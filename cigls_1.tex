\documentclass[12pt,letterpaper,twoside,final]{memoir}
%M-x shell
%latexmk -pdf -e '$pdflatex=q/xelatex %O %S/' *.tex
%latexmk -pdf -pvc -e '$pdflatex=q/xelatex %O %S/' *.tex
\usepackage[no-math]{fontspec}
\usepackage{xltxtra}
\defaultfontfeatures{Scale=MatchLowercase,Mapping=tex-text}
\setmainfont[Mapping=tex-text]{Junicode} %CMU Serif
\setsansfont[Mapping=tex-text]{Junicode} %CMU Sans Serif
\newfontfamily\greekfont{Linux Libertine} %CMU Serif,Linux Libertine O,Junicode
%\setmainfont[Mapping=tex-text]{CMU Serif}
\usepackage{xkeyval}
\usepackage{polyglossia}
\setdefaultlanguage[variant=american]{english}
\setotherlanguage[variant=ancient,numerals=arabic]{greek}
\setotherlanguage[spelling=new]{german}
\setotherlanguages{latin,french,italian,spanish,sanskrit}

\usepackage{xeindex}
\makeindex
\IndexList{mylist}{gold,silver,*Ethiopia=Ethiopia,*Ethiopian?=Ethiopians,*India=India,*Indian?=Indians,ψῆγμα=ψῆγμα,ψήγματος=ψήγμα}

\usepackage[babel=once,english=american,autostyle=tryonce,strict=true]{csquotes}
\usepackage[backend=biber,style=authoryear,sorting=debug,bibstyle=authoryear,citestyle=authoryear,useprefix=false,firstinits=false,url=false,usetranslator=true]{biblatex}%was: firstinits=true
%\DeclareAutoCiteCommand{plain}{\cite}{\cites}
\DeclareAutoCiteCommand{plain}{\textcite}{\textcites}
\DeclareAutoCiteCommand{inline}{\textcite}{\textcites}
%\DeclareAutoCiteCommand{footnote}[l]{\footcite}{\footcites}
%\DeclareAutoCiteCommand{footnote}[f]{\smartcite}{\smartcites}
\bibliography{classics_bib} %also avail india_bib
\usepackage[final]{hyperref}%?%hyperfootnotes=false
\hypersetup{bookmarks=false,        % show bookmarks bar?
    unicode=true,           % non-Latin characters in Acrobat’s bookmarks
    pdftoolbar=true,        % show Acrobat’s toolbar?
    pdfmenubar=true,        % show Acrobat’s menu?
    pdffitwindow=false,     % window fit to page when opened
    pdfstartview={FitH},    % fits the width of the page to the window
    pdftitle={Ethics of Leadership: Organization and Decision--Making in Caesar's \emph{Bellum~Gallicum}},
    pdfauthor={Kyle P. Johnson},     % author
    pdfsubject={Dissertation on organization and decision--making in Julius Caesar's Bellum Gallicum},   % subject of the document
    pdfcreator={Kyle P. Johnson},   % creator of the document
    pdfproducer={Kyle P. Johnson}, % producer of the document
    pdfkeywords={Julius Caesar, Bellum Gallicum, Gallic War, communication, deliberation, decision--making, leadership, organization, Xenophon, Anabasis}, % list of keywords
    pdfnewwindow=true,      % links in new window
    colorlinks=true,       % false: boxed links; true: colored links
    linkcolor=blue,          % color of internal links
    citecolor=blue,        % color of links to bibliography
    filecolor=blue,      % color of file links
    urlcolor=blue           % color of external links
}
\usepackage{microtype}
\usepackage{xecolor}
\definergbcolor{blue}{0000FF}
\definergbcolor{red}{FF0000} %\textxecolor{colorname}{text}
\XeTeXdashbreakstate=1
\usepackage{minitoc}
\usepackage{indentfirst}
\usepackage{outline}
\usepackage{verbatim}
\usepackage{enumerate}
%\usepackage{tikz}
%\usetikzlibrary{shapes,backgrounds}
\usepackage{longtable}
%\usepackage{lscape}
%\usepackage{verse}
%\usepackage{rotating}
\usepackage{ccicons}
\usepackage{bookmark}
\usepackage{eledmac}
\usepackage{eledpar}
\usepackage{etoolbox}
%!%%%%%%%%%%%%%%%%%%%%%for non-italicized headings%%%%%
% http://tex.stackexchange.com/questions/32655/remove-italic-from-memoir-headings-pagestyle
\makeevenhead{headings}{\leftmark}{}{}
\makeoddhead{headings}{\rightmark}{}{}
\makeevenfoot{headings}{}{\thepage}{}
\makeoddfoot{headings}{}{\thepage}{}
%!%%%%%%%%%%%%%%%%%%%%%%%%%%%%%%%%%%%%%%%%%%%%%%%%%%%%%%
%!%%%%%%%%%%%%%%%%%%%%%for margins, l=1.5; rest=1.0, maybe add 0.1in%%%%%
\setstocksize{11in}{8.5in}
\settrimmedsize{11.0in}{8.5in}{*} %\settrimmedsize{ height }{ width }{ ratio }
\settypeblocksize{7.75in}{5.8in}{*} %\settypeblocksize{ height }{ width }{ ratio } %note: 7.25h gives margins of 1.5 on top/bottom w/ ubratio of 1
\setlrmargins{1.5in}{*}{*} %\setlrmargins{ spine }{ edge }{ ratio } %spine = left, edge = right; only answer one or two of these values
%\setlrmarginsandblock{1.5in}{1.0in}{*} %\setlrmarginsandblock{ spine }{ edge }{ ratio } %
%\setulmargins{*}{*}{*} %\setulmargins{ upper }{ lower }{ ratio }
%\setulmarginsandblock{0.5in}{*}{*} %\setulmarginsandblock{ upper }{ lower }{ ratio }
%\setheadfoot{*}{*} %\setheadfoot{ headheight }{ footskip }
%\setheaderspaces{1.0in}{*}{*} %\setheaderspaces{ headdrop }{ headsep }{ ratio }
%\checkandfixthelayout[lines]
\flushbottom
%!%%%%%%%%%%%%%%%%%%%%%%%%%%%%%%%%%%%%%%%%%%%%%%%%%%%%%%
\apptocmd{\sloppy}{\hbadness 10000\relax}{}{}
%\renewcommand{\@pnumwidth}{3em} %taken from memman p. 153
%\renewcommand{\@tocrmarg}{4em} %taken from memman p. 153
\setsecnumdepth{subparagraph}
\makeatletter
\renewcommand\@makefntext{\hspace*{2em}\@thefnmark. }
\newenvironment*{singlespcquote}
        {\quote\SingleSpacing}
        {\endquote}
\SetBlockThreshold{0}
\SetBlockEnvironment{singlespcquote}
\SetCiteCommand{\parencite} %default is \cite
  %csquote + biblatex; see csquotes.pdf section 5 + 8.6
  %\textcquote[prenote][postnote]{key}[punct]{text}[tpunct]
  %\blockcquote[prenote][postnote]{key}[punct]{text}[tpunct]
  %usually:
  %\textcquote[page#]{key}{quote}
  %\blockcquote[page#]{key}[.]{quote}
  %\textcites(pre)(post)[pre][post]{key}...[pre][post]{key}
  %example: \textcites(and chapter 3)[35]{riggsby2006}[78]{hammond1996}[23]{levene2010}
\title{Eurasian Dharma Corpus}
\author{Kyle~P.~Johnson}
%\date{January 31, 2008}
\begin{document}
%\fussy
%\hyphenpenalty=5000   %1000 default=?
%\tolerance=1000        %1000 %200= default
%\setlength{\emergencystretch}{3em}
%\midsloppy
%\fussy
\sloppy
\vbadness=10000 % badness above which bad vboxes are shown. (Default = 10000?)
\frontmatter
\hyphenation{}

\SingleSpacing

%%%%%%%%%%for minitoc%%%%%%%%%%%%%%%%%%%%%
\dominitoc
\dominilof
\dominilot
%%%%%%%%%%%%%%%%%%%%%%%%%%%%%%%%%%%%%%%%%%%
\tableofcontents

\mainmatter

\chapter{Introduction}
\minitoc
Organization of this corpus? Options that come to mind are by (a) purported original reporter (e.g., Nicoalus of Damascus, non--extant historian), (b) reporting extant author (e.g., Strabo citing, but not quoting, Nicolaus at \emph{Geographica} 15.1.72--73), (c) by event (e.g., an embassy from Indian kingdom to Augustus in 19 BC, (d) types of people (e.g., gymnosophists or sramana), (e) story types (parallel fables or tropes, like walking on the water), (f) philosophical or religious concepts (like love and compassion). Further, I should consider whether there may be a home in this document for material culture such as statuary, painting, and coinage.

Another issue is one of perspective: from whose side am I recording this? Indians as they seemed to the Greeks? Greeks as they seemed to the Indians? Might either option lose subtleties of particular authors?

And what about distinct but sometimes related or confused cultures, namely the Ethiopians. 

%\part{India in Greek and Roman sources}

\chapter{Pre--classical Greek sources}
\dominitoc


\section{Aesopus et Aesopica}
\blockquote[From Wikipedia\footnote{\url{http://en.wikipedia.org/wiki/Aesop\%27s_Fables}}]{Aesop's Fables or the Aesopica is a collection of fables credited to Aesop, a slave and story-teller believed to have lived in ancient Greece between 620 and 560 BCE. Of diverse origins, the stories associated with Aesop's name have descended to modern times through a number of sources. They continue to be reinterpreted in different verbal registers and in popular as well as artistic mediums.}
\begin{greek}

Aesopus et Aesopica Scr. Fab., Fabulae tabulis ceratis Assendelftianis servatae (0096: 003)
“Corpus fabularum Aesopicarum, vol. 1.2, 2nd edn.”, Ed. Hausrath, A., Hunger, H.
Leipzig: Teubner, 1959.
Fable 1, line 5

                     τοιγ(α)ροῦν ἀλώπηξ κατὰ τὴν ἑαυτῆς 
ἀγχίνοιαν (ἐνό)ησε τὰς ἐνέδρας καὶ οὕτ(ω)ς ἑαυτὴν ἥρμο-
σεν, ἵνα καὶ τὴν χά(ριν) ἀναπληρώσῃ καὶ (τὸν) κ[1ίνδυν)ον 
ἐκφύγῃ. 



Aesopus et Aesopica Scr. Fab., Fabulae Aphthonii rhetoris (0096: 006)
“Corpus fabularum Aesopicarum, vol. 1.2, 2nd edn.”, Ed. Hausrath, A., Hunger, H.
Leipzig: Teubner, 1959.
Fable 6bis, line 2

λέγει δὲ (Αἴσωπος ὁ μυθοποιός), ὅτι 
 ὁ πριάμενος τὸν Ἰνδὸν ἐδυσχέραινε πρὸς τὴν χροιάν. 



Aesopus et Aesopica Scr. Fab., Fabulae Aphthonii rhetoris 
Fable 6bis, line 6

                                             ὡς δὲ αὐτῷ τὸ μὲν χρῶμα   
ἔμενεν ἀκραιφνές, τὸ δέρμα δὲ ἐξετρίβετο, φιλονεικῶν [δὲ] κατέμαθε 
μόλις, ὅτι ἡ χροιὰ φύσις ἦν τῷ Ἰνδῷ καὶ ὅτι ῥᾷον αὐτοῦ τὴν ψυχὴν 
ἢ τὴν μελανίαν ἀφαιρήσεσθαι. 
ΜΥΘΟΣ Ο ΤΗΣ ΠΑΡΘΕΝΟΥ ΚΑΙ ΤΟΥ ΛΕΟΝΤΟΣ 
ΗΔΟΝΩΝ ΑΠΟΤΡΕΠΩΝ


 λέων ἤρα παρθένου καὶ προσελθὼν τῷ πατρὶ τῆς παιδὸς 
ἐγγυῆσαι τὴν κόρην πρὸς γάμον ἐδεῖτο. 



Aesopus et Aesopica Scr. Fab., Fabulae Syntipae philosophi (0096: 009)
“Corpus fabularum Aesopicarum, vol. 1.2, 2nd edn.”, Ed. Hausrath, A., Hunger, H.
Leipzig: Teubner, 1959.
Fable 41, line t

ΑΙΘΙΟΨ ΙΝΔ*οΣ ΛΟΥΟΜΕΝΟΣ


 ἀνήρ τις ἑωρακὼς Αἰθίοπά τινα Ἰνδὸν λουόμενον ἐν τῷ 
ποταμῷ ἔφη πρὸς αὐτόν· “μὴ συνταράσσῃς καὶ θολοποιῇς 
τὸ ὕδωρ. 
\end{greek}



\section{Dionysius Chalcus}%???
Confirm person. Date?
\blockquote[From Wikipedia\footnote{\url{http://en.wikipedia.org/wiki/Dionysius_Chalcus}}]{tion, search

Dionysius Chalcus (Greek: Διονύσιος ὁ Χαλκοῦς) was an ancient Athenian poet and orator. According to Athenaeus, he was called Chalcus ("brazen") because he advised the Athenians to adopt a brass coinage (xv. p. 669). His speeches have not survived, but his poems are referred to and quoted by such authors as Plutarch (Nicias, 5), Aristotle (Rhetoric, iii. 2), and Athenaeus (xv, p. 668, 702; x, p. 443; xiii, p. 602). The extant fragments are chiefly elegies on symposiac subjects and are characterized by extravagant metaphors.

Plutarch credits Dionysius Chalcus with leading the band of Athenian colonists who founded Thurii in 443 BC.}
\begin{greek}

Dionysius Epic., Fragmenta (1326: 001)
“Die griechischen Dichterfragmente der römischen Kaiserzeit, vol. 1, 2nd edn.”, Ed. Heitsch, E.
Göttingen: Vandenhoeck \& Ruprecht, 1963.
Fragment 5b, line 2

ἐν δέ τε Κάσπειροι ποσσικλυτοί, ἐν δ' Ἀριηνοί, 
Κοσσαῖος γενεὴν Κασπειρόθεν, οἵ ρά τε πάντων 
Ἰνδῶν, ὅσσοι ἔασιν, ἀφάρτερα γούνατ' ἔχουσιν· 
ὅσσον γάρ τ' ἐν ὄρεσσιν ἀριστεύουσι λέοντες, 
ἢ ὁπόσον δελφῖνες ἔσω ἁλὸς ἠχηέσσης, 
αἰετὸς εἰν ὄρνισι μεταπρέπει ἀγρομένοισι 
ἵπποι τε πλακόεντος ἔσω πεδίοιο θέοντες, 
τόσσον ἐλαφροτάτοισι περιπροφέρουσι πόδεσσιν 
Κάσπειροι μετὰ φῦλα, τά τ' ἄφθιτος ἔλλαχεν ἠώς. 



Dionysius Epic., Fragmenta 
Fragment 9v, line 29

φῆ καὶ μέσσον ὄρουσεν ἀνὰ στρατὸν, ἔν[θα μάλιστα] 
Κηθαῖοι πυρίκαυτον ἐπὶ μόθ̣ο̣ν ἐκ̣λ̣ο[νέοντο,] 
[ς]τὰς [δ'] ὅ γε Δηριαδῆα καὶ ἄλ[λου]ς ἴαχ[εν αὐδῶν·] 
[ὧ]δ' ἄρα νῦν φράζεσθε γυναικ̣[ῶ]ν̣ ἀτμ̣έ[νες Ἰνδοί,]   
[Δ]ηριάδῃ δ' ἔκπαγλον ἐ̣[πιστά]μ̣ενος τ[άδε φράζω·] 
ο[ὐ] γάρ κεν πρὶν τοῦτο κατὰ̣ [στ]ένος αἴθο[πος ὁρμήν] 
οἴνου ἐρωήσαιτε καὶ ἐκ κακότητα φύ[γοιτε,] 
πρίν κε θοῇ ἐνὶ νυκτὶ διάλλυδις εἰρύς[σαντες] 
ὠμάδια κρέα θηρὸς ἀπὸ ζωοῖ̣ο φάγη[τε. 



Dionysius Epic., Fragmenta 
Fragment 11r, line 9

[]μμενα̣ι ἀλ̣[] 
[κ]ατεναντίον [] 
Φηρῶν ἄ̣σχετον ὕβριν ἀ[λ]υ̣σκά[ζ] 
ο̣ἱ̣ δ̣' ἅ̣μ̣α Πορθάωνος ἀτασθαλί[ῃ] 
Αἴθιππος Προκάων τε Τυρηθ[] 
ἠδὲ λιπὼ[ν] Μοτύης τε θεῶν [] 
ἑζόμενο[. κ]ονίῃσιν ἀφελκε[] 
ἔστι δ[....]υοεντος ὑπὲρ πεδί[οιο]   
μ̣εσσ̣[όθι τύμ]β̣ος ἐρεμνὸς α.[] 
..χα.[.........]σ̣ος ἱερὰ.[] 
Ἰνδι̣[..........]ικ..[] 
τοί δ' α[....]ε..[...]ι̣ο̣ν̣[] 
γαῖαν ὅσην Γάγγης τε π̣[] 
ἔνθα δὲ κῆρα φυγόντες ἀπ̣[] 
Πεισίνοος τε Δάμας τε Πολ̣[] 
τῶν οἱ μὲν σιγῇ δεδμη[μένοι] 
τρεῖς ἔσαν ὅττι ἑ πρόσθε[] 
αὐτὰρ ὁ καὶ μύθοισι καὶ α̣[] 
Παρπαδίκης Κόκα̣[λός τε?] 
Πεισίνοος προφέρ̣[ιστος] 

\end{greek}


\chapter{Classical Greek sources}
\minitoc


\section{Hellanicus of Mytilene}%???
Confirm person.
\blockquote[From Wikipedia\footnote{\url{http://en.wikipedia.org/wiki/Hellanicus_of_Mytilene}}]{Hellanicus of Lesbos (Ancient Greek: Ἑλλάνικος) was an ancient Greek logographer who flourished during the latter half of the 5th century BC. He was born in Mytilene on the isle of Lesbos in 490 BC and is reputed to have lived to the age of 85. According to the Suda, he lived for some time at the court of one of the kings of Macedon, and died at Perperene, a city in Aeolis on the plateau of Kozak near Pergamon, opposite Lesbos.

His work includes the first mention of the legendary founding of Rome by the Trojans; he writes that the city was founded by Aeneas when accompanying Odysseus on travels through Latium. However, he supported the idea that the Etruscans lay behind the origins of the Pelasgians, an ancient Greek people who were thought to have antedated the great Achaean invasions.

Some thirty works are attributed to him—chronological, historical and episodical. Mention may be made of:

    The Priestesses of Hera at Argon: a chronological compilation, arranged according to the order of succession of these functionaries
    The Carneonikae: a list of the victors in the Carnean games (the chief Spartan musical festival), including notices of literary events
    An Atthis, giving the history of Attica from 683 to the end of the Peloponnesian War (404), which is referred to by Thucydides (1.97), who says that he treated the events of the years 480-431 briefly and superficially, and with little regard to chronological sequence
    Phoronis: chiefly genealogical, with short notices of events from the times of Phoroneus, primordial king in Peloponnesus.
    Troica and Persica: histories of Troy and Persia.

Hellanicus authored works of chronology, geography, and history, particularly concerning Attica, in which he made a distinction between what he saw as Greek mythology from history. His influence on the historiography of Athens was considerable, lasting until the time of Eratosthenes (3rd century BC).

He transcended the narrow local limits of the older logographers, and was not content to merely repeat the traditions that had gained general acceptation through the poets. He tried to give the traditions as they were locally current, and availed himself of the few national or priestly registers that presented something like contemporary registration.

He endeavoured to lay the foundations of a scientific chronology, based primarily on the list of the Argive priestesses of Hera, and secondarily on genealogies, lists of magistrates (e.g. the archons at Athens), and Oriental dates, in place of the old reckoning by generations. But his materials were insufficient and he often had recourse to the older methods.

On account of his deviations from common tradition, Hellanicus is often called an untrustworthy writer by the ancients themselves, and it is a curious fact that he appears to have made no systematic use of the many inscriptions which were ready to hand. Dionysius of Halicarnassus censures him for arranging his history, not according to the natural connection of events, but according to the locality or the nation he was describing; and undoubtedly he never, like his contemporary Herodotus, rose to the conception of a single current of events wider than the local distinction of race. His style, like that of the older logographers, was dry and bald.}
\begin{greek}


Hellanicus Hist., Testimonia (0539: 001)
“FGrH \#4, \#323a, \#687a”.
Volume-Jacobyʹ-T 1a,4,T, fragment 19, line 3

STRABON I 2, 35: Θεόπομπος (II) δὲ ἐξομολογεῖται φήσας, 
ὅτι καὶ μύθους ἐν ταῖς ἱστορίαις ἐρεῖ, κρεῖττον ἢ ὡς Ἡρόδοτος καὶ Κτησίας 
(III) καὶ Ἑλλάνικος καὶ οἱ τὰ Ἰνδικὰ συγγράψαντες (III). 
 AVIEN. 



Hellanicus Hist., Fragmenta (0539: 002)
“FGrH \#4, \#323a, \#601a, \#608a, \#645a, \#687a”.
Volume-Jacobyʹ-F 1a,4,F, fragment 87, line 16

             .. καὶ τρίτον ἐπὶ τούτοις θεὸν ἀσώματον πτέρυγας ἐπὶ τῶν ὤμων ἔχοντα 
χρυσᾶς, ὃς ἐν μὲν ταῖς λαγόσι προσπεφυκυίας εἶχε ταύρων κεφαλάς, ἐπὶ δὲ τῆς κεφαλῆς 
δράκοντα πελώριον παντοδαπαῖς μορφαῖς θηρίων ἰνδαλλόμενον’. 



Hellanicus Hist., Fragmenta 
Volume-Jacobyʹ-F 1a,4,F, fragment 154b, line 5

                                               Κράτης δὲ Ἐρεμνοὺς γράφει καὶ 
ἀποδίδωσι τοὺς Ἰνδούς, ἐπεὶ μέλανες οὗτοι. 



Hellanicus Hist., Fragmenta 
Volume-Jacobyʹ-F 1a,4,F, fragment 185, line 5

          .. ἐν ἀριστερᾶι δὲ οἱ πρὸς ἕω Σκύθαι νομάδες καὶ οὗτοι μέχρι 
τῆς ἑώιας θαλάττης καὶ τῆς Ἰνδικῆς παρατείνοντες. 



Hellanicus Hist., Fragmenta 
Volume-Jacobyʹ-F 1a,4,F, fragment 190, line 1

                     ροηδ. 36: Ἑλλάνικος ἐν Ἰνδοῖς εἶναί 
φησι κρήνην Σίλλην καλουμένην, ἐφ' ἧς καὶ τὰ ἐλαφρότατα κατα-
ποντίζεται. 

\end{greek}


\section{Hecataeus}
\blockquote[From Wikipedia\footnote{\url{http://en.wikipedia.org/wiki/Hecataeus_of_Miletus}}]{

Hecataeus of Miletus (c. 550 BC – c. 476 BC[1]) (Greek Ἑκαταῖος), named after the Greek goddess Hecate, was an early Greek historian of a wealthy family. He flourished during the time of the Persian invasion. After having travelled extensively, he settled in his native city, where he occupied a high position, and devoted his time to the composition of geographical and historical works. When Aristagoras held a council of the leading Ionians at Miletus to organize a revolt against the Persian rule, Hecataeus in vain tried to dissuade his countrymen from the undertaking.[2] In 494 BC, when the defeated Ionians were obliged to sue for terms, he was one of the ambassadors to the Persian satrap Artaphernes, whom he persuaded to restore the constitution of the Ionic cities.[3] Hecataeus is the first known Greek historian,[4] and was one of the first classical writers to mention the Celtic people.}
 of Miletus?
\begin{greek}
       
Hecataeus Hist., Fragmenta (0538: 002)
“FGrH \#1”.
Volume-Jacobyʹ-F 1a,1,F, fragment 33, line 4

                                                            .. καὶ αὐτοῖς 
παίδευμα τὴν φρουρὰν <ἔχειν> οὐ κατανυστάζουσι· διδάσκονται γάρ τοι 
σοφίαι τινὶ Ἰνδικῆι καὶ τοῦτο. 



Hecataeus Hist., Fragmenta 
Volume-Jacobyʹ-F 1a,1,F, fragment 294a, line 1

                                                                        > (es 
folgt F 292 b). 
  – s. Γανδάραι· Ἰνδῶν ἔθνος. 



Hecataeus Hist., Fragmenta 
Volume-Jacobyʹ-F 1a,1,F, fragment 296, line 1

           ιι 70 β: (φ 292a) καὶ περὶ τὸν Ἰνδὸν δέ φησι ποταμὸν 
γίνεσθαι τὴν κυνάραν. 



Hecataeus Hist., Fragmenta 
Volume-Jacobyʹ-F 1a,1,F, fragment 297, line 1

              s. Ἀργάντη· πόλις Ἰνδίας, ὡς Ἑκαταῖος. 



Hecataeus Hist., Fragmenta 
Volume-Jacobyʹ-F 1a,1,F, fragment 298, line 1

– s. Καλατίαι· γένος Ἰνδικόν. 



Hecataeus Hist., Fragmenta 
Volume-Jacobyʹ-F 1a,1,F, fragment 299, line 1

– s. Ὠπίαι· ἔθνος Ἰνδικόν. 



Hecataeus Hist., Fragmenta 
Volume-Jacobyʹ-F 1a,1,F, fragment 299, line 2

                                     Ἑκαταῖος Ἀσίαι «<ἐν δ' αὐτοῖς 
οἰκέουσιν ἄνθρωποι παρὰ τὸν Ἰνδὸν ποταμὸν Ὠπίαι, ἐν 
δὲ τεῖχος βασιλήιον. 



Hecataeus Hist., Fragmenta 
Volume-Jacobyʹ-F 1a,1,F, fragment 299, line 4

                                                 ἀπὸ δὲ τούτου 
ἐρημίη μέχρις Ἰνδῶν. 

\end{greek}


\section{Aeschylus}
\blockquote[From Wikipedia\footnote{\url{}}]{}
\begin{greek}

Aeschylus Trag., Supplices (0085: 001)
“Aeschyli tragoediae, 2nd edn.”, Ed. Murray, G.
Oxford: Clarendon Press, 1955, Repr. 1960.
Line 284

καὶ Νεῖλος ἂν θρέψειε τοιοῦτον φυτόν, 
Κύπριος χαρακτήρ τ' ἐν γυναικείοις τύποις 
εἰκὼς πέπληκται τεκτόνων πρὸς ἀρσένων· 
Ἰνδάς τ' ἀκούω νομάδας ἱπποβάμοσιν 
εἶναι καμήλοις ἀστραβιζούσας χθόνα,   
παρ' Αἰθίοψιν ἀστυγειτονουμένας. 



Aeschylus Trag., Fragmenta (0085: 008)
“Die Fragmente der Tragödien des Aischylos”, Ed. Mette, H.J.
Berlin: Akademie–Verlag, 1959.
Tetralogy 10, play A, fragment 74a, line 6

Ξενοφῶντα δὲ γένος τι Ἰνδῶν φάναι τὸν ‘χλούνην’ εἶναι, καθάπερ καὶ 
<παρ' Αἰσχύλωι ἐν Ἠδωνοῖς·> 
         ‘μακροσκελὴς μέν. 

\end{greek}




\section{Democritus}
\blockquote[From Wikipedia\footnote{\url{http://en.wikipedia.org/wiki/Democritus}}]{Democritus (Greek: Δημόκριτος, Dēmokritos, "chosen of the people") (ca. 460 BC – ca. 370 BC) was an Ancient Greek philosopher born in Abdera, Thrace, Greece.[1] He was an influential pre-Socratic philosopher and pupil of Leucippus, who formulated an atomic theory for the universe.[2]

His exact contributions are difficult to disentangle from those of his mentor Leucippus, as they are often mentioned together in texts. Their speculation on atoms, taken from Leucippus, bears a passing and partial resemblance to the nineteenth-century understanding of atomic structure that has led some to regard Democritus as more of a scientist than other Greek philosophers; however their ideas rested on very different bases.[3] Largely ignored in ancient Athens, Democritus was nevertheless well known to his fellow northern-born philosopher Aristotle. Plato is said to have disliked him so much that he wished all his books burned.[1] Many consider Democritus to be the "father of modern science".[4]

He traveled to Asia, and was even said to have reached India and Ethiopia.[10] (Cicero, de Finibus, v.19; Strabo, xvi).}
\begin{greek}

Democritus Phil., Testimonia (1304: 001)
“Die Fragmente der Vorsokratiker, vol. 2, 6th edn.”, Ed. Diels, H., Kranz, W.
Berlin: Weidmann, 1952, Repr. 1966.
Fragment 1, line 15

τοῖς τε Γυμνοσοφισταῖς φασί τινες συμμῖξαι αὐτὸν ἐν Ἰνδίαι καὶ εἰς Αἰθιοπίαν 
ἐλθεῖν. 



Democritus Phil., Testimonia 
Fragment 2, line 5

SUIDAS Δημόκριτος Ἡγησιστράτου (οἱ δὲ Ἀθηνοκρίτου ἢ Δαμασίππου) 
γεγονὼς ὅτε καὶ Σωκράτης ὁ φιλόσοφος κατὰ τὴν <οζ> ὀλυμπιάδα [472 – 469] (οἱ 
δὲ κατὰ τὴν <π> [460 – 457] φασίν)· Ἀβδηρίτης ἐκ Θράικης, φιλόσοφος, μαθητὴς 
κατά τινας Ἀναξαγόρου καὶ Λευκίππου, ὡς δέ τινες, καὶ μάγων καὶ Χαλδαίων 
Περσῶν· ἦλθε γὰρ καὶ εἰς Πέρσας καὶ Ἰνδοὺς καὶ Αἰγυπτίους καὶ τὰ παρ' ἑκάστοις 
ἐπαιδεύθη σοφά. 



Democritus Phil., Testimonia 
Fragment 16, line 4

                                                     ἧκεν οὖν πρὸς τοὺς Χαλδαίους καὶ 
εἰς Βαβυλῶνα καὶ πρὸς τοὺς μάγους καὶ τοὺς σοφιστὰς τῶν Ἰνδῶν. 



Democritus Phil., Testimonia 
Fragment 40, line 3

               Δαμασίππου Ἀβδηρίτης πολλοῖς συμβαλὼν γυμνοσοφισταῖς ἐν 
Ἰνδοῖς καὶ ἱερεῦσιν ἐν Αἰγύπτωι καὶ ἀστρολόγοις καὶ ἐν Βαβυλῶνι μάγοις. 

\end{greek}



\section{Aristocrates}%???
%right person?
\blockquote[From Wikipedia\footnote{\url{http://en.wikipedia.org/wiki/Aristocrates_of_Athens}}]{Aristocrates (in Greek Aριστoκρατης; lived 4th century BC) was a person against whom Demosthenes wrote an oration, still extant, entitled Against Aristocrates (Kατα Aριστoκρατoυς). He wrote it shortly before 352 BC for Euthycles, who accused Aristocrates of proposing an illegal decree in relation to Charidemus, a Euboean adventurer who acted as chief minister for the Thracian prince Cersobleptes and desired to assume with Athenian help full control of king Cotys former dominions.}
\begin{greek}

Aristocrates Hist., Fragmenta (1189: 003)
“FHG 4”, Ed. Müller, K.
Paris: Didot, 1841–1870.
Fragment 2, line 2

Plutarch. Lyc. c. 4: Ὅτι δὲ καὶ Λιβύην καὶ Ἰβη-
ρίαν ἐπῆλθεν ὁ Λυκοῦργος, καὶ περὶ τὴν Ἰνδικὴν πλα-
νηθεὶς τοῖς γυμνοσοφισταῖς ὡμίλησεν, οὐδένα πλὴν 
Ἀριστοκράτη τὸν Ἱππάρχου Σπαρτιάτην εἰρηκότα 
γινώσκομεν. 

\end{greek}


\section{\emph{Scholia In Aeschylum}}
\blockquote[From Wikipedia\footnote{\url{}}]{}
\begin{greek}

Scholia In Aeschylum, Scholia in Aeschylum (scholia vetera) (5010: 001)
“Scholia Graeca in Aeschylum quae exstant omnia, vols. 1 \& 2.2”, Ed. Smith, O.L.
Leipzig: Teubner, 1:1976; 2.2:1982.
Play Supp, hypothesis-epigram-scholion 559, line 1

χιονόβοσκον] φασὶ γὰρ λυομένης χιόνος παρὰ Ἰνδοῖς πληροῦσθαι 
αὐτόν. 


Scholia In Aeschylum, Scholia in Prometheum vinctum (scholia vetera) (5010: 005)
“The older scholia on the Prometheus bound”, Ed. Herington, C.J.
Leiden: Brill, 1972.
Vita-argumentum-scholion-epigram sch, verse 844b, line 3

                                                   A.   
PPd (post 844a) et Ya (iuxta Prom. 852): Ἰστέον 
δὲ ὅτι ὅπερ ὁ Νεῖλος ἐν τῇ Αἰγύπτῳ ποιεῖ, τοῦτο καὶ ὁ 
ἐν τῇ Ἰνδίᾳ ῥέων ποταμός. 



Scholia In Aeschylum, Scholia in Prometheum vinctum (scholia vetera) 
Vita-argumentum-scholion-epigram sch, verse 844b, line 6

                                    ἄρδει γὰρ καὶ οὗτος τὰ 
ἐκεῖσε πεδία, διαφέρει δέ τι τοῦ Νείλου· μετὰ δὲ τὸ 
ἀρδεῦσαι τοῦτον τὰ ἐκεῖσε χωράφια, συνελθόντες οἱ ἐν 
τῇ Ἰνδίᾳ οἰκοῦντες φωνὰς εὐχαριστηρίους μετὰ κυμβά-
λων καὶ ἤχων ὀργάνων τούτῳ τῷ ποταμῷ ἀναπέμ-
πουσι. 



Scholia In Aeschylum, Scholia in Aeschylum (scholia recentiora) (5010: 009)
“Aeschyli tragoediae superstites et deperditarum fragmenta, vol. 3 [Scholia Graeca ex codicibus aucta et emendata]”, Ed. Dindorf, W.
Oxford: Oxford University Press, 1851, Repr. 1962.
Play Pr, hypothesis-verse of play 851, line 5

                                                       O.P. 
Ἰστέον δὲ ὅτι ὅπερ ὁ Νεῖλος ἐν τῇ Αἰγύπτῳ ποιεῖ, τοῦτο καὶ ὁ ἐν 
τῇ Ἰνδίᾳ ῥέων ποταμός. 



Scholia In Aeschylum, Scholia in Aeschylum (scholia recentiora) 
Play Pr, hypothesis-verse of play 851, line 7

                             μετὰ δὲ τὸ ἀρδεῦσαι τοῦτον τὰ ἐκεῖσε 
χωράφια, συνελθόντες οἱ ἐν τῇ Ἰνδίᾳ οἰκοῦντες φωνὰς εὐχαριστηρίους 
μετὰ κυμβάλων καὶ ἤχων ὀργάνων τούτῳ τῷ ποταμῷ ἀναπέμπουσι. 

\end{greek}


\section{Hecataeus}
\blockquote[From Wikipedia\footnote{\url{http://en.wikipedia.org/wiki/Hecataeus_of_Miletus}}]{

Hecataeus of Miletus (c. 550 BC – c. 476 BC[1]) (Greek Ἑκαταῖος), named after the Greek goddess Hecate, was an early Greek historian of a wealthy family. He flourished during the time of the Persian invasion. After having travelled extensively, he settled in his native city, where he occupied a high position, and devoted his time to the composition of geographical and historical works. When Aristagoras held a council of the leading Ionians at Miletus to organize a revolt against the Persian rule, Hecataeus in vain tried to dissuade his countrymen from the undertaking.[2] In 494 BC, when the defeated Ionians were obliged to sue for terms, he was one of the ambassadors to the Persian satrap Artaphernes, whom he persuaded to restore the constitution of the Ionic cities.[3] Hecataeus is the first known Greek historian,[4] and was one of the first classical writers to mention the Celtic people.}
Correct Hecataeus?

\begin{greek}

Hecataeus Hist., Fragmenta (1390: 002)
“FGrH \#264”.
Volume-Jacobyʹ-F 3a,264,F, fragment 25, line 620

                                                                             (2) ἔπειτ' εἰς μὲν τὴν 
Ἐρυθρὰν θάλατταν ἀπέστειλε στόλον νεῶν τετρακοσίων, πρῶτος τῶν ἐγχωρίων μακρὰ 
σκάφη ναυπηγησάμενος, καὶ τάς τε νήσους τὰς ἐν τοῖς τόποις κατεκτήσατο, καὶ τῆς ἠπεί-
ρου τὰ παρὰ θάλατταν μέρη κατεστρέψατο μέχρι τῆς Ἰνδικῆς· αὐτὸς δὲ μετὰ τῆς δυνάμεως 
πεζῆι τὴν πορείαν [διὰ Συρίας] ποιησάμενος κατεστρέψατο πᾶσαν τὴν Ἀσίαν. 



Hecataeus Hist., Fragmenta 
Volume-Jacobyʹ-F 3a,264,F, fragment 25, line 624

                                                                       (4) καὶ γὰρ τὸν Γάγγην ποταμὸν 
διέβη, καὶ τὴν Ἰνδικὴν ἐπῆλθε πᾶσαν ἕως ὠκεανοῦ, καὶ τὰ τῶν Σκυθῶν ἔθνη μέχρι Τα-
νάιδος ποταμοῦ τοῦ διορίζοντος τὴν Εὐρώπην ἀπὸ τῆς Ἀσίας· ὅτε δή φασι τῶν Αἰγυ-
πτίων καταλειφθέντας περὶ τὴν Μαιῶτιν λίμνην συστήσασθαι τὸ τῶν Κόλχων ἔθνος. 

\end{greek}


\section{Orphica, \emph{Lithica kerygmata}}
Orphica, Fragmenta (0579: 010)
“Die Fragmente der Vorsokratiker, vol. 1, 6th edn.”, Ed. Diels, H., Kranz, W.
Berlin: Weidmann, 1951, Repr. 1966.
Fragment 13, line 20

\begin{greek}
     .. καὶ τρίτον ἐπὶ τούτοις <θεὸν ἀσώματον,> πτέρυγας ἐπὶ τῶν 
ὤμων ἔχοντα χρυσᾶς, ὃς ἐν μὲν ταῖς λαγόσι προσπεφυκυίας εἶχε 
ταύρων κεφαλάς, ἐπὶ δὲ τῆς κεφαλῆς δράκοντα πελώριον παντοδαπαῖς 
μορφαῖς θηρίων ἰνδαλλόμενον . 



Orphica, Lithica kerygmata (0579: 012)
“Les lapidaires grecs”, Ed. Halleux, R., Schamp, J.
Paris: Les Belles Lettres, 1985.
Section 8, line 3

Γεννᾶται δὲ ἐν τῇ Ἰνδικῇ. 



Orphica, Lithica kerygmata 
Section 26, line 11

           Οὗτος ὁ λίθος γεννᾶται ἐν Ἰνδίᾳ, ὅπου ὁ 
Φισὼν ποταμὸς ἐκ τοῦ παραδείσου ἔρχεται. 



Orphica, Lithica kerygmata 
Section 29, line 14

                                           Γίνεται δὲ ἐν 
τῇ Ἰνδικῇ, ὅπου καὶ οἱ προγεγραμμένοι. 



Orphica, Lithica kerygmata 
Section 32, line 1

                     Οὗτος ἐν τῇ Ἰνδικῇ γίνεται 
λευκὰς ζώνας πλείστας ἔχων ἐν ἑαυτῷ ἀεριζούσας. 


\end{greek}

\section{Herodotus}

\subsection{About Herodotus}
\blockquote[From Wikipedia]{Herodotus ( /hɨˈrɒdətəs/; Ancient Greek: Ἡρόδοτος Hēródotos) was an ancient Greek historian who was born in Halicarnassus, Caria (modern day Bodrum, Turkey) and lived in the fifth century BC (c.484 – 425 BC). He has been called the "Father of History", and was the first historian known to collect his materials systematically, test their accuracy to a certain extent and arrange them in a well-constructed and vivid narrative.[1] The Histories—his masterpiece and the only work he is known to have produced—is a record of his "inquiry" (or ἱστορία historía, a word that passed into Latin and acquired its modern meaning of "history"), being an investigation of the origins of the Greco-Persian Wars and including a wealth of geographical and ethnographical information. Although some of his stories were fanciful, he claimed he was reporting only what had been told to him. Little is known of his personal history.}

On gold--digging ants:
\blockquote[From Wikipedia]{His accounts of India are among the oldest records of Indian civilization by an outsider.[56]

Discoveries made since the end of the 19th century have both added to and detracted from his credibility. His description of Gelonus, located in Scythia, as a city thousands of times larger than Troy was widely disbelieved until it was rediscovered in 1975. The archaeological study of the now-submerged ancient Egyptian city of Heracleion and the recovery of the so-called "Naucratis stela" give extensive credibility to Herodotus's previously unsupported claim that Heracleion was founded during the Egyptian New Kingdom.

Other claims he made are inconsistent with archeological and cuneiform document evidence. For instance, his account of the Medes appears to accord poorly with Assyrian and Babylonian records and with archeological evidence.[citation needed]

One of the most recent developments in Herodotus scholarship was made by the French ethnologist Michel Peissel. On his journeys to India and Pakistan, Peissel claims to have discovered an animal species that may finally illuminate one of the most bizarre passages in Herodotus's Histories. In Book 3, passages 102 to 105, Herodotus reports that a species of fox-sized, furry "ants" lives in one of the far eastern, Indian provinces of the Persian Empire. This region, he reports, is a sandy desert, and the sand there contains a wealth of fine gold dust. These giant ants, according to Herodotus, would often unearth the gold dust when digging their mounds and tunnels, and the people living in this province would then collect the precious dust. Now, Peissel says that in an isolated region of northern Pakistan, on the Deosai Plateau in Gilgit–Baltistan province, there exists a species of marmot, (the Himalayan marmot), (a type of burrowing squirrel) that may have been what Herodotus called giant "ants". Much like the province that Herodotus describes, the ground of the Deosai Plateau is rich in gold dust. According to Peissel, he interviewed the Minaro tribal people who live in the Deosai Plateau, and they have confirmed that they have, for generations, been collecting the gold dust that the marmots bring to the surface when they are digging their underground burrows. The story seems to have been widespread in the ancient world, because later authors like Pliny the Elder mentioned it in his gold mining section of the Naturalis Historia.

Even more tantalizing, in his book, The Ants' Gold: The Discovery of the Greek El Dorado in the Himalayas, Peissel offers the theory that Herodotus may have become confused because the old Persian word for "marmot" was quite similar to that for "mountain ant". Because research suggests that Herodotus probably did not know any Persian (or any other language except his native Greek), he was forced to rely on a multitude of local translators when travelling in the vast multilingual Persian Empire. Therefore, he may have been the unwitting victim of a simple misunderstanding in translation. As Herodotus never claims to have himself seen these "ant/marmot" creatures, it is likely that he was simply reporting what other travellers were telling him, no matter how bizarre or unlikely he personally may have found it to be. In the age when most of the world was still mysterious and unknown and before the modern science of biology, the existence of a giant ant may not have seemed so far-fetched. The suggestion that he completely made up the tale may continue to be thrown into doubt as more research is conducted.[57][58]

With that said, Herodotus did follow up in passage 105 of Book 3, with the claim that the "ants/marmots" are said to chase and devour full-grown camels; again, this could simply be dutiful reporting of what was in reality a tall tale or legend told by the local tribes to frighten foreigners from seeking this relatively easy access to gold dust. On the other hand, the details of the "ants" seem somewhat similar to the description of the camel spider (Solifugae), which are said to chase camels, have lots of hair bristles, and could quite easily be mistaken for ants. On account of the fear of encountering one, there have been "many myths and exaggerations about their size".[59] Images of camel spiders[60][61] could give the impression that this could be mistaken for a giant ant, but certainly not the size of a fox.}

\subsection{Histories}
Text: Herodotus Hist., Historiae (0016: 001)
“Hérodote. Histoires, 9 vols.”, Ed. Legrand, Ph.–E.
Paris: Les Belles Lettres, 1:1932; 2;1930; 3:1939; 4 (3rd edn.): 1960; 5:1946; 6:1948; 7:1951; 8:1953; 9:1954, Repr. 1:1970; 2:1963; 3:1967; 5:1968; 6:1963; 7:1963; 8:1964; 9:1968.
\begin{greek}
Book 1, section 192, line 18

                                                    Κυνῶν δὲ 
Ἰνδικῶν τοσοῦτο δή τι πλῆθος ἐτρέφετο ὥστε τέσσερες 
τῶν ἐν τῷ πεδίῳ κῶμαι μεγάλαι, τῶν ἄλλων ἐοῦσαι ἀτελέες, 
τοῖσι κυσὶ προσετετάχατο σιτία παρέχειν. 
Go to Context



Herodotus Hist., Historiae 
Book 3, section 38, line 15

                               Δαρεῖος δὲ μετὰ ταῦτα καλέσας 
Ἰνδῶν τοὺς καλεομένους Καλλατίας, οἳ τοὺς γονέας κατ-
εσθίουσι, εἴρετο, παρεόντων τῶν Ἑλλήνων καὶ δι' ἑρμηνέος 
μανθανόντων τὰ λεγόμενα, ἐπὶ τίνι χρήματι δεξαίατ' ἂν 
τελευτῶντας τοὺς πατέρας κατακαίειν πυρί· οἱ δὲ ἀμβώ-
σαντες μέγα εὐφημέειν μιν ἐκέλευον. 
Go to Context



Herodotus Hist., Historiae 
Book 3, section 94, line 7

               Μόσχοισι δὲ καὶ Τιβαρηνοῖσι καὶ Μάκρωσι 
καὶ Μοσσυνοίκοισι καὶ Μαρσὶ τριηκόσια τάλαντα προεί-
ρητο· νομὸς εἴνατος καὶ δέκατος οὗτος Ἰνδῶν δὲ πλῆθός 
τε πολλῷ πλεῖστόν ἐστι πάντων τῶν ἡμεῖς ἴδμεν 
ἀνθρώπων καὶ φόρον ἀπαγίνεον πρὸς πάντας τοὺς ἄλλους 
ἑξήκοντα καὶ τριηκόσια τάλαντα ψήγματος· νομὸς εἰκοστὸς 
οὗτος. 
Go to Context



Herodotus Hist., Historiae 
Book 3, section 97, line 9

                                        .. οἳ περί τε Νύσην 
τὴν ἱρὴν κατοίκηνται καὶ τῷ Διονύσῳ ἀνάγουσι τὰς ὁρτάς· 
[οὗτοι οἱ Αἰθίοπες καὶ οἱ πλησιόχωροι τούτοισι σπέρματι 
μὲν χρέωνται τῷ αὐτῷ τῷ καὶ οἱ Καλλαντίαι Ἰνδοί, οἰκή-
ματα δὲ ἔκτηνται κατάγαια]· οὗτοι συναμφότεροι διὰ τρί-
του ἔτεος ἀγίνεον, ἀγινέουσι δὲ καὶ τὸ μέχρις ἐμέο, δύο   
χοίνικας ἀπύρου χρυσίου καὶ διηκοσίας φάλαγγας ἐβένου 
καὶ πέντε παῖδας Αἰθίοπας καὶ ἐλέφαντος ὀδόντας μεγά-
λους εἴκοσι. 
Go to Context



Herodotus Hist., Historiae 
Book 3, section 98, line 1

Τὸν δὲ χρυσὸν τοῦτον τὸν πολλὸν οἱ Ἰνδοί, ἀπ' οὗ τὸ 
ψῆγμα τῷ βασιλέϊ τὸ εἰρημένον κομίζουσι, τρόπῳ τοιῷδε 
κτῶνται. 
Go to Context



Herodotus Hist., Historiae 
Book 3, section 98, line 3

          Ἔστι τῆς Ἰνδικῆς χώρης τὸ πρὸς ἥλιον ἀνί-
σχοντα ψάμμος· τῶν γὰρ ἡμεῖς ἴδμεν, τῶν καὶ πέρι ἀτρεκές 
τι λέγεται, πρῶτοι πρὸς ἠῶ καὶ ἡλίου ἀνατολὰς οἰκέουσι 
ἀνθρώπων τῶν ἐν τῇ Ἀσίῃ Ἰνδοί· Ἰνδῶν γὰρ τὸ πρὸς τὴν 
ἠῶ ἐρημίη ἐστὶ διὰ τὴν ψάμμον. 
Go to Context



Herodotus Hist., Historiae 
Book 3, section 98, line 8

Ἔστι δὲ πολλὰ ἔθνεα Ἰνδῶν καὶ οὐκ ὁμόφωνα σφίσι, καὶ 
οἱ μὲν αὐτῶν νομάδες εἰσί, οἱ δὲ οὔ, οἱ δὲ ἐν τοῖσι ἕλεσι 
οἰκέουσι τοῦ ποταμοῦ καὶ ἰχθῦς σιτέονται ὠμούς, τοὺς 
αἱρέουσι ἐκ πλοίων καλαμίνων ὁρμώμενοι· καλάμου δὲ ἓν 
γόνυ πλοῖον ἕκαστον ποιέεται. 
Go to Context



Herodotus Hist., Historiae 
Book 3, section 98, line 12

                                   Οὗτοι μὲν δὴ τῶν Ἰνδῶν 
φορέουσι ἐσθῆτα φλοΐνην· ἐπεὰν ἐκ τοῦ ποταμοῦ φλοῦν   
ἀμήσωνται καὶ κόψωσι, τὸ ἐνθεῦτεν φορμοῦ τρόπον κα-
ταπλέξαντες ὡς θώρηκα ἐνδύνουσι. 
Go to Context



Herodotus Hist., Historiae 
Book 3, section 99, line 1

                                      Ἄλλοι δὲ τῶν Ἰνδῶν 
πρὸς ἠῶ οἰκέοντες τούτων νομάδες εἰσί, κρεῶν ἐδεσταὶ 
ὠμῶν, καλέονται δὲ Παδαῖοι. 
Go to Context



Herodotus Hist., Historiae 
Book 3, section 100, line 1

                                     Ἑτέρων δέ ἐστι Ἰνδῶν 
ὅδε ἄλλος τρόπος· οὔτε κτείνουσι οὐδὲν ἔμψυχον οὔτε τι 
σπείρουσι οὔτε οἰκίας νομίζουσι ἐκτῆσθαι ποιηφαγέουσί τε, 
καὶ αὐτοῖσι <ὄσπριόν τι> ἔστι ὅσον κέγχρος τὸ μέγαθος ἐν 
κάλυκι, αὐτόματον ἐκ τῆς γῆς γινόμενον, τὸ συλλέγοντες 
αὐτῇ τῇ κάλυκι ἕψουσί τε καὶ σιτέονται. 
Go to Context



Herodotus Hist., Historiae 
Book 3, section 101, line 2

                                                     Μίξις δὲ 
τούτων τῶν Ἰνδῶν τῶν κατέλεξα πάντων ἐμφανής ἐστι   
κατά περ τῶν προβάτων, καὶ τὸ χρῶμα φορέουσι ὅμοιον 
πάντες καὶ παραπλήσιον Αἰθίοψι. 
Go to Context



Herodotus Hist., Historiae 
Book 3, section 101, line 7

                                        Οὗτοι μὲν τῶν Ἰνδῶν 
ἑκαστέρω τῶν Περσέων οἰκέουσι καὶ πρὸς νότου ἀνέμου 
καὶ Δαρείου βασιλέος οὐδαμὰ ὑπήκουσαν. 
Go to Context



Herodotus Hist., Historiae 
Book 3, section 102, line 2

                                               Ἄλλοι δὲ τῶν 
Ἰνδῶν Κασπατύρῳ τε πόλι καὶ τῇ Πακτυϊκῇ χώρῃ εἰσὶ 
πρόσοικοι, πρὸς ἄρκτου τε καὶ βορέω ἀνέμου κατοικημένοι 
τῶν ἄλλων Ἰνδῶν, οἳ Βακτρίοισι παραπλησίην ἔχουσι δίαι-
ταν. 
Go to Context



Herodotus Hist., Historiae 
Book 3, section 102, line 5

     Οὗτοι καὶ μαχιμώτατοί εἰσι Ἰνδῶν καὶ οἱ ἐπὶ τὸν 
χρυσὸν στελλόμενοί εἰσι οὗτοι· κατὰ γὰρ τοῦτό ἐστι ἐρημίη 
διὰ τὴν ψάμμον. 
Go to Context



Herodotus Hist., Historiae 
Book 3, section 102, line 16

           Ἐπὶ δὴ ταύτην τὴν ψάμμον στέλλονται ἐς τὴν 
ἔρημον οἱ Ἰνδοί, ζευξάμενος ἕκαστος καμήλους τρεῖς, σειρη-  
φόρον μὲν ἑκατέρωθεν ἔρσενα παρέλκειν, θήλεαν δὲ ἐς 
μέσον· ἐπὶ ταύτην δὴ αὐτὸς ἀναβαίνει, ἐπιτηδεύσας ὅκως 
ἀπὸ τέκνων ὡς νεωτάτων ἀποσπάσας ζεύξει· αἱ γάρ σφι 
κάμηλοι ἵππων οὐκ ἥσσονες ἐς ταχυτῆτά εἰσι· χωρὶς δὲ 
ἄχθεα δυνατώτεραι πολλὸν φέρειν. 
Go to Context



Herodotus Hist., Historiae 
Book 3, section 104, line 1

                          Οἱ δὲ δὴ Ἰνδοὶ τρόπῳ τοιούτῳ καὶ 
ζεύξι τοιαύτῃ χρεώμενοι ἐλαύνουσι ἐπὶ τὸν χρυσὸν λελογις-
μένως ὅκως [ἂν] καυμάτων τῶν θερμοτάτων ἐόντων ἔσον-
ται ἐν τῇ ἁρπαγῇ· ὑπὸ γὰρ τοῦ καύματος οἱ μύρμηκες 
ἀφανέες γίνονται ὑπὸ γῆν. 
Go to Context



Herodotus Hist., Historiae 
Book 3, section 104, line 12

                                Θερμότατος δέ ἐστι ὁ ἥλιος 
τούτοισι τοῖσι ἀνθρώποισι τὸ ἑωθινόν, οὐ κατά περ τοῖσι 
ἄλλοισι μεσαμβρίης, ἀλλ' ὑπερτείλας μέχρις οὗ ἀγορῆς δια-
λύσιος· τοῦτον δὲ τὸν χρόνον καίει πολλῷ μᾶλλον ἢ τῇ 
μεσαμβρίῃ τὴν Ἑλλάδα, οὕτω ὥστε ἐν ὕδατι λόγος αὐτούς 
ἐστι βρέχεσθαι τηνικαῦτα· μεσοῦσα δὲ ἡ ἡμέρη σχεδὸν 
παραπλησίως καίει τούς <τε> ἄλλους ἀνθρώπους καὶ τοὺς 
Ἰνδούς· ἀποκλινομένης δὲ τῆς μεσαμβρίης γίνεταί σφι ὁ   
ἥλιος κατά περ τοῖσι ἄλλοισι ὁ ἑωθινός· καὶ τὸ ἀπὸ τούτου 
ἀπιὼν ἐπὶ μᾶλλον ψύχει, ἐς ὃ ἐπὶ δυσμῇσι ἐὼν καὶ τὸ 
κάρτα ψύχει. 
Go to Context



Herodotus Hist., Historiae 
Book 3, section 105, line 1

               Ἐπεὰν δὲ ἔλθωσι ἐς τὸν χῶρον οἱ Ἰνδοὶ 
ἔχοντες θυλάκια, ἐμπλήσαντες ταῦτα τῆς ψάμμου τὴν 
ταχίστην ἐλαύνουσι ὀπίσω· αὐτίκα γὰρ οἱ μύρμηκες ὀδμῇ, 
ὡς δὴ λέγεται ὑπὸ Περσέων, μαθόντες διώκουσι. 
Go to Context



Herodotus Hist., Historiae 
Book 3, section 105, line 6

                                                       Εἶναι δὲ 
ταχυτῆτα οὐδενὶ ἑτέρῳ ὅμοιον, οὕτω ὥστε, εἰ μὴ προλαμ-
βάνειν τοὺς Ἰνδοὺς τῆς ὁδοῦ ἐν ᾧ τοὺς μύρμηκας συλλέγε-
σθαι, οὐδένα ἄν σφεων ἀποσῴζεσθαι. 
Go to Context



Herodotus Hist., Historiae 
Book 3, section 105, line 12

                  Τὸν μὲν δὴ πλέω τοῦ χρυσοῦ οὕτω οἱ 
Ἰνδοὶ κτῶνται, ὡς Πέρσαι φασί· ἄλλος δὲ σπανιώτερός 
ἐστι ἐν τῇ χώρῃ ὀρυσσόμενος. 
Go to Context



Herodotus Hist., Historiae 
Book 3, section 106, line 4

             Τοῦτο μὲν γὰρ πρὸς τὴν ἠῶ ἐσχάτη τῶν οἰκεο-
μένων ἡ Ἰνδική ἐστι, ὥσπερ ὀλίγῳ πρότερον εἴρηκα· ἐν 
ταύτῃ τοῦτο μὲν τὰ ἔμψυχα, <τὰ> τετράποδά τε καὶ τὰ 
πετεινά, πολλῷ μέζω ἢ ἐν τοῖσι ἄλλοισι χωρίοισί ἐστι, πά-
ρεξ τῶν ἵππων (οὗτοι δὲ ἑσσοῦνται ὑπὸ τῶν Μηδικῶν,   
Νησαίων δὲ καλεομένων ἵππων), τοῦτο δὲ χρυσὸς ἄπλετος 
αὐτόθι ἐστί, ὁ μὲν ὀρυσσόμενος, ὁ δὲ καταφορεόμενος ὑπὸ 
[τῶν] ποταμῶν, ὁ δὲ ὥσπερ ἐσήμηνα ἁρπαζόμενος. 
Go to Context



Herodotus Hist., Historiae 
Book 3, section 106, line 12

                                                         Τὰ δὲ 
δένδρεα τὰ ἄγρια αὐτόθι φέρει καρπὸν εἴρια καλλονῇ τε 
προφέροντα καὶ ἀρετῇ τῶν ἀπὸ τῶν ὀΐων· καὶ ἐσθῆτι Ἰνδοὶ 
ἀπὸ τούτων τῶν δενδρέων χρέωνται. 
Go to Context



Herodotus Hist., Historiae 
Book 4, section 40, line 7

       Μέχρι δὲ τῆς Ἰνδικῆς οἰκέεται [ἡ] Ἀσίη· τὸ δὲ   
ἀπὸ ταύτης ἔρημος ἤδη τὸ πρὸς τὴν ἠῶ, οὐδὲ ἔχει οὐδεὶς 
φράσαι οἷον δή τι ἐστί. 
Go to Context



Herodotus Hist., Historiae 
Book 4, section 44, line 2

Τῆς δὲ Ἀσίης τὰ πολλὰ ὑπὸ Δαρείου ἐξευρέθη, ὃς 
βουλόμενος Ἰνδὸν ποταμόν, ὃς κροκοδείλους δεύτερος 
οὗτος ποταμῶν πάντων παρέχεται, τοῦτον τὸν ποταμὸν 
εἰδέναι τῇ ἐς θάλασσαν ἐκδιδοῖ, πέμπει πλοίοισι ἄλλους τε   
τοῖσι ἐπίστευε τὴν ἀληθείην ἐρέειν καὶ δὴ καὶ Σκύλακα 
ἄνδρα Καρυανδέα. 
Go to Context



Herodotus Hist., Historiae 
Book 4, section 44, line 12

                                                         Μετὰ 
δὲ τούτους περιπλώσαντας Ἰνδούς τε κατεστρέψατο 
Δαρεῖος καὶ τῇ θαλάσσῃ ταύτῃ ἐχρᾶτο. 
Go to Context



Herodotus Hist., Historiae 
Book 5, section 3, line 1

Θρηίκων δὲ ἔθνος μέγιστόν ἐστι μετά γε Ἰνδοὺς πάντων 
ἀνθρώπων· εἰ δὲ ὑπ' ἑνὸς ἄρχοιτο ἢ φρονέοι κατὰ τὠυτό, 
ἄμαχόν τ' ἂν εἴη καὶ πολλῷ κράτιστον πάντων ἐθνέων κατὰ 
γνώμην τὴν ἐμήν· ἀλλὰ γὰρ τοῦτο ἄπορόν σφι καὶ ἀμήχανον 
μή κοτε ἐγγένηται· εἰσὶ δὴ κατὰ τοῦτο ἀσθενέες. 
Go to Context



Herodotus Hist., Historiae 
Book 7, section 9, line 6

                                                   Καὶ γὰρ δεινὸν 
ἂν εἴη πρῆγμα, εἰ Σάκας μὲν καὶ Ἰνδοὺς καὶ Αἰθίοπάς τε 
καὶ Ἀσσυρίους ἄλλα τε ἔθνεα πολλὰ καὶ μεγάλα, ἀδικήσαντα 
Πέρσας οὐδέν, ἀλλὰ δύναμιν προσκτᾶσθαι βουλόμενοι, 
καταστρεψάμενοι δούλους ἔχομεν, Ἕλληνας δὲ ὑπάρξαντας 
ἀδικίης οὐ τιμωρησόμεθα. 
Go to Context



Herodotus Hist., Historiae 
Book 7, section 65, line 1

                                                   Βακτρίων 
δὲ καὶ Σακέων ἦρχε Ὑστάσπης ὁ Δαρείου τε καὶ Ἀτόσσης 
τῆς Κύρου. Ἰνδοὶ δὲ εἵματα μὲν ἐνδεδυκότες ἀπὸ ξύλων 
πεποιημένα, τόξα δὲ καλάμινα εἶχον καὶ ὀϊστοὺς καλαμί-
νους· ἐπὶ δὲ σίδηρος ἦν· ἐσταλμένοι μὲν δὴ ἦσαν οὕτω 
Ἰνδοί, προσετετάχατο δὲ συστρατευόμενοι Φαρναζάθρῃ τῷ 
Ἀρταβάτεω. 
Go to Context



Herodotus Hist., Historiae 
Book 7, section 65, line 4

              Ἰνδοὶ δὲ εἵματα μὲν ἐνδεδυκότες ἀπὸ ξύλων 
πεποιημένα, τόξα δὲ καλάμινα εἶχον καὶ ὀϊστοὺς καλαμί-
νους· ἐπὶ δὲ σίδηρος ἦν· ἐσταλμένοι μὲν δὴ ἦσαν οὕτω 
Ἰνδοί, προσετετάχατο δὲ συστρατευόμενοι Φαρναζάθρῃ τῷ 
Ἀρταβάτεω. 
Go to Context



Herodotus Hist., Historiae 
Book 7, section 70, line 3

             Τῶν μὲν δὴ ὑπὲρ Αἰγύπτου Αἰθιόπων καὶ 
Ἀραβίων ἦρχε Ἀρσάμης, οἱ δὲ ἀπὸ ἡλίου ἀνατολέων 
Αἰθίοπες (διξοὶ γὰρ δὴ ἐστρατεύοντο) προσετετάχατο τοῖσι 
Ἰνδοῖσι, διαλλάσσοντες εἶδος μὲν οὐδὲν τοῖσι ἑτέροισι, 
φωνὴν δὲ καὶ τρίχωμα μοῦνον· οἱ μὲν γὰρ ἀπὸ ἡλίου Αἰ-
θίοπες ἰθύτριχές εἰσι, οἱ δ' ἐκ τῆς Λιβύης οὐλότατον τρί-
χωμα ἔχουσι πάντων ἀνθρώπων. 
Go to Context



Herodotus Hist., Historiae 
Book 7, section 70, line 7

                                  Οὗτοι δὲ οἱ ἐκ τῆς Ἀσίης 
Αἰθίοπες τὰ μὲν πλέω κατά περ Ἰνδοὶ ἐσεσάχατο, προμε-
τωπίδια δὲ ἵππων εἶχον ἐπὶ τῇσι κεφαλῇσι σύν τε τοῖσι 
ὠσὶ ἐκδεδαρμένα καὶ τῇ λοφιῇ· καὶ ἀντὶ μὲν λόφου ἡ λοφιὴ 
κατέχρα, τὰ δὲ ὦτα τῶν ἵππων ὀρθὰ πεπηγότα εἶχον· προ-
βλήματα δὲ ἀντ' ἀσπίδων ἐποιεῦντο γεράνων δοράς. 
Go to Context



Herodotus Hist., Historiae 
Book 7, section 86, line 2

                                                 Μῆδοι δὲ τήν 
περ ἐν τῷ πεζῷ εἶχον σκευήν· καὶ Κίσσιοι ὡσαύτως. Ἰνδοὶ 
δὲ σκευῇ μὲν ἐσεσάχατο τῇ αὐτῇ καὶ ἐν τῷ πεζῷ, ἤλαυνον 
δὲ κέλητας καὶ ἅρματα· ὑπὸ δὲ τοῖσι ἅρμασι ὑπῆσαν ἵπποι 
καὶ ὄνοι ἄγριοι. 
Go to Context



Herodotus Hist., Historiae 
Book 7, section 187, line 5

           Γυναικῶν δὲ σιτοποιῶν καὶ παλλακέων καὶ 
εὐνούχων οὐδεὶς ἂν εἴποι ἀτρεκέα ἀριθμόν· οὐδ' αὖ ὑποζυ-
γίων τε καὶ τῶν ἄλλων κτηνέων τῶν ἀχθοφόρων καὶ κυνῶν 
Ἰνδικῶν τῶν ἑπομένων, οὐδ' ἂν τούτων ὑπὸ πλήθεος οὐδεὶς 
ἂν εἴποι ἀριθμόν. 
Go to Context



Herodotus Hist., Historiae 
Book 8, section 113, line 12

           Ὡς δὲ ἀπίκατο ἐς τὴν Θεσσαλίην, ἐνθαῦτα 
Μαρδόνιος ἐξελέγετο πρώτους μὲν τοὺς μυρίους Πέρσας 
τοὺς Ἀθανάτους καλεομένους, πλὴν Ὑδάρνεος τοῦ στρατη-
γοῦ (οὗτος γὰρ οὐκ ἔφη λείψεσθαι βασιλέος), μετὰ δὲ τῶν 
ἄλλων Περσέων τοὺς θωρηκοφόρους καὶ τὴν ἵππον τὴν 
χιλίην, καὶ Μήδους τε καὶ Σάκας καὶ Βακτρίους [τε] καὶ 
Ἰνδούς, καὶ τὸν πεζὸν καὶ τὴν ἵππον. 
Go to Context



Herodotus Hist., Historiae 
Book 9, section 31, line 19

                                   Μετὰ δὲ Βακτρίους 
ἔστησε Ἰνδούς· οὗτοι δὲ ἐπέσχον Ἑρμιονέας τε καὶ Ἐρε-
τριέας καὶ Στυρέας τε καὶ Χαλκιδέας. 
Go to Context

\end{greek}

\begin{pages}
\begin{Leftside}
\begin{greek}
\beginnumbering
\pstart
\subsection{\textgreek{νομὸς Ἰνδῶν}} % <--
 (94) Παρικάνιοι δὲ καὶ Αἰθίοπες οἱ ἐκ τῆς Ἀσίης τετρακόσια τάλαντα ἀπαγίνεον· νομὸς ἕβδομος καὶ δέκατος οὗτος. Ματιηνοῖσι δὲ καὶ Σάσπειρσι καὶ Ἀλαροδίοισι διηκόσια ἐπετέτακτο τάλαντα· νομὸς ὄγδοος καὶ δέκατος οὗτος. Μόσχοισι δὲ καὶ Τιβαρηνοῖσι καὶ Μάκρωσι καὶ Μοσσυνοίκοισι καὶ Μαρσὶ τριηκόσια τάλαντα προείρητο· νομὸς εἴνατος καὶ δέκατος οὗτος Ἰνδῶν δὲ πλῆθός τε πολλῷ πλεῖστόν ἐστι πάντων τῶν ἡμεῖς ἴδμεν ἀνθρώπων καὶ φόρον ἀπαγίνεον πρὸς πάντας τοὺς ἄλλους ἑξήκοντα καὶ τριηκόσια τάλαντα ψήγματος· νομὸς εἰκοστὸς οὗτος. \pend

\pstart (95) Τὸ μὲν δὴ ἀργύριον τὸ Βαβυλώνιον πρὸς τὸ Εὐβοϊκὸν συμβαλλόμενον τάλαντον γίνεται τεσσεράκοντα καὶ πεντακόσια καὶ εἰνακισχίλια τάλαντα. Τὸ δὲ χρυσίον τρισκαιδεκαστάσιον λογιζομένων, τὸ ψῆγμα εὑρίσκεται ἐὸν Εὐβοϊκῶν ταλάντων ὀγδώκοντα καὶ ἑξακοσίων καὶ τετρακισχιλίων. Τούτων ὦν πάντων συντιθέμενον τὸ πλῆθος Εὐβοϊκὰ τάλαντα συνελέγετο ἐς τὸν ἐπέτειον φόρον Δαρείῳ μύρια καὶ τετρακισχίλια καὶ πεντακόσια καὶ ἑξήκοντα· τὸ δ' ἔτι   τούτων ἔλασσον ἀπιεὶς οὐ λέγω. (96) Οὗτος Δαρείῳ προσήιε φόρος ἀπὸ τῆς τε Ἀσίης καὶ τῆς Λιβύης ὀλιγαχόθεν· προϊόντος μέντοι τοῦ χρόνου καὶ ἀπὸ νήσων προσήιε ἄλλος φόρος καὶ τῶν ἐν τῇ Εὐρώπῃ μέχρι Θεσσαλίης οἰκημένων. Τοῦτον τὸν φόρον θησαυρίζει βασιλεὺς τρόπῳ τοιῷδε· ἐς πίθους κεραμίνους τήξας καταχέει, πλήσας δὲ τὸ ἄγγος περιαιρέει τὸν κέραμον· ἐπεὰν δὲ δεηθῇ χρημάτων, κατακόπτει τοσοῦτο ὅσου ἂν ἑκάστοτε δέηται. \pend

\pstart (97) Αὗται μέν νυν ἀρχαί τε ἦσαν καὶ φόρων ἐπιτάξιες. Ἡ Περσὶς δὲ χώρη μούνη μοι οὐκ εἴρηται δασμοφόρος· ἀτελέα γὰρ Πέρσαι νέμονται χώρην. Οἵδε δὲ φόρον μὲν οὐδένα ἐτάχθησαν φέρειν, δῶρα δὲ ἀγίνεον· Αἰθίοπες οἱ πρόσουροι Αἰγύπτῳ, τοὺς Καμβύσης ἐλαύνων ἐπὶ τοὺς μακροβίους Αἰθίοπας κατεστρέψατο, ... οἳ περί τε Νύσην τὴν ἱρὴν κατοίκηνται καὶ τῷ Διονύσῳ ἀνάγουσι τὰς ὁρτάς· [οὗτοι οἱ Αἰθίοπες καὶ οἱ πλησιόχωροι τούτοισι σπέρματι μὲν χρέωνται τῷ αὐτῷ τῷ καὶ οἱ Καλλαντίαι Ἰνδοί, οἰκήματα δὲ ἔκτηνται κατάγαια]· οὗτοι συναμφότεροι διὰ τρίτου ἔτεος ἀγίνεον, ἀγινέουσι δὲ καὶ τὸ μέχρις ἐμέο, δύο   χοίνικας ἀπύρου χρυσίου καὶ διηκοσίας φάλαγγας ἐβένου καὶ πέντε παῖδας Αἰθίοπας καὶ ἐλέφαντος ὀδόντας μεγάλους εἴκοσι. Κόλχοι δὲ ταξάμενοι ἐς τὴν δωρεὴν καὶ οἱ προσεχέες μέχρι Καυκάσιος ὄρεος (ἐς τοῦτο γὰρ τὸ ὄρος ὑπὸ Πέρσῃσι ἄρχεται, τὰ δὲ πρὸς βορέην ἄνεμον τοῦ Καυκάσιος Περσέων οὐδὲν ἔτι φροντίζει), οὗτοι ὦν δῶρα τὰ ἐτάξαντο ἔτι καὶ ἐς ἐμὲ διὰ πεντετηρίδος ἀγίνεον, ἑκατὸν παῖδας καὶ ἑκατὸν παρθένους. Ἀράβιοι δὲ χίλια τάλαντα ἀγίνεον λιβανωτοῦ ἀνὰ πᾶν ἔτος. Ταῦτα μὲν οὗτοι δῶρα πάρεξ τοῦ φόρου βασιλέϊ ἐκόμιζον. \pend

\pstart (98) Τὸν δὲ χρυσὸν τοῦτον τὸν πολλὸν οἱ Ἰνδοί, ἀπ' οὗ τὸ ψῆγμα τῷ βασιλέϊ τὸ εἰρημένον κομίζουσι, τρόπῳ τοιῷδε κτῶνται. Ἔστι τῆς Ἰνδικῆς χώρης τὸ πρὸς ἥλιον ἀνίσχοντα ψάμμος· τῶν γὰρ ἡμεῖς ἴδμεν, τῶν καὶ πέρι ἀτρεκές τι λέγεται, πρῶτοι πρὸς ἠῶ καὶ ἡλίου ἀνατολὰς οἰκέουσι ἀνθρώπων τῶν ἐν τῇ Ἀσίῃ Ἰνδοί· Ἰνδῶν γὰρ τὸ πρὸς τὴν ἠῶ ἐρημίη ἐστὶ διὰ τὴν ψάμμον.  Ἔστι δὲ πολλὰ ἔθνεα Ἰνδῶν καὶ οὐκ ὁμόφωνα σφίσι, καὶ οἱ μὲν αὐτῶν νομάδες εἰσί, οἱ δὲ οὔ, οἱ δὲ ἐν τοῖσι ἕλεσι οἰκέουσι τοῦ ποταμοῦ καὶ ἰχθῦς σιτέονται ὠμούς, τοὺς αἱρέουσι ἐκ πλοίων καλαμίνων ὁρμώμενοι· καλάμου δὲ ἓν γόνυ πλοῖον ἕκαστον ποιέεται. Οὗτοι μὲν δὴ τῶν Ἰνδῶν φορέουσι ἐσθῆτα φλοΐνην· ἐπεὰν ἐκ τοῦ ποταμοῦ φλοῦν   ἀμήσωνται καὶ κόψωσι, τὸ ἐνθεῦτεν φορμοῦ τρόπον καταπλέξαντες ὡς θώρηκα ἐνδύνουσι. (99) Ἄλλοι δὲ τῶν Ἰνδῶν πρὸς ἠῶ οἰκέοντες τούτων νομάδες εἰσί, κρεῶν ἐδεσταὶ ὠμῶν, καλέονται δὲ Παδαῖοι. Νομαίοισι δὲ τοιοισίδε λέγονται χρᾶσθαι. Ὃς ἂν κάμῃ τῶν ἀστῶν, ἤν τε γυνὴ ἤν τε ἀνήρ, τὸν μὲν ἄνδρα ἄνδρες οἱ μάλιστά οἱ ὁμιλέοντες κτείνουσι, φάμενοι αὐτὸν τηκόμενον τῇ νούσῳ τὰ κρέα σφίσι διαφθείρεσθαι· ὁ δὲ ἄπαρνός ἐστι μὴ μὲν νοσέειν, οἱ δὲ οὐ συγγινωσκόμενοι ἀποκτείναντες κατευωχέονται· ἣ δὲ ἂν γυνὴ κάμῃ, ὡσαύτως αἱ ἐπιχρεώμεναι μάλιστα γυναῖκες ταὐτὰ τοῖσι ἀνδράσι ποιεῦσι. Τὸν γὰρ δὴ ἐς γῆρας ἀπικόμενον θύσαντες κατευωχέονται. Ἐς δὲ τούτου λόγον οὐ πολλοί τινες αὐτῶν ἀπικνέονται· πρὸ γὰρ τοῦ τὸν ἐς νοῦσον πίπτοντα πάντα κτείνουσι. (100) Ἑτέρων δέ ἐστι Ἰνδῶν ὅδε ἄλλος τρόπος· οὔτε κτείνουσι οὐδὲν ἔμψυχον οὔτε τι σπείρουσι οὔτε οἰκίας νομίζουσι ἐκτῆσθαι ποιηφαγέουσί τε, καὶ αὐτοῖσι <ὄσπριόν τι> ἔστι ὅσον κέγχρος τὸ μέγαθος ἐν κάλυκι, αὐτόματον ἐκ τῆς γῆς γινόμενον, τὸ συλλέγοντες αὐτῇ τῇ κάλυκι ἕψουσί τε καὶ σιτέονται. Ὃς δ' ἂν ἐς νοῦσον αὐτῶν πέσῃ, ἐλθὼν ἐς τὴν ἔρημον κεῖται· φροντίζει δὲ οὐδεὶς οὔτε ἀποθανόντος οὔτε κάμνοντος. (101) Μίξις δὲ τούτων τῶν Ἰνδῶν τῶν κατέλεξα πάντων ἐμφανής ἐστι   κατά περ τῶν προβάτων, καὶ τὸ χρῶμα φορέουσι ὅμοιον πάντες καὶ παραπλήσιον Αἰθίοψι. Ἡ γονὴ δὲ αὐτῶν, τὴν ἀπίενται ἐς τὰς γυναῖκας, οὐ κατά περ τῶν ἄλλων ἀνθρώπων ἐστὶ λευκή, ἀλλὰ μέλαινα κατά περ τὸ χρῶμα· τοιαύτην δὲ καὶ Αἰθίοπες ἀπίενται θορήν. Οὗτοι μὲν τῶν Ἰνδῶν ἑκαστέρω τῶν Περσέων οἰκέουσι καὶ πρὸς νότου ἀνέμου καὶ Δαρείου βασιλέος οὐδαμὰ ὑπήκουσαν. (102) Ἄλλοι δὲ τῶν Ἰνδῶν Κασπατύρῳ τε πόλι καὶ τῇ Πακτυϊκῇ χώρῃ εἰσὶ πρόσοικοι, πρὸς ἄρκτου τε καὶ βορέω ἀνέμου κατοικημένοι τῶν ἄλλων Ἰνδῶν, οἳ Βακτρίοισι παραπλησίην ἔχουσι δίαιταν. Οὗτοι καὶ μαχιμώτατοί εἰσι Ἰνδῶν καὶ οἱ ἐπὶ τὸν χρυσὸν στελλόμενοί εἰσι οὗτοι· κατὰ γὰρ τοῦτό ἐστι ἐρημίη διὰ τὴν ψάμμον. \pend

\pstart  Ἐν δὴ ὦν τῇ ἐρημίῃ ταύτῃ καὶ τῇ ψάμμῳ γίνονται μύρμηκες μεγάθεα ἔχοντες κυνῶν μὲν ἐλάσσω, ἀλωπέκων δὲ μέζω· εἰσὶ γὰρ αὐτῶν καὶ παρὰ βασιλέϊ τῷ Περσέων ἐνθεῦτεν θηρευθέντες. Οὗτοι ὦν οἱ μύρμηκες ποιεύμενοι οἴκησιν ὑπὸ γῆν ἀναφέρουσι [τὴν] ψάμμον κατά περ οἱ ἐν τοῖσι Ἕλλησι μύρμηκες κατὰ τὸν αὐτὸν τρόπον, εἰσὶ δὲ καὶ τὸ εἶδος ὁμοιότατοι· ἡ δὲ ψάμμος ἡ ἀναφερομένη ἐστὶ χρυσῖτις. Ἐπὶ δὴ ταύτην τὴν ψάμμον στέλλονται ἐς τὴν ἔρημον οἱ Ἰνδοί, ζευξάμενος ἕκαστος καμήλους τρεῖς, σειρη-   φόρον μὲν ἑκατέρωθεν ἔρσενα παρέλκειν, θήλεαν δὲ ἐς μέσον· ἐπὶ ταύτην δὴ αὐτὸς ἀναβαίνει, ἐπιτηδεύσας ὅκως ἀπὸ τέκνων ὡς νεωτάτων ἀποσπάσας ζεύξει· αἱ γάρ σφι κάμηλοι ἵππων οὐκ ἥσσονες ἐς ταχυτῆτά εἰσι· χωρὶς δὲ ἄχθεα δυνατώτεραι πολλὸν φέρειν. (103) Τὸ μὲν δὴ εἶδος ὁκοῖόν τι ἔχει ἡ κάμηλος, ἐπισταμένοισι τοῖσι Ἕλλησι οὐ συγγράφω· τὸ δὲ μὴ ἐπιστέαται αὐτῆς, τοῦτο φράσω· κάμηλος ἐν τοῖσι ὀπισθίοισι σκέλεσι ἔχει τέσσερας μηροὺς καὶ γούνατα τέσσερα, τά τε αἰδοῖα διὰ τῶν ὀπισθίων σκελέων πρὸς τὴν οὐρὴν τετραμμένα. (104) Οἱ δὲ δὴ Ἰνδοὶ τρόπῳ τοιούτῳ καὶ ζεύξι τοιαύτῃ χρεώμενοι ἐλαύνουσι ἐπὶ τὸν χρυσὸν λελογιςμένως ὅκως [ἂν] καυμάτων τῶν θερμοτάτων ἐόντων ἔσονται ἐν τῇ ἁρπαγῇ· ὑπὸ γὰρ τοῦ καύματος οἱ μύρμηκες ἀφανέες γίνονται ὑπὸ γῆν. Θερμότατος δέ ἐστι ὁ ἥλιος τούτοισι τοῖσι ἀνθρώποισι τὸ ἑωθινόν, οὐ κατά περ τοῖσι ἄλλοισι μεσαμβρίης, ἀλλ' ὑπερτείλας μέχρις οὗ ἀγορῆς διαλύσιος· τοῦτον δὲ τὸν χρόνον καίει πολλῷ μᾶλλον ἢ τῇ μεσαμβρίῃ τὴν Ἑλλάδα, οὕτω ὥστε ἐν ὕδατι λόγος αὐτούς ἐστι βρέχεσθαι τηνικαῦτα· μεσοῦσα δὲ ἡ ἡμέρη σχεδὸν παραπλησίως καίει τούς <τε> ἄλλους ἀνθρώπους καὶ τοὺς Ἰνδούς· ἀποκλινομένης δὲ τῆς μεσαμβρίης γίνεταί σφι ὁ   ἥλιος κατά περ τοῖσι ἄλλοισι ὁ ἑωθινός· καὶ τὸ ἀπὸ τούτου ἀπιὼν ἐπὶ μᾶλλον ψύχει, ἐς ὃ ἐπὶ δυσμῇσι ἐὼν καὶ τὸ κάρτα ψύχει. (105) Ἐπεὰν δὲ ἔλθωσι ἐς τὸν χῶρον οἱ Ἰνδοὶ ἔχοντες θυλάκια, ἐμπλήσαντες ταῦτα τῆς ψάμμου τὴν ταχίστην ἐλαύνουσι ὀπίσω· αὐτίκα γὰρ οἱ μύρμηκες ὀδμῇ, ὡς δὴ λέγεται ὑπὸ Περσέων, μαθόντες διώκουσι. Εἶναι δὲ ταχυτῆτα οὐδενὶ ἑτέρῳ ὅμοιον, οὕτω ὥστε, εἰ μὴ προλαμβάνειν τοὺς Ἰνδοὺς τῆς ὁδοῦ ἐν ᾧ τοὺς μύρμηκας συλλέγεσθαι, οὐδένα ἄν σφεων ἀποσῴζεσθαι. Τοὺς μέν νυν ἔρσενας τῶν καμήλων, εἶναι γὰρ ἥσσονας θέειν τῶν θηλέων, παραλύεσθαι ἐπελκομένους, οὐκ ὁμοῦ ἀμφοτέρους· τὰς δὲ θηλέας ἀναμιμνησκομένας τῶν ἔλιπον τέκνων ἐνδιδόναι μαλακὸν οὐδέν. Τὸν μὲν δὴ πλέω τοῦ χρυσοῦ οὕτω οἱ Ἰνδοὶ κτῶνται, ὡς Πέρσαι φασί· ἄλλος δὲ σπανιώτερός ἐστι ἐν τῇ χώρῃ ὀρυσσόμενος. \pend

\pstart (106) Αἱ δ' ἐσχατιαί κως τῆς οἰκεομένης τὰ κάλλιστα ἔλαχον, κατά περ ἡ Ἑλλὰς τὰς ὥρας πολλόν τι κάλλιστα κεκρημένας ἔλαχε. Τοῦτο μὲν γὰρ πρὸς τὴν ἠῶ ἐσχάτη τῶν οἰκεομένων ἡ Ἰνδική ἐστι, ὥσπερ ὀλίγῳ πρότερον εἴρηκα· ἐν ταύτῃ τοῦτο μὲν τὰ ἔμψυχα, <τὰ> τετράποδά τε καὶ τὰ πετεινά, πολλῷ μέζω ἢ ἐν τοῖσι ἄλλοισι χωρίοισί ἐστι, πάρεξ τῶν ἵππων (οὗτοι δὲ ἑσσοῦνται ὑπὸ τῶν Μηδικῶν,   Νησαίων δὲ καλεομένων ἵππων), τοῦτο δὲ χρυσὸς ἄπλετος αὐτόθι ἐστί, ὁ μὲν ὀρυσσόμενος, ὁ δὲ καταφορεόμενος ὑπὸ [τῶν] ποταμῶν, ὁ δὲ ὥσπερ ἐσήμηνα ἁρπαζόμενος. Τὰ δὲ δένδρεα τὰ ἄγρια αὐτόθι φέρει καρπὸν εἴρια καλλονῇ τε προφέροντα καὶ ἀρετῇ τῶν ἀπὸ τῶν ὀΐων· καὶ ἐσθῆτι Ἰνδοὶ ἀπὸ τούτων τῶν δενδρέων χρέωνται. \pend
\endnumbering% <--
\end{greek}
\end{Leftside}
\begin{Rightside}
\beginnumbering
\pstart
\subsection{Law of the Indians}
 (94) The Paricanians and Ethiopians in Asia brought in four hundred talents: this is the seventeenth division. To the Matienians and Saspeirians and Alarodians was appointed a tribute of two hundred talents: this is the eighteenth division. To the Moschoi and Tibarenians and Macronians and Mossynoicoi and Mares three hundred talents were ordered: this is the nineteenth division. Of the Indians the number is far greater than that of any other race of men of whom we know; and they brought in a tribute larger than all the rest, that is to say three hundred and sixty talents of gold-dust: this is the twentieth division.\pend

\pstart (95) Now if we compare Babylonian with Euboïc talents, the silver is found to amount to nine thousand eight hundred and eighty 82 talents; and if we reckon the gold at thirteen times the value of silver, weight for weight, the gold-dust is found to amount to four thousand six hundred and eighty Euboïc talents. These being all added together, the total which was collected as yearly tribute for Dareios amounts to fourteen thousand five hundred and sixty Euboïc talents: the sums which are less than these 83 I pass over and do not mention. (96) This was the tribute which came in to Dareios from Asia and from a small part of Libya: but as time went on, other tribute came in also from the islands and from those who dwell in Europe as far as Thessaly. This tribute the king stores up in his treasury in the following manner:—he melts it down and pours it into jars of earthenware, and when he has filled the jars he takes off the earthenware jar from the metal; and when he wants money he cuts off so much as he needs on each occasion.\pend

\pstart (97) These were the provinces and the assessments of tribute: and the Persian land alone has not been mentioned by me as paying a contribution, for the Persians have their land to dwell in free from payment. The following moreover had no tribute fixed for them to pay, but brought gifts, namely the Ethiopians who border upon Egypt, whom Cambyses subdued as he marched against the Long-lived Ethiopians, those 84 who dwell about Nysa, which is called "sacred," and who celebrate the festivals in honour of Dionysos: these Ethiopians and those who dwell near them have the same kind of seed as the Callantian Indians, and they have underground dwellings. 85 These both together brought every other year, and continue to bring even to my own time, two quart measures 86 of unmelted gold and two hundred blocks of ebony and five Ethiopian boys and twenty large elephant tusks. The Colchians also had set themselves among those who brought gifts, and with them those who border upon them extending as far as the range of the Caucasus (for the Persian rule extends as far as these mountains, but those who dwell in the parts beyond Caucasus toward the North Wind regard the Persians no longer),—these, I say, continued to bring the gifts which they had fixed for themselves every four years 87 even down to my own time, that is to say, a hundred boys and a hundred maidens. Finally, the Arabians brought a thousand talents of frankincense every year. Such were the gifts which these brought to the king apart from the tribute.\pend

\pstart (98) Now this great quantity of gold, out of which the Indians bring in to the king the gold-dust which has been mentioned, is obtained by them in a manner which I shall tell:—That part of the Indian land which is towards the rising sun is sand; for of all the peoples in Asia of which we know or about which any certain report is given, the Indians dwell furthest away towards the East and the sunrising; seeing that the country to the East of the Indians is desert on account of the sand. Now there are many tribes of Indians, and they do not agree with one another in language; and some of them are pastoral and others not so, and some dwell in the swamps of the river 88 and feed upon raw fish, which they catch by fishing from boats made of cane; and each boat is made of one joint of cane. These Indians of which I speak wear clothing made of rushes: they gather and cut the rushes from the river and then weave them together into a kind of mat and put it on like a corslet. (99) Others of the Indians, dwelling to the East of these, are pastoral and eat raw flesh: these are called Padaians, and they practise the following customs:—whenever any of their tribe falls ill, whether it be a woman or a man, if a man then the men who are his nearest associates put him to death, saying that he is wasting away with the disease and his flesh is being spoilt for them: 89 and meanwhile he denies stoutly and says that he is not ill, but they do not agree with him; and after they have killed him they feast upon his flesh: but if it be a woman who falls ill, the women who are her greatest intimates do to her in the same manner as the men do in the other case. For 90 in fact even if a man has come to old age they slay him and feast upon him; but very few of them come to be reckoned as old, for they kill every one who falls into sickness, before he reaches old age. (100) Other Indians have on the contrary a manner of life as follows:—they neither kill any living thing nor do they sow any crops nor is it their custom to possess houses; but they feed on herbs, and they have a grain of the size of millet, in a sheath, which grows of itself from the ground; this they gather and boil with the sheath, and make it their food: and whenever any of them falls into sickness, he goes to the desert country and lies there, and none of them pay any attention either to one who is dead or to one who is sick. (101) The sexual intercourse of all these Indians of whom I have spoken is open like that of cattle, and they have all one colour of skin, resembling that of the Ethiopians: moreover the seed which they emit is not white like that of other races, but black like their skin; and the Ethiopians also are similar in this respect. These tribes of Indians dwell further off than the Persian power extends, and towards the South Wind, and they never became subjects of Dareios. (102) Others however of the Indians are on the borders of the city of Caspatyros and the country of Pactyïke, dwelling towards the North 91 of the other Indians; and they have a manner of living nearly the same as that of the Bactrians: these are the most warlike of the Indians, and these are they who make expeditions for the gold. \pend

\pstart For in the parts where they live it is desert on account of the sand; and in this desert and sandy tract are produced ants, which are in size smaller than dogs but larger than foxes, for 92 there are some of them kept at the residence of the king of Persia, which are caught here. These ants then make their dwelling under ground and carry up the sand just in the same manner as the ants found in the land of the Hellenes, which they themselves 93 also very much resemble in form; and the sand which is brought up contains gold. To obtain this sand the Indians make expeditions into the desert, each one having yoked together three camels, placing a female in the middle and a male like a trace-horse to draw by each side. On this female he mounts himself, having arranged carefully that she shall be taken to be yoked from young ones, the more lately born the better. For their female camels are not inferior to horses in speed, and moreover they are much more capable of bearing weights. (103) As to the form of the camel, I do not here describe it, since the Hellenes for whom I write are already acquainted with it, but I shall tell that which is not commonly known about it, which is this:—the camel has in the hind legs four thighs and four knees, 94 and its organs of generation are between the hind legs, turned towards the tail. (104) The Indians, I say, ride out to get the gold in the manner and with the kind of yoking which I have described, making calculations so that they may be engaged in carrying it off at the time when the greatest heat prevails; for the heat causes the ants to disappear underground. Now among these nations the sun is hottest in the morning hours, not at midday as with others, but from sunrise to the time of closing the market: and during this time it produces much greater heat than at midday in Hellas, so that it is said that then they drench themselves with water. Midday however has about equal degree of heat with the Indians as with other men, while after midday their sun becomes like the morning sun with other men, and after this, as it goes further away, it produces still greater coolness, until at last at sunset it makes the air very cool indeed. (105) When the Indians have come to the place with bags, they fill them with the sand and ride away back as quickly as they can, for forthwith the ants, perceiving, as the Persians allege, by the smell, begin to pursue them: and this animal, they say, is superior to every other creature in swiftness, so that unless the Indians got a start in their course, while the ants were gathering together, not one of them would escape. So then the male camels, for they are inferior in speed of running to the females, if they drag behind are even let loose 95 from the side of the female, one after the other; 96 the females however, remembering the young which they left behind, do not show any slackness in their course. 97 Thus it is that the Indians get most part of the gold, as the Persians say; there is however other gold also in their land obtained by digging, but in smaller quantities.\pend

\pstart (106) It seems indeed that the extremities of the inhabited world had allotted to them by nature the fairest things, just as it was the lot of Hellas to have its seasons far more fairly tempered than other lands: for first, India is the most distant of inhabited lands towards the East, as I have said a little above, and in this land not only the animals, birds as well as four-footed beasts, are much larger than in other places (except the horses, which are surpassed by those of Media called Nessaian), but also there is gold in abundance there, some got by digging, some brought down by rivers, and some carried off as I explained just now: and there also the trees which grow wild produce wool which surpasses in beauty and excellence that from sheep, and the Indians wear clothing obtained from these trees.
\pend
\endnumbering% <--
\end{Rightside}
\Pages
\end{pages}

\section{Ctesias of Cnidus}

\subsection{About Ctesias}
\blockquote[From Wikipedia]{Ctesias of Cnidus ( /ˈtiːʒəs/; Ancient Greek: Κτησίας) was a Greek physician and historian from Cnidus in Caria. Ctesias, who lived in the 5th century BC, was physician to Artaxerxes Mnemon, whom he accompanied in 401 BC on his expedition against his brother Cyrus the Younger.

Ctesias was the author of treatises on rivers, and on the Persian revenues, of an account of India entitled Indica (which is of value as recording the beliefs of the Persians about India), and of a history of Assyria and Persia in 23 books, called Persica, written in opposition to Herodotus in the Ionic dialect, and professedly founded on the Persian royal archives.}

About \emph{Indica}:
\blockquote[From Wikipedia]{A record of the view of Persians of India, under the title Indica includes descriptions of god like people, philosophers, artisans, unquantifiable gold, among other riches and wonders.[3] The book only remains in fragments and in reports made about the book by later authors.}

\subsection{Ἰνδ-- appearing in testimonia and fragmenta}
Text: Ctesias Hist., Med., Testimonia (0845: 001)
“FGrH \#688”.
Volume-Jacobyʹ-T 3c,688,T, fragment 10, line 1

\begin{greek}
        Bibl. 72 p. 45 a 20: ἀνεγνώσθη δὲ αὐτοῦ καὶ τὰ Ἰνδικὰ ἐν ἑνὶ 
βιβλίωι, ἐν οἷς μᾶλλον ἰωνίζει. 
Go to Context



Ctesias Hist., Med., Testimonia 
Volume-Jacobyʹ-T 3c,688,T, fragment 11b, line 3

– 1, 2, 35: Θεόπομπος (115 F 381) δὲ ἐξομολογεῖται φήσας ὅτι καὶ μύθους 
ἐν ταῖς ἱστορίαις ἐρεῖ κρεῖττον ἢ ὡς Ἡρόδοτος καὶ Κτησίας καὶ Ἑλλάνικος 
καὶ οἱ τὰ Ἰνδικὰ συγγράψαντες. 
Go to Context



Ctesias Hist., Med., Testimonia 
Volume-Jacobyʹ-T 3c,688,T, fragment 11f, line 1

                    8, 28 p. 606a 8: ἐν δὲ τῆι Ἰνδικῆι, 
ὥς φησι Κτησίας (F 45kα), οὐκ ὢν ἀξιόπιστος. 
Go to Context



Ctesias Hist., Med., Testimonia 
Volume-Jacobyʹ-T 3c,688,T, fragment 11h, line 2

            Verae narr. 1, 3: Κτησίας 
ὁ Κτησιόχου ὁ Κνίδιος, ὃς συνέγραψεν περὶ τῆς Ἰνδῶν χώρας καὶ τῶν παρ' 
αὐτοῖς ἃ μήτε αὐτὸς εἶδεν μήτε ἄλλου ἀληθεύοντος ἤκουσεν. 
Go to Context



Ctesias Hist., Med., Testimonia 
Volume-Jacobyʹ-T 3c,688,T, fragment 13, line 6

                      τῶν μέντοι γε μύθων, ἐν οἷς ἐκείνωι λοιδορεῖται, οὐδ' οὗτος 
ἀφίσταται (T 11), καὶ μάλιστα ἐν τοῖς ἐπιγραφομένοις αὐτῶι ’Ἰνδικά’. 
Go to Context



Ctesias Hist., Med., Fragmenta (0845: 002)
“FGrH \#688”.
Volume-Jacobyʹ-F 3c,688,F, fragment 1b, line 37

(2) οὕτω δὲ τῶν πραγμάτων τῶι Νίνωι προχωρούντων, δεινὴν ἐπιθυμίαν 
ἔσχε τοῦ καταστρέψασθαι τὴν Ἀσίαν ἅπασαν τὴν ἐντὸς Τανάιδος καὶ Νείλου· 
ὡς ἐπίπαν γὰρ τοῖς εὐτυχοῦσιν ἡ τῶν †πραγμάτων ἐπίρροια τὴν τοῦ πλείονος ἐπιθυμίαν 
παρίστησι· διόπερ τῆς μὲν Μηδίας σατράπην ἕνα τῶν περὶ αὑτὸν φίλων κατέστησεν, αὐτὸς δ' ἐπήιει τὰ κατὰ τὴν Ἀσίαν ἔθνη καταστρεφόμενος· καὶ χρόνον 
ἑπτακαιδεκαετῆ καταναλώσας πλὴν Ἰνδῶν καὶ Βακτριανῶν τῶν ἄλλων ἁπάντων κύριος ἐγένετο. 
Go to Context



Ctesias Hist., Med., Fragmenta 
Volume-Jacobyʹ-F 3c,688,F, fragment 1b, line 431

                            (2) πυνθανομένη δὲ τὸ τῶν Ἰνδῶν ἔθνος μέγιστον 
εἶναι τῶν κατὰ τὴν οἰκουμένην καὶ πλείστην τε καὶ καλλίστην χώραν 
νέμεσθαι, διενοεῖτο στρατεύειν εἰς τὴν Ἰνδικήν, ἧς ἐβασίλευε μὲν Σταβρο-
βάτης κατ' ἐκείνους τοὺς χρόνους, στρατιωτῶν δ' εἶχεν ἀναρίθμητον πλῆθος· 
ὑπῆρχον δ' αὐτῶι καὶ ἐλέφαντες πολλοὶ καθ' ὑπερβολήν, λαμπρῶς κεκοσμη-
μένοι τοῖς εἰς τὸν πόλεμον καταπληκτικοῖς. 
Go to Context



Ctesias Hist., Med., Fragmenta 
Volume-Jacobyʹ-F 3c,688,F, fragment 1b, line 436

                                                 (3) ἡ γὰρ Ἰνδικὴ χώρα διάφορος 
οὖσα τῶι κάλλει καὶ πολλοῖς διειλημμένη ποταμοῖς, ἀρδεύεταί τε πολλαχοῦ, καὶ 
διττοὺς καθ' ἕκαστον ἐνιαυτὸν ἐκφέρει καρπούς· διὸ καὶ τῶν πρὸς τὸ ζῆν ἐπιτηδείων 
τοσοῦτον ἔχει πλῆθος, ὥστε διὰ παντὸς ἄφθονον ἀπόλαυσιν τοῖς ἐγχωρίοις παρέχεσθαι· 
λέγεται δὲ μηδέποτε κατ' αὐτὴν γεγονέναι σιτοδείαν ἢ φθορὰν καρπῶν διὰ τὴν εὐκρα-
σίαν τῶν τόπων. 
Go to Context



Ctesias Hist., Med., Fragmenta 
Volume-Jacobyʹ-F 3c,688,F, fragment 1b, line 446

                        ὑπὲρ ὧν τὰ κατὰ μέρος ἡ Σεμίραμις ἀκούσασα προήχθη 
μηδὲν προαδικηθεῖσα τὸν πρὸς Ἰνδοὺς ἐξενεγκεῖν πόλεμον. 
Go to Context



Ctesias Hist., Med., Fragmenta 
Volume-Jacobyʹ-F 3c,688,F, fragment 1b, line 454

                                      (7) ὁ γὰρ Ἰνδὸς ποταμός, μέγιστος ὢν τῶν   
περὶ τοὺς τόπους καὶ τὴν βασιλείαν αὐτῆς ὁρίζων, πολλῶν προσεδεῖτο πλοίων 
πρός τε τὴν διάβασιν καὶ πρὸς τὸ τοὺς Ἰνδοὺς ἀπὸ τούτων ἀμύνασθαι· περὶ δὲ τὸν 
ποταμὸν οὐκ οὔσης ὕλης, ἀναγκαῖον ἦν ἐκ τῆς Βακτριανῆς πεζῆι παρακομίζεσθαι 
τὰ πλοῖα. 
Go to Context



Ctesias Hist., Med., Fragmenta 
Volume-Jacobyʹ-F 3c,688,F, fragment 1b, line 460

            (8) θεωροῦσα δ' ἡ Σεμίραμις ἑαυτὴν ἐν τῆι τῶν ἐλεφάντων χρείαι πολὺ 
λειπομένην, ἐπενοήσατο †τι κατασκευάζειν ἰδίωμα† τούτων τῶν ζώιων, ἐλ-
πίζουσα καταπλήξεσθαι τοὺς Ἰνδοὺς διὰ τὸ νομίζειν αὐτοὺς μηδ' εἶναι τὸ 
σύνολον ἐλέφαντας ἐκτὸς τῶν κατὰ τὴν Ἰνδικήν. 
Go to Context



Ctesias Hist., Med., Fragmenta 
Volume-Jacobyʹ-F 3c,688,F, fragment 1b, line 471

                  (10) οἱ δὲ ταῦτα κατασκευάζοντες αὐτῆι τεχνῖται προσεκαρ-
τέρουν τοῖς ἔργοις ἔν τινι περιβόλωι περιωικοδομημένωι καὶ πύλας ἔχοντι 
τηρουμένας ἐπιμελῶς, ὥστε μηδένα μήτε τῶν ἔσωθεν ἐξιέναι τεχνιτῶν μήτε 
τῶν ἔξωθεν εἰσιέναι πρὸς αὐτούς· τοῦτο δ' ἐποίησεν, ὅπως μηδεὶς τῶν ἔξωθεν 
ἴδηι τὸ γινόμενον μηδὲ διαπέσηι φήμη πρὸς Ἰνδοὺς περὶ τούτων. 
Go to Context



Ctesias Hist., Med., Fragmenta 
Volume-Jacobyʹ-F 3c,688,F, fragment 1b, line 483

       .... (4) ὁ δὲ τῶν Ἰνδῶν βασιλεὺς Σταβροβάτης πυνθανόμενος τά τε 
μεγέθη τῶν ὀνομαζομένων δυνάμεων καὶ τὴν ὑπερβολὴν τῆς εἰς τὸν πόλεμον   
παρασκευῆς, ἔσπευδεν ἐν ἅπασιν ὑπερθέσθαι τὴν Σεμίραμιν. 
Go to Context



Ctesias Hist., Med., Fragmenta 
Volume-Jacobyʹ-F 3c,688,F, fragment 1b, line 486

                                                                    (5) καὶ πρῶτον 
μὲν ἐκ τοῦ καλάμου κατεσκεύασε πλοῖα ποτάμια τετρακισχίλια· ἡ γὰρ Ἰνδικὴ 
παρά τε τοὺς ποταμοὺς καὶ τοὺς ἑλώδεις τόπους φέρει καλάμου πλῆθος, οὗ τὸ 
πάχος οὐκ ἂν ῥαιδίως ἄνθρωπος περιλάβοι· λέγεται δὲ καὶ τὰς ἐκ τούτων 
κατασκευαζομένας ναῦς διαφόρους κατὰ τὴν χρείαν ὑπάρχειν, οὔσης ἀσήπτου 
ταύτης τῆς ὕλης. 
Go to Context



Ctesias Hist., Med., Fragmenta 
Volume-Jacobyʹ-F 3c,688,F, fragment 1b, line 491

                     (6) ποιησάμενος δὲ καὶ τῆς τῶν ὅπλων κατασκευῆς πολλὴν 
ἐπιμέλειαν καὶ πᾶσαν ἐπελθὼν τὴν Ἰνδικήν, ἤθροισε δύναμιν πολὺ μείζονα τῆς 
Σεμιράμιδι συναχθείσης. 
Go to Context



Ctesias Hist., Med., Fragmenta 
Volume-Jacobyʹ-F 3c,688,F, fragment 1b, line 502

      (2) ἡ δὲ Σεμίραμις ἀναγνοῦσα τὴν ἐπιστολὴν καὶ καταγελάσασα τῶν 
γεγραμμένων, διὰ τῶν ἔργων ἔφησε τὸν Ἰνδὸν πειράσεσθαι τῆς περὶ αὐτὴν 
ἀρετῆς. 
Go to Context



Ctesias Hist., Med., Fragmenta 
Volume-Jacobyʹ-F 3c,688,F, fragment 1b, line 503

          ἐπεὶ δὲ προάγουσα μετὰ τῆς δυνάμεως ἐπὶ τὸν Ἰνδὸν ποταμὸν παρε-
γενήθη, κατέλαβε τὰ τῶν πολεμίων πλοῖα πρὸς μάχην ἕτοιμα. 
Go to Context



Ctesias Hist., Med., Fragmenta 
Volume-Jacobyʹ-F 3c,688,F, fragment 1b, line 512

                 μετὰ δὲ ταῦθ' ὁ μὲν τῶν Ἰνδῶν βασιλεὺς ἀπήγαγε τὴν δυνάμιν 
ἀπὸ τοῦ ποταμοῦ, προσποιούμενος μὲν ἀναχωρεῖν διὰ φόβον, τῆι δ' ἀληθείαι 
βουλόμενος τοὺς πολεμίους προτρέψασθαι διαβῆναι τὸν ποταμόν. 
Go to Context



Ctesias Hist., Med., Fragmenta 
Volume-Jacobyʹ-F 3c,688,F, fragment 1b, line 518

                                                                    (6) ἡ δὲ 
Σεμίραμις κατὰ νοῦν αὐτῆι τῶν πραγμάτων προχωρούντων, ἔζευξε τὸν ποταμὸν 
κατασκευάσασα πολυτελῆ καὶ μεγάλην γέφυραν, δι' ἧς ἅπασαν διακομίσασα 
τὴν δύναμιν ἐπὶ μὲν τοῦ ζεύγματος φυλακὴν κατέλιπεν ἀνδρῶν ἑξακισμυρίων, 
τῆι δ' ἄλληι στρατιᾶι προῆγεν ἐπιδιώκουσα τοὺς Ἰνδούς, προηγουμένων τῶν 
εἰδώλων, ὅπως οἱ τῶν πολεμίων κατάσκοποι τῶι βασιλεῖ ἀπαγγείλωσι τὸ   
πλῆθος τῶν παρ' αὐτῆι θηρίων. 
Go to Context



Ctesias Hist., Med., Fragmenta 
Volume-Jacobyʹ-F 3c,688,F, fragment 1b, line 521

                                   (7) οὐ διεψεύσθη δὲ κατά γε τοῦτο τῆς ἐλπίδος, 
ἀλλὰ τῶν ἐπὶ κατασκοπὴν ἐκπεμφθέντων τοῖς Ἰνδοῖς ἀπαγγελλόντων τὸ 
πλῆθος τῶν παρὰ τοῖς πολεμίοις ἐλεφάντων, ἅπαντες διηποροῦντο πόθεν αὐτῆι 
συνακολουθεῖ τοσοῦτο πλῆθος θηρίων. 
Go to Context



Ctesias Hist., Med., Fragmenta 
Volume-Jacobyʹ-F 3c,688,F, fragment 1b, line 528

                                        (8) οὐ μὴν ἔμεινέ γε τὸ ψεῦδος πλείω 
χρόνον κρυπτόμενον· τῶν γὰρ παρὰ τῆι Σεμιράμιδι στρατευομένων τινὲς 
κατελήφθησαν νυκτὸς ἐν τῆι στρατοπεδείαι ῥαιθυμοῦντες τὰ περὶ τὰς φυλακάς, 
φοβηθέντες δὲ τὴν ἐπακολουθοῦσαν τιμωρίαν ηὐτομόλησαν πρὸς τοὺς πολε-
μίους καὶ τὴν κατὰ τοὺς ἐλέφαντας πλάνην ἀπήγγειλαν· ἐφ' οἷς θαρρήσας ὁ 
τῶν Ἰνδῶν βασιλεὺς καὶ τῆι δυνάμει διαγγείλας τὰ περὶ τῶν εἰδώλων, ἐπέ-
στρεψεν ἐπὶ τοὺς Ἀσσυρίους διατάξας τὴν δύναμιν. 
Go to Context



Ctesias Hist., Med., Fragmenta 
Volume-Jacobyʹ-F 3c,688,F, fragment 1b, line 531

                                                          (19) τὸ δ' αὐτὸ καὶ τῆς 
Σεμιράμιδος ἐπιτελούσης, ὡς ἤγγισαν ἀλλήλοις τὰ στρατόπεδα, Σταβροβάτης 
ὁ τῶν Ἰνδῶν βασιλεὺς προαπέστειλε πολὺ πρὸ τῆς φάλαγγος τοὺς ἱππεῖς μετὰ 
τῶν ἁρμάτων. 
Go to Context



Ctesias Hist., Med., Fragmenta 
Volume-Jacobyʹ-F 3c,688,F, fragment 1b, line 534

                (2) δεξαμένης δὲ τῆς βασιλίσσης εὐρώστως τὴν ἔφοδον τῶν 
ἱππέων, καὶ τῶν κατεσκευασμένων ἐλεφάντων πρὸ τῆς φάλαγγος ἐν ἴσοις δια-
στήμασι τεταγμένων, συνέβαινε πτύρεσθαι τοὺς τῶν Ἰνδῶν ἵππους. 
Go to Context



Ctesias Hist., Med., Fragmenta 
Volume-Jacobyʹ-F 3c,688,F, fragment 1b, line 536

                                                                          (3) τὰ 
γὰρ εἴδωλα πόρρωθεν μὲν ὁμοίαν εἶχε τὴν πρόσοψιν τοῖς ἀληθινοῖς θηρίοις, 
οἷς συνήθεις ὄντες οἱ τῶν Ἰνδῶν ἵπποι τεθαρρηκότως προσίππευον· τοῖς δ' 
ἐγγίσασιν ἥ τε ὀσμὴ προσέβαλλεν ἀσυνήθης, καὶ τἄλλα διαφορὰν ἔχοντα πάντα 
παμμεγέθη τοὺς ἵππους ὁλοσχερῶς συνετάραττε· διὸ καὶ τῶν Ἰνδῶν οἱ μὲν 
ἐπὶ τὴν γῆν ἔπιπτον, οἱ δὲ τῶν ζώιων ἀπειθούντων τοῖς χαλινοῖς ὡς ἐτύγχανεν 
εἰς τοὺς πολεμίους ἐξέπιπτον μετὰ τῶν κομιζόντων αὐτοὺς ἵππων. 
Go to Context



Ctesias Hist., Med., Fragmenta 
Volume-Jacobyʹ-F 3c,688,F, fragment 1b, line 542

                                                                           (4) ἡ δὲ 
Σεμίραμις μετὰ στρατιωτῶν ἐπιλέκτων μαχομένη καὶ τῶι προτερήματι δεξιῶς 
χρησαμένη, τοὺς Ἰνδοὺς ἐτρέψατο· ὧν φυγόντων πρὸς τὴν [τῶν Ἰνδῶν] φά-
λαγγα, Σταβροβάτης ὁ βασιλεὺς οὐ καταπλαγεὶς ἐπήγαγε τὰς τῶν πεζῶν 
τάξεις, προηγουμένων τῶν ἐλεφάντων, αὐτὸς δ' ἐπὶ τοῦ δεξιοῦ κέρατος τεταγ-
μένος καὶ τὴν μάχην ἐπὶ τοῦ κρατίστου θηρίου ποιούμενος ἐπήγαγε κατα-
πληκτικῶς ἐπὶ τὴν βασίλισσαν κατ' αὐτὸν τυχικῶς τεταγμένην. 
Go to Context



Ctesias Hist., Med., Fragmenta 
Volume-Jacobyʹ-F 3c,688,F, fragment 1b, line 554

                  (7) τραπέντος οὖν τοῦ πλήθους παντός, ὁ βασιλεὺς τῶν Ἰνδῶν 
ἐπ' αὐτὴν ἐβιάζετο τὴν Σεμίραμιν. 
Go to Context



Ctesias Hist., Med., Fragmenta 
Volume-Jacobyʹ-F 3c,688,F, fragment 1b, line 562

(8) πάντων δὲ φευγόντων ἐπὶ τὴν σχεδίαν, καὶ τοσούτου πλήθους εἰς ἕνα καὶ 
στενὸν βιαζομένου τόπον, οἱ μὲν τῆς βασιλίσσης ὑπ' ἀλλήλων ἀπέθνησκον, 
συμπατούμενοι καὶ φυρόμενοι παρὰ φύσιν ἀναμὶξ ἱππεῖς τε καὶ πεζοί, τῶν δὲ 
Ἰνδῶν ἐπικειμένων, ὠσμὸς ἐγένετο βίαιος ἐπὶ τῆς γεφύρας διὰ τὸν φόβον, 
ὥστε πολλοὺς ἐξωθουμένους ἐφ' ἑκάτερα μέρη τῆς γεφύρας ἐμπίπτειν εἰς τὸν 
ποταμόν. 
Go to Context



Ctesias Hist., Med., Fragmenta 
Volume-Jacobyʹ-F 3c,688,F, fragment 1b, line 567

          (9) ἡ δὲ Σεμίραμις ἐπειδὴ τὸ πλεῖστον μέρος τῶν ἀπὸ τῆς μάχης 
διασωζομένων διὰ τὸν ποταμὸν ἔτυχε τῆς ἀσφαλείας, ἀπέκοψε τοὺς συνέχοντας 
δεσμοὺς [τὴν γέφυραν]· ὧν λυθέντων, ἡ μὲν σχεδία κατὰ πολλὰ διαιρεθεῖσα 
μέρη καὶ συχνοὺς ἐφ' ἑαυτῆς ἔχουσα τῶν διωκόντων Ἰνδῶν ὑπὸ τῆς τοῦ ῥεύ-
ματος σφοδρότητος ὡς ἔτυχε κατηνέχθη, καὶ πολλοὺς μὲν τῶν Ἰνδῶν διέ-
φθειρε, τῆι δὲ Σεμιράμιδι πολλὴν ἀσφάλειαν παρεσκεύασε, κωλύσασα τὴν τῶν 
πολεμίων ἐπ' αὐτὴν διάβασιν. 
Go to Context



Ctesias Hist., Med., Fragmenta 
Volume-Jacobyʹ-F 3c,688,F, fragment 1b, line 570

                                  (10) μετὰ δὲ ταῦθ' ὁ μὲν τῶν Ἰνδῶν βασιλεύς, 
διοσημείων αὐτῶι γενομένων, καὶ τῶν μάντεων ἀποφαινομένων σημαίνεσθαι τὸν 
ποταμὸν μὴ διαβαίνειν, ἡσυχίαν ἔσχεν, ἡ δὲ Σεμίραμις ἀλλαγὴν ποιησαμένη 
τῶν αἰχμαλώτων, ἐπανῆλθεν εἰς Βάκτρα, δύο μέρη τῆς δυνάμεως ἀποβεβληκυῖα. 
Go to Context



Ctesias Hist., Med., Fragmenta 
Volume-Jacobyʹ-F 3c,688,F, fragment 1b, line 586

                  αὕτη μὲν οὖν 
βασιλεύσασα τῆς Ἀσίας ἁπάσης πλὴν Ἰνδῶν ἐτελεύτησε τὸν προειρημένον τρό-
πον, βιώσασα μὲν ἔτη ἑξήκοντα καὶ δύο, βασιλεύσασα δὲ δύο πρὸς τοῖς τεττα-
ράκοντα. 
Go to Context



Ctesias Hist., Med., Fragmenta 
Volume-Jacobyʹ-F 3c,688,F, fragment 9, line 67

καὶ πίπτει καὶ αὐτὸς Κῦρος ἐκ τοῦ ἵππου, καὶ Ἰνδὸς ἀνήρ – συνεμάχουν γὰρ 
καὶ Ἰνδοὶ τοῖς Δερβίκεσιν, ἐξ ὧν καὶ τοὺς ἐλέφαντας ἔφερον – οὗτος ὁ Ἰνδὸς 
πεπτωκότα Κῦρον βάλλει ἀκοντίωι ὑπὸ τὸ ἰσχίον εἰς τὸν μηρόν· ἐξ οὗ καὶ 
τελευτᾶι. 
Go to Context



Ctesias Hist., Med., Fragmenta 
Volume-Jacobyʹ-F 3c,688,F, fragment 11, line 2

              s.v. Δυρβαῖοι· ἔθνος καθῆκον εἰς Βάκτρους 
καὶ τὴν Ἰνδικήν. 
Go to Context



Ctesias Hist., Med., Fragmenta 
Volume-Jacobyʹ-F 3c,688,F, fragment 11, line 3

                      <Κτησίας ἐν Περσικῶν <ι>· «χώρα δὲ πρὸς νότον 
πρόσκειται Δυρβαῖοι, <πρὸς τὴν Βακτρίαν καὶ Ἰνδικὴν κατα-
τείνοντες>. 
Go to Context



Ctesias Hist., Med., Fragmenta 
Volume-Jacobyʹ-F 3c,688,F, fragment 11, line 5

              οὗτοι εὐδαίμονες ἄνδρες καὶ πλούσιοι καὶ κάρτα 
δίκαιοί εἰσι> [πρὸς τὴν Βακτρίαν καὶ Ἰνδικὴν κατατείνοντες]· <οὗτοι οὔτε 
ἀδικοῦσιν οὔτε ἀποκτεννύουσιν ἀνθρώπων οὐδένα· ἐὰν δέ τι 
εὕρωσι ἐν τῆι ὁδῶι χρυσίον ἢ ἱμάτιον ἢ ἀργύριον ἢ ἄλλο τι, 
οὐδὲν ἀποκινήσειαν. 
Go to Context



Ctesias Hist., Med., Fragmenta 
Volume-Jacobyʹ-F 3c,688,F, fragment 33, line 4

        Bibl. 
72 p. 45 a 1 – 4: (76) ἀπὸ 
Ἐφέσου μέχρι Βάκτρων καὶ 
Ἰνδικῆς ἀριθμὸς σταθμῶν, 
ἡμερῶν, παρασαγγῶν. 
Go to Context



Ctesias Hist., Med., Fragmenta 
Volume-Jacobyʹ-F 3c,688,F, fragment 45, line 3

        Bibl. 72    
p. 45 a 21 – 50 a 4: (T 10) λέ-
γει περὶ τοῦ Ἰνδοῦ ποταμοῦ τὸ 
μὲν στενὸν αὐτοῦ τὸ εὖρος <μ> 
σταδίων εἶναι, τὸ δὲ πλατύτα-
τον καὶ διακοσίων. 
Go to Context



Ctesias Hist., Med., Fragmenta 
Volume-Jacobyʹ-F 3c,688,F, fragment 45, line 7

(2) λέγει περὶ αὐτῶν τῶν Ἰνδῶν, ὅτι πλείους σχεδὸν συμπάντων ἀνθρώπων 
(cf. F 49). 
Go to Context



Ctesias Hist., Med., Fragmenta 
Volume-Jacobyʹ-F 3c,688,F, fragment 45, line 10

             (5) ὅτι οὐκ ὕει, ἀλλ' ὑπὸ τοῦ ποταμοῦ ποτίζεται ἡ Ἰνδική. 
Go to Context



Ctesias Hist., Med., Fragmenta 
Volume-Jacobyʹ-F 3c,688,F, fragment 45, line 28

καὶ περὶ τοῦ ὀρνέ-
ου τοῦ βιττάκου, 
ὅτι γλῶσσαν ἀν-
θρωπίνην ἔχει καὶ φωνήν, μέγεθος μὲν ὅσον ἱέραξ, πορφύρεον δὲ πρό-
σωπον· καὶ πώγωνα φέρει μέλανα, αὐτὸ δὲ κυάνεόν ἐστιν ὡς τὸν τρά-
χηλον ὥσπερ κιννάβαρι· διαλέγεσθαι δὲ αὐτὸ ὥσπερ ἄνθρωπον Ἰνδιστί, ἂν δὲ 
Ἑλληνιστὶ μάθηι, καὶ Ἑλληνιστί. 
Go to Context



Ctesias Hist., Med., Fragmenta 
Volume-Jacobyʹ-F 3c,688,F, fragment 45, line 39

        (10) περὶ τῶν κυνῶν τῶν Ἰνδικῶν ὅτι μέγιστοί εἰσιν, ὡς καὶ λέοντι 
μάχεσθαι. 
Go to Context



Ctesias Hist., Med., Fragmenta 
Volume-Jacobyʹ-F 3c,688,F, fragment 45, line 47

(14) ὅτι ὁ Ἰνδὸς ποταμὸς ῥέων διὰ 
πεδίων καὶ δι' ὀρέων ῥεῖ, ἐν οἷς καὶ ὁ 
λεγόμενος Ἰνδικὸς κάλαμος φύεται, 
πάχος μὲν ὅσον δύω ἄνδρε περιωργυι-   
ωμένοι <μόλις> περιλάβοιεν, τὸ δὲ ὕψος 
ὅσον μυριοφόρου νεὼς ἱστός· εἰσὶ καὶ 
ἔτι μείζους καὶ ἐλάττους, οἵους εἰκὸς 
ἐν ὄρει μεγάλωι. 
Go to Context



Ctesias Hist., Med., Fragmenta 
Volume-Jacobyʹ-F 3c,688,F, fragment 45, line 121

ἔστι δὲ πολλὰ ἐν τῆι 
Ἰνδικῆι. 
Go to Context



Ctesias Hist., Med., Fragmenta 
Volume-Jacobyʹ-F 3c,688,F, fragment 45, line 126

(16) περὶ τῶν Ἰνδῶν ὅτι δικαιότατοι· καὶ περὶ τῶν ἐθῶν καὶ νομίμων αὐτῶν. 
Go to Context



Ctesias Hist., Med., Fragmenta 
Volume-Jacobyʹ-F 3c,688,F, fragment 45, line 131

                                                        (18) ὅτι βρονταὶ καὶ ἀστρα-
παὶ καὶ ὑετοὶ οὔκ εἰσιν ἐν τῆι Ἰνδικῆι, ἄνεμοι δὲ πολλοὶ καὶ πρηστῆρες πολλοί· 
καὶ ἁρπάζουσιν ὅ τι ἂν λάβωσιν. 
Go to Context



Ctesias Hist., Med., Fragmenta 
Volume-Jacobyʹ-F 3c,688,F, fragment 45, line 133

                                        ὁ δὲ ἥλιος ἀνίσχων τὸ ἥμισυ τῆς ἡμέρας 
ψύχος ποιεῖ, τὸ δ' ἄλλο λίαν ἀλεεινὸν ἐν τοῖς πλείστοις τῶν τῆς Ἰνδικῆς τόπων. 
Go to Context



Ctesias Hist., Med., Fragmenta 
Volume-Jacobyʹ-F 3c,688,F, fragment 45, line 134

(19) ὅτι οἱ Ἰνδοὶ οὐχ ὑπὸ τοῦ ἡλίου εἰσὶ μέλανες (§ 44) ἀλλὰ φύσει· εἶναι γάρ 
φησιν ἐν αὐτοῖς καὶ ἄνδρας καὶ γυναῖκας λευκοτάτους πάντων, εἰ καὶ ἐπ' ἔλαττον· 
ἰδεῖν δὲ καὶ αὐτὸν τοιαύτας Ἰνδὰς δύο γυναῖκας καὶ πέντε ἄνδρας. 
Go to Context



Ctesias Hist., Med., Fragmenta 
Volume-Jacobyʹ-F 3c,688,F, fragment 45, line 137

                                                                                 (20) ὅτι 
πιστῶσαι τὰ περὶ τοῦ ἡλίου βουλόμενος ὡς ἐν <λε> ἡμέραις ἐν Ἰνδίαι ψύχει, λέγει 
ὅτι καὶ τὸ πῦρ ἐκ τῆς Αἴτνης ῥέον τὸν μέσον χῶρον, ἅτε δικαίων ἀνδρῶν ὄντων, 
οὐ φθείρει, φθεῖρον τὰ ἄλλα· καὶ ἐν Ζακύνθωι κρηνῖδας ἰχθυοφόρους εἶναι, ἐξ ὧν 
αἴρεται πίσσα· καὶ ἐν Νάξωι κρήνην, ἐξ ἧς οἶνος ἐνίοτε ῥεῖ καὶ μάλα ἡδύς· καὶ 
ὅτι πῦρ ἐστὶν ἐγγὺς Φασήλιδος 
ἐν Λυκίαι ἀθάνατον, καὶ ὅτι ἀεὶ 
καίεται ἐπὶ πέτρας καὶ νύκτα 
καὶ ἡμέραν, καὶ ὕδατι μὲν οὐ 
σβέννυται, ἀλλὰ ἀναφλέγει, 
φορυτῶι δὲ σβέννυται. 
Go to Context



Ctesias Hist., Med., Fragmenta 
Volume-Jacobyʹ-F 3c,688,F, fragment 45, line 147

(21) ὅτι <ἐν> μέσηι Ἰνδικῆι 
ἄνθρωποί εἰσι μέλανες – κα-
λοῦνται Πυγμαῖοι – ὁμόγλως-
σοι τοῖς ἄλλοις Ἰνδοῖς. 
Go to Context



Ctesias Hist., Med., Fragmenta 
Volume-Jacobyʹ-F 3c,688,F, fragment 45, line 177

               (23) ἕπονται 
δὲ τῶι βασιλεῖ τῶν Ἰνδῶν τού-
των τῶν Πυγμαίων ἄνδρες 
τρισχίλιοι· σφόδρα γάρ εἰσι το-
ξόται. 
Go to Context



Ctesias Hist., Med., Fragmenta 
Volume-Jacobyʹ-F 3c,688,F, fragment 45, line 182

        δικαιότατοι δέ εἰσι, καὶ 
νόμοισι χρῶνται ὥσπερ καὶ οἱ 
Ἰνδοί. 
Go to Context



Ctesias Hist., Med., Fragmenta 
Volume-Jacobyʹ-F 3c,688,F, fragment 45, line 203

            ἔστι δὲ 
καὶ χρυσὸς ἐν τῆι 
Ἰνδικῆι χώραι, 
οὐκ ἐν τοῖς ποτα-
μοῖς εὑρισκόμενος 
καὶ πλυνόμενος, 
ὥσπερ ἐν τῶι 
Πακτωλῶι ποτα-
μῶι, ἀλλ' ὄρη πολ-
λὰ καὶ μεγάλα, ἐν 
οἷς οἰκοῦσι γρῦ-
πες, ὄρνεα
Go to Context

τετρά-



Ctesias Hist., Med., Fragmenta 
Volume-Jacobyʹ-F 3c,688,F, fragment 45, line 226

(27) ὅτι τὰ πρό-
βατα τῶν Ἰνδῶν 
καὶ αἱ αἶγες μεί-
ζους ὄνων εἰσί. 
Go to Context



Ctesias Hist., Med., Fragmenta 
Volume-Jacobyʹ-F 3c,688,F, fragment 45, line 239

             ὗς δὲ 
οὔτε ἥμερος οὔτε 
ἄγριός ἐστιν ἐν τῆι 
Ἰνδικῆι. 
Go to Context



Ctesias Hist., Med., Fragmenta 
Volume-Jacobyʹ-F 3c,688,F, fragment 45, line 241

            (28) οἱ 
δὲ φοίνικες οἱ ἐν 
Ἰνδοῖς καὶ οἱ τού-
των βάλανοι τρι-
πλάσιοι τῶν ἐν Βα-
βυλῶνι. 
Go to Context



Ctesias Hist., Med., Fragmenta 
Volume-Jacobyʹ-F 3c,688,F, fragment 45, line 255

(32) ὅτι φησὶν ὡς Ἰνδῶν οὐδεὶς κεφαλαλγεῖ οὐδὲ ὀφθαλμιᾶι οὐδὲ ὀδονταλ-
γεῖ οὐδὲ ἑλκοῦται τὸ στόμα οὐδὲ σηπεδόνα οὐδεμίαν ἴσχει. 
Go to Context



Ctesias Hist., Med., Fragmenta 
Volume-Jacobyʹ-F 3c,688,F, fragment 45, line 320

(36) ἔστι δὲ καὶ ποταμὸς διαρρέων διὰ τῆς 
Ἰνδικῆς, οὐ μέγας μὲν ἀλλ' ὡς ἐπὶ δύο σταδίους τὸ 
εὖρος· ὄνομα δὲ τῶι ποταμῶι Ἰνδιστὶ μὲν Ὕπαρ-
χος (?), Ἑλληνιστὶ δὲ φέρων πάντα τὰ ἀγαθά. 
Go to Context



Ctesias Hist., Med., Fragmenta 
Volume-Jacobyʹ-F 3c,688,F, fragment 45, line 330

           τῶι δενδρέωι δὲ τούτωι ὄνομά ἐστιν 
Ἰνδιστὶ σιπταχόρα, <ὅπερ> Ἑλληνιστὶ σημαίνει 
γλυκύ, ἡδύ· κἀκεῖθεν οἱ Ἰνδοὶ συλλέγουσι τὸ ἤλεκ-
τρον. 
Go to Context



Ctesias Hist., Med., Fragmenta 
Volume-Jacobyʹ-F 3c,688,F, fragment 45, line 344

                                οἰκοῦ-
σι δὲ ἐν τοῖς ὄρεσι μέχρι τοῦ Ἰνδοῦ 
ποταμοῦ. 
Go to Context



Ctesias Hist., Med., Fragmenta 
Volume-Jacobyʹ-F 3c,688,F, fragment 45, line 346

          μέλανες δέ εἰσι καὶ δίκαιοι 
πάνυ (§ 43), ὥσπερ καὶ οἱ ἄλλοι Ἰνδοί 
(§ 30), οἷς καὶ ἐπιμίγνυνται. 
Go to Context



Ctesias Hist., Med., Fragmenta 
Volume-Jacobyʹ-F 3c,688,F, fragment 45, line 352

            καλοῦνται δὲ ὑπὸ τῶν Ἰνδῶν 
Καλύστριοι, ὅπερ ἐστὶν Ἑλληνιστὶ 
Κυνοκέφαλοι. 
Go to Context



Ctesias Hist., Med., Fragmenta 
Volume-Jacobyʹ-F 3c,688,F, fragment 45, line 369

                              ταῦτα οὖν τὰ 
θηρία τρίβοντες οἱ Ἰνδοὶ βάπτουσι τὰς 
φοινικίδας καὶ τοὺς χιτῶνας καὶ ἄλλο 
ὅ τι ἂν βούλωνται· καὶ εἰσὶ βελτίω τῶν 
παρὰ Πέρσαις βαμμάτων. 
Go to Context



Ctesias Hist., Med., Fragmenta 
Volume-Jacobyʹ-F 3c,688,F, fragment 45, line 388

                                 (41) οἱ 
δὲ Κυνοκέφαλοι σχεδίαν ποιησάμενοι 
καὶ ἐπιθέντες ἀπάγουσι φόρτον τούτου    
καὶ τῆς πορφύρας, τὸ ἄνθος καθαρὸν ποιήσαντες, [καὶ τοῦ ἠλέκτρου] <ξ> καὶ <σ> 
τάλαντα τοῦ ἐνιαυτοῦ· καὶ ὅτωι τὸ φοινίκιον βάπτεται, τοῦ φαρμάκου ἕτερα 
τοσαῦτα· καὶ ἠλέκτρου χίλια τάλαντα ἀπάγουσι κατ' ἐνιαυτὸν τῶι Ἰνδῶν βασιλεῖ. 
Go to Context



Ctesias Hist., Med., Fragmenta 
Volume-Jacobyʹ-F 3c,688,F, fragment 45, line 389

καὶ ἕτερα δὲ κατάγοντες πωλοῦσι τοῖς Ἰνδοῖς πρὸς ἄρτους καὶ ἄλφιτα καὶ ξύλινα 
ἱμάτια. 
Go to Context



Ctesias Hist., Med., Fragmenta 
Volume-Jacobyʹ-F 3c,688,F, fragment 45, line 410

οὗτοι δὲ οἱ ἄνθρωποι μέλανες μέν εἰσιν ὥσπερ οἱ ἄλλοι Ἰνδοί (§ 19), ἐργάζονται 
δὲ οὐδέν, οὐδὲ ἐσθίουσι σῖτον, οὐδὲ πίνουσιν ὕδωρ, πρόβατα δὲ πολλὰ τρέφουσι 
καὶ βοῦς καὶ αἶγας καὶ ὄις, πίνουσι δὲ τὸ γάλα, ἄλλο δὲ οὐδέν· ὅταν δὲ γένηταί 
τινι αὐτῶν παιδίον, οὐ τέτρηται τὴν πυγὴν οὐδὲ ἀποπατεῖ, ἀλλὰ τὰ μὲν ἰσχία 
ἔχει, τὸ δὲ τρῆμα συμπέφυκε· διὸ ἀποπατοῦσι μὲν οὔ, οὐρεῖν δὲ ὥσπερ τυρὸν 
αὐτούς φασιν οὐ πάνυ παχὺν ἀλλὰ θολερόν. 
Go to Context



Ctesias Hist., Med., Fragmenta 
Volume-Jacobyʹ-F 3c,688,F, fragment 45, line 420

(45) ὅτι εἰσὶν ὄνοι ἄγριοι ἐν 
τοῖς Ἰνδοῖς, ἴσοι ἵπποις καὶ μεί-
ζους. 
Go to Context



Ctesias Hist., Med., Fragmenta 
Volume-Jacobyʹ-F 3c,688,F, fragment 45, line 473

(46) ὅτι ἐν τῶι ποτα-
μῶι τῶι Ἰνδῶι σκώληξ 
(§ 3) γίνεται, τὸ μὲν εἶ-
δος οἷόν περ ἐν ταῖς συ-
καῖς εἴωθε γίνεσθαι, τὸ 
δὲ μῆκος πήχεις ἑπτά, 
καὶ μείζους δὲ καὶ ἐλάτ-
τους· τὸ δὲ πάχος δεκαε-
τέα παῖδα μόλις φασὶ 
ταῖς χερσὶ περιλαβεῖν. 
Go to Context



Ctesias Hist., Med., Fragmenta 
Volume-Jacobyʹ-F 3c,688,F, fragment 45, line 511

         ὅταν δὲ παρέλ-
θωσιν αἱ τριάκοντα ἡμέ-
ραι, ἀπορρίπτουσι τὸν 
σκώληκα, καὶ τὸ ἔλαιον 
ἀσφαλισάμενοι ἄγουσι 
τῶι βασιλεῖ μόνωι τῶν 
Ἰνδῶν· ἄλλωι δὲ οὐκ 
ἔξεστιν ἐξ αὐτοῦ ἔχειν. 
Go to Context



Ctesias Hist., Med., Fragmenta 
Volume-Jacobyʹ-F 3c,688,F, fragment 45, line 519

(47) ὅτι ἐστὶ δένδρα ἐν Ἰνδοῖς ὑψηλὰ ὥσπερ κέδρος ἢ κυπάριττος, τὰ δὲ 
φύλλα ὥσπερ φοῖνιξ, ὀλίγον πλατύτερα· καὶ μασχαλίδας οὐκ ἔχει. 
Go to Context



Ctesias Hist., Med., Fragmenta 
Volume-Jacobyʹ-F 3c,688,F, fragment 45, line 521

                                            ὀνομάζεται δὲ Ἰνδιστὶ μὲν κάρπιον, Ἑλλη-
νιστὶ δὲ μυρορόδα. 
Go to Context



Ctesias Hist., Med., Fragmenta 
Volume-Jacobyʹ-F 3c,688,F, fragment 45, line 526

                     καὶ ἔπεμψεν ὁ Ἰνδῶν τῶι Περσῶν βασιλεῖ, καί <φησιν ἰδεῖν 
αὐτὸ Κτησίας,> καὶ ὀσφρανθῆναι ὀσμῆς οἵας οὔτε εἰπεῖν ἦν οὔτε εἰκάσαι. 
Go to Context



Ctesias Hist., Med., Fragmenta 
Volume-Jacobyʹ-F 3c,688,F, fragment 45, line 530

                             > 
 (49) ὅτι κρήνην ἐν Ἰνδοῖς φησιν, 
ὅσον πέντε ὀργυιῶν ἡ περίμετρος, τε-
τράγωνος δέ. 
Go to Context



Ctesias Hist., Med., Fragmenta 
Volume-Jacobyʹ-F 3c,688,F, fragment 45, line 536

           λούονται δὲ ἐν αὐτῶι οἱ ἐπι-
σημότατοι τῶν Ἰνδῶν καὶ ἄνδρες καὶ 
παῖδες καὶ γυναῖκες, κολυμβῶσι δὲ ἐπὶ 
πόδας ῥίπτοντες ἑαυτούς· ὅταν δὲ εἰς-
πηδῶσιν, ἐκβάλλει αὐτοὺς τὸ ὕδωρ 
ἄνω. 
Go to Context



Ctesias Hist., Med., Fragmenta 
Volume-Jacobyʹ-F 3c,688,F, fragment 45, line 550

     καλεῖται δὲ Ἰνδιστὶ βαλλάδη, 
Ἑλληνιστὶ δὲ ὠφελίμη. 
Go to Context



Ctesias Hist., Med., Fragmenta 
Volume-Jacobyʹ-F 3c,688,F, fragment 45, line 552

(50) εἰσὶν ἐν τοῖς ὄρεσι τοῖς Ἰνδικοῖς, ὅπου ὁ 
κάλαμος (§ 15) αὐτῶν φύεται, ἄνθρωποι τὸ πλῆθος 
αὐτῶν ἄχρι καὶ τριῶν μυριάδων. 
Go to Context



Ctesias Hist., Med., Fragmenta 
Volume-Jacobyʹ-F 3c,688,F, fragment 45, line 564

                                                     εἰσὶ δὲ σφόδρα πολεμισταί, καὶ 
βασιλεῖ τῶν Ἰνδῶν ἕπονται τοξόται τούτων πεντακισχίλιοι, καὶ ἀκοντισταί. 
Go to Context



Ctesias Hist., Med., Fragmenta 
Volume-Jacobyʹ-F 3c,688,F, fragment 45a, line 4

Dion. Per. 1143): Κτησίας μὲν δή, εἰ δή τωι 
ἱκανὸς καὶ Κτησίας ἐς τεκμηρίωσιν (T 11g), 
ἵνα μὲν στενότατος αὐτὸς αὑτοῦ ὁ Ἰνδός ἐστι, 
τεσσαράκοντα σταδίους <λέγει> ὅτι διέχουσιν 
αὐτῶι αἱ ὄχθαι, ἵνα δὲ πλατύτατος, καὶ ἑκατόν· 
τὸ πολὺ δὲ εἶναι αὐτοῦ τὸ μέσον τούτοιν. 
Go to Context



Ctesias Hist., Med., Fragmenta 
Volume-Jacobyʹ-F 3c,688,F, fragment 45b, line 1

                 17, 29: τοῦ Ἰνδῶν βασιλέως ἐλαύ-
νοντος ἐπὶ τοὺς πολεμίους δέκα μυριάδες ἐλεφάντων προ-
ηγοῦνται μαχίμων. 
Go to Context



Ctesias Hist., Med., Fragmenta 
Volume-Jacobyʹ-F 3c,688,F, fragment 45b, line 11

             ἰδεῖν δὲ ἐν Βαβυλῶνι ὁ αὐτὸς λέ-  
γει> τοὺς φοίνικας αὐτορρίζους ἀνατρεπομένους ὑπὸ τῶν 
ἐλεφάντων τὸν αὐτὸν τρόπον, ἐμπιπτόντων τῶν θηρίων 
αὐτοῖς βιαιότατα· δρῶσι δ' ἄρα, ἂν ὁ Ἰνδὸς ὁ πωλεύων 
αὐτοὺς κελεύσηι δρᾶσαι τοῦτο αὐτοῖς. 
Go to Context



Ctesias Hist., Med., Fragmenta 
Volume-Jacobyʹ-F 3c,688,F, fragment 45c, line 4

         Chil. 7, 738: 
εἰ θαῦμα δὲ νομίζοι τις Ἀρράβων τοὺς 
καλάμους (Uranios 675 F 21), / ὁ 
Τζέτζης λέγει, τοὺς Ἰνδῶν καλάμους   
τῶι Κτησίαι / ὡς διοργυίους γράφοντι 
τὸ πλάτος τίς πιστεύσοι; 
Go to Context



Ctesias Hist., Med., Fragmenta 
Volume-Jacobyʹ-F 3c,688,F, fragment 45d,Alpha, line 4

                                                             ἔστι 
δέ τι, <εἰ δεῖ πιστεῦσαι Κτησίαι·> ἐκεῖνος γὰρ τὸ ἐν 
Ἰνδοῖς θηρίον, ὧι ὄνομα εἶναι μαρτιχόραν, τοῦτ' ἔχειν ἐπ' 
ἀμφότερά φησι τριστοίχους τοὺς ὀδόντας. 
Go to Context



Ctesias Hist., Med., Fragmenta 
Volume-Jacobyʹ-F 3c,688,F, fragment 45d,Beta, line 2

                 4, 21 (PHILES De an. propr. 38): 
θηρίον Ἰνδικὸν βίαιον τὴν ἀλκήν, μέγεθος κατὰ τὸν λέοντα 
τὸν μέγιστον, τὴν δὲ χρόαν ἐρυθρόν, ὡς κιννάβαριν εἶναι 
δοκεῖν, δασὺ δὲ ὡς κύνες· φωνῆι τῆι Ἰνδῶν μαρτιχόρας 
ὠνόμασται. 
Go to Context



Ctesias Hist., Med., Fragmenta 
Volume-Jacobyʹ-F 3c,688,F, fragment 45d,Beta, line 23

                                            <λέγει δ' ἄρα Κτη-
σίας, καί φησιν ὁμολογεῖν αὐτῶι τοὺς Ἰνδούς,> 
ἐν ταῖς χώραις τῶν ἀπολυομένων ἐκείνων κέντρων ὑπανα-
φύεσθαι ἄλλα, ὡς εἶναι τοῦ κακοῦ τοῦδε ἐπιγονήν. 
Go to Context



Ctesias Hist., Med., Fragmenta 
Volume-Jacobyʹ-F 3c,688,F, fragment 45d,Beta, line 32

                              ὅτι δὲ κρεῶν ἀνθρωπείων ἐμπιμ-
πλάμενον τόδε τὸ ζῶιον ὑπερήδεται, κατηγορεῖ καὶ τὸ 
ὄνομα· νοεῖται γὰρ τῆι Ἑλλήνων φωνῆι ἡ Ἰνδῶν ἀνθρωπο-
φάγον αὐτὸ εἶναι· ἐκ δὲ τοῦ ἔργου καὶ κέκληται. 
Go to Context



Ctesias Hist., Med., Fragmenta 
Volume-Jacobyʹ-F 3c,688,F, fragment 45d,Beta, line 35

                               τὰ βρέφη δὲ τῶνδε τῶν ζώιων 
Ἰνδοὶ θηρῶσιν ἀκέντρους τὰς οὐρὰς ἔχοντα· καὶ λίθωι δὲ 
ἔτι διαθλῶσιν αὐτάς, ἵνα ἀδυνατῶσι τὰ κέντρα ἀναφύειν· 
φωνὴν δὲ σάλπιγγος ὡς ὅτι ἐγγυτάτω προίεται. 
Go to Context



Ctesias Hist., Med., Fragmenta 
Volume-Jacobyʹ-F 3c,688,F, fragment 45d,Beta, line 39

                                                      λέγει δὲ 
<καὶ ἑωρακέναι τόδε τὸ ζῶιον ἐν Πέρσαις Κτη-  
σίας ἐξ Ἰνδῶν κομισθὲν δῶρον τῶι Περσῶν 
βασιλεῖ, εἰ δή τωι ἱκανὸς τεκμηριῶσαι ὑπὲρ 
τῶν τοιούτων Κτησίας. 
Go to Context



Ctesias Hist., Med., Fragmenta 
Volume-Jacobyʹ-F 3c,688,F, fragment 45d,Gamma, line 2

PAUSAN. 9, 21, 4: θηρίον δὲ <τὸ> <ἐν τῶι Κτησίου 
λόγωι τῶι ἐς Ἰνδούς> – μαρτι<χ>όρα ὑπὸ τῶν Ἰνδῶν, 
ὑπὸ δὲ Ἑλλήνων φησὶν ἀνδροφάγον λελέχθαι – εἶναι πεί-
θομαι τὸν τίγριν. 
Go to Context



Ctesias Hist., Med., Fragmenta 
Volume-Jacobyʹ-F 3c,688,F, fragment 45d,Gamma, line 8

                    ὀδόντας δὲ αὐτὸ τριστοίχους κατὰ ἑ<κα>-
τέραν τὴν γένυν καὶ κέντρα ἐπὶ ἄκρας ἔχειν τῆς οὐρᾶς· 
τούτοις δὲ τοῖς κέντροις ἐγγύθεν ἀμύνεσθαι καὶ ἀποπέμ-
πειν ἐς τοὺς πορρωτέρω τοξότου ἀνδρὸς ὀιστῶι ἴσον, 
ταύτην οὐκ ἀληθῆ τὴν φήμην οἱ Ἰνδοὶ δέξασθαι δοκοῦσί μοι παρ' 
ἀλλήλων ὑπὸ τοῦ ἄγαν ἐς τὸ θηρίον δείματος. 
Go to Context



Ctesias Hist., Med., Fragmenta 
Volume-Jacobyʹ-F 3c,688,F, fragment 45f*,Alpha, line 3

                  De an. 
2, 67 (Suppl. Aristot. I 1 ed. Lambros p. 53, 
27): εἰσὶ δὲ ἐν τοῖς Ἰνδικοῖς οἳ Πυγμαῖοι 
καλοῦνται. 
Go to Context



Ctesias Hist., Med., Fragmenta 
Volume-Jacobyʹ-F 3c,688,F, fragment 45f*,Alpha, line 5

            χώρα δ' αὐτῶν πολλὴ μέσηι ἐν τῆι   
Ἰνδικῆι, καὶ ἄνθρωποί εἰσι μέλανες ὥσπερ οἱ 
ἄλλοι Ἰνδοὶ καὶ ὁμόγλωσσοι ἐκείνοις, μικροὶ 
δὲ κάρτα· καὶ οἱ μακρότατοι αὐτῶν εἰσι πή-
χεων δύο, οἱ δὲ πλεῖστοι πήχεος καὶ ἡμίσεος, 
ἄνδρες καὶ γυναῖκες. 
Go to Context



Ctesias Hist., Med., Fragmenta 
Volume-Jacobyʹ-F 3c,688,F, fragment 45f*,Alpha, line 28

                                           αὐτοὶ δέ 
εἰσι σιμοί τε καὶ αἰσχροὶ καὶ οὐδὲν ἐοικότες   
τοῖς ἄλλοις Ἰνδοῖς· εἰσὶ δὲ καὶ αἱ γυναῖκες 
αὐτῶν μικραὶ καὶ αἰσχραί, ὥσπερ οἱ ἄνδρες. 
Go to Context



Ctesias Hist., Med., Fragmenta 
Volume-Jacobyʹ-F 3c,688,F, fragment 45g, line 2

                 4, 26: τοὺς λαγὼς καὶ τὰς 
ἀλώπεκας θηρῶσιν οἱ Ἰνδοὶ τὸν τρόπον τοῦτον· κυνῶν ἐς 
τὴν ἄγραν οὐ δέονται, ἀλλὰ νεοττοὺς συλλαβόντες ἀετῶν καὶ 
κοράκων καὶ ἰκτίνων προσέτι τρέφουσι καὶ ἐκπαιδεύουσι 
τὴν θήραν. 
Go to Context



Ctesias Hist., Med., Fragmenta 
Volume-Jacobyʹ-F 3c,688,F, fragment 45h, line 2

                 4, 27: τὸν γρῦπα ἀκούω τὸ 
ζῶιον τὸ Ἰνδικὸν τετράπουν εἶναι κατὰ τοὺς λέοντας, καὶ   
ἔχειν ὄνυχας καρτεροὺς ὡς ὅτι μάλιστα, καὶ τούτους μέντοι 
τοῖς τῶν λεόντων παραπλησίους. 
Go to Context



Ctesias Hist., Med., Fragmenta 
Volume-Jacobyʹ-F 3c,688,F, fragment 45h, line 13

                                                               καὶ 
Βάκτριοι μὲν γειτνιῶντες Ἰνδοῖς λέγουσιν αὐτοὺς φύλακας 
εἶναι τοῦ χρυσοῦ <τοῦ> αὐτόθι· καὶ ὀρύττειν τε αὐτόν φασιν 
αὐτοὺς καὶ ἐκ τούτου τὰς καλιὰς ὑποπλέκειν, τὸ δὲ ἀπορ-
ρέον Ἰνδοὺς λαμβάνειν. 
Go to Context



Ctesias Hist., Med., Fragmenta 
Volume-Jacobyʹ-F 3c,688,F, fragment 45i,Beta, line 3

                    De an. 2, 556 (Suppl. 
Aristot. I 1 p. 139, 13): <ἐκ τοῦ Κτησίου· «τὰ πρό-
βατα τῶν Ἰνδῶν αἵ τε αἶγες μείζονες ὄνων τῶν 
μεγίστων εἰσί. 
Go to Context



Ctesias Hist., Med., Fragmenta 
Volume-Jacobyʹ-F 3c,688,F, fragment 45k,Alpha, line 2

                  8, 28 p. 606 a 8: ἐν δὲ 
τῆι Ἰνδικῆι, <ὥς φησι Κτησίας οὐκ ὢν ἀξιόπιστος,> 
οὔτ' ἄγριος οὔτε ἥμερος ὗς, τὰ δ' ἄναιμα καὶ τὰ φολιδωτὰ 
πάντα μεγάλα. 
Go to Context



Ctesias Hist., Med., Fragmenta 
Volume-Jacobyʹ-F 3c,688,F, fragment 45k,Beta, line 3

                   3, 3: ἴδια δὲ ἄρα 
φύσεως ζώιων καὶ ταῦτα ἦν· ὗν οὔτε ἄγριον οὔτε ἥμερον 
ἐν Ἰνδοῖς γίνεσθαι <λέγει Κτησίας. 
Go to Context



Ctesias Hist., Med., Fragmenta 
Volume-Jacobyʹ-F 3c,688,F, fragment 45k,Gamma, line 3

                 De an. 2, 572 (Suppl. Arist. I 1) 
p. 143, 17: <ἐκ τοῦ Κτησίου· «ὗς οὔτε ἥμερός 
ἐστιν οὔτε ἄγριος ἐν τῆι Ἰνδικῆι ὅλως γῆι, 
οὐδ' ἂν φάγοι Ἰνδῶν οὐδεὶς ὑὸς κρέας οὐδέν 
περ μᾶλλον ἢ ἀνθρώπου». 
Go to Context



Ctesias Hist., Med., Fragmenta 
Volume-Jacobyʹ-F 3c,688,F, fragment 45l*, line 1

                    4, 36: ἡ τῶν Ἰνδῶν γῆ, 
<φασὶν αὐτὴν οἱ συγγραφεῖς> πολυφάρμακόν τε καὶ 
τῶν βλαστημάτων τῶνδε δεινῶς πολύγονον εἶναι· καὶ τὰ 
μὲν σώζειν αὐτῶν καὶ ἐκ τῶν κινδύνων ῥύεσθαι τοὺς ὑπὸ 
τῶν δακέτων ὁμοῦ τῶι θανάτωι ὄντας (πολλὰ δὲ ἐκεῖθι   
τοιαῦτα), τὰ δὲ ἀπολλύναι καὶ διαφθείρειν ὀξύτατα, ὧν 
ἕν περ καὶ τὸ ἐκ τοῦ ὄφεως <τοῦ πορφυροῦ> γινόμενον 
εἴη ἄν. 
Go to Context



Ctesias Hist., Med., Fragmenta 
Volume-Jacobyʹ-F 3c,688,F, fragment 45l*, line 14

                               ὀδόντων δὲ ἄγονός ἐστιν ὁ 
ὄφις οὗτος· εὑρίσκεται δ' ἐν τοῖς πυρωδεστάτοις τῆς 
Ἰνδικῆς χωρίοις. 
Go to Context



Ctesias Hist., Med., Fragmenta 
Volume-Jacobyʹ-F 3c,688,F, fragment 45m*, line 2

                    4, 41: γένος ὀρνίθων 
Ἰνδικῶν βραχυτάτων καὶ τοῦτο εἴη ἄν. 
Go to Context



Ctesias Hist., Med., Fragmenta 
Volume-Jacobyʹ-F 3c,688,F, fragment 45m*, line 6

καὶ Ἰνδοὶ μὲν αὐτὸ φωνῆι τῆι σφετέραι δίκαιρον φιλοῦσιν 
ὀνομάζειν, Ἕλληνες δέ, <ὡς ἀκούω,> δίκαιον. 
Go to Context



Ctesias Hist., Med., Fragmenta 
Volume-Jacobyʹ-F 3c,688,F, fragment 45m*, line 13

             σπουδὴν δὲ ἄρα τὴν ἀνωτάτω τίθενται Ἰνδοὶ 
ἐς τὴν κτῆσιν αὐτοῦ· κακῶν γὰρ αὐτὸ ἐπίληθον ἡγοῦνται 
τῶι ὄντι. 
Go to Context



Ctesias Hist., Med., Fragmenta 
Volume-Jacobyʹ-F 3c,688,F, fragment 45m*, line 16

             καὶ οὖν καὶ ἐν τοῖς δώροις τοῖς μέγα τιμίοις τῶι 
Περσῶν βασιλεῖ ὁ Ἰνδῶν πέμπει καὶ τοῦτο· ὁ δὲ καὶ τῶν 
ἄλλων ἁπάντων προτιμᾶι λαβών, καὶ ἀποθησαυρίζει 
κακῶν ἀνιάτων ἀντίπαλόν τε καὶ ἀμυντήριον, εἰ ἀνάγκη 
καταλάβοι. 
Go to Context



Ctesias Hist., Med., Fragmenta 
Volume-Jacobyʹ-F 3c,688,F, fragment 45m*, line 22

                                                         ⟦καὶ διὰ 
ταῦτα ἀντικρίνοντες βασανίσωμεν τῶν φαρμάκων τοῦ τε 
Ἰνδικοῦ καὶ τοῦ Αἰγυπτίου ὁπότερον ἦν προτιμότερον· 
ἐπεὶ τὸ μὲν ἐφ' ἡμέραν αὐτὴν ἀνεῖχέ τε καὶ ἀνέστελλε τὰ 
δάκρυα τὸ Αἰγύπτιον, τὸ δὲ λήθην κακῶν παρεῖχεν αἰώ-
νιον τὸ Ἰνδικόν· καὶ τὸ μὲν γυναικὸς δῶρον ἦν, τὸ δὲ 
ὄρνιθος ἢ ἀπορρήτου φύσεως, δεσμῶν τῶν ὄντως βαρυ-
τάτων ἀπολυούσης δι' ὑπηρέτου τοῦ προειρημένου· καὶ 
Ἰνδοὺς κτήσασθαι αὐτὸ εὐτυχήσαντας, ὡς τῆς ἐνταυθοῖ 
φρουρᾶς ἀπολυθῆναι ὅταν ἐθέλωσιν⟧. 
Go to Context



Ctesias Hist., Med., Fragmenta 
Volume-Jacobyʹ-F 3c,688,F, fragment 45n,Alpha, line 2

           Hist. 
mir. 17: <Κτησίας παρ' Ἰνδοῖς 
ξύλον γίνεσθαί φησιν,> ὃ καλεῖται 
πάρηβον. 
Go to Context



Ctesias Hist., Med., Fragmenta 
Volume-Jacobyʹ-F 3c,688,F, fragment 45p,Beta, line 3

Chil. 7, 713: <καὶ ὁ Κτησίας ἐν 
Ἰνδοῖς> εἶναι τοιαῦτα <λέγει,> / ἠλεκ-
τροφόρα δένδρα τε καὶ τοὺς Κυνοκε-
φάλους· / δικαίους πάνυ δέ φησι, ζῆν 
δ' ἐκ τῶν ἀγρευμάτων. 
Go to Context



Ctesias Hist., Med., Fragmenta 
Volume-Jacobyʹ-F 3c,688,F, fragment 45p,Gamma, line 2

       4, 46: ἐν Ἰνδοῖς γίνεται θηρία τὸ 
μέγεθος ὅσον γένοιντο ἂν οἱ κάνθαροι, 
καί ἐστιν ἐρυθρά· κινναβάρει δὲ εἰκά-
σαις ἄν, εἰ πρῶτον θεάσαιο αὐτά. 
Go to Context



Ctesias Hist., Med., Fragmenta 
Volume-Jacobyʹ-F 3c,688,F, fragment 45p,Gamma, line 11

                θηρῶσι δὲ αὐτὰ οἱ 
Ἰνδοὶ καὶ ἀποθλίβουσι, καὶ ἐξ αὐτῶν 
βάπτουσι τάς τε φοινικίδας καὶ τοὺς 
ὑπ' αὐταῖς χιτῶνας καὶ πᾶν ὅ τι ἂν 
ἐθέλωσιν ἄλλο ἐς τήνδε τὴν χρόαν ἐκ-
τρέψαι τε καὶ χρῶσαι. 
Go to Context



Ctesias Hist., Med., Fragmenta 
Volume-Jacobyʹ-F 3c,688,F, fragment 45p,Gamma, line 24

γίνονται δὲ ἐνταῦθα τῆς Ἰνδικῆς ἔνθα 
οἱ κάνθαροι καὶ οἱ καλούμενοι Κυνο-
κέφαλοι, οἷς τὸ ὄνομα ἔδωκεν ἡ τοῦ 
σώματος ὄψις τε καὶ φύσις· τὰ δὲ ἄλλα 
ἀνθρώπων ἔχουσι, καὶ ἠμφιεσμένοι 
βαδίζουσι δορὰς θηρίων. 
Go to Context



Ctesias Hist., Med., Fragmenta 
Volume-Jacobyʹ-F 3c,688,F, fragment 45p,Gamma, line 32

                           καί εἰσι δί-
καιοι, καὶ ἀνθρώπων λυποῦσιν οὐδένα· 
καὶ φθέγγονται μὲν οὐδέν, ὠρύονται 
δέ, τῆς γε μὴν Ἰνδῶν φωνῆς ἐπαίουσι. 
Go to Context



Ctesias Hist., Med., Fragmenta 
Volume-Jacobyʹ-F 3c,688,F, fragment 45q, line 3

                 4, 52: ὄνους 
ἀγρίους οὐκ ἐλάττους ἵππων τὰ μεγέθη ἐν 
Ἰνδοῖς γίνεσθαι πέπυσμαι. 
Go to Context



Ctesias Hist., Med., Fragmenta 
Volume-Jacobyʹ-F 3c,688,F, fragment 45q, line 12

      ἐκ δὴ τῶνδε τῶν ποικίλων κεράτων 
πίνειν Ἰνδοὺς ἀκούω, καὶ ταῦτα οὐ πάντας 
ἀλλὰ τοὺς τῶν Ἰνδῶν κρατίστους, ἐκ δια-
στημάτων αὐτοῖς χρυσὸν περιχέαντας, 
οἱονεὶ ψελλίοις τισὶ κοσμήσαντας βραχίονα 
ὡραῖον ἀγάλματος. 
Go to Context



Ctesias Hist., Med., Fragmenta 
Volume-Jacobyʹ-F 3c,688,F, fragment 45q, line 27

                   πεπίστευται δὲ τοὺς ἄλλους 
τοὺς ἀνὰ πᾶσαν τὴν γῆν ὄνους καὶ ἡμέρους 
καὶ ἀγρίους καὶ τὰ ἄλλα τὰ μώνυχα θηρία   
ἀστραγάλους οὐκ ἔχειν οὐδὲ μὴν ἐπὶ τῶι 
ἥπατι χολήν, ὄνους δὲ τοὺς Ἰνδοὺς <λέγει 
Κτησίας> τοὺς ἔχοντας τὸ κέρας ἀστρα-
γάλους φορεῖν καὶ ἀχόλους μὴ εἶναι. 
Go to Context



Ctesias Hist., Med., Fragmenta 
Volume-Jacobyʹ-F 3c,688,F, fragment 45q, line 40

        διατριβαὶ δὲ τοῖς ὄνοις τῶν Ἰνδι-
κῶν πεδίων τὰ ἐρημότατα. 
Go to Context



Ctesias Hist., Med., Fragmenta 
Volume-Jacobyʹ-F 3c,688,F, fragment 45q, line 42

                              ἐπιόντων δὲ τῶν 
Ἰνδῶν ἐπὶ τὴν ἄγραν αὐτῶν, τὰ μὲν ἁπαλὰ 
καὶ ἔτι νεαρὰ ἑαυτῶν νέμεσθαι κατόπιν 
ἐῶσιν, αὐτοὶ δὲ ὑπερμαχοῦσι, καὶ ἴασι τοῖς 
ἱππεῦσιν ὁμόσε, καὶ τοῖς κέρασι παίουσι. 
Go to Context



Ctesias Hist., Med., Fragmenta 
Volume-Jacobyʹ-F 3c,688,F, fragment 45q, line 60

      ζῶντα μὲν οὖν τέλειον οὐκ ἂν λάβοις, 
βάλλονται δὲ ἀκοντίοις καὶ ὀιστοῖς, καὶ τὰ 
κέρατα οὕτω τὰ ἐξ αὐτῶν Ἰνδοὶ [νεκρῶν] 
σκυλεύσαντες ὡς εἶπον περιέπουσιν. 
Go to Context



Ctesias Hist., Med., Fragmenta 
Volume-Jacobyʹ-F 3c,688,F, fragment 45q, line 62

                                        ὄνων 
δὲ Ἰνδῶν ἄβρωτόν ἐστι <τὸ> κρέας· τὸ δὲ 
αἴτιον, πέφυκεν εἶναι πικρότατον. 
Go to Context



Ctesias Hist., Med., Fragmenta 
Volume-Jacobyʹ-F 3c,688,F, fragment 45r, line 2

                 5, 3: ὁ ποταμὸς ὁ 
Ἰνδὸς ἄθηρός ἐστι, μόνος δὲ ἐν αὐτῶι τίκτεται σκώ-
ληξ φασί. 
Go to Context



Ctesias Hist., Med., Fragmenta 
Volume-Jacobyʹ-F 3c,688,F, fragment 45r, line 42

τοῦτο δὴ τὸ ἔλαιον βασιλεῖ τῶν Ἰνδῶν κομίζουσι, 
σημεῖα ἐπιβάλλοντες· ἔχειν γὰρ αὐτοῦ ἄλλον οὐδὲ 
ὅσον ῥανίδα ἐφεῖται. 
Go to Context



Ctesias Hist., Med., Fragmenta 
Volume-Jacobyʹ-F 3c,688,F, fragment 45r, line 51

                        τούτωι τοί φασι τὸν τῶν 
Ἰνδῶν βασιλέα καὶ τὰς πόλεις αἱρεῖν τὰς ἐς ἔχθραν 
προελθούσας οἱ, καὶ μήτε κριοὺς μήτε χελώνας μήτε 
τὰς ἄλλας ἑλεπόλεις ἀναμένειν, ἐπεὶ καταπιμπρὰς 
ἥιρηκεν· ἀγγεῖα γὰρ κεραμεᾶ ὅσον κοτύλην ἕκαστον 
χωροῦντα ἐμπλήσας αὐτοῦ καὶ ἀποφράξας ἄνωθεν 
ἐς τὰς πύλας σφενδονᾶι, ὅταν τε τύχηι <τῶν> θυρί-
δων, τὰ μὲν ἀγγεῖα προσαράττεται καὶ ἀπερράγη, 
καὶ τὸ ἔλαιον κατώλισθε, καὶ τῶν θυρῶν πῦρ κατε-
χύθη, καὶ ἄσβεστόν ἐστι· καὶ ὅπλα δὲ κάει καὶ ἀν-
θρώπους μαχομένους, καὶ ἄπλητόν ἐστι τὴν ἰσχύν. 
Go to Context



Ctesias Hist., Med., Fragmenta 
Volume-Jacobyʹ-F 3c,688,F, fragment 45s,Alpha, line 4

          Hist. 
mir. 150: περὶ δὲ λιμνῶν <Κτησίαν 
μὲν ἱστορεῖν> λέγει (scil. Καλλίμα-
χος [F 407 XXII Pf.]) τῶν ἐν Ἰνδοῖς 
λιμνῶν τὴν μὲν τὰ εἰς αὐτὴν ἀφιέμενα 
<μὴ> καταδέχεσθαι, καθάπερ τὴν ἐν 
Σικελίαι καὶ Μήδοις, πλὴν χρυσίον 
καὶ σίδηρον καὶ χαλκόν· καὶ ἄν τι 
ἐμπέσηι πλάγιον, ὀρθὸν ἐκβάλλειν· 
ἰᾶσθαι δὲ τὴν καλουμένην λεύκην. 
Go to Context



Ctesias Hist., Med., Fragmenta 
Volume-Jacobyʹ-F 3c,688,F, fragment 45s,Beta, line 2

          φλορ. 3: κρήνη ἐν Ἰν-
δοῖς, ἣ τοὺς κολυμβῶντας ἐπὶ τὴν γῆν   
ἐκβάλλει ὡς ἀπ' ὀργάνου, <ὡς ἱστορεῖ 
Κτησίας. 
Go to Context



Ctesias Hist., Med., Fragmenta 
Volume-Jacobyʹ-F 3c,688,F, fragment 46a, line 2

                 16, 31: <λέγει δὲ ἄρα Κτησίας ἐν λόγοις 
Ἰνδικοῖς> τοὺς καλουμένους Κυναμολγοὺς τρέφειν κύνας πολλοὺς κατὰ τοὺς 
Ὑρκανοὺς τὸ μέγεθος, καὶ εἶναί γε ἰσχυρῶς κυνοτρόφους. 
Go to Context



Ctesias Hist., Med., Fragmenta 
Volume-Jacobyʹ-F 3c,688,F, fragment 46b, line 3

   POLLUX 5, 41: οἱ δὲ Κυναμολγοὶ κύνες εἰσὶ περὶ τὰ ἕλη τὰ μεσημ-
βρινά, γάλα δὲ βοῶν ποιοῦνται τὴν τροφήν· καὶ τοὺς ἐπιόντας τοῦ θέρους τῶι 
ἔθνει βοῦς Ἰνδικοὺς καταγωνίζονται, ὡς ἱστορεῖ Κτησίας. 
Go to Context



Ctesias Hist., Med., Fragmenta 
Volume-Jacobyʹ-F 3c,688,F, fragment 47a, line 1

           Hist. mir. 146: (F 1 l α) τὴν δ' ἐν τοῖς Ἰνδοῖς 
κρήνην <Σ>ίλαν οὐδὲ τὸ κουφότατον τῶν <ἐμ>βληθέντων ἐᾶν ἐπιμένειν, ἀλλὰ 
πάντα· καθέλκειν· καὶ ταῦτα δὲ πλείους εἰρήκασιν καὶ ἐπὶ πλειόνων ὑδάτων. 
Go to Context



Ctesias Hist., Med., Fragmenta 
Volume-Jacobyʹ-F 3c,688,F, fragment 49a, line 1

          Ind. 3, 6: Κτησίης δὲ ὁ Κνίδιος τὴν Ἰνδῶν γῆν ἴσην 
τῆι ἄλληι Ἀσίηι λέγει. 
Go to Context



Ctesias Hist., Med., Fragmenta 
Volume-Jacobyʹ-F 3c,688,F, fragment 49b, line 2

   STRABON 15, 1, 12: Κτησίου μὲν οὐκ ἐλάττω 
τῆς ἄλλης Ἀσίας τὴν Ἰνδικὴν λέγοντος. 
Go to Context



Ctesias Hist., Med., Fragmenta 
Volume-Jacobyʹ-F 3c,688,F, fragment 50, line 2

                                      Od. σ 3): Κτησίας δὲ παρ' 
Ἰνδοῖς φησιν οὐκ εἶναι τῶι βασιλεῖ μεθυσθῆναι. 
Go to Context



Ctesias Hist., Med., Fragmenta 
Volume-Jacobyʹ-F 3c,688,F, fragment 51b, line 2

   TZETZ. Chil. 7, 629: Καρυανδέως Σκύλακος (709 F 7) ὑπάρχει τι βιβλίον / περὶ τὴν 
Ἰνδικὴν γράφον ἀνθρώπους πεφυκέναι, / οὕσπερ φασὶ Σκιάποδας καί γε τοὺς Ὠτολίκ-
νους / . 
Go to Context



Ctesias Hist., Med., Fragmenta 
Volume-Jacobyʹ-F 3c,688,F, fragment 63, line 4

              De mens. 4, 14: ἡ γένεσις τοῦ πιπέρεως <κατὰ τοὺς 
παλαιοὺς καὶ Κτησίαν τὸν Κνίδιον> τοιαύτη· ἔθνος ἐστὶ κατὰ τὴν Ἀζού-
μην, Βησσάδαι τοὔνομα, σώμασι σμικροῖς καὶ ἀδρανεστάτοις κεχρημένοι, κε-
φαλαῖς μεγάλαις καὶ ἀκάρτοις καὶ παρὰ τὴν Ἰνδῶν φύσιν ἁπλόθριξιν· σπηλαίοις 
δὲ ἐνοικοῦσιν ὑπογείοις, καὶ κρημνοβατεῖν ἐπιστάμενοι διὰ τὴν τοῦ τόπου συν-
τροφίαν. 
Go to Context



Ctesias Hist., Med., Fragmenta 
Volume-Jacobyʹ-F 3c,688,F, fragment 63, line 7

                         ὁ δὲ Μάξιμός φησι· «φυτόν ἐστι ἐν τῆι Ἰνδίαι πρῶτον μὲν 
ἀνάκανθον, φυτουργούμενον δὲ ὡς ἄμπελος ἀναδενδρὰς ἢ ὑπὸ χάρακα, φέρει δὲ τὸν καρπὸν 
βοτρυώδη ὡς τερέβινθος, ἔχει δὲ φύλλον κισσῶδες ὑπόμακρον. 
Go to Context



Ctesias Hist., Med., Fragmenta 
Volume-Jacobyʹ-F 3c,688,F, fragment 71, line 2

TZETZ. Chil. 8, 985/92: Ἡρόδοτος (3, 110/2), Διόδωρος (2, 49), <Κτησίας,> 
πάντες ἄλλοι / λέγουσι τὴν Εὐδαίμονα τυγχάνειν Ἀραβίαν, / ὥσπερ καὶ γῆν τὴν Ἰνδικήν, 
εὐωδεστάτην ἄγαν, / ἀρώμασιν ἐκπνέουσαν ὡς καὶ τὴν γῆν ἐκείνης, / καὶ λίθους κοπτο-
μένους δὲ ταύτης ἀρωματίζειν. 
Go to Context
\end{greek}

\section{Ephippus Comic}

\blockquote[From Wikipedia\footnote{\url{http://en.wikipedia.org/wiki/Ephippus_of_Athens}.}]{Ephippus (Ephippos) of Athens was an Ancient Greek comic poet of the middle comedy.

We learn this from the testimonies of Suidas and Antiochus of Alexandria[1], and from the allusions in his fragments to Plato, and the Academic philosophers,[2] and to Alexander of Pherae and his contemporaries, Dionysius the Elder, Cotys, Theodorus[disambiguation needed], and others.[3]}

\begin{greek}

Ephippus Comic., Fragmenta (0450: 002)
“Fragmenta comicorum Graecorum, vol. 3”, Ed. Meineke, A.
Berlin: Reimer, 1840, Repr. 1970.
Play Ger, fragment 1, line 7

καὶ περιοίκους εἶναι ταύτης 
Ἰνδούς, Λυκίους, Μυγδονιώτας, 
Κραναούς, Παφίους. 

\end{greek}


\section{Menander}
\blockquote[From Wikipedia\footnote{\url{http://en.wikipedia.org/wiki/Menander}}]{Menander (Greek: Μένανδρος, Menandros; c. 341/42– c. 290 BC) was a Greek dramatist and the best-known representative of Athenian New Comedy.[1] He was the author of more than a hundred comedies, and took the prize at the Lenaia festival eight times.[2] His record at the City Dionysia is unknown but may well have been similarly spectacular.}
\begin{greek}

Menander Comic., Fragmenta (0541: 045)
“Menandri quae supersunt, vol. 2, 2nd edn.”, Ed. Körte, A., Thierfelder, A.
Leipzig: Teubner, 1959.
Fragment 24, line 3

εὐποροῦμεν, οὐδὲ μετρίως· ἐκ Κυΐνδων χρυσίον, 
Περσικαὶ στολαὶ δ' ἐκεῖναι πορφυρᾶ τε στρώματα 
ἔνδον ἔστ', ἄνδρες, ποτήρι' <Ἰν>δι<κ>ά <τε> τορεύματα 
κἀκτυπωμάτων πρόσωπα, τραγέλαφοι, λαβρώνια. 

\end{greek}


\section{Clearchus of Soli}
\blockquote[From Wikipedia\footnote{\url{http://en.wikipedia.org/wiki/Clearchus_of_Soli}}]{Clearchus of Soli (Greek: Kλέαρχoς, Klearkhos) was a Greek philosopher of the 4th-3rd century BCE, belonging to Aristotle's Peripatetic school. He was born in Soli in Cyprus.

He wrote extensively on eastern cultures, and is thought to have traveled to the Bactrian city of Ai-Khanoum (Alexandria on the Oxus) in modern Afghanistan.
Writings

Clearchus wrote extensively around 320 BCE on eastern cultures, from Persia to India, and several fragments from him are known. His book "Of Education" (Greek: Περὶ παιδείας, Peri paideiās) was preserved by Diogenes Laertius.

Clearchus in particular expressed several theories on the connection between western and eastern religions. In "Of Education", he wrote that "the gymnosophists are descendants of the Magi". In another text, quoted by Josephus, Clearchus reported a dialogue with Aristotle, where the philosopher states that the Hebrews were descendants of the Indian philosophers:

    "Jews are derived from the Indian philosophers; they are named by the Indians Calami, and by the Syrians Judaei, and took their name from the country they inhabit, which is called Judea; but for the name of their city, it is a very awkward one, for they call it Jerusalem." Josephus, Contra Apionem, I, 22.[1] }
\begin{greek}


Clearchus Phil., Fragmenta (1270: 001)
“Klearchos”, Ed. Wehrli, F.
Basel: Schwabe, 1969; Die Schule des Aristoteles, vol. 3, 2nd edn..
Fragment 6, line 15

κἀκεῖνος τοίνυν τὸ μὲν γένος ἦν Ἰουδαῖος ἐκ τῆς Κοίλης Συρίας, οὗτοι δ' εἰσὶν   
ἀπόγονοι τῶν ἐν Ἰνδοῖς φιλοσόφων. 



Clearchus Phil., Fragmenta 
Fragment 6, line 16

                                          καλοῦνται δέ, ὥς φασιν, οἱ φιλόσοφοι παρὰ 
μὲν Ἰνδοῖς Καλανοί, παρὰ δὲ Σύροις Ἰουδαῖοι, τοὔνομα λαβόντες ἀπὸ τοῦ τόπου. 

\end{greek}


\section{Eudoxus of Cnidus}
\blockquote[From Wikipedia\footnote{\url{http://en.wikipedia.org/wiki/Eudoxus_of_Cnidus}}]{Eudoxus /ˈjuːdəksəs/ of Cnidus (Greek: Εὔδοξος ὁ Κνίδιος Eúdoxos ho Knídios; 408 BC – 355 BC) was a Greek astronomer, mathematician, scholar and student of Plato. All of his works are lost, though some fragments are preserved in Hipparchus' commentary on Aratus's poem on astronomy.[1] Theodosius of Bithynia's important work, Sphaerics, may be based on a work of Eudoxus.}
\begin{greek}

Eudoxus Astron., Fragmenta (1358: 001)
“Die Fragmente des Eudoxos von Knidos”, Ed. Lasserre, F.
Berlin: De Gruyter, 1966.
Fragment 240, line 2

PARAPEGMA MILES. 456 D I 1 [Χηλαὶ ἑσπέριαι δύνο]υσιν 
κατ' Εὔ[δοξον, κατὰ δὲ Ἰ]νδῶν Καλλανέα [die insequenti Χηλαὶ ἑς-
π]έριαι δύνουσιν. 

\end{greek}


\section{Aristoxenus}
\blockquote[From Wikipedia\footnote{\url{http://en.wikipedia.org/wiki/Aristoxenus}}]{Aristoxenus of Tarentum (Greek: Ἀριστόξενος; fl. 335 BC) was a Greek Peripatetic philosopher, and a pupil of Aristotle. Most of his writings, which dealt with philosophy, ethics and music, have been lost, but one musical treatise, Elements of Harmony, survives incomplete, as well as some fragments concerning rhythm and meter. The Elements is the chief source of our knowledge of ancient Greek music.[1]}
\begin{greek}


Aristoxenus Mus., Fragmenta (0088: 006)
“Aristoxenos”, Ed. Wehrli, F.
Basel: Schwabe, 1967; Die Schule des Aristoteles, vol. 2, 2nd edn..
Fragment 53, line 3

                                                                           ... φησὶ 
δ' Ἀριστόξενος ὁ μουσικὸς Ἰνδῶν εἶναι τὸν λόγον τοῦτον. 



Aristoxenus Mus., Fragmenta 
Fragment 53, line 6

                        τοῦ δ' εἰπόντος, ὅτι ζητῶν περὶ τοῦ ἀνθρωπίνου βίου, 
καταγελάσαι τὸν Ἰνδόν, λέγοντα μὴ δύνασθαί τινα τὰ ἀνθρώπινα κατιδεῖν 
ἀγνοοῦντά γε τὰ θεῖα. 


\end{greek}


\section{Aristoteles et Corpus Aristotelicum}
\blockquote[From Wikipedia\footnote{\url{http://en.wikipedia.org/wiki/Aristotle}}]{Aristotle (Ancient Greek: Ἀριστοτέλης [aristotélɛːs], Aristotélēs) (384 BC – 322 BC)[1] was a Greek philosopher and polymath, a student of Plato and teacher of Alexander the Great. His writings cover many subjects, including physics, metaphysics, poetry, theater, music, logic, rhetoric, linguistics, politics, government, ethics, biology, and zoology. Together with Plato and Socrates (Plato's teacher), Aristotle is one of the most important founding figures in Western philosophy. Aristotle's writings were the first to create a comprehensive system of Western philosophy, encompassing ethics, aesthetics, logic, science, politics, and metaphysics.}
\begin{greek}

Aristoteles et Corpus Aristotelicum Phil., De caelo (0086: 005)
“Aristote. Du ciel”, Ed. Moraux, P.
Paris: Les Belles Lettres, 1965.
Bekker page 298a, line 11

           Διὸ τοὺς ὑπολαμβάνοντας συν-
άπτειν τὸν περὶ τὰς Ἡρακλείας στήλας τόπον τῷ περὶ τὴν 
Ἰνδικήν, καὶ τοῦτον τὸν τρόπον εἶναι τὴν θάλατταν μίαν, μὴ 
λίαν ὑπολαμβάνειν ἄπιστα δοκεῖν· λέγουσι δὲ τεκμαιρόμενοι 
καὶ τοῖς ἐλέφασιν, ὅτι περὶ ἀμφοτέρους τοὺς τόπους τοὺς 
ἐσχάτους ὄντας τὸ γένος αὐτῶν ἐστιν, ὡς τῶν ἐσχάτων διὰ τὸ 
συνάπτειν ἀλλήλοις τοῦτο πεπονθότων. 



Aristoteles et Corpus Aristotelicum Phil., Ethica Eudemia (0086: 009)
“[Aristotelis ethica Eudemia]”, Ed. Susemihl, F.
Leipzig: Teubner, 1884, Repr. 1967.
Bekker page 1226a, line 29

                                                         διὸ οὐ βου-
λευόμεθα περὶ τῶν ἐν Ἰνδοῖς, οὐδὲ πῶς ἂν ὁ κύκλος τετρα-
γωνισθείη. 



Aristoteles et Corpus Aristotelicum Phil., De generatione animalium (0086: 012)
“Aristotelis de generatione animalium”, Ed. Drossaart Lulofs, H.J.
Oxford: Clarendon Press, 1965, Repr. 1972.
Bekker page 746a, line 34

                                                          σπάνια μὲν 
οὖν γίγνεται τὰ τοιαῦτα ἐπὶ τῶν ἄλλων, γίγνεται δὲ καὶ ἐπὶ 
κυνῶν καὶ ἀλωπέκων καὶ λύκων <καὶ θώων>· καὶ οἱ Ἰνδικοὶ δὲ κύ-
νες ἐκ θηρίου τινὸς κυνώδους γεννῶνται καὶ κυνός. 



Aristoteles et Corpus Aristotelicum Phil., Historia animalium (0086: 014)
“Aristote. Histoire des animaux, vols. 1–3”, Ed. Louis, P.
Paris: Les Belles Lettres, 1:1964; 2:1968; 3:1969.
Bekker page 499b, line 19

                                   Μονοκέρατα δὲ καὶ μώ-
νυχα ὀλίγα, οἷον ὁ Ἰνδικὸς ὄνος. 



Aristoteles et Corpus Aristotelicum Phil., Historia animalium 
Bekker page 499b, line 20

        Καὶ ἀστράγαλον δ' ὁ Ἰνδικὸς ὄνος ἔχει τῶν μωνύχων 
μόνον· ἡ γὰρ ὗς, ὥσπερ ἐλέχθη πρότερον, ἐπαμφοτερίζει, 
διὸ καὶ οὐ καλλιαστράγαλόν ἐστιν. 



Aristoteles et Corpus Aristotelicum Phil., Historia animalium 
Bekker page 501a, line 26

        Ἔστι δέ τι, εἰ δεῖ πιστεῦσαι Κτησίᾳ· ἐκεῖνος γὰρ τὸ   
ἐν Ἰνδοῖς θηρίον, ᾧ ὄνομα εἶναι μαρτιχόραν, τοῦτ' ἔχειν ἐπ' 
ἀμφότερά φησι τριστοίχους τοὺς ὀδόντας· εἶναι δὲ μέγεθος 
μὲν ἡλίκον λέοντα καὶ δασὺ ὁμοίως, καὶ πόδας ἔχειν ὁμοί-
ους, πρόσωπον δὲ καὶ ὦτα ἀνθρωποειδές, τὸ δ' ὄμμα 
γλαυκόν, τὸ δὲ χρῶμα κινναβάρινον, τὴν δὲ κέρκον ὁμοίαν 
τῇ τοῦ σκορπίου τοῦ χερσαίου, ἐν ᾗ κέντρον ἔχειν καὶ τὰς ἀπο-
φυάδας ἀπακοντίζειν, φθέγγεσθαι δ' ὅμοιον φωνῇ ἅμα 
σύριγγος καὶ σάλπιγγος, ταχὺ δὲ θεῖν οὐχ ἧττον τῶν ἐλά-
φων, καὶ εἶναι ἄγριον καὶ ἀνθρωποφάγον. 



Aristoteles et Corpus Aristotelicum Phil., Historia animalium 
Bekker page 571b, line 34

                     Ἐξαγριαίνον-  
ται δὲ καὶ οἱ ἐλέφαντες περὶ τὴν ὀχείαν, διόπερ φασὶν οὐκ 
ἐᾶν αὐτοὺς ὀχεύειν τὰς θηλείας τοὺς τρέφοντας ἐν τοῖς Ἰν-
δοῖς· ἐμμανεῖς γὰρ γινομένους ἐν τοῖς χρόνοις τούτοις ἀνα-
τρέπειν τὰς οἰκήσεις αὐτῶν ἅτε φαύλως ᾠκοδομημένας, 
καὶ ἄλλα πολλὰ ἐργάζεσθαι. 



Aristoteles et Corpus Aristotelicum Phil., Historia animalium 
Bekker page 597b, line 27

                               Ὅλως δὲ τὰ γαμψώνυχα 
πάντα βραχυτράχηλα καὶ πλατύγλωττα καὶ μιμητικά· 
καὶ γὰρ τὸ Ἰνδικὸν ὄρνεον ἡ ψιττάκη, τὸ λεγόμενον ἀνθρω-
πόγλωττον, τοιοῦτόν ἐστι· καὶ ἀκολαστότερον δὲ γίνεται, ὅταν 
πίῃ οἶνον. 



Aristoteles et Corpus Aristotelicum Phil., Historia animalium 
Bekker page 606a, line 8

                                                          Ἐν δὲ Λι-
βύῃ πάσῃ οὔτε σῦς ἄγριός ἐστιν οὔτ' ἔλαφος οὔτ' αἲξ ἄγριος· 
ἐν δὲ τῇ Ἰνδικῇ, ὡς φησὶ Κτησίας οὐκ ὢν ἀξιόπιστος, οὔτ' 
ἄγριος οὔτε ἥμερος ὗς, τὰ δ' ἄναιμα καὶ τὰ φολιδωτὰ 
πάντα μεγάλα. 



Aristoteles et Corpus Aristotelicum Phil., Historia animalium 
Bekker page 607a, line 4

                                                           Φασὶ 
δὲ καὶ ἐκ τοῦ τίγριος καὶ κυνὸς γίνεσθαι τοὺς Ἰνδικούς, οὐκ εὐ-
θὺς δ' ἀλλ' ἐπὶ τῆς τρίτης μίξεως· τὸ γὰρ πρῶτον γεννη-
θὲν θηριῶδες γίνεσθαί φασιν. 



Aristoteles et Corpus Aristotelicum Phil., Historia animalium 
Bekker page 607a, line 34

                                                     Ἔστι δὲ καὶ ἐν 
τῇ Ἰνδικῇ ὀφείδιόν τι, οὗ μόνου φάρμακον οὐκ ἔχουσιν. 



Aristoteles et Corpus Aristotelicum Phil., Historia animalium 
Bekker page 610a, line 19

        Χρῶνται δ' οἱ Ἰνδοὶ πολεμιστηρίοις, καθάπερ τοῖς ἄρ-
ρεσι, καὶ ταῖς θηλείαις· εἰσὶ μέντοι καὶ ἐλάττονες αἱ θή-
λειαι καὶ ἀψυχότεραι πολύ. 



Aristoteles et Corpus Aristotelicum Phil., Magna moralia (0086: 022)
“Aristotle, vol. 18 (ed. G.C. Armstrong)”, Ed. Susemihl, F.
Cambridge, Mass.: Harvard University Press, 1935, Repr. 1969.
Book 1, chapter 17, section 4, line 6

                         πολλάκις γὰρ διανοούμεθα   
ὑπὲρ τῶν ἐν Ἰνδοῖς, ἀλλ' οὔτι καὶ προαιρούμεθα. 



Aristoteles et Corpus Aristotelicum Phil., Meteorologica (0086: 026)
“Aristotelis meteorologicorum libri quattuor”, Ed. Fobes, F.H.
Cambridge, Mass.: Harvard University Press, 1919, Repr. 1967.
Bekker page 350a, line 25

                                         ῥεῖ δὲ καὶ ὁ Ἰνδὸς ἐξ αὐ-
τοῦ, πάντων τῶν ποταμῶν ῥεῦμα πλεῖστον. 



Aristoteles et Corpus Aristotelicum Phil., Meteorologica 
Bekker page 362b, line 21

τὸ γὰρ ἀπὸ Ἡρακλείων στηλῶν μέχρι τῆς Ἰνδικῆς τοῦ ἐξ Αἰ-
θιοπίας πρὸς τὴν Μαιῶτιν καὶ τοὺς ἐσχατεύοντας τῆς Σκυθίας 
τόπους πλέον ἢ πέντε πρὸς τρία τὸ μέγεθός ἐστιν, ἐάν τέ 
τις τοὺς πλοῦς λογίζηται καὶ τὰς ὁδούς, ὡς ἐνδέχεται λαμ-
βάνειν τῶν τοιούτων τὰς ἀκριβείας. 



Aristoteles et Corpus Aristotelicum Phil., Meteorologica 
Bekker page 362b, line 28

                                                                 τὰ 
δὲ τῆς Ἰνδικῆς ἔξω καὶ τῶν στηλῶν τῶν Ἡρακλείων διὰ τὴν 
θάλατταν οὐ φαίνεται συνείρειν, τῷ συνεχῶς εἶναι πᾶσαν 
οἰκουμένην)· ἐπεὶ δ' ὁμοίως ἔχειν ἀνάγκη τόπον τινὰ πρὸς 
τὸν ἕτερον πόλον ὥσπερ ὃν ἡμεῖς οἰκοῦμεν πρὸς τὸν ὑπὲρ 
ἡμῶν, δῆλον ὡς ἀνάλογον ἕξει τά τ' ἄλλα καὶ τῶν πνευ-
μάτων ἡ στάσις· ὥστε καθάπερ ἐνταῦθα βορέας ἐστίν, κἀκεί-
νοις ἀπὸ τῆς ἐκεῖ ἄρκτου τις ἄνεμος οὕτως ὤν, ὃν οὐδὲν δυ-
νατὸν διέχειν δεῦρο, ἐπεὶ οὐδ' ὁ βορέας οὗτος εἰς τὴν ἐνταῦθα 
οἰκουμένην πᾶσάν ἐστιν· ἔστιν γὰρ ὥσπερ ἀπόγειον τὸ πνεῦμα 
τὸ βόρειον [ἕως ὁ βορέας οὗτος εἰς τὴν ἐνταῦθα οἰκουμένην 




Aristoteles et Corpus Aristotelicum Phil., Mirabilium auscultationes (0086: 027)
“Aristotelis opera, vol. 2”, Ed. Bekker, I.
Berlin: Reimer, 1831, Repr. 1960.
Bekker page 834a, line 1

Φασὶ δὲ καὶ ἐν Ἰνδοῖς τὸν χαλκὸν οὕτως εἶναι λαμ-
πρὸν καὶ καθαρὸν καὶ ἀνίωτον, ὥστε μὴ διαγινώσκεσθαι τῇ 
χρόᾳ πρὸς τὸν χρυσόν, ἀλλ' ἐν τοῖς Δαρείου ποτηρίοις 
βατιακὰς εἶναί τινας καὶ πλείους, ἃς εἰ μὴ τῇ ὀσμῇ, ἄλ-
λως οὐκ ἦν διαγνῶναι πότερόν εἰσι χαλκαῖ ἢ χρυσαῖ. 



Aristoteles et Corpus Aristotelicum Phil., Mirabilium auscultationes 
Bekker page 835a, line 6

Θαυμαστὸν δέ τί φασιν ἐν Ἰνδοῖς περὶ τὸν ἐκεῖ μό-
λυβδον συμβαίνειν· ὅταν γὰρ τακεὶς εἰς ὕδωρ καταχυθῇ 
ψυχρόν, ἐκπηδᾶν ἐκ τοῦ ὕδατος. 



Aristoteles et Corpus Aristotelicum Phil., Mirabilium auscultationes 
Bekker page 835b, line 5

Ἐν Ἰνδοῖς ἐν τῷ Κέρατι καλουμένῳ ἰχθύδιά φασι 
γίνεσθαι ἃ ἐν τῷ ξηρῷ πλανᾶται καὶ πάλιν ἀποτρέχει 
εἰς τὸν ποταμόν. 



Aristoteles et Corpus Aristotelicum Phil., De mundo (0086: 028)
“Aristotelis qui fertur libellus de mundo”, Ed. Lorimer, W.L.
Paris: Les Belles Lettres, 1933.
Bekker page 393b, line 3

          Πρός γε μὴν ταῖς ἀνασχέσεσι τοῦ ἡλίου πάλιν εἰς-
ρέων ὁ Ὠκεανός, τὸν Ἰνδικόν τε καὶ Περσικὸν διανοίξας κόλ-
πον, ἀναφαίνει συνεχῆ τὴν Ἐρυθρὰν θάλασσαν διειληφώς. 



Aristoteles et Corpus Aristotelicum Phil., De mundo 
Bekker page 393b, line 15

                 Τούτων δὲ οὐκ ἐλάττους ἥ τε Ταπροβάνη πέραν 
Ἰνδῶν, λοξὴ πρὸς τὴν οἰκουμένην, καὶ ἡ † Φεβὸλ καλουμένη, 
κατὰ τὸν Ἀραβικὸν κειμένη κόλπον. 



Aristoteles et Corpus Aristotelicum Phil., De mundo 
Bekker page 397b, line 18

                                              Διὸ καὶ τῶν πα-
λαιῶν εἰπεῖν τινες προήχθησαν ὅτι πάντα ταῦτά ἐστι θεῶν 
πλέα τὰ καὶ δι' ὀφθαλμῶν ἰνδαλλόμενα ἡμῖν καὶ δι' ἀκοῆς 
καὶ πάσης αἰσθήσεως, τῇ μὲν θείᾳ δυνάμει πρέποντα κατα-
βαλλόμενοι λόγον, οὐ μὴν τῇ γε οὐσίᾳ. 



Aristoteles et Corpus Aristotelicum Phil., De mundo 
Bekker page 398a, line 28

                                                              Τὴν 
δὲ σύμπασαν ἀρχὴν τῆς Ἀσίας, περατουμένην Ἑλλησπόν-
τῳ μὲν ἐκ τῶν πρὸς ἑσπέραν μερῶν, Ἰνδῷ δὲ ἐκ τῶν πρὸς 
ἕω, διειλήφεσαν κατὰ ἔθνη στρατηγοὶ καὶ σατράπαι καὶ 
βασιλεῖς, δοῦλοι τοῦ μεγάλου βασιλέως, ἡμεροδρόμοι τε 
καὶ σκοποὶ καὶ ἀγγελιαφόροι φρυκτωριῶν τε ἐποπτῆρες. 



Aristoteles et Corpus Aristotelicum Phil., De partibus animalium (0086: 030)
“Aristote. Les parties des animaux”, Ed. Louis, P.
Paris: Les Belles Lettres, 1956.
Bekker page 643b, line 6

                                                   Πάντα γὰρ ὡς 
εἰπεῖν, ὅσα ἥμερα καὶ ἄγρια τυγχάνει ὄντα, οἷον ἄνθρω-
ποι, ἵπποι, βόες, κύνες ἐν τῇ Ἰνδικῇ, ὕες, αἶγες, πρόβατα· 
ὧν ἕκαστον, εἰ μὲν ὁμώνυμον, οὐ διῄρηται χωρίς, εἰ δὲ ταῦτα 
ἓν εἴδει, οὐχ οἷόν τ' εἶναι διαφορὰς τὸ ἄγριον καὶ τὸ ἥμερον. 



Aristoteles et Corpus Aristotelicum Phil., De partibus animalium 
Bekker page 663a, line 19

                Ἔστι δὲ τὰ πλεῖστα τῶν κερατοφόρων διχαλά, 
λέγεται δὲ καὶ μώνυχον, ὃν καλοῦσιν Ἰνδικὸν ὄνον. 



Aristoteles et Corpus Aristotelicum Phil., De partibus animalium 
Bekker page 663a, line 23

                                                              Τὰ μὲν 
οὖν πλεῖστα, καθάπερ καὶ τὸ σῶμα διῄρηται τῶν ζῴων οἷς 
ποιεῖται τὴν κίνησιν, δεξιὸν καὶ ἀριστερόν, καὶ κέρατα δύο 
πέφυκεν ἔχειν διὰ τὴν αἰτίαν ταύτην· ἔστι δὲ καὶ μονοκέ-
ρατα, οἷον ὅ τε ὄρυξ καὶ ὁ Ἰνδικὸς καλούμενος ὄνος. 



Aristoteles et Corpus Aristotelicum Phil., Politica (0086: 035)
“Aristotelis politica”, Ed. Ross, W.D.
Oxford: Clarendon Press, 1957, Repr. 1964.
Bekker page 1332b, line 24

                               εἰ μὲν τοίνυν εἴησαν τοσοῦτον 
διαφέροντες ἅτεροι τῶν ἄλλων ὅσον τοὺς θεοὺς καὶ τοὺς 
ἥρωας ἡγούμεθα τῶν ἀνθρώπων διαφέρειν, εὐθὺς πρῶτον 
κατὰ τὸ σῶμα πολλὴν ἔχοντας ὑπερβολήν, εἶτα κατὰ   
τὴν ψυχήν, ὥστε ἀναμφισβήτητον εἶναι καὶ φανερὰν τὴν 
ὑπεροχὴν τοῖς ἀρχομένοις τὴν τῶν ἀρχόντων, δῆλον ὅτι 
βέλτιον ἀεὶ τοὺς αὐτοὺς τοὺς μὲν ἄρχειν τοὺς δ' ἄρχεσθαι 
καθάπαξ· ἐπεὶ δὲ τοῦτ' οὐ ῥᾴδιον λαβεῖν οὐδ' ἔστιν ὥσπερ ἐν 
Ἰνδοῖς φησι Σκύλαξ εἶναι τοὺς βασιλέας τοσοῦτον δια-
φέροντας τῶν ἀρχομένων, φανερὸν ὅτι διὰ πολλὰς αἰτίας 
ἀναγκαῖον πάντας ὁμοίως κοινωνεῖν τοῦ κατὰ μέρος ἄρχειν 
καὶ ἄρχεσθαι. 



Aristoteles et Corpus Aristotelicum Phil., Problemata (0086: 036)
“Aristotelis opera, vol. 2”, Ed. Bekker, I.
Berlin: Reimer, 1831, Repr. 1960.
Bekker page 895b, line 25

                                                 καὶ γὰρ ἄνθρωποί 
που φαίνονται ἄγριοι ὄντες καὶ κύνες ἐν Ἰνδοῖς καὶ ἵπποι 
ἄλλοθι, ἀλλ' οὐ λέοντες ἥμεροι οὐδὲ παρδάλεις οὐδ' ἔχεις οὐδ' 
ἄλλα πολλά. 



Aristoteles et Corpus Aristotelicum Phil., Sophistici elenchi (0086: 040)
“Aristotelis topica et sophistici elenchi”, Ed. Ross, W.D.
Oxford: Clarendon Press, 1958, Repr. 1970 (1st edn. corr.).
Bekker page 167a, line 8

ὁμοίως δὲ καὶ τὸ παρὰ τὸ πῂ καὶ τὸ ἁπλῶς· οἷον ὁ 
Ἰνδός, ὅλος μέλας ὤν, λευκός ἐστι τοὺς ὀδόντας· λευκὸς ἄρα 
καὶ οὐ λευκός ἐστιν. 



Aristoteles et Corpus Aristotelicum Phil., Topica (0086: 044)
“Aristotelis topica et sophistici elenchi”, Ed. Ross, W.D.
Oxford: Clarendon Press, 1958, Repr. 1970 (1st edn. corr.).
Bekker page 116a, line 38

                                       ἔστι δὲ τοῦτο ταὐτὸ 
τῷ πρὸ αὐτοῦ, διαφέρει δὲ τῷ τρόπῳ· τὸ μὲν γὰρ τοὺς φί-
λους δικαίους εἶναι δι' αὑτὸ αἱρούμεθα, καὶ εἰ μηδὲν ἡμῖν 
μέλλει ἔσεσθαι, κἂν ἐν Ἰνδοῖς ὦσιν· τὸ δὲ τοὺς ἐχθροὺς δι' 
ἕτερον, ὅπως μηθὲν ἡμᾶς βλάπτωσιν. 



Aristoteles et Corpus Aristotelicum Phil., Fragmenta varia (0086: 051)
“Aristotelis qui ferebantur librorum fragmenta”, Ed. Rose, V.
Leipzig: Teubner, 1886, Repr. 1967.
Category 1, treatise title 3, fragment 35, line 4

                   γεγενῆσθαι γὰρ παρὰ μὲν Πέρσαις μάγους,   
παρὰ δὲ Βαβυλωνίοις ἢ Ἀσσυρίοις Χαλδαίους, καὶ γυμνο-
σοφιστὰς παρ' Ἰνδοῖς, παρά τε Κελτοῖς καὶ Γαλάταις τοὺς 
καλουμένους δρυίδας καὶ σεμνοθέους, καθά φησιν <Ἀριστο-
τέλης> ἐν τῷ μαγικῷ καὶ Σωτίων ἐν εἰκοστῷ τρίτῳ τῆς δια-
δοχῆς. 



Aristoteles et Corpus Aristotelicum Phil., Fragmenta varia 
Category 6, treatise title 37, fragment 263, line 1

                   c. 49: φασὶ δὲ καὶ ἐν Ἰνδοῖς τὸν χαλκὸν 
οὕτως εἶναι λαμπρὸν καὶ καθαρὸν καὶ ἀνίωτον, ὡς μὴ 
διαγινώσκεσθαι τῇ χρόᾳ πρὸς τὸν χρυσόν, ἀλλ' ἐν τοῖς Δα-
ρείου ποτηρίοις βατιάκας εἶναί τινας καπήλους, ἃς εἰ μὴ τῇ 
ὀσμῇ, ἄλλως οὐκ ἦν διαγνῶναι πότερόν εἰσι χαλκαῖ ἢ χρυσαῖ. 



Aristoteles et Corpus Aristotelicum Phil., Fragmenta varia 
Category 6, treatise title 37, fragment 264, line 7

                                                           (61) 
θαυμαστὸν δέ τι φασὶν ἐν Ἰνδοῖς περὶ τὸν ἐκεῖ μόλυβδον 
συμβαίνειν· ὅταν γὰρ τακεὶς εἰς ὕδωρ καταχυθῇ ψυχρόν, 
ἐκπηδᾶν ἐκ τοῦ ὕδατος. 

\end{greek}


\section{Megasthenes}
\blockquote[From Wikipedia\footnote{\url{http://en.wikipedia.org/wiki/Megasthenes}}]{Megasthenes[pronunciation?] (Μεγασθένης, ca. 350 – 290 BCE) was a Greek ethnographer and explorer in the Hellenistic period, author of the work Indica. He was born in Asia Minor (modern day Turkey) and became an ambassador of Seleucus I of the Seleucid dynasty possibly to Chandragupta Maurya in Pataliputra, India. However the exact date of his embassy is uncertain. Scholars place it before 298 BC, the date of Chandragupta's death.

Arrian explains that Megasthenes lived in Arachosia, with the satrap Sibyrtius, from where he visited India:

    "Megasthenes lived with Sibyrtius, satrap of Arachosia, and often speaks of his visiting Sandracottus, the king of the Indians." Arrian, Anabasis Alexandri [1]

We have more definite information regarding the parts of India Megasthenes visited. He entered the subcontinent through the district of the Pentapotamia, providing a full account of the rivers found there (thought to be the five affluents of the Indus that form the Punjab region), and proceeded from there by the royal road to Pataliputra. There are accounts of Megasthenes having visited Madurai (then, a bustling city and capital of the Pandyas), but he appears not to have visited any other parts of India.

At the beginning of his Indica, he refers to the older Indians who know about the prehistoric arrival of Dionysus and Hercules in India, which was a story very popular amongst the Greeks during the Alexandrian period. Particularly important are his comments on the religions of the Indians. He mentions the devotees of Heracles and Dionysus but he does not mention Buddhists, something that gives support to the theory that the latter religion was not widely known before the reign of Ashoka.[2]

His Indica served as an important source for many later writers such as Strabo and Arrian. He describes such features as the Himalayas and the island of Sri Lanka. He also describes a caste system different from the one that exists today, which shows that the caste system may to some extent be fluid and evolving. However, it might be that, being a foreigner, he was not adequately informed about the caste system. His description follows:

    The first is formed by the collective body of the Philosophers, which in point of number is inferior to the other classes, but in point of dignity preeminent over all. The philosopher who errs in his predictions incurs censure, and then observes silence for the rest of his life.
    The second caste consists of the Husbandmen, who appear to be far more numerous than the others. They devote the whole of their time to tillage; nor would an enemy coming upon a husbandman at work on his land do him any harm, for men of this class, being regarded as public benefactors, are protected from all injury.
    The third caste consists of the Shepherds and in general of all herdsmen who neither settle in towns nor in villages, but live in tents.
    The fourth caste consists of the Artizans. Of these some are armourers, while others make the implements that husbandmen and others find useful in their different callings. This class is not only exempted from paying taxes, but even receives maintenance from the royal exchequer.
    The fifth caste is the Military. It is well organized and equipped for war, holds the second place in point of numbers, and gives itself up to idleness and amusement in the times of peace. The entire force--men-at-arms, war-horses, war-elephants, and all--are maintained at the king's expense.
    The sixth caste consists of the Overseers. It is their province to inquire into and superintend all that goes on in India, and make report to the king, or, where there is not a king, to the magistrates.
    The seventh caste consists of the Councillors and Assessors,--of those who deliberate on public affairs. It is the smallest class, looking to number, but the most respected, on account of the high character and wisdom of its members; for from their ranks the advisers of the king are taken, and the treasurers, of the state, and the arbiters who settle disputes. The generals of the army also, and the chief magistrates, usually belong to this class.

Later writers such as Arrian, Strabo, Diodorus, and Pliny refer to Indica in their works. Of these writers, Arrian speaks most highly of Megasthenes, while Strabo and Pliny treat him with less respect. Indica contained many legends and fabulous stories, similar to those we find in the Indica of Ctesias.

Megasthenes' Indica is the first well-known Western account of India and he is regarded as one of the founders of the study of Indian history in the West. He is also the first foreign Ambassador to be mentioned in Indian history.

Megasthenes also comments on the presence of pre-Socratic views among the Brahmans and Jews. Five centuries later Clement of Alexandria, in his Stromateis, may have misunderstood Megasthenes to be responding to claims of Greek primacy by admitting Greek views of physics were preceded by those of Jews and Indians. Megasthenes, like Numenius of Apamea, was simply comparing the ideas of the different ancient cultures.[3]}
\begin{greek}


Megasthenes Hist., Fragmenta (1489: 003)
“FHG 2”, Ed. Müller, K.
Paris: Didot, 1841–1870.
Fragment t1-43, line 1

ΙΝΔ*ιΚΑ. 




Megasthenes Hist., Fragmenta 
Fragment 1, line 1

(ΕΠΙΤΟΜΗ.)


 Diodorus II, 35: Ἡ τοίνυν Ἰνδικὴ τετράπλευρος 
οὖσα τῷ σχήματι, τὴν μὲν πρὸς ἀνατολὰς νεύουσαν 
πλευρὰν καὶ τὴν πρὸς τὴν μεσημβρίαν ἡ μεγάλη πε-
ριέχει θάλαττα, τὴν δὲ πρὸς τὰς ἄρκτους τὸ Ἠμωδὸν 
ὄρος διείργει τῆς Σκυθίας, ἣν κατοικοῦσι τῶν Σκυθῶν 
οἱ προσαγορευόμενοι Σάκαι· τὴν δὲ τετάρτην τὴν πρὸς 
δύσιν ἐστραμμένην διείληφεν ὁ Ἰνδὸς προσαγορευόμενος 
ποταμὸς, μέγιστος ὢν σχεδὸν τῶν ἁπάντων μετὰ τὸν 
Νεῖλον. 



Megasthenes Hist., Fragmenta 
Fragment 1, line 9

          (2) Τὸ δὲ μέγεθος τῆς ὅλης Ἰνδικῆς φασιν ὑπάρ-
χειν ἀπὸ μὲν ἀνατολῶν πρὸς δύσιν δισμυρίων ὀκτακις-
χιλίων σταδίων, ἀπὸ δὲ τῶν ἄρκτων πρὸς μεσημβρίαν 
τρισμυρίων δισχιλίων. 



Megasthenes Hist., Fragmenta 
Fragment 1, line 15

                        Τηλικαύτη δὲ οὖσα τὸ μέγεθος 
δοκεῖ μάλιστα τοῦ κόσμου περιέχειν τὸν τῶν θερι-
νῶν τροπῶν κύκλον, καὶ πολλαχῇ μὲν ἐπ' ἄκρας τῆς 
Ἰνδικῆς ἰδεῖν ἔστιν ἀσκίους ὄντας τοὺς γνώμονας, 
νυκτὸς δὲ τὰς ἄρκτους ἀθεωρήτους· ἐν δὲ τοῖς ἐσχά-
τοις οὐδ' αὐτὸν τὸν ἀρκτοῦρον φαίνεσθαι· καθ' ὃν 
δὴ τρόπον φασὶ καὶ τὰς σκιὰς κεκλίσθαι πρὸς μεσημ-
βρίαν. 



Megasthenes Hist., Fragmenta 
Fragment 1, line 20

3. Ἡ δ' οὖν Ἰνδικὴ πολλὰ μὲν ὄρη καὶ μεγάλα ἔχει 
δένδρεσι παντοδαποῖς καρπίμοις πληθύοντα, πολλὰ δὲ 
πεδία καὶ μεγάλα καρποφόρα, τῷ μὲν κάλλει διάφορα, 
ποταμῶν δὲ πλήθεσι διαιρούμενα. 



Megasthenes Hist., Fragmenta 
Fragment 1, line 31

                  (4) Καὶ πλείστους δὲ καὶ μεγίστους 
ἐλέφαντας ἐκτρέφει, χορηγοῦσα τὰς τροφὰς ἀφθόνως, 
δι' ἃς ταῖς ῥώμαις τὰ θηρία ταῦτα πολὺ προέχει τῶν 
κατὰ τὴν Λιβύην γεννωμένων· διὸ καὶ πολλῶν θηρευο-
μένων ὑπὸ τῶν Ἰνδῶν καὶ πρὸς τοὺς πολεμικοὺς ἀγῶνας 
κατασκευαζομένων μεγάλας συμβαίνει γίνεσθαι ῥοπὰς 
πρὸς τὴν νίκην. 



Megasthenes Hist., Fragmenta 
Fragment 1, line 46

(7) Χωρὶς δὲ τῶν δημητριακῶν καρπῶν φύεται κατὰ 
τὴν Ἰνδικὴν πολλὴ μὲν κέγχρος, ἀρδευομένη τῇ τῶν 
ποταμίων ναμάτων δαψιλείᾳ, πολὺ δὲ ὄσπριον καὶ διά-
φορον, ἔτι δὲ ὄρυζα καὶ τὸ προσαγορευόμενον βόσπορον, 
καὶ μετὰ ταῦτ' ἄλλα πολλὰ τῶν πρὸς διατροφὴν χρη-
σίμων· καὶ τούτων τὰ πολλὰ ὑπάρχει αὐτοφυῆ· οὐκ 
ὀλίγους δὲ καὶ ἄλλους ἐδωδίμους καρποὺς φέρει δυνα-
μένους τρέφειν ζῷα, περὶ ὧν μακρὸν ἂν εἴη γράφειν. 



Megasthenes Hist., Fragmenta 
Fragment 1, line 53

(8) Διὸ καί φασι μηδέποτε τὴν Ἰνδικὴν ἐπισχεῖν λιμὸν 
ἢ καθόλου σπάνιν τῶν πρὸς τροφὴν ἥμερον ἀνηκόντων. 



Megasthenes Hist., Fragmenta 
Fragment 1, line 61

Διττῶν γὰρ ὄμβρων ἐν αὐτῇ γινομένων καθ' ἕκαστον 
ἔτος, τοῦ μὲν χειμερινοῦ, καθὰ παρὰ τοῖς ἄλλοις ὁ σπό-
ρος τῶν πυρίνων γίνεται καρπῶν, τοῦ δ' ἑτέρου κατὰ 
τὴν θερινὴν τροπὴν, καθ' ἣν σπείρεσθαι συμβαίνει τὴν 
ὄρυζαν καὶ τὸ βόσπορον, ἔτι δὲ σήσαμον καὶ κέγχρον, 
κατὰ [δὲ] τὸ πλεῖστον ἀμφοτέροις τοῖς καρποῖς οἱ κατὰ   
τὴν Ἰνδικὴν ἐπιτυγχάνουσι· πάντων δὲ (μὴ) τελεσφο-
ρουμένων, θατέρου τῶν καρπῶν οὐκ ἀποτυγχάνουσιν. 



Megasthenes Hist., Fragmenta 
Fragment 1, line 73

                        (10) Συμβάλλονται δὲ παρὰ 
τοῖς Ἰνδοῖς καὶ τὰ νόμιμα πρὸς τὸ μηδέποτε ἔνδειαν 
τροφῆς παρ' αὐτοῖς εἶναι· παρὰ μὲν γὰρ τοῖς ἄλλοις 
ἀνθρώποις οἱ πολέμιοι καταφθείροντες τὴν χώραν 
ἀγεώργητον κατασκευάζουσι· παρὰ δὲ τούτοις τῶν γεωρ-
γῶν ἱερῶν καὶ ἀσύλων ἐωμένων, οἱ πλησίον τῶν παρα-
τάξεων γεωργοῦντες ἀνεπαίσθητοι τῶν κινδύνων εἰσίν. 



Megasthenes Hist., Fragmenta 
Fragment 1, line 85

XXXVII. 11. Ἔχει δὲ καὶ ποταμοὺς ἡ χώρα τῶν 
Ἰνδῶν πολλοὺς καὶ μεγάλους πλωτοὺς, οἳ τὰς πηγὰς 
ἔχοντες ἐν τοῖς ὄρεσι τοῖς πρὸς τὰς ἄρκτους κεκλιμένοις 
φέρονται διὰ τῆς πεδιάδος, ὧν οὐκ ὀλίγοι συμμίσγοντες 
ἀλλήλοις ἐμβάλλουσιν εἰς ποταμὸν τὸν ὀνομαζόμενον 
Γάγγην. 



Megasthenes Hist., Fragmenta 
Fragment 1, line 100

                                            Καὶ γὰρ Ἀλέ-
ξανδρος ὁ Μακεδὼν ἁπάσης τῆς Ἀσίας κρατήσας μό-
νους τοὺς Γανδαρίδας οὐκ ἐπολέμησε· καταντήσας γὰρ 
ἐπὶ τὸν Γάγγην ποταμὸν μετὰ πάσης τῆς δυνάμεως, 
καὶ τοὺς ἄλλους Ἰνδοὺς καταπολεμήσας, ὡς ἐπύθετο 
τοὺς Γανδαρίδας ἔχειν τετρακισχιλίους ἐλέφαντας πο-
λεμικῶς κεκοσμημένου, ἀπέγνω τὴν ἐπ' αὐτοὺς στρα-
τείαν. 



Megasthenes Hist., Fragmenta 
Fragment 1, line 104

        (14) Ὁ δὲ παραπλήσιος τῷ Γάγγῃ ποταμὸς, 
προσαγορευόμενος δὲ Ἰνδὸς, ἄρχεται μὲν ὁμοίως ἀπὸ 
τῶν ἄρκτων, ἐμβάλλων δὲ εἰς τὸν Ὠκεανὸν ἀφορίζει 
τὴν Ἰνδικήν· πολλὴν δὲ διεξιὼν πεδιάδα χώραν δέχε-
ται ποταμοὺς οὐκ ὀλίγους πλωτοὺς, ἐπιφανεστάτους δὲ 
Ὕπανιν καὶ Ὑδάσπην καὶ Ἀκεσῖνον. 



Megasthenes Hist., Fragmenta 
Fragment 1, line 114

                          (16) Τοῦ δὲ κατὰ τοὺς ποτα-
μοὺς πλήθους καὶ τῆς τῶν ὑδάτων ὑπερβολῆς αἰτίαν 
φέρουσιν οἱ παρ' αὐτοῖς φιλόσοφοι καὶ φυσικοὶ τοιαύτην· 
τῆς Ἰνδικῆς φασι τὰς περικειμένας χώρας, τήν τε 
Σκυθῶν καὶ Βακτριανῶν, ἔτι δὲ καὶ τῶν Ἀριανῶν, 
ὑψηλοτέρας εἶναι τῆς Ἰνδικῆς· ὥστε εὐλόγως εἰς τὴν 
ὑποκειμένην χώραν πανταχόθεν συρρεούσας τὰς λιβά-
δας ἐκ τοῦ κατ' ὀλίγον ποιεῖν τοὺς τόπους καθύγρους 
καὶ γεννᾶν ποταμῶν πλῆθος. 



Megasthenes Hist., Fragmenta 
Fragment 1, line 120

                               (17) Ἴδιον δέ τι συμ-
βαίνει περί τινα τῶν κατὰ τὴν Ἰνδικὴν ποταμῶν τὸν   
ὀνομαζόμενον Σίλλαν, ῥέοντα δὲ ἔκ τινος ὁμωνύμου 
κρήνης· ἐπὶ γὰρ τούτου μόνου τῶν ἁπάντων ποταμῶν 
οὐδὲν τῶν ἐμβαλλομένων εἰς αὐτὸν ἐπιπλεῖ, πάντα 
δ' εἰς τὸν βυθὸν καταδύεται παραδόξως. 



Megasthenes Hist., Fragmenta 
Fragment 1, line 125

XXXVIII. 18. Τὴν δ' ὅλην Ἰνδικὴν οὖσαν ὑπερμε-
γέθη λέγεται κατοικεῖν ἔθνη πολλὰ καὶ παντοδαπὰ, 
καὶ τούτων μηδὲν ἔχειν τὴν ἐξ ἀρχῆς γένεσιν ἔπηλυν, 
ἀλλὰ πάντα δοκεῖν ὑπάρχειν αὐτόχθονα, πρὸς δὲ τούτοις 
μήτε ξενικὴν ἀποικίαν προσδέχεσθαι πώποτε μήτε εἰς 
ἄλλο ἔθνος ἀπεσταλκέναι. 



Megasthenes Hist., Fragmenta 
Fragment 1, line 139

20. Μυθολογοῦσι δὲ παρὰ τοῖς Ἰνδοῖς οἱ λογιώτατοι 
περὶ ὧν καθῆκον ἂν εἴη συντόμως· διελθεῖν. 



Megasthenes Hist., Fragmenta 
Fragment 1, line 144

                                                     Φασὶ γὰρ 
ἐν τοῖς ἀρχαιοτάτοις χρόνοις, παρ' αὐτοῖς ἔτι τῶν 
ἀνθρώπων κωμηδὸν οἰκούντων, παραγενέσθαι τὸν 
Διόνυσον ἐκ τῶν πρὸς ἑσπέρας τόπων ἔχοντα δύναμιν 
ἀξιόλογον· ἐπελθεῖν δὲ τὴν Ἰνδικὴν ἅπασαν, μηδεμιᾶς 
οὔσης ἀξιολόγου πόλεως τῆς δυναμένης ἀντιτάξασθαι. 



Megasthenes Hist., Fragmenta 
Fragment 1, line 158

        (22) Μετὰ δὲ ταῦτα τῆς παραθέσεως τῶν καρ-
πῶν ἐπιμεληθέντα μεταδιδόναι τοῖς Ἰνδοῖς, καὶ τὴν 
εὕρεσιν τοῦ οἴνου καὶ τῶν ἄλλων τῶν εἰς τὸν βίον χρη-
σίμων παραδοῦναι. 



Megasthenes Hist., Fragmenta 
Fragment 1, line 169

             Βασιλεύσαντα δὲ πάσης τῆς Ἰνδικῆς ἔτη 
δύο πρὸς τοῖς πεντήκοντα γήρᾳ τελευτῆσαι· διαδεξαμέ-
νους δὲ τοὺς υἱοὺς αὐτοῦ τὴν ἡγεμονίαν ἀεὶ τοῖς ἀφ' 
ἑαυτῶν ἀπολιπεῖν τὴν ἀρχήν· τὸ δὲ τελευταῖον πολ-
λαῖς γενεαῖς ὕστερον καταλυθείσης τῆς ἡγεμονίας δη-
μοκρατηθῆναι τὰς πόλεις. 



Megasthenes Hist., Fragmenta 
Fragment 1, line 177

XXXIX. 24. Περὶ μὲν οὖν τοῦ Διονύσου καὶ τῶν 
ἀπογόνων αὐτοῦ τοιαῦτα μυθολογοῦσιν οἱ τὴν ὀρεινὴν 
τῆς Ἰνδικῆς κατοικοῦντες. 



Megasthenes Hist., Fragmenta 
Fragment 1, line 185

                        (25) Γαμήσαντα δὲ πλείους γυ-  
ναῖκας, υἱοὺς μὲν πολλοὺς, θυγατέρα δὲ μίαν γεννῆ-
σαι· καὶ τούτων ἐνηλίκων γενομένων, πᾶσαν τὴν 
Ἰνδικὴν διελόμενον εἰς ἴσας τοῖς τέκνοις μερίδας ἅπαν-
τας τοὺς υἱοὺς ἀποδεῖξαι βασιλέας, μίαν δὲ θυγατέρα 
θρέψαντα καὶ ταύτην βασίλισσαν ἀποδεῖξαι. 



Megasthenes Hist., Fragmenta 
Fragment 1, line 201

      (28) Νομίμων δ' ὄντων παρὰ τοῖς Ἰνδοῖς ἐνίων 
ἐξηλλαγμένων θαυμασιώτατον ἄν τις ἡγήσαιτο τὸ κα-
ταδειχθὲν ὑπὸ τῶν ἀρχαίων παρ' αὐτοῖς φιλοσόφων· 
νενομοθέτηται γὰρ παρ' αὐτοῖς δοῦλον μηδένα τὸ πα-
ράπαν εἶναι, ἐλευθέρους δ' ὑπάρχοντας τὴν ἰσότητα 
τιμᾶν ἐν πᾶσι. 



Megasthenes Hist., Fragmenta 
Fragment 1, line 211

XL. 29. Τὸ δὲ πᾶν πλῆθος τῶν Ἰνδῶν εἰς ἑπτὰ μέρη 
διῄρηται, ὧν ἐστι τὸ μὲν πρῶτον σύστημα φιλοσόφων, 
πλήθει μὲν τῶν ἄλλων μερῶν λειπόμενον, τῇ δ' ἐπι-
φανείᾳ πάντων πρωτεῦον· ἀλειτούργητοι γὰρ ὄντες οἱ 
φιλόσοφοι πάσης ὑπουργίας οὔθ' ἑτέρων κυριεύουσιν 
οὔθ' ὑφ' ἑτέρων δεσπόζονται. 



Megasthenes Hist., Fragmenta 
Fragment 1, line 222

                                   (30) Παραλαμβάνονται δὲ 
ὑπὸ μὲν τῶν ἰδιωτῶν εἴς τε τὰς ἐν τῷ βίῳ θυσίας καὶ 
εἰς τὰς τῶν τετελευτηκότων ἐπιμελείας, ὡς θεοῖς γεγο-
νότες προσφιλέστατοι καὶ περὶ τῶν ἐν ᾍδου μάλιστα 
ἐμπείρως ἔχοντες, ταύτης τε τῆς ὑπουργίας δῶρά 
τε καὶ τιμὰς λαμβάνουσιν ἀξιολόγους· τῷ δὲ κοινῷ τῶν 
Ἰνδῶν μεγάλας παρέχονται χρείας παραλαμβανόμενοι 
μὲν κατὰ τὸ νέον ἔτος ἐπὶ τὴν μεγάλην σύνοδον, προ-
λέγοντες δὲ τοῖς πλήθεσι περὶ αὐχμῶν καὶ ἐπομβρίας, 
ἔτι δὲ ἀνέμων εὐπνοίας καὶ νόσων καὶ τῶν ἄλλων τῶν 
δυναμένων τοὺς ἀκούοντας ὠφελῆσαι. 



Megasthenes Hist., Fragmenta 
Fragment 1, line 244

                                    Τῆς δὲ χώρας 
μισθοὺς τελοῦσι τῷ βασιλεῖ διὰ τὸ πᾶσαν τὴν Ἰνδικὴν 
βασιλικὴν εἶναι, ἰδιώτῃ δὲ μηδενὶ γῆν ἐξεῖναι κεκτῆσθαι· 
χωρὶς δὲ τῆς μισθώσεως τετάρτην εἰς τὸ βασιλικὸν 
τελοῦσι. 



Megasthenes Hist., Fragmenta 
Fragment 1, line 253

34. Τρίτον δ' ἐστὶ φῦλον τὸ τῶν βουκόλων καὶ ποι-
μένων καὶ καθόλου πάντων τῶν νομέων, οἳ πόλιν μὲν 
ἢ κώμην οὐκ οἰκοῦσι, σκηνίτῃ δὲ βίῳ χρῶνται· οἱ 
δ' αὐτοὶ καὶ κυνηγοῦντες καθαρὰν ποιοῦσι τὴν χώραν 
ὀρνέων καὶ θηρίων· εἰς ταῦτα δὲ ἀσκοῦντες καὶ φιλο-
πονοῦντες ἐξημεροῦσι τὴν Ἰνδικὴν, πλήθουσαν πολλῶν 
καὶ παντοδαπῶν θηρίων τε καὶ ὀρνέων τῶν κατεσθιόν-
των τὰ σπέρματα τῶν γεωργῶν. 



Megasthenes Hist., Fragmenta 
Fragment 1, line 268

37. Ἕκτον δ' ἐστὶ τὸ τῶν ἐφόρων· οὗτοι δὲ πολυ-
πραγμονοῦντες πάντα καὶ ἐφορῶντες τὰ κατὰ τὴν Ἰν-
δικὴν ἀπαγγέλλουσι τοῖς βασιλεῦσιν, ἐὰν δὲ ἡ πόλις 
αὐτῶν ἀβασίλευτος ᾖ, τοῖς ἄρχουσιν. 



Megasthenes Hist., Fragmenta 
Fragment 1, line 278

39. Τὰ μὲν οὖν μέρη τῆς διῃρημένης πολιτείας 
παρ' Ἰνδοῖς σχεδὸν ταῦτά ἐστιν. 



Megasthenes Hist., Fragmenta 
Fragment 1, line 282

XLII. 40. Ἔχει δ' ἡ τῶν Ἰνδῶν χώρα πλείστους 
καὶ μεγίστους ἐλέφαντας, ἀλκῇ τε καὶ μεγέθει πολὺ 
διαφέροντας. 



Megasthenes Hist., Fragmenta 
Fragment 1, line 292

41. Εἰσὶ δὲ παρ' Ἰνδοῖς καὶ ἐπὶ τοὺς ξένους ἄρχον-
τες τεταγμένοι καὶ φροντίζοντες ὅπως μηδεὶς ξένος 
ἀδικῆται· τοῖς δ' ἀρρωστοῦσι τῶν ξένων ἰατροὺς εἰσά-
γουσι καὶ τὴν ἄλλην ἐπιμέλειαν ποιοῦνται, καὶ τελευ-
τήσαντας θάπτουσιν, ἔτι δὲ τὰ καταλειφθέντα χρήματα 
τοῖς προσήκουσιν ἀποδιδόασιν. 



Megasthenes Hist., Fragmenta 
Fragment 1, line 300

                                           Περὶ μὲν οὖν 
τῆς Ἰνδικῆς καὶ τῶν κατ' αὐτὴν ἀρχαιολογουμένων 
ἀρκεσθησόμεθα τοῖς ῥηθεῖσι. 



Megasthenes Hist., Fragmenta 
Fragment 2, line 3

                    ϝ, 6, 2: Τῆς δὲ ὡς ἐπὶ νότον 
Ἀσίας τετραχῆ αὖ τεμνομένης μεγίστην μὲν μοῖραν 
τῶν Ἰνδῶν γῆν ποιεῖ Ἐρατοσθένης τε καὶ Μεγασθένης, 
ὃς ξυνῆν μὲν Σιβυρτίῳ τῷ σατράπῃ τῆς Ἀραχωσίας· 
πολλάκις δὲ λέγει ἀφικέσθαι παρὰ Σανδράκοττον τὸν 
Ἰνδῶν βασιλέα· ἐλαχίστην δὲ ὅσην ὁ Εὐφράτης ποτα-
μὸς ἀπείργει ὡς πρὸς τὴν ἡμετέραν θάλασσαν· δύο δὲ 
αἱ μεταξὺ Εὐφράτου τε ποταμοῦ καὶ τοῦ Ἰνδοῦ ἀπειρ-
γόμεναι, αἱ δύο ξυντεθεῖσαι μόλις ἄξιαι τῇ Ἰνδῶν γῇ 
ξυμβαλεῖν· ἀπείργεσθαι δὲ τὴν Ἰνδῶν χώραν πρὸς μὲν 
ἕω τε καὶ ἀπηλιώτην ἄνεμον ἔστε ἐπὶ μεσημβρίαν 
τῇ μεγάλῃ θαλάσσῃ· τὸ πρὸς βορρᾶν δὲ αὐτῆς

ἀπείρ-



Megasthenes Hist., Fragmenta 
Fragment 2, line 15

Ἰνδῶν βασιλέα· ἐλαχίστην δὲ ὅσην ὁ Εὐφράτης ποτα-
μὸς ἀπείργει ὡς πρὸς τὴν ἡμετέραν θάλασσαν· δύο δὲ 
αἱ μεταξὺ Εὐφράτου τε ποταμοῦ καὶ τοῦ Ἰνδοῦ ἀπειρ-
γόμεναι, αἱ δύο ξυντεθεῖσαι μόλις ἄξιαι τῇ Ἰνδῶν γῇ 
ξυμβαλεῖν· ἀπείργεσθαι δὲ τὴν Ἰνδῶν χώραν πρὸς μὲν 
ἕω τε καὶ ἀπηλιώτην ἄνεμον ἔστε ἐπὶ μεσημβρίαν 
τῇ μεγάλῃ θαλάσσῃ· τὸ πρὸς βορρᾶν δὲ αὐτῆς ἀπείρ-
γειν τὸν Καύκασον τὸ ὄρος ἔστε ἐπὶ τοῦ Ταύρου τὴν 
ξυμβολήν· τὴν δὲ πρὸς ἑσπέραν τε καὶ ἄνεμον Ἰάπυγα 
ἔστε ἐπὶ τὴν μεγάλην θάλασσαν, ὁ Ἰνδὸς ποταμὸς ἀπο-
τέμνεται. 



Megasthenes Hist., Fragmenta 
Fragment 2, line 31

      ... Εἰ δὴ οὖν εἷς τε ποταμὸς παρ' ἑκάστοις, καὶ 
οὐ μεγάλοι οὗτοι ποταμοὶ, ἱκανοὶ γῆν πολλὴν ποιῆσαι, 
ἐς θάλασσαν προχεόμενοι, ὁπότε ἰλὺν καταφέροιεν καὶ 
πηλὸν ἐκ τῶν ἄνω τόπων, ἔνθενπερ αὐτοῖς αἱ πηγαί 
εἰσιν, οὐδὲ ὑπὲρ τῆς Ἰνδῶν ἄρα χώρας ἐς ἀπιστίαν ἰέναι 
ἄξιον, ὅπως πεδίον τε ἡ πολλή ἐστι, καὶ ἐκ τῶν ποτα-
μῶν τὸ πεδίον ἔχει προσκεχωσμένον. 



Megasthenes Hist., Fragmenta 
Fragment 2, line 36

                                         Ἕρμον μὲν γὰρ 
καὶ Κάϋστρον καὶ Κάϊκόν τε καὶ Μαίανδρον ἢ ὅσοι 
πολλοὶ ποταμοὶ τῆς Ἀσίας ἐς τήνδε τὴν ἐντὸς θάλασσαν 
ἐκδιδοῦσιν, οὐδὲ ξύμπαντας ξυντεθέντας ἑνὶ τῶν Ἰνδῶν 
ποταμῶν ἄξιον ξυμβαλεῖν πλήθους ἕνεκα τοῦ ὕδατος, 
μὴ ὅτι τῷ Γάγγῃ τῷ μεγίστῳ, ὅτῳ οὔτε Νείλου ὕδωρ τοῦ 
Αἰγυπτίου οὔτε ὁ Ἴστρος ὁ κατὰ τὴν Εὐρώπην ῥέων 
ἄξιοι ξυμβαλεῖν· ἀλλ' οὐδὲ τῷ Ἰνδῷ ποταμῷ ἐκεῖνοί γε 
πάντες ξυμμιχθέντες ἐς ἴσον ἔρχονται· ὃς μέγας τε 
εὐθὺς ἀπὸ τῶν πηγῶν ἀνίσχει, καὶ πεντεκαίδεκα ποτα-
μοὺς πάντας τῶν Ἀσιανῶν μείζονας παραλαβὼν, καὶ 
τῇ ἐπωνυμίᾳ κρατήσας, οὕτως ἐκδιδοῖ ἐς θάλασσαν. 



Megasthenes Hist., Fragmenta 
Fragment 2, line 45

Ταῦτά μοι ἐν τῷ παρόντι περὶ Ἰνδῶν τῆς χώρας λε-
λέχθω· τὰ δὲ ἄλλα ἀποκείσθω ἐς τὴν Ἰνδικὴν ξυγ-
γραφήν. 



Megasthenes Hist., Fragmenta 
Fragment 3, line 1

Strabo XV: Τὴν Ἰνδικὴν περιώρικεν ἀπὸ 
μὲν τῶν ἄρκτων τοῦ Ταύρου τὰ ἔσχατα· ἀπὸ δὲ τῆς 
Ἀριανῆς μέχρι τῆς ἑῴας θαλάττης, ἅπερ οἱ ἐπιχώριοι 
κατὰ μέρος Παροπάμισόν τε καὶ Ἠμωδὸν καὶ Ἴμαον 
καὶ ἄλλα ὀνομάζουσι, Μακεδόνες δὲ Καύκασον· ἀπὸ 
τῆς ἑσπέρας ὁ Ἰνδὸς ποταμός· τὸ δὲ νότιον καὶ τὸ προς-
εῷον πλευρὸν πολὺ μείζω τῶν ἑτέρων ὄντα προπέπτω-
κεν εἰς τὸ Ἀτλαντικὸν πέλαγος, καὶ γίνεται ῥομβοειδὲς 
τὸ τῆς χώρας σχῆμα, τῶν μειζόνων πλευρῶν ἑκατέρου 
πλεονεκτοῦντος παρὰ τὸ ἀπεναντίον πλευρὸν τρισχιλίοις 




Megasthenes Hist., Fragmenta 
Fragment 3, line 16

                                                    Τῆς μὲν 
οὖν ἑσπερίου πλευρᾶς ἀπὸ τῶν Καυκασίων ὀρῶν ἐπὶ 
τὴν νότιον θάλατταν στάδιοι μάλιστα λέγονται μύριοι 
τρισχίλιοι παρὰ τὸν Ἰνδὸν ποταμὸν μέχρι τῶν ἐκβο-
λῶν αὐτοῦ· ὥστ' ἀπεναντίον ἡ ἑωθινὴ προσλαβοῦσα 
τοὺς τῆς ἄκρας τρισχιλίους ἔσται μυρίων καὶ ἑξα-
κισχιλίων σταδίων. 



Megasthenes Hist., Fragmenta 
Fragment 3, line 34

                                                           .. 
Ἐκ δὲ τούτων πάρεστιν ὁρᾶν, ὅσον διαφέρουσιν αἱ τῶν 
ἄλλων ἀποφάσεις, Κτησίου μὲν οὐκ ἐλάττω τῆς 
ἄλλης Ἀσίας τὴν Ἰνδικὴν λέγοντος, Ὀνησικρίτου δὲ τρί-
τον μέρος τῆς οἰκουμένης, Νεάρχου δὲ μηνῶν ὁδὸν τετ-
τάρων τὴν δι' αὐτοῦ τοῦ πεδίου, Μεγασθένους δὲ καὶ 
Δηϊμάχου μετριασάντων μᾶλλον· ὑπὲρ γὰρ δισμυρίους 
τιθέασι σταδίους, τὸ ἀπὸ τῆς νοτίου θαλάττης ἐπὶ τὸν 
Καύκασον. 



Megasthenes Hist., Fragmenta 
Fragment 4, line 4

Strabo II.: Φησὶν ὁ Ἵππαρχος ἐν τῷ   
δευτέρῳ Ὑπομνήματι αὐτὸν τὸν Ἐρατοσθένη διαβάλλειν 
τὴν τοῦ Πατροκλέους πίστιν ἐκ τῆς πρὸς Μεγασθένη 
διαφωνίας περὶ τοῦ μήκους τῆς Ἰνδικῆς τοῦ κατὰ τὸ 
βόρειον πλευρὸν, τοῦ μὲν Μεγασθένους λέγοντος στα-
δίων μυρίων ἑξακισχιλίων, τοῦ δὲ Πατροκλέους χιλίοις 
λείπειν φαμένου. 



Megasthenes Hist., Fragmenta 
Fragment 6, line 2

Arrian. Ind. 3, 7: Μεγασθένει δὲ τὸ ἀπὸ ἀνατο-
λέων ἐς ἑσπέρην πλάτος ἐστὶ τῆς Ἰνδῶν γῆς, ὅ τι περ 
οἱ ἄλλοι μῆκος ποιέουσι· καὶ λέγει Μεγασθένης, μυρίων 
καὶ ἑξακισχιλίων σταδίων εἶναι, ἵναπερ τὸ βραχύτατον 
αὐτοῦ. 



Megasthenes Hist., Fragmenta 
Fragment 7a, line 3

Strabo II: 
Μεγασθένει φήσαντι 
ἐν τοῖς νοτίοις μέρεσι τῆς Ἰνδικῆς τάς τε ἄρκτους ἀπο-
κρύπτεσθαι καὶ τὰς σκιὰς ἀντιπίπτειν. 



Megasthenes Hist., Fragmenta 
Fragment 7b, line 2

                 Δηιμάχου φήσαντος μηδαμοῦ τῆς 
Ἰνδικῆς μήτ' ἀποκρύπτεσθαι τὰς ἄρκτους μήτ' ἀντι-
πίπτειν τὰς σκιὰς ὅπερ ὑπείληφεν ὁ Μεγασθένης. 



Megasthenes Hist., Fragmenta 
Fragment 9, line 2

Plinius H. N. VII, 22, 6: Ab iis   
in interiore situ Mo-
nedes et Suari, quorum mons Maleus, in quo umbrae 
ad septemtrionem cadunt hieme, aestate in austrum, 
per senos menses. Septemtriones eo tractu semel in 
anno apparere, nec nisi quindecim diebus, Baeton 
auctor est: hoc idem pluribus locis Indiae 
fieri, Megasthenes. 
 Strabo XV: Μεγασθένης δὲ τὴν εὐδαι-
μονίαν τῆς Ἰνδικῆς ἐπισημαίνεται τῷ δίκαρπον εἶναι 
καὶ δίφορον, καθάπερ καὶ Ἐρατοσθένης ἔφη, τὸν μὲν 
εἰπὼν σπόρον χειμερινὸν, τὸν δὲ θερινὸν, καὶ ὄμβρον 
ὁμοίως· οὐδὲν γὰρ ἔτος εὑρίσκεσθαί φησι πρὸς ἀμφοτέρους 
καιροὺς ἄνομβρον· ὥστε εὐετηρίαν ἐκ τούτου συμβαίνειν, 
ἀφόρου μηδέποτε τῆς γῆς οὔσης· τούς τε ξυλίνους καρ-
ποὺς γεννᾶσθαι πολλοὺς, καὶ τὰς ῥίζας τῶν φυτῶν, καὶ 
μάλιστα τῶν μεγάλων καλάμων, γλυκείας καὶ φύσει 
καὶ ἑψήσει, χλιαινομένου τοῦ ὕδατος τοῖς ἡλίοις, τοῦ 
τ' ἐκπίπτοντος ἐκ Διὸς, καὶ τοῦ ποταμίου. 



Megasthenes Hist., Fragmenta 
Fragment 11, line 2

Aelian. N A. XVII, 39: Ἐν τῇ Πραξιακῇ χώρᾳ, 
Ἰνδῶν δὲ αὕτη ἐστίν, Μεγασθένης φησὶ πιθήκους εἶναι 
τῶν μεγίστων κυνῶν οὐ μείους, ἔχειν δὲ καὶ οὐρὰς πη-  
χῶν πέντε· προσπεφυκέναι δὲ ἄρα αὐτοῖς καὶ προκόμια, 
καὶ πώγωνας καθειμένους καὶ βαθεῖς· καὶ τὸ μὲν πρός-
ωπον εἶναι πᾶν λευκούς, τὸ σῶμα δὲ μέλανας ἰδεῖν, 
ἡμέρους δὲ καὶ φιλανθρωποτάτους, καὶ τὸ τοῖς ἀλλα-
χόθι πιθήκοις συμφυὲς οὐκ ἔχειν τὸ κακόηθες. 



Megasthenes Hist., Fragmenta 
Fragment 12, line 2

Aelian. N. A. XVI, 41: Μεγασθένης φησὶ κατὰ 
τὴν Ἰνδικὴν σκορπίους γίνεσθαι πτερωτοὺς μεγέθει με-
γίστους, τὸ κέντρον δὲ ἐγχρίπτειν τοῖς Εὐρωπαίοις 
παραπλησίως· γίνεσθαι δὲ καὶ ὄφεις αὐτόθι, καὶ τούτους 
πτηνούς· ἐπιφοιτᾶν δὲ οὐ καθ' ἡμέραν, ἀλλὰ νύκτωρ, 
καὶ ἀφιέναι ἐξ αὑτῶν οὖρον, ὅπερ οὖν ἐὰν κατά τινος 
ἀποστάξῃ σώματος, σῆψιν ἐργάζεται παραχρῆμα. 



Megasthenes Hist., Fragmenta 
Fragment 13b, line 4

Aelianus XVI, 
20: 
Ἐν 
τοῖς χωρίοις τοῖς ἐν Ἰνδίᾳ, λέγω δὴ τοῖς ἐνδοτάτω, 
ὄρη φασὶν εἶναι δύσβατά τε καὶ ἔνθηρα· καὶ ἔχειν ζῷα 
ὅσα καὶ ἡ καθ' ἡμᾶς τρέφει γῆ, ἄγρια δέ· καὶ γάρ τοι 
καὶ τὰς ὄϊς τὰς ἐκεῖ φασιν εἶναι καὶ ταύτας θηρία, καὶ 
κύνας καὶ αἶγας καὶ βοῦς, αὐτόνομά τε ἀλᾶσθαι καὶ 
ἐλεύθερα, ἀφειμένα νομευτικῆς ἀρχῆς. 



Megasthenes Hist., Fragmenta 
Fragment 13b, line 11

                                            Πλήθη δὲ αὐτῶν 
καὶ ἀριθμοῦ πλείω φασὶν οἱ τούτων συγγραφεῖς, καὶ 
οἱ τῶν Ἰνδῶν λόγιοι· ἐν δὴ τοῖς καὶ τοὺς Βραχμᾶνας 
ἀριθμεῖν ἄξιον, καὶ γάρ τοι καὶ ἐκεῖνοι ὑπὲρ τῶνδε 
ὁμολογοῦσι τὰ αὐτά. 



Megasthenes Hist., Fragmenta 
Fragment 13b, line 36

                                                       Νομὰς 
δὲ ἐρήμους ἀσπάζεται καὶ πλανᾶται μόνον· ὥρᾳ δὲ   
ἀφροδίτης τῆς σφετέρας συνδυασθεὶς πρὸς τὴν θήλειαν 
πεπράϋνται, καὶ μέντοι καὶ συννόμω ἐστόν· εἶτα ταύτης 
παραδραμούσης καὶ τῆς θηλείας κυούσης, ἐκθηριοῦται 
αὖθις, καὶ μονίας ἐστὶν ὅδε ὁ Ἰνδὸς καρτάζωνος. 



Megasthenes Hist., Fragmenta 
Fragment 13b, line 42

      c. 21: Ὑπερελθόντι τὰ ὄρη τὰ γειτνιῶντα τοῖς 
Ἰνδοῖς κατὰ τὴν ἐνδοτάτην πλευρὰν φανοῦνται, φασὶν, 
αὐλῶνες δασύτατοι, καὶ καλεῖταί γε ὑπ' Ἰνδῶν ὁ χῶρος 
Κόρουδα· ἀλῶνται δὲ ἄρα, φασὶν, ἐν 
τοῖσδε τοῖς αὐλῶσι ζῷα Σατύροις ἐμφερῆ τὰς μορφὰς, 
τὸ πᾶν σῶμα λάσια, καὶ ἔχει κατὰ τῆς ἰξύος ἵππουριν. 



Megasthenes Hist., Fragmenta 
Fragment 15, line 2

Plinius VIII, 14, 1: Megasthenes scribit in 
India serpentes in tantam magnitudinem adolescre, 
ut solidos hauriant cervos taurosque. 
 Aelian. N. A. VIII, 7: Μεγασθένους ἀκούω λέ-
γοντος περὶ τὴν τῶν Ἰνδῶν θάλατταν γίγνεσθαί τι ἰχθύ-
διον, καὶ τοῦτο μὲν ὅταν ζῇ, ἀθέατον εἶναι, κάτω που 
νηχόμενον καὶ ἐν βυθῷ, ἀποθανὸν δὲ ἀναπλεῖν· οὗ τὸν 
ἁψάμενον λειποθυμεῖν καὶ ἐκθνήσκειν τὰ πρῶτα, εἶτα 
μέντοι καὶ ἀποθνήσκειν. 



Megasthenes Hist., Fragmenta 
Fragment 17, line 2

Plinius H. N. VI, 24: Megasthenes flumine 
dividi (Taprobanen insulam tradit) incolasque 
palaeogonos appellari, aurique margaritarumque 
fertiliores quam Indos.   
 Antigonus Hist. Mirab. c. 147: 
Μεγασθένην δὲ τὸν τὰ Ἰνδικὰ γεγραφότα 
ἱστορεῖν, ἐν τῇ κατὰ τὴν Ἰνδικὴν θαλάττῃ δένδρεα 
φύεσθαι. 



Megasthenes Hist., Fragmenta 
Fragment 18a, line 2

Arrianus Ind. 4, 2 – 13: Αὐτοῖν δὲ τοῖν μεγί-
στοιν ποταμοῖν τοῦ τε Γάγγεω καὶ τοῦ Ἰνδοῦ τὸν Γάγ-
γεα μεγέθει πολύ τι ὑπερφέρειν Μεγασθένης ἀνέγραψεν, 
καὶ ὅσοι ἄλλοι μνήμην τοῦ Γάγγεω ἔχουσιν· (3) αὐτόν 
τε γὰρ μέγαν ἀνίσχειν ἐκ τῶν πηγέων, δέχεσθαί τε ἐς 
αὑτὸν τόν τε Καϊνὰν ποταμὸν, καὶ τὸν Ἐραννοβόαν 
καὶ τὸν Κοσσόανον, πάντας 
πλωτούς· ἔτι δὲ Σῶνόν τε ποταμὸν καὶ Σιττόκατιν 
καὶ Σολόματιν, καὶ τούτους πλω-
τούς. 



Megasthenes Hist., Fragmenta 
Fragment 18a, line 14

       (4) Ἐπὶ δὲ Κονδοχάτην τε καὶ Σάμβον καὶ 
Μάγωνα καὶ Ἀγόρανιν καὶ Ὤμαλιν· 
ἐμβάλλουσι δὲ ἐς αὐτὸν Κομμενάσης 
τε μέγας ποταμὸς καὶ Κάκουθις καὶ Ἀνδώματις 
ἐξ ἔθνεος Ἰνδικοῦ τοῦ Μανδιαδινῶν ῥέων· (5) καὶ ἐπὶ 
τούτοισιν Ἄμυστις παρὰ πόλιν Καταδούπην, καὶ Ὀξύ-
ματις ἐπὶ 
Παζάλαισι καλου-
μένοισι, καὶ Ἐρρένυσις 
ἐν Μάθαισιν, ἔθνεϊ Ἰνδικῷ, ξυμβάλλει τῷ Γάγγῃ. 



Megasthenes Hist., Fragmenta 
Fragment 18a, line 26

(8) Τῷ δὲ Ἰνδῷ ἐς ταὐτὸν ἔρχεται. 



Megasthenes Hist., Fragmenta 
Fragment 18a, line 33

                                                           (10) Ὁ δὲ 
Ἀκεσίνης ἐν Μαλλοῖς ξυμβάλλει τῷ Ἰνδῷ· καὶ Τούταπος 
δὲ μέγας ποταμὸς ἐς τὸν Ἀκεσίνην ἐκδιδοῖ. 



Megasthenes Hist., Fragmenta 
Fragment 18a, line 36

                                                    Τούτων 
ὁ Ἀκεσίνης ἐμπλησθεὶς καὶ τῇ ἐπικλήσει ἐκνικήσας αὐτὸς 
τῷ ἑωϋτοῦ ἤδη οὐνόματι ἐσβάλλει ἐς τὸν Ἰνδόν· (11) Κω-
φὴν δὲ ἐν Πευκελαιήτιδι, ἅμα οἷ ἄγων Μαλάμαντόν 
τε καὶ Σόαστον καὶ Γαρροίαν, ἐκδιδοῖ 
ἐς τὸν Ἰνδόν. 



Megasthenes Hist., Fragmenta 
Fragment 18a, line 41

                   (12) Καθύπερθε δὲ τούτων Πάρενος 
καὶ Σάπαρνος, οὐ πολὺ διέχοντες, ἐμβάλ-
λουσιν ἐς τὸν Ἰνδόν. 



Megasthenes Hist., Fragmenta 
Fragment 18b, line 2

Τὸ δὲ αἴτιον ὅστις ἐθέλει φράζειν τοῦ 
πλήθεός τε καὶ μεγέθεος τῶν Ἰνδῶν ποταμῶν, φραζέτω· 
ἐμοὶ δὲ καὶ ταῦτα ὡς ἀκοὴ ἀναγεγράφθω. 



Megasthenes Hist., Fragmenta 
Fragment 18b, line 5

                                                 Ἐπεὶ καὶ 
ἄλλων πολλῶν ποταμῶν οὐνόματα Μεγασθένης ἀνέγρα-
ψεν, οἳ ἔξω τοῦ Γάγγεώ τε καὶ τοῦ Ἰνδοῦ ἐκδιδοῦσιν 
ἐς τὸν ἑῷόν τε καὶ μεσημβρινὸν τὸν ἔξω πόντον· ὥστε 
τοὺς πάντας ὀκτὼ καὶ πεντήκοντα λέγει ὅτι εἰσὶν Ἰνδοὶ 
ποταμοὶ ναυσίποροι πάντες. 



Megasthenes Hist., Fragmenta 
Fragment 18b, line 9

                              Ἀλλ' οὐδὲ Μεγασθένης   
πολλὴν δοκέει μοι ἐπελθεῖν τῆς Ἰνδῶν χώρης, πλήν 
γε ὅτι πλεῦνα ἢ οἱ ξὺν Ἀλεξάνδρῳ τῷ Φιλίππου ἐπελ-
θόντες. 



Megasthenes Hist., Fragmenta 
Fragment 19a, line 2

Arrianus Ind. 6, 2: Καὶ τόδε λέγει Μεγασθένης 
ὑπὲρ ποταμοῦ Ἰνδικοῦ, Σίλαν μὲν εἶναί οἱ οὔνομα, ῥέειν 
δὲ ἀπὸ κρήνης ἐπωνύμου τῷ ποταμῷ διὰ τῆς χώρης 
τῆς Σιλέων, καὶ τούτων ἐπωνύμων τοῦ ποταμοῦ τε καὶ 
τῆς κρήνης· (3) τὸ δὲ ὕδωρ παρέχεσθαι τοῖόνδε· οὐδὲν 
εἶναι ὅτεῳ ἀντέχει τὸ ὕδωρ, οὔτε τι νήχεσθαι ἐπ' αὐτοῦ 
οὔτε τι ἐπιπλέειν, ἀλλὰ πάντα γὰρ ἐς βυσσὸν δύνειν· 
οὕτω τι ἀμενηνότερον πάντων εἶναι τὸ ὕδωρ ἐκεῖνο καὶ 
ἠεροειδέστερον. 



Megasthenes Hist., Fragmenta 
Fragment 20, line 2

E LIBRO SECUNDO.


 Strabo XV: Ἡμῖν δὲ τίς ἂν δικαία γένοιτο 
πίστις περὶ τῶν Ἰνδικῶν ἐκ τῆς τοιαύτης στρατείας 
τοῦ Κύρου ἢ τῆς Σεμιράμιδος; 



Megasthenes Hist., Fragmenta 
Fragment 20, line 5

Συναποφαίνεται δέ πως καὶ Μεγασθένης τῷ λόγῳ 
τούτῳ, κελεύων ἀπιστεῖν ταῖς ἀρχαίαις περὶ Ἰνδῶν 
ἱστορίαις· οὔτε γὰρ παρ' Ἰνδῶν ἔξω σταλῆναί ποτε 
στρατιὰν, οὔτ' ἐπελθεῖν ἔξωθεν καὶ κρατῆσαι, πλὴν 
τῆς μεθ' Ἡρακλέους καὶ Διονύσου, καὶ τῆς νῦν μετὰ 
Μακεδόνων. 



Megasthenes Hist., Fragmenta 
Fragment 20, line 16

                          Ἰδάνθυρσον δὲ τὸν Σκύθην ἐπι-
δραμεῖν τῆς Ἀσίας μέχρι Αἰγύπτου· τῆς Ἰνδικῆς δὲ 
μηδένα τούτων ἅψασθαι. 



Megasthenes Hist., Fragmenta 
Fragment 20, line 19

                          Πέρσας δὲ μισθοφόρους μὲν ἐκ 
τῆς Ἰνδικῆς μεταπέμψασθαι Ὕδρακας· ἐκεῖ δὲ μὴ 
στρατεῦσαι, ἀλλ' ἐγγὺς ἐλθεῖν μόνον, ἡνίκα Κῦρος 
ἤλαυνεν ἐπὶ Μασσαγέτας. 



Megasthenes Hist., Fragmenta 
Fragment 21, line 2

Arrianus Ind. 5, 4: Οὗτος ὦν ὁ Μεγασθένης λέγει, 
οὔτε Ἰνδοὺς ἐπιστρατεῦσαι οὐδαμοῖσιν ἀνθρώποισιν, 
οὔτε Ἰνδοῖσιν ἄλλους ἀνθρώπους· (5) ἀλλὰ Σέσωστριν 
μὲν τὸν Αἰγύπτιον, τῆς Ἀσίης καταστρεψάμενον τὴν 
πολλὴν, ἔστε ἐπὶ τὴν Εὐρώπην σὺν στρατιῇ ἐλάσαντα 
ὀπίσω ἀπονοστῆσαι· (6) Ἰνδάθυρσιν δὲ τὸν Σκύθεα ἐκ 
Σκυθίης ὁρμηθέντα πολλὰ μὲν τῆς Ἀσίης ἔθνεα κατα-
στρέψασθαι, ἐπελθεῖν δὲ καὶ τὴν Αἰγυπτίων γῆν κρα-
τέοντα· (7) Σεμίραμιν δὲ τὴν Ἀσσυρίην ἐπιχειρέειν, 
μὲν στέλλεσθαι εἰς Ἰνδοὺς, ἀποθανεῖν δὲ πρὶν τέλος 
ἐπιθεῖναι τοῖσι βουλεύμασιν· ἀλλὰ Ἀλέξανδρον γὰρ 




Megasthenes Hist., Fragmenta 
Fragment 21, line 12

οὔτε Ἰνδοῖσιν ἄλλους ἀνθρώπους· (5) ἀλλὰ Σέσωστριν 
μὲν τὸν Αἰγύπτιον, τῆς Ἀσίης καταστρεψάμενον τὴν 
πολλὴν, ἔστε ἐπὶ τὴν Εὐρώπην σὺν στρατιῇ ἐλάσαντα 
ὀπίσω ἀπονοστῆσαι· (6) Ἰνδάθυρσιν δὲ τὸν Σκύθεα ἐκ 
Σκυθίης ὁρμηθέντα πολλὰ μὲν τῆς Ἀσίης ἔθνεα κατα-
στρέψασθαι, ἐπελθεῖν δὲ καὶ τὴν Αἰγυπτίων γῆν κρα-
τέοντα· (7) Σεμίραμιν δὲ τὴν Ἀσσυρίην ἐπιχειρέειν, 
μὲν στέλλεσθαι εἰς Ἰνδοὺς, ἀποθανεῖν δὲ πρὶν τέλος 
ἐπιθεῖναι τοῖσι βουλεύμασιν· ἀλλὰ Ἀλέξανδρον γὰρ 
στρατεῦσαι ἐπὶ Ἰνδοὺς μοῦνον. 



Megasthenes Hist., Fragmenta 
Fragment 21, line 14

                                     (8) Καὶ πρὸ Ἀλεξάν-
δρου Διονύσου μὲν πέρι πολλὸς λόγος κατέχει, ὡς καὶ 
τούτου στρατεύσαντος ἐς Ἰνδοὺς, καὶ καταστρεψαμένου 
Ἰνδούς· Ἡρακλέος δὲ πέρι οὐ πολλός. 



Megasthenes Hist., Fragmenta 
Fragment 21, line 18

                                              (9) Διονύσου 
μέν γε καὶ Νύσσα πόλις μνῆμα οὐ φαῦλον τῆς στρατη-
λασίης καὶ ὁ Μηρὸς τὸ ὄρος, καὶ ὁ κισσὸς ὅτι ἐν τῷ 
ὄρεϊ τούτῳ φύεται· καὶ αὐτοὶ οἱ Ἰνδοὶ ὑπὸ τυμπάνων 
τε καὶ κυμβάλων στελλόμενοι ἐς τὰς μάχας· καὶ ἐσθὴς 
αὐτοῖσι κατάστικτος ἐοῦσα, καθάπερ τοῦ Διονύσου τοῖσι 
βάκχοισιν· (10) Ἡρακλέος δὲ οὐ πολλὰ ὑπομνήματα. 



Megasthenes Hist., Fragmenta 
Fragment 21, line 30

                                                   (12) Καὶ 
ἐν Σίβαισιν, Ἰνδικῷ γένεϊ, ὅτι δορὰς ἀμπεχομένους 
εἶδον τοὺς Σίβας, ἀπὸ τῆς Ἡρακλέος στρατηλασίης 
ἔφασκον τοὺς ὑπολειφθέντας εἶναι τοὺς Σίβας· καὶ γὰρ 
καὶ σκυτάλην φέρουσί τε οἱ Σίβαι, καὶ τοῖς βουσὶν αὐ-
τῶν ῥόπαλον ἐπικέκαυται· καὶ τοῦτο ἐς μνήμην ἀνέφε-
ρον τοῦ ῥοπάλου τοῦ Ἡρακλέος. 



Megasthenes Hist., Fragmenta 
Fragment 22a, line 8

          <Καὶ Μεγασθένης> δὲ <ἐν τῇ δʹ> (l. δευτέρᾳ) 
<τῶν Ἰνδικῶν> μνημονεύει αὐτῶν, δι' ἧς ἀποφαίνειν 
πειρᾶται τοῦτον τὸν βασιλέα τῇ ἀνδρείᾳ καὶ τῷ μεγέ-
θει τῶν πράξεων ὑπερβεβηκότα τὸν Ἡρακλέα· κατα-
στρέψασθαι γὰρ αὐτόν φησι Λιβύης τὴν πολλὴν καὶ 
Ἰβηρίαν. 



Megasthenes Hist., Fragmenta 
Fragment 22a, line 14

            Καὶ Διοκλῆς δὲ ἐν τῇ δευτέρᾳ τῶν Περσικῶν 
μνημονεύει τούτου τοῦ βασιλέως, καὶ Φιλόστρατος ἐν 
ταῖς Ἰνδικαῖς αὐτοῦ καὶ Φοινικικαῖς ἱστορίαις, ὅτι οὗ-
τος ὁ βασιλεὺς ἐπολιόρκησε Τύρον ἔτεσι τρισὶ καὶ δέκα, 
βασιλεύοντος κατ' ἐκεῖνον τὸν καιρὸν Ἰθωβάλου τῆς 
Τύρου. 



Megasthenes Hist., Fragmenta 
Fragment 22b, line 3

Syncellus: Τὸν 
Ναβουχοδονόσωρ ὁ Μεγασθένης ἐν τῇ δʹ (l. δευτέρᾳ) 
τῶν Ἰνδικῶν Ἡρακλέους ἀλκιμώτερον ἀποφαίνει, ὃς 
ἀνδρείᾳ μεγάλῃ Λιβύης τὸ πλεῖστον καὶ Ἰβηρίας κα-
τεστρέψατο. 



Megasthenes Hist., Fragmenta 
Fragment 23, line 1

Arrianus Ind. c. 7: Ἔθνεα δὲ Ἰνδικὰ εἴκοσι καὶ 
ἑκατόν τι ἅπαντα λέγει <Μεγασθένης> δυοῖν δέοντα. 



Megasthenes Hist., Fragmenta 
Fragment 23, line 3

[Καὶ πολλὰ μὲν εἶναι ἔθνεα Ἰνδικὰ καὶ αὐτὸς ξυμφέ-
ρομαι Μεγασθένεϊ· τὸ δὲ ἀτρεκὲς οὐκ ἔχω εἰκάσαι ὅπως 
ἐκμαθὼν ἀνέγραψεν, οὐδὲ πολλοστὸν μέρος τῆς Ἰνδῶν 
γῆς ἐπελθὼν, οὐδὲ ἐπιμιξίης πᾶσι τοῖς γένεσιν ἐούσης 
ἐς ἀλλήλους. 



Megasthenes Hist., Fragmenta 
Fragment 23, line 8

Πάλαι μὲν δὴ νομάδας εἶναι Ἰνδοὺς κατάπερ Σκυ-
θέων τοὺς οὐκ ἀροτῆρας, οἳ ἐπὶ τῇσιν ἁμάξῃσι πλανώ-
μενοι ἄλλοτε ἄλλην τῆς Σκυθίης ἀμείβουσιν, οὔτε πό-
λιας οἰκέοντες οὔτε ἱερὰ θεῶν σέβοντες· οὕτω μηδὲ 
Ἰνδοῖσι πόλιας εἶναι μηδὲ ἱερὰ θεῶν δεδομημένα· ἀλλ' 
ἀμπέχεσθαι μὲν δορὰς θηρίων ὅσων κατακτάνοιεν· σι-  
τέεσθαι δὲ τῶν δενδρέων τὸν φλοιόν· καλέεσθαι δὲ τὰ 
δένδρεα ταῦτα τῇ Ἰνδῶν φωνῇ <Τάλα·> καὶ φύεσθαι 
ἐπ' αὐτῶν κατάπερ τῶν φοινίκων ἐπὶ τῇσι κορυφῇσιν 
οἷά περ τολύπας. 



Megasthenes Hist., Fragmenta 
Fragment 23, line 19

                     Σιτέεσθαι δὲ καὶ τῶν θηρίων ὅσα 
ἕλοιεν ὠμοφαγέοντας, πρὶν δὴ Διόνυσον ἐλθεῖν ἐς τὴν 
χώρην τῶν Ἰνδῶν. 



Megasthenes Hist., Fragmenta 
Fragment 23, line 20

                      Διόνυσον δὲ ἐλθόντα, ὡς καρτερὸς 
ἐγένετο Ἰνδῶν, πόλιάς τε οἰκῆσαι καὶ νόμους θέσθαι 
τῇσι πόλισιν, οἴνου τε δοτῆρα Ἰνδοῖς γενέσθαι κατάπερ 
Ἕλλησι, καὶ σπείρειν διδάξαι τὴν γῆν διδόντα αὐτὸν 
σπέρματα· ἢ οὐκ ἐλάσαντος ταύτῃ Τριπτολέμου, ὅτε 
περ ἐκ Δήμητρος ἐστάλη σπείρειν τὴν γῆν πᾶσαν, 
ἢ πρὸ Τριπτολέμου τις οὗτος Διόνυσος ἐπελθὼν τὴν 
Ἰνδῶν γῆν σπέρματά σφισιν ἔδωκε καρποῦ τοῦ ἡμέ-
ρου· βόας τε ὑπ' ἀρότρῳ ζεῦξαι Διόνυσον πρῶτον, καὶ 
ἀροτῆρας ἀντὶ νομάδων ποιῆσαι Ἰνδῶν τοὺς πολλοὺς 
καὶ ὁπλίσαι ὅπλοισι τοῖσιν ἀρηΐοισι. 



Megasthenes Hist., Fragmenta 
Fragment 23, line 33

                                              Καὶ θεοὺς σέβειν 
ὅτι ἐδίδαξε Διόνυσος ἄλλους τε καὶ μάλιστα δὴ ἑωυτὸν 
κυμβαλίζοντας καὶ τυμπανίζοντας· καὶ ὄρχησιν δὲ ἐκ-
διδάξαι τὴν σατυρικὴν, τὸν κόρδακα παρ' Ἕλλησι κα-
λεόμενον· καὶ κομᾶν Ἰνδοὺς τῷ θεῷ, μιτρηφορέειν τε 
ἀναδεῖξαι καὶ μύρων ἀλοιφὰς ἐκδιδάξαι, ὥστε καὶ εἰς 
Ἀλέξανδρον ἔτι ὑπὸ κυμβάλων τε καὶ τυμπάνων ἐς τὰς 
μάχας Ἰνδοὶ καθίσταντο. 



Megasthenes Hist., Fragmenta 
Fragment 23, line 37

C. 8. Ἀπιόντα δὲ ἐκ τῆς Ἰνδῶν γῆς, ὥς οἱ ταῦτα κε-
κοσμέατο, καταστῆσαι βασιλέα τῆς χώρης Σπατέμβαν, 
τῶν ἑταίρων ἕνα τὸν βακχωδέστατον· τελευτήσαντος δὲ 
Σπατέμβα τὴν βασιληίην ἐκδέξασθαι Βουδύαν τὸν 
τούτου παῖδα· καὶ τὸν μὲν πεντήκοντα καὶ δύο ἔτεα 
βασιλεῦσαι Ἰνδῶν, τὸν πατέρα· τὸν δὲ παῖδα εἴκοσιν 
ἔτεα· καὶ τούτου παῖδα ἐκδέξασθαι τὴν βασιληίην Κρα-
δεύαν· καὶ τὸ ἀπὸ τοῦδε τὸ πολὺ μὲν κατὰ γένος ἀμεί-
βειν τὴν βασιληίην, παῖδα παρὰ πατρὸς ἐκδεκόμενον· 
εἰ δὲ ἐκλείποι τὸ γένος, οὕτω δὴ ἀριστίνδην

καθίστα-



Megasthenes Hist., Fragmenta 
Fragment 23, line 47

κοσμέατο, καταστῆσαι βασιλέα τῆς χώρης Σπατέμβαν, 
τῶν ἑταίρων ἕνα τὸν βακχωδέστατον· τελευτήσαντος δὲ 
Σπατέμβα τὴν βασιληίην ἐκδέξασθαι Βουδύαν τὸν 
τούτου παῖδα· καὶ τὸν μὲν πεντήκοντα καὶ δύο ἔτεα 
βασιλεῦσαι Ἰνδῶν, τὸν πατέρα· τὸν δὲ παῖδα εἴκοσιν 
ἔτεα· καὶ τούτου παῖδα ἐκδέξασθαι τὴν βασιληίην Κρα-
δεύαν· καὶ τὸ ἀπὸ τοῦδε τὸ πολὺ μὲν κατὰ γένος ἀμεί-
βειν τὴν βασιληίην, παῖδα παρὰ πατρὸς ἐκδεκόμενον· 
εἰ δὲ ἐκλείποι τὸ γένος, οὕτω δὴ ἀριστίνδην καθίστα-
σθαι Ἰνδοῖσι βασιλέας. 



Megasthenes Hist., Fragmenta 
Fragment 23, line 48

Ἡρακλέα δὲ, ὅντινα ἐς Ἰνδοὺς ἀφικέσθαι λόγος κα-
τέχει, παρ' αὐτοῖσιν Ἰνδοῖσι γηγενέα λέγεσθαι. 



Megasthenes Hist., Fragmenta 
Fragment 23, line 51

                                                       Τοῦ-
τον τὸν Ἡρακλέα μάλιστα πρὸς Σουρασηνῶν γεραίρε-
σθαι Ἰνδικοῦ ἔθνεος, ἵνα δύο πόλιες μεγάλαι Μέθορά 
τε καὶ Κλεισόβορα, καὶ ποταμὸς Ἰωβάρης πλωτὸς διαρ-
ρέει τὴν χώρην αὐτῶν. 



Megasthenes Hist., Fragmenta 
Fragment 23, line 55

                           Τὴν σκευὴν δὲ οὗτος ὁ Ἡρα-
κλέης ἥντινα ἐφόρεε, <Μεγασθένης> λέγει ὅτι ὁμοίην 
τῷ Θηβαίῳ Ἡρακλέϊ, ὡς αὐτοὶ Ἰνδοὶ ἀπηγέονται· καὶ 
τούτῳ ἄρσενας μὲν παῖδας πολλοὺς κάρτα γενέσθαι ἐν 
τῇ Ἰνδῶν γῇ, (πολλῇσι γὰρ δὴ γυναιξὶν ἐς γάμον ἐλ-
θεῖν καὶ τοῦτον τὸν Ἡρακλέα,) θυγατέρα δὲ μουνο-
γενέην· οὔνομα δὲ εἶναι τῇ παιδὶ Πανδαίην, καὶ τὴν 
χώρην ἵνα τε ἐγένετο καὶ ἧστινος ἐπέτρεψεν αὐτὴν ἄρ-
χειν Ἡρακλέης, Πανδαίην, τῆς παιδὸς ἐπώνυμον· καὶ 
ταύτῃ ἐλέφαντας μὲν γενέσθαι ἐκ τοῦ πατρὸς ἐς πεν-
τακοσίους, ἵππον δὲ ἐς τετρακισχιλίην, πεζῶν δὲ ἐς τὰς 
τρεῖς καὶ δέκα μυριάδας. 



Megasthenes Hist., Fragmenta 
Fragment 23, line 64

                             Καὶ τάδε μετεξέτεροι Ἰνδῶν 
περὶ Ἡρακλέος λέγουσιν· ἐπελθόντα αὐτὸν πᾶσαν γῆν 
καὶ θάλασσαν καὶ καθήραντα ὅ τι περ κακὸν κίναδος, 
ἐξευρεῖν ἐν τῇ θαλάσσῃ κόσμον γυναικήϊον· [ὅντινα καὶ 
εἰς τοῦτο ἔτι οἵ τε ἐξ Ἰνδῶν τῆς χώρης τὰ ἀγώγιμα 
παρ' ἡμέας ἀγινέοντες σπουδῇ ὠνεόμενοι ἐκκομίζουσι· 
καὶ Ἑλλήνων δὲ πάλαι καὶ Ῥωμαίων νῦν ὅσοι πολυ-
κτέανοι καὶ εὐδαίμονες, μέζονι ἔτι σπουδῇ ὠνέονται·]   
τὸν μαργαρίτην δὴ τὸν θαλάσσιον, οὕτω τῇ Ἰνδῶν 
γλώσσῃ καλεόμενον· τὸν γὰρ Ἡρακλέα, ὡς καλόν οἱ 




Megasthenes Hist., Fragmenta 
Fragment 23, line 75

περὶ Ἡρακλέος λέγουσιν· ἐπελθόντα αὐτὸν πᾶσαν γῆν 
καὶ θάλασσαν καὶ καθήραντα ὅ τι περ κακὸν κίναδος, 
ἐξευρεῖν ἐν τῇ θαλάσσῃ κόσμον γυναικήϊον· [ὅντινα καὶ 
εἰς τοῦτο ἔτι οἵ τε ἐξ Ἰνδῶν τῆς χώρης τὰ ἀγώγιμα 
παρ' ἡμέας ἀγινέοντες σπουδῇ ὠνεόμενοι ἐκκομίζουσι· 
καὶ Ἑλλήνων δὲ πάλαι καὶ Ῥωμαίων νῦν ὅσοι πολυ-
κτέανοι καὶ εὐδαίμονες, μέζονι ἔτι σπουδῇ ὠνέονται·]   
τὸν μαργαρίτην δὴ τὸν θαλάσσιον, οὕτω τῇ Ἰνδῶν 
γλώσσῃ καλεόμενον· τὸν γὰρ Ἡρακλέα, ὡς καλόν οἱ 
ἐφάνη τὸ φόρημα, ἐκ πάσης τῆς θαλάσσης ἐς τὴν Ἰν-
δῶν γῆν συναγινέειν τὸν μαργαρίτην δὴ τοῦτον, τῇ θυ-
γατρὶ τῇ ἑωυτοῦ εἶναι κόσμον. 



Megasthenes Hist., Fragmenta 
Fragment 23, line 87

                                   Καὶ εἶναι γὰρ καὶ παρ' 
Ἰνδοῖσι τὸν μαργαρίτην τριστάσιον κατὰ τιμὴν πρὸς 
χρυσίον τὸ ἄπεφθον, καὶ τοῦτο ἐν τῇ Ἰνδῶν γῇ ὀρυς-
σόμενον. 



Megasthenes Hist., Fragmenta 
Fragment 23, line 94

                                    Καὶ ὑπὲρ τούτου λεγό-
μενον λόγον εἶναι παρ' Ἰνδοῖσιν· Ἡρακλέα, ὀψιγόνου 
οἱ γενομένης τῆς παιδὸς, ἐπεί τε δὴ ἐγγὺς ἔμαθεν ἑωυτῷ 
ἐοῦσαν τὴν τελευτὴν, οὐκ ἔχοντα ὅτεῳ ἀνδρὶ ἐκδῷ τὴν 
παῖδα ἑωυτοῦ ἐπαξίῳ, αὐτὸν μιγῆναι τῇ παιδὶ ἑπταέτεϊ 
ἐούσῃ, ὡς γένος ἐξ οὗ τε κἀκείνης ὑπολείπεσθαι Ἰνδῶν 
βασιλέας. 



Megasthenes Hist., Fragmenta 
Fragment 23, line 119

Ἀπὸ μὲν δὴ Διονύσου βασιλέας ἠρίθμεον Ἰνδοὶ ἐς 
Σανδράκοττον τρεῖς καὶ πεντήκοντα καὶ ἑκατόν· ἔτεα 
δὲ δύο καὶ τεσσαράκοντα καὶ ἑξακισχίλια· ἐν δὲ τού-
τοισι τρὶς τὸ πᾶν εἰς ἐλευθερίην ** τὴν δὲ καὶ ἐς τριη-
κόσια· τὴν δὲ εἴκοσί τε ἐτέων καὶ ἑκατόν (*)· πρεσβύ-  
τερόν τε Διόνυσον Ἡρακλέος δέκα καὶ πέντε γενεῇ-
σιν Ἰνδοὶ λέγουσιν· ἄλλον δὲ οὐδένα ἐμβαλεῖν ἐς γῆν 
τὴν Ἰνδῶν ἐπὶ πολέμῳ, οὐδὲ Κῦρον τὸν Καμβύσεω, καί-
τοι ἐπὶ Σκύθας ἐλάσαντα καὶ τἄλλα πολυπραγμονέστα-
τον δὴ τῶν κατὰ τὴν Ἀσίην βασιλέων γενόμενον τὸν 




Megasthenes Hist., Fragmenta 
Fragment 23, line 132

                                             Οὐ μὲν δὴ 
οὐδὲ Ἰνδῶν τινα ἔξω τῆς οἰκηίης σταλῆναι ἐπὶ πολέμῳ 
διὰ δικαιότητα. 



Megasthenes Hist., Fragmenta 
Fragment 26, line 2

Arrian. Ind. c. 10: Λέγεται δὲ καὶ τάδε, μνημήια 
ὅτι Ἰνδοὶ τοῖς τελευτήσασιν οὐ ποιέουσιν, ἀλλὰ τὰς ἀρε-
τὰς γὰρ τῶν ἀνδρῶν ἱκανὰς ἐς μνήμην τίθενται τοῖσιν 
ἀποθανοῦσι, καὶ τὰς ᾠδὰς αἳ αὐτοῖσιν ἐπᾴδονται. 



Megasthenes Hist., Fragmenta 
Fragment 26, line 6

                                                              (2) Πο-
λίων δὲ ἀριθμὸν οὐκ εἶναι ἂν ἀτρεκὲς ἀναγράψαι τῶν 
Ἰνδικῶν ὑπὸ πλήθεος· ἀλλὰ γὰρ ὅσαι παραποτάμιαι αὐ-
τέων ἢ παραθαλάσσιαι, ταύτας μὲν ξυλίνας ποιέεσθαι· 
(3) οὐ γὰρ εἶναι ἐκ πλίνθου ποιεομένας διαρκέσαι ἐπὶ 
χρόνον τοῦ τε ὕδατος ἕνεκα τοῦ ἐξ οὐρανοῦ, καὶ ὅτι οἱ 
ποταμοὶ αὐτοῖσιν ὑπερβάλλοντες ὑπὲρ τὰς ὄχθας ἐμπι-
πλᾶσι τοῦ ὕδατος τὰ πεδία. 



Megasthenes Hist., Fragmenta 
Fragment 26, line 14

                                 (4) Ὅσαι δὲ ἐν ὑπερδε-
ξίοισί τε καὶ μετεώροισι τόποισι καὶ τούτοισιν ὑψηλοῖσιν, 
ᾠκισμέναι εἰσὶ, ταύτας δὲ ἐκ πλίνθου τε καὶ πηλοῦ 
ποιέεσθαι· (5) μεγίστην δὲ πόλιν ἐν Ἰνδοῖσιν εἶναι Παλίμ-
βοθρα καλεομένην, ἐν τῇ Πρασίων γῇ, ἵνα αἱ συμβολαί 
εἰσι τοῦ τε Ἐραννοβόα ποταμοῦ καὶ τοῦ Γάγγεως· τοῦ 
μὲν Γάγγεω, τοῦ μεγίστου ποταμῶν· ὁ δὲ Ἐραννοβόας 
τρίτος μὲν ἂν εἴη τῶν Ἰνδῶν ποταμῶν, μέζων δὲ τῶν 
ἄλλῃ καὶ οὗτος· ἀλλὰ ξυγχωρέει αὐτὸς τῷ Γάγγῃ, ἐπει-
δὰν ἐμβάλλῃ ἐς αὐτὸν τὸ ὕδωρ. 



Megasthenes Hist., Fragmenta 
Fragment 26, line 28

                           (8) Εἶναι δὲ καὶ τόδε μέγα 
ἐν τῇ Ἰνδῶν γῇ, πάντας Ἰνδοὺς εἶναι ἐλευθέρους, οὐδέ 
τινα δοῦλον εἶναι Ἰνδόν· τοῦτο μὲν Λακεδαιμονίοισιν 
ἐς ταὐτὸ συμβαίνει καὶ Ἰνδοῖσι. 



Megasthenes Hist., Fragmenta 
Fragment 26, line 32

                                        (9) Λακεδαιμονίοισι 
μέν γε οἱ εἵλωτες δοῦλοί εἰσι καὶ τὰ δούλων ἐργάζονται· 
Ἰνδοῖσι δὲ οὐδὲ ἄλλος δοῦλός ἐστι μήτι γε Ἰνδῶν τις. 



Megasthenes Hist., Fragmenta 
Fragment 27, line 2

Strabo XV: Εὐτελεῖς δὲ κατὰ τὴν 
δίαιταν οἱ Ἰνδοὶ πάντες, μᾶλλον δ' ἐν ταῖς στρατείαις· 
οὐδ' ὄχλῳ περιττῷ χαίρουσι· διόπερ εὐκοσμοῦσι. 



Megasthenes Hist., Fragmenta 
Fragment 27, line 46

                         Δούλοις δὲ οὗτος μέν φησι 
μηδένα Ἰνδῶν χρῆσθαι· [Ὀνησίκριτος δὲ τῶν 
ἐν τῇ Μουσικανοῦ τοῦτ' ἴδιον ἀποφαίνει κτλ. 



Megasthenes Hist., Fragmenta 
Fragment 28, line 2

Athenaeus IV: <Μεγασθένης ἐν τῇ 
δευτέρᾳ τῶν Ἰνδικῶν> τοῖς Ἰνδοῖς φησιν ἐν τῷ 
δείπνῳ παρατίθεσθαι ἑκάστῳ τράπεζαν, ταύτην δ' εἶ-
ναι ὁμοίαν ταῖς ἐγγυθήκαις· καὶ ἐπιτίθεσθαι ἐπ' αὐτῇ 
τρυβλίον χρυσοῦν, εἰς ὃ ἐμβαλεῖν αὐτοὺς πρῶτον μὲν 
τὴν ὄρυζαν ἑφθὴν, ὡς ἄν τις ἑψήσειε χόνδρον, ἔπειτα 
ὄψα πολλὰ κεχειρουργημένα ταῖς Ἰνδικαῖς σκευασίαις. 



Megasthenes Hist., Fragmenta 
Fragment 29, line 2

Strabo II: Ἅπαντες μὲν τοίνυν οἱ περὶ τῆς 
Ἰνδικῆς γράψαντες ὡς ἐπὶ τὸ πολὺ ψευδολόγοι γεγό-
νασι, καθ' ὑπερβολὴν δὲ Δηίμαχος· τὰ δὲ δεύτερα λέγει 
Μεγασθένης· Ὀνησίκριτος δὲ καὶ Νέαρχος καὶ ἄλλοι 
τοιοῦτοι παραψελλίζοντες ἤδη. 



Megasthenes Hist., Fragmenta 
Fragment 34, line 2

praeterea locustis eos ali et esse pernices.] Man-
dorum nomen iis dedit Cli-
tarchus et <Megasthenes>, trecentosque eorum vi-
cos annumerat. Feminas septimo aetatis anno 
parere, senectam quadragesimo accidere. 
 Plutarch. De fac. in luna c. 24: 
Τὴν μὲν γὰρ Ἰνδικὴν ῥίζαν, ἥν φησι <Μεγα-
σθένης> μήτ' ἐσθίοντας μήτε πίνοντας ἀλλ' ἀστόμους 
ὄντας ὑποτύφειν καὶ θυμιᾶν καὶ τρέφεσθαι τῇ ὀσμῇ, πό-
θεν ἄν τις ἐκεῖ φυομένην λάβοι μὴ βρεχομένης τῆς 
σελήνης; 



Megasthenes Hist., Fragmenta 
Fragment 35, line 2

E LIBRO TERTIO.


 Arrian. Ind. c. 11: Νενέμηνται δὲ οἱ πάντες 
Ἰνδοὶ ἐς ἑπτὰ μάλιστα γενεάς· ἐν μὲν αὐτοῖσιν οἱ σο-
φισταί εἰσι, πλήθεϊ μὲν μείους τῶν ἄλλων, δόξῃ δὲ καὶ 
τιμῇ γεραρώτατοι. 



Megasthenes Hist., Fragmenta 
Fragment 35, line 8

                     (2) Οὔτε γάρ τι τῷ σώματι ἐργά-
ζεσθαι ἀναγκαίη σφὶν προσκέεται, οὔτε τι ἀποφέρειν 
ἀπ' ὅτου πονέουσιν ἐς τὸ κοινόν· οὐδέ τι ἄλλο ἀνάγκης 
ἁπλῶς ἐπεῖναι τοῖσι σοφιστῇσιν, ὅτι μὴ θύειν τὰς θυσίας 
τοῖσι θεοῖσιν ὑπὲρ τοῦ κοινοῦ τῶν Ἰνδῶν· (3) καὶ 
δὲ ἰδίᾳ θύει, ἐξηγητὴς αὐτῷ τῆς θυσίης τῶν τις σοφιστέων 
τούτων γίνεται, ὡς οὐκ ἂν ἄλλως κεχαρισμένα τοῖς 
θεοῖσι θύσαντας. 



Megasthenes Hist., Fragmenta 
Fragment 35, line 12

                   (4) Εἰσὶ δὲ καὶ μαντικῆς οὗτοι μοῦνοι 
Ἰνδῶν δαήμονες, οὐδὲ ἐφεῖται ἄλλῳ μαντεύεσθαι ὅτι μὴ 
σοφῷ ἀνδρί. 



Megasthenes Hist., Fragmenta 
Fragment 35, line 32

9. <Δεύτεροι> δ' ἐπὶ τούτοισιν οἱ γεωργοί εἰσιν· 
πλήθεϊ πλεῖστοι Ἰνδῶν ἐόντες· καὶ τούτοισιν οὔτε ὅπλα 
ἐστὶν ἀρήια οὔτε μέλει τὰ πολέμια ἔργα, ἀλλὰ τὴν 
χώρην οὗτοι ἐργάζονται· καὶ τοὺς φόρους τοῖσι τε βασι-
λεῦσι καὶ τῇσι πόλισιν, ὅσαι αὐτόνομοι, οὗτοι ἀποφέ-
ρουσι· (10) καὶ εἰ πόλεμος ἐς ἀλλήλους τοῖσιν Ἰνδοῖσι 
τύχοι, τῶν ἐργαζομένων τὴν γῆν οὐ θέμις σφὶν ἅπτεσθαι, 
οὐδὲ αὐτὴν τὴν γῆν τάμνειν· ἀλλὰ οἱ μὲν πολεμέουσι καὶ 
κατακαίνουσιν ἀλλήλους ὅπως τύχοιεν, οἱ δὲ πλησίον 
αὐτῶν κατ' ἡσυχίην ἀροῦσιν ἢ τρυγῶσιν ἢ κλαδοῦσιν 
ἢ θερίζουσιν. 



Megasthenes Hist., Fragmenta 
Fragment 35, line 42

11. <Τρίτοι> δέ εἰσιν Ἰνδοῖσιν οἱ νομέες, οἱ 
τε καὶ βουκόλοι, καὶ οὗτοι οὔτε κατὰ πόλιας οὔτε ἐν 
τῇσι κώμῃσιν οἰκέουσι. 



Megasthenes Hist., Fragmenta 
Fragment 35, line 55

2. <Πέμπτον> δὲ γένος ἐστὶν Ἰνδοῖσιν οἱ πολεμισταὶ, 
πλήθεϊ μὲν δεύτερον μετὰ τοὺς γεωργοὺς, πλείστῃ δὲ 
ἐλευθερίῃ τε καὶ εὐθυμίῃ ἐπιχρεόμενον· καὶ οὗτοι ἀσκη-
ταὶ μούνων τῶν πολεμικῶν ἔργων εἰσί. 



Megasthenes Hist., Fragmenta 
Fragment 35, line 67

5. <Ἕκτοι> δέ εἰσιν Ἰνδοῖσιν οἱ ἐπίσκοποι καλεόμε-
νοι. 



Megasthenes Hist., Fragmenta 
Fragment 35, line 70

     Οὗτοι ἐφορῶσι τὰ γινόμενα κατά τε τὴν χώρην καὶ 
κατὰ τὰς πόλιας· καὶ ταῦτα ἀναγγέλλουσι τῷ βασιλέϊ, 
ἵναπερ βασιλεύονται Ἰνδοὶ, ἢ τοῖσι τέλεσιν, ἵναπερ αὐ-
τόνομοι εἰσί· καὶ τούτοισιν οὐ θέμις ψεῦδος ἀγγεῖλαι οὐ-
δέν· οὐδέ τις Ἰνδῶν αἰτίην ἔσχε ψεύσασθαι. 



Megasthenes Hist., Fragmenta 
Fragment 36, line 2

Strabo XV: (1) <Φησὶ> δὴ τὸ τῶν 
Ἰνδῶν πλῆθος εἰς ἑπτὰ μέρη διῃρῆσθαι, καὶ πρώτους 
μὲν τοὺς φιλοσόφους εἶναι κατὰ τιμὴν, ἐλαχίστους δὲ 
κατ' ἀριθμόν· χρῆσθαι δ' αὐτοῖς ἰδίᾳ μὲν ἑκάστῳ τοὺς 
θύοντας ἢ τοὺς ἐναγίζοντας, κοινῇ δὲ τοὺς βασιλέας 
κατὰ τὴν μεγάλην λεγομένην σύνοδον, καθ' ἣν τοῦ 
νέου ἔτους ἅπαντες οἱ φιλόσοφοι τῷ βασιλεῖ συνελθόντες 
ἐπὶ θύρας, ὅ τι ἂν ἕκαστος αὐτῶν συντάξῃ τῶν χρη-
σίμων, ἢ τηρήσῃ πρὸς εὐετηρίαν καρπῶν τε καὶ ζώων 
καὶ περὶ πολιτείας, προφέρει τοῦτο εἰς τὸ μέσον· ὃς 
δ' ἂν τρὶς ἐψευσμένος ἁλῷ, νόμος ἐστὶ σιγᾶν διὰ 




Megasthenes Hist., Fragmenta 
Fragment 36b, line 2

Aelianus XIII, 9: 
 Ἵππον δὲ ἄρα Ἰνδὸν 
κατασχεῖν καὶ ἀνακροῦσαι προπηδῶντα καὶ ἐκθέοντα 
οὐ παντὸς ἦν, ἀλλὰ τῶν ἐκ παιδὸς ἱππείαν πεπαιδευ-
μένων. 



Megasthenes Hist., Fragmenta 
Fragment 38a, line 2

Arrianus Ind. c. 13: Θηρῶσι 
δὲ Ἰνδοὶ τὰ μὲν ἄλλα ἄγρια θηρία, κατάπερ καὶ Ἕλ-
ληνες· ἡ δὲ τῶν ἐλεφάντων σφὶν θήρη οὐδέν τι ἄλλῃ 
ἔοικεν, ὅτι καὶ ταῦτα τὰ θηρία οὐδαμοῖσιν ἄλλοισι θη-
ρίοισιν ἐπέοικεν. 



Megasthenes Hist., Fragmenta 
Fragment 38a, line 66

                     (2) Ἄγοντες δὲ εἰς τὰς κώμας τοὺς 
ἁλόντας τοῦ τε χλωροῦ καλάμου καὶ τῆς ποίης τὰ πρῶτα 
ἐμφαγεῖν ἔδοσαν· (3) οἱ δὲ ὑπὸ ἀθυμίης οὐκ ἐθέλουσιν 
οὐδὲν σιτέεσθαι, τοὺς δὲ περιϊστάμενοι οἱ Ἰνδοὶ ᾠδαῖσί 
τε καὶ τυμπάνοισι καὶ κυμβάλοισιν ἐν κύκλῳ κρούοντές 
τε καὶ ἐπᾴδοντες κατευνάζουσι. 



Megasthenes Hist., Fragmenta 
Fragment 38a, line 94

                     Ταῦτα παρ' Ἰνδοῖσίν ἐστιν αὐτοῖσιν 
ἰήματα. 



Megasthenes Hist., Fragmenta 
Fragment 38b, line 1

Aelianus N. A. XII, 44: Ἐν Ἰνδοῖς ἂν ἁλῷ τέλειος 
ἐλέφας, ἡμερωθῆναι χαλεπός ἐστι, καὶ τὴν ἐλευθερίαν 
ποθῶν φονᾷ· ἐὰν δὲ αὐτὸν καὶ δεσμοῖς διαλάβῃς, ἔτι καὶ 
μᾶλλον ἐς τὸν θυμὸν ἐξάπτεται, καὶ δεσπότην οὐχ ὑπο-
νέμει. 



Megasthenes Hist., Fragmenta 
Fragment 38b, line 5

        Ἀλλ' οἱ Ἰνδοὶ καὶ ταῖς τροφαῖς κολακεύουσιν αὐ-
τὸν, καὶ ποικίλοις καὶ ἐφολκοῖς δελέασι πραΰνειν πει-
ρῶνται, παρατιθέντες, ὡς πληροῦν τὴν γαστέρα καὶ 
θέλγειν τὸν θυμόν· ὁ δὲ ἄχθεται αὐτοῖς, καὶ ὑπερορᾷ· 
Τί οὖν ἐκεῖνοι κατασοφίζονται καὶ δρῶσι; 



Megasthenes Hist., Fragmenta 
Fragment 38c, line 2

Idem XIII, 6: Τῶν τεθηραμένων ἐλεφάντων ἰῶν-
ται τὰ τραύματα οἱ Ἰνδοὶ τὸν τρόπον τοῦτον. 



Megasthenes Hist., Fragmenta 
Fragment 39a, line 4

Strabo XV: 
Μεγασθένης δὲ περὶ τῶν μυρμήκων οὕτω 
φησὶν, ὅτι ἐν Δέρδαις, ἔθνει μεγάλῳ τῶν προσεῴων καὶ 
ὀρεινῶν Ἰνδῶν, ὀροπέδιον εἴη τρισχιλίων πως τὸν κύ-
κλον σταδίων· ὑποκειμένων δὲ τούτῳ χρυσωρυχείων, οἱ 
μεταλλεύοντες εἶεν μύρμηκες, θηρίων ἀλωπέκων οὐκ 
ἐλάττους, τάχος ὑπερφυὲς ἔχοντες, καὶ ζῶντες ἀπὸ θή-
ρας. 



Megasthenes Hist., Fragmenta 
Fragment 39b, line 9

Arrianus Ind. c. 5, 4: Μεγασθένης δὲ καὶ ἀτρεκέας 
εἶναι ὑπὲρ τῶν μυρμήκων τὸν λόγον ἱστορέει, τούτους 
εἶναι τοὺς τὸν χρυσὸν ὀρύσσοντας, οὐκ αὐτοῦ τοῦ χρυ-
σοῦ ἕνεκα, ἀλλὰ φύσι γὰρ κατὰ τῆς γῆς ὀρύσσουσιν, ἵνα 
φωλεύοιεν· κατάπερ οἱ ἡμέτεροι οἱ σμικροὶ μύρμηκες ὀλί-
γον τῆς γῆς ὀρύσσουσιν· (6) ἐκείνους δὲ, εἶναι γὰρ 
πέκων μέζονας, πρὸς λόγον τοῦ μεγέθεος σφῶν καὶ τὴν 
γῆν ὀρύσσειν· τὴν δὲ γῆν χρυσῖτιν εἶναι, καὶ ἀπὸ ταύ-
της γίνεσθαι Ἰνδοῖσι τὸν χρυσόν. 



Megasthenes Hist., Fragmenta 
Fragment 41a, line 3

Clem. Alex. Strom. I: 
Μεγασθένης ὁ συγγραφεὺς ὁ Σελεύκῳ τῷ Νικάτορι 
συμβεβιωκὼς ἐν τῇ τρίτῃ τῶν Ἰνδικῶν ὧδε γράφει· 
»Ἅπαντα μέντοι τὰ περὶ φύσεως εἰρημένα παρὰ τοῖς 
ἀρχαίοις λέγεται καὶ παρὰ τοῖς ἔξω τῆς Ἑλλάδος φιλο-
σοφοῦσι, τὰ μὲν παρ' Ἰνδοῖς ὑπὸ τῶν Βραχμάνων, τὰ 
δὲ ἐν τῇ Συρίᾳ ὑπὸ τῶν καλουμένων Ἰουδαίων. 



Megasthenes Hist., Fragmenta 
Fragment 41b, line 8

Clemens l. l.: Διττὸν δὲ τούτων 
τὸ γένος· οἱ μὲν Σαρμᾶναι αὐτῶν, οἱ δὲ Βραχμᾶναι 
καλούμενοι· καὶ τῶν Σαρμανῶν οἱ Ὑλόβιοι προσαγο-
ρευόμενοι οὔτε πόλεις οἰκοῦσιν οὔτε στέγας ἔχουσιν, 
δένδρων δὲ ἀμφιέννυνται φλοιοῖς, καὶ ἀκρόδρυα σιτοῦν-
ται καὶ ὕδωρ ταῖς χερσὶ πίνουσιν· οὐ γάμον, οὐ παιδο-
ποιίαν ἴσασιν, [ὥσπερ οἱ νῦν Ἐγκρατηταὶ καλούμενοι, 
εἰσὶ δὲ τῶν Ἰνδῶν οἱ τοῖς Βούττα πειθόμενοι παραγ-
γέλμασιν, ὃν δι' ὑπερβολὴν σεμνότητος ὡς θεὸν τετιμή-
κασι. 



Megasthenes Hist., Fragmenta 
Fragment 42, line 16

καὶ ὁ Κάλανος, ἀκόλαστος ἄνθρωπος, καὶ ταῖς Ἀλε-
ξάνδρου τραπέζαις δεδουλωμένος· τοῦτον μὲν οὖν ψέ-
γεσθαι, τὸν δὲ Μάνδανιν ἐπαινεῖσθαι, ὃς τῶν τοῦ Ἀλε-
ξάνδρου ἀγγέλων καλούντων πρὸς τὸν Διὸς υἱὸν, 
πειθομένῳ τε δῶρα ἔσεσθαι ὑπισχνουμένων, ἀπει-
θοῦντι δὲ κόλασιν· μήτε ἐκεῖνον φαίη Διὸς υἱὸν, ὅς γε 
ἄρχει μηδὲ πολλοστοῦ μέρους τῆς γῆς· μηδὲ αὐτῷ 
δεῖν τῶν παρ' ἐκείνου δωρεῶν, ᾧ οὐδεὶς κόρος· μήτε 
δὲ ἀπειλῆς εἶναι φόβον, ᾧ ζῶντι μὲν ἀρκοῦσα εἴη τρο-
φὸς ἡ Ἰνδικὴ, ἀποθανόντι δὲ ἀπαλλάξαιτο τῆς σαρκὸς 
ἀπὸ γήρως τετρυχωμένης, μεταστὰς εἰς βελτίω καὶ 
καθαρώτερον βίον· ὥστ' ἐπαινέσαι τὸν Ἀλέξανδρον καὶ 
συγχωρῆσαι. 



\end{greek}



\chapter{Hellenistic Greek sources}%hell
Up to conquest of the eastern Mediterranean by Rome. 
\minitoc

\section{Historia Alexandri Magni}
\blockquote[From Wikipedia\footnote{\url{}}]{}
\begin{greek}

Historia Alexandri Magni, Recensio α sive Recensio vetusta (1386: 001)
“Historia Alexandri Magni, vol. 1”, Ed. Kroll, W.
Berlin: Weidmann, 1926.
Book 2, chapter 7, section 9, line 3

                                                                     ὥστε καὶ 
σὺ μετάπεμψαι τῆς ὅλης ἠπείρου τοὺς σατράπας· ἔστι γάρ σοι ἔθνη Περσῶν 
καὶ Πάρθων καὶ Ἐλυμαίων καὶ Βαβυλωνίων καὶ τῶν κατὰ τὴν Μεσοποτα-
μίαν καὶ τὴν Ἰλλυρίαν χώραν, ἵνα μή σοι τὰ Βάκτρων καὶ τὰ Ἰνδῶν ἢ 
τῶν Σεμιράμεως μελάθρων εἴπω. 



Historia Alexandri Magni, Recensio α sive Recensio vetusta 
Book 2, chapter 12, section 1, line 2

Γράφει δὲ καὶ τῷ Πώρῳ δεόμενος βοηθείας· λαβὼν δὲ ὁ Πῶρος 
γράφει· ‘Πῶρος βασιλεὺς Ἰνδῶν Δαρείῳ βασιλεῖ Περσῶν χαίρειν. 



Historia Alexandri Magni, Recensio α sive Recensio vetusta 
Book 2, chapter 19, section 1, line 3

Ἤδη δὲ πάλιν ὁ Δαρεῖος ἐστέλλετο πρὸς ἑτέραν συμβολήν· καὶ γράφει 
Πώρῳ ἐπιστολὴν περιέχουσαν οὕτως· [Ἐπιστολὴ Δαρείου Πώρῳ βασιλεῖ 
Ἰνδῶν. 



Historia Alexandri Magni, Recensio α sive Recensio vetusta 
Book 2, chapter 22, section 17, line 2

Θύσας δὲ τοῖς ἐγχωρίοις θεοῖς καὶ ἀναλαβὼν τὴν δύναμιν μαθὼν 
Πῶρον συμμαχήσοντα Δαρείῳ τὴν ὁδοιπορίαν ἐποιεῖτο πρὸς Ἰνδούς. 



Historia Alexandri Magni, Recensio α sive Recensio vetusta 
Book 3, chapter 1, section 3, line 1

                                                     τί ἄρτι κάμνομεν πορευό-
μενοι πρὸς Ἰνδοὺς εἰς θηριώδεις τόπους μὴ προσήκοντας τῇ Ἑλλάδι; 



Historia Alexandri Magni, Recensio α sive Recensio vetusta 
Book 3, chapter 2, section 1, line 1

Καὶ μεθ' ἡμέρας εἰς τοὺς τῆς Ἰνδικῆς <ὅρους> ἐγένοντο· ὑπήντησαν 
δὲ αὐτῷ γραμματηφόροι Πώρου καὶ ἐπέδωκαν αὐτῷ [τὰ] γράμματα. 



Historia Alexandri Magni, Recensio α sive Recensio vetusta 
Book 3, chapter 2, section 2, line 1

                                                                            ὁ δὲ 
ἀνεγίνωσκεν οὕτως· 
 [Ἐπιστολὴ Πώρου βασιλέως Ἰνδῶν Ἀλεξάνδρῳ. 



Historia Alexandri Magni, Recensio α sive Recensio vetusta 
Book 3, chapter 2, section 2, line 2

                                                           Βασιλεὺς Πῶρος 
Ἰνδῶν Ἀλεξάνδρῳ τῷ τὰς πόλεις λεηλατοῦντι προστάττω. 



Historia Alexandri Magni, Recensio α sive Recensio vetusta 
Book 3, chapter 2, section 3, line 1

ἐγὼ γὰρ ἀήττητός εἰμι, οὐ μόνον ἀνθρώπων βασιλεὺς ὤν, ἀλλὰ καὶ θεῶν· 
παρόντα γὰρ ὃν λέγουσι Διόνυσον ἀπήλασαν τῇ ἰδίᾳ δυνάμει οἱ Ἰνδοί. 



Historia Alexandri Magni, Recensio α sive Recensio vetusta 
Book 3, chapter 2, section 9, line 1

                                                                       ἔτι προθυ-
μοτέρους ἡμᾶς ἐποίησας εἰς μάχην σοι ὀτρυνθῆναι, λέγων τὴν Ἑλλάδα 
μηδὲν ἄξιον ἔχειν τῆς Ἰνδῶν χώρας, ὑμᾶς δὲ τοὺς Ἰνδοὺς πάντα κεκτῆσθαι. 



Historia Alexandri Magni, Recensio α sive Recensio vetusta 
Book 3, chapter 3, section 2, line 1

Οὕτως γράψας ἐκπέμπει· Πῶρος δὲ ὠτρύνθη ἀναγνοὺς τὰ γράμματα 
<καὶ> συνάγει τὰ πλήθη καὶ πλείστους ἐλέφαντας καὶ ἕτερα γενναῖα θηρία, 
ἅτινα συνεμάχοντο τοῖς Ἰνδοῖς. 



Historia Alexandri Magni, Recensio α sive Recensio vetusta 
Book 3, chapter 3, section 5, line 1

Οἱ δὲ Πέρσαι μᾶλλον καταδυναστεύουσι τοὺς Ἰνδοὺς καὶ τούτους ἀπε-
δίωκον τοξοβολίαις καὶ ἱππομαχίαις· πολλὴ δὲ αὐτῶν γίνεται μάχη ἀναι-
ρούντων καὶ ἀναιρουμένων. 



Historia Alexandri Magni, Recensio α sive Recensio vetusta 
Book 3, chapter 4, section 5, line 2

Ἤρξαντο δὲ αἱ στρατιαὶ πρὸς ἄλληλα πολεμεῖν· καὶ ὁ Ἀλέξανδρος   
εἶπεν· ‘Τάλανες Ἰνδοί, τίνι πολεμεῖτε τοῦ βασιλέως ὑμῶν ἀναιρεθέντος; 



Historia Alexandri Magni, Recensio α sive Recensio vetusta 
Book 3, chapter 4, section 7, line 2

                                                   Ταῦτα εἶπεν, ὅτι οὐκ ἀνη-
λόγει τὸ στράτευμα αὐτοῦ πρὸς τοὺς Ἰνδούς. 



Historia Alexandri Magni, Recensio α sive Recensio vetusta 
Book 3, chapter 4, section 8, line 1

[Ὑπετάξατο δὲ ὁ Ἀλέξανδρος καὶ τοὺς λοιποὺς τόπους τῆς Ἰνδικῆς 
βασιλείας, ἐχειρώσατο δὲ καὶ τοὺς ὑπὸ Παυσανίαν Ἰνδούς. 



Historia Alexandri Magni, Recensio α sive Recensio vetusta 
Book 3, chapter 4, section 10, line 4

                                                                              ἐκελεύσατο 
κατασκευασθῆναι πασσάλους σιδηροῦς καὶ τούτους ἐμπαγῆναι εἰς τοὺς 
κρημνούς, δι' ὧν ἀναβαίνοντες οἱ Μακεδόνες † ὠχυρωμένοι τοῖς ὑπερμαχο-
μένοις Ἰνδοῖς ἴσχυσαν καταλαβέσθαι τὴν πέτραν. 



Historia Alexandri Magni, Recensio α sive Recensio vetusta 
Book 3, chapter 4, section 12, line 3

       ἔνθεν <δέ> τις καταμάθῃ τὴν εὐτολμίαν μάλιστα αὐτοῦ, ἦν πόλις 
τῆς Ἰνδικῆς, εἰς ἣν πολλοὶ πεφευγότες ἐκ τῶν ἄλλων πόλεων ἐληλύθεισαν 
ὡς ἂν μεγίστην καὶ ὀχυρωτάτην. 



Historia Alexandri Magni, Recensio α sive Recensio vetusta 
Book 3, chapter 17, section 2, line 2

                                                   τὸ συμβεβηκὸς ἡμῖν παρά-
δοξον ἐπὶ τῆς Ἰνδικῆς χώρας ἀναγκαῖον ἐξειπεῖν. 



Historia Alexandri Magni, Recensio α sive Recensio vetusta 
Book 3, chapter 17, section 2, line 3

                                                           παραγενομένων γὰρ 
ἡμῶν εἰς τὴν Πρασιακὴν πόλιν, ἥτις ἐδόκει μητρόπολις εἶναι τῆς Ἰνδικῆς 
χώρας, κατελάβομεν παρ' αὐτὴν ἐναργὲς ἀκρωτήριον τῆς θαλάσσης. 



Historia Alexandri Magni, Recensio α sive Recensio vetusta 
Book 3, chapter 17, section 26, line 1

            γενομένου δὲ <τούτου> καί μου τὰ πέριξ κατὰ φύσιν οἰκονο-
μήσαντος καὶ τῶν Ἰνδῶν προθύμως συνελθόντων ἔλεγόν μοι· ‘Βασιλεῦ 
Ἀλέξανδρε, λήψῃ πόλεις καὶ βασιλείας καὶ ὄρη καὶ ἔθνη, εἰς ἃ οὐδεὶς τῶν 
ζώντων ἐπέβη <ποτὲ βασιλεύς>. 



Historia Alexandri Magni, Recensio α sive Recensio vetusta 
Book 3, chapter 17, section 33, line 2

                                  καὶ προσκαλοῦμαι ἐκ τῶν συνακολουθησάντων 
μοι Ἰνδῶν, ἵνα ἑρμηνείας τύχω παρ' αὐτῶν. 



Historia Alexandri Magni, Recensio α sive Recensio vetusta 
Book 3, chapter 17, section 33, line 4

                                       .. ἅμα τῷ δῦναι τὸν ἥλιον φωνὴ 
ἠνέχθη Ἰνδικὴ ἐκ τοῦ δένδρου, ἣ ἑρμηνεύθη μοι ὑπὸ τῶν Ἰνδῶν τῶν ὄντων 
σὺν ἡμῖν. 



Historia Alexandri Magni, Recensio α sive Recensio vetusta 
Book 3, chapter 17, section 34, line 2

             καὶ φοβούμενοι οὐκ ἤθελον μεθερμηνεῦσαι· σύννους ἐγενάμην 
καὶ εἵλκυσα αὐτοὺς κατὰ μόνας, καὶ εἶπον τοῦτο οἱ Ἰνδοί· ‘Ταχὺ ἀπολέσθαι 
ἔχεις ὑπὸ τῶν ἰδίων. 



Historia Alexandri Magni, Recensio α sive Recensio vetusta 
Book 3, chapter 17, section 38, line 5

                                                                                          .. 
περιλύπου δέ μου διακειμένου καὶ λίαν δυσφοροῦντος ὅ τε Παρμενίων καὶ 
ὁ Φίλιππος παρεκάλουν με περὶ τὸν ὕπνον γενέσθαι· μὴ βουληθέντος δέ 
μου ἀναστὰς ὤρθρισα <καὶ> περὶ τὴν ἀνατολὴν σὺν τοῖς ιʹ φίλοις καὶ τῷ 
ἱερεῖ καὶ τοῖς Ἰνδοῖς πάλιν εἰς τὸ ἱερὸν ἀπελθὼν καὶ διαστολὰς δοὺς 
προσελθών τε εἰς τὸ ἱερὸν σὺν τῷ ἱερεῖ καὶ ἐπιθεὶς τὴν χεῖρα πρὸς τὸ 
δένδρον ἐπηρώτησα λέγων· ‘Εἰ πεπλήρωταί μοι τὰ τῆς ζωῆς ἔτη, τοῦτο 
βούλομαι παρ' ὑμῶν μαθεῖν, εἰ ἀνακομισθήσομαι εἰς Μακεδονίαν καὶ ἀσπά-
σομαι τὴν μητέρα μου καὶ τὴν γυναῖκα, καὶ τότε † ἀπαναλῦσαι. 



Historia Alexandri Magni, Recensio α sive Recensio vetusta 
Book 3, chapter 22, section 12, line 1

                                                        ὁ Περσολέτης, ὁ Ἰνδο-
λέτης, ὁ καθελὼν τρόπαια Μήδων καὶ Πάρθων νῦν χωρὶς πολέμων καὶ 
στρατείας ὑποχείριος ἐγίνου Κανδάκῃ βασιλίσσῃ· ὥστε γίνωσκε, Ἀλέξανδρε, 
ὅτι ὅστις δοκεῖ τῶν ἀνθρώπων φρονεῖν μέγα, ἄλλος μείζονα τούτου τὴν 
φρόνησιν σχῇ’. 



Historia Alexandri Magni, Recensio α sive Recensio vetusta 
Book 3, chapter 25, section 3, line 2

                            ἐκεῖθεν δὲ εἰς τοὺς Ἰνδοὺς ἐστρατεύσαμεν καὶ 
εὑρόντες τοὺς † ἑπομένους αὐτῷ βασιλεῖς, ὄντας δὲ καὶ γυμνοσοφιστάς, 
καὶ λαβόντες φόρον παρ' αὐτῶν ἀφήκαμεν ἐπὶ τῶν ἰδίων τόπων μένειν 
παρακαλέσαντας ἡμᾶς καὶ ἐν εἰρήνῃ τὴν χώραν κατεστήσαμεν, ὥστε ἡδέως 
ἡμᾶς προσδέξασθαι καὶ θυσίαν ὑπὲρ ἡμῶν ποιῆσαι. 



Historia Alexandri Magni, Recensio α sive Recensio vetusta 
Book 3, chapter 26, section 7, line 18

                   ταῦτα ἀκούσας ὁ φρενήρης Ἀλέξανδρος προνομεύσας τὴν 
παραποταμίαν καὶ τὴν ἄλλην χώραν τῶν Ἰνδῶν καὶ οὕτω βωμοὺς κτίσας 
ἐμπύρους διὰ τοῦ στρατεύματος αὐτοῦ θυσίας τοῖς θεοῖς ἐτελείωσεν. 



Historia Alexandri Magni, Recensio α sive Recensio vetusta 
Book 3, chapter 26, section 7, line 26

                                                                ἐν γὰρ τῇ τῶν 
Ἰνδῶν χώρᾳ δὶς καὶ πλεονάκις εἰς χειμῶνας εἰσπεσὼν διεσώθης· εἰς δὲ 
ἐκείνην τὴν χώραν ἐάν τις εἰσελθεῖν τολμήσῃ, ἐπιφανῶν καὶ θαυμαστῶν 
ἔργων δόξαν λήψεται. 



Historia Alexandri Magni, Recensio α sive Recensio vetusta 
Book 3, chapter 28, section 4, line 2

                                                     .. προσπλεύσαντες οὖν 
εὕρομεν πόλιν τοῦ Ἡλίου, ἧς ἦν περίμετρον σταδίων ρκʹ [Ἡλιοπολίτης], 
πύργοι δὲ ἦσαν ιδʹ χρυσῷ καὶ σμαράγδῳ ᾠκοδομημένοι· τὸ δὲ τεῖχος ἐκ 
λίθου Ἰνδικοῦ. 



Historia Alexandri Magni, Recensio α sive Recensio vetusta 
Book 3, chapter 33, section 21, line 1

Ἀποδείκνυσι βασιλεὺς Ἀλέξανδρος Ἰνδικῆς βασιλέα τῆς μὲν παρατει-
νούσης παρὰ τῷ Ὑδάσπῃ ποταμῷ Ταξίλην, τῆς δὲ περιεχομένης ἀπὸ τοῦ 
Ὑδάσπου <μέχρις Ἰνδοῦ> ποταμοῦ Πῶρον, ἐπὶ δὲ Παροπανισαδῶν Ὀξυδράκην 
τὸν Βακτριανὸν τὸν Ῥωξάνης πατέρα τῆς Ἀλεξάνδρου γυναικός. 



Historia Alexandri Magni, Recensio β (1386: 002)
“Der griechische Alexanderroman. Rezension β”, Ed. Bergson, L.
Stockholm: Almqvist \& Wiksell, 1965.
Book 1, section 2, line 8

                                                                                     εἰσὶ γὰρ οἱ 
ἐπερχόμενοι ἡμῖν Ἰνδοὶ καὶ Νοκυμαῖοι, Ὀξύδρακες, Ἴβηρες, Καύχωνες, 
Ἀέλαπες, Βόσποροι, Βασταρνοί, Ἀζανοί, Χάλυβες καὶ ὅσα ἄλλα ἐπὶ τῆς 
ἀνατολῆς παράκεινται ἔθνη μεγάλα, ἀναριθμήτων ἀνδρῶν στρατόπεδα ἐπὶ τὴν 
Αἴγυπτον ἐπερχόμενα. 



Historia Alexandri Magni, Recensio β 
Book 2, section 7, line 29

ὥστε οὖν καὶ σύ, βασιλεῦ, <μετάπεμψαι> τοιούτους σατράπας καὶ ὅσα ἔθνη σοί 
ἐστι Περσῶν καὶ Πάρθων καὶ Μήδων καὶ Ἐλυμαίων καὶ Βαβυλωνίων καὶ τῶν 
κατὰ τὴν Μεσοποταμίαν καὶ Ἰλλύρων χώραν, ἵνα μή σοι λέγω τὰ Βάκτρων 
καὶ τὰ Ἰνδῶν ὀνόματα. 



Historia Alexandri Magni, Recensio β 
Book 2, section 11, line 20

                                                                                 ἔγραψε 
δὲ καὶ Πώρῳ τῷ βασιλεῖ τῶν Ἰνδῶν δεόμενος βοηθείας τυχεῖν παρ' αὐτοῦ. 



Historia Alexandri Magni, Recensio β 
Book 2, section 12, line 3

Δεξάμενος δὲ Πῶρος ὁ βασιλεὺς τὰ γράμματα Δαρείου καὶ ἀναγνοὺς τὰς 
συμφορὰς τὰς γινομένας αὐτῷ ἐλυπήθη καὶ ἀντιγράφει αὐτῷ οὕτως· “Πῶρος 
βασιλεὺς Ἰνδῶν βασιλεῖ Περσῶν Δαρείῳ χαίρειν. 



Historia Alexandri Magni, Recensio β 
Book 2, section 19, line 2

                                                                                  καὶ 
γράφει Πώρῳ βασιλεῖ Ἰνδῶν οὕτως· 
 “Βασιλεὺς Δαρεῖος βασιλεῖ Ἰνδῶν Πώρῳ χαίρειν· . 



Historia Alexandri Magni, Recensio β 
Book 3, section 1, line 2

Μετὰ δὲ ταῦτα πάντα τὴν ὁδοιπορίαν ἐποιεῖτο Ἀλέξανδρος ἀναλαβὼν τὴν 
δύναμιν αὐτοῦ πρὸς Πῶρον βασιλέα τῶν Ἰνδῶν. 



Historia Alexandri Magni, Recensio β 
Book 3, section 1, line 6

                                                                     τί οὖν κάμνομεν 
πορευόμενοι πρὸς Ἰνδοὺς εἰς θηριώδεις τόπους μὴ προσήκοντας τῇ Ἑλλάδι; 



Historia Alexandri Magni, Recensio β 
Book 3, section 2, line 2

Καὶ παραγενομένου αὐτοῦ σὺν πάσῃ τῇ δυνάμει αὐτοῦ εἰς τοὺς ὅρους τῆς   
Ἰνδικῆς χώρας ὑπήντησαν αὐτῷ γραμματοφόροι σταλέντες παρὰ Πώρου βασι-
λέως Ἰνδῶν καὶ ἐπέδωκαν αὐτῷ τὰ γράμματα Πώρου. 



Historia Alexandri Magni, Recensio β 
Book 3, section 2, line 5

                                                              καὶ λαβὼν Ἀλέξανδρος 
ἀνέγνω ἐπὶ τῶν στρατοπέδων αὐτοῦ περιέχοντα οὕτως· 
 “Βασιλεὺς Πῶρος Ἰνδῶν Ἀλεξάνδρῳ πόλεις λεηλατοῦντι· προστάσσω σοι 
ἀναχωρεῖν· ἄνθρωπος γὰρ ὢν τί δύνασαι πρὸς θεόν; 



Historia Alexandri Magni, Recensio β 
Book 3, section 2, line 14

                                           εἰ γὰρ χρείαν εἴχομεν τῆς Ἑλλάδος, πάλαι 
πρὶν Ξέρξου κατεδουλωσάμεθα αὐτὴν Ἰνδοί. 



Historia Alexandri Magni, Recensio β 
Book 3, section 2, line 30

                                                                                   ἔτι 
μᾶλλον περισσοτέρως ἡμᾶς προθύμους ἐποίησας πρὸς μάχην σοι ὀτρυνθῆναι 
λέγων τὴν Ἑλλάδα μηδὲν ἄξιον ἔχειν βασιλικῆς θεωρίας, ἀλλ' ὑμᾶς τοὺς Ἰν-
δοὺς πάντα κεκτῆσθαι καὶ χώρας τε καὶ πόλεις. 



Historia Alexandri Magni, Recensio β 
Book 3, section 3, line 3

Πῶρος δὲ δεξάμενος τὰ γράμματα Ἀλεξάνδρου καὶ ἀναγνοὺς ὠτρύνθη 
σφόδρα καὶ εὐθέως συνήγαγε τὰ πλήθη τῶν βαρβάρων καὶ ἐλέφαντας καὶ ἕτερα 
πολλὰ θηρία, ἅτινα συνεμάχοντο τοῖς Ἰνδοῖς. 



Historia Alexandri Magni, Recensio β 
Book 3, section 3, line 8

                                                                 οἱ δὲ Ἰνδοὶ τοῦτον 
θεασάμενοι εὐθέως παρέστησαν αὐτὸν Πώρῳ τῷ βασιλεῖ. 



Historia Alexandri Magni, Recensio β 
Book 3, section 3, line 24

                                        οἱ δὲ Πέρσαι κατεδυνάστευον τοὺς Ἰν-
δοὺς καὶ τούτους ἐπεδίωκον τοξοβολίαις καὶ ἱππομαχίαις. 



Historia Alexandri Magni, Recensio β 
Book 3, section 4, line 12

                                                                             ὁ δὲ Ἀλέξανδρος 
κυλλάνας τοὺς πόδας [Πώρου] ἐμπηδᾷ εἰς αὐτὸν καὶ ἐντίθησι τὸ ξίφος αὐτοῦ 
εἰς τὰς λαγόνας αὐτοῦ, καὶ παραυτίκα ἀναιρεῖ Πῶρον τὸν βασιλέα Ἰνδῶν. 



Historia Alexandri Magni, Recensio β 
Book 3, section 4, line 14

                                                                         ὁ οὖν 
Ἀλέξανδρος λέγει πρὸς τοὺς Ἰνδούς· “τάλανες Ἰνδοί, τίνα πολεμεῖτε τοῦ 
βασιλέως ὑμῶν ἀναιρεθέντος; 



Historia Alexandri Magni, Recensio β 
Book 3, section 4, line 19

                                                       ταῦτα δὲ εἶπεν εἰδὼς ὅτι 
οὐκ ἀναλογεῖ τὸ στρατόπεδον αὐτοῦ πρὸς τὸ τῶν Ἰνδῶν μάχεσθαι. 



Historia Alexandri Magni, Recensio β 
Book 3, section 17, line 3

Καὶ τούτου γενομένου ὑπεχώρησεν ἀπ' αὐτῶν ὁ Ἀλέξανδρος ὑποστρέψας εἰς 
τὴν κατὰ φύσιν ὁδὸν τὴν φέρουσαν εἰς τὴν Πρασιακὴν πόλιν, ἥτις δοκεῖ μητρό-
πολις εἶναι τῆς Ἰνδικῆς χώρας, ἔνθα Πῶρος ἦν βασιλεύων. 



Historia Alexandri Magni, Recensio β 
Book 3, section 17, line 5

                                         καὶ πάντα κατὰ φύσιν διοικονομήσαντος 
καὶ τῶν Ἰνδῶν προθύμως συνελθόντων ἔλεγόν τινες ἐξ αὐτῶν τῷ Ἀλεξάνδρῳ· 
“μέγιστε βασιλεῦ, λήψῃ πόλεις θαυμαστὰς καὶ βασιλείας καὶ ὄρη, εἰς ἃ οὐδεὶς 
τῶν ζώντων ἐπέβη ποτὲ βασιλεύς. 



Historia Alexandri Magni, Recensio β 
Book 3, section 17, line 28

                                                                            προσκαλεῖται 
οὖν ἐκ τῶν συνακολουθησάντων αὐτῷ Ἰνδῶν, ἵνα ἑρμηνείας τύχῃ παρ' αὐτῶν. 



Historia Alexandri Magni, Recensio β 
Book 3, section 17, line 31

Ἐγένετο δὲ ἅμα τῷ δῦναι τὸν ἥλιον, φωνὴ ἠνέχθη Ἰνδικὴ ἀπὸ τοῦ δένδρου. 



Historia Alexandri Magni, Recensio β 
Book 3, section 17, line 32

οἱ δὲ συνόντες αὐτῷ Ἰνδοὶ φοβούμενοι οὐκ ἠθέλησαν μεθερμηνεῦσαι. 



Historia Alexandri Magni, Recensio β 
Book 3, section 17, line 44

                                                           περίλυπος δὲ γενόμενος 
Ἀλέξανδρος ἀναστὰς ὄρθρου σὺν τοῖς ἱερεῦσι καὶ τοῖς φίλοις αὐτοῦ καὶ τοῖς 
Ἰνδοῖς πάλιν εἰς τὸ ἱερὸν εἰσῆλθεν. 



Historia Alexandri Magni, Recensio β 
Book 3, section 17, line 53

                                       καὶ ἐξελθὼν ἐκεῖθεν ἐκίνησεν ἀναχωρῶν ἀπὸ 
τῆς Ἰνδικῆς, καὶ παραγίνεται ἐν Περσίδι. 



Historia Alexandri Magni, Recensio β 
Book 3, section 22, line 31

            ὁ Περσολέτης, ὁ Ἰνδολέτης, ὁ καθελὼν τρόπαια Μήδων καὶ Πάρθων 
καὶ ὅλην τὴν ἀνατολὴν καταβαλὼν νῦν χωρὶς πολέμου καὶ στρατιᾶς ὑποχείριος 
γέγονας Κανδάκης. 



Historia Alexandri Magni, Recensio β 
Book 3, section 22, line 43

                                                                            ἐὰν γὰρ 
γνώσωσί σε ὄντα Ἀλέξανδρον, ἀναιροῦσί σε εὐθέως, ὅτι σὺ Πῶρον τὸν βασιλέα 
Ἰνδῶν ἀνῄρησας. 



Historia Alexandri Magni, Recensio β 
Book 3, section 25, line 6

                                                  τὴν μὲν πρὸς Δαρεῖον μάχην 
οἴομαι ὑμᾶς ἀκηκοέναι, ἐκεῖθεν δὲ εἰς τοὺς Ἰνδοὺς ἐπεστρατεύσαμεν καὶ ἡττή-
σαμεν τοὺς ἡγουμένους αὐτῶν καὶ κατεδουλώσαμεν αὐτοὺς διὰ τῆς ἄνω προ-
νοίας. 



Historia Alexandri Magni, Recensio β 
Book 3, section 28, line 11

                                                             τὸ δὲ τεῖχος τῆς πόλεως 
ἐκείνης Ἰνδικὸν ἦν. 



Historia Alexandri Magni, Recensio γ (lib. 1) (1386: 003)
“Der griechische Alexanderroman. Rezension γ. Buch I”, Ed. von Lauenstein, U.
Meisenheim am Glan: Hain, 1962; Beiträge zur klassischen Philologie 4.
Section 2, line 11

                                 εἰσὶ γὰρ οἱ ἐπερχόμενοι ἡμῖν· 
Ἰνδοὶ καὶ Νοκημαῖοι Ὀξύδρακες Ἴβηροι Καύχονες Ἀέλαπες 
Βοσπορηνοὶ Βαστρανοὶ Ἀξανοὶ Χάλυβες καὶ ὅσα ἄλλα ἐπὶ τῆς 
ἀνατολῆς παράκεινται ἔθνη. 



Historia Alexandri Magni, Recensio γ (lib. 2) (1386: 004)
“Der griechische Alexanderroman. Rezension γ. Buch II”, Ed. Engelmann, H.
Meisenheim am Glan: Hain, 1963; Beiträge zur klassischen Philologie 12.
Section 11, line 21

        ἔγραψε δὲ καὶ ἐπιστολὴν πρὸς Πῶρον, βασιλέα τῶν Ἰν-
δῶν, περιέχουσαν οὕτως·   
τῷ ἐν θεοῖς θεῷ μεγάλῳ βασιλεῖ Πώρῳ χαίρειν Δαρεῖος ὁ 
δυστυχής· τὰ περὶ τῶν ἡμετέρων δυστυχημάτων καὶ γράφειν 
ἀδύνατον· οἶμαι δὲ σὲ τὸν ἐμὸν δεσπότην ἀκηκοέναι τῶν 
πολλῶν ὀλίγα, ὡς παῖς εἰς ἡμᾶς Μακεδόνων ἐπιβὰς λῃστρικῶς 
καταναγκάζει μετανάστας τῶν οἰκείων γενέσθαι, τὴν δουλι-
κὴν ἠθετηκὼς τύχην· δεσποτικῶς ἡμᾶς σπουδάζει διατεθῆναι 
καὶ τὴν τῆς ἑῴ⌊ας⌋ ἀρχὴν ταῖς δυσμαῖς παραπέμψαι. 



Historia Alexandri Magni, Recensio γ (lib. 2) 
Section 11, line 34

                                                           ἐπίσταμαι 
γὰρ ἀκαταμάχητον εἶναι τὸ τῶν Ἰνδῶν στῖφος· ἐπικάμφθητι 
δή μου τοῖς γράμμασιν καὶ τὴν ἐκ ψυχῆς ⌊αἴτησιν⌋ πλήρωσον 
καὶ τοὺς κατεπείγον⌊τάς με Μα⌋κεδόνας ἀμύνασθαι θέλησον. 



Historia Alexandri Magni, Recensio γ (lib. 2) 
Section 12, line 11

γράφει δὲ καὶ ἐπιστολὴν πρὸς Δαρεῖον περιέχουσαν οὕτως·   
Πῶρος βασιλεὺς Ἰνδῶν βασιλεῖ Περσῶν Δαρείῳ χαίρειν· 
ἀναγνοὺς τὰ γεγραμμένα ἡμῖν ὑπὸ σοῦ ἐλυπήθην σφόδρα καὶ 
ἀπορῶ θέλων σοι συντυχεῖν καὶ βουλεύσασθαι περὶ τ⌊ῶν συμ⌋-
φερόντων· κωλύομαι δὲ ὑπὸ τῆς συνεχούσης με σωματικῆς νό-
σου· εὐθύμως οὖν δίαγε, ὡς ἡμῶν συμπαρόντων σοι καὶ μὴ δυ-
ναμένων στέργειν τὴν ὕβριν ταύτην. 



Historia Alexandri Magni, Recensio γ (lib. 2) 
Section 19, line 2

ὁ δὲ Δαρεῖος ηὐτρεπίζετο εἰς ἕτερον πόλεμον, γράφει δὲ 
καὶ Πώρῳ, βασιλεῖ Ἰνδῶν, οὕτως·   
βασιλεὺς Δαρεῖος βασιλεῖ Ἰνδῶν Πώρῳ χαίρειν· < > 
<    > ἐπὶ τῇ γενομένῃ καταστροφῇ τοῦ οἴκου μου ἐν ταῖς 
ἡμέραις ταύταις, ἐπειδὴ ἐπιβὰς ὁ Μακεδὼν θηρὸς ἀγρίου τύ-
χην ἔχων οὐ βούλεται τὴν μητέρα μου καὶ τὴν γυναῖκα καὶ 
τὰ τέκνα ἀποδοῦναι· ἐμοῦ δὲ ἐπαγγειλαμένου καὶ θησαυροὺς 
καὶ ἄλλα τινὰ πλείονα παρασχεῖν αὐτῷ οὐ πείθεται. 



Historia Alexandri Magni, Recensio γ (lib. 2) 
Section 35a,44, line 11

                                                       ἀλλὰ καὶ ἐν ὀλί-
γῃ πορφύρᾳ στιλβούσῃ ἐν ἱματίῳ ἐναβρύνεσθε μεγάλως, τῶν Ἰν-
δῶν ὁλοπορφύρων ὄντων καὶ τῶν δούλων αὐτῶν ὁλοπόρφυρα φορούν-
των· καὶ ὑμεῖς ἐπίκαλον ἡγεῖσθε τὴν πορφύραν κἂν ὀλίγῃ χρῆσθε. 



Historia Alexandri Magni, Recensio γ (lib. 2) 
Section 35a,57, line 2

σὺ δὲ, βασιλεῦ Ἀλέξανδρε, τὰ ἡμέτερα φρονεῖν παραγενό-
μενος εἰς Ἰνδοὺς καὶ Βραγμᾶνας ἰδών ἐπ' ἐρημίας οἴκησον 
γυμνός· ἄλλως γάρ σε οὐ δυνάμεθα ἡμεῖς δέξασθαι, ἐὰν μὴ   
πρῶτον τὰς ἀρχὰς ἀπορρίψῃς ἀπὸ σεαυτοῦ, ἐφ' οἷς νῦν γέγηθας 
καὶ μέγα φρονεῖς· ἅψονται δέ σου προνοίας λόγοι, οὕσπερ 
πρότερον εἰρήκειν σοι· καὶ ἀγαπήσεις ἐκ καρδίας σου πάντα, 
ὅσα τότε ἐπῄνεσας θαυμάσας· καὶ εἰ ἐμοὶ πεισθῇς καὶ ποιή-
σῃς ταῦτα, οὐδεὶς ὅλως οὐκέτι πολεμήσει σε οὐδ' ἀφελέσθαι 
σου τίς τι δυνήσεταί ποτε λοιπόν, ὧν οὐ κέκτησαι· ἐὰν γὰρ 
ἐμοὶ πεισθῇς κατὰ κράτος καὶ ἀσφαλῶς, οὐδεὶς οὐδὲν τοῦ 
ὑμετέρου βίου εὑρήσει παρά σοι· ὗλαι γάρ σε θρέψουσιν,

λοι-



Historia Alexandri Magni, Recensio γ (lib. 2) 
Section 43, line 78

                                              τανῦν δὲ Πῶρον τὸν   
τῶν Ἰνδῶν βασιλέα εὐτρεπιζόμεθα πολεμῆσαι καὶ ὅσα περὶ 
ἡμῶν ἡ θεία εὐοδώσειεν πρόνοια. 



Historia Alexandri Magni, Recensio γ (lib. 2) 
Section 44, line 2

       ἀπάρας οὖν τῶν ἐκεῖ ὥρμησεν κατὰ Ἰνδῶν· καὶ δὴ τὴν 
Ἡλίου καταλαβόμενος χώραν εἰσελθὼν ἐν τῇ πόλει. 



Historia Alexandri Magni, Recensio γ (lib. 3) (1386: 005)
“Der griechische Alexanderroman. Rezension γ. Buch III”, Ed. Parthe, F.
Meisenheim am Glan: Hain, 1969; Beiträge zur klassischen Philologie 33.
Section 2, line 2

τῶν ἐκεῖσε οὖν διελθόντων ἡμῶν ἄγγελοι ἤλθοσαν Πώρου τῶν 
Ἰνδῶν βασιλέως πρὸς Ἀλέξανδρον ἐπιστολὴν κομισάμενοι 
περιέχουσαν οὕτως· 
ἐπειδήπερ, Ἀλέξανδρε, ἡ κατὰ Δαρείου νίκη εἰς ὑπεροψίαν 
μείζονά σε ἐθράσυνε, καὶ κατὰ θεὸν ἦρας τὸ δόρυ σου, ἀπεί-
ρως ἔχων ὅση παρ' ἐμοὶ δύναμις καὶ ὅτι τῷ θυμῷ τῷ ἐμῷ οὐχ 
ὑποστήσεται ἡ ὑπὸ οὐρανόν· σὺ γὰρ ἄνθρωπος ὢν τί δύνα-
σαι πρὸς θεόν; 



Historia Alexandri Magni, Recensio γ (lib. 3) 
Section 2, line 20

                                         εἰ γὰρ χρείαν εἴχομεν 
τῆς Ἑλλάδος, πάλαι <ἂν> πρὶν Ξέρξου κατεδουλωσάμεθα 
αὐτὴν οἱ Ἰνδοί. 



Historia Alexandri Magni, Recensio γ (lib. 3) 
Section 3, line 3

Πῶρος δεξάμενος τὰ γράμματα Ἀλεξάνδρου καὶ ἀναγνοὺς 
ὠτρύνθη σφόδρα καὶ εὐθέως συνήγαγε τὰ πλήθη τῶν βαρβάρων 
καὶ ἐλέφαντας καὶ ἕτερα πολλὰ θηρία, ἅτινα τοῖς Ἰνδοῖς 
συνεμάχοντο. 



Historia Alexandri Magni, Recensio γ (lib. 3) 
Section 3, line 8

                  τί οὖν κάμνομεν εἰς Ἰνδοὺς εἰς θηριώδεις 
τόπους, μὴ προσήκοντας τῇ Ἑλλάδι; 



Historia Alexandri Magni, Recensio γ (lib. 3) 
Section 3, line 21

                                                           τοῦτο 
μέντοι ὑμᾶς ὑπομνήσω, {ὡς} ὅτι κἀκείνους τοὺς πολέμους 
ἐγὼ μόνος ἐνίκησα· καὶ ὅσους βούλομαι λαβεῖν τῶν Ἰνδῶν, 
πάλιν ἐγὼ μόνος νικήσω· ἐμοῦ γὰρ ἓν βούλημα πρὸς τὸν 
πόλεμον τὰς ψυχὰς πάντων ἐθάρσυνε ἤδη ἀδρανοῦντας πρὸς 
τὰ Δαρείου πλήθη. 



Historia Alexandri Magni, Recensio γ (lib. 3) 
Section 3, line 34

εἰς τοὺς ὅρους τῆς Ἰνδικῆς χώρας ὡς δὲ ἦλθον ἐγγὺς οἱ 
Μακεδόνες καὶ Πέρσαι, ἰδὼν Ἀλέξανδρος τὴν παράταξιν 
Πώρου ἐφοβήθη οὐ τοὺς ὄχλους, ἀλλὰ τοὺς θῆρας· θεωρήσας 
γὰρ τὸ ξένον τῶν θηρίων ἐθαύμασεν· ἀνθρώποις γὰρ εἶχεν   
ἔθος μάχεσθαι καὶ οὐ θηρίοις. 



Historia Alexandri Magni, Recensio γ (lib. 3) 
Section 3, line 41

                     οἱ δὲ Ἰνδοὶ τοῦτον θεασάμενοι εὐθέως 
παρέστησαν αὐτὸν Πώρῳ τῷ βασιλεῖ. 



Historia Alexandri Magni, Recensio γ (lib. 3) 
Section 3, line 62

     ὡς δὲ τὸν πόλεμον συνεκρότησαν, γενναίως 
ἀντίκεινται Πέρσαι καὶ Μακεδόνες τῷ Ἰνδικῷ στρατεύματι· 
καὶ τούτους ἐπεδίωκον τοξοβολίαις καὶ ἱππομαχίαις. 



Historia Alexandri Magni, Recensio γ (lib. 3) 
Section 3, line 66

τῇ δὲ ἐπιούσῃ ἡμέρᾳ πολέμου συγκροτηθέντος ἄφνω ἐλέφαντες 
ἤλθοσαν τῇ τῶν Ἰνδῶν παρατάξει ξυλίνοις ἐπιφερόμενοι 
τείχεσι· καὶ ἐπ' αὐτοὺς ἄνδρες ἔνοπλοι ἵσταντο καὶ δόρατα 
ταῖς χερσὶν ἔχοντες. 



Historia Alexandri Magni, Recensio γ (lib. 3) 
Section 3, line 84

                 ναί, ναί, κτείνατε δή με, πάντας ἱκετεύω, 
εἴ γε βούλεσθε Ἰνδοῖς παραδοῦναι. 



Historia Alexandri Magni, Recensio γ (lib. 3) 
Section 3, line 96

                                                 † 
καὶ δὴ ἡμέρας καταλαβούσης τὸ Ἰνδικὸν εἰς μάχην ἐξῄει 
στρατόπεδον· εἰς ἐλέφαντας δέ, ὡς προεῖπον, ἐπιβεβηκότες 
δίκην περιπατοῦσαι πόλεις ἐφαίνοντο τοῖς τείχεσι κατωχυ-
ρωμέναι. 



Historia Alexandri Magni, Recensio γ (lib. 3) 
Section 3, line 106

                                καὶ δειλία Ἰνδοῖς ἐκ τούτου 
ἐπέπεσεν. 



Historia Alexandri Magni, Recensio γ (lib. 3) 
Section 3, line 110

                                       τούτων οὕτως γενομένων, ὁ 
πόλεμος πάλιν συνεκροτεῖτο καὶ ὑπερισχύουσιν Ἰνδοὶ τῷ 
στρατοπέδῳ Ἀλεξάνδρου. 



Historia Alexandri Magni, Recensio γ (lib. 3) 
Section 3, line 112

                            ὁ δὲ Ἀλέξανδρος ἐκεῖσε θαυμαστῶς 
ἐμαχήσατο, καὶ μόλις τὸ Ἰνδικὸν ἠδυνήθη ἀποστρέψαι στῖφος, 
ὥστε παρ' Ἰνδῶν κυκλωθέντα μόλις ἡ πρόνοια τῷ Βουκεφάλῳ 
ἵππῳ ἐπιβάντα διέσωσε. 



Historia Alexandri Magni, Recensio γ (lib. 3) 
Section 4, line 23

                             ὡς δὲ κύριος Ἀλέξανδρος καὶ 
Ἰνδοῖς ἐγεγόνει, τὸ σκυθρωπὸν Μακεδόνων ῥᾳδίως ἐθεράπευσεν 
καὶ ἀμνήμων ἦν τῶν ἐναγῶν βουλευμάτων. 



Historia Alexandri Magni, Recensio γ (lib. 3) 
Section 17, line 3

ἔδοξεν οὖν αὐτῷ τὸ καταβόρειον μέρος ὑπεξελθεῖν καὶ τὴν 
ὁδὸν τὴν φέρουσαν εἰς τὴν Πρασιακήν, ἥτις δοκεῖ μητρόπολις 
εἶναι τῆς Ἰνδικῆς χώρας, ἔνθα ἦν Πῶρος βασιλεύων. 



Historia Alexandri Magni, Recensio γ (lib. 3) 
Section 17, line 5

                                                          καὶ πάντα 
κατὰ φύσιν διοικονομήσαντος καὶ τῶν Ἰνδῶν προθύμως συν-
ελθόντων, ἔλεγόν τινες τῷ Ἀλεξάνδρῳ· “μέγιστε βασιλεῦ, 
λήψει πόλεις θαυμαστὰς καὶ βασιλείας καὶ ὄρη, εἰς ἃ οὐδεὶς 
τῶν ζώντων ἐπέβη ποτὲ βασιλεύς. 



Historia Alexandri Magni, Recensio γ (lib. 3) 
Section 17, line 37

            συνῆλθον δὲ τῷ Ἀλεξάνδρῳ ἄνδρες ἱκανοί, 
καὶ κελεύει καθοπτεῦσαι τὸν τόπον κύκλῳ· προσκαλεῖται 
οὖν ἐκ συνακολουθησάντων αὐτῷ Ἰνδῶν, ἵνα ἑρμηνείας τύχῃ 
παρ' αὐτῶν. 



Historia Alexandri Magni, Recensio γ (lib. 3) 
Section 17, line 40

                                                               ἐγένετο 
ἅμα τὸ δῦναι τὸν ἥλιον, φωνὴ ἠνέχθη Ἰνδικὴ ἀπὸ τοῦ 
δένδρου. 



Historia Alexandri Magni, Recensio γ (lib. 3) 
Section 17, line 41

          οἱ δὲ συνόντες αὐτῷ Ἰνδοὶ φοβούμενοι οὐκ ἠθέ-
λησαν μεθερμηνεῦσαι. 



Historia Alexandri Magni, Recensio γ (lib. 3) 
Section 17, line 46

                                                           οἱ δὲ 
εἶπον· “Ἀλέξανδρε βασιλεῦ, ἐν τάχει ἀπολέσθαι ἔχεις ἀπὸ 
τῶν † Ἰνδῶν †. 



Historia Alexandri Magni, Recensio γ (lib. 3) 
Section 17, line 58

              περίλυπος δὲ γενόμενος Ἀλέξανδρος ἀναστὰς 
ὄρθρου σὺν τοῖς ἱερεῦσιν καὶ φίλοις αὐτοῦ <καὶ> τοῖς Ἰν-
δοῖς πάλιν εἰς τὸ ἱερὸν εἰσῆλθε καὶ προσευξάμενος προσῆλθε 
σὺν τῷ ἱερεῖ. 



Historia Alexandri Magni, Recensio γ (lib. 3) 
Section 17, line 67

                                          ταῦτα ἀκούσας 
περίλυπος σφόδρα γέγονε· καὶ ἐξελθὼν ἐκεῖθεν ἐκίνησεν 
ἀναχωρῶν ἀπὸ τῆς Ἰνδικῆς. 



Historia Alexandri Magni, Recensio γ (lib. 3) 
Section 20, line 2

ταῦτα δὲ ἀκούσας Ἁντίοχος ἐξῆλθε παραλαβὼν τὸν Κανδάβλην 
καὶ χιλιάδας ἑκατὸν ἀπό τε Περσῶν, Ἰνδῶν καὶ Μακεδόνων 
καὶ ὥρμησεν ἐπὶ Εὐαγρίδην. 



Historia Alexandri Magni, Recensio γ (lib. 3) 
Section 22, line 41

                ὁ Περσολέτης, ὁ Ἰνδολέτης, ὁ καθελὼν 
τρόπῳ Μήδους καὶ Πάρθους καὶ ὅλην τὴν ἀνατολὴν καταλαβὼν, 
νῦν χωρὶς πολέμου καὶ στρατιᾶς ὑποχείριος γέγονας Κανδάκης· 
ὥστε νῦν γίνωσκε, Ἀλέξανδρε, ὅτι ὅστις δοκεῖ τῶν ἀνθρώ-
πων ὑπερφρονεῖν μέγα, καὶ ἄλλος μείζονα τούτου φρονήσει. 



Historia Alexandri Magni, Recensio γ (lib. 3) 
Section 25, line 6

                                              ἐκεῖθεν δὲ εἰς τοὺς 
Ἰνδοὺς ἐπεστρατεύσαμεν καὶ ἡττήσαμεν τοὺς ἡγουμένους 
αὐτῶν· καὶ κατεδουλωσάμεθα αὐτοὺς διὰ τῆς ἄνω προνοίας. 



Historia Alexandri Magni, Recensio γ (lib. 3) 
Section 28, line 9

             ἀπὸ δὲ τοῦ ποταμοῦ ἐκείνου ἀποπλεύσαντες 
ἤλθομεν εἰς νῆσόν τινα μεγάλην· καὶ εὕρομεν ἐκεῖ πόλιν 
τὴν τοῦ Ἡλίου καλουμένην, ἐν ᾗ πύργοι ἦσαν δώδεκα ἀπὸ 
χρυσίου καὶ σμαράγδων ᾠκοδομημένοι· τὸ δὲ τεῖχος τῆς 
πόλεως ἐκείνης Ἰνδικὸν ἦν· ἐν μέσῳ δὲ ταύτης ἦν βωμὸς 
χρυσίῳ καὶ σμαράγδῳ ᾠκοδομημένος ἔχων ἀναβαθμοὺς ἑπτά· 
ἐπάνω δὲ ἵστατο ἅρμα ἵππων καὶ ὁ ἱππηλάτης ἐκ χρυσίου 
καὶ σμαράγδου ᾠκοδομημένα· τὸ δὲ τεῖχος αὐτῆς ὑψηλόν. 



Historia Alexandri Magni, Recensio γ (lib. 3) 
Section 30aC, line 9

                 ἐδόκει δέ μοι περὶ τῆς ἄνω προνοίας βελτίον 
δηλῶσαι σε, καὶ τὰ ἐμοὶ συμβάντα ἐν ταύτῃ ἀνάγνωθι τῇ 
ἐπιστολῇ· 
καθὼς γὰρ πρῷόν σοι εἶπον, ὅτι εἰς τὴν Ἰνδίαν παραγενό-
μενοι ἐν τρισὶ συμβολαῖς τοὺς Ἰνδοὺς ἐπολεμήσαμεν, τὸ 
δὲ τελευταῖον μετὰ Πώρου τοῦ βασιλέως μονομαχήσας τοῦτον 
ἀπέκτεινα καὶ πᾶσαν τὴν Ἰνδίαν παραλαβὼν καὶ ὁμόνοιαν 
Μακεδόσιν καὶ Πέρσαις καὶ Ἰνδοῖς ποιησάμενος τοὺς 
πάντας ἀνέλαβον. 



Historia Alexandri Magni, Recensio γ (lib. 3) 
Section 33, line 13

ἀνθυποστραφεὶς δὲ Πῶρον ἀπέκτεινα τὸν τῶν Ἰνδῶν βασιλέα. 



Historia Alexandri Magni, Recensio γ (lib. 3) 
Section 33C, line 60

εἴθ' οὕτως διανέμει τὴν ἀρχὴν ὁ Ἀλέξανδρος τοῖς ἑαυτοῦ 
ἄρχουσι, ἤγουν Σελεύκῳ, Πτολεμαίῳ τῷ Φιλίππῳ, Ἀντιόχῳ, 
Φίλωνι τῷ Βύζᾳ· Σέλευκον μὲν Πέρσαις καὶ Ἰνδοῖς βασιλεύ-
ειν, Φίλιππον δὲ τὸν καὶ Πτολεμαῖον Αἴγυπτον ἄρχειν, 
Ἀντίοχον δὲ τὴν μέσην γαῖαν τῶν Ἀσσυρίων, Φίλωνα δὲ 
τὸν καὶ Βύζαντα τὴν Μακεδονικὴν διέπειν ἀρχήν, τὰ περὶ 
Ὀλυμπιάδος πλεῖστα παραινέσας αὐτῷ. 



Historia Alexandri Magni, Recensio γ (lib. 3) 
Section 35, line 36

ονας, Κανζιότας, Κανζήτας, Ῥυσπηρήτας, Χαρουρίτας, 
Ὀφιομάγους, Ὀφιόποδας, Ἐλεφαντινόποδας, Σκηβρυότας, 
Ἐξαμάρους, Λογγιβάρδους, Λεβεσεντιανούς, Ἐβρίδας, 
Δερματησίους, Ἀβάσγους, Ἀρμενίους, Ῥουσίους, Ὄχλους, 
Σαρακηνούς, Σύρους, Ἀλανούς, Ἐβρεπάους, Ἐβρεξάους, 
Ἑξαχείρους, Ἑξαστίχους, Λωροπόδους, Ὑποφαλαγγίους, 
Πρίσκους, Λακούς, Πολύποδας, Πατησόφους, Λέβεις, Λυκοκρά-
νους, Κυοκεφάλους (sic), Λοκομίτας, Ὀστρικούς, Πανζήτας, 
Δελεήμας, Σανδαλεῖς, Κανσάδεις, Κασανδριώτας, Αἰγιώτας, 
Ὑποβιώτας, Ὑποβοτίους, Ἰνδούς, Σινδιανούς, Σουγδάους, 
Βαρμαίους, Αἰγυπτίους <σὺν> τοῖς κατοικοῦσιν τοῖς 
σκοτεινοῖς τόποις, Ἑβραίους, Θρυμβήτας, Κούσκους, Χα-
ζάρους, Βουλγάρους, Χουνάβους, Πίνσας, Αἰθίοπας καὶ 
Ῥωμαίους τοὺς νικήτορας στρατιώτας. 



Historia Alexandri Magni, Recensio ε (1386: 006)
“Anonymi Byzantini vita Alexandri regis Macedonum”, Ed. Trumpf, J.
Stuttgart: Teubner, 1974.
Chapter 17, section 4, line 2

Δαρεῖος δὲ μετ' ὀλίγων ἱππέων φυγὼν διασώζεται καὶ 
γράψας ἐπιστολὴν πρὸς Πῶρον τὸν τῶν Ἰνδῶν βασιλέα, ὃς 
ἦν αὐτῷ προσφιλής, περιέχουσαν τάδε· 
 Τῷ ἐν θεοῖς θεῷ μεγάλῳ βασιλεῖ Πώρῳ Δαρεῖος ὁ δυστυ-
 χὴς χαίρειν. 



Historia Alexandri Magni, Recensio ε 
Chapter 17, section 4, line 18

              ἐπίσταμαι γὰρ ἀκαταμάχητον εἶναι τὸ τῶν Ἰν-
 δῶν στῖφος ὡς καί σε ἴσα θεοῖς ὄντα πάντα τὰ δοκούμενα 
 δυνατά. 



Historia Alexandri Magni, Recensio ε 
Chapter 17, section 6, line 2

                               μαθὼν δὲ τὴν τῶν Ἰνδῶν ἔλευσιν 
Δαρεῖος ὀλίγον τι ἀνέθορε καὶ πρὸς ἐλπίδα ἦλθε. 



Historia Alexandri Magni, Recensio ε 
Chapter 17, section 6, line 6

                                             καὶ δὴ τῶν Ἰνδῶν ἐν 
Περσίδα φθασάντων καὶ τῶν ἀμφοτέρων λαῶν εἰς ἑνότητα 
γενομένων καὶ εἰς πλήθη ἄπειρα γεγονότα ὥρμησαν κατὰ 
Ἀλεξάνδρου. 



Historia Alexandri Magni, Recensio ε 
Chapter 17, section 7, line 1

Οἱ δὲ τῶν Ἰνδῶν στρατηγοὶ ἀποστέλλουσί <τινας> κα-
τασκοπεῦσαι Ἀλέξανδρον, οἵτινες παρὰ τῶν ἔξω βιγλῶν ἐκρα-
τήθησαν. 



Historia Alexandri Magni, Recensio ε 
Chapter 17, section 7, line 20

                    οἱ δὲ τὴν φάλαγγα φυλάσσοντες Μακεδόνες 
φασὶ τοῖς κατασκόποις· Ἰδοὺ καὶ τὸ Μακεδονικὸν λῃστήριον· 
εἰ οὐκ αὔταρκες τῷ Ἰνδικῷ καὶ Περσικῷ μάχεσθαι στρατοπέ-
δῳ, ἰδοὺ τὰς φάλαγγας ἡμῶν βλέπετε. 



Historia Alexandri Magni, Recensio ε 
Chapter 18, section 1, line 7

      φόβος δὲ καὶ τρόμος πᾶσιν ἐγένετο Πέρσαις ὁμοῦ καὶ Ἰν-
δοῖς, ἱδρῶτες δὲ περὶ αὐτοὺς ἐγένοντο. 



Historia Alexandri Magni, Recensio ε 
Chapter 18, section 5, line 3

Ἀκούσας δὲ ταῦτα κελεύει Σέλευκον τὸν στρατηγὸν 
αὐτοῦ ἀπελθόντα πρὸς τὸ Περσικὸν στράτευμα, εἰπὼν αὐτῷ· 
Τοὺς Ἰνδοὺς ἀπόστειλον ἀπελθεῖν εἰς τὰ ἴδια ἀφαιρεθέντων 
τῶν ἰδίων ὅπλων. 



Historia Alexandri Magni, Recensio ε 
Chapter 32, section 4, line 4

                                                    ἀνακαλεῖταί σοι 
ἡ ἀνατολὴ καὶ Πώρου τῶν Ἰνδῶν βασιλεία, ἣν νικήσεις. 



Historia Alexandri Magni, Recensio ε 
Chapter 34, section 9, line 1

Τανῦν δὲ πρὸς Πῶρον τὸν τῶν Ἰνδῶν βασιλέα εὐτρεπι-
ζόμεθα πολεμῆσαι, καὶ ὅσα περὶ ἡμῶν ἡ θεία εὐοδώσειε πρό-
νοια. 



Historia Alexandri Magni, Recensio ε 
Chapter 35, section 1, line 2

                       ἀπάρας τῶν ἐκεῖ ὥρμησε κατὰ Ἰνδῶν, 
καὶ δὴ τὴν Ἡλίου καταλαβόμενος χώραν εἰσελθὼν ἐν τῇ πό-
λει, ἱερὸν ἐλέγετο ἐκεῖσε τοῦ ἡλίου εἶναι καὶ ἱερά τινα δένδρα. 



Historia Alexandri Magni, Recensio ε 
Chapter 36, section 4, line 2

Ὡς δὲ τὸν πόλεμον συνεκρότησαν, γενναίως ἀντίκεινται 
Πέρσαι καὶ Μακεδόνες τῷ Ἰνδικῷ στρατεύματι. 



Historia Alexandri Magni, Recensio ε 
Chapter 36, section 4, line 5

     ἐπὶ τούτοις ἄφνω ἐλέφαντες ἐμφαίνονται τῇ Ἰνδῶν παρα-
τάξει ξύλινα τείχη ἐπιφερόμενοι, καὶ ἐπ' αὐτῶν ἄνδρες ἔνοπλοι 
ἦσαν λίθους καὶ δόρατα ταῖς χερσὶν ἔχοντες. 



Historia Alexandri Magni, Recensio ε 
Chapter 36, section 5, line 12

                                             ναί, ναί, κτείνατε 
δή με πάντες ἱκετεύω, εἰ καὶ βούλεσθε Ἰνδοῖς παραδοῦναι. 



Historia Alexandri Magni, Recensio ε 
Chapter 36, section 6, line 1

Καὶ δὴ ἡμέρας καταλαβούσης τὸ Ἰνδικὸν εἰς μάχην 
ἐξῄει στρατόπεδον. 



Historia Alexandri Magni, Recensio ε 
Chapter 36, section 6, line 11

                     καὶ δειλία Ἰνδοῖς ἐκ τούτου ἐπέπεσε καί· 
ὦ ἡμῖν ἀπὸ τοῦδε, εἶπον, τοῦ μειρακίου, καὶ τοὺς λέοντας ἀπέ-  
στρεψε καὶ τοὺς ἐλέφαντας ἀπεδίωξε, ποία δὲ λοιπὸν ἐλπὶς 
ἡμῶν; 



Historia Alexandri Magni, Recensio ε 
Chapter 36, section 6, line 15

         – ἐπὶ τούτοις πάλιν πόλεμος συγκεκρότηται καὶ ὑπερις-
χύουσιν Ἰνδοὶ τῆς τοῦ Ἀλεξάνδρου φάλαγγος. 



Historia Alexandri Magni, Recensio ε 
Chapter 36, section 6, line 16

                                                     ἐκεῖ πλεῖστα 
Ἀλέξανδρος ἐμαχήσατο, καὶ μόλις τὸ Ἰνδικὸν ἠδυνήθη ἀπο-
στρέψαι στῖφος, ὥστε καὶ παρ' Ἰνδῶν κυκλωθέντα μόλις ἡ 
πρόνοια τῷ Βουκεφάλῳ ἵππῳ ἐποχούμενον διεσώσατο. 



Historia Alexandri Magni, Recensio ε 
Chapter 37, section 1, line 5

        εἶπεν οὖν αὐτὸν καὶ ὑπὲρ αὐτῶν πάντως ἑτοίμως ἔχειν   
ἀποθανεῖν ὡς μονομαχήσοντα ἐν λόγῳ τοιῷδε· ἐπὰν τὸ Ἰνδι-
κὸν ἐπελθὸν θεασώμεθα στρατόπεδον, Πώρῳ τάδε πρὸ τῆς 
συμβολῆς φθέγξομαι, τούτῳ ἐμφανισθεὶς πρὸς μονομαχίαν· εἰ 
μὲν ἐπελθὼν πατάξει με, ἔστε ὑμεῖς αὐτῷ εἰς δουλείαν, εἰ δὲ 
καὶ παρ' ἐμοῦ τεθνήξεται, ἔσονται ἡμῖν καὶ Ἰνδοὶ εἰς δουλείαν. 



Historia Alexandri Magni, Recensio ε 
Chapter 37, section 2, line 4

                     ἡ δὲ Ἀλεξάνδρου παράταξις τοιαύτη ἦν· 
ῥομβοειδὴς καὶ φοβερὰ ὡς οὐπώποτε ἐνεφαίνετο ὥστε καὶ λίαν 
ἐπ' αὐτῇ δυσχεραίνειν Ἰνδοὺς καί· τίς γε τοιαύτης, εἰπεῖν, περι-
γενέσθαι δυνήσεται φάλαγγος; 



Historia Alexandri Magni, Recensio ε 
Chapter 37, section 2, line 6

                                 – καὶ δὴ πλησίον γενομένων, 
Ἀλέξανδρος ἐξελθὼν Ἰνδοῖς ἐξεφώνησεν· εἴπατε Πώρῳ 
ἐξελθόντι μονομαχῆσαι μετ' ἐμοῦ. 



Historia Alexandri Magni, Recensio ε 
Chapter 37, section 2, line 12

                                   – ὁ δὲ λόγος ἔδοξεν ἀρέσαι 
ἐνώπιον Ἰνδῶν καὶ Πώρου τοῦ βασιλέως. 



Historia Alexandri Magni, Recensio ε 
Chapter 41, section 1, line 2

Ταῦτα ἀκούσας ἐξῆλθε παραλαβὼν τὸν Κανδαύλην καὶ 
χιλιάδας ἑκατὸν ἀπὸ τῶν Περσῶν καὶ Ἰνδῶν καὶ Μακεδόνων 
καὶ ὥρμησεν κατὰ τοῦ Εὐαγρίδου. 



Historia Alexandri Magni, Recensio ε 
Chapter 45, section 1, line 13

                         ἀνθυποστραφεὶς δὲ Πῶρον ἀπέκτεινα 
τὸν τῶν Ἰνδῶν βασιλέα, καὶ μέχρι τῆς ἐσχάτου διώδευσα θα-
λάσσης ἔνθα Ἡλίου ὀνομάζεται χώρα. 



Historia Alexandri Magni, Recensio ε 
Chapter 46, section 1, line 3

Σέλευκον μὲν Πέρσαις καὶ Ἰνδοῖς κυριεύειν, Φίλιππον δὲ Αἰ-
γύπτου ἄρχειν, Ἀντίοχον δὲ τὴν μέσην γαῖαν τῶν Ἀσσυρίων, 
Φίλωνα δὲ {τὸν καὶ Βύζαντα} τὴν Μακεδονικὴν διέπειν ἀρχήν,   
τὰ περὶ Ὀλυμπιάδος πλεῖστα παραινέσας αὐτῷ. 



Historia Alexandri Magni, Recensio λ (lib. 3) (1386: 007)
“Die Rezension λ des Pseudo–Kallisthenes”, Ed. van Thiel, H.
Bonn: Habelt, 1959.
Page 37, line 3

Τούτου δὲ γενομένου ὑπέστρεψεν Ἀλέξανδρος καὶ ἤρξατο τῆς 
ὁδοῦ τῆς φερούσης εἰς τὴν Πρασιακὴν πόλιν, ἥτις δοκεῖ 
εἶναι μητρόπολις τῆς <Ἰνδικῆς> χώρας. 



Historia Alexandri Magni, Recensio λ (lib. 3) 
Page 39, line 1

                                         ὁ δὲ γέρων λέγει· “τὸ 
μὲν [ἓν] δένδρον ὁ ἥλιος ἄρχεται βαρβαρικῶς τῷ λόγῳ, ἐν δὲ 
τῷ τέλει Ἑλληνικῶς πληροῖ, τὸ δὲ [ἕτερον] δένδρον ἡ σελήνη   
ἄρχεται μὲν Ἑλληνικῶς, τελεῖται δὲ Ἰνδικῶς. 



Historia Alexandri Magni, Recensio λ (lib. 3) 
Page 48, line 13

                                       ὁ Περσολέτης, ὁ Ἰνδο-
λέτης, ὁ καθελὼν τρόπαια μεγάλων βασιλέων Δαρείου καὶ 
Πώρου, Πάρθων καὶ Μήδων, καὶ ὅλην τὴν ἀνατολὴν καταλαβὼν 
νῦν δὲ χωρὶς πολέμου καὶ στρατῶν ὑποχείριος γέγονας μιᾶς 
γυναικός. 



Historia Alexandri Magni, Recensio λ (lib. 3) 
Page 57, line 15

                                                                   ὁ 
δὲ Πτολεμαῖος εἶπεν· “οὐχ οὕτως ἁρμόζει <βασιλεῦ>, ἀλλὰ 
πορευσώμεθα μετὰ τῶν στρατοπέδων [καὶ] τὰ τῶν Ἰνδῶν βασί-
λεια ἀφανῆσαι. 



Historia Alexandri Magni, Recensio λ (lib. 3) 
Page 60, line 8

                         μᾶλλον δὲ φρενήρης ὤν, ἔξελθε πρὸς 
τοὺς υἱούς μου καὶ εἰρήνευσον αὐτούς, διότι ὁ υἱός μου ὁ 
πρεσβύτερος ὀργίζεται κατά σου, διότι ἀνεῖλες Πῶρον τὸν 
βασιλέα Ἰνδῶν. 



Historia Alexandri Magni, Recensio β (e cod. Paris. gr. 1685 et cod. Messinensi 62) (1386: 011)
“Der griechische Alexanderroman. Rezension β”, Ed. Bergson, L.
Stockholm: Almqvist \& Wiksell, 1965.
Book 3, section 29, line 41

εὗρον δὲ ἐκεῖ καὶ Κανδαύλην τὸν υἱὸν Κανδάκης τῆς βασιλίσσης Ἰνδικῆς χώ-
ρας σὺν τῇ αὐτοῦ γυναικὶ ἐν οἴκῳ ἠσφαλισμένους. 



Historia Alexandri Magni, Recensio F (cod. Flor. Laurentianus Ashburn 1444) (1386: 012)
“Ps.–Kallisthenes. Zwei mittelgriechische Prosa–Fassungen des Alexanderromans, 2 vols.”, Ed. Lolos, A., Konstantinopulos, V.L.
Meisenheim am Glan: Hain, 1983; Beiträge zur klassischen Philologie 141 \& 150.
Chapter 52, section 6, line 5

Καὶ εἶπεν· “Ἔλα, Ἀλέξανδρε, μὲ ἀγάπην καὶ προσκύνει τὸ θεὸν τὸ Σα-
βαὼθ καὶ πάντοτε καὶ νύκταν καὶ ἡμέραν παρακάλει τὸν θεὸν εἰς 
βοήθειάν σου καὶ τὴν δύναμιν ὅλην τῆς Περσίας θέλεις χαλάσειν· 
καὶ ἐσὺ ὑπηγαίνεις εἰς τὴν Αἴγυπτον καὶ θέλεις τὴν ἐπάρειν· καὶ 
τὸν Πῶρον τῆς Ἰνδίας θέλεις σκοτώσειν· καὶ εἰς ἀσθένειαν με-
γάλην θέλεις πέσειν. 



Historia Alexandri Magni, Recensio F (cod. Flor. Laurentianus Ashburn 1444) 
Chapter 65, section 3, line 5

                              Καὶ ὡς ἤκουσεν ἡ στρατία τῆς Ἴνδιας 
τοὺς λόγους, ἐδειλίασεν πολλὰ καὶ ὡσὰν καταστανέον εἰς τὸν πό-
λεμον ἔρχουνται. 



Historia Alexandri Magni, Recensio F (cod. Flor. Laurentianus Ashburn 1444) 
Chapter 66, section 1, line 2

Τόμου ἐμαζώκθησαν τὰ δύο φουσάτα, ὁ ἥλιος ἀπὸ τὸν κορ-
νιακτὸν ἐσκοτίσθην· φόβος εἰς τὸ φουσάτον τῆς Ἴνδιας καὶ εἰς 
τοὺς Μακεδόνας ἐσέβην. 



Historia Alexandri Magni, Recensio F (cod. Flor. Laurentianus Ashburn 1444) 
Chapter 66, section 2, line 2

   Τῆς 
Ἴνδιας τὰ φουσάτα καὶ τῶν Μακεδόνων καὶ τῆς Περσίας ἐσέβην φό-
βος μέγας καὶ αὐτὴν τὴ νύκταν ἀρχέρισαν οἱ Πέρσοι νὰ φεύγουν· 
ἰδόντας τῆς Ἴνδιας τὰ φουσάτα ἔφυγαν καὶ αὐτοὶ καὶ ἐποίησαν 
πλέον ἐντροπήν. 



Historia Alexandri Magni, Recensio F (cod. Flor. Laurentianus Ashburn 1444) 
Chapter 66, section 3, line 1

   Τῆς Ἴνδιας οἱ στρατηγοὶ   
ἰδόντας τὸν Ἀλέξανδρον καὶ οἱ Πέρσοι ὅτι ἔφθασεν ἀτός του ὁ 
βασιλεὺς ἔδωκαν νὰ φεύγουν. 



Historia Alexandri Magni, Recensio F (cod. Flor. Laurentianus Ashburn 1444) 
Chapter 86, section 3, line 2

Καὶ ἀπ' αὐτοῦ ἐσηκώθην καὶ ἐπερπάτησεν ἕξι ἡμέρες καὶ 
ἦλθεν εἰς τὸ σύνορον τῆς Ἰνδίας. 



Historia Alexandri Magni, Recensio F (cod. Flor. Laurentianus Ashburn 1444) 
Chapter 87, section t, line 2

Περὶ ὅταν ἦλθεν ὁ Ἀλέξανδρος 
εἰς τὸν τόπον τῆς Ἰνδίας. 




Historia Alexandri Magni, Recensio F (cod. Flor. Laurentianus Ashburn 1444) 
Chapter 89, section 1, line 21

                                  Καὶ ὅσον ἔλθεις μὲ πολὺ πλῆθος 
φουσάτον τόσον ἐγὼ μὲ πολλὴν ἀνδρείαν σὲ θέλω τζακίσειν καὶ 
οὐδὲ εἰς τὸν κόσμον ὅλον σὲ θέλω διώξειν, μόνον εἰς τὴν 
Ἴνδιαν σὲ θέλω φάειν. 



Historia Alexandri Magni, Recensio F (cod. Flor. Laurentianus Ashburn 1444) 
Chapter 91, section 3, line 6

  Ὦ ἠγαπημένοι μου καὶ ἀνδρειωμένοι μου Μακεδόνες, ἀ-
πὸ ὅλες τὲς γλῶσσες ἐνδοξότατοι καὶ τιμημένοι καὶ πρωτοκαβαλ-
λαραῖοι χρυσοπτερνιστηράτοι, τὸ πῶς ὅλον τὸν κόσμον ἐπαραλάβε-
τε καὶ ὅλην τὴν οἰκουμένην ἐπατάξετε μὲ τὴν δεξιά σας χεῖρα, 
καὶ τὴν σήμερον ἡμέραν ἐσκιάσθητε ἀπὸ τοὺς ἀνάνδρους καὶ σκια-
σταραίους καὶ οὐδετιποτέϊνους τοὺς Ἰνδιῶτες; 



Historia Alexandri Magni, Recensio F (cod. Flor. Laurentianus Ashburn 1444) 
Chapter 92, section 1, line 7

                                                                  Νὰ ἠ-
ξεύρης ὅτι ὅλον τὸν κόσμον ἐπαράλαβα καὶ ἐξεβήκαμεν γεροὶ καὶ 
θέλομε νὰ πολεμήσωμεν μὲ τὸν βασιλέα τῆς Ἴνδιας τὸν Πῶρον. 



Historia Alexandri Magni, Recensio F (cod. Flor. Laurentianus Ashburn 1444) 
Chapter 92, section 1, line 9

Καὶ τόμου ἰδεῖς τὴν γραφή μου, αὐτὴν τὴν ὥρα μὲ ὅλα τὰ φουσά-
τα τῆς Δύσης νὰ φθάσης εἰς τὴν Ἴνδιαν. 



Historia Alexandri Magni, Recensio F (cod. Flor. Laurentianus Ashburn 1444) 
Chapter 93, section 2, line 3

Καὶ ὡς εἶδεν ὁ Πῶρος, ἦλθεν εἰς τὸ φουσάτον του καὶ 
ἔκραξεν ὅλους, ὁποὺ εἶχεν, καὶ εἶπεν τους· “Ὦ ἀνδρειωμένοι 
καὶ μεγιστάνοι στρατιῶτες τῆς Ἴνδιας, μὲ τοὺς Μακεδόνους ἐ-
πολεμήσαμεν καὶ ἐσκοτώθησαν πολὺ πλῆθος ἀπὸ ἐμᾶς καὶ ζημία με-
γάλην ἐγίνη. 



Historia Alexandri Magni, Recensio F (cod. Flor. Laurentianus Ashburn 1444) 
Chapter 94, section 2, line 11

                                                          Καὶ αὐτοῦ 
στέκει ὁ Σελεύκιος ὁ Φιλόνης καὶ εἶπε τοῦ Ἀλεξάνδρου· “Ἠγα-
πημένε καὶ ἐνκαρδιακέ μου ἀφέντη καὶ ὁλονοῦ τοῦ κόσμου βασι-
λεύς, βασιλέα ὑψηλότατε καὶ μέγα χονκιάρη καὶ κάστρον τῆς οἰ-
κουμένης ὁλονῆς, Ἀλέξανδρε βασιλέα, οὐδὲν πρέπει ἐσέναν ἀν-
τίκρυτα νὰ ἐστέκεσαι μὲ τῆς Ἴνδιας τὸν Πῶρον· καὶ μηδὲν ἀρ-
γῆς, ἀμὴ ἔτζι σύρε καταπάνου τους ἐγλήγορα, ὅτι τὸ φουσάτο μας ἔ-  
ναι πολλὰ δυνατόν, ὁποὺ δὲν εἶδαν τοὺς πολέμους σας καὶ θέλο-
με τοὺς τζακίσειν”. 



Historia Alexandri Magni, Recensio F (cod. Flor. Laurentianus Ashburn 1444) 
Chapter 94, section 3, line 3

                                                Καὶ οἱ Ἰνδιῶτες ὡς 
ἤκουσαν, φόβος καὶ τρόμος τοὺς ὑπόπεσεν πολλά. 



Historia Alexandri Magni, Recensio F (cod. Flor. Laurentianus Ashburn 1444) 
Chapter 95, section 1, line 12

         Καὶ πρέπει ἐγὼ νὰ πολεμήσω, ὁποὺ κρατῶ τὸ βασίλειον 
τῆς Περσίας καὶ εἶμαι μὲ τὸ ῥιζικόν σου αὐθέντης τῆς Περσίας, 
καὶ ὁ Πῶρος ἔναι τῆς Ἴνδιας”. 



Historia Alexandri Magni, Recensio F (cod. Flor. Laurentianus Ashburn 1444) 
Chapter 96, section 2, line 3

Καὶ ὁ Ἀλέξανδρος ἔπεσεν εἰς τὸ πέσιμον τοῦ Πώρου μὲ 
τὸ φουσάτον του καὶ ἀπόστειλεν κουρσατόρους καὶ ἐκούρσεψεν 
τὸν τόπον τῆς Ἴνδιας. 



Historia Alexandri Magni, Recensio F (cod. Flor. Laurentianus Ashburn 1444) 
Chapter 98, section 1, line 9

                                               Ἀμὴ ἐσὺ ἔχεις μανί-
αν καὶ κάκητα ὁληνῆς τῆς Ἴνδιας· δείχνεσαι ἀτός σου ἀπὸ τὸ   
ἔργον σου. 



Historia Alexandri Magni, Recensio F (cod. Flor. Laurentianus Ashburn 1444) 
Chapter 98, section 1, line 11

                                                                       Καὶ 
ἐὰν τοὺς στρατιῶτες τῆς Ἴνδιας οὐδὲν ποθῆς καὶ ἀγαπᾶς, ἐγὼ 
πολλὰ πονῶ καὶ θέλω τοὺς Μακεδόνας. 



Historia Alexandri Magni, Recensio F (cod. Flor. Laurentianus Ashburn 1444) 
Chapter 98, section 1, line 18

                                                             Καὶ ἂν σκο-
τώσης ἐσὺ ἐμένα, νὰ εἶσαι νοικοκύρης εἰς ὅλον μου τὸ φουσάτον, 
νὰ ἔναι δικό σου· εἰ δὲ σκοτώσω ἐσέναν, ἐγὼ αὐτὸ μὲ τὸ δί-
καιο <νὰ εἶμαι> αὐθέντης ὁληνῆς τῆς Ἴνδιας. 



Historia Alexandri Magni, Recensio F (cod. Flor. Laurentianus Ashburn 1444) 
Chapter 99, section 1, line 4

                                      Καὶ ἐχάρην ὅλος ὁ λαὸς 
τῆς Ἴνδιας καὶ ἐπαραπάρθη πολλὰ ὁ Πῶρος καὶ ἐχάρη ἡ Οὑλιού-
πονλη, καὶ εἶπεν· “Ἐσὺ εἶσαι ὁ Φιλόνης, τῆς Περσίας ἀφέντης; 



Historia Alexandri Magni, Recensio F (cod. Flor. Laurentianus Ashburn 1444) 
Chapter 99, section 1, line 13

                       Καὶ ἂς εἶσαι ἐδικός μου καὶ ἀφέντευε τὴν 
Περσίαν καὶ ἀπὸ τὴν Ἴνδιαν νὰ σοῦ δίδω τὸ τέταρτον μερτικόν”. 



Historia Alexandri Magni, Recensio F (cod. Flor. Laurentianus Ashburn 1444) 
Chapter 101, section 1, line 2

  Θεέ μου ὑψηλότατε, ὁποὺ ἀναπεύεσαι εἰς τοὺς ἁγίους, 
γενοῦ βοηθός μου τὴν σήμερον ἡμέραν εἰς τὸν Πῶρον τῆς Ἴνδιας. 



Historia Alexandri Magni, Recensio F (cod. Flor. Laurentianus Ashburn 1444) 
Chapter 102, section 1, line 14

                               Καὶ τῆς Ἴνδιας ὡς εἶδαν τὰ φουσάτα, 
ἔδωκαν τὸ φυγίο νὰ φεύγουν. 



Historia Alexandri Magni, Recensio F (cod. Flor. Laurentianus Ashburn 1444) 
Chapter 102, section 2, line 3

Ἡ βασίλισσα ἡ Κλετεμήτρα τὰ μαλλία της ἀπόλυκεν ἕως 
τὴν γῆν καὶ τὸ πολυτίμητον φόρεμα ἔσκισεν καὶ μὲ δέκα χιλιά-
δες ἀρχόντισσες τῆς Ἴνδιας ἦλθεν καὶ ἐσυναπάντησε μὲ κλαυθ-
μὸν καὶ μὲ δαρμὸν πολὺν τὸ κορμὶ τοῦ Πώρου, καὶ μὲ θλῖψιν με-
γάλην καὶ θρῆνον ἐσυναπάντησαν τὸν βασιλέα τὸν Πῶρον. 



Historia Alexandri Magni, Recensio F (cod. Flor. Laurentianus Ashburn 1444) 
Chapter 103, section 2, line 16

          Καὶ ἤφεραν τοῦ Ἀλεξάνδρου ἑκατὸ χιλιάδες φαρία τοῦ 
στάβλου τῆς Ἴνδιας ὅλο ἀρματωμένα μὲ κουβέρτες ἀπὸ σύρμα. 



Historia Alexandri Magni, Recensio F (cod. Flor. Laurentianus Ashburn 1444) 
Chapter 103, section 2, line 29

      Ἀλέξανδρος τὸν ἐδικόν του τὸν βοεβόντα τὸν Ἀντίοχον ἐ-
ποίησεν αὐθέντη τῆς Ἴνδιας ὅλης. 



Historia Alexandri Magni, Recensio F (cod. Flor. Laurentianus Ashburn 1444) 
Chapter 111, section 1, line 3

     Καὶ ὅρισεν ὁ Ἀντίοχος καὶ ἐδώρισαν τοῦ Κανταυλούσην ἀπα-
νωφόριν μακεδονίτικον πολυτίμητον καὶ φαρὶν ἀπὸ τὴν Ἴνδια ἀ-
ποκάτου κουβερτιασμένον ἀπὸ τὸν κορκόνδειλον. 



Historia Alexandri Magni, Recensio F (cod. Flor. Laurentianus Ashburn 1444) 
Chapter 113, section 3, line 6

           Καὶ αὐτὸς τὸν εἶπεν· “Ἐγὼ ἤμουν ὁ Σωνσόχος ὁ βασι-
λεὺς τῆς Ἴνδιας, ὁποὺ οὑκάποτες ἠπῆρα τὸν κόσμον ὅλον, καὶ 
ἀπὸ τὴν παραφρασίαν μου τὴν πολλὴν πολλὰ ὑψώθηκα καὶ εἰς τὴν 
Ἀνατολὴν εἰς τὴν ἄκραν τῆς γῆς ἤθελα νὰ ὑπηγαίνω. 



Historia Alexandri Magni, Recensio F (cod. Flor. Laurentianus Ashburn 1444) 
Chapter 114, section 1, line 18

                                    Καὶ ὁ Τάρειος τὸν εἶπεν· “Σύρε 
παραπίσω· αὐτοῦ θέλεις ἰδεῖν τὸν Πῶρον τῆς Ἴνδιας τὸν βασι-
λέαν”. 



Historia Alexandri Magni, Recensio F (cod. Flor. Laurentianus Ashburn 1444) 
Chapter 115, section 1, line 2

Ὁ Ἀλέξανδρος τὸν βασιλέαν τὸν Πῶρον εἶδεν καὶ εἶπεν του 
“Ὦ μεγαλειότατε Πῶρε τῆς Ἴνδιας, μίαν βολὰν ἴσα μὲ τὸν θεὸν 
ὀνομάζεσουν, καὶ τώρα ὡσὰν ἕνας ἄτυχος ἄνθρωπος κολάζεσαι ἐ-
δῶ”. 



Historia Alexandri Magni, Recensio F (cod. Flor. Laurentianus Ashburn 1444) 
Chapter 117, section 4, line 25

                                                    Ὁ Δορυφόρος τὴν μη-
τέρα του οὐδὲν ἤκουσεν, ἀμὴ ἔλεγεν· “Ἔναν ἄνθρωπον τοῦ Ἀλε-
ξάνδρου οὐδὲν μὲ ἀφήνετε νὰ σκοτώσω, ὅτι αὐτὸς πολλὲς χιλιάδες 
ἐδικές μας λαὸν ἐσκότωσε ἀνθρώπους, καὶ τὸν πεθερόν μου τὸν 
Πῶρον τῆς Ἴνδιας ἐσκότωσεν. 



Historia Alexandri Magni, Recensio F (cod. Flor. Laurentianus Ashburn 1444) 
Chapter 117, section 5, line 22

                     Καὶ ὁ αὐθέντης του τὸν λωλὸν τὸν πεθερόν σου, 
τὸν Πῶρον, μὲ ἕνα σπαθὶν τὸν ἔσφαξεν τῆς Ἴνδιας τὸν βασιλέαν”. 



Historia Alexandri Magni, Recensio F (cod. Flor. Laurentianus Ashburn 1444) 
Chapter 119, section 1, line 6

                                                            Ἀρχὴ τὸν 
Ἀντίοχον ἔδωκεν τὴν Ἴνδιαν ὅλη νὰ ἀφεντεύη ὅρισεν. 



Historia Alexandri Magni, Recensio F (cod. Flor. Laurentianus Ashburn 1444) 
Chapter 124, section 3, line 3

Καὶ αὐτοῦ ὁ Ἀλέξανδρος τὸν Ἀριστοτέλην τὸν ἐδικόν 
του ἐδώρισεν ἔναν στέμμα καὶ ἀπανωφόριν μέγαν τοῦ Πώρου τοῦ βα-
σιλέως τῆς Ἴνδιας καὶ δέκα χιλιάδες χρυσὰ τάλαντα καὶ τρεῖς 
χιλιάδες μόδια μαργαριτάριν. 



Historia Alexandri Magni, Recensio F (cod. Flor. Laurentianus Ashburn 1444) 
Chapter 125, section 5, line 8

        Καὶ αὐτοῦ ἀρχίρισεν ὁ Ἀλέξανδρος νὰ δείχνη τοὺς πολέ-
μους τοὺς μεγάλους, ὀποὺ εἶχεν ποιήσειν μὲ τὸν Τάρειον τῆς 
Περσίας τὸν βασιλέαν καὶ μὲ τὸν Πῶρον τῆς Ἴνδιας τὸν βασιλέ-
αν καὶ μὲ τοὺς ἄλλους βασιλεῖς τῆς Ἀνατολῆς καὶ τῆς Δύσης. 



Historia Alexandri Magni, Recensio F (cod. Flor. Laurentianus Ashburn 1444) 
Chapter 125, section 11, line 12

                                                                  Ὅ-  
λοι οἱ Πέρσηδες ἐθαύμασαν τὴν φρόνεσή του καὶ τὰ ποιήματά 
του, καὶ τῆς Ἴνδιας τὰ φουσάτα καὶ ὅλος ὁ κόσμος ἀποκάτου 
τὸν ἥλιον ἐτρόμαξεν, ἡ Ἀνατολὴ καὶ ἡ Δύσις, Ἄρ<κ>τος καὶ Με-
σημβρία. 



Historia Alexandri Magni, Recensio E (cod. Eton College 163) (1386: 013)
“Ps.–Kallisthenes. Zwei mittelgriechische Prosa–Fassungen des Alexanderromans, 2 vols.”, Ed. Lolos, A., Konstantinopulos, V.L.
Meisenheim am Glan: Hain, 1983; Beiträge zur klassischen Philologie 141 \& 150.
Chapter 52, section 6, line 5

    Ἔλα, Ἀλέξανδρε, μὲ ἀγάπην καὶ 
προσκύνησον τὸν θεὸν τὸν Σαβαὼθ καὶ πάντοτε νύκταν καὶ ἡμέραν 
παρακάλειέ τον εἰς τὴν βοήθειάν σου καὶ τὴν δύναμιν τῆς Περσί-
ας θέλει χαλάσει· καὶ ἐσὺ πήγαινε εἰς τὴν Αἴγυπτον, θέλεις τὴν 
παραλάβει· καὶ τὸν Πῶρον τῆς Ἰ<ν>δίας θέλεις σκοτώσει· καὶ αὐ-
θέντης μέγας θέλεις γίνει. 



Historia Alexandri Magni, Recensio E (cod. Eton College 163) 
Chapter 59, section 1, line 5

Ὁ Ἀλέξανδρος αὐτὴν τὴν νύκταν ἐφάνην του εἰς τὸν ὕ-
πνον του ὁ προφήτης Ἱερεμίας μὲ τὴν ἀλλαγὴν τῶν ἀρχιερέων, ὅ-
ταν ἦτον εἰς τὴν Ἱερουσαλὴμ εἰς τὴν ἁγίαν Σιὼν καὶ εἶπε του· 
“Ἄνθρωπέ μου Ἀλέξανδρε, γίνου ἀτός σου ἀποκρισάρης καὶ τὸν 
βασιλέα τὸν Τάρειον καταπάτησε καὶ ἰδὲς τῆς Ἰνδίας τὰ φουσάτα, 
ὁποὺ φέρνει καταπάνω σου, καὶ νὰ ἰδῆς ἀτός σου τὰ φουσάτα μὲ 
τὴν καρδίαν σου· καὶ ἂν σὲ ἐγνωρίσουν ἐκεῖ εἰς τὸν Τάρειον, ὁ 
θεὸς ὁ παντοκράτωρ Σαβαὼθ νὰ σὲ ἐβγάλη καλὰ ἀπ' ἐκεῖ· ἀμὴ οὐδὲν 
ἔχης ἔγνοιαν. 



Historia Alexandri Magni, Recensio E (cod. Eton College 163) 
Chapter 64, section t, line 2

Περὶ ὅταν ἔστειλεν ὁ Τάρειος εἰς τὸν Πῶρον 
τὸν βασιλέα τῆς Ἰνδίας <ἐπιστολήν>. 




Historia Alexandri Magni, Recensio E (cod. Eton College 163) 
Chapter 64, section 1, line 5

                                Εἰς τριάντα ἕξι βασίλεια ἤμουν καὶ ἐ-
γὼ ὁκάποτες, ὡσὰν τὸ ἤκουσεν ἡ βασιλεία σου, καὶ ἐσὺ εἶσαι τῆς 
Ἰνδίας ὁ θεὸς καὶ ἡ δεξιά σου ἡ στερεὰ ἔνι εἰς ὅλα τὰ βασί-
λεια τοῦ κόσμου ὁλουνοῦ, τῆς Ἰνδίας ὑψηλότατε βασιλέα. 



Historia Alexandri Magni, Recensio E (cod. Eton College 163) 
Chapter 64, section 4, line 2

   Πάντοτε ἔνι τιμημένον τῆς Ἰν-
δίας τὸ φουσάτον, ὑψηλότατε, καὶ ἐσὺ ὅμοιος τὸν θεὸν καὶ ἐλε-
ημόνησε τὴν παρακάλεσίν μου καὶ στεῖλε μου φουσάτον καὶ ἐλευθέ-
ρωσέ με ἀπὸ τοὺς λωλοὺς καὶ ἀνελεήμονας Μακεδόνας καὶ οὐδὲν 
πρέπει νὰ κακοπαθῶ, ὁποὺ ἔνι ἡ δύναμίς σου ἡ φρικτή, ὁποὺ ἔχεις. 



Historia Alexandri Magni, Recensio E (cod. Eton College 163) 
Chapter 64, section 7, line 1

Ὡς ἤκουσεν ὁ Τάρειος ὅτι ἦλθαν τὰ φουσάτα τῆς Ἰνδίας 
ἦλθεν ἀπὸ τὴν μεγάλην θλῖψιν εἰς χαράν. 



Historia Alexandri Magni, Recensio E (cod. Eton College 163) 
Chapter 64, section 7, line 5

                                                 Καὶ ὅρισαν τοὺς Πέρσας 
καὶ ἐμαζώχθησαν καὶ ηὗρε τὸ φουσάτον του δέκα φορὲς ἀπὸ ἑκατὸ 
χιλιάδες· καὶ ἔτζι ἐκίνησεν πρὸς τὸν Ἀλέξανδρον μὲ ὅλους τοὺς 
βοηβοντάδες τῆς Ἰνδίας. 



Historia Alexandri Magni, Recensio E (cod. Eton College 163) 
Chapter 65, section 2, line 5

                                                      Καὶ οἱ βοηβοντάδες 
τῆς Ἰνδίας τοὺς ἐρωτῆσαν περὶ τὸ φουσάτον τοῦ Ἀλεξάνδρου καὶ 
αὐτοὶ τοὺς ἀπεκρίθησαν 
         “Ἡμεῖς ὅσον ἠδυνήθημεν καὶ ἐγρυκή-
σαμεν νὰ σᾶς ὁμολογήσωμεν· εἴδαμεν φουσάτον πολὺν καὶ ἀνδρειωμένον καὶ 
δυνατὰ ἔρχονται καταπάνω σας· ἔγνοιαν καμμίαν οὐδὲν ἔχουν. 



Historia Alexandri Magni, Recensio E (cod. Eton College 163) 
Chapter 65, section 3, line 5

     Καὶ ὡς ἤκουσαν οἱ στρατιῶτες τῆς Ἰνδίας, ἐδειλίασαν πολ-
λὰ καὶ ὡσὰν ἀποστανέον εἰς τὸν πόλεμον ἔρχουντα. 



Historia Alexandri Magni, Recensio E (cod. Eton College 163) 
Chapter 66, section 1, line 2

Τόμου ἐμαζώχθησαν τὰ δύο φουσάτα, ὁ ἥλιος ἀπὸ τὸν κο-
νιαρτὸν ἐσκοτίσθην· φόβος εἰς τῆς Ἰνδίας τὰ φουσάτα καὶ εἰς τῆς 
Μακεδονίας ἐσέβη. 



Historia Alexandri Magni, Recensio E (cod. Eton College 163) 
Chapter 66, section 2, line 1

   Εἰς τῆς Ἰνδίας 
τὰ φουσάτα καὶ εἰς τῆς Μακεδονίας {ἐσέβη} καὶ εἰς τῆς Περσίας 
ἐσέβη μέγας φόβος ἀπὸ τὸ δὸς καὶ λάβε· καὶ αὐτὴν τὴν ἡμέραν ἀρ-
χίνισαν νὰ φεύγουν. 



Historia Alexandri Magni, Recensio E (cod. Eton College 163) 
Chapter 66, section 3, line 1

   Τῆς Ἰνδίας οἱ στρατηγοὶ ἰδόντες   
τὸν Ἀλέξανδρον καὶ οἱ Πέρσηδες ὅτι ἔφθασεν ἀτός του ἔδωκαν 
καὶ φεύγουν. 



Historia Alexandri Magni, Recensio E (cod. Eton College 163) 
Chapter 66, section 6, line 3

Ὁ Ἀλέξανδρος ἔκραξεν ἕναν ἀπὸ τοὺς ἰδικούς του τοὺς 
μεγιστάνους, τὸ ὄνομά του Φοῖνιξ, καὶ εἶπε τον· “Σύρε εἰς τῆς 
Ἰνδίας τὰ φουσάτα καὶ ἕως τῆς Περσίας καὶ εἴπετέ τους· “Νὰ ἠ-
ξεύρετε ὅτι ὁ βασιλεὺς ὁ ἰδικός σας ἐσκοτώθη· καὶ στέκεστε <καὶ> 
μηδὲν φεύγετε! Εἰ δὲ ἀρχινίσετε νὰ φεύγετε, τὴν σήμερον ἡμέραν 
θέλετε ἀποθάνει ἐκ τὸ σπαθίν μου. 



Historia Alexandri Magni, Recensio E (cod. Eton College 163) 
Chapter 66, section 7, line 2

   Τὸν Σελευκούσην ἔστειλεν 
εἰς τῆς Ἰνδίας τὰ φουσάτα νὰ ἐπάρη τὰ ἄλογα καὶ τὰ ἄρματά τους   
καὶ “αὐτουνοὺς ζωντανοὺς νὰ τοὺς στείλετε εἰς τὸν βασιλέαν τους. 



Historia Alexandri Magni, Recensio E (cod. Eton College 163) 
Chapter 66, section 7, line 6

Ὅλα τὰ φλάμπουρα τοῦ Πώρου τοῦ βασιλέως τῆς Ἰνδίας καὶ οἱ τρουμπέ-
τες οἱ μεγάλες διακοσίες, ὁποὺ ἦσαν, ἔδωκαν <νὰ φεύγουν> καὶ οἱ ἄλλες 
τόσες ἀνακαράδες τὰ ἄλογά τους ὅλα καὶ τὰ ἄρματά τους ἐπαρέδω-
καν τοῦ Φιλόνη· καὶ ἐπῆραν συμπάθιον καὶ ὑπῆγαν πρὸς τὸν τόπον 
τους. 



Historia Alexandri Magni, Recensio E (cod. Eton College 163) 
Chapter 66, section 9, line 1

   Οἱ Περσῆδες ἐχωρίσθησαν ἀπὸ τῆς Ἰνδίας τὸ φου-
σάτον καὶ ἐζύγωσαν καὶ προσεκύνησαν τὸν Φιλόνη ἀντίτοπα τὸν Ἀ-
λέξανδρον καὶ ἐσμίχθηκαν μὲ τοὺς Μακεδόνες καὶ ἐχάρηκαν χαρὰν 
μεγάλην, τὸ πὼς τοὺς ἐπαρέδωκεν ὁ θεὸς νὰ δουλεύουν τὸν Ἀλέξαν-
δρον. 



Historia Alexandri Magni, Recensio E (cod. Eton College 163) 
Chapter 82, section 1, line 6

          Ἀμὴ σύρε ἐγλήγορα, ὅτι σὲ ἀκαρτερεῖ τὸ φουσάτον σου 
τῆς Ἰνδίας καὶ τοῦ Πώρου ἡ αὐθεντία καὶ ὁ μέγας βασιλεὺς ὁ 
Πῶρος· καὶ αὐτοῦ τὴν δύναμιν θέλεις χαλάσει παντελῶς καὶ τὸν 
θέλεις σκοτώσει. 



Historia Alexandri Magni, Recensio E (cod. Eton College 163) 
Chapter 86, section 3, line 2

Καὶ ἀπ' αὐτοῦ ἐσηκώθη καὶ ἐπερπάτησεν ἡμέρες ϛʹ καὶ 
ἦλθεν εἰς τὸ σύνορον τῆς Ἰνδίας. 



Historia Alexandri Magni, Recensio E (cod. Eton College 163) 
Chapter 87, section t, line 2

Περὶ ὅταν εἰσῆλθεν ὁ Ἀλέξανδρος 
εἰς τὸ σύνορον τῆς Ἰνδίας. 




Historia Alexandri Magni, Recensio E (cod. Eton College 163) 
Chapter 88, section 1, line 1

Καὶ ὡς ἤκουσεν ὁ Πῶρος ὅτι ἔρχεται ὁ Ἀλέξανδρος, ἔ-
στειλεν ἐπιστολάτορα μὲ ἐπιστολήν· ἔγραφεν οὕτως· 
 “Πῶρος ὁ τρανὸς καὶ ὑψηλότατος βασιλεύς, τῆς Ἰνδίας 
ὁ θεός, τὸν Ἀλέξανδρον τῆς Μακεδονίας γράφω. 



Historia Alexandri Magni, Recensio E (cod. Eton College 163) 
Chapter 89, section 1, line 22

                                                             Καὶ ὅσον θέλεις 
ἔλθει ἐσὺ μὲ πλῆθος φουσάτον, τόσον {θέλω} καὶ ἐγὼ μὲ πολλὴν 
ἀνδρείαν σὲ θέλω τζακίσει καὶ σὲ θέλω διώξει, μόνον εἰς τὴν 
Ἴνδιαν σὲ θέλω φάει. 



Historia Alexandri Magni, Recensio E (cod. Eton College 163) 
Chapter 90, section 2, line 1

Καὶ ἦλθα εἰς τὸν βασιλέα τῆς Ἰνδίας ἢ νὰ σκοτώση αὐ-
τὸς ἐμένα ἢ ἐγὼ νὰ σκοτώσω αὐτὸν καὶ νὰ τοὺς παραλάβωμεν. 



Historia Alexandri Magni, Recensio E (cod. Eton College 163) 
Chapter 90, section 10, line 2

Καὶ ἐξέβηκα δεξιὰ καὶ διὰ ἕνα χρόνον ἐξέβηκα εἰς τὴν 
οἰκουμένην, εἰς τῆς Ἰνδίας τὸν Πῶρον τὸν βασιλέα, καὶ θέλομεν 
νὰ πολεμήσωμεν μετ' αὐτόν, διότι νὰ γένη τὸ θέλημα τοῦ θεοῦ 
γληγορότερα. 



Historia Alexandri Magni, Recensio E (cod. Eton College 163) 
Chapter 91, section 1, line 1

Ὁ Πῶρος, ὁ βασιλεὺς τῆς Ἰνδίας, ὅλον τὸν κόσμον ἐσύ-
ναξε καὶ ἄλλα βασίλεια, φουσάτα πολλὰ ἀναρίθμητα, καὶ ἔγραψε 
τὰ φουσάτα του καὶ ηὗρε ἐννενήντα φορὲς ἀπὸ χίλιες χιλιάδες 
καὶ εἶχεν καὶ δέκα χιλιάδες λέοντας, ὁποὺ ἦσαν μαθημένοι εἰς 
τὸν πόλεμον. 



Historia Alexandri Magni, Recensio E (cod. Eton College 163) 
Chapter 91, section 3, line 6

  Ὦ ἠγαπημένοι καὶ ἀνδρειωμένοι Μακεδόνες, ἀπὸ ὅλον τὸ 
γένος ἐνδοξότατοι καὶ τετιμημένοι καβαλλαραῖοι χρυσοπτερνιστη-
ράτοι, ὅλον τὸν κόσμον ἐπαραλάβετε καὶ ὅλην τὴν οἰκουμένην ἐ-
πατάξατε μὲ τὴν δεξιάν σας χεῖρα, καὶ τὴν σήμερον ἡμέραν ἐσκι-
αστήκετε ἀπὸ τοὺς ἄτυχους ἄνδρας καὶ σκιασταραίους καὶ οὐδετι-
ποτέϊνους τοὺς Ἰνδιῶτες; 



Historia Alexandri Magni, Recensio E (cod. Eton College 163) 
Chapter 91, section 4, line 6

Οἱ Μακεδόνες ὡς ἤκουσαν τοὺς λόγους τοῦ Ἀλεξάνδρου, 
ἔκλαυσαν πολλὰ καὶ εἶπαν τοῦ βασιλέως· “Ὦ βασιλεῦ Ἀλέξανδρε 
καὶ μέγα αὐθέντη χρυσοκέφαλε, κάλλιον ἔνι νὰ ἀποθάνωμεν ὅλοι 
ἀντάμα παρὰ νὰ ζήσωμεν πολύ· ἀμὴ ἡ δημηγερσία ἐτούτη οὐδὲν ἔ-
νι ἀπὸ ἡμᾶς, ἀμὴ οἱ δημηγέρτες καὶ οἱ σκιασταραῖοι οἱ Πέρση-
δες καὶ <οἱ> ἄνανδροι οἱ γειτόνοι τῆς Ἰνδίας τὰ φουσάτα, ἤθελαν 
νὰ μᾶς σκιάσουν. 



Historia Alexandri Magni, Recensio E (cod. Eton College 163) 
Chapter 91, section 4, line 7

                    Ἀμὴ ἐμεῖς καλὰ σᾶς ἐγνωρίσαμεν τῆς Ἰνδίας τὰ φουσάτα, ὅ-
ταν εἶχεν στείλει εἰς τὴν Περσίαν τὰ φουσάτα του ὁ Πῶρος εἰς 
βοήθειαν τοῦ Ταρείου τοῦ βασιλέως”. 



Historia Alexandri Magni, Recensio E (cod. Eton College 163) 
Chapter 93, section 2, line 4

Καὶ ὡς εἶδεν ὁ Πῶρος τὴν ἀπώλειαν τοῦ πολέμου, ἦλθεν 
εἰς τὸ φουσάτον καὶ ἔκραξε τοὺς αὐθέντας καὶ βασιλεῖς, ὀποὺ εἶ-
χεν, καὶ εἶπεν τους· “Ὦ πολυανδρειωμένοι στρατιῶτες τῆς Ἰν-
δίας, μὲ τοὺς Μακεδόνας ἐπολεμήσαμεν καὶ ἐσκοτώθηκεν πολὺ πλῆ-
θος ἀπ' ἐμᾶς καὶ ζημία μεγάλη ἐγίγνετον. 



Historia Alexandri Magni, Recensio E (cod. Eton College 163) 
Chapter 94, section 2, line 12

                                                          Καὶ αὐτοῦ 
ἔστεκεν ὁ Σελευκούσης ὁ Φιλόνης καὶ εἶπεν τοῦ Ἀλεξάνδρου· “Ὦ 
ἠγαπημένε μου καὶ ἐγκαρδιακέ μου αὐθέντη, ὁλουνοῦ τοῦ κόσμου 
βασιλεῦ, βασιλεῦ ὑψηλότατε καὶ αὐθέντη χρυσοκέφαλε καὶ κάστρον 
τῆς οἰκουμένης ὅλης, Ἀλέξανδρε βασιλέα, οὐδὲν πρέπει ἐσὺ νὰ 
στέκεσαι ἀντίκρυς τὸν βασιλέα τῆς Ἰνδίας, τὸν Πῶρον· καὶ μηδὲν 
ἀργῆς καὶ ἔτζι σύρε ἀπάνω του ἐγλήγορα, ὅτι τὸ φουσάτον μας   
τώρα ἔνι δυνατόν, ὅτι οὐδὲν εἶδαν τὸν πόλεμον”. 



Historia Alexandri Magni, Recensio E (cod. Eton College 163) 
Chapter 94, section 3, line 4

Καὶ οἱ Ἰνδιῶτες ὡς ἤκουσαν, εἰς φόβον μέγαν καὶ εἰς τρόμον 
ἔπεσαν. 



Historia Alexandri Magni, Recensio E (cod. Eton College 163) 
Chapter 95, section 1, line 11

                         Καὶ πρέπει ἐγὼ νὰ πολεμήσω, ὁποὺ κρατῶ τὸ 
βασίλειον τῆς Περσίας, καὶ ὁ Πῶρος ἔνι τῆς Ἰνδίας”. 



Historia Alexandri Magni, Recensio E (cod. Eton College 163) 
Chapter 95, section 1, line 25

                         Καὶ τῆς Ἴνδιας τὰ φουσάτα πολλὰ ἐπολέμι-
ζαν. 



Historia Alexandri Magni, Recensio E (cod. Eton College 163) 
Chapter 96, section 2, line 3

Καὶ ὁ Ἀλέξανδρος ἔπεσε εἰς τὸ πέσιμον τοῦ Πώρου μὲ 
τὰ φουσάτα του καὶ ἀπέστειλαν κουρσάτορας καὶ ἐκούρσευαν τὸν 
τόπον τῆς Ἰνδίας. 



Historia Alexandri Magni, Recensio E (cod. Eton College 163) 
Chapter 97, section 2, line 2

Καὶ ὡς ἤκουσαν τοῦ Πώρου τὲς ἐπιστολές, ὁποὺ ἦσαν εἰς 
τοῦ Βορέως τὸ μέρος, ἦλθαν εἰς βοήθειαν τοῦ Πώρου τοῦ βασιλέως τῆς Ἰνδίας. 



Historia Alexandri Magni, Recensio E (cod. Eton College 163) 
Chapter 98, section 1, line 8

                                                                                      Ἀ-
μὴ ἐσὺ ἂν ἔχης μανίαν καὶ κάκηταν ὅλης τῆς Ἰνδίας, δείχνεσαι ἀτός σου ἀπὸ τὸ   
ἔργον σου. 



Historia Alexandri Magni, Recensio E (cod. Eton College 163) 
Chapter 98, section 1, line 10

       Καὶ ἂν ἐσὺ τοὺς στρατιώτας τῆς Ἰνδίας οὐδὲν ποθῆς καὶ 
ἀγαπᾶς, ἐγὼ ποθῶ καὶ θέλω τοὺς Μακεδόνας. 



Historia Alexandri Magni, Recensio E (cod. Eton College 163) 
Chapter 98, section 1, line 16

                                                                Εἰ δὲ ἐγὼ 
σκοτώσω πάλι ἐσένα, νὰ εἶμαι αὐθέντης ὅλης τῆς Ἰνδίας. 



Historia Alexandri Magni, Recensio E (cod. Eton College 163) 
Chapter 99, section 1, line 4

                                   Καὶ ἐχάρηκεν ὁ λαὸς ὅλος τῆς Ἰν-
δίας καὶ ἐπαρεπάρθην πολλὰ ὁ Πῶρος καὶ ἐχάρηκε πολλὰ καὶ εἶπεν· 
“Ἐσὺ εἶσαι ὁ Φιλόνης τοῦ Ταρείου ὁ ἀντίτοπος τῆς Περσίας ὁ 
αὐθέντης; 



Historia Alexandri Magni, Recensio E (cod. Eton College 163) 
Chapter 99, section 1, line 12

                                                Καὶ ἂς εἶσαι ἰδικός 
μου καὶ αὐθέντευε τὴν Περσίαν καὶ ἀπὸ τὴν Ἰνδίαν νὰ σὲ δώσω 
τὸ τέταρτον μερδικόν”. 



Historia Alexandri Magni, Recensio E (cod. Eton College 163) 
Chapter 99, section 1, line 15

                          Καὶ ὁ Φιλόνης τὸν ἀπεκρίθην· “Ἐγὼ τὴν 
σήμερον ἡμέραν εἶμαι αὐθέντης τῆς Περσίας καὶ ἐσὺ μὲ δίδεις τὸ 
τέταρτον μερτικὸν ἀπὸ τὴν Ἰνδίαν. 



Historia Alexandri Magni, Recensio E (cod. Eton College 163) 
Chapter 99, section 1, line 16

                                          Ὁ Ἀλέξανδρος ὅλην τὴν 
Ἰνδίαν μοῦ ἔδωκεν ὅταν τὴν περιλάβη νὰ μὲ κάμη αὐθέντην τῆς 
Ἰνδίας”. 



Historia Alexandri Magni, Recensio E (cod. Eton College 163) 
Chapter 99, section 1, line 23

                 Καὶ ὁ Πῶρος τὸν ἀπεκρίθη καὶ εἶπε πάλε τοῦ Φι-
λόνη· “Γένου ἰδικός μου καὶ νὰ σοῦ δώσω τὴν θυγατέραν μου καὶ 
εἰς τὸν θάνατόν μου νὰ σὲ κάμω αὐθέντην εἰς τὴν Ἴνδιαν ὅλην”. 



Historia Alexandri Magni, Recensio E (cod. Eton College 163) 
Chapter 101, section 1, line 2

                                                          Καὶ ἔτζι ἀρχί-
νισεν ὁ Ἀλέξανδρος νὰ παρακαλῆ τὸν θεὸν τοῦ οὐρανοῦ καὶ τῆς 
γῆς Σαβαώθ· 
 “Θεέ μου ὑψηλότατε, ὁποὺ ἀναπαύεσαι εἰς τοὺς ἁγίους, 
γένου βοηθός μου τὴν σήμερον εἰς τὸν Πῶρον τῆς Ἰνδίας. 



Historia Alexandri Magni, Recensio E (cod. Eton College 163) 
Chapter 102, section 1, line 15

                                     Καὶ τῆς Ἰνδίας τὰ φουσάτα ἔδωκαν 
εἰς τὸ φυγίον νὰ φεύγουν. 



Historia Alexandri Magni, Recensio E (cod. Eton College 163) 
Chapter 102, section 2, line 3

                                                                        Καὶ 
λʹ χιλιάδες ἀρχόντισσες τῆς Ἰνδίας ἦλθαν καὶ ἐσυναπάντησαν <μὲ> τὸν 
κλαυθμὸν καὶ δάκρυα πολλὰ τὸ κορμὶν τοῦ Πώρου, καὶ μὲ θλῖψιν καὶ θρῆνον ἐ-
συναπάντησαν τὸν Πῶρον. 



Historia Alexandri Magni, Recensio E (cod. Eton College 163) 
Chapter 103, section 2, line 12

                                                              Καὶ ἤφεραν 
τοῦ Ἀλεξάνδρου ἑκατὸν φαρία τοῦ στάβλου τῆς Ἰνδίας ὅλα ἀρμα-
τωμένα καὶ κουβέρτες ἀπὸ σύρμα. 



Historia Alexandri Magni, Recensio E (cod. Eton College 163) 
Chapter 103, section 2, line 24

                           Ὁ Ἀλέξανδρος τὸν ἠγαπημένον τὸν ἰδικόν του τὸν 
βοϊβόντα του τὸν Ἀντίοχον ἐποίησεν αὐθέντην τῆς Ἰνδίας ὅλης. 



Historia Alexandri Magni, Recensio E (cod. Eton College 163) 
Chapter 111, section 1, line 3

         Καὶ ὅρισεν ὁ Ἀντίοχος καὶ ἐδώρισαν τὸν Καταυλούσην ἀ-
πανωφόριν μακεδονίτικον καὶ φαρὶν ἀπὸ τὴν Ἰνδίαν ἀποκάτω κουβερ-
τιασμένον ἀπὸ κορκόνδειλον. 



Historia Alexandri Magni, Recensio E (cod. Eton College 163) 
Chapter 113, section 3, line 5

                                                            Καὶ αὐτὸς τὸν 
εἶπεν· “Ἐγὼ εἶμαι ὁ Σωχὸν ὁ βασιλεὺς τῆς Ἰνδίας. 



Historia Alexandri Magni, Recensio E (cod. Eton College 163) 
Chapter 114, section 1, line 17

                       Καὶ ὁ Τάρειος τὸν εἶπεν· “Σύρε παραπέσω καὶ αὐτοῦ 
θέλεις ἰδεῖ τὸν Πῶρον τὸν βασιλέα τῆς Ἰνδίας”. 



Historia Alexandri Magni, Recensio E (cod. Eton College 163) 
Chapter 115, section 1, line 2

Ὁ Ἀλέξανδρος τὸν Πῶρον τὸν βασιλέα εἶδε τον καὶ εἶδε 
του· “Ὦ μεγαλειότατε Πῶρε τῆς Ἰνδίας, ἐσὺ μὲ τὸν θεὸν ἴσια 
ὀνομάζεσουν, καὶ τώρα ὡσὰν ἕνας <ἄτυχος ἅνθρωπος> κολάζεσαι ἐ-
δῶ”. 



Historia Alexandri Magni, Recensio E (cod. Eton College 163) 
Chapter 117, section 4, line 26

                         Ὁ Δορυφόρος τὴν μητέραν του οὐδὲν τὴν ἤκου-
σεν, ἀμὴ ἔλεγεν· “Ἔναν ἄνθρωπον τοῦ Ἀλεξάνδρου οὐδὲν μὲ ἀφή-
νετε νὰ σκοτώσω· αὐτὸς πολλοὺς ἀνθρώπους ἰδικούς μου ἐσκότωσεν, 
χιλιάδες καὶ μυριάδες, καὶ τὸν πεθερόν μου τὸν Πῶρον, τῆς Ἰν-
δίας τὸν βασιλέα, ἐσκότωσεν. 



Historia Alexandri Magni, Recensio E (cod. Eton College 163) 
Chapter 117, section 5, line 23

         Καὶ ὁ αὐθέντης του τὸν λωλὸν τὸν πεθερόν σου, τὸν Πῶ-
ρον, διὰ τὴν ἐπαρσίαν του καὶ τὴν κακήν του φρόνεσιν μὲ μίαν 
σπαθέα τὸν ἐθανάτωσε τῆς Ἴνδιας τὸν βασιλέα”. 



Historia Alexandri Magni, Recensio E (cod. Eton College 163) 
Chapter 119, section 1, line 7

                                                           Ἀρχὴ τὴν 
Ἰνδίαν ὅλην τὴν ἔδωκεν τοῦ Ἀντιόχου νὰ αὐθεντεύη καὶ νὰ τὴν 
ὁρίζη. 



Historia Alexandri Magni, Recensio E (cod. Eton College 163) 
Chapter 120, section 4, line 12

                      Καὶ ἦσαν καὶ πλεώτεροι, ὁποὺ ἤβλεπαν, ἕως 
χιλιάδες ἑκατὸν ἄνθρωποι μαζωμένοι· ἀρχὴ οἱ Πέρσηδες, τῆς Ἰν-
δίας τὰ φουσάτα, οἱ Σιριάνηδες καὶ οἱ Ἑβραῖοι, οἱ Δῆδες, 
Φοινικαῖοι, Γελφοί, Ἀλαμπετάνοι, πολλοὶ ἐκ τὴν Ἀνατολὴν καὶ 
τῆς Ἰνδίας ἦσαν. 



Historia Alexandri Magni, Recensio E (cod. Eton College 163) 
Chapter 123, section 2, line 2

Καὶ αὐτὴν τὴν ἡμέραν ἤφεραν ἕναν ἄνθρωπον ἀπὸ τὴν Ἴν-
διαν δεξιώτην εἰς τὸν Ἀλέξανδρον καὶ ἔλεγαν ὅτι τέτοιος δεξι-
ώτης ἔνι, ὅτι ἐκ τὸ δακτυλίδιν διαβάζει τὴν σαΐταν. 



Historia Alexandri Magni, Recensio E (cod. Eton College 163) 
Chapter 124, section 3, line 3

Καὶ αὐτοῦ ὁ Ἀλέξανδρος τὸν ἰδικόν του τὸν Ἀριστοτέ-
λην ἐδώρισέν τον ἔνα στέμμα καὶ τὸ ἀπανωφόριν τοῦ Πώρου τοῦ 
βασιλέως τῆς Ἰνδίας καὶ δέκα χιλιάδες χρυσὰ τάλαντα καὶ δέκα 
χιλιάδες μόδια μαργαριτάριν. 



Historia Alexandri Magni, Recensio E (cod. Eton College 163) 
Chapter 125, section 5, line 8

                   Καὶ αὐτοῦ ἀρχίνισεν ὁ Ἀλέξανδρος νὰ δείχνη 
τοὺς πολέμους τοὺς μεγάλους, ὁποὺ ἐποίησε μὲ τὸν Τάρειον τῆς Περσίας τὸν βα-
σιλέα καὶ μὲ τὸν Πῶρον τῆς Ἰνδίας καὶ τῆς Ἀνατολῆς καὶ τῆς 
Δύσης. 



Historia Alexandri Magni, Recensio E (cod. Eton College 163) 
Chapter 125, section 11, line 11

Ὅλοι οἱ Πέρσηδες ἐθαύμασαν καὶ τῆς Ἰνδίας τὰ φουσάτα καὶ ὅ-
λος ὁ κόσμος ἀποκάτω τὴν Ἴνδιαν καὶ τοῦ ἡλίου ἐτρόμαξαν, ἡ 
Ἀνατολὴ καὶ ἡ Δύσις, Ἄρκτος καὶ Μεσημβρία. 



Historia Alexandri Magni, Recensio V (cod. Vind. theol. gr. 244) (1386: 014)
“Der byzantinische Alexanderroman nach dem Codex Vind. Theol. gr. 244”, Ed. Mitsakis, K.
Munich: Institut für Byzantinistik und neugriechische Philologie der Universität, 1967; Miscellanea Byzantina Monacensia 7.
Page 72, line 20

Ὁ Ἀλέξανδρος αὐτὴν τὴν ὥραν, ἤγουν τὴν νύκταν, εἶδεν 
ὄνειρον ὅτι ἦλθεν ὁ προφήτης Ἱερεμίας καὶ ἦτον ἐντυμένος ἀρχιε-
ρατικὴν στολὴν καὶ εἶπεν του· «Ἐγλήγορα, τέκνον Ἀλέξανδρε, 
σῦρε καὶ γένου ἀποκρισιάρης εἰς τὸν Δάρειον καὶ καταπάτησέ 
τον αὐτὸν καὶ τὰ φουσάτα του καὶ τῆς Ἰνδίας ὁποὺ φέρνει κατα-
πάνου σου· καὶ ἐὰν σὲ ἐγνωρίσουν εἰς τὸν Δάρειον, ὁ θεὸς τοῦ 
οὐρανοῦ καὶ τῆς γῆς Σαβαὼθ σὲ θέλει φυλάξει καὶ θέλεις σωθῆ 
ἀβλαβὴς ἀπέκει. 



Historia Alexandri Magni, Recensio V (cod. Vind. theol. gr. 244) 
Page 79, line 23

                                  Καὶ οὕτως ὁ Ἀλέξανδρος ὄρθω-
σεν καὶ ὑπῆγεν εἰς τῆς Ἰνδίας τὰ φουσάτα, ἤγουν εἰς τὸν βασιλέα. 



Historia Alexandri Magni, Recensio V (cod. Vind. theol. gr. 244) 
Page 79, line 24

   Ὁ Πῶρος καὶ μέγας καὶ ὑψηλότατος βασιλέας τῆς Ἰνδίας 
ὅλης», τοιούτως λέγει ἡ γραφή, «ὅτι κάτι ἤκουσα, Ἀλέξανδρε, 
ὅτι τὸν Δάρειον ἐσκότωσες καὶ ὑψώθης πολλὰ καὶ ἀπὸ τὴν ἀνα-
γνωσίαν σου τὴν πολλὴν ἦλθες εἰς τὸ σύνορόν μου νὰ χαθῆς. 



Historia Alexandri Magni, Recensio V (cod. Vind. theol. gr. 244) 
Page 82, line 9

Καὶ τῆς Ἰνδίας τὰ φουσάτα ὡς εἶδαν, ἔδωκαν νὰ φεύγουν· Ὁ 
Ἀλέξανδρος κατόπισθεν τοὺς ἐδίωχνεν καὶ [f. 42v] ἐσκότωσεν 
τρεῖς χιλιάδες. 



Historia Alexandri Magni, Recensio V (cod. Vind. theol. gr. 244) 
Page 83, line 2

                       Ὁ Ἀλέξανδρος τὸ βοηβόντα τὸν Ἀντίο-
χον ἔκαμεν αὐθέντην τῆς Ἰνδίας. 



Historia Alexandri Magni, Recensio φ (1386: 016)
“Ἡ φυλλάδα τοῦ μεγαλέξαντρου. Διήγησις Ἀλεξάνδρου τοῦ Μακεδόνος”, Ed. Veloudis, G.
Athens: Hermes, 1977; Νέα Ἑλληνικὴ Βιβλιοθήκη 39.
Section 103, line 11

                                              Καὶ τὸν βασιλέα τῆς Ἰνδίας 
θέλεις σκοτώσει· καὶ εἰς ἀσθένειαν θέλεις πέσει· καὶ ὁ Θεὸς θέλει 
σὲ βοηθήσει καὶ θέλεις γένει ὅλου τοῦ κόσμου βασιλεύς· καὶ εἰς 
τὸν 
         Παράδεισον κοντὰ θέλεις ὑπάγει καὶ θέλεις εὕρει ἐκεῖ ἄνδρες 
καὶ γυναῖκες εἰς ἕνα νησὶ φυλακωμένους, ὁποὺ εἶναι τὸ φαγί τους 
πωρικὰ καὶ τὸ ὄνομά τους εἶναι Μακάριοι· καὶ θέλουν σοῦ ὁμολο-
γήσει διὰ τὴν ζωήν σου καὶ διὰ τὸν θάνατόν σου. 



Historia Alexandri Magni, Recensio φ 
Section 136, line 9

Αὐτὴν τὴν νύκτα εἶδεν ὁ Ἀλέξανδρος ὄνειρον ὅτι ἦλθεν ὁ 
προφήτης Ἱερεμίας καὶ ἦτον ἐνδεδυμένος ἀρχιερατικὴν στολήν, ὡσὰν 
τὸν εἶδεν εἰς τὴν Ἱερουσαλήμ, καὶ εἶπε του: Ὀγλήγορα ὕπαγε εἰς 
τὸν Δάρειον ἀποκρισάρης καὶ καταπάτησέ τον, ὁμοίως καὶ τὰ στρα-
τεύματα τῆς Ἰνδίας, ὁποὺ ἔρχουνται κατεπάνω σου· καὶ ἐὰν σὲ γνω-
ρίσουν, ὁ Θεὸς θέλει σὲ φυλάξει. 



Historia Alexandri Magni, Recensio φ 
Section 148, line 9

Καὶ ὅρισε νὰ γράψουν χαρτὶ παραπονετικὸν εἰς τὸν Πῶρον, τὸν βα-
σιλέα τῆς Ἰνδίας. 



Historia Alexandri Magni, Recensio φ 
Section 148, line 13

   Εἰς τὸν βασιλέα τὸν ὑψηλότατον παρὰ ὅλους τοὺς βασιλεῖς 
τῆς γῆς, ὁποὺ εἶσαι κύριος τῶν αὐθεντάδων τῆς οἰκουμένης καὶ 
ἐπίγειος θεὸς τῆς Ἰνδίας, ἡ δεξιά σου εἶναι στε-
        ρεὰ εἰς ὅλα τὰ 
βασίλεια τῆς γῆς. 



Historia Alexandri Magni, Recensio φ 
Section 150, line 12

                                                 Καὶ ἐκίνησαν τὰ φου-
σάτα τῆς Ἰνδίας εἰς βοήθειαν τοῦ Δαρείου. 



Historia Alexandri Magni, Recensio φ 
Section 150, line 13

                                                    Καὶ ὡς ἤκουσεν ὁ Δά-
ρειος ὅτι ἦλθαν τὰ φουσάτα τῆς Ἰνδίας, 
         ἐσύναξε καὶ αὐτὸς τὰ 
φουσάτα του τῆς Περσίας καὶ ἔγραψέ τα καὶ ηὗρε τα χίλιες χιλιάδες. 



Historia Alexandri Magni, Recensio φ 
Section 151, line 13

                                                            Καὶ ἐρώτησάν 
τους οἱ ἄρχοντες τῆς Ἰνδίας διὰ τὰ φουσάτα τοῦ Ἀλεξάνδρου, καὶ 
αὐτοὶ εἶπαν: Ἡμεῖς καθὼς εἴδαμεν, ἔτζι μαρτυροῦμεν. 



Historia Alexandri Magni, Recensio φ 
Section 152, line 3

          Καὶ ὡσὰν ἤκουσαν τὰ φουσάτα τῆς Ἰνδίας, ἐδειλίασαν 
καὶ ὡσὰν ἐγγαρεμένοι ἤρχονταν εἰς τὸν πόλεμον. 



Historia Alexandri Magni, Recensio φ 
Section 152, line 12

                                     Καὶ αὐτὴν τὴν ὥραν δὲν ἐγνωρί-
ζετο τίς εἶναι Πέρσης ἢ Ἴνδης ἢ Μακεδών. 



Historia Alexandri Magni, Recensio φ 
Section 153, line 2

          Καὶ ὡσὰν ἐνύκτωσεν, ἐφοβήθησαν οἱ Ἰνδοὶ καὶ οἱ Πέρσαι 
καὶ ἄρχισαν νὰ φύγουν. 



Historia Alexandri Magni, Recensio φ 
Section 153, line 4

Καὶ ὡσὰν εἶδεν ὁ Ἀλέξανδρος ὅτι ἐγύρισαν οἱ Ἴνδαι, δὲν ἠμπόρεσε   
νὰ βαστάξη, μόνον εἰσέβη εἰς τὸ στράτευμα τῆς Ἰνδίας μὲ τὸ τάγμα 
του, ἑκατὸν χιλιάδες, ὁποὺ ἦσαν ὅλοι διαλεκτοί, καὶ ἔκαμε μεγάλους 
φόνους καὶ πολλούς. 



Historia Alexandri Magni, Recensio φ 
Section 153, line 7

                       Καὶ οἱ Ἴνδαι, ὡσὰν εἶδαν τὸν Ἀλέξανδρον, 
ἔδωκαν νὰ φεύγουν. 



Historia Alexandri Magni, Recensio φ 
Section 153, line 9t

Φυγὴ τῶν Ἰνδῶν. 




Historia Alexandri Magni, Recensio φ 
Section 153, line 10

Καὶ ὡσὰν εἶδαν οἱ Πέρσαι τὰ φουσάτα τῆς Ἰνδίας πὼς φεύγουν, 
ἄρχισαν καὶ αὐτοὶ νὰ φεύγουν, μὲ τὸ νὰ μὴν δύνωνται νὰ πολεμοῦν. 



Historia Alexandri Magni, Recensio φ 
Section 155, line 3

                                                     Ἔστειλε καὶ τὸν Φι-
λόνην εἰς τὸ φουσάτον τῆς Ἰνδίας νὰ πάρη τὰ ἄλογά τους καὶ αὐτου-
νοὺς νὰ τοὺς ἀφήση νὰ ὑπάγουν εἰς τὸν αὐθέντην τους τὸν Πῶρον. 



Historia Alexandri Magni, Recensio φ 
Section 155, line 6

Καὶ ὑπῆγεν ὁ Φιλόνης καὶ εἶπε τὸν ὁρισμὸν τοῦ Ἀλεξάνδρου εἰς 
τοὺς Ἰνδούς. 



Historia Alexandri Magni, Recensio φ 
Section 155, line 14

                                                     Καὶ οἱ Πέρσαι ἐξε-
χωρίσθηκαν ἀπὸ τὰ φουσάτα τῆς Ἰνδίας καὶ ἐπροσκύνησαν τὸν Φι-
λόνην καὶ ἦλθαν εἰς μίξιν μὲ τὰ φουσάτα τῆς 
         Μακεδονίας. 



Historia Alexandri Magni, Recensio φ 
Section 187, line 5

                                                 Μόνον κάμε ἐτοῦτο ὁποὺ 
σοῦ λέγομεν καὶ γύρισε ὀπίσω δεξιὰ καὶ σύρε πάλιν νὰ ἀκολουθή-
σης τοὺς συνηθισμένους σου πολέμους, ὅτι καρτερεῖ σε καὶ ὁ βα-
σιλεὺς τῆς Ἰνδίας νὰ πολεμήσετε. 



Historia Alexandri Magni, Recensio φ 
Section 192, line 4

                          Καὶ ἀπ' αὐτοῦ ἐπεριπάτησεν ἡμέρας ἓξ καὶ 
ἦλθεν εἰς τὸ σύνορον τῆς Ἰνδίας καὶ ἐξέβη εἰς τὸν κόσμον. 



Historia Alexandri Magni, Recensio φ 
Section 192, line 10

                                              Ἔμαθεν δὲ ὁ Πῶρος, ὁ βα-
σιλεὺς τῆς Ἰνδίας, ὅτι ἦλθεν ὁ Ἀλέξανδρος εἰς τὸ σύνορόν του καὶ 
τοῦ ἔγραψεν ἐπιστολὴν τοιαύτην: 
Ἐπιστολὴ Πώρου πρὸς Ἀλέξανδρον. 




Historia Alexandri Magni, Recensio φ 
Section 193, line 2

   Ὁ Πῶρος, ὁ μέγας βασιλεὺς τῆς Ἰνδίας, ὁποὺ λάμπει ὡσὰν 
θεός, εἰς τὸν Ἀλέξανδρον τὸν βασιλέα. 



Historia Alexandri Magni, Recensio φ 
Section 194, line 9

   Ἀλέξανδρος, ὁ βασιλεὺς τῶν βασιλέων, ὄχι μὲ τὸ ἐδικόν μου 
θέλημα, ἀμὴ μὲ τοῦ παντοκράτορος Σαβαώθ, ὁποὺ ἔκαμε τὸν οὐρανὸν 
καὶ τὴν γῆν, εἰς τὸν βασιλέα τῆς Ἰνδίας Πῶρον, ὁποὺ λέγει πὼς 
λάμπει ὡσὰν ὁ ἥλιος καὶ γίνεται θεὸς ἐπίγειος, ὁ οὐτιδανός, ὁ γάϊ-
δαρος, ὁ τρελός, ὁ ἄτυχος ἄνθρωπος τοῦ κόσμου ὅλου. 



Historia Alexandri Magni, Recensio φ 
Section 195, line 2

                                                                  Καὶ αὐτὸς 
τέτοιας λογῆς ἐθεοποιεῖτο εἰς ὅλην τὴν Περσίαν, ὡσὰν 
         καὶ ἐσένα 
τώρα εἰς τὴν Ἰνδίαν, καὶ ἐγὼ τὸν ἐσκότωσα μὲ τὴν δύναμιν τοῦ παντο-
κράτορος Σαβαώθ. 



Historia Alexandri Magni, Recensio φ 
Section 196, line 6

                                                                  Ἀκόμη 
καὶ εἰς τοὺς ἄλλους βασιλεῖς τῆς Ἰνδίας στείλετε γράμματα νὰ ἔλ-
θουν ὅλοι. 



Historia Alexandri Magni, Recensio φ 
Section 197, line 8

Καὶ ὡς τὸ ἤκουσεν ὁ Ἀλέξανδρος, ὅρισε καὶ ἐσυμμαζώχθησαν 
τὰ φουσάτα του ὅλα ἔμπροσθέν του καὶ εἶπε: Ὦ ἠγαπημένοι μου 
Μακεδόνες καὶ ἀπὸ ὅλες τὶς γενεὲς ὑψηλότεροι, τὸν κόσμον ὅλον 
ἐπήραμεν καὶ τὴν οἰκουμένην ἐπεριλάβαμεν καὶ τὴν σήμερον ἐφο-
βηθήκατε ἀπὸ τοὺς ἀνάνδρους Ἰνδούς. 



Historia Alexandri Magni, Recensio φ 
Section 200, line 6

                               Ἂς ἠξεύρετε πὼς ὅλον τὸν κόσμον 
εἴδαμεν καὶ ἐγυρίσαμεν ὑγιεῖς καὶ ἤλθαμεν εἰς τὴν Ἰνδίαν καὶ θέλο-
μεν νὰ πολεμήσωμεν μὲ τὸν Πῶρον. 



Historia Alexandri Magni, Recensio φ 
Section 200, line 9

                                       Καὶ τὴν ὥραν ὁποὺ νὰ ἰδῆτε 
τὴν γραφήν μου, νὰ φθάσετε τὸ ὀγληγορώτερον μὲ ὅλα τῆς Δύσεως 
τὰ φουσάτα εἰς τὴν Ἰνδίαν». 



Historia Alexandri Magni, Recensio φ 
Section 204, line 9

                        Καὶ ἐστάθη ὁ Φιλόνης καὶ ὁ Σέλευκος καὶ εἶ-
παν: «Ὦ ἠγαπημένε αὐθέντη καὶ τοῦ κόσμου ὅλου βασιλεῦ, δὲν 
πρέπει νὰ στέκης ἀντίκρυ τοῦ βασιλέως τῆς Ἰνδίας νὰ τὸν βλέπης, 
ἀμὴ σύρε κατεπάνω του καὶ μὴν ἀργῆς, ὅτι ὁ Πῶρος ἔχει φουσάτα 
πολλὰ παρὰ ἡμᾶς. 



Historia Alexandri Magni, Recensio φ 
Section 205, line 9

                       Καὶ δὲν πρέπει ἐσένα νὰ πολεμήσης μὲ τὸν   
Πῶρον, ὅτι ἐσὺ εἶσαι τῆς οἰκουμένης ὅλης βασιλεύς, ἀμὴ νὰ πολε-
μήσω ἐγώ, ὁποὺ εἶμαι μὲ τὸ ἐδικόν σου ῥιζικὸν αὐθέντης τῆς Περσίας 
καὶ αὐτὸς εἶναι τῆς Ἰνδίας καὶ εἴμεσθεν παρόμοιοι. 



Historia Alexandri Magni, Recensio φ 
Section 206, line 8

                       Καὶ ἐπολέμησαν τὰ φουσάτα τῆς Ἰνδίας πολλὰ 
ἀνδρειωμένα καὶ ὕστερον ἐνικήθησαν καὶ ἄρχισαν νὰ φεύγουν. 



Historia Alexandri Magni, Recensio φ 
Section 207, line 8

                                    Ὁ δὲ Ἀλέξανδρος ἦλθεν εἰς τὸ κονάκι 
τοῦ Πώρου καὶ ἔστειλε τὰ φουσάτα του καὶ ἐκούρσευσαν τὸν τό-
πον τῆς Ἰνδίας. 



Historia Alexandri Magni, Recensio φ 
Section 209, line 13

                                                                  Ἀμὴ 
ὡσὰν βλέπω, ἐσὺ ἔχεις ἔχθρητα τῶν στρατιωτῶν τῆς Ἰνδίας καὶ 
θέλεις 
         νὰ τοὺς χάσης ὅλους. 



Historia Alexandri Magni, Recensio φ 
Section 210, line 12

                                        Καὶ ὡς ἤκουσεν ὅλον τὸ 
φουσάτον τῆς Ἰνδίας, ἐχάρη. 



Historia Alexandri Magni, Recensio φ 
Section 212, line 8

         Καὶ ὡσὰν εἶδεν τὸν θάνατον τοῦ Πώρου τῆς Ἰνδίας τὸ φου-
σάτον, ἔδωκαν νὰ φεύγουν. 



Historia Alexandri Magni, Recensio φ 
Section 213, line 1

                                     Καὶ ἡ βασίλισσα ἡ Κλυταιμνή-
στρα ἀπέλυσε τὰ μαλλιά της ἕως τὴν γῆν καὶ τὸ πολύτιμόν της φό-
ρεμα ἔσκισεν καὶ μὲ δέκα χιλιάδες 
         ἀρχόντισσες τῆς Ἰνδίας ἦλθε 
καὶ ἐσυναπάντησε τὸ λείψανον τοῦ Πώρου μὲ κλαυθμὸν καὶ ὀδυρμὸν 
καὶ μὲ θλῖψιν μεγάλην. 



Historia Alexandri Magni, Recensio φ 
Section 215, line 4

                                               Ἠκούσατε πὼς τὸν Δάρειον, 
τὸν βασιλέα τῆς Περσίας, ἀφάνισα, ὁμοίως καὶ τὸν φρικτὸν Πῶρον 
τῆς Ἰνδίας καὶ πολλοὺς βασιλεῖς ὁποὺ ἐδούλωσα καὶ τοὺς ἔκαμα 
διὰ νὰ μὲ ἔχουν ὅλοι αὐθέντην τους καὶ νὰ μοῦ δίδουν ὅλοι χαράτζιον. 



Historia Alexandri Magni, Recensio φ 
Section 239, line 1

                                                Καὶ ὁ Δάρειος τοῦ εἶπε· 
Σύρε παρεμπρὸς νὰ ἰδῆς καὶ τὸν Πῶρον, τὸν βασιλέα 
         τῆς Ἰνδίας. 



Historia Alexandri Magni, Recensio φ 
Section 248, line 14

                                                     ὁ Δορυφόρος εἶπε: 
Ἄφησόν με, μητέρα μου, νὰ σκοτώσω ἕνα ἄνθρωπον τοῦ Ἀλεξάν-
δρου, ὁποὺ αὐτὸς πολλὲς χιλιάδες λαὸν μοῦ ἐσκότωσεν, ἀκόμη καὶ 
τὸν πενθερόν μου Πῶρον, τὸν βασιλέα τῆς Ἰνδίας, καὶ ὁποὺ ἡ 
         γυ-
ναίκα μου κλαίει μερόνυκτον διὰ τὸν θάνατόν του· καὶ πίστευσόν 
με ὅτι ὁ Ἀντίοχος πλέον ζωὴν δὲν ἔχει. 



Historia Alexandri Magni, Recensio φ 
Section 265, line 2

                        Καὶ ἀνεγίνωσκε τὴν ἐπιστολὴν ὁ Ἀριστοτέλης, 
καὶ ὁ Ἀλέξανδρος ἐκάθετο εἰς ἕναν θρόνον, ὁποὺ εἶχε βήματα δώ-
δε-
        κα, ἐγκοσμημένος· ὁ ὁποῖος ἦτον τοῦ Πώρου, τοῦ βασιλέως 
τῆς Ἰνδίας. 



Historia Alexandri Magni, Recensio φ 
Section 268, line 10

        Καὶ αὐτὴν τὴν ἡμέραν ἤφεράν τον ἕναν ἄνθρωπον ἀπὸ τὴν 
Ἰνδίαν, καὶ ἔλεγεν ὅτι τέτοιος τοξότης εἶναι, ὁποὺ ἀπὸ τὴν μέσην 
τοῦ δακτυλιδίου διαβαίνει τὴν σαΐτταν. 



Historia Alexandri Magni, Recensio φ 
Section 285, line 1

      Προσέτι ἀφήνω τοῦ Πτολεμαίου ὅλην τὴν Αἴγυπτον καὶ τὴν 
Ἀλεξάνδρειαν· Σελεύκου τὴν Περσίαν καὶ Βαβυλώνα· Ἀντιγόνου   
τὴν Κιλικίαν· Φίλωνος τὴν Μηδίαν· Πύθωνος τὴν Φρυγίαν καὶ Λυ-
δίαν· Μελεάγρου τὴν Παφλαγονίαν· Εὐμενίου τὴν Καππαδοκίαν· 
Κασσάνδρου τὴν Λυκίαν καὶ Ἑλλήσποντον· Λυσιμάχου τὴν Θράκην· 
Ἀντιπάτρου τὴν Ποντικήν· Ὀξυάρτου τὴν Βακτριανήν· Φιλίππου 
τὴν Δραγαΐνην· Ἀντιόχου τὴν Ἰνδίαν· Φραταφέρνου τὴν Παρθίαν 
καὶ τὴν Ὑρκανίαν· τὴν δὲ Περσίαν καὶ Μεσοποταμίαν ἀφήνω τοῦ 
Ὀβλίου. 



Historia Alexandri Magni, Recensio Byzantina poetica (cod. Marcianus 408) (1386: 018)
“Das byzantinische Alexandergedicht nach dem codex Marcianus 408 herausgegeben”, Ed. Reichmann, S.
Meisenheim am Glan: Hain, 1963; Beiträge zur klassischen Philologie 13.
Line 96

Ἐπέρχεται γὰρ μέγιστον ἔθνος ἡμῖν βαρβάρων, 
οὐχ ἓν ὑπάρχον τὸ φανέν, πολλαὶ δὲ μυριάδες 
ἡμῖν κατεπερχόμεναι· πρῶτον Ἰνδῶν τὰ γένη, 
Ἴβηρες Ὀξυδόρκηες σὺν Νοημαίοις πᾶσιν, 
Βοσπόροι καὶ Νοκίμαροι, Χάλβαι, πολύ τι πλῆθος, 
ὅσα τε πρὸς ἀνατολὴν ἔθνη Περσῶν ὑπάρχει, 
ἀναριθμήτοις μαχηταῖς, ἀνδράσι μεγιστάνοις, 
αὐτὴν πᾶσαν τὴν Αἴγυπτον βουλόμενοι συστρέψαι. 



Historia Alexandri Magni, Recensio Byzantina poetica (cod. Marcianus 408) 
Line 2971

  Ὅταν ἐπέμφθην παρὰ σοῦ,» προσείρηκε σατράπης, 
»πρὸς τὸν πατέρα Φίλιππον τούτου τοῦ Μακεδόνος 
λαβέσθαι φόρους παρ' αὐτοῦ, εἶδον αὐτὸν ἐκεῖσε· 
ἔγνων γὰρ τούτου φρόνησιν, ἰσχὺν καὶ χαρακτῆρα· 
εἴπερ οὖν δέχῃ συμβουλὴν ἐμοῦ τοῦ σοῦ σατράπου, 
τοὺς Πέρσας ἅπαντας ἐλθεῖν ἐκ πάσης οἰκουμένης 
μετὰ σπουδῆς σὺ πρόσταξον, ὡσαύτως ἔθνη πάντα 
τὰ δουλικῶς συνόντα σοι, φημὶ Βαβυλωνίων, 
τῶν Ἐλιμαίων, Πάρθων τε καὶ Μεσοποταμίων, 
τῶν Ἰλαυρίων, Χωρανῶν, Βάκτρων, Ἰνδῶν τῶν ἔσω, 
ἀλλὰ καὶ τοὺς τὰ μέλεθρα ταῦτα Σεμιραμέως,   
(ἔστι σοι δ' ὀγδοήκοντα σὺν ἑκατὸν τὰ γένη,) 
στράτευσον ἅπαντας αὐτοὺς Θεῶν τῇ συμμαχίᾳ, 
καὶ Μακεδόσιν Ἕλλησιν ἀντιπαραταχθῶμεν· 
κἂν μὲν μὴ πάρεσμεν αὐτοὶ δυνάμει μαχεσθῆναι 
καὶ πολεμῆσαι τοῖς ἐχθροῖς, ἀλλ' οὖν γε τούτους μᾶλλον 
τῷ πλήθει προσφοβήσομεν ποιήσαντες φυγάδας. 



Historia Alexandri Magni, Recensio Byzantina poetica (cod. Marcianus 408) 
Line 3287

Μαθὼν δὲ Πῶρος ἅπαντα τὰ παρὰ τοῦ Δαρείου 
τούτῳ γραφέντα δυστυχῶς, πληροῦται λύπῃ πλείστῃ, 
αὐτῷ δὲ προσαντέγραψεν ἐπιστολὴν τοίανδε· 
»Πῶρος Ἰνδῶν ὁ βασιλεὺς τῷ βασιλεῖ Δαρείῳ 
Περσῶν, προσαγορεύω σοι καὶ προσλαλῶ σοι χαίρειν. 



Historia Alexandri Magni, Recensio Byzantina poetica (cod. Marcianus 408) 
Line 3323

Αὐτὴν δὲ τὴν ἐπιστολὴν πᾶσαν ἰδὼν Δαρεῖος 
ἐδάκρυσεν ἀναμνησθεὶς μητρός, συζύγου, τέκνων, 
ὁμοῦ προσπαρασσόμενος ψυχήν τε καὶ καρδίαν, 
καὶ χώραν Βακτηριανὴν καὶ πᾶσαν τὴν Ἰνδίαν 
ὅλην προσεκαλέσατο πρὸς μάχην Ἀλεξάνδρου. 



Historia Alexandri Magni, Recensio Byzantina poetica (cod. Marcianus 408) 
Line 3804

Δαρεῖος ηὐτρεπίζετο καὶ πάλιν εἰς τὴν μάχην 
καὶ πολεμεῖν Ἀλέξανδρον ὁ δυστυχὴς ἐκ τρίτου· 
ὅθεν Ἰνδῶν τῷ βασιλεῖ τῷ Πώρῳ γράφει ταῦτα· 
»Δαρεῖος βασιλεὺς Περσῶν ἄνακτι Πώρῳ χαίρειν. 



Historia Alexandri Magni, Recensio Byzantina poetica (cod. Marcianus 408) 
Line 3822

Λοιπὸν αὐτὸς συνάθροισον Ἰνδῶν πάντα τὰ γένη 
εἰς Πύλας τὰς Κασπιακὰς καὶ τοῖς συνερχομένοις 
ἀνδράσι σὺ χορήγησον χρυσὸν πολὺν καὶ σῖτον 
καὶ πλεῖστα τὰ χορτάσματα· πάντα γοῦν τῶν λαφύρων 
ἔσται σοι τὰ ἡμίσεια· χαρίσομαί σοι μᾶλλον 
τὸν ἵππον τὸν Βουκέφαλον ὃν οὗτος ἐποχεῖται, 
ὁ Μακεδόνων τύραννος, ὡσαύτως δίδωμί σοι 
τὴν πανοπλίαν τὴν αὐτοῦ σὺν παλλακαῖς ταῖς τούτου. 



Historia Alexandri Magni, Recensio Byzantina poetica (cod. Marcianus 408) 
Line 3832

Λαβὼν οὖν σὺ τὰ γράμματα μετὰ σπουδῆς τῆς πλείστης 
ἄθροισον πᾶσαν στρατιὰν καὶ στεῖλον ἡμῖν τάχει· 
ἔρρωσο, Πῶρε, βασιλεῦ μέγιστε τῆς Ἰνδίας. 



Historia Alexandri Magni, Recensio Byzantina poetica (cod. Marcianus 408) 
Line 4475

Μετὰ δὲ ταῦτ' Ἀλέξανδρος ἐποίει τὴν πορείαν, 
τῶν Μακεδόνων ἅπαντα παραλαβὼν τὰ πλήθη, 
πρὸς βασιλέα τῶν Ἰνδῶν, τὸν προρρηθέντα Πῶρον. 



Historia Alexandri Magni, Recensio Byzantina poetica (cod. Marcianus 408) 
Line 4482

Πολλὴν οὖν ἔρημον ὁδὸν καὶ τόπους διελθόντες 
ἀνύδρους, φαραγγώδεις τε καὶ φοβερούς, κρημνώδεις, 
πρὸς Μακεδόνας ἔλεγον ἅπαντες οἱ σατράπαι· 
»Μέχρι Περσίδος ἀρκετὸν ἡμᾶς ποιεῖσθαι μάχην 
καὶ τὸν Δαρεῖον καθελεῖν, ὡς ἀπαιτοῦντα φόρους 
ἡμᾶς καὶ πάντας Ἕλληνας· κάμωμεν τί γὰρ ἄρτι 
ἐρχόμενοι πρὸς τοὺς Ἰνδοὺς εἰς θηριώδεις τόπους, 
Ἑλλάδι μὴ προσήκοντας, ἠγριωμένους ὄντας; 



Historia Alexandri Magni, Recensio Byzantina poetica (cod. Marcianus 408) 
Line 4498

Λοιπὸν ὑπόμνησιν ὑμῖν ταύτην ἐγὼ ποιήσω· 
ὡς τοὺς πολέμους ἔγωγε τοὺς δυνατοὺς ἐκείνους 
μόνος αὐτὸς ἐνίκησα, τὴν τῶν Ἰνδῶν δὲ μάχην 
ἐγὼ κατατροπώσομαι καλλίστῃ συμβουλίᾳ. 



Historia Alexandri Magni, Recensio Byzantina poetica (cod. Marcianus 408) 
Line 4518

Ἀλέξανδρος οὖν τάχιστα πρὸς τῶν Ἰνδῶν τὴν χώραν 
ἀνδρείως εἰσελήλυθε σὺν πάσῃ τῇ δυνάμει, 
καὶ προσυπήντησαν αὐτῷ Πώρου γραμματοφόροι 
ἐπιστολὰς κομίζοντες· ἃς καὶ λαβὼν εἰς χεῖρας 
προσέταξε μέσον στρατοῦ ταύτας ἀναγνωσθῆναι. 



Historia Alexandri Magni, Recensio Byzantina poetica (cod. Marcianus 408) 
Line 4524

Ἔλεγον δὲ τὰ γράμματα κατὰ λεπτὸν τοιάδε· 
»Πῶρος Ἰνδῶν ὁ βασιλεὺς ὁ μέγας Ἀλεξάνδρῳ, 
τῷ πόλεις καταστρέφοντι, λεηλατοῦντι ταύτας. 



Historia Alexandri Magni, Recensio Byzantina poetica (cod. Marcianus 408) 
Line 4530

Νομίζεις εἶναι σθεναρὸς ὑπὲρ Ἰνδοὺς καὶ μέγας· 
γίνωσκε γοῦν, Ἀλέξανδρε, μείζων τυγχάνω πάντων, 
μὴ μόνον μέγας βασιλεὺς ὢν ἐπὶ γῆς ἀνθρώπων, 
ἀλλ' ἔχω καὶ Διόνυσον Θεὸν προσσυμμαχοῦντα, 
ὃν καὶ πατέρα τῶν Θεῶν λέγουσι πάντες εἶναι. 



Historia Alexandri Magni, Recensio Byzantina poetica (cod. Marcianus 408) 
Line 4544

Ἂν δὲ καὶ χρείαν εἴχομεν Ἑλλάδος, σῆς πατρίδος, 
Ἰνδοὶ καθυπετάξασι πρῶτοι παρὰ τὸν Ξέρξην, 
ἀλλ' ὡς μηδὲν κατέχοντας ἄξιον θεωρίας 
βασιλικῆς ἢ γνώσεως, μᾶλλον δ' ἀχρεῖον ἔθνος, 
προσαπεστράφημεν ὑμᾶς, μή που καὶ μιανθῶμεν· 
τὸ κρεῖττον πᾶς ἐπιθυμεῖ καὶ σπεύδει συλλαβέσθαι. 



Historia Alexandri Magni, Recensio Byzantina poetica (cod. Marcianus 408) 
Line 4569

Περισσοτέρως μᾶλλον σὺ προθύμους ἀπειργάσω 
πρὸς μάχην ὀτρυνθῆναί σοι τοὺς τῶν Ἑλλήνων παῖδας, 
ὡς τὴν Ἑλλάδα λέγοντα μηδὲν χρηστὸν κατέχειν 
θεώρημα βασιλικόν, ὑμᾶς δ' Ἰνδοὺς κεκτῆσθαι 
πάντα τὰ τιμιώτερα, πόλεις χρηστὰς καὶ χώρας. 



Historia Alexandri Magni, Recensio Byzantina poetica (cod. Marcianus 408) 
Line 4602

Ὡς δὲ πλησίον ἤλθοσαν τούτων οἱ Μακεδόνες, 
Πέρσαι καὶ πάντες Ἕλληνες, Ἀλέξανδρος προσβλέψας 
μεγάλην τὴν παράταξιν τοῦ Πώρου, ἐφοβήθη 
οὐ στρατιώτας τοὺς αὐτοῦ, τούτου δὲ μᾶλλον θῆρας· 
ξένον γὰρ ἦν θεώρημα τῶν φοβερῶν θηρίων· 
οὐκ ἦν ἡ μάχη πρὸς Ἰνδούς, ἀλλὰ πρὸς τὰ θηρία. 



Historia Alexandri Magni, Recensio Byzantina poetica (cod. Marcianus 408) 
Line 4606

Σχῆμα δὲ στρατιωτικὸν περιβαλὼν εἰσῆλθεν, 
Ἰνδοῖς δόξας ὀψώνια τροφὰς ἐξαγοράσαι· 
ὅντινα τάχος βασιλεῖ τῷ Πώρῳ παριστῶσι. 



Historia Alexandri Magni, Recensio Byzantina poetica (cod. Marcianus 408) 
Line 4634

Κατετροποῦντο δὲ λοιπὸν Ἰνδοὺς οἱ Μακεδόνες 
καὶ τούτους κατεδίωκον σφάττοντες κατὰ κράτος, 
ἔχοντες Πέρσας ἰσχυροὺς συμμάχους καὶ βαρβάρους, 
βάλλοντες τούτους βέλεσι, πυκναῖς τοξοβολίαις, 
πολλὴ δὲ γέγονεν αὐτοῖς ἡ μάχη τοῦ πολέμου. 



Historia Alexandri Magni, Recensio Byzantina poetica (cod. Marcianus 408) 
Line 4639

Νῶτα δεδώκασιν Ἰνδοὶ πᾶσι τοῖς Μακεδόσιν, 
αὐτὸ δὲ στράτευμα Περσῶν γενναίως ἠνδραγάθει· 
ἀλλ' ἵππος ὁ Βουκέφαλος ἐξασθενήσας πίπτει 
καὶ λύπην προεξένησε μεγάλην Ἀλεξάνδρῳ· 
λοιπὸν χαῦνος ὁ πόλεμος ὕστερον ἐγεγόνει. 



Historia Alexandri Magni, Recensio Byzantina poetica (cod. Marcianus 408) 
Line 4644

Ἡμέραις τοίνυν εἴκοσιν Ἰνδοὶ σὺν Μακεδόσι 
καὐτοῖς τοῖς Πέρσαις ἅπασι μάχην πολλὴν ἱστῶντες   
καὶ πόλεμον ἀθροίζοντες ἅπαντες κατ' ἀλλήλων, 
οὐκ ἴσχυεν Ἀλέξανδρος Ἰνδοὺς κατατροπῶσαι· 
πολὺ γὰρ ἦν τὸ στράτευμα τὸ τῶν Ἰνδῶν καὶ μέγα· 
ὅθεν καὶ δειλιάσαντες Πέρσαι σὺν Μακεδόσι 
σαυτοὺς κατεβουλεύσαντο τῷ Πώρῳ προδιδόναι. 



Historia Alexandri Magni, Recensio Byzantina poetica (cod. Marcianus 408) 
Line 4655

Εἶθ' οὕτως ἔγραψεν αὐτὸς ἐπιστολὴν τῷ Πώρῳ, 
λέγων· «Οὐκ ἔστι, βασιλεῦ, τοῖς σοῖς Ἰνδοῖς καὶ Πέρσαις 
καὶ πᾶσι Μακεδόσι μου χρηστὰ τὰ τοῦ πολέμου· 
πάντες γὰρ οὗτοι φθείρονται μαχόμενοι κακίστως. 



Historia Alexandri Magni, Recensio Byzantina poetica (cod. Marcianus 408) 
Line 4673

Ἔσπευδεν οὖν Ἀλέξανδρον ὁ Πῶρος θανατῶσαι· 
τὰ πλήθη δὲ παρέστασαν πάντα τῶν στρατευμάτων 
ἐξ ἑκατέρων τῶν μερῶν, Ἰνδῶν καὶ τῶν Ἑλλήνων. 



Historia Alexandri Magni, Recensio Byzantina poetica (cod. Marcianus 408) 
Line 4685

Ἤρξαντο γοῦν ἀμφότερα τὰ πλήθη στρατευμάτων 
μάχην ποιεῖν πρὸς ἄλληλα καὶ πολεμεῖν καὶ πάλιν· 
καὶ δοὺς Ἀλέξανδρος φωνὴν αὐτοῖς Ἰνδοῖς προσέφη· 
»Ὦ τάλανες καὶ δυστυχεῖς Ἰνδεῖς, τί πολεμεῖτε; 



Historia Alexandri Magni, Recensio Byzantina poetica (cod. Marcianus 408) 
Line 4694

Εἶπε δ' Ἀλέξανδρος αὐτοῖς· «Παύσασθε πολεμοῦντες· 
ἀπελευθέρους γὰρ ὑμᾶς ὡς ἀναιτίους ἔχω· 
ἀπέλθετε, συστράφητε, μή τινα φόβον σχόντες· 
οὐ γὰρ κατετολμήσατε πρὸς τὸ στρατόπεδόν μου· 
Πῶρος γὰρ κατηνάγκασεν Ἰνδοὺς μαχέσασθαί με. 



Historia Alexandri Magni, Recensio Byzantina poetica (cod. Marcianus 408) 
Line 4811

Αἱ δὲ γυναῖκες πόρρωθεν εἰσὶν ἀποικισμέναι 
πλησίον Γάγγου ποταμοῦ τοῖς Ἰνδικοῖς τοῖς τόποις· 
οἱ γοῦν Βραχμᾶνοι πρὸς αὐτὰς περῶσι τὰς γυναῖκας 
καθ' ὃν καιρὸν τὰ πρὸς βορρᾶ ἥλιος ἀνατρέχει, 
ἐν Ἰουλίῳ τῷ μηνί, τῷ μετ' αὐτὸν Αὐγούστῳ· 
οἵτινες εὐκρατότεροι γενόμενοι, Βραχμᾶνοι 
πρὸς οἶστρον διεγείρονται, μιγνύμενοι γυναῖκας. 



Historia Alexandri Magni, Recensio Byzantina poetica (cod. Marcianus 408) 
Line 4909

Εἶθ' οὕτως ἀνεχώρησαν ἀμφότερα τὰ μέρη· 
ὑπέστρεψεν Ἀλέξανδρος ὁδὸν τὴν κατ' εὐθεῖαν 
τὴν ἐπὶ πρόσω φέρουσαν Πρασιακὴν τὴν πόλιν,   
ἥτις ἐστὶ μητρόπολις τῆς Ἰνδικῆς τῆς χώρας, 
εἰς ἣν καὶ Πῶρος πρότερον ἔτυχε βασιλεύων. 



Historia Alexandri Magni, Recensio Byzantina poetica (cod. Marcianus 408) 
Line 4914

Αὐτῶν δὲ πάντα τῶν Ἰνδῶν καλῶς οἰκονομήσας, 
ἦλθον τινὲς ἐντόπιοι λέγοντες χρησμολόγοι 
πρὸς τὸν Ἀλέξανδρον Ἰνδεῖς τὰ μέλλοντα συμβῆναι· 
»Μέγιστε πάντων βασιλεῦ, λήψῃ μεγάλας πόλεις, 
ἀναριθμήτους, θαυμαστάς, ὄρη καὶ βασιλείας, 
ἅπερ οὐδεὶς τῶν ἔκπαλαι κατῆρξε βασιλέων. 



Historia Alexandri Magni, Recensio Byzantina poetica (cod. Marcianus 408) 
Line 4934

Ἐκάλουν οὖν ὀνόματα δένδροις δυσὶν ἐκείνοις, 
τοῖς φθεγγομένοις τὴν φωνὴν ὁμοίαν τῶν ἀνθρώπων, 
τὸ πρῶτον ἄρσεν ἥλιον, τὸ θῆλυ δὲ σελήνην· 
Ἰνδεῖς δὲ ταῦτα λέγουσι μουθεαμάθους κλῆσιν. 



Historia Alexandri Magni, Recensio Byzantina poetica (cod. Marcianus 408) 
Line 4961

Εἶθ' οὕτως ἦλθε σὺν αὐτοῖς ἔσωθεν τῶν ἀδύτων, 
ἄνδρας κατέχων ἱκανούς· ὅθεν κελεύει κύκλῳ 
τούτους τὸν τόπον κατιδεῖν καὶ πάντα κατοπτεῦσαι· 
καί τινας δ' ἀπὸ τῶν Ἰνδῶν ἐκεῖ προσεκαλεῖτο 
καταμαθέσθαι παρ' αὐτῶν πάντα τὰ κεκρυμμένα. 



Historia Alexandri Magni, Recensio Byzantina poetica (cod. Marcianus 408) 
Line 4968

Τοῦ δῦναι δὲ τὸν ἥλιον, ὡς σύνηθες, τὸ δένδρον 
φωνὴν Ἰνδῶν ἐφώνησε φθεγγόμενον ἀρίστως· 
οἱ δὲ συνόντες σὺν αὐτῷ, πάντα τὰ λαληθέντα 
καλῶς κατανοήσαντες καὶ σφόδρα φοβηθέντες, 
οὐκ ἤθελον τῷ βασιλεῖ τὰ τῆς φωνῆς μαθέσθαι· 
σύννους οὖν οὗτος γεγονὼς καὶ τούτους κατὰ μόνας 
παραλαβὼν ἠρώτησε· «Τίς ἡ φωνὴ τοῦ δένδρου; 



Historia Alexandri Magni, Recensio Byzantina poetica (cod. Marcianus 408) 
Line 4975

Κρυφίως δ' οὗτοι πρὸς αὐτόν, «Ἀλέξανδρε», προσεῖπον, 
»ἔχεις κακῶς ἐκ τῶν Ἰνδῶν τὸν βίον ἐξελθεῖν σε. 



Historia Alexandri Magni, Recensio Byzantina poetica (cod. Marcianus 408) 
Line 4984

Ἀνατελλούσης δὲ λοιπὸν ὡς ἔθος τῆς σελήνης, 
τὸ δένδρον προσεφώνησεν ἑλληνικῇ τῇ γλώσσῃ· 
»Ἐν Βαβυλῶνί σε θανεῖν, Ἀλέξανδρε, προλέγω· 
σὺ φονευθεὶς ἐκ τῶν Ἰνδῶν καὶ μᾶλλον οὐ δυνήσῃ, 
γῆν Μακεδόνων κατιδεῖν ἢ τὴν Ὀλυμπιάδα. 



Historia Alexandri Magni, Recensio Byzantina poetica (cod. Marcianus 408) 
Line 5006

Ἅμα δὲ τὴν ἀνατολὴν ἡλίου γενομένην 
καὶ τὴν αὐγὴν ἐλθῆναί τε πρὸς τὴν μορφὴν τοῦ δένδρου, 
ὀξεῖα φωνὴ γέγονε λέγουσα διαρρήδην· 
»Ἀλέξανδρε, πεπλήρωται σὸς χρόνος ὁ τοῦ βίου, 
πλὴν οὐ ταφῇς σὺ πρὸς τὴν γῆν τὴν σὴν Μακεδονίαν 
ἢ κατιδεῖν μητέρα σὴν οὐκ ἔχεις, ἀλλ' ἐν ταύτῃ 
τῇ Βαβυλῶνι πρὸς μικρὸν κακῶς καταφθαρήσῃ, 
καὶ μετ' ὀλίγον μήτηρ σή, σύζυγός σου Ῥωξάνη   
μέλλουσιν αὗται παρ' Ἰνδῶν δεινῶς ἀποκτανθῆναι· 
λοιπὸν ἐμὲ τοῖς ἔμπροσθεν πλέον μὴ διενόχλει· 
οὐ γὰρ ἀκούσῃς ἕτερον· ἄπελθε γοῦν τῶν ὧδε. 



Historia Alexandri Magni, Recensio Byzantina poetica (cod. Marcianus 408) 
Line 5010

Περίλυπος δὲ γεγονὼς Ἀλέξανδρος καὶ πάλιν, 
ἐκεῖθεν ἀνεχώρησεν Ἰνδῶν ἐάσας χώραν, 
καὶ πρὸς τὰ Σεμιράμεως ἀνάκτορα κατῆλθεν. 



Historia Alexandri Magni, Recensio Byzantina poetica (cod. Marcianus 408) 
Line 5024

Εἰς Αἴγυπτον ἐρχόμενος σὺν πάσῃ μου δυνάμει, 
κατέμαθον μέσον ὁδοῦ τῶν ἀρετῶν ἐνεῖναι 
βωμὸν ἐν τόποις τοῖς ὑμῶν, ὡς μέγα τοῦτον τέρας· 
ὡσαύτως προσανέμαθον εἶναι Θεοῦ μνημεῖον 
ἐν πεδιάσιν ἰνδικοῖς, μέγιστον ἡρωεῖον, 
ἅπερ ἰδεῖν ἐπιθυμῶ καλῶς καὶ κατοπτεῦσαι, 
καὶ ταῦτα λέξαι πρὸς ὑμᾶς, ὡς πάμπολυν τὸν χρόνον 
ὑμεῖς ἐκυριεύσατε πάσης τῆς γῆς Αἰγύπτου, 
τυραννικῶς κατάρχοντες τόπους τοὺς ἀλλοτρίους. 



Historia Alexandri Magni, Recensio Byzantina poetica (cod. Marcianus 408) 
Line 5031

Εἰ δ' οὔ, πρὸς μάχην ἔρχομαι κακῶς ἐξολοθρεῦσαι 
χώραν ὑμῶν τὴν τῶν Ἰνδῶν αὐτὴν τὴν ἐσωτέραν. 



Historia Alexandri Magni, Recensio Byzantina poetica (cod. Marcianus 408) 
Line 5292

ὁ τῶν Περσῶν καὶ τῶν Ἰνδῶν ὀλέσας δυναστείαν, 
ὁ στήσας πολλὰ τρόπαια Μήδων τε καὶ τῶν Πάρθων, 
καὶ πᾶσαν τὴν Ἀνατολὴν καὶ Δύσιν ἐκπορθήσας, 
ὁ κόσμον ἐκνενικηκώς, χωρὶς αὐτοῦ πολέμου 
καὶ πάσης ἄλλης στρατιᾶς, γέγονας τῆς Κανδάκης 
τῆς χήρας ὑποχείριος· λοιπὸν μὴ φρόνει μέγα. 



Historia Alexandri Magni, Recensio Byzantina poetica (cod. Marcianus 408) 
Line 5317

ἥτις καὶ πρὸς αὐτόν φησιν· «Ἀλέξανδρε, καὶ τοῦτο 
γενναῖον καὶ βασιλικὸν κακῶς προσεβουλεύσω· 
ἀλλὰ μηδὲν ἀγωνιᾷς· μὴ φοβηθῇς, ὦ τέκνον· 
ὥσπερ γὰρ σύ μου τὸν υἱὸν καὶ τὴν γυναῖκα τούτου 
ἐκ τῶν Βεβρύκων ἔσωσας μεγίστῃ σου φρονήσει, 
οὕτω κἀγὼ φυλάξω σε πάλιν ἐπανελθεῖν σε 
πρὸς τὰ στρατόπεδα τὰ σά, φόβον μὴ σχὼν μηδένα, 
σὲ προσφωνοῦσ' Ἀντίγονον· εἰ γὰρ καὶ γνωριοῦσιν 
εἶναί σε τὸν Ἀλέξανδρον, ἀποκτενοῦσι ξίφει, 
ὡς Πῶρον ἀποκτείναντα τὸν τῶν Ἰνδῶν δεσπότην· 
ἡ γὰρ γυνὴ θἀτέρου μου τέκνου, θυγάτηρ Πώρου. 



Historia Alexandri Magni, Recensio Byzantina poetica (cod. Marcianus 408) 
Line 5471

Κατεδουλώσαμεν Ἰνδούς, καὶ βασιλέας τούτων 
σὺν ἡγουμένοις τοῖς αὐτῶν, μεγάλα καυχωμένους, 
ἡττήσαμεν, τῆς ἄνωθεν Προνοίας συμμαχούσης· 
ὅθεν καὶ φόρους παρ' αὐτῶν λαβόντες οὐκ ὀλίγους, 
πρὸς τοὺς Βραχμάνους ἤλθομεν καὶ γῆν τῶν μακροβίων, 
καὶ τούτους κατοπτεύσαντες μεθ' ἱλαροῦ τοῦ τρόπου, 
καὶ πᾶσαν χώραν τὴν αὐτῶν καὶ πάντα βίον τούτων 
ἀναμαθόντες ἀκριβῶς, πολλά δ' ὠφεληθέντες, 
καὶ τούτους καταστήσαντες, μή τι λαβόντες τούτων 
(οὐ πράγματα γὰρ πάρεισιν ἐκείνοις τοῖς ἀνδράσιν,) 




Historia Alexandri Magni, Recensio Byzantina poetica (cod. Marcianus 408) 
Line 5600

Εἶτ' ἀποπλεύσαντες ἡμεῖς τοῦ ποταμοῦ ταχίστως, 
νῆσον μεγάλην εὕρομεν ἀπέχουσαν ἠπείρου 
ἐκμετρητοὺς πρὸς ἑκατὸν εἴκοσι τοὺς σταδίους· 
εἰς ἣν ἐλθόντες εἴδομεν πόλιν αὐτὴν Ἡλίου, 
ἔχουσαν πύργους δώδεκα χρυσέους σμαραγδίνους·   
τὰ τείχη δὲ τῆς πόλεως ἐξ ἰνδικοῦ χρυσίου. 



Historia Alexandri Magni, Recensio Byzantina poetica (cod. Marcianus 408) 
Line 6120

Ἐγράφη δὲ πρὸς ἔτεσι τοῖς ἑξακισχιλίοις 
ἓξ σὺν τοῖς ἐνενήκοντα καὶ τοῖς ὀκτακοσίοις, 
Ἰνδικτιῶνος ἐν αὐτοῖς τρεχούσης ἑνδεκάτης. 



Historia Alexandri Magni, Recensio poetica (recensio R) (1386: 019)
“Διήγησις τοῦ Ἀλεξάνδρου. The tale of Alexander. The rhymed version”, Ed. Holton, D.
Thessalonica: n.p., 1974; Βυζαντινὴ καὶ Νεοελληνικὴ βιβλιοθήκη.
Line 1769

Εἰς ἕξι μέρες τὸ λοιπὸν στὰ ὄρη Ἴνδων σώνουν,   
ἐκεῖ γὰρ ἀπεζεύσασι ὅλοι τους καὶ τεντώνουν. 



Historia Alexandri Magni, Recensio poetica (recensio R) 
Line 1996

Νὰ γράψω ἐπεθύμησα ἐσένα τί ἐγίνη 
στὴν χώραν γὰρ τὴν Ἰνδικήν, ὁπὄκαμαν ἐκεῖνοι. 



Historia Alexandri Magni, Recensio poetica (recensio R) 
Line 2026

Ἐλέφαντας ηὑρίσκαμεν κι ὅλοι καβαλικέψαν, 
στὴ χώρα γὰρ τὴν Ἰνδικὴν ἐπῆγαν κ' ἐπεζέψαν. 



Historia Alexandri Magni, Recensio poetica (recensio R) 
Line 2031

Στὴν Ἰνδικὴν ἐδιάβημαν καὶ περικαρτεροῦμε, 
στὴν ἔρημον κινήσαμε τότες καὶ περπατοῦμε. 



Historia Alexandri Magni, Recensio poetica (recensio R) 
Line 2094

Οἱ ἐντόπιοι μᾶς ὁδηγοῦν κ' ἐπήγαμε μὲ κόπον, 
μὲ βία γὰρ ἐσώσαμε στῆς Ἰνδικῆς τὸν τόπον. 



Historia Alexandri Magni, Recensio poetica (recensio R) 
Line 2095

Κ' ἤλθασι ὀκ τοὺς Ἰνδικοὺς ὀμπρός μου νὰ σταθοῦσι 
κ' εἴπασι· ‘Δῶ ηὑρίσκονται δένδρα ὁποὺ λαλοῦσι. 



Historia Alexandri Magni, Recensio poetica (recensio R) 
Line 2114

Καὶ βασιλεύει ἥλιος, τὰ δένδρη γὰρ λαλοῦσι, 
καὶ ἡ φωνή τους ἰνδική, δὲν ξεύρω τί μιλοῦσι. 


\end{greek}

\section{Claudius Ptolemaeus}
\blockquote[From Wikipedia\footnote{\url{http://en.wikipedia.org/wiki/Ptolemy}}]{Claudius Ptolemy (/ˈtɒləmi/; Greek: Κλαύδιος Πτολεμαῖος, Klaudios Ptolemaios, pronounced [kláwdios ptolɛmɛ́ːos]; Latin: Claudius Ptolemaeus; c. AD 90 – c. AD 168) was a Greco-Roman writer of Alexandria, known as a mathematician, astronomer, geographer, astrologer, and poet of a single epigram in the Greek Anthology.[1][2] He lived in the city of Alexandria in the Roman province of Egypt, wrote in Greek, and held Roman citizenship.[3] Beyond that, few reliable details of his life are known. His birthplace has been given as Ptolemais Hermiou in the Thebaid in an uncorroborated statement by the 14th century astronomer Theodore Meliteniotes.[4] This is very late, however, and there is no other reason to suppose that he ever lived anywhere else than Alexandria,[4] where he died around AD 168.[5]}
\begin{greek}


Claudius Ptolemaeus Math., Apotelesmatica (= Tetrabiblos) (0363: 007)
“Claudii Ptolemaei opera quae exstant omnia, vol. 3.1”, Ed. Boll, F., Boer, E.
Leipzig: Teubner, 1940, Repr. 1957.
Book 2, chapter 3, section 22, line 3

τοῦ δὲ δευτέρου τεταρτημορίου τοῦ κατὰ τὸ νότιον 
μέρος τῆς Μεγάλης Ἀσίας τὰ μὲν ἄλλα μέρη τὰ περι-
έχοντα Ἰνδικὴν Ἀριανὴν Γεδρουσίαν Παρθίαν Μηδίαν   
Περσίδα Βαβυλωνίαν Μεσοποταμίαν Ἀσσυρίαν καὶ τὴν 
θέσιν ἔχοντα πρὸς νοταπηλιώτην τῆς ὅλης οἰκουμένης 
εἰκότως καὶ αὐτὰ συνοικειοῦται μὲν τῷ νοταπηλιωτικῷ 
τριγώνῳ Ταύρου καὶ Παρθένου καὶ Αἰγόκερω· οἰκοδε-
σποτεῖται δὲ ὑπὸ τοῦ τῆς Ἀφροδίτης καὶ τοῦ τοῦ Κρόνου 
ἐπὶ ἑῴων σχημάτων. 



Claudius Ptolemaeus Math., Apotelesmatica (= Tetrabiblos) 
Book 2, chapter 3, section 28, line 6

                                                  τῇ δὲ Παρθένῳ 
καὶ τῷ τοῦ Ἑρμοῦ τὰ περὶ τὴν Βαβυλωνίαν καὶ Μεσοπο-
ταμίαν καὶ Ἀσσυρίαν, διὸ καὶ παρὰ τοῖς ἐνταῦθα τὸ 
μαθηματικὸν καὶ παρατηρητικὸν τῶν πέντε ἀστέρων 
ἐξαίρετον συνέπεσε, τῷ δὲ Αἰγόκερῳ καὶ τῷ τοῦ Κρόνου 
τὰ περὶ τὴν Ἰνδικὴν καὶ Ἀριανὴν καὶ Γεδρουσίαν, ὅθεν 
καὶ τὸ τῶν νεμομένων τὰς χώρας ἐκείνας ἄμορφον καὶ 
ἀκάθαρτον καὶ θηριῶδες. 



Claudius Ptolemaeus Math., Geographia (lib. 1–3) (0363: 009)
“Claudii Ptolemaei geographia, vol. 1.1”, Ed. Müller, K.
Paris: Didot, 1883.
Book 1, chapter 7, section 6, line 5

         Φησὶ γὰρ, «ὅτι καὶ οἱ μὲν τῆς 
Ἰνδικῆς εἰς τὴν Λιμυρικὴν πλέοντες, ὥς φησι   
Διόδωρος ὁ Σάμιος ἐν τῷ τρίτῳ, ἔχουσι τὸν 
Ταῦρον μεσουρανοῦντα καὶ τὴν Πλειάδα κατὰ 
μέσην τὴν κεραίαν· οἱ δ' εἰς τὴν Ἀζανίαν ἀπὸ 
τῆς Ἀραβίας ἀναγόμενοι εὐθύνουσι τὸν πλοῦν 
πρὸς μεσημβρίαν καὶ τὸν Κάνωβον ἀστέρα, 
ὅστις ἐκεῖ λέγεται Ἵππος καί ἐστι νοτιώτατος. 



Claudius Ptolemaeus Math., Geographia (lib. 1-3) 
Book 1, chapter 9, section 1, line 3

Ἔπειτα καὶ κατὰ τὸν μεταξὺ τῶν Ἀρωμά-
των καὶ τῶν Ῥάπτων πλοῦν, «Διογένη μέν 
τινά φησι τῶν εἰς τὴν Ἰνδικὴν πλεόντων ὑπο-
στρέφοντα τὸ δεύτερον, ὅτε ἐγένετο κατὰ τὰ 
Ἀρώματα, ἀπωσθῆναι ἀπαρκτίᾳ, καὶ ἐν δεξιᾷ 
ἔχοντα τὴν Τρωγλοδυτικὴν ἐπὶ ἡμέρας εἴκοσι 
πέντε παραγενέσθαι εἰς τὰς λίμνας, ὅθεν ὁ Νεῖ-
λος ῥεῖ, ὧν ἐστι τὸ τῶν Ῥάπτων ἀκρωτήριον 
ὀλίγῳ νοτιώτερον· Θεόφιλον δέ τινα τῶν εἰς τὴν 
Ἀζανίαν πλεόντων ἀπὸ τῶν Ῥάπτων ἀναχθῆ-
ναι νότῳ, καὶ εἰκοστῇ ἡμέρᾳ ἐληλυθέναι εἰς 
τὰ Ἀρώματα. 



Claudius Ptolemaeus Math., Geographia (lib. 1-3) 
Book 1, chapter 13, section 1, line 3

Στοχάσαιτο δ' ἄν τις τηλικοῦτον εἶναι τὸ   
μῆκος καὶ δι' ὧν ἐκτίθεται διαστημάτων κατὰ 
τὸν πλοῦν τὸν ἀπὸ τῆς Ἰνδικῆς μέχρι τοῦ τῶν 
Σινῶν κόλπου καὶ Καττιγάρων, ἐὰν τὸ παρὰ 
τὰς κολπώσεις καὶ τὰς ἀνωμαλίας τῶν πλῶν 
καὶ ἔτι τὰς θέσεις ἐπιλογίζηται κατὰ συνεγγι-
σμὸν τῶν ἐπιβολῶν. 



Claudius Ptolemaeus Math., Geographia (lib. 1-3) 
Book 1, chapter 14, section 7, line 2

   Ἀλλ' ὁ μὲν διὰ 
τῆς ἀρχῆς τοῦ Ἰνδοῦ ποταμοῦ μεσημβρινὸς 
ὀλίγῳ δυτικώτερός ἐστι τοῦ βορείου τῆς Ταπρο-
βάνης ἀκρωτηρίου κατὰ τὸν Μαρῖνον, ὅπερ 
ἀντίκειται τῷ Κῶρυ· τούτου δ' ἀφέστηκεν ὁ διὰ 
τῶν ἐκβολῶν τοῦ Βαίτιος ποταμοῦ ὡριαῖα δια-
στήματα ηʹ, μοίρας δὲ ρκʹ, καὶ ἔτι ὁ διὰ τῶν ἐκ-
βολῶν τοῦ Βαίτιος τοῦ διὰ τῶν Μακάρων νήσων 
μοίρας εʹ· ὥστε καὶ ὁ μὲν διὰ τοῦ Κῶρυ μεσημ-
βρινὸς ἀπέχει τοῦ διὰ τῶν Μακάρων νήσων μι-
κρῷ πλέον μοιρῶν ρκεʹ, ὁ δὲ διὰ Καττιγάρων   




Claudius Ptolemaeus Math., Geographia (lib. 1-3) 
Book 1, chapter 16, section 1, line 10

Παρῆλθε δέ τινα αὐτὸν καὶ κατὰ τοὺς πε-
ριορισμούς, ὡς ὅταν τὴν μὲν Μυσίαν πᾶσαν ἀπ' 
ἀνατολῶν ὁρίζῃ τῇ Ποντικῇ θαλάσσῃ, τὴν δὲ 
Θρᾴκην ἀπὸ δυσμῶν Μυσίᾳ τῇ ἄνω, καὶ τὴν 
μὲν Ἰταλίαν ἀπ' ἄρκτων μὴ Ῥαιτίᾳ καὶ Νω-
ρικῷ μόνον, ἀλλὰ καὶ Παννονίᾳ, τὴν δὲ Παν-
νονίαν ἀπὸ μεσημβρίας Δαλματίᾳ μόνῃ καὶ 
μηκέτι τῇ Ἰταλίᾳ, καὶ τοὺς μὲν μεσογείους 
Σογδιανοὺς καὶ τοὺς Σάκας γειτνιάζειν ἀπὸ με-
σημβρίας τῇ Ἰνδικῇ, τοὺς δὲ βορειοτέρους τοῦ 
Ἰμάου ὄρους, ὅ ἐστιν ἀρκτικώτατον τῆς Ἰνδι-
κῆς, δύο παραλλήλους τόν τε δι' Ἑλλησπόν-
του καὶ τὸν διὰ Βυζαντίου μὴ γράφῃ διὰ τῶν 
εἰρημένων ἐθνῶν, ἀλλὰ πρώτως τὸν διὰ μέσου 
Πόντου. 



Claudius Ptolemaeus Math., Geographia (lib. 1-3) 
Book 1, chapter 16, section 1, line 11

ριορισμούς, ὡς ὅταν τὴν μὲν Μυσίαν πᾶσαν ἀπ' 
ἀνατολῶν ὁρίζῃ τῇ Ποντικῇ θαλάσσῃ, τὴν δὲ 
Θρᾴκην ἀπὸ δυσμῶν Μυσίᾳ τῇ ἄνω, καὶ τὴν 
μὲν Ἰταλίαν ἀπ' ἄρκτων μὴ Ῥαιτίᾳ καὶ Νω-
ρικῷ μόνον, ἀλλὰ καὶ Παννονίᾳ, τὴν δὲ Παν-
νονίαν ἀπὸ μεσημβρίας Δαλματίᾳ μόνῃ καὶ 
μηκέτι τῇ Ἰταλίᾳ, καὶ τοὺς μὲν μεσογείους 
Σογδιανοὺς καὶ τοὺς Σάκας γειτνιάζειν ἀπὸ με-
σημβρίας τῇ Ἰνδικῇ, τοὺς δὲ βορειοτέρους τοῦ 
Ἰμάου ὄρους, ὅ ἐστιν ἀρκτικώτατον τῆς Ἰνδι-
κῆς, δύο παραλλήλους τόν τε δι' Ἑλλησπόν-
του καὶ τὸν διὰ Βυζαντίου μὴ γράφῃ διὰ τῶν 
εἰρημένων ἐθνῶν, ἀλλὰ πρώτως τὸν διὰ μέσου 
Πόντου. 



Claudius Ptolemaeus Math., Geographia (lib. 1-3) 
Book 1, chapter 17, section 3, line 2

   Καὶ πάλιν 
τὰ Σήμυλλα τὸ τῆς Ἰνδικῆς ἐμπόριον μὴ μόνον 
τοῦ Μαρέως ἀκρωτηρίου δυτικώτερον ὑπ' αὐτοῦ   
τεθειμένον, ἀλλὰ καὶ τοῦ Ἰνδοῦ ποταμοῦ· μό-
νον γὰρ μεσημβρινώτερον ὁμολογεῖται τῶν στο-
μάτων εἶναι τοῦ ποταμοῦ παρά τε τῶν ἐντεῦθεν 
εἰσπλευσάντων καὶ χρόνον πλεῖστον ἐπελθόν-
των τοὺς τόπους καὶ παρὰ τῶν ἐκεῖθεν ἀφικομέ-
νων πρὸς ἡμᾶς, καλούμενον ὑπὸ τῶν ἐγχωρίων 
Τίμουλα· 
         παρ' ὧν καὶ τά τε ἄλλα τὰ 




Claudius Ptolemaeus Math., Geographia (lib. 1-3) 
Book 1, chapter 17, section 4, line 2

τοῦ Μαρέως ἀκρωτηρίου δυτικώτερον ὑπ' αὐτοῦ   
τεθειμένον, ἀλλὰ καὶ τοῦ Ἰνδοῦ ποταμοῦ· μό-
νον γὰρ μεσημβρινώτερον ὁμολογεῖται τῶν στο-
μάτων εἶναι τοῦ ποταμοῦ παρά τε τῶν ἐντεῦθεν 
εἰσπλευσάντων καὶ χρόνον πλεῖστον ἐπελθόν-
των τοὺς τόπους καὶ παρὰ τῶν ἐκεῖθεν ἀφικομέ-
νων πρὸς ἡμᾶς, καλούμενον ὑπὸ τῶν ἐγχωρίων 
Τίμουλα· 
         παρ' ὧν καὶ τά τε ἄλλα τὰ 
περὶ τὴν Ἰνδικὴν μερικώτερον καὶ κατ' ἐπαρ-
χίας ἐμάθομεν καὶ τὰ ταύτης τῆς χώρας ἐνδο-
τέρω μέχρι τῆς Χρυσῆς Χερσονήσου καὶ ἐντεῦ-
θεν ἕως τῶν Καττιγάρων, τὸ μὲν ὅτι πρὸς 
ἀνατολάς ἐστιν ὁ πλοῦς εἰσπλεόντων καὶ πάλιν 
ἐξιόντων πρὸς δυσμὰς συνιστορούντων, τὸ δ' 
ἄτακτον καὶ ἀνώμαλον τοῦ χρόνου τῶν διανύ-
σεων προσομολογούντων, καὶ ὅτι ὑπέρκειται 
τῶν Σινῶν ἥ τε τῶν Σηρῶν χώρα καὶ ἡ μητρό-
πολις, καὶ τὰ ἀνατολικώτερα τούτων ἄγνωστός 




Claudius Ptolemaeus Math., Geographia (lib. 1-3) 
Book 1, chapter 17, section 4, line 17

ἄτακτον καὶ ἀνώμαλον τοῦ χρόνου τῶν διανύ-
σεων προσομολογούντων, καὶ ὅτι ὑπέρκειται 
τῶν Σινῶν ἥ τε τῶν Σηρῶν χώρα καὶ ἡ μητρό-
πολις, καὶ τὰ ἀνατολικώτερα τούτων ἄγνωστός 
ἐστι γῆ λίμνας ἔχουσα ἑλώδεις, ἐν αἷς κάλα-
μοι μεγάλοι φύονται καὶ συνεχεῖς οὕτως, ὥστε 
ἐχομένους αὐτῶν ποιεῖσθαι τὰς διαπεραιώσεις· 
καὶ ὅτι οὐ μόνον ἐπὶ τὴν Βακτριανὴν ἐντεῦθέν 
ἐστιν ὁδὸς διὰ τοῦ Λιθίνου πύργου, ἀλλὰ καὶ 
ἐπὶ τὴν Ἰνδικὴν διὰ Παλιμβόθρων· ἡ δὲ ἀπὸ 
τῆς μητροπόλεως τῶν Σινῶν ἐπὶ τὸν ὅρμον τὰ 
Καττίγαρα πρὸς δυσμάς ἐστι καὶ μεσημβρίαν,   
ὡς διὰ τοῦτο μὴ πίπτειν αὐτὴν κατὰ τὸν διὰ 
τῆς Σήρας καὶ τῶν Καττιγάρων μεσημβρινὸν, 
ἐξ ὧν φησιν ὁ Μαρῖνος, ἀλλὰ κατά τινα τῶν 
ἀνατολικωτέρων. 



Claudius Ptolemaeus Math., Geographia (lib. 4–8) (0363: 014)
“Claudii Ptolemaei geographia, vols. 1–2”, Ed. Nobbe, C.F.A.
Leipzig: Teubner, 1:1843; 2:1845, Repr. 1966.

Claudius Ptolemaeus Math., Geographia (lib. 4-8) 
Book 6, chapter 8, section 2, line 4

Ἡ Καρμανία περιορίζεται ἀπὸ μὲν <ἄρκτων> 
τῇ ἐκτεθειμένῃ μεσημβρινῇ πλευρᾷ τῆς Ἐρήμου Καρ-
μανίας, 
 ἀπὸ δὲ <ἀνατολῶν> Γεδρωσίᾳ παρὰ Περσι-
κὰ ὄρη κατὰ τὴν δι' αὐτῶν ἐπιζευγνυμένην μεσημβρι-
νὴν γραμμὴν, ἀπὸ τοῦ πρὸς τῇ Ἐρήμῳ πέρατος μέχρι 
τοῦ Ἰνδικοῦ πελάγους κατὰ θέσιν 
ἐπέχουσαν μοίρας . 



Claudius Ptolemaeus Math., Geographia (lib. 4-8) 
Book 6, chapter 8, section 6, line 1

                    ......... ϙ<δ> κ<β> 𐅵ʹ· 
 ἀπὸ δὲ <μεσημβρίας> μέρει τοῦ Ἰνδικοῦ πε-
λάγους τῷ μέχρι τοῦ εἰρημένου πέρατος, οὗ ἡ περι-
γραφὴ ἔχει οὕτως·   
Μετὰ τὴν Καρπέλαν ἄκραν ἐν τῷ Παράγοντι 
  κόλπῳ, 
  Κανθάτις πόλις . 



Claudius Ptolemaeus Math., Geographia (lib. 4-8) 
Book 6, chapter 8, section 10, line 1

              ............ <ρ>γ δʹ <κ> ϛʹ· 
 μεθ' ἃ τὸ εἰρημένον μέχρι τοῦ Ἰνδικοῦ πελά-
γους πέρας ἐπέχει μοίρας . 



Claudius Ptolemaeus Math., Geographia (lib. 4-8) 
Book 6, chapter 8, section 16, line 1

               ............ ϙ<ε> γʹ κ<ε> 𐅵ʹ· 
ἐν δὲ τῷ Ἰνδικῷ πελάγει 
  Πάλλα ἢ Πόλλα . 



Claudius Ptolemaeus Math., Geographia (lib. 4-8) 
Book 6, chapter 12, section 6, line 4

                             ..... ρι<γ> <μ>δ γʹ 
  Ἰνδικομορδάνα . 



Claudius Ptolemaeus Math., Geographia (lib. 4-8) 
Book 6, chapter 15, section 1, line 9

      ..................... <ρ>ξ λ<ε> 
ἀπὸ δὲ <μεσημβρίας> μέρει τῆς ἐκτὸς Γάγγου ποτα-
μοῦ Ἰνδικῆς κατὰ τὴν ἐπιζευγνύουσαν τὰ ἐκτεθειμένα 
πέρατα κατὰ παράλληλον γραμμήν. 



Claudius Ptolemaeus Math., Geographia (lib. 4-8) 
Book 6, chapter 16, section 1, line 10

      ..................... <ρ>π λε 
ἀπὸ δὲ <μεσημβρίας> τῷ τε λοιπῷ μέρει τῆς ἐκτὸς 
Γάγγου Ἰνδικῆς διὰ τῆς αὐτῆς παραλλήλου γραμμῆς 
μέχρι πέρατος, οὗ ἡ θέσις ἐπέχει 
μοίρας . 



Claudius Ptolemaeus Math., Geographia (lib. 4-8) 
Book 6, chapter 18, section 1, line 4

Οἱ Παροπανισάδαι περιορίζονται ἀπὸ μὲν 
<δύσεως> Ἀρείᾳ παρὰ τὴν εἰρημένην πλευρὰν, ἀπὸ δὲ   
<ἄρκτων> τῷ ἐκτεθειμένῳ μέρει τῆς Βακτριανῆς, ἀπὸ 
δὲ <ἀνατολῶν>20 Ἰνδικῆς μέρει τῇ ἀπὸ τῶν πηγῶν τοῦ 
Ὤξου ποταμοῦ διὰ τῶν Καυκασίων ὀρῶν ἐκβαλλομένῃ 
μεσημβρινῇ γραμμῇ μέχρι πέρατος, 
οὗ ἡ θέσις ἐπέχει μοίρας . 



Claudius Ptolemaeus Math., Geographia (lib. 4-8) 
Book 6, chapter 20, section 1, line 4

Ἡ Ἀραχωσία περιορίζεται ἀπὸ μὲν <δύσεως> 
Δραγγιανῇ, ἀπὸ δὲ <ἄρκτων> Παροπανισάδαις κατὰ 
τὰς ἐκτεθειμένας αὐτῶν πλευρὰς, ἀπὸ δὲ <ἀνατολῶν>20 
Ἰνδικῆς μέρει κατὰ μεσημβρινὴν γραμμὴν τὴν ἐκβαλ-
λομένην ἀπὸ τοῦ πρὸς τοῖς Παροπανισάδαις ὁρίου 
μέχρι πέρατος, οὗ ἡ θέσις ἐπέχει 
μοίρας . 



Claudius Ptolemaeus Math., Geographia (lib. 4-8) 
Book 6, chapter 20, section 2, line 2

Ἐμβάλλει δὲ εἰς τὴν χώραν ποταμὸς ἀπὸ τοῦ   
Ἰνδοῦ ἐκτρεπόμενος, οὗ αἱ πηγαὶ 
ἐπέχουσι μοίρας . 



Claudius Ptolemaeus Math., Geographia (lib. 4-8) 
Book 6, chapter 21, section 1, line 5

Ἡ Γεδρωσία περιορίζεται ἀπὸ μὲν <δύσεως> 
Καρμανίᾳ κατὰ τὴν ἐκτεθειμένην μέχρι θαλάσσης με-
σημβρινὴν γραμμὴν, ἀπὸ δὲ ἄρκτων Δραγγιανῇ καὶ 
Ἀραχωσίᾳ παρὰ τὰς διωρισμένας αὐτῶν μεσημβρινὰς 
γραμμὰς, ἀπὸ δὲ <ἀνατολῶν>20 Ἰνδικῆς μέρει παρὰ τὸν 
Ἰνδὸν ποταμὸν κατὰ τὴν ἐκβαλλομένην γραμμὴν ἀπὸ 
τοῦ πρὸς τῇ Ἀραχωσίᾳ ὁρίου μέχρι τοῦ ἐπὶ θαλάς-
σης πέρατος, ὃ ἐπέχει μοίρας . 



Claudius Ptolemaeus Math., Geographia (lib. 4-8) 
Book 6, chapter 21, section 1, line 9

                                    .... <ρ>θ <κ> 
ἀπὸ δὲ <μεσημβρίας> μέρει τοῦ Ἰνδικοῦ πελάγους, οὗ 
ἡ περιγραφὴ ἔχει οὕτως· 
μετὰ τὸ πρὸς τῇ Καρμανίᾳ πέρας 
  Ἀράβιος ποταμοῦ ἐκβολαί . 



Claudius Ptolemaeus Math., Geographia (lib. 4-8) 
Book 6, chapter 21, section 3, line 5

      ..................... ρι<γ> <κ>ϛ 𐅵ʹ 
ἀφ' ὧν εἰς τὸν Ἰνδὸν ἐμβάλλουσί τινες ποταμοὶ, ὧν 
τοῦ ἑνὸς ἡ πηγή . 



Claudius Ptolemaeus Math., Geographia (lib. 4-8) 
Book 6, chapter 21, section 4, line 9

Τὰ μὲν οὖν ἐπὶ θαλάς-
σῃ τῆς χώρας κατέχουσιν <Ἀρβιτῶν> κῶμαι, 
τὰ δὲ παρὰ τὴν Καρμανίαν <Παρσίδαι> 
(ἢ Παρσίραι), 
τὰ δὲ παρὰ τὴν Ἀραχωσίαν <Μουσαρναῖοι>, 
ἡ δὲ μέση τῆς χώρας πᾶσα κα-
λεῖται <Παραδηνή>, 
καὶ ὑπ' αὐτὴν <Παρισιηνή>, 
μεθ' ἣν τὰ πρὸς τῷ Ἰνδῷ κατέ-
χουσι <Ῥάμναι>· 
Πόλεις δὲ καὶ κῶμαι τῆς Γεδρωσίας καταλέγον-
  ται αἵδε· 
  Κοῦνι . 



Claudius Ptolemaeus Math., Geographia (lib. 4-8) 
Book 7, chapter C, section 1, line 1

[Πίναξ δέκατος] 
Ἰνδικῆς τῆς ἐντὸς Γάγγου ποταμοῦ· 
 [Πίναξ ἑνδέκατος] 
Ἰνδικῆς τῆς ἐκτὸς Γάγγου ποταμοῦ· 
Σινῶν· 
 [Πίναξ δωδέκατος] 
Ταπροβάνης νήσου καὶ τῶν περὶ αὐτήν· 
 Ὑπογραφὴ κεφαλαιώδης τοῦ τῆς οἰκου-
μένης πίνακος· 
Κρικωτῆς σφαίρας μετὰ τῆς οἰκουμένης κατα-
 γραφή· 




Claudius Ptolemaeus Math., Geographia (lib. 4-8) 
Book 7, chapter 1, section T, line 1

Τῆς ἐντὸς Γάγγου Ἰνδικῆς θέσις. 




Claudius Ptolemaeus Math., Geographia (lib. 4-8) 
Book 7, chapter 1, section 1, line 1

Ἡ ἐντὸς Γάγγου ποταμοῦ Ἰνδικὴ περιορίζεται 
ἀπὸ μὲν <δύσεως> Παροπανισάδαις καὶ Ἀραχωσίᾳ καὶ 
Γεδρωσίᾳ παρὰ τὰς ἐκτεθειμένας αὐτῶν ἀνατολικὰς 
πλευρὰς, ἀπὸ δὲ <ἄρκτων> Ἰμάῳ ὄρει παρὰ τοὺς ὑπερ-
κειμένους αὐτοῦ Σογδιαίους καὶ Σάκας· ἀπὸ δὲ <ἀνα-
τολῶν> τῷ Γάγγῃ ποταμῷ, ἀπὸ δὲ <μεσημβρίας> καὶ 
ἔτι <δύσεως> μέρει τοῦ Ἰνδικοῦ πελάγους, οὗ ἡ παρα-
λία ἔχει περιγραφὴν τοιαύτην· 
 Συραστρηνῆς ἐν κόλπῳ, καλουμένῳ Κάνθι, 
ναύσταθμον ὅρμος . 



Claudius Ptolemaeus Math., Geographia (lib. 4-8) 
Book 7, chapter 1, section 2, line 3

                     ........... ρ<θ> 𐅵ʹ <κ> 
τὸ δυσμικώτατον τοῦ Ἰνδοῦ ποταμοῦ 
στόμα, ὃ καλεῖται Σάγαπα . 



Claudius Ptolemaeus Math., Geographia (lib. 4-8) 
Book 7, chapter 1, section 2, line 5

                                ..... ρ<ι> γʹ ι<θ> 𐅵ʹγ 
τὸ ἐφεξῆς αὐτοῦ τοῦ Ἰνδοῦ, ὃ κα-
λεῖται Σίνθων . 



Claudius Ptolemaeus Math., Geographia (lib. 4-8) 
Book 7, chapter 1, section 19, line 2

                                         . ρμ<η> 𐅵ʹ <ι>η δʹ· 
 Ὄρη δὲ ὀνομάζεται ἐν τῷ καλουμένῳ τμή-
ματι τῆς Ἰνδικῆς τά τε Ἀπόκοπα, ἃ καλεῖται Ποιναὶ 
Θεῶν, ὧν τὰ πέρατα ἐπέχει μοίρας. 



Claudius Ptolemaeus Math., Geographia (lib. 4-8) 
Book 7, chapter 1, section 26, line 1

Ἡ δὲ τάξις τῶν εἰς τὸν Ἰνδὸν ῥεόντων πο-
ταμῶν ἀπὸ τοῦ Ἰμάου ὄρους οὕτως ἔχει·   
Κώα ποταμοῦ πηγαί . 



Claudius Ptolemaeus Math., Geographia (lib. 4-8) 
Book 7, chapter 1, section 26, line 5

                            ...... ρκ<β> 𐅵ʹ λ<ϛ> 
τοῦ Ἰνδοῦ ποταμοῦ πηγαί . 



Claudius Ptolemaeus Math., Geographia (lib. 4-8) 
Book 7, chapter 1, section 27, line 2

                            ...... ρλ<α> λ<ε> 𐅵ʹ· 
 Ζαράδρου ποταμοῦ πηγαί ρλ<β> λ<ϛ> 
συμβολὴ Κώα καὶ Ἰνδοῦ . 



Claudius Ptolemaeus Math., Geographia (lib. 4-8) 
Book 7, chapter 1, section 27, line 4

                                ... ρκ<β> 𐅵ʹ λ<α> γοʹ 
συμβολὴ Ζαράδρου καὶ Ἰνδοῦ . 



Claudius Ptolemaeus Math., Geographia (lib. 4-8) 
Book 7, chapter 1, section 28, line 1

                                    .. ρκ<ϛ> 𐅵ʹ λ<α> 𐅵ʹ 
συμβολὴ Βιδάσπου καὶ Σανδαβάλ ρκ<ϛ> γοʹ λ<β> γοʹ 
 ἐκτροπὴ ἀπὸ τοῦ Ἰνδοῦ 
ποταμοῦ εἰς τὸ Οὐΐνδιον ὄρος . 



Claudius Ptolemaeus Math., Geographia (lib. 4-8) 
Book 7, chapter 1, section 28, line 4

                          .......... ρκ<ζ> κ<ζ> 
ἐκτροπὴ τοῦ Ἰνδοῦ εἰς τὴν Ἀραχω-
σίαν . 



Claudius Ptolemaeus Math., Geographia (lib. 4-8) 
Book 7, chapter 1, section 28, line 9

                          ......... ρι<ε> κ<δ> 𐅵ʹ 
ἐκτροπὴ τοῦ Ἰνδοῦ εἰς τὰ Ἄρβιτα ὄρη ρι<ζ> κ<ε> ϛʹ 
ἐκτροπὴ τοῦ Ἰνδοῦ εἰς τοὺς Παρο-
πανισάδας . 



Claudius Ptolemaeus Math., Geographia (lib. 4-8) 
Book 7, chapter 1, section 28, line 12

            ................ ρκ<δ> 𐅵ʹ λ<α> γʹ 
ἐκτροπὴ τοῦ Ἰνδοῦ εἰς τὸ Σάγαπα 
στόμα . 



Claudius Ptolemaeus Math., Geographia (lib. 4-8) 
Book 7, chapter 1, section 28, line 14

        ................... ρι<γ> γοʹ κ<γ> δʹ 
ἀπὸ δὲ τοῦ Σάγαπα εἰς τὸν Ἰνδὸν ρι<α> κ<α> 𐅵ʹ 
ἐκτροπὴ τοῦ Ἰνδοῦ εἰς τὸ Χρυσοῦν   
στόμα . 



Claudius Ptolemaeus Math., Geographia (lib. 4-8) 
Book 7, chapter 1, section 28, line 17

        ................... ρι<β> 𐅵ʹ κ<β> 
ἐκτροπὴ τοῦ Ἰνδοῦ εἰς τὸ Χαρίφου 
στόμα . 



Claudius Ptolemaeus Math., Geographia (lib. 4-8) 
Book 7, chapter 1, section 42, line 7

Ἡ δὲ τῶν ἐν τούτῳ τῷ τμήματι χωρῶν καὶ 
τῶν ἐν αὐτῷ πόλεων ἢ κωμῶν τάξις ἔχει τὸν τρόπον 
τοῦτον· ὑπὸ μὲν γὰρ τὰς τοῦ 
Κώα πηγὰς ἵδρυνται <Λαμβάται>, 
καὶ ἡ ὀρεινὴ αὐτῶν ἀνατείνει μέχρι τῆς τῶν Κωμηδῶν, 
ὑπὸ δὲ τὰς τοῦ Σουάστου ἡ <Σουαστηνή>, 
ὑπὸ δὲ τὰς τοῦ Ἰνδοῦ <Δαράδραι>, 
καὶ ἡ ὀρεινὴ αὐτῶν ὑπέρκειται· 
ὑπὸ δὲ τὰς τοῦ Βιδάσπου καὶ τοῦ 
Σανδαβὰλ καὶ τοῦ Ἄδριος πηγὰς ἡ <Κασπειρία>, 
ὑπὸ δὲ τὰς Βιβάσιος καὶ τοῦ Ζαράδρου καὶ τοῦ Δια-
μούνα καὶ τοῦ Γάγγου ἡ <Κυλινδρινή>, 
καὶ ὑπὸ μὲν τοὺς Λαμβάτας καὶ 
τὴν Σουαστηνὴν ἡ <Γωρυαῖα>, 
καὶ πόλεις αἵδε· 
Καίσανα . 



Claudius Ptolemaeus Math., Geographia (lib. 4-8) 
Book 7, chapter 1, section 44, line 1

              ............. ρ<κ> 𐅵ʹ <λ>β 𐅵ʹ· 
μεταξὺ δὲ τοῦ Σουάστου καὶ τοῦ Ἰνδοῦ <Γαν-
  δάραι> καὶ πόλεις αἵδε· 
  Προκλαΐς . 



Claudius Ptolemaeus Math., Geographia (lib. 4-8) 
Book 7, chapter 1, section 45, line 1

             .............. ρκ<δ> γʹ <λ>γ γʹ·   
μεταξὺ δὲ τοῦ Ἰνδοῦ καὶ τοῦ Βιδάσπου πρὸς μὲν 
  τῷ Ἰνδῷ ἡ <Ἄρσα> χώρα καὶ πόλεις αἵδε· 
  Ἰθάγουρος . 



Claudius Ptolemaeus Math., Geographia (lib. 4-8) 
Book 7, chapter 1, section 49, line 5

             .............. ρκ<η> <λ> 𐅵ʹ 
  Ἀρδόνη ............... ρ<κ>ϛ δʹ <λ> ϛʹ 
  Ἰνδάβαρα . 



Claudius Ptolemaeus Math., Geographia (lib. 4-8) 
Book 7, chapter 1, section 55, line 2

Πάλιν ἡ μὲν παρὰ τὸ λοιπὸν μέρος τοῦ Ἰν-
δοῦ πᾶσα καλεῖται κοινῶς μὲν <Ἰνδοσκυθία>, 
ταύτης δὲ ἡ μὲν παρὰ τὸν διαμε-
ρισμὸν τῶν στομάτων <Παταληνή>, 
καὶ ἡ ὑπερκειμένη αὐτῆς <Ἀβιρία>, 
ἡ δὲ περὶ τὰ στόματα τοῦ Ἰνδοῦ 
καὶ ἡ περὶ τὸν Κάνθι κόλπον <Συραστρηνή>· 
καὶ πόλεις τῆς Ἰνδοσκυθίας αἵδε· 
ἀπὸ μὲν <δύσεως> τοῦ ποταμοῦ ἄποθεν· 
Ἀρτοάρτα . 



Claudius Ptolemaeus Math., Geographia (lib. 4-8) 
Book 7, chapter 1, section 62, line 1

Τῆς δὲ Ἰνδοσκυθίας τὰ ἀπὸ ἀνατολῶν, τὰ 
μὲν ἀπὸ θαλάσσης κατέχει ἡ <Λαρικὴ> χώρα, ἐν ᾗ με-
σόγειοι ἀπὸ μὲν δύσεως τοῦ Ναμάδου ποταμοῦ πό-
λις ἥδε· 
  Βαρύγαζα ἐμπόριον . 



Claudius Ptolemaeus Math., Geographia (lib. 4-8) 
Book 7, chapter 1, section 64, line 4

Τὰ δὲ ὑπερ-
κείμενα κατανέμονται <Πουλῖνδαι Ἀγριοφάγοι>, 
καὶ ὑπὲρ αὐτοὺς ἔτι <Χατριαῖοι>, 
ἐν οἷς ἀπὸ δύσεως καὶ ἀπ' ἀνατολῶν τοῦ Ἰνδοῦ πο-
ταμοῦ πόλεις αἵδε· 
  Νιγρανίγραμμα . 



Claudius Ptolemaeus Math., Geographia (lib. 4-8) 
Book 7, chapter 1, section 83, line 6

           ............... <ρ>κ <κ> δʹ 
  Ταβασώ ............... ρκ<α> 𐅵ʹ <κ> γοʹ 
  Ἴνδη . 



Claudius Ptolemaeus Math., Geographia (lib. 4-8) 
Book 7, chapter 1, section 94, line 1

Νῆσοι δὲ παράκεινται τῷ ἐκκειμένῳ τῆς Ἰνδικῆς 
  μέρει, ἐν μὲν τῷ Κάνθι κόλπῳ, 
  Βαράκη . 



Claudius Ptolemaeus Math., Geographia (lib. 4-8) 
Book 7, chapter 2, section T, line 1

Τῆς ἐκτὸς Γάγγου Ἰνδικῆς θέσις. 




Claudius Ptolemaeus Math., Geographia (lib. 4-8) 
Book 7, chapter 2, section 1, line 1

Ἡ ἐκτὸς Γάγγου Ἰνδικὴ περιορίζεται ἀπὸ 
μὲν <δύσεως> τῷ Γάγγῃ ποταμῷ, ἀπὸ δὲ <ἄρκτων> 
τοῖς ἐκτεθειμένοις μέρεσι τῆς τε Σκυθίας καὶ τῆς 
Σηρικῆς, ἀπὸ δὲ <ἀνατολῶν> τοῖς τε Σίναις κατὰ τὴν 
ἀπὸ τοῦ πρὸς τῇ Σηρικῇ πέρατος ἐκβαλλομένην με-
σημβρινὴν γραμμὴν μέχρι τοῦ καλουμένου Μεγάλου 
κόλπου, καὶ αὐτῷ τῷ κόλπῳ, ἀπὸ δὲ <μεσημβρίας> 
τῷ τε Ἰνδικῷ πελάγει καὶ μέρει τῆς Πρασώδους θα-
λάσσης, ἥτις ἀπὸ τῆς Μενουθιάδος νήσου διατείνει 
κατὰ παράλληλον γραμμὴν μέχρι τῶν ἀντικειμένων 




Claudius Ptolemaeus Math., Geographia (lib. 4-8) 
Book 7, chapter 2, section 18, line 5

Πάλιν δὲ μεταξὺ τοῦ Βηπύῤῥου ὄρους καὶ 
τῶν Δαμάσσων ὀρέων τὰ μὲν ἀρκτικώτατα κατέχου-
σιν <Ἀνινάχαι> (ἢ Ἀμι-
       νάχαι), 
ὑπὸ δὲ τούτους <Ἰνδαπρᾶθαι>, 
μεθ' οὓς <Ἰβηρίγγαι>, 
εἶτα <Δαβάσαι> (ἢ Δα-
        μάσσαι?). 



Claudius Ptolemaeus Math., Geographia (lib. 4-8) 
Book 7, chapter 2, section 20, line 6

ὃ σημαίνει <γυμνῶν κόσμος>· 
 καὶ μεταξὺ τῶν Δαμάσσων ὀρῶν καὶ τοῦ πρὸς 
τοὺς Σίνας ὁρίου ἀρκτικώτατοι <Κάκοβαι>· 
καὶ ὑπὸ τούτους <Βασανᾶραι>· 
 εἶτα ἡ <Χαλκῖτις> χώρα, 
ἐν ᾗ πλεῖστα μέταλλα χαλκοῦ· 
ὑπὸ δὲ ταύτην μέχρι τοῦ Με-
γάλου κόλπου <Κουδοῦται> 
καὶ <Βάῤῥαι> 
μεθ' οὓς <Ἰνδοί>, 
εἶτα <Δοᾶναι> 
παρὰ τὸν ὁμώνυμον ποταμόν· 
 καὶ μετὰ τούτους ὀρεινὴ συνάπτουσα τῇ 
τῶν <Λῃστῶν> χώρᾳ, τίγρεις ἔχουσα καὶ ἐλέφαντας· 
αὐτοὺς δὲ τοὺς τῶν Λῃστῶν χώραν κατανεμομένους 
θηριώδεις τε εἶναι λέγουσι καὶ ἐν σπηλαίοις οἰκοῦν-
τας καὶ τὸ δέρμα ἔχοντας παραπλήσιον <ἵππων> πο-
ταμίων, ὡς μὴ διακόπτεσθαι βέλεσιν. 



Claudius Ptolemaeus Math., Geographia (lib. 4-8) 
Book 7, chapter 2, section 26, line 2

Νῆσοι δὲ φέρονται κατὰ τοῦ ἐκκειμένου τῆς Ἰν-
  δικῆς τμήματος αἵδε· 
  Βαζακάτα . 



Claudius Ptolemaeus Math., Geographia (lib. 4-8) 
Book 7, chapter 3, section 1, line 4

Οἱ Σῖναι περιορίζονται ἀπὸ μὲν <ἄρκτων> τῷ 
ἐκτεθειμένῳ μέρει τῆς Σηρικῆς, ἀπὸ δὲ <ἀνατολῶν> 
καὶ <μεσημβρίας> ἀγνώστῳ γῇ, ἀπὸ δὲ <δύσεως> τῇ 
ἐκτὸς Γάγγου Ἰνδικῇ κατὰ τὴν διωρισμένην μέχρι τοῦ 
Μεγάλου κόλπου γραμμὴν, καὶ αὐτῷ τῷ Μεγάλῳ 
κόλπῳ καὶ τοῖς ἐφεξῆς αὐτῷ κειμένοις, τῷ τε καλου-
μένῳ Θηριῴδει καὶ τῷ τῶν Σινῶν, ὃν περιοικοῦσιν 
<Ἰχθυοφάγοι Αἰθίοπες>, κατὰ περιγραφὴν τοι-
αύτην· 
 μετὰ τὸ πρὸς τῇ Ἰνδικῇ τοῦ κόλπου ὅριον, 
Ἀσπίθρα ποταμοῦ ἐκβολαί . 



Claudius Ptolemaeus Math., Geographia (lib. 4-8) 
Book 7, chapter 4, section 1, line 1

Τῷ δὲ Κῶρυ ἀκρωτηρίῳ, τῷ τῆς Ἰνδικῆς, ἀν-
τίκειται τὸ τῆς Ταπροβάνης νήσου ἄκρον, ἥτις ἐκαλεῖ-
το πάλαι Σιμούνδου, νῦν δὲ Σαλίκη· καὶ οἱ κατέχον-
τες αὐτὴν κοινῶς <Σάλαι>, μαλλοῖς γυναικείοις εἰς 
ἅπαν ἀναδεδεμένοι. 



Claudius Ptolemaeus Math., Geographia (lib. 4-8) 
Book 7, chapter 5, section 2, line 6

Τῆς γῆς τὸ κατὰ τὴν ἡμετέραν οἰκουμένην 
μέρος περιορίζεται ἀπὸ μὲν <ἀνατολῶν> ἀγνώστῳ γῇ 
τῇ παρακειμένῃ τοῖς ἀνατολικοῖς ἔθνεσι τῆς Μεγάλης 
Ἀσίας Σίναις τε καὶ τοῖς ἐν τῇ Σηρικῇ, ἀπὸ δὲ <με-
σημβρίας> ὁμοίως ἀγνώστῳ γῇ νῇ περικλειούσῃ τὸ 
Ἰνδικὸν πέλαγος καὶ τῇ περιεχούσῃ τὴν ἀπὸ μεσημ-
βρίας τῆς Λιβύης Αἰθιοπίαν Ἀγίσυμβα χώραν καλου-
μένην, ἀπὸ δὲ <δυσμῶν> τῇ τε ἀγνώστῃ γῇ τῇ περι-
λαμβανούσῃ τὸν Αἰθιοπικὸν κόλπον τῆς Λιβύης, καὶ 
τῷ ἐφεξῆς δυτικῷ Ὠκεανῷ, παρακειμένῳ τοῖς τῆς Λι-
βύης καὶ τῆς Εὐρώπης δυσμικωτάτοις μέρεσιν, ἀπ' 
<ἄρκτων> δὲ τῷ συνημμένῳ ὠκεανῷ, τῷ περιέχοντι τὰς 
Βρετανικὰς νήσους καὶ τὰ βορειότατα τῆς Εὐρώπης, 
καλουμένῳ δὲ Δουηκαλυδονίῳ τε καὶ Σαρματικῷ καὶ 
τῇ ἀγνώστῳ γῇ παρακειμένῃ ταῖς ἀρκτικωτάταις

χώ-



Claudius Ptolemaeus Math., Geographia (lib. 4-8) 
Book 7, chapter 5, section 4, line 4

Ἡ δὲ Ὑρκανία ἡ καὶ Κασπία θάλασσα 
πάντοθεν ὑπὸ τῆς γῆς περικέκλεισται νήσῳ κατὰ τὸ   
ἀντικείμενον παραπλησίως· ὁμοίως δὲ καὶ ἡ περὶ τὸ 
Ἰνδικὸν πέλαγος πᾶσα μετὰ τῶν συνημμένων αὐτῇ 
κόλπων παρά τε τὸν Ἀράβιον κόλπον καὶ τὸν Περσι-
κὸν, καὶ τὸν Γαγγητικὸν, καὶ τὸν ἰδίως καλούμενον 
Μέγαν κόλπον, περιεχομένη καὶ αὐτὴ πάντοθεν ὑπὸ 
τῆς γῆς. 



Claudius Ptolemaeus Math., Geographia (lib. 4-8) 
Book 7, chapter 5, section 5, line 5

Διὸ καὶ τῶν τριῶν ἠπείρων ἡ μὲν <Ἀσία> συν-
άπτει τῇ τε Λιβύῃ, καὶ διὰ τοῦ κατὰ τὴν Ἀραβίαν 
αὐχένος, ὃς καὶ χωρίζει τὴν καθ' ἡμᾶς θάλασσαν ἀπὸ 
τοῦ Ἀραβικοῦ κόλπου, καὶ διὰ τῆς περιεχούσης τὸ 
Ἰνδικὸν πέλαγος ἀγνώστου γῆς. 



Claudius Ptolemaeus Math., Geographia (lib. 4-8) 
Book 7, chapter 5, section 9, line 3

Ὁμοίως δὲ καὶ τῶν εἰρημένων ἐμπεριέχεσθαι 
τῇ γῇ <θαλασσῶν>, πρώτη μέν ἐστι μεγέθει πάλιν ἡ 
κατὰ τὸ Ἰνδικὸν πέλαγος, δευτέρα δὲ ἡ καθ' ἡμᾶς, 
τρίτη δὲ ἡ Ὑρκανία ἡ καὶ Κασπία. 



Claudius Ptolemaeus Math., Geographia (lib. 4-8) 
Book 8, chapter 1, section 3, line 2

   Παρὰ γὰρ ταύτην 
τὴν αἰτίαν τὸ μὲν Ἰνδικὸν πέλαγος μετὰ τὴν Ταπρο-
βάνην ἐπὶ τὰς ἄρκτους ἀπέστρεψαν, ἐνστάντος αὐτοῖς 
τοῦ πίνακος πρὸς τὴν ἐπὶ τὰς ἀνατολὰς προχώρησιν, 
ἐπεὶ μηδὲν εἶχε τοιοῦτον ἐπὶ τῆς ὑπερκειμένης κατὰ τὸ 
βόρειον Σκυθίας ἀντιπαραγράφειν. 



Claudius Ptolemaeus Math., Geographia (lib. 4-8) 
Book 8, chapter 1, section 4, line 4

Τὸν δὲ δυτικὸν Ὠκεανὸν ἐπὶ τὰς ἀνατολὰς 
ἀπέστρεψαν πάλιν, ἐνστάντος αὐτοῖς τοῦ πίνακος ἐπὶ 
τὴν μεσημβρινὴν διάστασιν, ἐπεὶ μηδ' ἐνταῦθα τὸ τῆς 
ἐντὸς Λιβύης βάθος ἢ τὸ τῆς Ἰνδικῆς εἶχέ τι δυνάμε-
νον κατὰ τὸ συνεχὲς ἀντιπαρασταθῆναι τῇ δυτικῇ 
παραλίῳ· ὡς καὶ διὰ τὰ τοιαῦτα τὴν περὶ τοῦ περιῤ-
ῥεῖσθαι τὴν γῆν ὅλην τῷ Ὠκεανῷ δόξαν ἄρξασθαι 
μὲν ἀπὸ γραφικῶν ἁμαρτημάτων, καταστρέψαι δὲ εἰς 
ἀσύστατον ἱστορίαν. 



Claudius Ptolemaeus Math., Geographia (lib. 4-8) 
Book 8, chapter 16, section 2, line 3

Περιορίζεται δὲ ὁ πίναξ ἀπὸ μὲν <ἀνατολῶν> 
Ἀραβίῳ κόλπῳ καὶ Ἐρυθρᾷ θαλάσσῃ καὶ Βαρβαρικῷ 
πελάγει καὶ μέρει τοῦ Ἰνδικοῦ πελάγους, ἀπὸ δὲ <με-
σημβρίας> ἀγνώστῳ γῇ, καὶ ἀπὸ <δύσεως> ἀγνώστῳ 
γῇ καὶ δυτικῷ Ὠκεανῷ, ἀπὸ δὲ <ἄρκτων> ταῖς τε δυσὶ 
Μαυριτανίαις καὶ Ἀφρικῇ καὶ Κυρηναϊκῇ καὶ Αἰγύπτῳ. 



Claudius Ptolemaeus Math., Geographia (lib. 4-8) 
Book 8, chapter 22, section 2, line 2

Περιορίζεται δὲ ὁ πίναξ ἀπὸ μὲν <ἀνατολῶν> 
Γεδρωσίᾳ καὶ Ἰνδικῷ πελάγει, ἀπὸ δὲ <μεσημβρίας> 
αὐτῷ τῷ Ἰνδικῷ πελάγει καὶ τῇ Ἐρυθρᾷ θαλάσσῃ, 
ἀπὸ δὲ <δύσεως> Ἀραβικῷ κόλπῳ, ἀπὸ δὲ <ἄρκτων>   
ταῖς δυσὶν Ἀραβίαις τῇ τε Πετραίᾳ καὶ τῇ Ἐρήμῳ καὶ 
τῷ Περσικῷ κόλπῳ καὶ τῇ Ἐρήμῳ Καρμανίᾳ. 



Claudius Ptolemaeus Math., Geographia (lib. 4-8) 
Book 8, chapter 23, section 2, line 3

Περιορίζεται δὲ ὁ πίναξ ἀπὸ μὲν <ἀνατολῶν> 
τῇ ἐντὸς Ἰμάου ὄρους Σκυθίᾳ, ἀπὸ δὲ <μεσημβρί-  
ας> τῷ τε ὑπὲρ τὴν ἐντός τε καὶ ἐκτὸς Γάγγου Ἰνδικὴν 
Ἰμάῳ ὄρει καὶ Παροπανισάδαις καὶ Ἀρείᾳ τῇ τε Παρ-
θίᾳ καὶ μέρει τῆς Ὑρκανίας θαλάσσης, ἀπὸ δὲ <δύ-
σεως> μέρει τε Μηδίας καὶ Ὑρκανίας θαλάσσης καὶ 
τῇ ἐν Ἀσίᾳ Σαρματίᾳ, ἀπὸ δὲ <ἄρκτων> ἀγνώστῳ γῇ. 



Claudius Ptolemaeus Math., Geographia (lib. 4-8) 
Book 8, chapter 24, section 2, line 3

Περιορίζεται δὲ ὁ πίναξ ἀπὸ μὲν <ἄρκτων> καὶ 
<ἀνατολῶν> ἀγνώστῳ γῇ, ἀπὸ δὲ <μεσημβρίας> Σί-
ναις τε καὶ μέρει τῆς Ἰνδικῆς, ἀπὸ δὲ <δύσεως> τοῖς τε 
Σάκαις καὶ τῇ ἐντὸς Ἰμάου ὄρους Σκυθίᾳ. 



Claudius Ptolemaeus Math., Geographia (lib. 4-8) 
Book 8, chapter 25, section 2, line 2

Περιορίζεται δὲ ὁ πίναξ ἀπὸ μὲν <ἀνατολῶν>20 
Ἰνδικῇ, ἀπὸ δὲ <μεσημβρίας>20 Ἰνδικῷ πελάγει, ἀπὸ 
δὲ <δύσεως> ταῖς δυσὶ Καρμανίαις καὶ Παρθίᾳ, ἀπὸ 
δὲ <ἄρκτων> Μαργιανῇ καὶ Βακτριανῇ. 



Claudius Ptolemaeus Math., Geographia (lib. 4-8) 
Book 8, chapter 26, section 1, line 2

Ὁ δέκατος πίναξ τῆς Ἀσίας περιέχει 
  τὴν ἐντὸς Γάγγου τοῦ ποταμοῦ <Ἰνδικὴν> 
σὺν ταῖς παρακειμέναις αὐτῇ νήσοις. 



Claudius Ptolemaeus Math., Geographia (lib. 4-8) 
Book 8, chapter 26, section 2, line 2

Περιορίζεται δὲ ὁ πίναξ ἀπὸ μὲν <ἀνατολῶν> 
τῇ ἐκτὸς Γάγγου Ἰνδικῇ, ἀπὸ δὲ <μεσημβρίας> μέρει 
τε τοῦ Γαγγητικοῦ κόλπου καὶ Ἰνδικῷ πελάγει, ἀπὸ 
δὲ <δύσεως> Γεδρωσία καὶ Ἀραχωσίᾳ καὶ τοῖς Παρο-
πανισάδαις, ἀπὸ δὲ <ἄρκτων> τῷ ὑπὸ τοὺς Σογδιανοὺς 
καὶ Σάκας Ἰμάῳ ὄρει. 



Claudius Ptolemaeus Math., Geographia (lib. 4-8) 
Book 8, chapter 27, section 1, line 2

Ὁ ἑνδέκατος πίναξ τῆς Ἀσίας περιέχει 
      τὴν ἐκτὸς Γάγγου <Ἰνδικὴν> 
καὶ <Σίνας> 
σὺν ταῖς παρακειμέναις νήσοις. 



Claudius Ptolemaeus Math., Geographia (lib. 4-8) 
Book 8, chapter 27, section 2, line 3

Περιορίζεται δὲ ὁ πίναξ ἀπὸ μὲν <ἀνατολῶν> 
ἀγνώστῳ γῇ, ἀπὸ δὲ <μεσημβρίας> μέρει τοῦ Γαγγη-
τικοῦ κόλπου καὶ Ἰνδικῷ πελάγει καὶ Μεγάλῳ κόλπῳ, 
ἔτι ἀγνώστῳ γῇ, ἀπὸ δὲ <δύσεως> τῇ ἐντὸς Γάγγου 
Ἰνδικῇ, ἀπὸ δὲ <ἄρκτων> μέρει τε Σακῶν καὶ Σκυθίᾳ 
τῇ ἐντὸς Ἰμάου ὄρους καὶ Σηρικῇ. 



Claudius Ptolemaeus Math., Geographia (lib. 4-8) 
Book 8, chapter 27, section 3, line 1

Τῶν οὖν ἐν τῇ ἐκτὸς Γάγγου Ἰνδικῇ διασήμων 
πόλεων   
 ἡ μὲν <Τάκωλα> τὴν μεγίστην ἡμέραν ἔχει ὡρῶν ι<β> 
δʹ, καὶ διέστηκεν Ἀλεξανδρείας πρὸς ἔω ὥραις <ϛ> καὶ 
γοʹ· λαμβάνει δὲ τὸν ἥλιον δὶς τοῦ ἔτους κατὰ κορυ-
φὴν, ἀπέχοντα τῆς θερινῆς τροπῆς ἐφ' ἑκάτερα μοίρας 
<ο>θ 𐅵ʹ. 



Claudius Ptolemaeus Math., Geographia (lib. 4-8) 
Book 8, chapter 28, section 2, line 1

Περιορίζεται δὲ ὁ πίναξ πάντοθεν Ἰνδικῷ 
πελάγει. 



Claudius Ptolemaeus Math., Geographia (lib. 4-8) 
Book 8, chapter 29, section 26, line 2

Ἀρεία 
Παροπανισάδαι 
Δραγγιανή 
Ἀραχωσία 
Γεδρωσία· 
      πίναξ ιʹ. 
Ἰνδικὴ ἡ ἐντὸς Γάγγου ποταμοῦ· 
      πίναξ ιαʹ. 



Claudius Ptolemaeus Math., Geographia (lib. 4-8) 
Book 8, chapter 29, section 27, line 2

Ἰνδικὴ ἡ ἐντὸς Γάγγου ποταμοῦ· 
      πίναξ ιαʹ. 
Ἰνδικὴ ἡ ἐκτὸς Γάγγου ποταμοῦ 
Σινῶν χώρα· 
      πίναξ ιβʹ. 


Claudius Ptolemaeus Math., Geographia (lib. 4-8) 
Book 8, chapter 30, section 26, line 4

Ὁ δωδέκατος καὶ τελευταῖος ἀπὸ μοιρῶν ρ<ι>ϛ 
ἕως μοιρῶν ρλ<ε>· γίνεται μῆκος μοιρῶν <ι>θ· πλάτος 
ἀπὸ βορείων μοιρῶν <ι>β 𐅵ʹ ἢ διὰ τὴν Κῶρυ ἀκρωτηρίου 
καταγραφὴν ὡς ἂν φαινόμενον τῆς πρὸς τὴν Ἰνδικὴν 
σχέσιν τῆς Ταπροβάνης ἐμφαίνῃ μοιρῶν ι<γ> γʹ· ἀπὸ 
δὲ νοτίων μοιρῶν <ϛ> 𐅵ʹ· ὡς συνάγεσθαι πλάτος μοιρῶν 
<ι>θ 𐅵ʹγ ἢ ὅλων εἴκοσι. 

\end{greek}


\section{Anonymi Historici (FGrH), Heraclis historia (Tabula Albana) (IG 14.1293)}
\blockquote[From Wikipedia\footnote{\url{http://referenceworks.brillonline.com/entries/brill-s-new-jacoby/anonymous-history-of-herakles-tabula-albana-ig-14-1293-heraclis-historia-40-a40}}]{Source Date: 1st century BC}
\begin{greek}

Anonymi Historici (FGrH), Heraclis historia (Tabula Albana) (IG 14.1293) (1139: 002)
“FGrH \#40”.
Volume-Jacobyʹ-F 1a,40,F, fragment 1a, line 36

             / Ἡρακλῆς δ' ἐπὶ τὸν Ἰνδὸν ἦλθε / ποταμὸν καὶ πόλιν Ἡράκλειαν / τὰν ἐν 
Σίβαις οἰκίζει. 

\end{greek}



\section{Adespota Papyracea (SH), Epigrammata}
\blockquote[From Wikipedia\footnote{\url{}}]{}
\begin{greek}

Adespota Papyracea (SH), Epigrammata (2648: 003)
“Supplementum Hellenisticum”, Ed. Lloyd–Jones, H., Parsons, P.
Berlin: De Gruyter, 1983.
Fragment 977, line 1

οὐκ ἐμὰ τ[ 
....[    
βάρβαρον εἴσιδε τοῦτο[ 
ουκικατονφοι̣.ονο.ι̣[ 
οὔριον ὁλκάδι λαῖφον.. 
       
τίς σε τὸν ευ..τηρα κ- 
       
οὐκ ἐμὰ ταῦτα λάφυρα 
οἱ τρισσοί σοι ταῦτα   
⊗ Ἰνδὸν ὅδ' ἀπύει τύμβος Ταύρωνα θανόντα 
 κεῖσθαι, ὁ δὲ κτείνας πρόσθ̣ε̣ν ἐπεῖδ' Ἀίδαν· 
θὴρ ἅπερ ἄντα δρακεῖν, συὸς ἤ ῥ' ἀπὸ τᾶς Καλυδῶνος 
 λεί̣ψ̣ανο̣ν̣, ε̣ὐ̣κάρπο̣ι̣σ̣ ἐ̣μ π̣ε̣δίοις τρέφετο 
Ἀρσινόας ἀτίνακτον, ἀπ' αὐχένος ἀθρόα φρίσσων 
 λ]ό̣χ̣μ̣α̣ι̣σ̣ κ̣α̣ὶ̣ γ̣ε̣[ν]ύ̣ων ἀφρὸν ἀμ̣ε̣ρ̣γόμενος· 
σὺν δὲ πεσὼν σκύλακος τόλμαι στήθη μὲν ἑτοίμως 
 ἠλόκισ', οὐ μέλλων δ' αὐχέν' ἔθηκ' ἐπὶ γᾶν, 
δρα]ξ̣ά̣μ̣ενος γὰρ ὁμοῦ λοφιᾶι μεγάλοιο τένοντος 
 ο]ὐ πρ[ὶ]ν̣ ἔμυσεν ὀδόντ' ἔσθ' ὑπέθηκ' Ἀίδαι. 



Adespota Papyracea (SH), Epigrammata 
Fragment 977, line 22

ἄλλο


 ⊗ σκύλαξ ὁ τύμβωι τῶιδ' ὕπ' ἐκτερισμένος 
 Ταύρων, ἐπ' αὐθένταισιν οὐκ ἀμήχανος· 
 κάπρωι γὰρ ὡς συνῆλθεν ἀντίαν ἔριν, 
 ὁ μέν τις ὡς ἄπλατος οἰδήσας γένυν 
 στῆθος κατηλόκιζε λευκαίνων ἀφρῶι· 
 ὁ δ' ἀμφὶ νώτωι δισσὸν ἐμβαλὼν ἴχνος 
 ἐδράξατο φρίσσοντος ἐκ στέρνων μέσων 
 καὶ γᾶι συνεσπείρασεν· Ἀίδαι δὲ δοὺς 
 τὸν αὐτόχειρ' ἔθναισκεν, Ἰνδὸν ὡς νόμος. 

\end{greek}


\section{Dionysius Scytobrachion}
Era?
\blockquote[From Wikipedia\footnote{\url{
%http://books.google.com/books?id=480Wd8G6LPYC&lpg=PA25&ots=-yOwcHIuJ3&dq=Dionysius\%20Scytobrachion&pg=PA25#v=onepage&q=Dionysius\%20Scytobrachion&f=false
}}]{}
\begin{greek}
Dionysius Scytobrachion Gramm., Fragmenta (1881: 002)
“FGrH \#32”.
Volume-Jacobyʹ-F 1a,32,F, fragment 8, line 108

                                            (4) γηγενὲς γὰρ ὑπάρχον καὶ 
φυσικῶς ἐκ τοῦ στόματος ἄπλατον ἐκβάλλον φλόγα τὸ μὲν πρῶτον φανῆναι 
περὶ τὴν Φρυγίαν καὶ κατακαῦσαι τὴν χώραν, ἣν μέχρι τοῦ νῦν Κατακεκαυ-
μένην Φρυγίαν ὀνομάζεσθαι· ἔπειτ' ἐπελθεῖν τὰ περὶ τὸν Ταῦρον ὄρη 
συνεχῶς καὶ κατακαῦσαι τοὺς ἑξῆς δρυμοὺς μέχρι τῆς Ἰνδικῆς· μετὰ δὲ 
ταῦτα πάλιν ἐπὶ θάλατταν τὴν ἐπάνοδον ποιησάμενον περὶ μὲν τὴν Φοι-
νίκην ἐμπρῆσαι τοὺς κατὰ τὸν Λίβανον δρυμούς, καὶ δι' Αἰγύπτου πορευθὲν 
ἐπὶ τῆς Λιβύης διελθεῖν τοὺς περὶ τὴν ἑσπέραν τόπους καὶ τὸ τελευταῖον 
εἰς τοὺς περὶ τὰ Κεραύνια δρυμοὺς ἐγκατασκῆψαι. 



Dionysius Scytobrachion Gramm., Fragmenta 
Volume-Jacobyʹ-F 1a,32,F, fragment 8, line 236

                                   (7) τὸν δ' οὖν Διόνυσόν φασι τὴν κατάβασιν 
ἐκ τῆς Ἰνδικῆς ἐπὶ τὴν θάλατταν ποιησάμενον καταλαβεῖν ἅπαντας τοὺς 
Τιτᾶνας ἠθροικότας δυνάμεις καὶ διαβεβηκότας εἰς Κρήτην ἐπ' Ἄμμωνα. 



Dionysius Scytobrachion Gramm., Fragmenta 
Volume-Jacobyʹ-F 1a,32,F, fragment 13, line 2

                  ρ ιι 904: ὦκα δὲ Καλλιχόροιο παρὰ προχοὰς ποταμοῖο 
ἤλυθον, ἔνθ' ἐνέπουσι Διὸς Νυσήιον υἷα Ἰνδῶν ἡνίκα φῦλα λιπὼν κατενάσσατο Θήβας, 
ὀργιάσαι] ποταμὸς Παφλαγονίας ἱερὸς Διονύσου περὶ Ἡράκλειαν, οὗ μέμνηται καὶ 
Καλλίμαχος (F 100c Schn.) . 



Dionysius Scytobrachion Gramm., Fragmenta 
Volume-Jacobyʹ-F 1a,32,F, fragment 13, line 5

                                                                                  ὅτι δὲ κατε-
πολέμησεν Ἰνδοὺς ὁ Διόνυσος Διονύσιός φησι καὶ Ἀριστόδημος ἐν πρώτωι Θηβαικῶν 
ἐπιγραμμάτων (III) καὶ Κλείταρχος ἐν ταῖς Περὶ Ἀλέξανδρον Ἱστορίαις (II) . 


\end{greek}


\section{Vettius Valens}
\blockquote[From Wikipedia\footnote{\url{http://en.wikipedia.org/wiki/Vettius_Valens}}]{Vettius Valens (February 8, 120 – c. 175) was a 2nd-century Hellenistic astrologer, a somewhat younger contemporary of Claudius Ptolemy.

Valens' major work is the Anthology, ten volumes in Greek written roughly within the period 150 to 175. The Anthology is the longest and most detailed treatise on astrology which has survived from that period. A working professional astrologer, Valens includes over a hundred sample charts from his case files in the Anthology.

Although originally a native of Antioch, he appears to have traveled widely in Egypt in search of specific astrological doctrines to bolster his practice. At the time Alexandria was still home to a number of astrologers of the older Babylonian, Greek and Egyptian traditions. He published much of what he learned from the tradition and through his practice in his Anthology, written in an engaging and instructional style. The Anthology is thus of great value in piecing together actual working techniques of the time.

Valens' work is also important because he cites the views of a number of earlier authors and authorities who would otherwise be unknown. The fragments from works attributed to the alleged pharaoh Nechepso and the high priest Petosiris, pseudopigraphal authors of the 2nd century BC, survive mainly through direct quotations in Valens' work.

The three manuscripts of the Anthology all date from 1300 or later.[1] The text, however, appears to be fairly reliable and complete, although disorganized in places.

Although Ptolemy, the astronomer, mathematician, astrologer of ancient Alexandria and author of Tetrabiblos (the most influential astrological text ever written), was generally regarded as the colossus of Hellenistic-period astrology in the many centuries following his death, it is most likely that the actual practical astrology of the period resembled the methods elaborated in Valens' Anthology. Modern scholars tend to counterpoise the two men, since both were roughly contemporary and lived in Alexandria; yet Valens' work elaborated the more practical techniques that arose from ancient tradition, while Ptolemy, very much the scientist, tended to focus more on creating a theoretically consistent model based on his Aristotelian causal framework. The balance given by Valens' Anthology is therefore very instructive. No other Hellenistic author has contributed as much to our understanding of the everyday, practical astrological methods of the early Roman/late Hellenistic era.

Deciding that the traditional religion was useless, he found in fate a substitute religion. For him absolute determination gave emotional satisfaction and aroused an almost mystical feeling. Knowing that everything was already predetermined gave one a sense of freedom from anxiety and a sense of salvation.}

\begin{greek}

Vettius Valens Astrol., Anthologiarum libri ix (1764: 001)
“Vettii Valentis anthologiarum libri”, Ed. Kroll, W.
Berlin: Weidmann, 1908, Repr. 1973.
Page 7, line 24

Ἐστὶ δὲ τὰ ὑποτεταγμένα κλίματα· κατὰ τὴν κεφαλὴν Μηδία 
καὶ οἱ συνεχεῖς τόποι· <τῷ δὲ στήθει Βαβυλωνία· τὰ πρὸς τῷ 
Ἡνιόχῳ δεξιά, Σκυθία>· ἡ Πλειάς, [ἡ] Κύπρος· τὰ ἀριστερά, 
Ἀραβία καὶ οἱ πέριξ τόποι· κατὰ τοὺς ὤμους Περσὶς καὶ τὰ 
Καυκάσια ὄρη, ὑπὸ τὸ κύρτωμα † ἄρχονται· ὑπὸ τὴν ὀσφὺν Αἰ-
θιοπία· ὑπὸ τὸ μέτωπον Ἐλυμαΐς· ὑπὸ τὰ κέρατα Καρχηδονία· 
μέσοις μέρεσιν Ἀρμενία, Ἰνδική, Γερμανία. 



Vettius Valens Astrol., Anthologiarum libri ix 
Page 8, line 20

Ἔστι δὲ Διδύμοις ὑποτεταγμένα κλίματα τάδε· ἐμπρόσθια, 
Ἰνδικὴ καὶ οἱ συνεχεῖς τόποι καὶ Κελτική· στῆθος, Κιλικία, Γα-
λατία, Θρᾴκη καὶ [ἡ] Βοιωτία. 



Vettius Valens Astrol., Anthologiarum libri ix 
Page 12, line 29

Πρόσκειται δὲ αὐτῷ κλίματα τάδε· ἐμπρόσθια, Συρία· μέσα, 
Εὐφράτης καὶ Τίγρις, Αἴγυπτος καὶ Λιβύη καὶ οἱ συνεχεῖς Αἰγυ-
πτίων ποταμοὶ καὶ Ἰνδὸς ποταμός· κατὰ δὲ τὸ μέσον τῆς Κάλπης 
Τάναϊς καὶ οἱ λοιποὶ ποταμοὶ ἐκ τοῦ † ὑπὸ τοὺς πόδας ῥέοντες 
πρὸς νότον καὶ ζέφυρον. 



Vettius Valens Astrol., Anthologiarum libri ix 
Page 13, line 17

                             πρόσκειται δὲ κλίματι τῆς Ἐρυθρᾶς θα-
λάσσης, ἔχον νήσους οὐκ ὀλίγας παρ' ἑαυτό, ἃς ὑπέρκειται ἡ Ἰνδία 
καὶ ὁ λεγόμενος Ἰνδικὸς ὠκεανός· ἐν δὲ τοῖς ἀπηλιωτικοῖς αὐτοῦ 
μέρεσι τὴν Παρθίαν ἔχει καὶ τὴν Ἰνδικὴν χώραν κατὰ θίξιν καὶ 
τὸν Ἀπηλιωτικὸν ὠκεανόν, ἐκ τῶν βορείων αὐτοῦ μερῶν τὴν Σκυ-
θικὴν χώραν· ἐκ δὲ τῶν πρὸς λίβα αὐτοῦ μερῶν ψαύει προσκλύ-
ζον Μυοσόρμου, Ὀρθοῦ ὅρμου καὶ τῶν πέριξ πόλεων. 



Vettius Valens Astrol., Anthologiarum libri ix 
Page 14, line 7

                                 μέσα, Συρία, Ἐρυθρὰ θάλασσα· [τὰ 
ἐμπρόσθια] Ἰνδικὴ μέση, Περσὶς καὶ οἱ συνεχεῖς τόποι, Ἀραβικὴ 
θάλασσα καὶ Ἐρυθρὰ καὶ Βορυσθένης ποταμός. 

\end{greek}

\section{Apollodorus of Artemita}
\blockquote[From Wikipedia\footnote{\url{http://en.wikipedia.org/wiki/Apollodorus_of_Artemita}}]{

Apollodorus of Artemita (Greek: Ἀπολλόδωρος Ἀρτεμιτηνός) (c. 130–87 BCE) was a Greek writer of the 1st century BCE.

Apollodorus wrote a history of the Parthian Empire, the Parthika (Greek: τὰ Παρθικὰ), in at least four books. He is quoted by Strabo and Athenaeus. Strabo stated that he was very reliable. Apollodorus seems to have used the archives of Artemita and Seleucia on the Tigris for his work. Some information on Greco-Bactrians are preserved in Strabo's work:

    "The Greeks who caused Bactria to revolt grew so powerful on account of the fertility of the country that they became masters, not only of Ariana, but also of India, as Apollodorus of Artemita says: and more tribes were subdued by them than by Alexander--by Menander in particular (at least if he actually crossed the Hypanis towards the east and advanced as far as the Imaus), for some were subdued by him personally and others by Demetrius, the son of Euthydemus the king of the Bactrians; and they took possession, not only of Patalena, but also, on the rest of the coast, of what is called the kingdom of Saraostus and Sigerdis. In short, Apollodorus says that Bactriana is the ornament of Ariana as a whole; and, more than that, they extended their empire even as far as the Seres and the Phryni." (Strabo, Geographia, 11.11.1)

He is also quoted for his general geographical knowledge of Central Asia:

    "Accordingly, if the distance from Hyrcania to Artemita in Babylonia is eight thousand stadia, as is stated by Apollodorus of Artemita, and the distance from there to the mouth of the Persian Sea another eight thousand, and again eight thousand, or a little less, to the places that lie on the same parallel as the extremities of Ethiopia, there would remain of the above-mentioned breadth of the inhabited world the distance which I have already given,14 from the recess of the Hyrcanian Sea to the mouth of that sea" (Strabo, Geographia, 11.11.1) }
\begin{greek}

Apollodorus Hist., Fragmenta (1164: 002)
“FHG 4”, Ed. Müller, K.
Paris: Didot, 1841–1870.
Fragment 5, line 6

Τοσοῦτον δὲ ἴσχυσαν οἱ ἀποστήσαντες Ἕλληνες αὐτὴν 
διὰ τὴν ἀρετὴν τῆς χώρας, ὥστε τῆς τε Ἀριανῆς ἐπε-
κράτουν καὶ τῶν Ἰνδῶν, ὥς φησιν Ἀπολλόδωρος ὁ Ἀρ-
τεμιτηνὸς, καὶ πλείω ἔθνη κατεστρέψαντο ἢ Ἀλέξαν-
δρος, καὶ μάλιστα Μένανδρος (εἴγε καὶ τὸν Ὕπανιν 
διέβη, πρὸς ἕω, καὶ μέχρι τοῦ Ἰσάμου 
προ-
ῆλθε)· τὰ μὲν γὰρ αὐτὸς, τὰ δὲ Δημήτριος ὁ Εὐθυ-
δήμου υἱὸς τοῦ Βακτρίων βασιλέως· οὐ μόνον δὲ τὴν 
Πατταληνὴν κατέσχον, ἀλλὰ καὶ τῆς ἄλλης παραλίας   
τήν τε Σαραόστου καλουμένην, καὶ τὴν Σιγέρδιδος 
βασιλείαν. 

Apollodorus Hist., Fragmenta 
Fragment 6, line 5

Strabo XV: Ἀπολλόδωρος γοῦν, ὁ τὰ 
Παρθικὰ ποιήσας, μεμνημένος καὶ τῶν τὴν Βακτριανὴν 
ἀποστησάντων Ἑλλήνων παρὰ τῶν Συριακῶν βασι-
λέων, τῶν ἀπὸ Σελεύκου τοῦ Νικάτορος, φησὶ μὲν αὐ-
τοὺς αὐξηθέντας ἐπιθέσθαι καὶ τῇ Ἰνδικῇ· οὐδὲν δὲ 
προσανακαλύπτει τῶν πρότερον ἐγνωσμένων, ἀλλὰ καὶ 
ἐναντιολογεῖ, πλείω τῆς Ἰνδικῆς ἐκείνους ἢ Μακεδό-
νας καταστρέψασθαι λέγων. 

\end{greek}


\section{\emph{Scholia In Homerum}}
\blockquote[From Wikipedia\footnote{\url{}}]{}
\begin{greek}

Scholia In Homerum, Scholia in Iliadem (scholia vetera) (5026: 001)
“Scholia Graeca in Homeri Iliadem (scholia vetera), vols. 1–5, 7”, Ed. Erbse, H.
Berlin: De Gruyter, 1:1969; 2:1971; 3:1974; 4:1975; 5:1977; 7:1988.
Book of Iliad 17, verse 213-4, line of scholion 1

                                                       b(BE3) 

Scholia In Homerum, Scholia in Odysseam (scholia vetera) (5026: 007)
“Scholia Graeca in Homeri Odysseam, 2 vols.”, Ed. Dindorf, W.
Oxford: Oxford University Press, 1855, Repr. 1962.
Book 1, hypothesis-verse 23, line 2

         H. 
διχθὰ δεδαίαται] ἀπὸ μεσημβρίας διερχόμενος ὁ Νεῖλος   
διορίζει τοὺς Αἰθίοπας, ἐκ μὲν ἀνατολῆς ἔχων τοὺς Ἰνδοὺς, ἐκ δυσμῶν 
δὲ νομάδας καὶ Βλέμυας. 



Scholia In Homerum, Scholia in Odysseam (scholia vetera) 
Book 4, hypothesis-verse 84, line 7

                                                                 τινὲς 
δὲ γράφουσιν Ἐρεμνούς, ὅ ἐστι τοὺς Ἰνδοὺς, οἱ δὲ τοὺς εἰς τὴν ἔραν 
δύνοντας διὰ τοὺς καύσωνας, οὓς καὶ Τρωγλοδύτας φασί. 



Scholia In Homerum, Scholia in Odysseam (scholia vetera) 
Book 4, hypothesis-verse 84, line 13

                           ρ. Ἐρεμβοὺς τοὺς Τρωγλοδύτας τοὺς   
Σαρακηνοὺς παρὰ τὸ εἰς τὴν ἔραν δύειν· οἱ δὲ τοὺς Ἰνδοὺς παρὰ τὸ 
ἔρεβος, μέλανες γὰρ, ὅθεν καὶ Κράτης τοὺς Ἐρεμνοὺς γράφει· οἱ δὲ 
ἴδιον ἔθνος. 



Scholia In Homerum, Scholia in Odysseam (scholia vetera) 
Book 4, hypothesis-verse 84, line 19

                                                             οἱ δὲ τοὺς 
Ἰνδοὺς παρὰ τὸ ἔρεβος, μέλανες γὰρ, ὅθεν καὶ Κράτης Ἐρέμνους 
γράφει. 




Scholia In Homerum, Scholia in Odysseam 1.1–309 (scholia vetera) (5026: 008)
“Scholia in Homeri Odysseae α 1–309 auctiora et emendatiora”, Ed. Ludwich, A.
Königsberg: Hartung, 1888–1890, Repr. 1966.
Hypothesis-verse of Odyssey 23, line of scholion 10

                                                                                 H Ma P Q V. ἀπὸ μεσημ-
βρίας διερχόμενος ὁ Νεῖλος διορίζει τοὺς Αἰθίοπας, ἐκ μὲν ἀνατολῆς ἔχων τοὺς Ἰνδούς, ἐκ δυς-
μῶν δὲ Νομάδας καὶ Βλέμυας. 



Scholia In Homerum, Scholia in Iliadem (scholia vetera) (= D scholia) (5026: 017)
“Homeri Ilias, 2 vols.”, Ed. Heyne, C.G.
Oxford: Oxford University Press, 1834.
Book of Iliad 6, verse 133, line of scholion 7

        Ἡ δὲ Νύσσα ἐν μέν τισι χώραις 
ὄρος, ὡς ἐν Βοιωτίᾳ καὶ Θρᾴκῃ, καὶ Ἀρα-
βίᾳ, καὶ Ἰνδικῇ, καὶ Λιβύῃ, καὶ Νάξῳ. 


\end{greek}


\section{Timon}
\blockquote[From Wikipedia\footnote{\url{http://en.wikipedia.org/wiki/Timon_(philosopher)}}]{
Timon of Phlius, 17th-century engraving

Timon of Phlius (Greek: Τίμων, gen.: Τίμωνος; c. 320 BC – c. 230 BC) was a Greek skeptic philosopher, a pupil of Pyrrho, and a celebrated writer of satirical poems called Silloi (Greek: Σίλλοι). He was born in Phlius, moved to Megara, and then he returned home and married. He next went to Elis with his wife, and heard Pyrrho, whose tenets he adopted. He also lived on the Hellespont, and taught at Chalcedon, before moving to Athens, where he lived until his death. His writings were said to have been very numerous. He composed poetry, tragedies, satiric dramas, and comedies, of which very little remains. His most famous composition was his Silloi, a satirical account of famous philosophers, living and dead, in hexameter verse. The Silloi has not survived intact, but it is mentioned and quoted by several ancient authors.}

Ἰνδαλμοί?

\begin{greek}

Timon Phil., Fragmenta et tituli (1735: 003)
“Supplementum Hellenisticum”, Ed. Lloyd–Jones, H., Parsons, P.
Berlin: De Gruyter, 1983.
Fragment 841-843(+844), line t

νῦν δέ με λευγαλέαις ἔρισιν εἵμαρτο δαμῆναι 
καὶ πενίῃ καὶ ὅσ' ἄλλα βροτοὺς κηφῆνας ἐλαστρεῖ.’   
Ἰνδαλμοί


τοῦτό μοι, ὦ Πύρρων, ἱμείρεται ἦτορ ἀκοῦσαι, 
 πῶς ποτ' †ανηροταγεις† ῥῇστα μεθ' ἡσυχίης 
αἰεὶ ἀφροντίστως καὶ ἀκινήτως κατὰ ταὐτὰ 
 μὴ προσέχων δίνοις ἡδυλόγου σοφίης, 
μοῦνος δ' ἀνθρώποισι θεοῦ τρόπον ἡγεμονεύεις, 
 ὃς περὶ πᾶσαν ἐλῶν γαῖαν ἀναστρέφεται, 
δεικνὺς εὐτόρνου σφαίρης πυρικαύτορα κύκλον. 

\end{greek}


\section{\emph{Lyrica Adespota}}
\blockquote[From Wikipedia\footnote{\url{}}]{}
\begin{greek}

Lyrica Adespota (CA), Fragmenta lyrica (0230: 001)
“Collectanea Alexandrina”, Ed. Powell, J.U.
Oxford: Clarendon Press, 1925, Repr. 1970.
Fragment 2, line 90

          [Βά]ρβαρον ἀνάγω χορὸν ἄπλετον, θεὰ Σελή[νη], 
πρὸς ῥυθμὸν ἄνετον βήματι βαρβάρῳ [προβαίνων]. 
Ἰνδῶν δὲ πρόμοι πρὸς ἱ[ε]ρόθρουν δότε [τυπανισμόν, 
[Σ]ηρικὸν ἰδίως θεαστικὸν βῆμα παρα̣λ̣[λάξ]. 

\end{greek}


\section{Theopompus}
\blockquote[From Wikipedia\footnote{\url{http://en.wikipedia.org/wiki/Theopompus}}]{Theopompus (Ancient Greek: Θεόπομπος) (born c. 380 BC) was a Greek historian and rhetorician

The works of Theopompus were chiefly historical, and are much quoted by later writers. They included an Epitome of Herodotus's History (Whether this work is actually his is debated[1]),the Hellenics, the History of Philip, and several panegyrics and hortatory addresses, the chief of which was the Letter to Alexander.}
\begin{greek}


Theopompus Hist., Fragmenta (0566: 002)
“FGrH \#115”.
Volume-Jacobyʹ-F 2b,115,F, fragment 340, line 2

– V 42: ἔνδοξος δὲ καὶ ὁ Ἠπειρωτικὸς Κέρβερος καὶ 
ὁ Ἀλεξάνδρου Περίτας, τὸ θρέμμα τὸ Ἰνδικόν· ἐκράτει δ' οὗτος λέοντος, 
ἑκατὸν μνῶν ἐωνημένος. 



Theopompus Hist., Fragmenta 
Volume-Jacobyʹ-F 2b,115,F, fragment 371, line 1

–  – s. Ἰνδάρα· Σικανῶν πόλις. 



Theopompus Hist., Fragmenta 
Volume-Jacobyʹ-F 2b,115,F, fragment 381, line 3

         ι 2, 35: Θεόπομπος δὲ ἐξομολογεῖται φήσας, ὅτι 
καὶ μύθους ἐν ταῖς ἱστορίαις ἐρεῖ κρεῖττον ἢ ὡς Ἡρόδοτος καὶ Κτησίας 
(III) καὶ Ἑλλάνικος (4) καὶ οἱ τὰ Ἰνδικὰ συγγράψαντες (III). 
  – VII 7, 5: <τῶν μὲν οὖν Ἠπειρωτῶν ἔθνη 
φησὶν εἶναι Θεόπομπος τετταρεσκαίδεκα. 

\end{greek}


\section{Anaxarchus}
\blockquote[From Wikipedia\footnote{\url{http://en.wikipedia.org/wiki/Anaxarchus}}]{Anaxarchus (play /ˌænəɡˈzɑrkəs/; Greek: Ἀνάξαρχος; c. 380 - c. 320 BC) was a Greek philosopher of the school of Democritus. Together with Pyrrho, he accompanied Alexander the Great into Asia. The reports of his philosophical views suggest that he was a forerunner of the Greek skeptics.



Anaxarchus was born at Abdera in Thrace. He was the companion and friend of Alexander the Great in his Asiatic campaigns. According to Diogenes Laertius, in response to Alexander's claim to have been the son of Zeus-Ammon, Anaxarchus pointed to his bleeding wound and remarked, "See the blood of a mortal, not ichor, such as flows from the veins of the immortal gods."[1] Diogenes Laertius also says that Nicocreon, the tyrant of Cyprus, commanded him to be pounded to death in a mortar, and that he endured this torture with fortitude and Cicero relates the same story.[2]

Plutarch tells a story that at Bactra, in 327 BC in a debate with Callisthenes, he advised all to worship Alexander as a god even during his lifetime, is with greater probability attributed to the Sicilian Cleon.

When Alexander was trying to show that he is divine so that the Greeks and Macedonians would perform proskynesis to him, Anaxarchus said that Alexander could "more justly be considered a god than Dionysus or Heracles" (Arrian, 104)
Philosophy

Very little is known about his philosophical views. It is thought that he represents a link between the atomism of Democritus, and the skepticism of Pyrrho.

Anaxarchus is said to have studied under Diogenes of Smyrna, who in turn studied under Metrodorus of Chios, who used to declare that he knew nothing, not even the fact that he knew nothing.[2] According to Sextus Empiricus, Anaxarchus "compared existing things to a scene-painting and supposed them to resemble the impressions experienced in sleep or madness."[3] It was under the influence of Anaxarchus that Pyrrho is said to have adopted "a most noble philosophy, . . . taking the form of agnosticism and suspension of judgement."[4] Anaxarchus is said to have praised Pyrrho's "indifference and sang-froid."[5] Anaxarchus is said to have possessed "fortitude and contentment in life," which earned him the epithet eudaimonikos ("fortunate"),[1] which may imply that he held the end of life to be eudaimonia.}

\begin{greek}

Anaxarchus Phil., Testimonia (0714: 001)
“Die Fragmente der Vorsokratiker, vol. 2, 6th edn.”, Ed. Diels, H., Kranz, W.
Berlin: Weidmann, 1952, Repr. 1966.
Fragment 2, line 3

                                       .. ἤκουσε Βρύσωνος τοῦ Στίλπωνος, ὡς 
Ἀλέξανδρος ἐν Διαδοχαῖς [fr. 146 FHG III 243], εἶτα Ἀναξάρχου ξυνακολουθῶν 
πανταχοῦ, ὡς καὶ τοῖς Γυμνοσοφισταῖς ἐν Ἰνδίαι συμμῖξαι καὶ τοῖς Μάγοις . 



Anaxarchus Phil., Testimonia 
Fragment 2, line 5

τοῦτο δὲ ποιεῖν ἀκούσαντα Ἰνδοῦ τινος ὀνειδίζοντος Ἀναξάρχωι, ὡς οὐκ ἂν ἕτερόν 
τινα διδάξαι οὗτος ἀγαθόν, αὐτὸς αὐλὰς βασιλικὰς θεραπεύων. 

\end{greek}


\section{Speusippus}
\blockquote[From Wikipedia\footnote{\url{http://en.wikipedia.org/wiki/Speusippus}}]{Speusippus (c. 408 – 339/8 BC[1]) was an ancient Greek philosopher. Speusippus was Plato's nephew by his sister Potone. After Plato's death, Speusippus inherited the Academy and remained its head for the next eight years. However, following a stroke, he passed the chair to Xenocrates. Although the successor to Plato in the Academy, he frequently diverged from Plato's teachings. He rejected Plato's Theory of Forms, and whereas Plato had identified the Good with the ultimate principle, Speusippus maintained that the Good was merely secondary. He also argued that it is impossible to have satisfactory knowledge of any thing without knowing all the differences by which it is separated from everything else.}

\begin{greek}

Speusippus Phil., Fragmenta (1692: 005)
“Speusippus of Athens”, Ed. Tarán, L.
Leiden: Brill, 1981; Philosophia Antiqua 39.
Fragment 67, line 63

                                            πάντα γὰρ ὡς εἰπεῖν, ὅσα ἥμερα καὶ 
ἄγρια τυγχάνει ὄντα, οἷον ἄνθρωποι, ἵπποι, βόες, κύνες ἐν τῇ Ἰνδικῇ, 
ὕες, αἶγες, πρόβατα· ὧν ἕκαστον, εἰ μὲν ὁμώνυμον, οὐ διῄρηται χωρίς, 
εἰ δὲ ταῦτα ἓν εἴδει, οὐχ οἷόν τ' εἶναι διαφορὰς τὸ ἄγριον καὶ τὸ ἥμερον. 

\end{greek}


\section{Callixenus of Rhodes}
\blockquote[From Wikipedia\footnote{\url{http://en.wikipedia.org/wiki/Callixenus_of_Rhodes}}]{

Callixenus of Rhodes was a Hellenistic author from Rhodes. He was a contemporary of Ptolemy II Philadelphus [309 BCE – 246 BCE]. He wrote two works, both of which are lost.
"Peri Alexandreias"

This consisted of at least four books, and was much used by Athenaeus (Athen. v. p. 196, \&c., ix. p. 387, xi. pp. 472, 474, 483; Harpocrat. s. v. eggythike). It contained the main account of the tessarakonteres.
Title unknown

This seems to have been a catalogue of painters and sculptors (zografonte kai andriantopoion anagrafe), of which Sopater, in the twelfth book of his Eclogae had made an abridgement. (Phot. Bibl. Cod. 161 ; comp. Preller, Polem. Fragm. p. 178, \&c.)}

\begin{greek}

Callixenus Hist., Fragmenta (1240: 003)
“FHG 3”, Ed. Müller, K.
Paris: Didot, 1841–1870.
Fragment 1, line 118

            Κατεναντίον δὲ τούτου ἄλλο συμπόσιον πολυ-
τελὲς περίπτερον· οἱ γὰρ κίονες αὐτοῦ ἐκ λίθων Ἰνδικῶν 
συνέκειντο. 



Callixenus Hist., Fragmenta 
Fragment 2, line 247

         Ἐπὶ δὲ ἄλλης τετρακύκλου, ἣ περιεῖχε τὴν 
ἐξ Ἰνδῶν κάθοδον Διονύσου, Διόνυσος ἦν δωδεκάπηχυς, 
ἐπ' ἐλέφαντος κατακείμενος, ἠμφιεσμένος προφυρίδα, 
καὶ στέφανον κισσοῦ καὶ ἀμπέλου χρυσοῦν ἔχων, εἶχεν 
ἐν ταῖς χερσὶ θυρσόλογχον χρυσοῦν· ὑπεδέδετο δ' ἐμ-
βάδας χρυσογραφεῖς. 



Callixenus Hist., Fragmenta 
Fragment 2, line 278

                      Αὗται δ' εἶχον σκηνὰς βαρβαρικὰς, 
ἐφ' ὧν ἐκάθηντο γυναῖκες Ἰνδαὶ καὶ ἕτεραι, κεκοσμη-
μέναι ὡς αἰχμάλωτοι. 



Callixenus Hist., Fragmenta 
Fragment 2, line 288

              Ἤγοντο δὲ καὶ κύνες δισχίλιοι τετρακό-
σιοι, οἱ μὲν Ἰνδοὶ, οἱ λοιποὶ δὲ Ὑρκανοὶ καὶ Μολοσσοὶ 
καὶ ἑτέρων γενῶν. 



Callixenus Hist., Fragmenta 
Fragment 2, line 296

                              Εἰπὼν δὲ καὶ 
ἄλλα πλεῖστα, καὶ καταλέξας ζῴων ἀγέλας, ἐπιφέρει· 
»Πρόβατα Αἰθιοπικὰ ἑκατὸν τριάκοντα, Ἀράβια 
τριακόσια, Εὐβοϊκὰ εἴκοσι, ὁλόλευκοι βόες Ἰνδικοὶ 
εἴκοσι ἓξ, Αἰθιοπικοὶ ὀκτὼ, ἄρκτος μὲν λευκὴ μεγάλη 
μία, παρδάλεις τεσσαρεσκαίδεκα, πάνθηρες ἑκκαίδεκα, 
λυγκία τέσσαρα, ἄρκηλοι τρεῖς, καμηλοπάρδαλις μία, 
ῥινόκερως Αἰθιοπικὸς εἷς. 



Callixenus Hist., Fragmenta 
Fragment 3d, line 5

Οὕτως γὰρ καὶ Πολέμων κ. τ. λ. 
 Harpocration v. ἐγγυθήκη: Εἴη δ' ἂν σκεῦός τι πρὸς 
τὸ κρατῆρας ἢ λέβητας ἤ τι τοιούτων οὐκ ἀλλότριον 
ἐπικεῖσθαι ἐπιτήδειον εἶναι, ὡς Καλλίξενός τε ἐν δ' Περὶ 
Ἀλεξανδρείας ὑποσημαίνει, καὶ Δαΐμαχος ὁ Πλαταιεὺς 
ἐν βʹ Περὶ Ἰνδικῆς. 

\end{greek}


\section{Paradoxographus Vaticanus}%??? date
%http://bmcr.brynmawr.edu/2010/2010-01-13.html
%http://de.wikipedia.org/wiki/Paradoxographoi
\blockquote[From Wikipedia\footnote{\url{http://en.wikipedia.org/wiki/Paradoxography}}]{Paradoxographoi (griechisch παραδοξογράφοι) oder Paradoxographien sind antike listenartige Sammlungen von wunderbaren Tatsachen. Die Bezeichnung selbst stammt nicht aus der Antike, sondern wurde nach ersten Belegen im zwölften Jahrhundert erst im neunzehnten Jahrhundert zur Sammelbezeichnung der Gattung.

Paradoxographische Schriften sind eine antike Literaturgattung. Als Erfinder der Gattung gilt Kallimachos. Bei vielen Paradoxographien handelt es sich um anonyme oder pseudepigraphische Schriften. Letzteres gilt etwa von der Aristoteles zugeschriebenen Schrift Περὶ θαυμασίων ἀκουσμάτων („Über Dinge, die wunderbar zu hören sind“). Zu den ersteren gehören die anonymen Sammlungen von Exzerpten älterer Paradoxographien: Paradoxographus Florentinus, Vaticanus und Palatinus.

Paradoxographische Schriften sind dadurch von z. B. mythologischen Schriften unterschieden, dass ihre Autoren davon ausgehen, dass es sich bei den berichteten außergewöhnlichen Ereignissen oder Gegebenheiten (die „Mirabilien“) um zwar seltsame, aber doch empirische Tatsachen handelt. Inhaltlich überwiegen Tier- und Wassermirabilien, es finden sich aber auch Berichte über (aus der Sicht der Verfasser) bemerkenswerte Eigentümlichkeiten fremder Völker.}
%http://de.wikipedia.org/wiki/Paradoxographoi
\begin{greek}

Paradoxographus Vaticanus, Admiranda (0582: 001)
“Paradoxographorum Graecorum reliquiae”, Ed. Giannini, A.
Milan: Istituto Editoriale Italiano, 1965.
Section 34, line 1

Περὶ τὴν Ἰνδικὴν ἔστι λίμνη, ἥτις πάντα †δέχεται† 
πλὴν χρυσοῦ καὶ ἀργύρου. 



Paradoxographus Vaticanus, Admiranda 
Section 35, line 1

Ἑλλάνικος ἐν Ἰνδοῖς εἶναί φησι κρήνην Σίλαν κα-
λουμένην, ἐφ' ἧς καὶ τὰ ἐλαφρότατα καταποντίζεται. 



Paradoxographus Vaticanus, Admiranda 
Section 42, line 1

Ἐν Παδαίοις, Ἰνδικῷ ἔθνει, ὁ συνετώτατος τῶν 
παρόντων κατάρχεται τῶν ἱερῶν· αἰτεῖται δὲ παρὰ τῶν θεῶν 
οὐδὲν ἄλλο πλὴν δικαιοσύνης. 



Paradoxographus Vaticanus, Admiranda 
Section 53, line 1

Παρὰ τοῖς Ἰνδοῖς ὁ τεχνίτου πηρώσας χεῖρα ἢ 
ὀφθαλμὸν θανάτῳ ζημιοῦται. 

\end{greek}


\section{Theophrastus}
\blockquote[From Wikipedia\footnote{\url{http://en.wikipedia.org/wiki/Theophrastus}}]{Theophrastus (Greek: Θεόφραστος; c. 371 – c. 287 BC[1]), a Greek native of Eresos in Lesbos, was the successor to Aristotle in the Peripatetic school. He came to Athens at a young age, and initially studied in Plato's school. After Plato's death he attached himself to Aristotle. Aristotle bequeathed to Theophrastus his writings, and designated him as his successor at the Lyceum. Theophrastus presided over the Peripatetic school for thirty-six years, during which time the school flourished greatly. After his death, the Athenians honoured him with a public funeral. His successor as head of the school was Strato of Lampsacus.

The interests of Theophrastus were wide-ranging, extending from biology and physics to ethics and metaphysics. His two surviving botanical works, Enquiry into Plants[2] and On the Causes of Plants, were an important influence on medieval science. There are also surviving works On Moral Characters, On Sensation, On Stones, and fragments on Physics and Metaphysics all written in Greek. In philosophy, he studied grammar and language, and continued Aristotle's work on logic. He also regarded space as the mere arrangement and position of bodies, time as an accident of motion, and motion as a necessary consequence of all activity. In ethics, he regarded happiness as depending on external influences as well as on virtue, and famously said that "life is ruled by fortune, not wisdom."}

\begin{greek}

Theophrastus Phil., Historia plantarum (0093: 001)
“Theophrastus. Enquiry into plants, 2 vols.”, Ed. Hort, A.
Cambridge, Mass.: Harvard University Press, 1916, Repr. 1:1968; 2:1961.
Book 1, chapter 7, section 3, line 1

Ἰδία δὲ ῥίζης φύσις καὶ δύναμις ἡ τῆς Ἰνδικῆς 
συκῆς· ἀπὸ γὰρ τῶν βλαστῶν ἀφίησι, μέχρι οὗ 
ἂν συνάψῃ τῇ γῇ καὶ ῥιζωθῇ, καὶ γίνεται περὶ τὸ 
δένδρον κύκλῳ συνεχὲς τὸ τῶν ῥιζῶν οὐχ ἁπτό-
μενον τοῦ στελέχους ἀλλ' ἀφεστηκός. 



Theophrastus Phil., Historia plantarum 
Book 4, chapter 4, section 1, line 5

Ἐν δὲ τῇ Ἀσίᾳ παρ' ἑκάστοις ἴδι' ἄττα 
τυγχάνει· τὰ μὲν γὰρ φέρουσιν αἱ χῶραι τὰ δ'   
οὐ φύουσιν· οἷον κιττὸν καὶ ἐλάαν οὔ φασιν εἶναι 
τῆς Ἀσίας ἐν τοῖς ἄνω τῆς Συρίας ἀπὸ θαλάττης 
πένθ' ἡμερῶν· ἀλλ' ἐν Ἰνδοῖς φανῆναι κιττὸν 
ἐν τῷ ὄρει τῷ Μηρῷ καλουμένῳ, ὅθεν δὴ καὶ τὸν 
Διόνυσον εἶναι μυθολογοῦσι. 



Theophrastus Phil., Historia plantarum 
Book 4, chapter 4, section 4, line 1

Ἡ δὲ Ἰνδικὴ χώρα τήν τε καλουμένην ἔχει 
συκῆν, ἣ καθίησιν ἐκ τῶν κλάδων τὰς ῥίζας ἀν' 
ἕκαστον ἔτος, ὥσπερ εἴρηται πρότερον· ἀφίησι 
δὲ οὐκ ἐκ τῶν νέων ἀλλ' ἐκ τῶν ἔνων καὶ ἔτι 
παλαιοτέρων· αὗται δὲ συνάπτουσαι τῇ γῇ 
ποιοῦσιν ὥσπερ δρύφακτον κύκλῳ περὶ τὸ δέν-
δρον, ὥστε γίνεσθαι καθάπερ σκηνήν, οὗ δὴ καὶ   
εἰώθασι διατρίβειν. 



Theophrastus Phil., Historia plantarum 
Book 4, chapter 4, section 5, line 3

Ἔστι δὲ καὶ ἕτερον δένδρον καὶ τῷ μεγέθει 
μέγα καὶ ἡδύκαρπον θαυμαστῶς καὶ μεγαλό-
καρπον· καὶ χρῶνται τροφῇ τῶν Ἰνδῶν οἱ σοφοὶ 
καὶ μὴ ἀμπεχόμενοι. 



Theophrastus Phil., Historia plantarum 
Book 4, chapter 4, section 11, line 8

                            ταῦτα μὲν οὖν κατὰ 
τὴν Ἰνδικήν. 



Theophrastus Phil., Historia plantarum 
Book 4, chapter 4, section 14, line 3

Περιττότερα δὲ τῶν φυομένων καὶ πλεῖστον 
ἐξηλλαγμένα πρὸς τὰ ἄλλα τὰ εὔοσμα τὰ περὶ 
Ἀραβίαν καὶ Συρίαν καὶ Ἰνδούς, οἷον ὅ τε 
λιβανωτὸς καὶ ἡ σμύρνα καὶ ἡ κασία καὶ τὸ 
ὀποβάλσαμον καὶ τὸ κινάμωμον καὶ ὅσα ἄλλα 
τοιαῦτα· περὶ ὧν ἐν ἄλλοις εἴρηται διὰ πλειόνων. 



Theophrastus Phil., Historia plantarum 
Book 4, chapter 7, section 3, line 1

Οἱ δέ, ὅτε ἀνάπλους ἦν τῶν ἐξ Ἰνδῶν ἀποστα-
λέντων ὑπὸ Ἀλεξάνδρου, τὰ ἐν τῇ θαλάττῃ 
φυόμενα, μέχρι οὗ μὲν ἂν ᾖ ἐν τῷ ὑγρῷ, χρῶμά 
φασιν ἔχειν ὅμοιον τοῖς φυκίοις, ὁπόταν δ' ἐξ-  
ενεχθέντα τεθῇ πρὸς τὸν ἥλιον, ἐν ὀλίγῳ χρόνῳ 
ἐξομοιοῦσθαι τῷ ἁλί. 



Theophrastus Phil., Historia plantarum 
Book 4, chapter 7, section 8, line 1

Γίνεται δὲ τοῦτο καὶ ἐν Ἰνδοῖς, ὥσπερ ἐλέχθη, 
καὶ ἐν Ἀραβίᾳ. 



Theophrastus Phil., Historia plantarum 
Book 4, chapter 11, section 13, line 6

                  .. 
 Ὁ δὲ Ἰνδικὸς ἐν μεγίστῃ διαφορᾷ καὶ ὥσπερ 
ἕτερον ὅλως τὸ γένος· ἔστι δὲ ὁ μὲν ἄρρην στερεός, 
ὁ δὲ θῆλυς κοῖλος· διαιροῦσι γὰρ καὶ τοῦτον τῷ 
ἄρρενι καὶ θήλει. 



Theophrastus Phil., Historia plantarum 
Book 7, chapter 13, section 8, line 11

                              μεγίστη δὲ καὶ ἰδιωτάτη 
διαφορὰ τῶν ἐριοφόρων· ἔστι γάρ τι γένος τοιοῦ-
τον, ὃ φύεται μὲν ἐν αἰγιαλοῖς ἔχει δὲ τὸ ἔριον 
ὑπὸ τοὺς πρώτους χιτῶνας, ὥστε ἀνὰ μέσον εἶναι   
τοῦ τε ἐδωδίμου τοῦ ἐντὸς καὶ τοῦ ἔξω· ὑφαίνεται 
δὲ ἐξ αὐτοῦ καὶ πόδεια καὶ ἄλλα ἱμάτια· δι' ὃ 
καὶ ἐριῶδες τοῦτο καὶ οὐχ ὥσπερ τὸ ἐν Ἰνδοῖς 
τριχῶδες. 



Theophrastus Phil., Historia plantarum 
Book 8, chapter 4, section 2, line 15

διαφορὰ δὲ μεγάλη καὶ τὸ παραβλαστητικὴν 
εἶναι, καθάπερ εἴπομεν τὴν Ἰνδικήν. 



Theophrastus Phil., Historia plantarum 
Book 9, chapter 1, section 2, line 10

Ἡ δ' ὑγρότης τῶν μὲν πάχος ἔχει μόνον, ὥσπερ 
τῶν ὀπωδῶν· τῶν δὲ καὶ δακρυώδης γίνεται, καθά-
περ ἐλάτης πεύκης τερεβίνθου πίτυος ἀμυγδαλῆς 
κεράσου προύμνης ἀρκεύθου κέδρου τῆς ἀκάνθης 
τῆς Αἰγυπτίας πτελέας, καὶ γὰρ αὕτη φέρει κόμμι   
πλὴν οὐκ ἐκ τοῦ φλοιοῦ ἀλλ' ἐν τῷ κωρύκῳ, ἔτι 
δὲ ἀφ' ὧν ὁ λίβανος καὶ ἡ σμύρνα, δάκρυα γὰρ 
καὶ ταῦτα, καὶ τὸ βάλσαμον καὶ <ἡ> χαλβάνη 
καὶ εἴ τι τοιοῦτον ἕτερον, οἷόν φασι τὴν ἄκανθαν 
τὴν Ἰνδικήν, ἀφ' ἧς γίνεταί τι ὅμοιον τῇ σμύρνῃ· 
συνίσταται δὲ καὶ ἐπὶ τῆς σχίνου καὶ ἐπὶ τῆς 
ἀκάνθης τῆς ἰξίνης καλουμένης, ἐξ ὧν ἡ μαστίχη. 



Theophrastus Phil., Historia plantarum 
Book 9, chapter 7, section 2, line 12

                                         τὰ δὲ ἄλλα 
πάντα τὰ εὔοσμα οἷς πρὸς τὰ ἀρώματα χρῶνται, 
τὰ μὲν ἐξ Ἰνδῶν κομίζεται κἀκεῖθεν ἐπὶ θάλατταν 
καταπέμπεται, τὰ δ' ἐξ Ἀραβίας, οἷον πρὸς τῷ 
κιναμώμῳ καὶ τῇ κασίᾳ καὶ κώμακον· ἕτερον δ' 
εἶναι τὸ κώμακον καρπόν· τὸ δ' ἕτερον παραμίς-
γουσιν εἰς τὰ σπουδαιότατα τῶν μύρων. 



Theophrastus Phil., Historia plantarum 
Book 9, chapter 7, section 2, line 18

                                           τὸ δὲ 
καρδάμωμον καὶ ἄμωμον οἱ μὲν ἐκ Μηδείας, οἱ δ' 
ἐξ Ἰνδῶν καὶ ταῦτα καὶ τὴν νάρδον καὶ τὰ ἄλλα 
ἢ τὰ πλεῖστα. 



Theophrastus Phil., Historia plantarum 
Book 9, chapter 15, section 2, line 10

          ἐν Ἰνδοῖς δὲ καὶ ἕτερα γένη πλείω, 
περιττότατα δέ, εἴπερ ἀληθῆ λέγουσιν, ἥ τε δυνα-
μένη τὸ αἷμα διαχεῖν καὶ οἷον ὑποφεύγειν, καὶ 
πάλιν ἡ συνάγουσα καὶ πρὸς ἑαυτὴν ἐπισπωμένη, 
ἃ δή φασιν εὑρῆσθαι πρὸς τὰ τῶν ὀφιδίων τῶν 
θανατηφόρων δήγματα. 



Theophrastus Phil., De lapidibus (= fr. 2, Wimmer) (0093: 004)
“Theophrastus. De lapidibus”, Ed. Eichholz, D.E.
Oxford: Clarendon Press, 1965.
Section 36, line 5

                                           γίνεται δὲ ἐν ὀστρείῳ τινὶ 
παραπλησίῳ ταῖς πίνναις <πλὴν ἐλάττονι· μέγεθος δὲ ἡλίκον ἰχθύος 
ὀφθαλμὸς εὐμεγέθης>, φέρει δ' ἥ τε Ἰνδικὴ χώρα καὶ νῆσοί τινες τῶν 
ἐν τῇ Ἐρυθρᾷ. 



Theophrastus Phil., De lapidibus (= fr. 2, Wimmer) 
Section 38, line 3

                                                              τρόπον δέ τιν' οὐ 
πόρρω τούτου τῇ φύσει καὶ ὁ Ἰνδικὸς κάλαμος ἀπολελιθωμένος. 



Theophrastus Phil., Physicorum opiniones (0093: 008)
“Doxographi Graeci”, Ed. Diels, H.
Berlin: Reimer, 1879, Repr. 1965.
Section 12, line 50

                                                                  τὸ παραπλήσιον μέντοι 
καὶ τοὺς κατὰ τὴν Ἰνδικὴν δράκοντάς φασι πάσχειν. 



Theophrastus Phil., Fragmenta (0093: 010)
“Theophrasti Eresii opera, quae supersunt, omnia”, Ed. Wimmer, F.
Paris: Didot, 1866, Repr. 1964.
Fragment 171, section 2, line 2

   Τὰ δ' ἐν Ἰν-
δοῖς ἰχθύδια τὰ ἐκ τῶν ποταμῶν εἰς τὴν γῆν ἐξιόντα 
καὶ πηδῶντα καὶ πάλιν εἰς τὸ ὕδωρ ἀπιόντα, καθάπερ 
οἱ βάτραχοι, θαυμαστὰ μὲν οὖν, οὐχ ὁμοίως δὲ τούτοις, 
ὅσῳ τὸ ὀλίγον χρόνον ἢ πολὺν καὶ τελευταῖον ἧττον 
θαυμαστόν· ἡ δ' ὄψις ὁμοία τούτων τοῖς μαζίναις 
καλουμένοις. 



Theophrastus Phil., De causis plantarum (lib. 2–6) (0093: 014)
“Theophrasti Eresii opera, quae supersunt, omnia”, Ed. Wimmer, F.
Paris: Didot, 1866, Repr. 1964.
Book 2, chapter 10, section 2, line 8

                             Ἐπεὶ καὶ τὰ μικρόκαρπα 
πάνθ' ὡς ἐπὶ τὸ πολὺ μείζω καὶ τὰ εἰς μέγεθος ὡρ-
μημένα μικροκαρπότερα καθάπερ καὶ ἡ ἰνδικὴ συκῆ 
καλουμένη. 



Theophrastus Phil., De causis plantarum (lib. 2-6) 
Book 3, chapter 3, section 3, line 7

              Τάχα δὲ καὶ ἡ ὁρμὴ γίνεται διὰ τὸ πε-
ριέχον· ὅπου δ' αὖ θερινὸς ὄμβρος πολὺς ὥσπερ ἐν Αἰ-
θιοπίᾳ καὶ ἐν Ἰνδοῖς ἢ περὶ Αἴγυπτον ὁ Νεῖλος ἐνταῦθα 
δὴ πρὸ τούτων ἢ μετὰ τούτους εἰκὸς τὴν φυτείαν ἁρ-
μόττειν· τηνικαῦτα γὰρ ἡ τοῦ ἀέρος κρᾶσις σύμμε-
τρος. 

\end{greek}


\section{Ephraem the Syrian}
\blockquote[From Wikipedia\footnote{\url{http://en.wikipedia.org/wiki/Ephraem}}]{Ephrem the Syrian (Syriac: ܡܪܝ ܐܦܪܝܡ ܣܘܪܝܝܐ, Mār Efrêm Sûryāyâ;[1] Greek: Ἐφραίμ ὁ Σῦρος; Latin: Ephraem Syrus; ca. 306 – 373) was a Syriac deacon and a prolific Syriac-language hymnographer and theologian of the 4th century.[2][3][4][5] His works are hailed by Christians throughout the world and many denominations venerate him as a saint. He has been declared a Doctor of the Church in Roman Catholicism. He is especially beloved in the Syriac Orthodox Church.}

\begin{greek}

Ephraem Hist., Poeta, Chronicon (3170: 001)
“Ephraemius”, Ed. Bekker, I.
Bonn: Weber, 1840; Corpus scriptorum historiae Byzantinae.


Ephraem Hist., Poeta, Chronicon 
Line 3137

Μουχούμετ ἀνὴρ ἡγεμὼν τῆς Περσίδος 
ἤρατο μάχην κατὰ Βαβυλωνίων 
Ἰνδῶν θ', ἑαυτῷ δυσμενῶν ἀντιπάλων, 
δι' ὧν τροποῦται τοὺς ἐχθροὺς κατὰ κράτος. 

\end{greek}


\section{\emph{Artaxerxis Epistulae}}%!see first line of these, below -- pseudographic ... http://www.archive.org/stream/epistolographoih00hercuoft#page/174/mode/2up
%??? time period totally unknown
\blockquote[From Wikipedia\footnote{\url{}}]{}
\begin{greek}

Artaxerxis Epistulae, Epistulae (0045: 001)
“Epistolographi Graeci”, Ed. Hercher, R.
Paris: Didot, 1873, Repr. 1965.
Epistle 1, line 1

Βασιλεὺς μέγας Ἀρταξέρξης τοῖς ἀπὸ Ἰνδικῆς ἕως 
τῆς Αἰθιοπίας ἑπτὰ καὶ εἴκοσι καὶ ἑκατὸν σατραπειῶν 
ἄρχουσι τάδε γράφει. 

\end{greek}

\section{Polemon Periegetes}
\blockquote[From Wikipedia\footnote{\url{http://en.wikipedia.org/wiki/Polemon_of_Athens}}]{Polemon (2nd century BCE) was a Stoic philosopher and geographer. Of Athenian citizenship, he is known as Polemon of Athens, but he was born either in Ilium, Samos, or Sicyon, and is also known as Polemon of Ilium and Polemon Periegetes. He travelled throughout Greece, and wrote about the places he visited. He also compiled a collection of the epigrams he saw on the monuments and votive offerings. None of these works survive, but many later writers quote from them.

n his travels, Polemon collected the epigrams he found into a work On the inscriptions to be found in cities (Greek: Περὶ τω̂ν κατὰ πόλεις ἐπιγραμμάτων).[4] In addition, other works of his are mentioned, upon the votive offerings and monuments in the Acropolis of Athens, at Lacedaemon, at Delphi, and elsewhere, which no doubt contained copies of numerous epigrams. His works may have been a chief source of the Garland of Meleager. Athenaeus and other writers make very numerous quotations from his works. They were chiefly descriptions of different parts of Greece; some are on paintings preserved in various places, and several are controversial, among which is one against Eratosthenes.

Sir James Frazer considered him the most learned of all Greek antiquaries. "His acquaintance both with the monuments and with the literature seems to have been extensive and profound. The attention which he bestowed on inscriptions earned for him the nickname of the 'monument-tapper.'"[5]}

\begin{greek}

Polemon Perieg., Fragmenta (0586: 001)
“FHG 3”, Ed. Müller, K.
Paris: Didot, 1853.
Fragment 92, line 1

Idem II: Καὶ περὶ τὸν Ἰνδὸν δέ φησι 
ποταμὸν γίνεσθαι τὴν κυνάραν. 



Polemon Perieg., Fragmenta 
Fragment 92, line 6

                                  Καὶ Σκύλαξ δὲ ἢ Πο-
λέμων γράφει· «εἶναι μὲν τὴν γῆν ὑδρηλὴν κρήνῃσι 
καὶ ὀχετοῖσιν, ἐν δὲ τοῖς οὔρεσι πέφυκε κυνάρα καὶ 
βοτάνη ἄλλη,» καὶ ἐν τοῖς ἑξῆς· «ἐντεῦθεν δὲ ὄρος 
παρέτεινε τοῦ ποταμοῦ τοῦ Ἰνδοῦ ἔνθεν καὶ ἔνθεν 
ὑψηλόν τε καὶ δασὺ ἀγρίῃ ὕλῃ καὶ ἀκάνθῃ κυνάρᾳ. 

\end{greek}


\section{Phylarchus}
\blockquote[From Wikipedia\footnote{\url{http://en.wikipedia.org/wiki/Phylarchus}}]{Phylarchus (Greek: Φύλαρχoς, Phylarkhos; lived 3rd century BC) was a Greek historical writer whose works have been lost, but not before having been considerably used by other historians whose works have survived.}
\begin{greek}

Phylarchus Hist., Fragmenta (1609: 002)
“FGrH \#81”.
Volume-Jacobyʹ-F 2a,81,F, fragment 35a, line 2

           ηist. mir. 18: Φύλαρχος ἐν <κ> τῶν Ἱστοριῶν 
ἐκ τῆς Ἰνδικῆς φησιν ἐνεχθῆναι λευκὴν ῥίζαν, ἣν κόπτοντας μεθ' ὕδατος 
καταπλάττειν τοὺς πόδας, τοὺς δὲ καταπλασθέντας ἄνδρας τῆς συνουσίας 
λήθην ἴσχειν καὶ γίγνεσθαι ὁμοίους εὐνούχοις. 



Phylarchus Hist., Fragmenta 
Volume-Jacobyʹ-F 2a,81,F, fragment 35b, line 2

           ι 32 
p. 18 D E: Φύλαρχος δὲ Σανδρόκοττόν φησι τὸν Ἰνδῶν βασιλέα Σελεύκωι 
μεθ' ὧν ἔπεμψε δώρων ἀποστεῖλαί τινας δυνάμεις στυτικὰς τοιαύτας ὡς 
ὑπὸ τοὺς πόδας τιθεμένας τῶν συνουσιαζόντων οἷς μὲν ὁρμὰς ἐμποιεῖν 
ὀρνίθων δίκην, οὓς δὲ καταπαύειν. 



Phylarchus Hist., Fragmenta 
Volume-Jacobyʹ-F 2a,81,F, fragment 36, line 5

          γράφει δὲ οὕτως· <«τούτωι δὲ τῶι ἐλέφαντι συνε-
τρέφετο θήλεια ἐλέφας, ἣν Νίκαιαν ἐκάλουν· ἧι τελευ-
τῶσα ἡ τοῦ τρέφοντος Ἰνδοῦ γυνὴ παιδίον αὑτῆς τρια-
κοσταῖον παρακατέθετο. 



Phylarchus Hist., Fragmenta 
Volume-Jacobyʹ-F 2a,81,F, fragment 41, line 15

                                                                             τὰς 
δὲ χρυσᾶς πλατάνους καὶ τὴν χρυσῆν ἄμπελον, ὑφ' ἣν οἱ Περσῶν βασι-
λεῖς ἐχρημάτιζον πολλάκις καθήμενοι, σμαραγδίνους βότρυς ἔχουσαν καὶ 
τῶν Ἰνδικῶν ἀνθράκων ἄλλων τε παντοδαπῶν λίθων ὑπερβαλλόντων ταῖς 
πολυτελείαις, ἐλάττω φησὶν ὁ Φύλαρχος φαίνεσθαι τῆς καθ' ἡμέραν ἑκάστοτε 
γινομένης παρ' Ἀλεξάνδρωι δαπάνης. 



Phylarchus Hist., Fragmenta 
Volume-Jacobyʹ-F 2a,81,F, fragment 78, line 4

          et οs. 29 p. 362 βξ: οὐ γὰρ ἄξιον προσέχειν 
τοῖς Φρυγίοις γράμμασιν, ἐν οἷς λέγεται Χάροπος μὲν τοῦ Ἡρακλέους 
γενέσθαι θυγάτηρ Ἶσις, Αἰακοῦ δὲ τοῦ Ἡρακλέους ὁ Τυφών· οὐδὲ Φυ-
λάρχου μὴ καταφρονεῖν γράφοντος, ὅτι πρῶτος εἰς Αἴγυπτον ἐξ Ἰνδῶν 
Διόνυσος ἤγαγε δύο βοῦς, ὧν ἦν τῶι μὲν Ἆπις ὄνομα τῶι δ' Ὄσιρις. 

\end{greek}

\section{Demosthenes}
\blockquote[From Wikipedia\footnote{\url{}}]{}
\begin{greek}


Demosthenes Orat., Epistulae (0014: 063)
“Demosthenis orationes, vol. 3”, Ed. Rennie, W.
Oxford: Clarendon Press, 1931, Repr. 1960.
Epistle 4, section 7, line 5

                                        καὶ ἐῶ Καππαδόκας καὶ 
Σύρους καὶ τοὺς τὴν Ἰνδικὴν χώραν κατοικοῦντας ἀνθρώ-
πους ἐπ' ἔσχατα γῆς· οἷς ἅπασι συμβέβηκε πολλὰ καὶ δεινὰ 
πεπονθέναι καὶ χαλεπά. 

\end{greek}



\section{\emph{Scholia In Apollonium Rhodium}}

\blockquote[From Wikipedia\footnote{\url{}}]{}

\begin{greek}

Scholia In Apollonium Rhodium, Scholia in Apollonii Rhodii Argonautica (scholia vetera) (5012: 001)
“Scholia in Apollonium Rhodium vetera”, Ed. Wendel, K.
Berlin: Weidmann, 1935, Repr. 1974.
Page 80, line 1

                                     Διονύσου ἐρασθεῖσα Ἀφροδίτη ἐμίγη   
αὐτῷ καὶ ἀναχωρήσαντος αὐτοῦ εἰς τὴν Ἰνδικὴν ἐμίγη τῷ Ἀδώνιδι. 



Scholia In Apollonium Rhodium, Scholia in Apollonii Rhodii Argonautica (scholia vetera) 
Page 193, line 20

                                     οὕτω δὲ κέκληται ὁ ποταμὸς ἀπὸ τοῦ 
τὸν Διόνυσον αὐτόθι στῆσαι χορόν, ὅτε ἀπὸ Ἰνδῶν ὑπέστρεφε. 



Scholia In Apollonium Rhodium, Scholia in Apollonii Rhodii Argonautica (scholia vetera) 
Page 193, line 21

                                  ὅτι δὲ ἐπολέμησεν Ἰνδοὺς ὁ Διόνυσος, 
Διονύσιός (32 fg 13 J.) φησι καὶ Ἀριστόδημος ἐν αʹ Θηβαϊκῶν ἐπιγραμ-  
μάτων (fg VII Radtke Herm. 36, 1901, 54) καὶ Κλείταρχος ἐν ταῖς 
Περὶ Ἀλέξανδρον ἱστορίαις (137 fg 17 J.), προσιστορῶν, ὅτι καὶ Νύσα 
ὄρος ἐστὶν ἐν Ἰνδικῇ καὶ κισσῷ προσόμοιον φυτὸν φυτεύεται ἐκεῖ, ὃ 
προσαγορεύεται σκινδαψός. 

\end{greek}


\section{Philochorus}

\blockquote[From Wikipedia\footnote{\url{http://en.wikipedia.org/wiki/Philochorus}}]{Philochorus, of Athens, Greek historian during the 3rd century BC, (d. circa 261 BCE), was a member of a priestly family. He was a seer and interpreter of signs, and a man of considerable influence.

He was strongly anti-Macedonian in politics, and a bitter opponent of Demetrius Poliorcetes. When Antigonus Gonatas, the son of the latter, besieged and captured Athens (261), Philochorus was put to death for having supported Ptolemy Philadelphus, who had encouraged the Athenians in their resistance to Macedonia.

His investigations into the usages and customs of his native Attica were embodied in an Atthis, in seventeen books, a history of Athens from the earliest times to 262 BC. Considerable fragments are preserved in the lexicographers, scholiasts, Athenaeus, and elsewhere. The work was epitomized by the author himself, and later by Asinius Pollio of Tralles (perhaps a freedman of the famous Gaius Asinius Pollio).

Philochorus also wrote on oracles, divination and sacrifices; the mythology and religious observances of the tetrapolis of Attica; the myths of Sophocles; the lives of Euripides and Pythagoras; the foundation of Salamis, Cyprus. He compiled chronological lists of the archons and Olympiads, and made a collection of Attic inscriptions, the first of its kind in Greece.}

\begin{greek}

Philochorus Hist., Fragmenta (0583: 002)
“FGrH \#328”.
Volume-Jacobyʹ-F 3b,328,F, fragment 7a, line 2

                                            p. 307, 
1 Bonn.): Διονύσου πράξεις καὶ τὰ περὶ Ἰνδούς, Λυκοῦργόν τε καὶ Ἀ-
κταίωνα καὶ Πενθέα, ὅπως τε Περσεῖ συστὰς εἰς μάχην ἀναιρεῖται, ὥς φησι 
Δείναρχος ὁ ποιητής, οὐχ ὁ ῥήτωρ. 

\end{greek}

\section{Aristodemus}%???

Who is this?

%\blockquote[From Wikipedia\footnote{\url{}}]{}

\begin{greek}
Aristodemus Hist., Myth., Fragmenta (1875: 002)
“FHG 3”, Ed. Müller, K.
Paris: Didot, 1841–1870.
Fragment 1d, line 2

ΘΗΒΑΙΚΑ ΕΠΙΓΡΑΜΜΑΤΑ

  
E LIBRO PRIMO.


 Schol. Apoll. Rhod. II, 904: Ὅτι κατεπολέμη-
σεν Ἰνδοὺς ὁ Διόνυσος, Διονύσιος φησὶ καὶ Ἀριστόδη-
μος ἐν πρώτῳ Θηβαϊκῶν ἐπιγραμμάτων. 

\end{greek}




\section{\emph{Scholia In Euclidem}}

\blockquote[From Wikipedia\footnote{\url{http://en.wikipedia.org/wiki/Euclid}.}]{Euclid (play /ˈjuːklɪd/ EWK-lid; Ancient Greek: Εὐκλείδης Eukleidēs), fl. 300 BC, also known as Euclid of Alexandria, was a Greek mathematician, often referred to as the "Father of Geometry". He was active in Alexandria during the reign of Ptolemy I (323–283 BC). His Elements is one of the most influential works in the history of mathematics, serving as the main textbook for teaching mathematics (especially geometry) from the time of its publication until the late 19th or early 20th century.[1][2][3] In the Elements, Euclid deduced the principles of what is now called Euclidean geometry from a small set of axioms. Euclid also wrote works on perspective, conic sections, spherical geometry, number theory and rigor.}

\begin{greek}
Scholia In Euclidem, Scholia in Euclidis elementa (scholia vetera et recentiora) (5022: 001)
“Euclidis opera omnia, vols. 5.1–5.2, 2nd edn.”, Ed. Stamatis, E.S. (post J.L. Heiberg)
Leipzig: Teubner, 1977.
Book 9, scholion 11, line 1

Ad prop. 8


Δῆλον ἐκ τῶνδε, διὰ τί ἐν τῇ Ἰνδικῇ ψήφῳ ἐν ταῖς 
τῶν πλευρῶν τῶν τετραγώνων λήψεσιν ἀνὰ μείζονα τὸ 
γίνεται, οὐ γίνεται, γίνεται, οὐ γίνεται λέγομεν, διότι ἥ 
τε μονὰς τετράγωνός ἐστι καὶ ὁ τρίτος ἀπ' αὐτῆς καὶ ὁ πά-
λιν τρίτος μετ' αὐτὸν καὶ ἑξῆς. 



Scholia In Euclidem, Scholia in Euclidis elementa (scholia vetera et recentiora) 
Book 10, scholion 9, line 66

ἢ ἐπειδὴ τὰ μέτρα θέσει ἐξ ἡμῶν αὐτῶν λαμβάνεται καὶ 
οὐ φύσει, καὶ εἰκός ἐστι παρ' ἡμῖν, εἰ οὕτως ἔτυχε, τὸν 
πῆχυν δέκα δακτύλων εἶναι, παρ' ἄλλοις δὲ οἷον Ἰνδοῖς 
ὀκτὼ δακτύλων καὶ παρ' ἄλλοις ἄλλων, διὰ τοῦτο πρόσκει-
ται τὸ δεῖν αἰτῆσαι πηλικότητα πήχεως, ὡς εἰ ἐλέγομεν· 
δεῖ λαβεῖν τὴν πηλικότητα τοῦ πήχεως ὡρισμένην, ὥσπερ 
κἂν τὸν πῆχυν ἡμᾶς ἔροιτό τις, πόσων ἐστὶ δακτύλων, δεῖ 
αἰτῆσαι τὸ πηλίκον αὐτοῦ· οὐδὲ γὰρ ὁ δάκτυλος οὐδ' 
ὁ ποῦς οὐδ' ὁ μέδιμνος οὐδ' ἄλλο οὐδὲν παρὰ πᾶσίν ἐστι τὰ 
αὐτά, ὡς εἴρηται. 


\end{greek}



\section{Ephorus}

\blockquote[From Wikipedia\footnote{\url{http://en.wikipedia.org/wiki/Ephorus}.}]{of Cyme in Aeolia, in Asia Minor, was an ancient Greek historian. Information on his biography is limited; he was the father of Demophilus, who followed in his footsteps as a historian, and to Plutarch's claim that Ephorus declined Alexander the Great's offer to join him on his Persian campaign as the official historiographer.[1] Together with the historian Theopompus, he was a pupil of Isocrates, in whose school he attended two courses of rhetoric.[citation needed] But he does not seem to have made much progress in the art, and it is said to have been at the suggestion of Isocrates himself that he took up literary composition and the study of history.[citation needed]}

\begin{greek}
Ephorus Hist., Fragmenta (0536: 003)
“FGrH \#70”.
Volume-Jacobyʹ-F 2a,70,F, fragment 30a, line 4

                                                   > 
 STRABON I 2, 28: μηνύει δὲ καὶ Ἔφορος τὴν παλαιὰν 
περὶ τῆς Αἰθιοπίας δόξαν, ὅς φησιν ἐν τῶι περὶ τῆς Εὐρώπης λόγωι,   
τῶν περὶ τὸν οὐρανὸν καὶ τὴν γῆν τόπων εἰς τέσσαρα μέρη διηιρημένων, 
τὸ πρὸς τὸν ἀπηλιώτην Ἰνδοὺς ἔχειν, πρὸς νότον δὲ Αἰθίοπας, πρὸς δύσιν 
δὲ Κελτούς, πρὸς δὲ βορρᾶν ἄνεμον Σκύθας. 



Ephorus Hist., Fragmenta 
Volume-Jacobyʹ-F 2a,70,F, fragment 30b, line 1

                                                               > 
 KOSMAS INDIKOPL. 



Ephorus Hist., Fragmenta 
Volume-Jacobyʹ-F 2a,70,F, fragment 30b, line 3

                                                        <«τὸν μὲν γὰρ ἀπηλιώ-
την καὶ τὸν ἐγγὺς ἀνατολῶν τόπον Ἰνδοὶ κατοικοῦσι· 
τὸν δὲ πρὸς νότον καὶ μεσημβρίαν Αἰθίοπες νέμονται· 
τὸν δὲ ἀπὸ ζεφύρου καὶ δυσμῶν Κελτοὶ κατέχουσι· τὸν 
δὲ κατὰ βορρᾶν καὶ τοὺς ἄρκτους Σκύθαι κατοικοῦσιν. 



Ephorus Hist., Fragmenta 
Volume-Jacobyʹ-F 2a,70,F, fragment 30b, line 9

ἔστιν μὲν οὖν οὐκ ἶσον ἕκαστον τούτων τῶν μερῶν, ἀλλὰ 
τὸ μὲν τῶν Σκυθῶν καὶ τῶν Αἰθιόπων μεῖζον, τὸ δὲ τῶν 
Ἰνδῶν καὶ τῶν Κελτῶν ἔλαττον. 



Ephorus Hist., Fragmenta 
Volume-Jacobyʹ-F 2a,70,F, fragment 30b, line 11

                                       καὶ παραπλήσιον ἑκα-
τέρων ἀλλήλοις ἔχει τοῦ τόπου τὸ μέγεθος· οἱ μὲν γὰρ 
<Ἰνδοί> εἰσι μεταξὺ θερινῶν καὶ χειμερινῶν ἀνατολῶν· 
Κελτοὶ δὲ τὴν ἀπὸ θερινῶν μέχρι χειμερινῶν δυσμῶν 
χώραν κατέχουσι· καὶ τοῦτο μὲν ἶσόν ἐστιν ἐκείνωι τῶι 
διαστήματι καὶ μάλιστά πως ἀντικείμενον. 



Ephorus Hist., Fragmenta 
Volume-Jacobyʹ-F 2a,70,F, fragment 30b, line 19

<<ΝΟΤΟΣ 
ΧΕΙΜΕΡΙΝΗ ΑΝΑΤΟΛΗ <ΑΙΘΙΟΠΕΣ> ΧΕΙΜΕΡΙΝΗ ΔΥΣΙΣ> 
<<ΑΠΗΛΙΩΤΗΣ <ΙΝΔ*οΙ>> <<ΚΕΛΤΟΙ> ΖΕΦΥΡΟΣ>> 
<ΘΕΡΙΝΗ ΑΝΑΤΟΛΗ <ΣΚΥΘΑΙ> ΘΕΡΙΝΗ ΔΥΣΙΣ 
ΒΟΡΡΑΣ>> 
ἀκριβῶς ὁ Ἔφορος καὶ λόγωι καὶ τῆι καταγραφῆι . 



Ephorus Hist., Fragmenta 
Volume-Jacobyʹ-F 2a,70,F, fragment 30c, line 4

                                                                      | τὴν μὲν γὰρ ἐντὸς 
ἀνατολῶν πᾶσαν σχεδὸν | οἰκοῦσιν Ἰνδοί, τὴν δὲ πρὸς μεσημβρίαν | Αἰθίοπες 
ἐγγὺς κείμενοι νότου πνοῆς· | τὸν ἀπὸ ζεφύρου Κελτοὶ δὲ μέχρι δυσμῶν τόπον | 
θερινῶν ἔχουσιν, τὸν δὲ πρὸς βορρᾶν Σκύθαι. 



Ephorus Hist., Fragmenta 
Volume-Jacobyʹ-F 2a,70,F, fragment 191, line 22

        παραταχ[θεί|ς]ας δὲ πολὺν χρόνο[ν] | πολλὰς μὲν τῶν κ̣[ιν]|δυνευουσῶν 
βαρβα̣[ρι]|κῶν νεῶν διέφθε[ι|ρ]εν, ἑκατὸν δ' αὐτοῖς | ἀνδράσιν εἷλε ζωγρή|[σας τ]ὸν 
π̣[.....]ων. col. II v. 77 – 83 zeilenanfänge. 
frg. 11: ........ τὸν μὲ]ν | [στρατηγὸ]ν̣ αὐτῶν | [Φερενδάτη]ν̣ (F 192) ἀδελ|[φιδοῦν 
ὄντ]α̣ τοῦ βας[ι|λέως ἐν τῆι] σκηνῆι 
frg. 12 – 13 col. I 89 – 90:  – col. II ............]ε̣ | [..]δ̣ι̣ετέλ[ουν ὄ]ν̣τες· | [ὥστ]ε 
νομίζοντες ἀ|πὸ τῆς ἠπείρ[ου] τὴν | ἔφοδον αὐτ[οῖς γεγ]ο|νέναι τῶν π[ο]λ̣ε̣μί|ων πρὸς 
τὰ[ς] ν̣α̣ῦ[ς] ἔ̣|φευγον, ὑπολαμβά|νοντες αὐτοῖς εἶναι | φιλίας. 



Ephorus Hist., Fragmenta 
Volume-Jacobyʹ-F 2a,70,F, fragment 208, line 5

                                                           (2) ἐπεὶ δὲ αἵ τε 
παρ' Ἰνδῶν καί τινων ἄλλων ἐθνῶν καθυστέρουν διὰ τὸ μακρὰν ἀφεστάναι 
τοὺς τόπους, μετὰ τῆς συναχθείσης στρατιᾶς ὥρμησεν ἀπαντήσων τῶι 
Κύρωι. 

\end{greek}


\section{Lycophron (scholia in)}

\blockquote[From Wikipedia\footnote{\url{}.}]{a Hellenistic Greek tragic poet, grammarian, and commentator on comedy, to whom the poem Alexandra is attributed (perhaps falsely).

One poem traditionally attributed to him, Alexandra or Cassandra,[1] has been preserved in its complete form, running to 1474 iambic trimeters. It consists of a prophecy uttered by Cassandra and relates the later fortunes of Troy and of the Greek and Trojan heroes. References to events of mythical and later times are introduced, and the poem ends with a reference to Alexander the Great, who was to unite Asia and Europe in his world-wide empire.

The style obtained for the poem's author, even among the ancients, the title of "obscure"; one modern scholar says the Alexandra "may be the most illegible piece of classical literature, one which nobody can read without a proper commentary and which even then makes very difficult reading."[2] The poem is evidently intended to display the writer's knowledge of obscure names and uncommon myths; it is full of unusual words of doubtful meaning gathered from the older poets, and long-winded compounds coined by the author. It was probably written as a show-piece for the Alexandrian school, rather than as straight poetry. It was very popular in the Byzantine period, and was read and commented on very frequently; the manuscripts of the Cassandra are numerous. Two explanatory paraphrases of the poem survive, and the collection of scholia by Isaac and John Tzetzes is very valuable (much used by, among others, Robert Graves in his Greek Myths).}

\begin{greek}

Scholia In Lycophronem, Scholia in Lycophronem (scholia vetera et recentiora partim Isaac et Joannis Tzetzae) (5030: 001)
“Lycophronis Alexandra, vol. 2”, Ed. Scheer, E.
Berlin: Weidmann, 1958.
Scholion 174, line 9

οἱ δὲ Κόλχοι Ἰνδικοὶ Σκύθαι εἰσὶν οἱ καὶ Λαζοὶ καλούμενοι 
πλησίον οἰκοῦντες Ἀβασγῶν τῶν πρὶν Μασσαγετῶν †T ὧν 
Κόλχων τὰ φάρμακα αὐθημερινὸν ἀναιροῦσιν. 



Scholia In Lycophronem, Scholia in Lycophronem (scholia vetera et recentiora partim Isaac et Joannis Tzetzae) 
Scholion 175, line 96

         ὕστερον δὲ τῷ Θησεῖ ἐπιβουλεύσασα ἐκβάλλεται 
τῆς πατρίδος μετὰ τοῦ παιδός, ὃς βαρβάρων ἐπικρατήσας 
τὴν ὑπ' αὐτὸν Μηδείαν ἐκάλεσε καὶ στρατεύσας ἐν Ἰνδοῖς 
τελευτᾷ. 



Scholia In Lycophronem, Scholia in Lycophronem (scholia vetera et recentiora partim Isaac et Joannis Tzetzae) 
Scholion 254, line 2a

                                                          * T 
τὸ <ἰνδάλλεται> οὐ μόνον 
ἐπὶ τῆς ὄψεως, ἀλλὰ καὶ ἀκοῆς 
τέταχε. 



Scholia In Lycophronem, Scholia in Lycophronem (scholia vetera et recentiora partim Isaac et Joannis Tzetzae) 
Scholion 254, line 5b

         ss3   
<ἰνδάλλεται> ἀπεικάζεται 
ὁμοιοῦται καταχρηστικῶς· κυ-
ρίως γὰρ ἐπ' ὀφθαλμοῖς λέ-
γομεν τὸ ἰνδάλλειν. 



Scholia In Lycophronem, Scholia in Lycophronem (scholia vetera et recentiora partim Isaac et Joannis Tzetzae) 
Scholion 595, line 3

                                    ss3 
 <οἳ θαλασσίαν>· οἵτινες φίλοι τοῦ Διομήδους ἐρω-
διοὶ ἢ λάροι γενόμενοι <αἰνήσουσι> τὴν ἐν θαλάσσῃ δια-
γωγὴν καὶ <δίκην> τῶν <πορκέων> καὶ ἁλιέων <ἰνδαλθέν-
τες> καὶ ὁμοιωθέντες εἰς τὴν <δομὴν> καὶ δέμας κατὰ μετα-
πλασμὸν <κύκνοις εὐγλήνοις> καὶ εὐοφθάλμοις. 

\end{greek}


\section{Phaenias of Eresus}

\blockquote[From Wikipedia\footnote{\url{http://en.wikipedia.org/wiki/Phaenias_of_Eresus}.}]{Phaenias of Eresus (Ancient Greek: Φαινίας ὁ Ἐρέσιος, Phainias; also Phanias) was a Greek philosopher from Lesbos, important as an immediate follower of and commentator on Aristotle. He came to Athens about 332 BCE, and joined his compatriot, Theophrastus, in the Peripatetic school. His writings on logic and science appear to have been commentaries or supplements to the works of Aristotle and Theophrastus. He also wrote extensively on history. None of his works have survived.}

\begin{greek}

Phaenias Phil., Fragmenta (1578: 001)
“Phainias von Eresos. Chamaileon. Praxiphanes”, Ed. Wehrli, F.
Basel: Schwabe, 1969; Die Schule des Aristoteles, vol. 9, 2nd edn..
Fragment 12, line 16

  ib. XXIII 422 d: ἐλέγχει δ' αὐτὸν (sc. τὸν ξένον) ὁ τῶν κόσμων ἀριθ-
 μὸς οὐκ ὢν Αἰγύπτιος οὐδ' Ἰνδὸς ἀλλὰ Δωριεὺς ἀπὸ Σικελίας, ἀνδρὸς 
 Ἱμεραίου τοὔνομα Πέτρωνος. 



Phaenias Phil., Fragmenta 
Fragment 45, line 7

                                                                     ὑφαίνεται δ' ἐξ αὐτοῦ 
  καὶ πόδεια καὶ ἄλλα ἱμάτια”, ὡς καὶ Φαινίας φησί· “τὸ δὲ ἐν Ἰνδοῖς τριχῶ-
  δές ἐστι. 
\end{greek}

\section{Berossus}

\blockquote[From Wikipedia\footnote{\url{http://en.wikipedia.org/wiki/Berossus}.}]{Berossus (play /bəˈrɒsəs/) or Berosus (play /bəˈroʊsəs/; Akkadian: Bēl-rē'ušu, "Bel is his shepherd"; Greek: Βήρωσσος[1]) was a Hellenistic-era Babylonian writer, a priest of Bel Marduk[2] and astronomer writing in Greek, who was active at the beginning of the 3rd century BC. Versions of two excerpts of his writings survive, at several removes.}

\begin{greek}

Beros(s)us Astrol., Hist., Fragmenta (1222: 003)
“FHG 2”, Ed. Müller, K.
Paris: Didot, 1841–1870.
Fragment 14c, line 15

           Περὶ τούτων γοῦν συμφωνεῖ καὶ Φιλόστρα-
τος ἐν ταῖς Ἱστορίαις, μεμνημένος τῆς Τύρου πολιορ-
κίας, καὶ Μεγασθένης ἐν τῇ τετάρτῃ τῶν Ἰνδικῶν, 
δι' ἧς ἀποφαίνειν πειρᾶται τὸν προειρημένον βασιλέα 
τῶν Βαβυλωνίων Ἡρακλέους ἀνδρείᾳ καὶ μεγέθει πρά-
ξεων διενηνοχέναι· καταστρέψασθαι γὰρ αὐτόν φησι 
καὶ Λιβύης τὴν πολλὴν καὶ Ἰβηρίαν. 

\end{greek}



\section{\emph{Scholia In Theocritum}}
I have no idea of the date of this.

Scholia In Theocritum, Scholia in Theocritum (scholia vetera) (5038: 001)
“Scholia in Theocritum vetera”, Ed. Wendel, K.
Leipzig: Teubner, 1914, Repr. 1967.
Prolegomenon-anecdote-poem 17, section-verse 106/107, line 5
\begin{greek}
               περὶ δὲ τῶν μυρμήκων τῶν μεταλλευόντων 
χρυσὸν ἐν Ἰνδικοῖς πολλοὶ ἱστορήκασιν. 
\end{greek}


\section{Megasthenes, \emph{Indica}}

\subsection{About Megasthenes}

\blockquote[From Wikipedia]{Megasthenes (Μεγασθένης, ca. 350 – 290 BCE) was a Greek ethnographer and explorer in the Hellenistic period, author of the work Indica. He was born in Asia Minor (modern day Turkey) and became an ambassador of Seleucus I of the Seleucid dynasty possibly to Chandragupta Maurya in Pataliputra, India. However the exact date of his embassy is uncertain. Scholars place it before 298 BC, the date of Chandragupta's death.

Arrian explains that Megasthenes lived in Arachosia, with the satrap Sibyrtius, from where he visited India:
"Megasthenes lived with Sibyrtius, satrap of Arachosia, and often speaks of his visiting Sandracottus, the king of the Indians." Arrian, Anabasis Alexandri [1]

We have more definite information regarding the parts of India Megasthenes visited. He entered the subcontinent through the district of the Pentapotamia, providing a full account of the rivers there (thought to be the five affluents of the Indus that form the Punjab region), and proceeded from there by the royal road to Pataliputra. There are accounts of Megasthenes having visited Madurai (then, a bustling city and capital of the Pandyas), but he appears not to have visited any other parts of India.

At the beginning of his Indica, he refers to the older Indians who know about the prehistoric arrival of Dionysus and Hercules in India, which was a story very popular amongst the Greeks during the Alexandrian period. Particularly important are his comments on the religions of the Indians. He mentions the devotees of Heracles (Lord Krishna) and Dionysus (Lord Shiva or King Lord Indra), but he does not mention Buddhists, something that gives support to the theory that the latter religion was not widely known before the reign of Ashoka.[2]

His Indica served as an important source for many later writers such as Strabo and Arrian. He describes such features as the Himalayas and the island of Sri Lanka. He also describes a caste system different from the one that exists today, which shows that the caste system may to some extent be fluid and evolve. However, it might be that, being a foreigner, he was not adequately informed about the caste system. His description follows:

``The first is formed by the collective body of the Philosophers, which in point of number is inferior to the other classes, but in point of dignity preeminent over all. The philosopher who errs in his predictions incurs censure, and then observes silence for the rest of his life.

``The second caste consists of the Husbandmen, who appear to be far more numerous than the others. They devote the whole of their time to tillage; nor would an enemy coming upon a husbandman at work on his land do him any harm, for men of this class, being regarded as public benefactors, are protected from all injury.

``The third caste consists of the Shepherds and in general of all herdsmen who neither settle in towns nor in villages, but live in tents.

``The fourth caste consists of the Artizans. Of these some are armourers, while others make the implements that husbandmen and others find useful in their different callings. This class is not only exempted from paying taxes, but even receives maintenance from the royal exchequer.

``The fifth caste is the Military. It is well organized and equipped for war, holds the second place in point of numbers, and gives itself up to idleness and amusement in the times of peace. The entire force--men-at-arms, war-horses, war-elephants, and all--are maintained at the king's expense.

``The sixth caste consists of the Overseers. It is their province to inquire into and superintend all that goes on in India, and make report to the king, or, where there is not a king, to the magistrates.

``The seventh caste consists of the Councillors and Assessors,--of those who deliberate on public affairs. It is the smallest class, looking to number, but the most respected, on account of the high character and wisdom of its members; for from their ranks the advisers of the king are taken, and the treasurers, of the state, and the arbiters who settle disputes. The generals of the army also, and the chief magistrates, usually belong to this class.

Later writers such as Arrian, Strabo, Diodorus, and Pliny refer to Indica in their works. Of these writers, Arrian speaks most highly of Megasthenes, while Strabo and Pliny treat him with less respect. Indica contained many legends and fabulous stories, similar to those we find in the Indica of Ctesias.''

Megasthenes' Indica is the first well-known Western account of India and he is regarded as one of the founders of the study of Indian history in the West. He is also the first foreigner Ambassador to be mentioned in the Indian history.

Megasthenes also comments on the presence of pre-Socratic views among the Brahmans and Jews. Five centuries later Clement of Alexandria, in his Stromateis, may have misunderstood Megasthenes to be responding to claims of Greek primacy by admitting Greek views of physics were preceded by those of Jews and Indians. Megasthenes, like Numenius of Apamea, was simply comparing the ideas of the different ancient cultures.[3]}

\subsection{\emph{Indica}}
Text: Megasthenes Hist., Fragmenta (1489: 003) | “FHG 2”, Ed. Müller, K. |Paris: Didot, 1841–1870.

\begin{greek}
ΙΝΔΙΚΑ.
(ΕΠΙΤΟΜΗ.)
1.1
 Diodorus II, 35: Ἡ τοίνυν Ἰνδικὴ τετράπλευρος
οὖσα τῷ σχήματι, τὴν μὲν πρὸς ἀνατολὰς νεύουσαν
πλευρὰν καὶ τὴν πρὸς τὴν μεσημβρίαν ἡ μεγάλη πε-
ριέχει θάλαττα, τὴν δὲ πρὸς τὰς ἄρκτους τὸ Ἠμωδὸν
1.5
ὄρος διείργει τῆς Σκυθίας, ἣν κατοικοῦσι τῶν Σκυθῶν
οἱ προσαγορευόμενοι Σάκαι· τὴν δὲ τετάρτην τὴν πρὸς
δύσιν ἐστραμμένην διείληφεν ὁ Ἰνδὸς προσαγορευόμενος
ποταμὸς, μέγιστος ὢν σχεδὸν τῶν ἁπάντων μετὰ τὸν
Νεῖλον. (2) Τὸ δὲ μέγεθος τῆς ὅλης Ἰνδικῆς φασιν ὑπάρ-
1.10
χειν ἀπὸ μὲν ἀνατολῶν πρὸς δύσιν δισμυρίων ὀκτακις-
χιλίων σταδίων, ἀπὸ δὲ τῶν ἄρκτων πρὸς μεσημβρίαν
τρισμυρίων δισχιλίων. Τηλικαύτη δὲ οὖσα τὸ μέγεθος
δοκεῖ μάλιστα τοῦ κόσμου περιέχειν τὸν τῶν θερι-
νῶν τροπῶν κύκλον, καὶ πολλαχῇ μὲν ἐπ' ἄκρας τῆς
1.15
Ἰνδικῆς ἰδεῖν ἔστιν ἀσκίους ὄντας τοὺς γνώμονας,
νυκτὸς δὲ τὰς ἄρκτους ἀθεωρήτους· ἐν δὲ τοῖς ἐσχά-
τοις οὐδ' αὐτὸν τὸν ἀρκτοῦρον φαίνεσθαι· καθ' ὃν
δὴ τρόπον φασὶ καὶ τὰς σκιὰς κεκλίσθαι πρὸς μεσημ-
βρίαν.
1.20
 3. Ἡ δ' οὖν Ἰνδικὴ πολλὰ μὲν ὄρη καὶ μεγάλα ἔχει
δένδρεσι παντοδαποῖς καρπίμοις πληθύοντα, πολλὰ δὲ
πεδία καὶ μεγάλα καρποφόρα, τῷ μὲν κάλλει διάφορα,
ποταμῶν δὲ πλήθεσι διαιρούμενα. Τὰ πολλὰ δὲ τῆς
χώρας ἀρδεύεται, καὶ διὰ τοῦτο διττοὺς ἔχει τοὺς κατ'
1.25
ἔτος καρπούς· ζῴων τε παντοδαπῶν γέμει διαφόρων
τοῖς μεγέθεσι καὶ ταῖς ἀλκαῖς, τῶν μὲν χερσαίων, τῶν
δὲ καὶ πτηνῶν. (4) Καὶ πλείστους δὲ καὶ μεγίστους
ἐλέφαντας ἐκτρέφει, χορηγοῦσα τὰς τροφὰς ἀφθόνως,
δι' ἃς ταῖς ῥώμαις τὰ θηρία ταῦτα πολὺ προέχει τῶν
1.30
κατὰ τὴν Λιβύην γεννωμένων· διὸ καὶ πολλῶν θηρευο-
μένων ὑπὸ τῶν Ἰνδῶν καὶ πρὸς τοὺς πολεμικοὺς ἀγῶνας
κατασκευαζομένων μεγάλας συμβαίνει γίνεσθαι ῥοπὰς
πρὸς τὴν νίκην.
 XXXVI. 5. Ὁμοίως δὲ καὶ τοὺς ἀνθρώπους ἡ πο-
1.35
λυκαρπία τρέφουσα τοῖς τε ἀναστήμασι τῶν σωμά-
των καὶ τοῖς ὄγκοις ὑπερφέροντας κατασκευάζει·
εἶναι δὲ αὐτοὺς συμβαίνει καὶ πρὸς τὰς τέχνας ἐπιστή-
μονας, ὡς ἂν ἀέρα μὲν ἕλκοντας καθαρὸν, ὕδωρ δὲ λε-
πτομερέστατον πίνοντας. (6) Ἡ δὲ γῇ πάμφορος οὖσα
1.40
τοῖς ἡμέροις καρποῖς ἔχει καὶ φλέβας καταγείους πολλῶν
καὶ παντοδαπῶν μετάλλων· γίνεται γὰρ ἐν αὐτῇ πολὺς
μὲν ἄργυρος καὶ χρυσὸς, οὐκ ὀλίγος δὲ χαλκὸς καὶ σί-
δηρος, ἔτι δὲ καττίτερος καὶ τἄλλα τὰ πρὸς κόσμον τε
καὶ χρείαν καὶ πολεμικὴν παρασκευὴν ἀνήκοντα.
1.45
(7) Χωρὶς δὲ τῶν δημητριακῶν καρπῶν φύεται κατὰ
τὴν Ἰνδικὴν πολλὴ μὲν κέγχρος, ἀρδευομένη τῇ τῶν
ποταμίων ναμάτων δαψιλείᾳ, πολὺ δὲ ὄσπριον καὶ διά-
φορον, ἔτι δὲ ὄρυζα καὶ τὸ προσαγορευόμενον βόσπορον,
καὶ μετὰ ταῦτ' ἄλλα πολλὰ τῶν πρὸς διατροφὴν χρη-
1.50
σίμων· καὶ τούτων τὰ πολλὰ ὑπάρχει αὐτοφυῆ· οὐκ
ὀλίγους δὲ καὶ ἄλλους ἐδωδίμους καρποὺς φέρει δυνα-
μένους τρέφειν ζῷα, περὶ ὧν μακρὸν ἂν εἴη γράφειν.
(8) Διὸ καί φασι μηδέποτε τὴν Ἰνδικὴν ἐπισχεῖν λιμὸν
ἢ καθόλου σπάνιν τῶν πρὸς τροφὴν ἥμερον ἀνηκόντων.
1.55
Διττῶν γὰρ ὄμβρων ἐν αὐτῇ γινομένων καθ' ἕκαστον
ἔτος, τοῦ μὲν χειμερινοῦ, καθὰ παρὰ τοῖς ἄλλοις ὁ σπό-
ρος τῶν πυρίνων γίνεται καρπῶν, τοῦ δ' ἑτέρου κατὰ
τὴν θερινὴν τροπὴν, καθ' ἣν σπείρεσθαι συμβαίνει τὴν
ὄρυζαν καὶ τὸ βόσπορον, ἔτι δὲ σήσαμον καὶ κέγχρον,
1.60
κατὰ [δὲ] τὸ πλεῖστον ἀμφοτέροις τοῖς καρποῖς οἱ κατὰ  
τὴν Ἰνδικὴν ἐπιτυγχάνουσι· πάντων δὲ (μὴ) τελεσφο-
ρουμένων, θατέρου τῶν καρπῶν οὐκ ἀποτυγχάνουσιν.
(9) Οἵ τε αὐτοματίζοντες καρποὶ καὶ αἱ κατὰ τοὺς ἑλώ-
δεις τόπους φυόμεναι ῥίζαι διάφοροι ταῖς γλυκύτησιν
1.65
οὖσαι πολλὴν παρέχονται τοῖς ἀνθρώποις δαψίλειαν·
πάντα γὰρ σχεδὸν τὰ κατὰ τὴν χώραν πεδία γλυκεῖαν
ἔχει τὴν ἀπὸ τῶν ποταμῶν ἰκμάδα καὶ τὴν ἀπὸ τῶν
ἐν τῷ θέρει [γινομένων] κατ' ἐνιαυτὸν κυκλικῇ τινι πε-
ριόδῳ παραδόξως εἰωθότων γίνεσθαι δαψίλειαν, χλιαρῶν
1.70
πιπτόντων ὑδάτων ἐκ τοῦ περιέχοντος ἀέρος, καὶ τὰς
ἐν τοῖς ἕλεσι ῥίζας ἕψοντος τοῦ καύματος, καὶ μάλιστα
τῶν μεγάλων καλάμων. (10) Συμβάλλονται δὲ παρὰ
τοῖς Ἰνδοῖς καὶ τὰ νόμιμα πρὸς τὸ μηδέποτε ἔνδειαν
τροφῆς παρ' αὐτοῖς εἶναι· παρὰ μὲν γὰρ τοῖς ἄλλοις
1.75
ἀνθρώποις οἱ πολέμιοι καταφθείροντες τὴν χώραν
ἀγεώργητον κατασκευάζουσι· παρὰ δὲ τούτοις τῶν γεωρ-
γῶν ἱερῶν καὶ ἀσύλων ἐωμένων, οἱ πλησίον τῶν παρα-
τάξεων γεωργοῦντες ἀνεπαίσθητοι τῶν κινδύνων εἰσίν.
Ἀμφότεροι γὰρ οἱ πολεμοῦντες ἀλλήλους μὲν ἀποκτεί-
1.80
νουσιν ἐν ταῖς μάχαις, τοὺς δὲ περὶ τὴν γεωργίαν ὄντας
ἐῶσιν ἀβλαβεῖς, ὡς κοινοὺς ὄντας ἁπάντων εὐεργέτας,
τάς τε χώρας τῶν ἀντιπολεμούντων οὔτ' ἐμπυρίζουσιν
οὔτε δενδροτομοῦσιν.
 XXXVII. 11. Ἔχει δὲ καὶ ποταμοὺς ἡ χώρα τῶν
1.85
Ἰνδῶν πολλοὺς καὶ μεγάλους πλωτοὺς, οἳ τὰς πηγὰς
ἔχοντες ἐν τοῖς ὄρεσι τοῖς πρὸς τὰς ἄρκτους κεκλιμένοις
φέρονται διὰ τῆς πεδιάδος, ὧν οὐκ ὀλίγοι συμμίσγοντες
ἀλλήλοις ἐμβάλλουσιν εἰς ποταμὸν τὸν ὀνομαζόμενον
Γάγγην. (12) Οὗτος δὲ τὸ πλάτος γενόμενος σταδίων 

1.90
τριάκοντα φέρεται μὲν ἀπὸ τῆς ἄρκτου πρὸς μεσημ-
βρίαν, ἐξερεύγεται δὲ εἰς τὸν Ὠκεανόν, ἀπολαμβάνων
εἰς τὸ πρὸς ἕω μέρος τὸ ἔθνος τὸ τῶν Γανδαριδῶν (Γαγ-
γαριδῶν?), πλείστους ἔχον καὶ μεγίστους ἐλέφαντας.
(13) Διὸ καὶ τῆς χώρας ταύτης οὐδεὶς πώποτε βασιλεὺς
1.95
ἔπηλυς ἐκράτησε, πάντων τῶν ἀλλοεθνῶν φοβουμένων
τὸ πλῆθος καὶ τὴν ἀλκὴν τῶν θηρίων. Καὶ γὰρ Ἀλέ-
ξανδρος ὁ Μακεδὼν ἁπάσης τῆς Ἀσίας κρατήσας μό-
νους τοὺς Γανδαρίδας οὐκ ἐπολέμησε· καταντήσας γὰρ
ἐπὶ τὸν Γάγγην ποταμὸν μετὰ πάσης τῆς δυνάμεως,
1.100
καὶ τοὺς ἄλλους Ἰνδοὺς καταπολεμήσας, ὡς ἐπύθετο
τοὺς Γανδαρίδας ἔχειν τετρακισχιλίους ἐλέφαντας πο-
λεμικῶς κεκοσμημένου, ἀπέγνω τὴν ἐπ' αὐτοὺς στρα-
τείαν. (14) Ὁ δὲ παραπλήσιος τῷ Γάγγῃ ποταμὸς,
προσαγορευόμενος δὲ Ἰνδὸς, ἄρχεται μὲν ὁμοίως ἀπὸ
1.105
τῶν ἄρκτων, ἐμβάλλων δὲ εἰς τὸν Ὠκεανὸν ἀφορίζει
τὴν Ἰνδικήν· πολλὴν δὲ διεξιὼν πεδιάδα χώραν δέχε-
ται ποταμοὺς οὐκ ὀλίγους πλωτοὺς, ἐπιφανεστάτους δὲ
Ὕπανιν καὶ Ὑδάσπην καὶ Ἀκεσῖνον. (15) Χωρὶς δὲ
τούτων ἄλλο πλῆθος ποταμῶν παντοδαπῶν διαρρεῖ,
1.110
καὶ ποιεῖ κατάφυτον πολλοῖς κηπεύμασι καὶ καρποῖς
παντοδαποῖς τὴν χώραν. (16) Τοῦ δὲ κατὰ τοὺς ποτα-
μοὺς πλήθους καὶ τῆς τῶν ὑδάτων ὑπερβολῆς αἰτίαν
φέρουσιν οἱ παρ' αὐτοῖς φιλόσοφοι καὶ φυσικοὶ τοιαύτην·
τῆς Ἰνδικῆς φασι τὰς περικειμένας χώρας, τήν τε
1.115
Σκυθῶν καὶ Βακτριανῶν, ἔτι δὲ καὶ τῶν Ἀριανῶν,
ὑψηλοτέρας εἶναι τῆς Ἰνδικῆς· ὥστε εὐλόγως εἰς τὴν
ὑποκειμένην χώραν πανταχόθεν συρρεούσας τὰς λιβά-
δας ἐκ τοῦ κατ' ὀλίγον ποιεῖν τοὺς τόπους καθύγρους
καὶ γεννᾶν ποταμῶν πλῆθος. (17) Ἴδιον δέ τι συμ-
1.120
βαίνει περί τινα τῶν κατὰ τὴν Ἰνδικὴν ποταμῶν τὸν  
ὀνομαζόμενον Σίλλαν, ῥέοντα δὲ ἔκ τινος ὁμωνύμου
κρήνης· ἐπὶ γὰρ τούτου μόνου τῶν ἁπάντων ποταμῶν
οὐδὲν τῶν ἐμβαλλομένων εἰς αὐτὸν ἐπιπλεῖ, πάντα
δ' εἰς τὸν βυθὸν καταδύεται παραδόξως.
1.125
 XXXVIII. 18. Τὴν δ' ὅλην Ἰνδικὴν οὖσαν ὑπερμε-
γέθη λέγεται κατοικεῖν ἔθνη πολλὰ καὶ παντοδαπὰ,
καὶ τούτων μηδὲν ἔχειν τὴν ἐξ ἀρχῆς γένεσιν ἔπηλυν,
ἀλλὰ πάντα δοκεῖν ὑπάρχειν αὐτόχθονα, πρὸς δὲ τούτοις
μήτε ξενικὴν ἀποικίαν προσδέχεσθαι πώποτε μήτε εἰς
1.130
ἄλλο ἔθνος ἀπεσταλκέναι. (19) Μυθολογοῦσι δὲ τοὺς
ἀρχαιοτάτους ἀνθρώπους τροφαῖς μὲν κεχρῆσθαι τοῖς
αὐτομάτως φυομένοις ἐκ τῆς γῆς καρποῖς, ἐσθῆσι δὲ
ταῖς δοραῖς τῶν ἐγχωρίων ζῴων, καθάπερ καὶ παρ'
Ἕλλησιν· ὁμοίως δὲ καὶ τῶν τεχνῶν τὰς εὑρέσεις καὶ
1.135
τῶν ἄλλων τῶν πρὸς βίον χρησίμων ἐκ τοῦ κατ' ὀλίγον
γενέσθαι, τῆς χρείας αὐτῆς ὑφηγουμένης εὐφυεῖ ζῴῳ
καὶ συνεργοὺς ἔχοντι πρὸς ἅπαντα χεῖρας καὶ λόγον καὶ
ψυχῆς ἀγχίνοιαν.
 20. Μυθολογοῦσι δὲ παρὰ τοῖς Ἰνδοῖς οἱ λογιώτατοι
1.140
περὶ ὧν καθῆκον ἂν εἴη συντόμως· διελθεῖν. Φασὶ γὰρ
ἐν τοῖς ἀρχαιοτάτοις χρόνοις, παρ' αὐτοῖς ἔτι τῶν
ἀνθρώπων κωμηδὸν οἰκούντων, παραγενέσθαι τὸν
Διόνυσον ἐκ τῶν πρὸς ἑσπέρας τόπων ἔχοντα δύναμιν
ἀξιόλογον· ἐπελθεῖν δὲ τὴν Ἰνδικὴν ἅπασαν, μηδεμιᾶς
1.145
οὔσης ἀξιολόγου πόλεως τῆς δυναμένης ἀντιτάξασθαι.
(21) Ἐπιγενομένων δὲ καυμάτων μεγάλων, καὶ τῶν τοῦ
Διονύσου στρατιωτῶν λοιμικῇ νόσῳ διαφθειρομένων,
συνέσει διαφέροντα τὸν ἡγεμόνα τοῦτον ἀπαγαγεῖν τὸ
στρατόπεδον ἐκ τῶν πεδινῶν τόπων εἰς τὴν ὀρεινήν·
1.150
ἐνταῦθα δὲ πνεόντων ψυχρῶν ἀνέμων καὶ τῶν ναμα-
τιαίων ὑδάτων καθαρῶν ῥεόντων πρὸς αὐταῖς ταῖς πη-
γαῖς, ἀπαλλαγῆναι τῆς νόσου τὸ στρατόπεδον· ὀνομά-
ζεσθαι δὲ τῆς ὀρεινῆς τὸν τόπον τοῦτον Μηρὸν, καθ' ὃν
ὁ Διόνυσος ἐξέτρεψε τὰς δυνάμεις ἐκ τῆς νόσου· ἀφ' οὗ
1.155
δὴ καὶ τοὺς Ἕλληνας περὶ τοῦ θεοῦ τούτου παραδεδω-
κέναι τοῖς μεταγενεστέροις τετράφθαι τὸν Διόνυσον ἐν
μηρῷ. (22) Μετὰ δὲ ταῦτα τῆς παραθέσεως τῶν καρ-
πῶν ἐπιμεληθέντα μεταδιδόναι τοῖς Ἰνδοῖς, καὶ τὴν
εὕρεσιν τοῦ οἴνου καὶ τῶν ἄλλων τῶν εἰς τὸν βίον χρη-
1.160
σίμων παραδοῦναι. Πρὸς δὲ τούτοις πόλεών τε ἀξιολό-
γων γενηθῆναι κτίστην, μεταγαγόντα τὰς κώμας εἰς
τοὺς εὐθέτους τόπους, τιμᾶν τε καταδεῖξαι τὸ θεῖον καὶ
νόμους εἰσηγήσασθαι καὶ δικαστήρια, καθόλου δὲ πολ-
λῶν καὶ καλῶν ἔργων εἰσηγητὴν γενόμενον θεὸν νομι-
1.165
σθῆναι καὶ τυχεῖν ἀθανάτων τιμῶν. (23) Ἱστοροῦσι
δ' αὐτὸν καὶ γυναικῶν πλῆθος μετὰ τοῦ στρατοπέδου
περιάγεσθαι, καὶ κατὰ τὰς ἐν τοῖς πολέμοις παρατάξεις
τυμπάνοις καὶ κυμβάλοις κεχρῆσθαι, μήπω σάλπιγγος
εὑρημένης. Βασιλεύσαντα δὲ πάσης τῆς Ἰνδικῆς ἔτη
1.170
δύο πρὸς τοῖς πεντήκοντα γήρᾳ τελευτῆσαι· διαδεξαμέ-
νους δὲ τοὺς υἱοὺς αὐτοῦ τὴν ἡγεμονίαν ἀεὶ τοῖς ἀφ'
ἑαυτῶν ἀπολιπεῖν τὴν ἀρχήν· τὸ δὲ τελευταῖον πολ-
λαῖς γενεαῖς ὕστερον καταλυθείσης τῆς ἡγεμονίας δη-
μοκρατηθῆναι τὰς πόλεις.
1.175
 XXXIX. 24. Περὶ μὲν οὖν τοῦ Διονύσου καὶ τῶν
ἀπογόνων αὐτοῦ τοιαῦτα μυθολογοῦσιν οἱ τὴν ὀρεινὴν
τῆς Ἰνδικῆς κατοικοῦντες. Τόν τε Ἡρακλέα φασὶ
παρ' αὑτοῖς γεγενῆσθαι, καὶ παραπλησίως τοῖς Ἕλλησι
τό τε ῥόπαλον καὶ τὴν λεοντῆν αὐτῷ προσάπτουσι· τῇ 

1.180
δὲ τοῦ σώματος ῥώμῃ καὶ ἀλκῇ πολλῷ τῶν ἄλλων ἀν-
θρώπων διενεγκεῖν, καὶ καθαρὰν ποιῆσαι τῶν θηρίων
γῆν τε καὶ θάλατταν. (25) Γαμήσαντα δὲ πλείους γυ-  
ναῖκας, υἱοὺς μὲν πολλοὺς, θυγατέρα δὲ μίαν γεννῆ-
σαι· καὶ τούτων ἐνηλίκων γενομένων, πᾶσαν τὴν
1.185
Ἰνδικὴν διελόμενον εἰς ἴσας τοῖς τέκνοις μερίδας ἅπαν-
τας τοὺς υἱοὺς ἀποδεῖξαι βασιλέας, μίαν δὲ θυγατέρα
θρέψαντα καὶ ταύτην βασίλισσαν ἀποδεῖξαι. (26) Κτί-
στην τε πόλεων οὐκ ὀλίγων γενέσθαι, καὶ τούτων τὴν
ἐπιφανεστάτην καὶ μεγίστην προσαγορεῦσαι Παλί-
1.190
βοθρα· κατασκευάσαι δ' ἐν αὐτῇ καὶ βασίλεια πολυ-
τελῆ καὶ πλῆθος οἰκητόρων καθιδρῦσαι· τήν τε πόλιν
ὀχυρῶσαι τάφροις ἀξιολόγοις ποταμίοις ὕδασι πληρου-
μέναις. (27) Καὶ τὸν μὲν Ἡρακλέα τὴν ἐξ ἀνθρώπων
μετάστασιν ποιησάμενον ἀθανάτου τυχεῖν τιμῆς· τοὺς
1.195
δ' ἀπογόνους αὐτοῦ βασιλεύσαντας ἐπὶ πολλὰς γενεὰς
καὶ πράξεις ἀξιολόγους μεταχειρισαμένους μήτε στρα-
τείαν ὑπερόριον ποιήσασθαι, μήτε ἀποικίαν εἰς ἄλλο
ἔθνος ἀποστεῖλαι. Ὕστερον δὲ πολλοῖς ἔτεσι τὰς πλεί-
στας μὲν τῶν πόλεων δημοκρατηθῆναι, τινῶν δὲ ἐθνῶν
1.200
τὰς βασιλείας διαμεῖναι μέχρι τῆς Ἀλεξάνδρου διαβά-
σεως. (28) Νομίμων δ' ὄντων παρὰ τοῖς Ἰνδοῖς ἐνίων
ἐξηλλαγμένων θαυμασιώτατον ἄν τις ἡγήσαιτο τὸ κα-
ταδειχθὲν ὑπὸ τῶν ἀρχαίων παρ' αὐτοῖς φιλοσόφων·
νενομοθέτηται γὰρ παρ' αὐτοῖς δοῦλον μηδένα τὸ πα-
1.205
ράπαν εἶναι, ἐλευθέρους δ' ὑπάρχοντας τὴν ἰσότητα
τιμᾶν ἐν πᾶσι. Τοὺς γὰρ μαθόντας μήθ' ὑπερέχειν
μήθ' ὑποπίπτειν ἄλλοις κράτιστον ἕξειν βίον πρὸς
ἁπάσας τὰς περιστάσεις· εὔηθες γὰρ εἶναι νόμους μὲν
ἐπ' ἴσης τιθέναι πᾶσι, τὰς δ' ἐξουσίας ἀνωμάλους κα-
1.210
τασκευάζειν.
 XL. 29. Τὸ δὲ πᾶν πλῆθος τῶν Ἰνδῶν εἰς ἑπτὰ μέρη
διῄρηται, ὧν ἐστι τὸ μὲν πρῶτον σύστημα φιλοσόφων,
πλήθει μὲν τῶν ἄλλων μερῶν λειπόμενον, τῇ δ' ἐπι-
φανείᾳ πάντων πρωτεῦον· ἀλειτούργητοι γὰρ ὄντες οἱ
1.215
φιλόσοφοι πάσης ὑπουργίας οὔθ' ἑτέρων κυριεύουσιν
οὔθ' ὑφ' ἑτέρων δεσπόζονται. (30) Παραλαμβάνονται δὲ
ὑπὸ μὲν τῶν ἰδιωτῶν εἴς τε τὰς ἐν τῷ βίῳ θυσίας καὶ
εἰς τὰς τῶν τετελευτηκότων ἐπιμελείας, ὡς θεοῖς γεγο-
νότες προσφιλέστατοι καὶ περὶ τῶν ἐν ᾍδου μάλιστα
1.220
ἐμπείρως ἔχοντες, ταύτης τε τῆς ὑπουργίας δῶρά
τε καὶ τιμὰς λαμβάνουσιν ἀξιολόγους· τῷ δὲ κοινῷ τῶν
Ἰνδῶν μεγάλας παρέχονται χρείας παραλαμβανόμενοι
μὲν κατὰ τὸ νέον ἔτος ἐπὶ τὴν μεγάλην σύνοδον, προ-
λέγοντες δὲ τοῖς πλήθεσι περὶ αὐχμῶν καὶ ἐπομβρίας,
1.225
ἔτι δὲ ἀνέμων εὐπνοίας καὶ νόσων καὶ τῶν ἄλλων τῶν
δυναμένων τοὺς ἀκούοντας ὠφελῆσαι. (31) Τὰ μέλλοντα
γὰρ προακούσαντες οἵ τε πολλοὶ καὶ ὁ βασιλεὺς ἐκπλη-
ροῦσιν ἀεὶ τὸ μέλλον ἐκλείπειν καὶ προκατασκευάζουσιν
ἀεί τι τῶν χρησίμων· ὁ δὲ ἀποτυχὼν τῶν φιλοσόφων
1.230
ἐν ταῖς προρρήσεσιν ἄλλην μὲν οὐδεμίαν ἀναδέχεται
τιμωρίαν ἢ βλασφημίαν, ἄφωνος δὲ διατελεῖ τὸν λοι-
πὸν βίον.
 32. Δεύτερον δ' ἐστὶ μέρος τὸ τῶν γεωργῶν, οἳ τῷ
πλήθει τῶν ἄλλων πολὺ προέχειν δοκοῦσιν· οὗτοι δὲ
1.235
πολέμων καὶ τῆς ἄλλης λειτουργίας ἀφειμένοι περὶ τὰς
γεωργίας ἀσχολοῦνται· καὶ οὐδεὶς ἂν πολέμιος περιτυ-
χὼν γεωργῷ κατὰ τὴν χώραν ἀδικήσειεν, ἀλλ' ὡς κοι-
νοὺς εὐεργέτας ἡγούμενοι πάσης ἀδικίας ἀπέχονται.
Διόπερ ἀδιάφθορος ἡ χώρα διαμένουσα καὶ καρποῖς
1.240
βρίθουσα πολλὴν ἀπόλαυσιν παρέχεται τῶν ἐπιτηδείων  
τοῖς ἀνθρώποις. (33) Βιοῦσι δ' ἐπὶ τῆς χώρας μετὰ
τέκνων καὶ γυναικῶν γεωργοὶ, καὶ τῆς εἰς τὴν πόλιν
καταβάσεως παντελῶς ἀφεστήκασι. Τῆς δὲ χώρας
μισθοὺς τελοῦσι τῷ βασιλεῖ διὰ τὸ πᾶσαν τὴν Ἰνδικὴν
1.245
βασιλικὴν εἶναι, ἰδιώτῃ δὲ μηδενὶ γῆν ἐξεῖναι κεκτῆσθαι·
χωρὶς δὲ τῆς μισθώσεως τετάρτην εἰς τὸ βασιλικὸν
τελοῦσι.
 34. Τρίτον δ' ἐστὶ φῦλον τὸ τῶν βουκόλων καὶ ποι-
μένων καὶ καθόλου πάντων τῶν νομέων, οἳ πόλιν μὲν
1.250
ἢ κώμην οὐκ οἰκοῦσι, σκηνίτῃ δὲ βίῳ χρῶνται· οἱ
δ' αὐτοὶ καὶ κυνηγοῦντες καθαρὰν ποιοῦσι τὴν χώραν
ὀρνέων καὶ θηρίων· εἰς ταῦτα δὲ ἀσκοῦντες καὶ φιλο-
πονοῦντες ἐξημεροῦσι τὴν Ἰνδικὴν, πλήθουσαν πολλῶν
καὶ παντοδαπῶν θηρίων τε καὶ ὀρνέων τῶν κατεσθιόν-
1.255
των τὰ σπέρματα τῶν γεωργῶν.
 XLI. 35. Τέταρτον δ' ἐστὶ μέρος τὸ τῶν τεχνιτῶν· καὶ
τούτων οἱ μέν εἰσιν ὁπλοποιοὶ, οἱ δὲ τοῖς γεωργοῖς
ἤ τισιν ἄλλοις τὰ χρήσιμα πρὸς ὑπηρεσίαν κατασκευά-
ζουσιν· οὗτοι δὲ οὐ μόνον ἀτελεῖς εἰσιν, ἀλλὰ καὶ
1.260
σιτομετρίαν ἐκ τοῦ βασιλικοῦ λαμβάνουσι.
 36. Πέμπτον δὲ στρατιωτικὸν, εἰς τοὺς πολέμους
εὐθετοῦν, τῷ μὲν πλήθει δεύτερον, ἀνέσει δὲ καὶ παι-
διᾷ πλείστῃ χρώμενον ἐν ταῖς εἰρήναις. Τρέφεται δ' ἐκ
τοῦ βασιλικοῦ πᾶν τὸ πλῆθος τῶν στρατιωτῶν καὶ τῶν
1.265
πολεμιστῶν ἵππων τε καὶ ἐλεφάντων.
 37. Ἕκτον δ' ἐστὶ τὸ τῶν ἐφόρων· οὗτοι δὲ πολυ-
πραγμονοῦντες πάντα καὶ ἐφορῶντες τὰ κατὰ τὴν Ἰν-
δικὴν ἀπαγγέλλουσι τοῖς βασιλεῦσιν, ἐὰν δὲ ἡ πόλις
αὐτῶν ἀβασίλευτος ᾖ, τοῖς ἄρχουσιν. 

1.270
 38. Ἕβδομον δ' ἐστὶ μέρος τὸ βουλεῦον μὲν καὶ συν-
εδρεῦον τοῖς ὑπὲρ τῶν κοινῶν βουλευομένοις, πλήθει
μὲν ἐλάχιστον, εὐγενείᾳ δὲ καὶ φρονήσει μάλιστα θαυ-
μαζόμενον· ἐκ τούτων γὰρ οἵ τε σύμβουλοι τοῖς βασι-
λεῦσίν εἰσιν οἵ τε διοικηταὶ τῶν κοινῶν καὶ οἱ δικασταὶ
1.275
τῶν ἀμφισβητουμένων, καὶ καθόλου τοὺς ἡγεμόνας καὶ
τοὺς ἄρχοντας ἐκ τούτων ἔχουσι.
 39. Τὰ μὲν οὖν μέρη τῆς διῃρημένης πολιτείας
παρ' Ἰνδοῖς σχεδὸν ταῦτά ἐστιν. Οὐκ ἔξεστι δὲ γα-
μεῖν ἐξ ἄλλου γένους ἢ προαιρέσεις ἢ τέχνας μεταχει-
1.280
ρίζεσθαι, οἷον στρατιώτην ὄντα γεωργεῖν ἢ τεχνίτην
ὄντα φιλοσοφεῖν.
 XLII. 40. Ἔχει δ' ἡ τῶν Ἰνδῶν χώρα πλείστους
καὶ μεγίστους ἐλέφαντας, ἀλκῇ τε καὶ μεγέθει πολὺ
διαφέροντας. Ὀχεύεται δὲ τοῦτο τὸ ζῷον οὐχ, ὥσπερ
1.285
τινές φασιν, ἐξηλλαγμένως, ἀλλ' ὁμοίως ἵπποις καὶ
τοῖς ἄλλοις τετραπόδοις ζῴοις· κύουσι δὲ τοὺς μὲν ἐλα-
χίστους μῆνας ἑκκαίδεκα, τοὺς δὲ πλείστους ὀκτωκαί-
δεκα. Τίκτουσι δὲ καθάπερ ἵπποι κατὰ τὸ πλεῖστον
ἓν, καὶ τρέφουσι τὸ γεννηθὲν αἱ μητέρες ἐπ' ἔτη ἕξ.
1.290
Ζῶσι δ' οἱ πλεῖστοι καθάπερ ὁ μακροβιώτατος ἄνθρω-
πος, οἱ δὲ μάλιστα γηράσαντες ἔτη διακόσια.
 41. Εἰσὶ δὲ παρ' Ἰνδοῖς καὶ ἐπὶ τοὺς ξένους ἄρχον-
τες τεταγμένοι καὶ φροντίζοντες ὅπως μηδεὶς ξένος
ἀδικῆται· τοῖς δ' ἀρρωστοῦσι τῶν ξένων ἰατροὺς εἰσά-
1.295
γουσι καὶ τὴν ἄλλην ἐπιμέλειαν ποιοῦνται, καὶ τελευ-
τήσαντας θάπτουσιν, ἔτι δὲ τὰ καταλειφθέντα χρήματα
τοῖς προσήκουσιν ἀποδιδόασιν. (42) Οἵ τε δικασταὶ
τὰς κρίσεις παρ' αὐτοῖς ἀκριβῶς διαγινώσκουσι, καὶ
πικρῶς τοῖς ἁμαρτάνουσι προσφέρονται. Περὶ μὲν οὖν
1.300
τῆς Ἰνδικῆς καὶ τῶν κατ' αὐτὴν ἀρχαιολογουμένων
ἀρκεσθησόμεθα τοῖς ῥηθεῖσι.  
t2-20.1

E LIBRO PRIMO.

2.1
 Arrianus Exp. Al. V, 6, 2: Τῆς δὲ ὡς ἐπὶ νότον
Ἀσίας τετραχῆ αὖ τεμνομένης μεγίστην μὲν μοῖραν
τῶν Ἰνδῶν γῆν ποιεῖ Ἐρατοσθένης τε καὶ Μεγασθένης,
ὃς ξυνῆν μὲν Σιβυρτίῳ τῷ σατράπῃ τῆς Ἀραχωσίας·
2.5
πολλάκις δὲ λέγει ἀφικέσθαι παρὰ Σανδράκοττον τὸν
Ἰνδῶν βασιλέα· ἐλαχίστην δὲ ὅσην ὁ Εὐφράτης ποτα-
μὸς ἀπείργει ὡς πρὸς τὴν ἡμετέραν θάλασσαν· δύο δὲ
αἱ μεταξὺ Εὐφράτου τε ποταμοῦ καὶ τοῦ Ἰνδοῦ ἀπειρ-
γόμεναι, αἱ δύο ξυντεθεῖσαι μόλις ἄξιαι τῇ Ἰνδῶν γῇ
2.10
ξυμβαλεῖν· ἀπείργεσθαι δὲ τὴν Ἰνδῶν χώραν πρὸς μὲν
ἕω τε καὶ ἀπηλιώτην ἄνεμον ἔστε ἐπὶ μεσημβρίαν
τῇ μεγάλῃ θαλάσσῃ· τὸ πρὸς βορρᾶν δὲ αὐτῆς ἀπείρ-
γειν τὸν Καύκασον τὸ ὄρος ἔστε ἐπὶ τοῦ Ταύρου τὴν
ξυμβολήν· τὴν δὲ πρὸς ἑσπέραν τε καὶ ἄνεμον Ἰάπυγα
2.15
ἔστε ἐπὶ τὴν μεγάλην θάλασσαν, ὁ Ἰνδὸς ποταμὸς ἀπο-
τέμνεται. Καί ἐστι πεδίον ἡ πολλὴ αὐτῆς· καὶ τοῦτο,
ὡς εἰκάζουσιν, ἐκ τῶν ποταμῶν προσκεχωσμένον· εἶναι
γὰρ οὖν καὶ τῆς ἄλλης χώρας ὅσα πεδία οὐ πρόσω
θαλάσσης τὰ πολλὰ τῶν ποταμῶν παρ' ἑκάστοις ποιή-
2.20
ματα, ὥστε καὶ τῆς χώρας τὴν ἐπωνυμίαν τοῖς ποτα-
μοῖς ἐκ παλαιοῦ προσκεῖσθαι, καθάπερ Ἕρμου τέ τι
πεδίον λέγεσθαι, ὃς κατὰ τὴν Ἀσίαν γῆν ἀνίσχων ἐξ
ὄρους Μητρὸς Δινδυμήνης παρὰ πόλιν Σμύρναν Αἰο-
λικὴν ἐκδιδοῖ ἐς θάλασσαν· καὶ ἄλλο Καΰστρου πεδίον
2.25
Λυδίου ποταμοῦ, καὶ Καΐκου ἄλλο ἐν Μυσίᾳ, καὶ
Μαιάνδρου τὸ Καρικὸν ἔστε ἐπὶ Μίλητον πόλιν Ἰωνι-
κήν .... Εἰ δὴ οὖν εἷς τε ποταμὸς παρ' ἑκάστοις, καὶ
οὐ μεγάλοι οὗτοι ποταμοὶ, ἱκανοὶ γῆν πολλὴν ποιῆσαι,
ἐς θάλασσαν προχεόμενοι, ὁπότε ἰλὺν καταφέροιεν καὶ
2.30
πηλὸν ἐκ τῶν ἄνω τόπων, ἔνθενπερ αὐτοῖς αἱ πηγαί
εἰσιν, οὐδὲ ὑπὲρ τῆς Ἰνδῶν ἄρα χώρας ἐς ἀπιστίαν ἰέναι
ἄξιον, ὅπως πεδίον τε ἡ πολλή ἐστι, καὶ ἐκ τῶν ποτα-
μῶν τὸ πεδίον ἔχει προσκεχωσμένον. Ἕρμον μὲν γὰρ
καὶ Κάϋστρον καὶ Κάϊκόν τε καὶ Μαίανδρον ἢ ὅσοι
2.35
πολλοὶ ποταμοὶ τῆς Ἀσίας ἐς τήνδε τὴν ἐντὸς θάλασσαν
ἐκδιδοῦσιν, οὐδὲ ξύμπαντας ξυντεθέντας ἑνὶ τῶν Ἰνδῶν
ποταμῶν ἄξιον ξυμβαλεῖν πλήθους ἕνεκα τοῦ ὕδατος,
μὴ ὅτι τῷ Γάγγῃ τῷ μεγίστῳ, ὅτῳ οὔτε Νείλου ὕδωρ τοῦ
Αἰγυπτίου οὔτε ὁ Ἴστρος ὁ κατὰ τὴν Εὐρώπην ῥέων
2.40
ἄξιοι ξυμβαλεῖν· ἀλλ' οὐδὲ τῷ Ἰνδῷ ποταμῷ ἐκεῖνοί γε
πάντες ξυμμιχθέντες ἐς ἴσον ἔρχονται· ὃς μέγας τε
εὐθὺς ἀπὸ τῶν πηγῶν ἀνίσχει, καὶ πεντεκαίδεκα ποτα-
μοὺς πάντας τῶν Ἀσιανῶν μείζονας παραλαβὼν, καὶ
τῇ ἐπωνυμίᾳ κρατήσας, οὕτως ἐκδιδοῖ ἐς θάλασσαν.
2.45
Ταῦτά μοι ἐν τῷ παρόντι περὶ Ἰνδῶν τῆς χώρας λε-
λέχθω· τὰ δὲ ἄλλα ἀποκείσθω ἐς τὴν Ἰνδικὴν ξυγ-
γραφήν.  
3.1
 Strabo XV: Τὴν Ἰνδικὴν περιώρικεν ἀπὸ
μὲν τῶν ἄρκτων τοῦ Ταύρου τὰ ἔσχατα· ἀπὸ δὲ τῆς
Ἀριανῆς μέχρι τῆς ἑῴας θαλάττης, ἅπερ οἱ ἐπιχώριοι
κατὰ μέρος Παροπάμισόν τε καὶ Ἠμωδὸν καὶ Ἴμαον
3.5
καὶ ἄλλα ὀνομάζουσι, Μακεδόνες δὲ Καύκασον· ἀπὸ
τῆς ἑσπέρας ὁ Ἰνδὸς ποταμός· τὸ δὲ νότιον καὶ τὸ προς-
εῷον πλευρὸν πολὺ μείζω τῶν ἑτέρων ὄντα προπέπτω-
κεν εἰς τὸ Ἀτλαντικὸν πέλαγος, καὶ γίνεται ῥομβοειδὲς
τὸ τῆς χώρας σχῆμα, τῶν μειζόνων πλευρῶν ἑκατέρου
3.10
πλεονεκτοῦντος παρὰ τὸ ἀπεναντίον πλευρὸν τρισχιλίοις
σταδίοις, ὅσον ἐστὶ τὸ κοινὸν ἄκρον τῆς τε ἑωθινῆς 


παραλίας καὶ τῆς μεσημβρινῆς, ἔξω προπεπτωκὸς ἐξ
ἴσης ἐφ' ἑκάτερον παρὰ τὴν ἄλλην ἠϊόνα. Τῆς μὲν
οὖν ἑσπερίου πλευρᾶς ἀπὸ τῶν Καυκασίων ὀρῶν ἐπὶ
3.15
τὴν νότιον θάλατταν στάδιοι μάλιστα λέγονται μύριοι
τρισχίλιοι παρὰ τὸν Ἰνδὸν ποταμὸν μέχρι τῶν ἐκβο-
λῶν αὐτοῦ· ὥστ' ἀπεναντίον ἡ ἑωθινὴ προσλαβοῦσα
τοὺς τῆς ἄκρας τρισχιλίους ἔσται μυρίων καὶ ἑξα-
κισχιλίων σταδίων. Τοῦτο μὲν οὖν πλάτος τῆς χώ-
3.20
ρας τό τ' ἐλάχιστον, καὶ τὸ μέγιστον. Μῆκος δὲ τὸ
ἀπὸ τῆς ἑσπέρας ἐπὶ τὴν ἕω· τούτου δὲ τὸ μὲν μέχρι
Παλιβόθρων ἔχοι τις ἂν βεβαιοτέρως εἰπεῖν· καταμε-
μέτρηται γὰρ σχοινίοις, καὶ ἔστιν
ὁδὸς βασιλικὴ σταδίων μυρίων· τὰ
3.25
δ' ἐπέκεινα στοχασμῷ λαμβάνεται διὰ τῶν ἀνάπλων
τῶν ἐκ θαλάττης διὰ τοῦ Γάγγου ποταμοῦ μέχρι Πα-
λιβόθρων· εἴη δ' ἂν σταδίων ἑξακισχιλίων. Ἔσται δὲ
τὸ πᾶν, ᾗ βραχύτατον, μυρίων ἑξακισχιλίων, ὡς ἔκ τε
τῆς ἀναγραφῆς τῶν σταθμῶν τῆς πεπιστευμένης μάλιστα
3.30
λαβεῖν Ἐρατοσθένης φησί· καὶ ὁ <Μεγασθένης> οὕτω
συναποφαίνεται· Πατροκλῆς δὲ χιλίοις ἔλαττόν φησι ...
Ἐκ δὲ τούτων πάρεστιν ὁρᾶν, ὅσον διαφέρουσιν αἱ τῶν
ἄλλων ἀποφάσεις, Κτησίου μὲν οὐκ ἐλάττω τῆς
ἄλλης Ἀσίας τὴν Ἰνδικὴν λέγοντος, Ὀνησικρίτου δὲ τρί-
3.35
τον μέρος τῆς οἰκουμένης, Νεάρχου δὲ μηνῶν ὁδὸν τετ-
τάρων τὴν δι' αὐτοῦ τοῦ πεδίου, Μεγασθένους δὲ καὶ
Δηϊμάχου μετριασάντων μᾶλλον· ὑπὲρ γὰρ δισμυρίους
τιθέασι σταδίους, τὸ ἀπὸ τῆς νοτίου θαλάττης ἐπὶ τὸν
Καύκασον.
4.1
 Strabo II.: Φησὶν ὁ Ἵππαρχος ἐν τῷ  
δευτέρῳ Ὑπομνήματι αὐτὸν τὸν Ἐρατοσθένη διαβάλλειν
τὴν τοῦ Πατροκλέους πίστιν ἐκ τῆς πρὸς Μεγασθένη
διαφωνίας περὶ τοῦ μήκους τῆς Ἰνδικῆς τοῦ κατὰ τὸ
4.5
βόρειον πλευρὸν, τοῦ μὲν Μεγασθένους λέγοντος στα-
δίων μυρίων ἑξακισχιλίων, τοῦ δὲ Πατροκλέους χιλίοις
λείπειν φαμένου.
5.1
 Strabo II.:
δυοῖν ἀντιμαρτυρούντων
αὐτῷ Δηιμάχου τε καὶ Μεγασθένους, οἳ καθ' οὓς μὲν
τόπους δισμυρίων τὸ διάστημά φασι τὸ ἀπὸ τῆς κατὰ
5.5
μεσημβρίαν θαλάττης, καθ' οὓς δὲ καὶ τρισμυρίων.
Τούτους γε δὴ τοιαῦτα λέγειν καὶ τοὺς ἀρχαίους πίνα-
κας τούτοις ὁμολογεῖν.
6.1
 Arrian. Ind. 3, 7: Μεγασθένει δὲ τὸ ἀπὸ ἀνατο-
λέων ἐς ἑσπέρην πλάτος ἐστὶ τῆς Ἰνδῶν γῆς, ὅ τι περ
οἱ ἄλλοι μῆκος ποιέουσι· καὶ λέγει Μεγασθένης, μυρίων
καὶ ἑξακισχιλίων σταδίων εἶναι, ἵναπερ τὸ βραχύτατον
6.5
αὐτοῦ. Τὸ δὲ ἀπὸ ἄρκτου πρὸς μεσημβρίην, τοῦτο δὲ αὐτῷ
μῆκος γίνεται, καὶ ἐπέχει τριηκοσίους καὶ δισχιλίους,
ἵναπερ τὸ στεινότατον αὐτοῦ.
7a.1
 Strabo II:
Μεγασθένει φήσαντι
ἐν τοῖς νοτίοις μέρεσι τῆς Ἰνδικῆς τάς τε ἄρκτους ἀπο-
κρύπτεσθαι καὶ τὰς σκιὰς ἀντιπίπτειν.
7b.1
 Strabo II:. Δηιμάχου φήσαντος μηδαμοῦ τῆς
Ἰνδικῆς μήτ' ἀποκρύπτεσθαι τὰς ἄρκτους μήτ' ἀντι-
πίπτειν τὰς σκιὰς ὅπερ ὑπείληφεν ὁ Μεγασθένης.
8.1
 Plinius H. N. VII, 22, 6: Ab iis  
in interiore situ Mo-
nedes et Suari, quorum mons Maleus, in quo umbrae
ad septemtrionem cadunt hieme, aestate in austrum,
8.5
per senos menses. Septemtriones eo tractu semel in
anno apparere, nec nisi quindecim diebus, Baeton
auctor est: hoc idem pluribus locis Indiae
fieri, Megasthenes.
9.1
 Strabo XV: Μεγασθένης δὲ τὴν εὐδαι-
μονίαν τῆς Ἰνδικῆς ἐπισημαίνεται τῷ δίκαρπον εἶναι
καὶ δίφορον, καθάπερ καὶ Ἐρατοσθένης ἔφη, τὸν μὲν
εἰπὼν σπόρον χειμερινὸν, τὸν δὲ θερινὸν, καὶ ὄμβρον
9.5
ὁμοίως· οὐδὲν γὰρ ἔτος εὑρίσκεσθαί φησι πρὸς ἀμφοτέρους
καιροὺς ἄνομβρον· ὥστε εὐετηρίαν ἐκ τούτου συμβαίνειν,
ἀφόρου μηδέποτε τῆς γῆς οὔσης· τούς τε ξυλίνους καρ-
ποὺς γεννᾶσθαι πολλοὺς, καὶ τὰς ῥίζας τῶν φυτῶν, καὶ
μάλιστα τῶν μεγάλων καλάμων, γλυκείας καὶ φύσει
9.10
καὶ ἑψήσει, χλιαινομένου τοῦ ὕδατος τοῖς ἡλίοις, τοῦ
τ' ἐκπίπτοντος ἐκ Διὸς, καὶ τοῦ ποταμίου. Τρόπον δή
τινα λέγειν βούλεται, διότι ἡ παρὰ τοῖς ἄλλοις λεγο-
μένη πέψις καὶ καρπῶν καὶ χυμῶν παρ' ἐκείνοις ἕψη-
σίς ἐστι· καὶ κατεργάζεται τοσοῦτον εἰς εὐστομίαν,
9.15
ὅσον καὶ ἡ διὰ πυρός· διὸ καὶ τοὺς κλάδους φησὶν εὐ-
καμπεῖς εἶναι τῶν δένδρων, ἐξ ὧν οἱ τροχοί· ἐκ δὲ τῆς
αὐτῆς αἰτίας ἐνίοις καὶ ἐπανθεῖν ἔριον.
10.1
 Strabo XV: Καὶ τίγρεις δ' ἐν τοῖς Πρα-
σίοις φησὶν ὁ Μεγασθένης γίνεσθαι μεγίστους, σχεδὸν
δέ τι καὶ διπλασίους λεόντων· δυνατοὺς δὲ ὥστε τῶν
ἡμέρων τινὰ, ἀγόμενον ὑπὸ τεττάρων, τῷ ὀπισθίῳ σκέ-
10.5
λει δραξάμενον ἡμιόνου, βιάσασθαι καὶ ἑλκύσαι πρὸς
ἑαυτόν. Κερκοπιθήκους δὲ μείζους τῶν μεγίστων κυνῶν,
λευκοὺς πλὴν τοῦ προσώπου· τοῦτο δ' εἶναι μέλαν·
παρ' ἄλλοις δ' ἀνάπαλιν· τὰς δὲ κέρκους μείζους δυοῖν
πήχεων· ἡμερωτάτους δὲ καὶ οὐ κακοήθεις περὶ ἐπιθέ-

10.10
σεις καὶ κλοπάς. Λίθους δ' ὀρύττεσθαι λιβανόχρους,
γλυκυτέρους σύκων ἢ μέλιτος. Ἀλλαχοῦ δὲ διπήχεις
ὄφεις ὑμενοπτέρους ὥσπερ αἱ νυκτερίδες, καὶ τούτους
δὲ νύκτωρ πέτεσθαι, σταλαγμοὺς ἀφιέντας οὔρων, τοὺς
δὲ ἱδρώτων, διασήποντας τὸν χρῶτα τοῦ μὴ φυλαξα-
10.15
μένου· καὶ σκορπίους εἶναι πτηνοὺς, ὑπερβάλλοντας
μεγέθεσι· φύεσθαι δὲ καὶ ἔβενον· εἶναι δὲ
καὶ κύνας ἀλκίμους, οὐ πρότερον μεθιέντας τὸ δηχθὲν
πρὶν εἰς τοὺς ῥώθωνας ὕδωρ καταχυθῆναι· ἐνίοις δ' ὑπὸ
προθυμίας ἐν τῷ δήγματι διαστρέφεσθαι τοὺς ὀφθαλ-
10.20
μοὺς, τοῖς δὲ καὶ ἐκπίπτειν· κατασχεθῆναι δὲ καὶ
λέοντα ὑπὸ κυνὸς καὶ ταῦρον· τὸν δὲ ταῦρον καὶ ἀπο-
θανεῖν κρατούμενον τοῦ ῥύγχους πρότερον ἢ ἀφεθῆναι.
11.1
 Aelian. N A. XVII, 39: Ἐν τῇ Πραξιακῇ χώρᾳ,
Ἰνδῶν δὲ αὕτη ἐστίν, Μεγασθένης φησὶ πιθήκους εἶναι
τῶν μεγίστων κυνῶν οὐ μείους, ἔχειν δὲ καὶ οὐρὰς πη-  
χῶν πέντε· προσπεφυκέναι δὲ ἄρα αὐτοῖς καὶ προκόμια,
11.5
καὶ πώγωνας καθειμένους καὶ βαθεῖς· καὶ τὸ μὲν πρός-
ωπον εἶναι πᾶν λευκούς, τὸ σῶμα δὲ μέλανας ἰδεῖν,
ἡμέρους δὲ καὶ φιλανθρωποτάτους, καὶ τὸ τοῖς ἀλλα-
χόθι πιθήκοις συμφυὲς οὐκ ἔχειν τὸ κακόηθες.
12.1
 Aelian. N. A. XVI, 41: Μεγασθένης φησὶ κατὰ
τὴν Ἰνδικὴν σκορπίους γίνεσθαι πτερωτοὺς μεγέθει με-
γίστους, τὸ κέντρον δὲ ἐγχρίπτειν τοῖς Εὐρωπαίοις
παραπλησίως· γίνεσθαι δὲ καὶ ὄφεις αὐτόθι, καὶ τούτους
12.5
πτηνούς· ἐπιφοιτᾶν δὲ οὐ καθ' ἡμέραν, ἀλλὰ νύκτωρ,
καὶ ἀφιέναι ἐξ αὑτῶν οὖρον, ὅπερ οὖν ἐὰν κατά τινος
ἀποστάξῃ σώματος, σῆψιν ἐργάζεται παραχρῆμα. Καὶ
τὰ μὲν τοῦ Μεγασθένους ταῦτα.
13a.1
 Strabo XV: Φησὶ γὰρ
τοὺς τὸν Καύκασον οἰκοῦντας ἐν τῷ φανερῷ γυναιξὶ
μίσγεσθαι καὶ σαρκοφαγεῖν τὰ τῶν συγγενῶν σώματα·
πετροκυλιστὰς δ' εἶναι κερκοπιθήκους, οἳ λίθους κατα-
13a.5
κυλίουσι κρημνοβατοῦντες ἐπὶ τοὺς διώκοντας, τά τε
παρ' ἡμῖν ἥμερα ζῷα τὰ πλεῖστα παρ' ἐκείνοις ἄγρια
εἶναι· ἵππους τε λέγει μονοκέρωτας ἐλαφοκράνους· κα-
λάμους δὲ μῆκος μὲν τριάκοντα ὀργυιῶν τοὺς ὀρθίους,
τοὺς δὲ χαμαικλινεῖς πεντήκοντα· πάχος δὲ, ὥστε τὴν
13a.10
διάμετρον τοῖς μὲν εἶναι τριπήχη, τοῖς δὲ διπλασίαν.
13b.1
 Aelianus XVI,
20:
Ἐν
τοῖς χωρίοις τοῖς ἐν Ἰνδίᾳ, λέγω δὴ τοῖς ἐνδοτάτω,
13b.5
ὄρη φασὶν εἶναι δύσβατά τε καὶ ἔνθηρα· καὶ ἔχειν ζῷα
ὅσα καὶ ἡ καθ' ἡμᾶς τρέφει γῆ, ἄγρια δέ· καὶ γάρ τοι
καὶ τὰς ὄϊς τὰς ἐκεῖ φασιν εἶναι καὶ ταύτας θηρία, καὶ
κύνας καὶ αἶγας καὶ βοῦς, αὐτόνομά τε ἀλᾶσθαι καὶ
ἐλεύθερα, ἀφειμένα νομευτικῆς ἀρχῆς. Πλήθη δὲ αὐτῶν
13b.10
καὶ ἀριθμοῦ πλείω φασὶν οἱ τούτων συγγραφεῖς, καὶ
οἱ τῶν Ἰνδῶν λόγιοι· ἐν δὴ τοῖς καὶ τοὺς Βραχμᾶνας
ἀριθμεῖν ἄξιον, καὶ γάρ τοι καὶ ἐκεῖνοι ὑπὲρ τῶνδε
ὁμολογοῦσι τὰ αὐτά. Λέγεται δὲ καὶ ζῷον ἐν τούτοις
εἶναι μονόκερων, καὶ ὑπ' αὐτῶν ὀνομάζεσθαι καρτά-
13b.15
ζωνον· καὶ μέγεθος μὲν ἔχειν ἵππου τοῦ τελείου, καὶ
λόφον, καὶ λάχνην ἔχειν ξανθὴν, ποδῶν δὲ ἄριστα εἰλη-
χέναι, καὶ εἶναι ὤκιστον, καὶ τοὺς μὲν πόδας ἀδιαρ-
θρώτους τε καὶ ἐμφερεῖς ἐλέφαντι συμπεφυκέναι, τὴν
δὲ οὐρὰν συός· μέσον δὲ τῶν ὀφρύων ἔχειν ἐκπεφυκὸς
13b.20
κέρας, οὐ λεῖον, ἀλλὰ ἑλιγμοὺς ἔχον τινὰς καὶ μάλα
αὐτοφυεῖς, καὶ εἶναι μέλαν τὴν χρόαν· λέγεται δὲ καὶ
ὀξύτατον εἶναι τὸ κέρας ἐκεῖνο· φωνὴν δὲ ἔχειν τὸ θηρίον
ἀκούω τοῦτο πάντων ἀπηχεστάτην τε καὶ γεγωνοτάτην·
καὶ τῶν μὲν ἄλλων αὐτῷ ζῴων προσιόντων φέρειν, καὶ
13b.25
πρᾷον εἶναι, λέγουσι δὲ ἄρα πρὸς τὸ ὁμόφυλον δύσεριν
εἶναί πως. Καὶ οὐ μόνον φασὶ τοῖς ἄρρεσιν εἶναί τινα
συμφυῆ κύριξίν τε πρὸς ἀλλήλους καὶ μάχην, ἀλλὰ καὶ
πρὸς τὰς θηλείας ἔχουσι θυμὸν τὸν αὐτὸν, καὶ προάγειν
τε τὴν φιλονεικίαν καὶ μέχρι θανάτου ἡττηθέντα ἐξά-
13b.30
γουσαν· ἔστι μὲν οὖν καὶ διὰ παντὸς τοῦ σώματος ῥω-
μαλέον, ἀλκὴ δέ οἱ τοῦ κέρατος ἄμαχός ἐστι. Νομὰς
δὲ ἐρήμους ἀσπάζεται καὶ πλανᾶται μόνον· ὥρᾳ δὲ  
ἀφροδίτης τῆς σφετέρας συνδυασθεὶς πρὸς τὴν θήλειαν
πεπράϋνται, καὶ μέντοι καὶ συννόμω ἐστόν· εἶτα ταύτης
13b.35
παραδραμούσης καὶ τῆς θηλείας κυούσης, ἐκθηριοῦται
αὖθις, καὶ μονίας ἐστὶν ὅδε ὁ Ἰνδὸς καρτάζωνος. Τούτων
οὖν πώλους πάνυ νεαροὺς κομίζεσθαί φασι τῷ τῶν
Πραισίων βασιλεῖ, καὶ τὴν ἀλκὴν ἐν ἀλλήλοις ἀποδεί-
κνυσθαι κατὰ τὰς πανηγυρικάς· τέλειον δὲ ἁλῶναί ποτε
13b.40
οὐδεὶς μέμνηται.
 Id. c. 21: Ὑπερελθόντι τὰ ὄρη τὰ γειτνιῶντα τοῖς
Ἰνδοῖς κατὰ τὴν ἐνδοτάτην πλευρὰν φανοῦνται, φασὶν,
αὐλῶνες δασύτατοι, καὶ καλεῖταί γε ὑπ' Ἰνδῶν ὁ χῶρος
Κόρουδα· ἀλῶνται δὲ ἄρα, φασὶν, ἐν
13b.45
τοῖσδε τοῖς αὐλῶσι ζῷα Σατύροις ἐμφερῆ τὰς μορφὰς,
τὸ πᾶν σῶμα λάσια, καὶ ἔχει κατὰ τῆς ἰξύος ἵππουριν.
Καὶ καθ' ἑαυτὰ μὲν μὴ ἐνοχλούμενα διατρίβειν ἐν τοῖς
δρυμοῖς ὑλοτραγοῦντα, ὅταν δὲ αἴσθωνται κυνηγετῶν
κτύπου, καὶ ἀκούσωσι κυνῶν ὑλακῆς, ἀναθέουσιν εἰς
13b.50
τὰς ἀκρωρείας αὐτὰς ἀμηχάνῳ τῷ τάχει· καὶ γάρ εἰσι
ταῖς ὀρειβασίαις ἐντριβεῖς. Καὶ ἀπομάχονται πέτρας
τινὰς κυλινδοῦντες κατὰ τῶν ἐπιόντων, καὶ καταλαμ-


βανόμενοί γε πολλοὶ διαφθείρονται. Καὶ ἐκ τούτων
εἰσὶν ἐκεῖνοι δυσάλωτοι· καὶ μόλις ποτὲ καὶ διὰ μακροῦ
13b.55
τινὰς αὐτῶν <εἰς Πραισίους> κομίζεσθαι λέγουσι· καὶ
τούτων μέντοι ἢ τὰ νοσοῦντα ἐκομίσθη, ἢ θήλεά τινα
καὶ κυοῦντα· καὶ συνέβη γε θηραθῆναι τοῖς μὲν διὰ τὴν
νωθείαν, ταῖς δὲ διὰ τὸν τῆς γαστρὸς ὄγκον.
14.1
 Plinius VIII, 14, 1: Megasthenes scribit in
India serpentes in tantam magnitudinem adolescre,
ut solidos hauriant cervos taurosque.
15.1
 Aelian. N. A. VIII, 7: Μεγασθένους ἀκούω λέ-
γοντος περὶ τὴν τῶν Ἰνδῶν θάλατταν γίγνεσθαί τι ἰχθύ-
διον, καὶ τοῦτο μὲν ὅταν ζῇ, ἀθέατον εἶναι, κάτω που
νηχόμενον καὶ ἐν βυθῷ, ἀποθανὸν δὲ ἀναπλεῖν· οὗ τὸν
15.5
ἁψάμενον λειποθυμεῖν καὶ ἐκθνήσκειν τὰ πρῶτα, εἶτα
μέντοι καὶ ἀποθνήσκειν.
16.1
 Plinius H. N. VI, 24: Megasthenes flumine
dividi (Taprobanen insulam tradit) incolasque
palaeogonos appellari, aurique margaritarumque
fertiliores quam Indos.  
17.1
 Antigonus Hist. Mirab. c. 147:
Μεγασθένην δὲ τὸν τὰ Ἰνδικὰ γεγραφότα
ἱστορεῖν, ἐν τῇ κατὰ τὴν Ἰνδικὴν θαλάττῃ δένδρεα
φύεσθαι.
18a.1
 Arrianus Ind. 4, 2 – 13: Αὐτοῖν δὲ τοῖν μεγί-
στοιν ποταμοῖν τοῦ τε Γάγγεω καὶ τοῦ Ἰνδοῦ τὸν Γάγ-
γεα μεγέθει πολύ τι ὑπερφέρειν Μεγασθένης ἀνέγραψεν,
καὶ ὅσοι ἄλλοι μνήμην τοῦ Γάγγεω ἔχουσιν· (3) αὐτόν
18a.5
τε γὰρ μέγαν ἀνίσχειν ἐκ τῶν πηγέων, δέχεσθαί τε ἐς
αὑτὸν τόν τε Καϊνὰν ποταμὸν, καὶ τὸν Ἐραννοβόαν
καὶ τὸν Κοσσόανον, πάντας
πλωτούς· ἔτι δὲ Σῶνόν τε ποταμὸν καὶ Σιττόκατιν
καὶ Σολόματιν, καὶ τούτους πλω-
18a.10
τούς. (4) Ἐπὶ δὲ Κονδοχάτην τε καὶ Σάμβον καὶ
Μάγωνα καὶ Ἀγόρανιν καὶ Ὤμαλιν·
ἐμβάλλουσι δὲ ἐς αὐτὸν Κομμενάσης
τε μέγας ποταμὸς καὶ Κάκουθις καὶ Ἀνδώματις
ἐξ ἔθνεος Ἰνδικοῦ τοῦ Μανδιαδινῶν ῥέων· (5) καὶ ἐπὶ
18a.15
τούτοισιν Ἄμυστις παρὰ πόλιν Καταδούπην, καὶ Ὀξύ-
ματις ἐπὶ
Παζάλαισι καλου-
μένοισι, καὶ Ἐρρένυσις
ἐν Μάθαισιν, ἔθνεϊ Ἰνδικῷ, ξυμβάλλει τῷ Γάγγῃ.
18a.20
(6) Τούτων λέγει Μεγασθένης οὐδένα εἶναι τοῦ Μαιάν-
δρου ἀποδέοντα, ἵναπερ ναυσίπορος ὁ Μαίανδρος.
(7) Εἶναι ὦν τὸ εὖρος τῷ Γάγγῃ, ἔνθα περ αὐτὸς ἑωϋτοῦ
στεινότατος, εἰς ἑκατὸν σταδίους· πολλαχῆ δὲ καὶ λι-
μνάζειν, ὡς μὴ ἄποπτον εἶναι τὴν πέρην χώρην, ἵναπερ
18a.25
χθαμαλή τε ἐστι καὶ οὐδαμῆ γηλόφοισιν ἀνεστηκυῖα.
(8) Τῷ δὲ Ἰνδῷ ἐς ταὐτὸν ἔρχεται. Ὑδραώτης μὲν
Καμβισθόλοισι παρειληφὼς τόν τε Ὕφασιν ἐν Ἀστρό-
βαισι καὶ τὸν Σαράγγην ἐκ Κηκέων
καὶ τὸν Νεῦ-
18a.30
δρον ἐξ Ἀττακηνῶν, ἐς Ἀκεσίνην ἐμβάλλουσιν. (9) Ὑδά-
σπης ἐν δὲ Ὀξυδράκαισιν ἄγων ἅμα οἱ τὸν Σίναρον ἐν
Ἀρίσπαισι, ἐς τὸν Ἀκεσίνην ἐκδιδοῖ καὶ οὗτος. (10) Ὁ δὲ
Ἀκεσίνης ἐν Μαλλοῖς ξυμβάλλει τῷ Ἰνδῷ· καὶ Τούταπος
δὲ μέγας ποταμὸς ἐς τὸν Ἀκεσίνην ἐκδιδοῖ. Τούτων
18a.35
ὁ Ἀκεσίνης ἐμπλησθεὶς καὶ τῇ ἐπικλήσει ἐκνικήσας αὐτὸς
τῷ ἑωϋτοῦ ἤδη οὐνόματι ἐσβάλλει ἐς τὸν Ἰνδόν· (11) Κω-
φὴν δὲ ἐν Πευκελαιήτιδι, ἅμα οἷ ἄγων Μαλάμαντόν
τε καὶ Σόαστον καὶ Γαρροίαν, ἐκδιδοῖ
ἐς τὸν Ἰνδόν. (12) Καθύπερθε δὲ τούτων Πάρενος
18a.40
καὶ Σάπαρνος, οὐ πολὺ διέχοντες, ἐμβάλ-
λουσιν ἐς τὸν Ἰνδόν. Σόανος δὲ, ἐκ τῆς ὀρεινῆς τῆς
Ἀβισσαρέων,
ἔρημος ἄλλου ποταμοῦ, ἐκ-
διδοῖ ἐς αὐτόν. Καὶ τούτων τοὺς πολλοὺς Μεγασθένης
18a.45
λέγει, ὅτι πλωτοί εἰσιν.
18b.1
 Τὸ δὲ αἴτιον ὅστις ἐθέλει φράζειν τοῦ
πλήθεός τε καὶ μεγέθεος τῶν Ἰνδῶν ποταμῶν, φραζέτω·
ἐμοὶ δὲ καὶ ταῦτα ὡς ἀκοὴ ἀναγεγράφθω. Ἐπεὶ καὶ
ἄλλων πολλῶν ποταμῶν οὐνόματα Μεγασθένης ἀνέγρα-
18b.5
ψεν, οἳ ἔξω τοῦ Γάγγεώ τε καὶ τοῦ Ἰνδοῦ ἐκδιδοῦσιν
ἐς τὸν ἑῷόν τε καὶ μεσημβρινὸν τὸν ἔξω πόντον· ὥστε
τοὺς πάντας ὀκτὼ καὶ πεντήκοντα λέγει ὅτι εἰσὶν Ἰνδοὶ
ποταμοὶ ναυσίποροι πάντες. Ἀλλ' οὐδὲ Μεγασθένης  
πολλὴν δοκέει μοι ἐπελθεῖν τῆς Ἰνδῶν χώρης, πλήν
18b.10
γε ὅτι πλεῦνα ἢ οἱ ξὺν Ἀλεξάνδρῳ τῷ Φιλίππου ἐπελ-
θόντες. Συγγενέσθαι γὰρ Σανδρακόττῳ λέγει, τῷ με-
γίστῳ βασιλέϊ καὶ Πώρῳ, ἔτι τούτου μέζονι
18c.1
 Strabo XV: Λέγουσιν οἱ μὲν
τριάκοντα σταδίων τοὐλάχιστον πλάτος, οἱ δὲ καὶ
τριῶν· Μεγασθένης δὲ, ὅταν ᾖ μέτριος, καὶ ἐς ἑκατὸν
εὐρύνεσθαι, βάθος δὲ εἴκοσι ὀργυιῶν τοὐλάχιστον. Ἐπὶ
18c.5
δὲ συμβολῇ τούτου κτλ.  
19a.1
 Arrianus Ind. 6, 2: Καὶ τόδε λέγει Μεγασθένης
ὑπὲρ ποταμοῦ Ἰνδικοῦ, Σίλαν μὲν εἶναί οἱ οὔνομα, ῥέειν
δὲ ἀπὸ κρήνης ἐπωνύμου τῷ ποταμῷ διὰ τῆς χώρης
τῆς Σιλέων, καὶ τούτων ἐπωνύμων τοῦ ποταμοῦ τε καὶ
19a.5
τῆς κρήνης· (3) τὸ δὲ ὕδωρ παρέχεσθαι τοῖόνδε· οὐδὲν 

εἶναι ὅτεῳ ἀντέχει τὸ ὕδωρ, οὔτε τι νήχεσθαι ἐπ' αὐτοῦ
οὔτε τι ἐπιπλέειν, ἀλλὰ πάντα γὰρ ἐς βυσσὸν δύνειν·
οὕτω τι ἀμενηνότερον πάντων εἶναι τὸ ὕδωρ ἐκεῖνο καὶ
ἠεροειδέστερον.  
19b.1
 Strabo XV: Ἐν δὲ τῇ ὀρεινῇ Σιλίαν
ποταμὸν εἶναι,
ᾧ μηδὲν ἐπιπλεῖ· Δημόκριτον μὲν οὖν ἀπιστεῖν ἅτε
πολλὴν τῆς Ἀσίας πεπλανημένον· καὶ Ἀριστοτέλης δὲ
19b.5
ἀπιστεῖ.

E LIBRO SECUNDO.

20.1
 Strabo XV: Ἡμῖν δὲ τίς ἂν δικαία γένοιτο
πίστις περὶ τῶν Ἰνδικῶν ἐκ τῆς τοιαύτης στρατείας
τοῦ Κύρου ἢ τῆς Σεμιράμιδος;
Συναποφαίνεται δέ πως καὶ Μεγασθένης τῷ λόγῳ
20.5
τούτῳ, κελεύων ἀπιστεῖν ταῖς ἀρχαίαις περὶ Ἰνδῶν
ἱστορίαις· οὔτε γὰρ παρ' Ἰνδῶν ἔξω σταλῆναί ποτε
στρατιὰν, οὔτ' ἐπελθεῖν ἔξωθεν καὶ κρατῆσαι, πλὴν
τῆς μεθ' Ἡρακλέους καὶ Διονύσου, καὶ τῆς νῦν μετὰ
Μακεδόνων. Καίτοι Σέσωστριν μὲν τὸν Αἰγύπτιον καὶ
20.10
Τεάρκωνα τὸν Αἰθίοπα ἕως Εὐρώπης προελθεῖν,
Ναβοκοδρόσορον δὲ τὸν παρὰ Χαλδαίοις εὐδοκιμή-
σαντα, Ἡρακλέους μᾶλλον, καὶ ἕως Στηλῶν ἐλάσαι·
μέχρι μὲν δὴ δεῦρο καὶ Τεάρκωνα ἀφικέσθαι· ἐκεῖνον
δὲ καὶ ἐκ τῆς Ἰβηρίας εἰς τὴν Θρᾴκην καὶ τὸν Πόντον
20.15
ἀγαγεῖν τὴν στρατιάν. Ἰδάνθυρσον δὲ τὸν Σκύθην ἐπι-
δραμεῖν τῆς Ἀσίας μέχρι Αἰγύπτου· τῆς Ἰνδικῆς δὲ
μηδένα τούτων ἅψασθαι. Καὶ Σεμίραμιν δ' ἀποθανεῖν
πρὸ τῆς ἐπιχειρήσεως. Πέρσας δὲ μισθοφόρους μὲν ἐκ
τῆς Ἰνδικῆς μεταπέμψασθαι Ὕδρακας· ἐκεῖ δὲ μὴ
20.20
στρατεῦσαι, ἀλλ' ἐγγὺς ἐλθεῖν μόνον, ἡνίκα Κῦρος
ἤλαυνεν ἐπὶ Μασσαγέτας.
 Καὶ τὰ περὶ Ἡρακλέους δὲ καὶ Διονύσου Μεγασθέ-
νης μὲν μετ' ὀλίγων πιστὰ ἡγεῖται· τῶν δ' ἄλλων οἱ
πλείους, ὧν ἐστὶ καὶ Ἐρατοσθένης, ἄπιστα καὶ μυ-
20.25
θώδη, καθάπερ καὶ τὰ παρὰ τοῖς Ἕλλησιν.
21.1
 Arrianus Ind. 5, 4: Οὗτος ὦν ὁ Μεγασθένης λέγει,
οὔτε Ἰνδοὺς ἐπιστρατεῦσαι οὐδαμοῖσιν ἀνθρώποισιν,
οὔτε Ἰνδοῖσιν ἄλλους ἀνθρώπους· (5) ἀλλὰ Σέσωστριν
μὲν τὸν Αἰγύπτιον, τῆς Ἀσίης καταστρεψάμενον τὴν
21.5
πολλὴν, ἔστε ἐπὶ τὴν Εὐρώπην σὺν στρατιῇ ἐλάσαντα
ὀπίσω ἀπονοστῆσαι· (6) Ἰνδάθυρσιν δὲ τὸν Σκύθεα ἐκ
Σκυθίης ὁρμηθέντα πολλὰ μὲν τῆς Ἀσίης ἔθνεα κατα-
στρέψασθαι, ἐπελθεῖν δὲ καὶ τὴν Αἰγυπτίων γῆν κρα-
τέοντα· (7) Σεμίραμιν δὲ τὴν Ἀσσυρίην ἐπιχειρέειν,
21.10
μὲν στέλλεσθαι εἰς Ἰνδοὺς, ἀποθανεῖν δὲ πρὶν τέλος
ἐπιθεῖναι τοῖσι βουλεύμασιν· ἀλλὰ Ἀλέξανδρον γὰρ
στρατεῦσαι ἐπὶ Ἰνδοὺς μοῦνον. (8) Καὶ πρὸ Ἀλεξάν-
δρου Διονύσου μὲν πέρι πολλὸς λόγος κατέχει, ὡς καὶ
τούτου στρατεύσαντος ἐς Ἰνδοὺς, καὶ καταστρεψαμένου
21.15
Ἰνδούς· Ἡρακλέος δὲ πέρι οὐ πολλός. (9) Διονύσου
μέν γε καὶ Νύσσα πόλις μνῆμα οὐ φαῦλον τῆς στρατη-
λασίης καὶ ὁ Μηρὸς τὸ ὄρος, καὶ ὁ κισσὸς ὅτι ἐν τῷ
ὄρεϊ τούτῳ φύεται· καὶ αὐτοὶ οἱ Ἰνδοὶ ὑπὸ τυμπάνων
τε καὶ κυμβάλων στελλόμενοι ἐς τὰς μάχας· καὶ ἐσθὴς
21.20
αὐτοῖσι κατάστικτος ἐοῦσα, καθάπερ τοῦ Διονύσου τοῖσι
βάκχοισιν· (10) Ἡρακλέος δὲ οὐ πολλὰ ὑπομνήματα.
Ἀλλὰ τὴν Ἄορνον γὰρ πέτρην, ἥντινα Ἀλέξανδρος βίῃ
ἐχειρώσατο, ὅτι Ἡρακλέης οὐ δυνατὸς ἐγένετο ἐξελεῖν,  
Μακεδονικὸν δοκέει μοί τι κόμπασμα, κατάπερ καὶ τὸν
21.25
Παραπάμισον Καύκασον ἐκάλεον Μακεδόνες, οὐδέν τι
προσήκοντα τοῦτον τῷ Καυκάσῳ. (11) Καί τι καὶ ἄντρον
ἐπιφρασθέντες ἐν Παραπαμισάδαισι, τοῦτο ἔφρασαν
ἐκεῖνο εἶναι τοῦ Προμηθέως τοῦ Τιτῆνος τὸ ἄντρον, ἐν
ὅτεῳ ἐκρέματο ἐπὶ τῇ κλοπῇ τοῦ πυρός. (12) Καὶ
21.30
ἐν Σίβαισιν, Ἰνδικῷ γένεϊ, ὅτι δορὰς ἀμπεχομένους
εἶδον τοὺς Σίβας, ἀπὸ τῆς Ἡρακλέος στρατηλασίης
ἔφασκον τοὺς ὑπολειφθέντας εἶναι τοὺς Σίβας· καὶ γὰρ
καὶ σκυτάλην φέρουσί τε οἱ Σίβαι, καὶ τοῖς βουσὶν αὐ-
τῶν ῥόπαλον ἐπικέκαυται· καὶ τοῦτο ἐς μνήμην ἀνέφε-
21.35
ρον τοῦ ῥοπάλου τοῦ Ἡρακλέος.
22a.1
 Josephus Ant. Iud. X, 11, 1: [Ἐν δὲ τοῖς βασι-
λείοις τούτοις ἀναλήμματα λίθινα ἀνοικοδομήσας (scil.
ὁ Ναβουχοδονόσορος), καὶ τὴν ὄψιν ἀποδοὺς ὁμοιοτάτην
τοῖς ὄρεσι καταφυτεύσας δένδρεσι παντοδαποῖς ἐξειρ-
22a.5
γάσατο, διὰ τὸ τὴν γυναῖκα αὐτοῦ ἐπιθυμεῖν τῆς
οἰκείας διαθέσεως ὡς τεθραμμένην ἐν τοῖς κατὰ Μηδίαν
τόποις.] <Καὶ Μεγασθένης> δὲ <ἐν τῇ δʹ> (l. δευτέρᾳ)
<τῶν Ἰνδικῶν> μνημονεύει αὐτῶν, δι' ἧς ἀποφαίνειν
πειρᾶται τοῦτον τὸν βασιλέα τῇ ἀνδρείᾳ καὶ τῷ μεγέ-
22a.10
θει τῶν πράξεων ὑπερβεβηκότα τὸν Ἡρακλέα· κατα-
στρέψασθαι γὰρ αὐτόν φησι Λιβύης τὴν πολλὴν καὶ
Ἰβηρίαν. Καὶ Διοκλῆς δὲ ἐν τῇ δευτέρᾳ τῶν Περσικῶν
μνημονεύει τούτου τοῦ βασιλέως, καὶ Φιλόστρατος ἐν
ταῖς Ἰνδικαῖς αὐτοῦ καὶ Φοινικικαῖς ἱστορίαις, ὅτι οὗ-
22a.15
τος ὁ βασιλεὺς ἐπολιόρκησε Τύρον ἔτεσι τρισὶ καὶ δέκα,
βασιλεύοντος κατ' ἐκεῖνον τὸν καιρὸν Ἰθωβάλου τῆς
Τύρου.
22b.1
 Syncellus: Τὸν
Ναβουχοδονόσωρ ὁ Μεγασθένης ἐν τῇ δʹ (l. δευτέρᾳ)
τῶν Ἰνδικῶν Ἡρακλέους ἀλκιμώτερον ἀποφαίνει, ὃς 


ἀνδρείᾳ μεγάλῃ Λιβύης τὸ πλεῖστον καὶ Ἰβηρίας κα-
22b.5
τεστρέψατο. Συμφωνεῖ δὲ αὐτῷ καὶ Φιλόστρατος ἐν ταῖς
Ἱστορίαις, ἔνθα καὶ περὶ τῆς Τύρου πολιορκίας καὶ τῶν
Φοινίκων πάλαι, αἵτινες ἱστορίαι φέρονται περὶ τοῦ
Ναβουχοδονόσωρ, ὅτι Συρίαν καὶ Αἴγυπτον καὶ πᾶσαν
τὴν Φοινίκην κατεστρέψατο πολέμοις.
22c.1
 Euseb. Pr. Ev. IX: Εὗρον δὲ καὶ ἐν τῇ
Ἀβυδηνοῦ Περὶ Ἀσσυρίων γραφῇ περὶ τοῦ Ναβουχο-
δονόσορ ταῦτα· Μεγασθένης δέ φησι Ναβουκοδρόσο-
ρον Ἡρακλέος ἀλκιμώτερον γεγονότα ἐπί τε Λιβύην
22c.5
καὶ Ἰβηρίην στρατεῦσαι· ταύτας δὲ χειρωσάμενον ἀπο-
δασμὸν αὐτέων εἰς τὰ δεξιὰ τοῦ Πόντου κατοικίσαι.
23.1
 Arrianus Ind. c. 7: Ἔθνεα δὲ Ἰνδικὰ εἴκοσι καὶ
ἑκατόν τι ἅπαντα λέγει <Μεγασθένης> δυοῖν δέοντα.
[Καὶ πολλὰ μὲν εἶναι ἔθνεα Ἰνδικὰ καὶ αὐτὸς ξυμφέ-
ρομαι Μεγασθένεϊ· τὸ δὲ ἀτρεκὲς οὐκ ἔχω εἰκάσαι ὅπως
23.5
ἐκμαθὼν ἀνέγραψεν, οὐδὲ πολλοστὸν μέρος τῆς Ἰνδῶν
γῆς ἐπελθὼν, οὐδὲ ἐπιμιξίης πᾶσι τοῖς γένεσιν ἐούσης
ἐς ἀλλήλους.]
 Πάλαι μὲν δὴ νομάδας εἶναι Ἰνδοὺς κατάπερ Σκυ-
θέων τοὺς οὐκ ἀροτῆρας, οἳ ἐπὶ τῇσιν ἁμάξῃσι πλανώ-
23.10
μενοι ἄλλοτε ἄλλην τῆς Σκυθίης ἀμείβουσιν, οὔτε πό-
λιας οἰκέοντες οὔτε ἱερὰ θεῶν σέβοντες· οὕτω μηδὲ
Ἰνδοῖσι πόλιας εἶναι μηδὲ ἱερὰ θεῶν δεδομημένα· ἀλλ'
ἀμπέχεσθαι μὲν δορὰς θηρίων ὅσων κατακτάνοιεν· σι-  
τέεσθαι δὲ τῶν δενδρέων τὸν φλοιόν· καλέεσθαι δὲ τὰ
23.15
δένδρεα ταῦτα τῇ Ἰνδῶν φωνῇ <Τάλα·> καὶ φύεσθαι
ἐπ' αὐτῶν κατάπερ τῶν φοινίκων ἐπὶ τῇσι κορυφῇσιν
οἷά περ τολύπας. Σιτέεσθαι δὲ καὶ τῶν θηρίων ὅσα
ἕλοιεν ὠμοφαγέοντας, πρὶν δὴ Διόνυσον ἐλθεῖν ἐς τὴν
χώρην τῶν Ἰνδῶν. Διόνυσον δὲ ἐλθόντα, ὡς καρτερὸς
23.20
ἐγένετο Ἰνδῶν, πόλιάς τε οἰκῆσαι καὶ νόμους θέσθαι
τῇσι πόλισιν, οἴνου τε δοτῆρα Ἰνδοῖς γενέσθαι κατάπερ
Ἕλλησι, καὶ σπείρειν διδάξαι τὴν γῆν διδόντα αὐτὸν
σπέρματα· ἢ οὐκ ἐλάσαντος ταύτῃ Τριπτολέμου, ὅτε
περ ἐκ Δήμητρος ἐστάλη σπείρειν τὴν γῆν πᾶσαν,
23.25
ἢ πρὸ Τριπτολέμου τις οὗτος Διόνυσος ἐπελθὼν τὴν
Ἰνδῶν γῆν σπέρματά σφισιν ἔδωκε καρποῦ τοῦ ἡμέ-
ρου· βόας τε ὑπ' ἀρότρῳ ζεῦξαι Διόνυσον πρῶτον, καὶ
ἀροτῆρας ἀντὶ νομάδων ποιῆσαι Ἰνδῶν τοὺς πολλοὺς
καὶ ὁπλίσαι ὅπλοισι τοῖσιν ἀρηΐοισι. Καὶ θεοὺς σέβειν
23.30
ὅτι ἐδίδαξε Διόνυσος ἄλλους τε καὶ μάλιστα δὴ ἑωυτὸν
κυμβαλίζοντας καὶ τυμπανίζοντας· καὶ ὄρχησιν δὲ ἐκ-
διδάξαι τὴν σατυρικὴν, τὸν κόρδακα παρ' Ἕλλησι κα-
λεόμενον· καὶ κομᾶν Ἰνδοὺς τῷ θεῷ, μιτρηφορέειν τε
ἀναδεῖξαι καὶ μύρων ἀλοιφὰς ἐκδιδάξαι, ὥστε καὶ εἰς
23.35
Ἀλέξανδρον ἔτι ὑπὸ κυμβάλων τε καὶ τυμπάνων ἐς τὰς
μάχας Ἰνδοὶ καθίσταντο.
 C. 8. Ἀπιόντα δὲ ἐκ τῆς Ἰνδῶν γῆς, ὥς οἱ ταῦτα κε-
κοσμέατο, καταστῆσαι βασιλέα τῆς χώρης Σπατέμβαν,
τῶν ἑταίρων ἕνα τὸν βακχωδέστατον· τελευτήσαντος δὲ
23.40
Σπατέμβα τὴν βασιληίην ἐκδέξασθαι Βουδύαν τὸν
τούτου παῖδα· καὶ τὸν μὲν πεντήκοντα καὶ δύο ἔτεα
βασιλεῦσαι Ἰνδῶν, τὸν πατέρα· τὸν δὲ παῖδα εἴκοσιν
ἔτεα· καὶ τούτου παῖδα ἐκδέξασθαι τὴν βασιληίην Κρα-
δεύαν· καὶ τὸ ἀπὸ τοῦδε τὸ πολὺ μὲν κατὰ γένος ἀμεί-
23.45
βειν τὴν βασιληίην, παῖδα παρὰ πατρὸς ἐκδεκόμενον·
εἰ δὲ ἐκλείποι τὸ γένος, οὕτω δὴ ἀριστίνδην καθίστα-
σθαι Ἰνδοῖσι βασιλέας.
 Ἡρακλέα δὲ, ὅντινα ἐς Ἰνδοὺς ἀφικέσθαι λόγος κα-
τέχει, παρ' αὐτοῖσιν Ἰνδοῖσι γηγενέα λέγεσθαι. Τοῦ-
23.50
τον τὸν Ἡρακλέα μάλιστα πρὸς Σουρασηνῶν γεραίρε-
σθαι Ἰνδικοῦ ἔθνεος, ἵνα δύο πόλιες μεγάλαι Μέθορά
τε καὶ Κλεισόβορα, καὶ ποταμὸς Ἰωβάρης πλωτὸς διαρ-
ρέει τὴν χώρην αὐτῶν. Τὴν σκευὴν δὲ οὗτος ὁ Ἡρα-
κλέης ἥντινα ἐφόρεε, <Μεγασθένης> λέγει ὅτι ὁμοίην
23.55
τῷ Θηβαίῳ Ἡρακλέϊ, ὡς αὐτοὶ Ἰνδοὶ ἀπηγέονται· καὶ
τούτῳ ἄρσενας μὲν παῖδας πολλοὺς κάρτα γενέσθαι ἐν
τῇ Ἰνδῶν γῇ, (πολλῇσι γὰρ δὴ γυναιξὶν ἐς γάμον ἐλ-
θεῖν καὶ τοῦτον τὸν Ἡρακλέα,) θυγατέρα δὲ μουνο-
γενέην· οὔνομα δὲ εἶναι τῇ παιδὶ Πανδαίην, καὶ τὴν
23.60
χώρην ἵνα τε ἐγένετο καὶ ἧστινος ἐπέτρεψεν αὐτὴν ἄρ-
χειν Ἡρακλέης, Πανδαίην, τῆς παιδὸς ἐπώνυμον· καὶ
ταύτῃ ἐλέφαντας μὲν γενέσθαι ἐκ τοῦ πατρὸς ἐς πεν-
τακοσίους, ἵππον δὲ ἐς τετρακισχιλίην, πεζῶν δὲ ἐς τὰς
τρεῖς καὶ δέκα μυριάδας. Καὶ τάδε μετεξέτεροι Ἰνδῶν
23.65
περὶ Ἡρακλέος λέγουσιν· ἐπελθόντα αὐτὸν πᾶσαν γῆν
καὶ θάλασσαν καὶ καθήραντα ὅ τι περ κακὸν κίναδος,
ἐξευρεῖν ἐν τῇ θαλάσσῃ κόσμον γυναικήϊον· [ὅντινα καὶ
εἰς τοῦτο ἔτι οἵ τε ἐξ Ἰνδῶν τῆς χώρης τὰ ἀγώγιμα
παρ' ἡμέας ἀγινέοντες σπουδῇ ὠνεόμενοι ἐκκομίζουσι·
23.70
καὶ Ἑλλήνων δὲ πάλαι καὶ Ῥωμαίων νῦν ὅσοι πολυ-
κτέανοι καὶ εὐδαίμονες, μέζονι ἔτι σπουδῇ ὠνέονται·]  
τὸν μαργαρίτην δὴ τὸν θαλάσσιον, οὕτω τῇ Ἰνδῶν
γλώσσῃ καλεόμενον· τὸν γὰρ Ἡρακλέα, ὡς καλόν οἱ
ἐφάνη τὸ φόρημα, ἐκ πάσης τῆς θαλάσσης ἐς τὴν Ἰν-
23.75
δῶν γῆν συναγινέειν τὸν μαργαρίτην δὴ τοῦτον, τῇ θυ-
γατρὶ τῇ ἑωυτοῦ εἶναι κόσμον.
 Καὶ λέγει <Μεγασθένης> (*), θηρεύεσθαι αὐτοῦ τὴν
κόγχην δικτύοισι, νέμεσθαι δ' ἐν τῇ θαλάσσῃ κατ'
αὐτὸ πολλὰς κόγχας, κατάπερ τὰς μελίσσας· καὶ εἶναι

23.80
γὰρ καὶ τοῖσι μαργαρίτῃσι βασιλέα ἢ βασίλισσαν, ὡς
τῇσι μελισσίῃσι. Καὶ ὅστις μὲν ἐκεῖνον κατ' ἐπιτυχίην
συλλάβοι, τοῦτον δὲ εὐπετέως περιβάλλειν καὶ τὸ ἄλλο
σμῆνος τῶν μαργαριτέων· εἰ δὲ διαφύγοι σφᾶς ὁ βασι-
λεὺς, τούτῳ δὲ οὐκέτι θηρατοὺς εἶναι τοὺς ἄλλους· τοὺς
23.85
ἁλόντας δὲ περιορᾶν κατασαπῆναί σφισι τὴν σάρκα, τῷ
δὲ ὀστέῳ ἐς κόσμον χρῆσθαι. Καὶ εἶναι γὰρ καὶ παρ'
Ἰνδοῖσι τὸν μαργαρίτην τριστάσιον κατὰ τιμὴν πρὸς
χρυσίον τὸ ἄπεφθον, καὶ τοῦτο ἐν τῇ Ἰνδῶν γῇ ὀρυς-
σόμενον.
23.90
 C. 9. Ἐν δὲ τῇ χώρῃ ταύτῃ, ἵνα ἐβασίλευσεν ἡ θυγά-
τηρ τοῦ Ἡρακλέος, τὰς μὲν γυναῖκας ἑπταέτεας ἐούσας
ἐς ὥρην γάμου ἰέναι, τοὺς δὲ ἄνδρας τεσσαράκοντα
ἔτεα τὰ πλεῖστα βιώσκεσθαι *. Καὶ ὑπὲρ τούτου λεγό-
μενον λόγον εἶναι παρ' Ἰνδοῖσιν· Ἡρακλέα, ὀψιγόνου
23.95
οἱ γενομένης τῆς παιδὸς, ἐπεί τε δὴ ἐγγὺς ἔμαθεν ἑωυτῷ
ἐοῦσαν τὴν τελευτὴν, οὐκ ἔχοντα ὅτεῳ ἀνδρὶ ἐκδῷ τὴν
παῖδα ἑωυτοῦ ἐπαξίῳ, αὐτὸν μιγῆναι τῇ παιδὶ ἑπταέτεϊ
ἐούσῃ, ὡς γένος ἐξ οὗ τε κἀκείνης ὑπολείπεσθαι Ἰνδῶν
βασιλέας. Ποιῆσαι ὦν αὐτὴν Ἡρακλέα ὡραίην γάμου·
23.100
καὶ ἐκ τοῦδε ἅπαν τὸ γένος τοῦτο ὅτου ἡ Πανδαίη
ἐπῆρξε, ταὐτὸ τοῦτον γέρας ἔχειν παρὰ Ἡρακλέος
[Ἐμοὶ δὲ δοκέει, εἴπερ ὦν τὰ ἐς τοσόνδε ἄτοπα Ἡρα-
κλέης οἷός τε ἦν ἐξεργάζεσθαι, καὶ αὑτὸν ἀποφῆναι μα-
κροβιώτερον, ὡς ὡραίῃ μιγῆναι τῇ παιδί. Ἀλλὰ γὰρ εἰ
23.105
ταῦτα ὑπὲρ τῆς ὥρης τῶν ταύτῃ παίδων ἀτρεκέα ἐστὶν,
ἐς ταὐτὸν φέρειν δοκέει ἔμοιγε ἐς ὅ τι περ καὶ ὑπὲρ τῶν
ἀνδρῶν τῆς ἡλικίης ὅτι τεσσαρακοντούτεες ἀποθνήσκου-
σιν οἱ πρεσβύτατοι αὐτῶν. Οἷς γὰρ τό τε γῆρας τοσῷδε
ταχύτερον ἐπέρχεται καὶ ὁ θάνατος ὁμοῦ τῷ γήραϊ,
23.110
πάντως που καὶ ἡ ἀκμὴ πρὸς λόγον τοῦ τέλεος ταχυ-
τέρη ἐπανθέει· ὥστε τριακοντούτεες μὲν ὠμογέροντες
ἄν που εἶεν αὐτοῖσιν οἱ ἄνδρες, εἴκοσι δὲ ἔτεα γεγονότες
οἱ ἔξω ἥβης νεηνίσκοι· ἡ δὲ ἀκροτάτη ἥβη ἀμφὶ τὰ
πεντεκαίδεκα ἔτεα· καὶ τῇσι γυναιξὶν ὥρη τοῦ γάμου
23.115
κατὰ λόγον ἂν οὕτω ἐς τὰ ἑπτὰ ἔτεα συμβαίνοι.] Καὶ
γὰρ τοὺς καρποὺς ἐν ταύτῃ τῇ χώρῃ πεπαίνεσθαί τε
ταχύτερον μὲν τῆς ἄλλης, αὐτὸς οὗτος <Μεγασθένης>
ἀνέγραψε, καὶ φθίνειν ταχύτερον.
 Ἀπὸ μὲν δὴ Διονύσου βασιλέας ἠρίθμεον Ἰνδοὶ ἐς
23.120
Σανδράκοττον τρεῖς καὶ πεντήκοντα καὶ ἑκατόν· ἔτεα
δὲ δύο καὶ τεσσαράκοντα καὶ ἑξακισχίλια· ἐν δὲ τού-
τοισι τρὶς τὸ πᾶν εἰς ἐλευθερίην ** τὴν δὲ καὶ ἐς τριη-
κόσια· τὴν δὲ εἴκοσί τε ἐτέων καὶ ἑκατόν (*)· πρεσβύ-  
τερόν τε Διόνυσον Ἡρακλέος δέκα καὶ πέντε γενεῇ-
23.125
σιν Ἰνδοὶ λέγουσιν· ἄλλον δὲ οὐδένα ἐμβαλεῖν ἐς γῆν
τὴν Ἰνδῶν ἐπὶ πολέμῳ, οὐδὲ Κῦρον τὸν Καμβύσεω, καί-
τοι ἐπὶ Σκύθας ἐλάσαντα καὶ τἄλλα πολυπραγμονέστα-
τον δὴ τῶν κατὰ τὴν Ἀσίην βασιλέων γενόμενον τὸν
Κῦρον· ἀλλὰ Ἀλέξανδρον γὰρ ἐλθεῖν τε καὶ κρατῆσαι
23.130
πάντων τοῖς ὅπλοις, ὅσους γε δὴ ἐπῆλθε· καὶ ἂν καὶ
πάντων κρατῆσαι, εἰ ἡ στρατιὴ ἤθελεν. Οὐ μὲν δὴ
οὐδὲ Ἰνδῶν τινα ἔξω τῆς οἰκηίης σταλῆναι ἐπὶ πολέμῳ
διὰ δικαιότητα.
24.1
 Phlegon Mirab. c. 33: Μεγασθένης δέ φησιν,
τὰς ἐν Παλαίᾳ κατοικούσας γυναῖκας
ἑξαετεῖς γινομένας τίκτειν.
25.1
 Strabo XV: Ἐπὶ δὲ τῇ συμβολῇ τούτου
(τοῦ Γάγγου) τε καὶ τοῦ ἄλλου ποταμοῦ τὰ Παλίβοθρα
ἱδρύσθαι (sc. Μεγασθένης φησὶ) σταδίων ὀγδοήκοντα
τὸ μῆκος, πλάτος δὲ πεντεκαίδεκα, ἐν παραλληλο-
25.5
γράμμῳ σχήματι, ξύλινον περίβολον ἔχουσαν κατατε-
τρημένον, ὥστε διὰ τῶν ὀπῶν τοξεύειν· προκεῖσθαι δὲ
καὶ τάφρον φυλακῆς τε χάριν καὶ ὑποδοχῆς τῶν ἐκ τῆς
πόλεως ἀπορροιῶν. Τὸ δ' ἔθνος, ἐν ᾧ ἡ πόλις αὕτη,
καλεῖσθαι Πρασίους, διαφορώτατον τῶν πάντων· τὸν
25.10
δὲ βασιλεύοντα ἐπώνυμον δεῖ τῆς πόλεως εἶναι Παλί-
βοθρον καλούμενον, πρὸς τῷ ἰδίῳ τῷ ἐκ γενετῆς ὀνό-
ματι, καθάπερ τὸν Σανδρόκοττον, πρὸς ὃν ἧκεν ὁ Με-
γασθένης πεμφθείς.  
26.1
 Arrian. Ind. c. 10: Λέγεται δὲ καὶ τάδε, μνημήια
ὅτι Ἰνδοὶ τοῖς τελευτήσασιν οὐ ποιέουσιν, ἀλλὰ τὰς ἀρε-
τὰς γὰρ τῶν ἀνδρῶν ἱκανὰς ἐς μνήμην τίθενται τοῖσιν
ἀποθανοῦσι, καὶ τὰς ᾠδὰς αἳ αὐτοῖσιν ἐπᾴδονται. (2) Πο-
26.5
λίων δὲ ἀριθμὸν οὐκ εἶναι ἂν ἀτρεκὲς ἀναγράψαι τῶν
Ἰνδικῶν ὑπὸ πλήθεος· ἀλλὰ γὰρ ὅσαι παραποτάμιαι αὐ-
τέων ἢ παραθαλάσσιαι, ταύτας μὲν ξυλίνας ποιέεσθαι·
(3) οὐ γὰρ εἶναι ἐκ πλίνθου ποιεομένας διαρκέσαι ἐπὶ
χρόνον τοῦ τε ὕδατος ἕνεκα τοῦ ἐξ οὐρανοῦ, καὶ ὅτι οἱ
26.10
ποταμοὶ αὐτοῖσιν ὑπερβάλλοντες ὑπὲρ τὰς ὄχθας ἐμπι-
πλᾶσι τοῦ ὕδατος τὰ πεδία. (4) Ὅσαι δὲ ἐν ὑπερδε-
ξίοισί τε καὶ μετεώροισι τόποισι καὶ τούτοισιν ὑψηλοῖσιν,
ᾠκισμέναι εἰσὶ, ταύτας δὲ ἐκ πλίνθου τε καὶ πηλοῦ
ποιέεσθαι· (5) μεγίστην δὲ πόλιν ἐν Ἰνδοῖσιν εἶναι Παλίμ-
26.15
βοθρα καλεομένην, ἐν τῇ Πρασίων γῇ, ἵνα αἱ συμβολαί
εἰσι τοῦ τε Ἐραννοβόα ποταμοῦ καὶ τοῦ Γάγγεως· τοῦ
μὲν Γάγγεω, τοῦ μεγίστου ποταμῶν· ὁ δὲ Ἐραννοβόας
τρίτος μὲν ἂν εἴη τῶν Ἰνδῶν ποταμῶν, μέζων δὲ τῶν
ἄλλῃ καὶ οὗτος· ἀλλὰ ξυγχωρέει αὐτὸς τῷ Γάγγῃ, ἐπει-
26.20
δὰν ἐμβάλλῃ ἐς αὐτὸν τὸ ὕδωρ. (6) Καὶ λέγει Μεγα-

σθένης, μῆκος μὲν ἐπέχειν τὴν πόλιν κατ' ἑκατέρην
τὴν πλευρὴν ἵναπερ μακροτάτη αὐτὴ ἑωυτῆς ᾤκισται
ἐς ὀγδοήκοντα σταδίους τὸ δὲ πλάτος ἐς πεντεκαίδεκα·
(7) τάφρον δὲ περιβεβλῆσθαι τῇ πόλι τὸ εὖρος ἑξάπλε-
26.25
θρον, τὸ δὲ βάθος τριήκοντα πήχεων· πύργους δὲ ἑβδο-
μήκοντα καὶ πεντακοσίους ἐπέχειν τὸ τεῖχος καὶ πύλας
τέσσαρας καὶ ἑξήκοντα. (8) Εἶναι δὲ καὶ τόδε μέγα
ἐν τῇ Ἰνδῶν γῇ, πάντας Ἰνδοὺς εἶναι ἐλευθέρους, οὐδέ
τινα δοῦλον εἶναι Ἰνδόν· τοῦτο μὲν Λακεδαιμονίοισιν
26.30
ἐς ταὐτὸ συμβαίνει καὶ Ἰνδοῖσι. (9) Λακεδαιμονίοισι
μέν γε οἱ εἵλωτες δοῦλοί εἰσι καὶ τὰ δούλων ἐργάζονται·
Ἰνδοῖσι δὲ οὐδὲ ἄλλος δοῦλός ἐστι μήτι γε Ἰνδῶν τις.
27.1
 Strabo XV: Εὐτελεῖς δὲ κατὰ τὴν
δίαιταν οἱ Ἰνδοὶ πάντες, μᾶλλον δ' ἐν ταῖς στρατείαις·
οὐδ' ὄχλῳ περιττῷ χαίρουσι· διόπερ εὐκοσμοῦσι. Πλεί-
στη δ' ἐπιχειρία περὶ τὰς κλοπάς. Γενομένους
27.5
δ' οὖν ἐν τῷ Σανδροκόττου στρατο-
πέδῳ φησὶν ὁ <Μεγασθένης,> τετταράκοντα μυριάδων
πλήθους ἱδρυμένου, μηδεμίαν ἡμέραν ἰδεῖν ἀνηνεγμένα
κλέμματα πλειόνων ἢ διακοσίων δραχμῶν ἄξια, ἀγρά-
φοις καὶ ταῦτα νόμοις χρωμένοις. (2) Οὐδὲ γὰρ γράμ-
27.10
ματα εἰδέναι αὐτοὺς, ἀλλ' ἀπὸ μνήμης ἕκαστα διοικεῖ-
σθαι (*)· εὖ πράττειν δ' ὅμως διὰ τὴν ἁπλότητα καὶ  
τὴν εὐτέλειαν· οἶνόν τε γὰρ οὐ πίνειν, ἀλλ' ἐν θυσίαις
μόνον· πίνειν δ' ἀπ' ὀρύζης ἀντὶ κριθίνου συντιθέντας.
Καὶ σιτία δὲ τὸ πλέον ὄρυζαν εἶναι ῥοφητήν. (3) Καὶ
27.15
ἐν τοῖς νόμοις δὲ καὶ ἐν τοῖς συμβολαίοις τὴν ἁπλότητα
ἐλέγχεσθαι ἐκ τοῦ μὴ πολυδίκους εἶναι· οὔτε γὰρ ὑπο-
θήκης οὔτε παρακαταθήκης εἶναι δίκας· οὐδὲ μαρτύ-
ρων, οὐδὲ σφραγίδων αὐτοῖς δεῖν, ἀλλὰ πιστεύειν πα-
ραβαλλομένους· καὶ τὰ οἴκοι δὲ τὸ πλέον ἀφρουρεῖν.
27.20
Ταῦτα μὲν δὴ σωφρονικά. Τἄλλ' οὐδ' ἄν τις ἀποδέ-
ξαιτο· τὸ μόνους διαιτᾶσθαι ἀεὶ, καὶ τὸ μὴ μίαν εἶναι
πᾶσιν ὥραν κοινὴν δείπνου τε καὶ ἀρίστου, ἀλλ' ὅπως
ἑκάστῳ φίλον. Πρὸς γὰρ τὸν κοινωνικὸν καὶ τὸν πολιτι-
κὸν βίον ἐκείνως κρεῖττον
27.25
 4. Γυμνάσιον δὲ μάλιστα τρίψιν δοκιμάζουσι, καὶ
ἄλλως καὶ διὰ σκυταλίδων ἐβενίνων λείων ἐξομαλίζον-
ται τὰ σώματα. Λιταὶ δὲ καὶ αἱ ταφαὶ, καὶ μικρὰ χώ-
ματα. Ὑπεναντίως δὲ τῇ ἄλλῃ λιτότητι κοσμοῦνται.
Χρυσοφοροῦσι γὰρ καὶ διαλίθῳ κόσμῳ χρῶνται, σιν-
27.30
δόνας τε φοροῦσιν εὐανθεῖς καὶ σκιάδια αὐτοῖς ἕπεται·
τὸ γὰρ κάλλος τιμῶντες ἀσκοῦσιν ὅσα καλλωπίζει τὴν
ὄψιν· ἀλήθειάν τε ὁμοίως καὶ ἀρετὴν ἀποδέχονται·
διόπερ οὐδὲ τῇ ἡλικίᾳ τῶν γερόντων προνομίαν διδόα-
σιν, ἂν μὴ καὶ τῷ φρονεῖν πλεονεκτῶσι. Πολλὰς δὲ
27.35
γαμοῦσιν ὠνητὰς παρὰ τῶν γονέων, λαμβάνουσί τε
ἀντιδιδόντες ζεῦγος βοῶν· ὧν τὰς μὲν εὐπειθείας χάριν,
τὰς δ' ἄλλας ἡδονῆς καὶ πολυτεκνίας· εἰ δὲ μὴ σωφρο-
νεῖν ἀναγκάσαιεν, πορνεύειν ἔξεστι. Θύει δὲ οὐδεὶς
ἐστεφανωμένος οὐδὲ σπένδει, οὐδὲ σφάττουσι τὸ ἱερεῖον
27.40
ἀλλὰ πνίγουσιν, ἵνα μὴ λελωβημένον ἀλλ' ὁλόκληρον
διδῶται τῷ θεῷ.
 5. Ψευδομαρτυρίας δ' ὁ ἁλοὺς ἀκρωτηριάζεται·
ὅ τε πηρώσας οὐ τὰ αὐτὰ μόνον ἀντιπάσχει, ἀλλὰ καὶ
χειροκοπεῖται· ἐὰν δὲ καὶ τεχνίτου χεῖρα ἢ ὀφθαλμὸν
27.45
ἀφέληται, θανατοῦται. Δούλοις δὲ οὗτος μέν φησι
μηδένα Ἰνδῶν χρῆσθαι· [Ὀνησίκριτος δὲ τῶν
ἐν τῇ Μουσικανοῦ τοῦτ' ἴδιον ἀποφαίνει κτλ.].
 6 Τῷ βασιλεῖ δ' ἡ μὲν τοῦ σώματος θεραπεία διὰ
γυναικῶν ἐστιν, ὠνητῶν καὶ αὐτῶν παρὰ τῶν πατέρων·
27.50
ἔξω δὲ τῶν θυρῶν οἱ σωματοφύλακες καὶ τὸ λοιπὸν
στρατιωτικόν· μεθύοντα δὲ κτείνασα γυνὴ βασιλέα γέ-
ρας ἔχει συνεῖναι τῷ ἐκεῖνον διαδεξαμένῳ· διαδέχονται
δ' οἱ παῖδες. Οὐδ' ὑπνοῖ μεθ' ἡμέραν ὁ βασιλεὺς, καὶ
νύκτωρ δὲ καθ' ὥραν ἀναγκάζεται τὴν κοίτην ἀλλάτ-
27.55
τειν διὰ τὰς ἐπιβουλάς.
 7. Τῶν γε μὴ κατὰ πόλεμον ἐξόδων μία μέν ἐστιν
ἡ ἐπὶ τὰς κρίσεις, ἐν αἷς διημερεύει διακούων οὐδὲν
ἧττον, κἂν ὥρα γένηται τῆς τοῦ σώματος θεραπείας·
αὕτη δ' ἐστὶν ἡ διὰ τῶν σκυταλίδων τρίψις· ἅμα γὰρ
27.60
καὶ διακούει καὶ τρίβεται τεττάρων περιστάντων τρι-  
βέων. Ἑτέρα δ' ἐστὶν ἡ ἐπὶ τὰς θυσίας ἔξοδος. Τρίτη
δ' ἐπὶ θήραν βακχική τις, κύκλῳ γυναικῶν περικεχυ-
μένων, ἔξωθεν δὲ τῶν δορυφόρων· παρεσχοίνισται
δ' ἡ ὁδός· τῷ δὲ παρελθόντι ἐντὸς μέχρι γυναικῶν θά-
27.65
νατος· προηγοῦνται δὲ τυμπανισταὶ καὶ κωδωνοφόροι.
Κυνηγετεῖ δ' ἐν μὲν τοῖς περιφράγμασιν ἀπὸ βήματος
τοξεύων· παρεστᾶσι δ' ἔνοπλοι δύο ἢ τρεῖς γυναῖκες·
ἐν δὲ ταῖς ἀφράκτοις θήραις ἀπ' ἐλέφαντος· αἱ δὲ γυναῖ-
κες αἱ μὲν ἐφ' ἁρμάτων, αἱ δ' ἐφ' ἵππων, αἱ δὲ καὶ ἐπ'
27.70
ἐλεφάντων, ὡς καὶ συστρατεύουσιν, ἠσκημέναι παντὶ
ὅπλῳ.
 [Ἔχει μὲν οὖν καὶ ταῦτα πολλὴν ἀήθειαν πρὸς τὰ
παρ' ἡμῖν· ἔτι μέντοι μᾶλλον τὰ τοιάδε.] Φησὶ γὰρ
τοὺς τὸν Καύκασον οἰκοῦντας ἐν τῷ φανερῷ γυναιξὶ
27.75
μίσγεσθαι, καὶ σαρκοφαγεῖν τὰ τῶν συγγενῶν σώματα.
Πετροκυλιστὰς δ' εἶναι κερκοπιθήκους κτλ.
28.1
 Athenaeus IV: <Μεγασθένης ἐν τῇ
δευτέρᾳ τῶν Ἰνδικῶν> τοῖς Ἰνδοῖς φησιν ἐν τῷ 

δείπνῳ παρατίθεσθαι ἑκάστῳ τράπεζαν, ταύτην δ' εἶ-
ναι ὁμοίαν ταῖς ἐγγυθήκαις· καὶ ἐπιτίθεσθαι ἐπ' αὐτῇ
28.5
τρυβλίον χρυσοῦν, εἰς ὃ ἐμβαλεῖν αὐτοὺς πρῶτον μὲν
τὴν ὄρυζαν ἑφθὴν, ὡς ἄν τις ἑψήσειε χόνδρον, ἔπειτα
ὄψα πολλὰ κεχειρουργημένα ταῖς Ἰνδικαῖς σκευασίαις.
29.1
 Strabo II: Ἅπαντες μὲν τοίνυν οἱ περὶ τῆς
Ἰνδικῆς γράψαντες ὡς ἐπὶ τὸ πολὺ ψευδολόγοι γεγό-
νασι, καθ' ὑπερβολὴν δὲ Δηίμαχος· τὰ δὲ δεύτερα λέγει
Μεγασθένης· Ὀνησίκριτος δὲ καὶ Νέαρχος καὶ ἄλλοι
29.5
τοιοῦτοι παραψελλίζοντες ἤδη. Καὶ ἡμῖν δ' ὑπῆρξεν ἐπὶ
πλέον κατιδεῖν ταῦτα, ὑπομνηματιζομένοις τὰς Ἀλε-
ξάνδρου πράξεις· διαφερόντως δ' ἀπιστεῖν ἄξιον Δηι-
μάχῳ τε καὶ Μεγασθένει. Οὗτοι γάρ εἰσιν οἱ τοὺς
Ἐνωτοκοίτας καὶ τοὺς Ἀστόμους καὶ Ἄρρινας ἱστοροῦν-
29.10
τες, Μονοφθάλμους τε καὶ Μακροσκελεῖς καὶ Ὀπισθο-
δακτύλους· ἀνεκαίνισαν δὲ καὶ τὴν Ὁμηρικὴν τῶν
Πυγμαίων γερανομαχίαν, τρισπιθάμους εἰπόντες· οὗτοι
δὲ καὶ τοὺς χρυσωρύχους μύρμηκας καὶ Πᾶνας σφηνο-
κεφάλους ὄφεις τε καὶ ἐλάφους σὺν κέρασι καταπίνον-
29.15
τας· περὶ ὧν ἕτερος τὸν ἕτερον ἐλέγχει, ὅπερ καὶ
Ἐρατοσθένης φησίν. Ἐπέμφθησαν μὲν γὰρ εἰς τὰ Πα-
λίμβοθρα, ὁ μὲν Μεγασθένης πρὸς Σανδρόκοττον, ὁ δὲ
Δηίμαχος πρὸς Ἀλλιτροχάδην τὸν ἐκείνου υἱὸν κατὰ
πρεσβείαν· ὑπομνήματα δὲ τῆς ἀποδημίας κατέλιπον
29.20
τοιαῦτα, ὑφ' ἡσδήποτε αἰτίας προαχθέντες. Πατρο-
κλῆς δὲ ἥκιστα τοιοῦτος· καὶ οἱ ἄλλοι δὲ μάρτυρες οὐκ
ἀπίθανοι, οἷς κέχρηται ὁ Ἐρατοσθένης.
30.1
 Strabo XV: Ὑπερεκπίπτων δ' ἐπὶ τὸ μυ-
θῶδες πεντασπιθάμους ἀνθρώπους λέγει καὶ τρισπιθά-
μους, ὧν τινας ἀμύκτηρας, ἀναπνοὰς ἔχοντας μόνον δύο
ὑπὲρ τοῦ στόματος· πρὸς δὲ τοὺς τρισπιθάμους πόλε-
30.5
μον εἶναι ταῖς γεράνοις (ὃν καὶ Ὅμηρον δηλοῦν) καὶ
τοῖς πέρδιξιν, οὓς χηνομεγέθεις εἶναι· τούτους δ' ἐκλέ-
γειν αὐτῶν τὰ ᾠὰ καὶ φθείρειν· ἐκεῖ γὰρ ᾠοτοκεῖν τὰς
γεράνους· διόπερ μηδαμοῦ μηδ' ᾠὰ εὑρίσκεσθαι γερά-
νων, μήτ' οὖν νεόττια· πλειστάκις δ' ἐκπίπτειν γέρανον  
30.10
χαλκῆν ἔχουσαν ἀκίδα ἀπὸ τῶν ἐκεῖθεν πληγμάτων.
Ὅμοια δὲ καὶ τὰ περὶ τῶν Ἐνωτοκοιτῶν καὶ τῶν ἀγρίων
ἀνθρώπων καὶ ἄλλων τερατωδῶν. Τοὺς μὲν οὖν ἀγρίους μὴ
κομισθῆναι παρὰ Σανδρόκοττον· ἀποκαρτερεῖν γάρ (*)·
ἔχειν δὲ τὰς μὲν πτέρνας πρόσθεν, τοὺς δὲ ταρσοὺς
30.15
ὄπισθεν καὶ τοὺς δακτύλους. Ἀστόμους δέ τινας ἀχθῆ-
ναι, ἀνθρώπους ἡμέρους· οἰκεῖν δὲ περὶ τὰς πηγὰς τοῦ
Γάγγου· τρέφεσθαι δ' ἀτμαῖς ὀπτῶν κρεῶν καὶ καρπῶν
καὶ ἀνθέων ὀσμαῖς, ἀντὶ τῶν στομάτων ἔχοντας ἀνα-
πνοάς· χαλεπαίνειν δὲ τοῖς δυσώδεσι, καὶ διὰ τοῦτο
30.20
περιγίνεσθαι μόλις καὶ μάλιστα ἐν στρατοπέδῳ. Περὶ
δὲ τῶν ἄλλων διηγεῖσθαι τοὺς φιλοσόφους Ὠκύποδάς τε
ἱστοροῦντας ἵππων μᾶλλον ἀπιόντας· Ἐνωτοκοίτας δὲ
ποδήρη τὰ ὦτα ἔχοντας, ὡς ἐγκαθεύδειν, ἰσχυροὺς δ'
ὥστ' ἀνασπᾶν δένδρα καὶ ῥήττειν νευράν· Μονομμά-
30.25
τους δὲ ἄλλους, ὦτα μὲν ἔχοντας κυνὸς, ἐν μέσῳ δὲ τῷ
μετώπῳ τὸν ὀφθαλμὸν, ὀρθοχαίτας, λασίους τὰ στήθη
τοὺς δὲ Ἀμύκτηρας εἶναι παμφάγους, ὠμοφάγους, ὀλι-
γοχρονίους, πρὸ γήρως θνήσκοντας· τοῦ δὲ στόματος τὸ
ἄνω προχειλότερον εἶναι πολύ. Περὶ δὲ τῶν χιλιετῶν
30.30
Ὑπερβορέων τὰ αὐτὰ λέγειν Σιμωνίδῃ καὶ Πινδάρῳ
καὶ ἄλλοις μυθολόγοις. [Μῦθος δὲ καὶ τὸ ὑπὸ Τιμαγέ-
νους λεχθὲν, ὡς ὅτι χαλκὸς ὕοιτο σταλαγμοῖς χαλκοῖς
καὶ σύροιτο. Ἐγγυτέρω δὲ πίστεώς φησιν ὁ <Μεγα-
σθένης,> ὅτι οἱ ποταμοὶ καταφέροιεν ψῆγμα χρυσοῦ, καὶ
30.35
ἀπ' αὐτοῦ φόρος ἀπάγοιτο τῷ βασιλεῖ· τοῦτο γὰρ καὶ
ἐν Ἰβηρίᾳ συμβαίνει.]
31.1
 Plinius VII, 2, 14: In monte, cui nomen est
Nulo, (Nullo v. l.), homines esse aversis plantis,
octonos digitos in singulis habentes, auctor est <Me-
gasthenes>. (15) In multis autem ontibus genus
31.5
hominum capitibus caninis ferarum pellibus velari,
pro voce latratum edere, unguibus armatum venatu
et aucupio vesci. [Horum supra centum viginti
millia fuisse prodente se Ctesias scribit,
et in quadam gente Indiae feminas semet in vita
31.10
parere genitosque confestim canescere, etc.]
32.1
 Solinus 52, 36: Ad montem, qui Nulo dicitur,
habitant, quibus adversae plantae sunt et octoni
digiti in plantis singulis. <Megasthenes> per di-
versos Indiae montes sees scribit nationes capitibus
32.5
caninis, armatas unguibus, amictas vestitu tergo-
rum, ad sermonem humanum nulla voce sed latra-
tibus tantum sonantes, asperis rictibus. [Apud
Ctesiam legitur, quasdam feminas ibi semel parere
natosque canos illico fieri, etc.]
33.1
 Plinius VII, 2, 18: <Megasthenes> gentem inter
Nomadas Indos narium loco foramina tantum ha-
bentem, anguium modo loripedem, vocari Scyritas.
Ad extremos fines Indiae ab oriente circa fontem
33.5
Gangis Astomorum gentem sine ore, corpore toto
hirtam vestiri frondium lanugine, halitu tantum
viventem et odore, quem naribus trahant. Nullum
iis cibum nullumque potum tantum radicum flo-
rumque varios odores et silvestrium malorum, quae   

33.10
secum portant itinere, ne desit olfactus:
graviore paulo odore haud difficulter exanimari.
 19. Supra hos extrema in parte montium Tri-
spithami Pygmaeique narrantur, ternas spithamas
longitudine, hoc est, ternos dodrantes non exce-
33.15
dentes, salubri caelo semperque vernante, montibus
ab aquilone oppositis: quos a gruibus infestari
Homerus quoque prodidit. Fama est, insidentes
arietum caprarumque dorsis, armatos sagittis veris
tempore universo agmine ad mare descendere et
33.20
ova pullosque earum alitum consumere: ternis ex-
peditionem eam mensibus confici, aliter futuris gre-
gibus non resisti. Casas eorum luto pennisque et
ovorum putaminibus construi. [Aristoteles in ca-
vernis vivere Pygmaeos tradit: cetera de his, ut
33.25
reliqui.]
 22. [Ctesias gentem ex his, quae appelle-
tur Pandore, in convallibus sitam, annos ducenos
vivere, in iuventa candido capillo, qui in sene-
ctute nigrescat. Contra alios quadragenos non
33.30
excedere annos iunctos Macrobiis, quorum feminae
semel pariant: idque et Agatharchides tradit;
praeterea locustis eos ali et esse pernices.] Man-
dorum nomen iis dedit Cli-
tarchus et <Megasthenes>, trecentosque eorum vi-
33.35
cos annumerat. Feminas septimo aetatis anno
parere, senectam quadragesimo accidere.
34.1
 Plutarch. De fac. in luna c. 24:
Τὴν μὲν γὰρ Ἰνδικὴν ῥίζαν, ἥν φησι <Μεγα-
σθένης> μήτ' ἐσθίοντας μήτε πίνοντας ἀλλ' ἀστόμους
ὄντας ὑποτύφειν καὶ θυμιᾶν καὶ τρέφεσθαι τῇ ὀσμῇ, πό-
34.5
θεν ἄν τις ἐκεῖ φυομένην λάβοι μὴ βρεχομένης τῆς
σελήνης;  
t35-43.1

E LIBRO TERTIO.

35.1
 Arrian. Ind. c. 11: Νενέμηνται δὲ οἱ πάντες
Ἰνδοὶ ἐς ἑπτὰ μάλιστα γενεάς· ἐν μὲν αὐτοῖσιν οἱ σο-
φισταί εἰσι, πλήθεϊ μὲν μείους τῶν ἄλλων, δόξῃ δὲ καὶ
τιμῇ γεραρώτατοι. (2) Οὔτε γάρ τι τῷ σώματι ἐργά-
35.5
ζεσθαι ἀναγκαίη σφὶν προσκέεται, οὔτε τι ἀποφέρειν
ἀπ' ὅτου πονέουσιν ἐς τὸ κοινόν· οὐδέ τι ἄλλο ἀνάγκης
ἁπλῶς ἐπεῖναι τοῖσι σοφιστῇσιν, ὅτι μὴ θύειν τὰς θυσίας
τοῖσι θεοῖσιν ὑπὲρ τοῦ κοινοῦ τῶν Ἰνδῶν· (3) καὶ
δὲ ἰδίᾳ θύει, ἐξηγητὴς αὐτῷ τῆς θυσίης τῶν τις σοφιστέων
35.10
τούτων γίνεται, ὡς οὐκ ἂν ἄλλως κεχαρισμένα τοῖς
θεοῖσι θύσαντας. (4) Εἰσὶ δὲ καὶ μαντικῆς οὗτοι μοῦνοι
Ἰνδῶν δαήμονες, οὐδὲ ἐφεῖται ἄλλῳ μαντεύεσθαι ὅτι μὴ
σοφῷ ἀνδρί. (5) Μαντεύουσι δὲ ὅσα ὑπὲρ τῶν ὡρέων
ἔτεος καὶ εἴ τις ἐς τὸ κοινὸν συμφορὴ καταλαμβάνει·
35.15
τὰ δὲ ἴδια ἑκάστοισιν οὔ σφιν μέλει μαντεύεσθαι, ἢ ὡς
οὐκ ἐξικνεομένης τῆς μαντικῆς ἐς τὰ σμικρότερα, ἢ ὡς
οὐκ ἄξιον ἐπὶ τούτοισι πονέεσθαι. (6) Ὅστις δὲ
ἐς τρὶς μαντευσάμενος, τούτῳ δὲ ἄλλο μὲν κακὸν γί-
νεσθαι οὐδὲν, σιωπᾶν δὲ εἶναι ἐπάναγκες τοῦ λοιποῦ·
35.20
καὶ οὐκ ἔστιν ὅστις ἐξαναγκάσει τὸν ἄνδρα τοῦτον φω-
νῆσαι, ὅτου ἡ σιωπὴ κατακέκριται. (7) Οὗτοι γυμνοὶ
διαιτῶνται οἱ σοφισταὶ, τοῦ μὲν χειμῶνος ὑπαίθριοι ἐν
τῷ ἡλίῳ, τοῦ δὲ θέρεος ἐπὴν ὁ ἥλιος κατέχῃ, ἐν τοῖσι  
λειμῶσι καὶ τοῖσιν ἕλεσιν ὑπὸ δένδρεσι μεγάλοισιν· ὧν
35.25
τὴν σκιὴν Νέαρχος λέγει ἐς πέντε πλέθρα ἐν κύκλῳ
ἐξικνέεσθαι, καὶ ἂν καὶ μυρίους ἀνθρώπους ὑπὸ ἑνὶ
δένδρεϊ σκιάζεσθαι· τηλικαῦτα εἶναι ταῦτα τὰ δένδρεα.
(8) Σιτέονται δὲ ὡραῖα καὶ τὸν φλοιὸν τῶν δένδρεων,
γλυκύν τε ὄντα τὸν φλοιὸν καὶ τρόφιμον οὐ μεῖον ἤπερ
35.30
αἱ βάλανοι τῶν φοινίκων.
 9. <Δεύτεροι> δ' ἐπὶ τούτοισιν οἱ γεωργοί εἰσιν·
πλήθεϊ πλεῖστοι Ἰνδῶν ἐόντες· καὶ τούτοισιν οὔτε ὅπλα
ἐστὶν ἀρήια οὔτε μέλει τὰ πολέμια ἔργα, ἀλλὰ τὴν
χώρην οὗτοι ἐργάζονται· καὶ τοὺς φόρους τοῖσι τε βασι-
35.35
λεῦσι καὶ τῇσι πόλισιν, ὅσαι αὐτόνομοι, οὗτοι ἀποφέ-
ρουσι· (10) καὶ εἰ πόλεμος ἐς ἀλλήλους τοῖσιν Ἰνδοῖσι
τύχοι, τῶν ἐργαζομένων τὴν γῆν οὐ θέμις σφὶν ἅπτεσθαι,
οὐδὲ αὐτὴν τὴν γῆν τάμνειν· ἀλλὰ οἱ μὲν πολεμέουσι καὶ
κατακαίνουσιν ἀλλήλους ὅπως τύχοιεν, οἱ δὲ πλησίον
35.40
αὐτῶν κατ' ἡσυχίην ἀροῦσιν ἢ τρυγῶσιν ἢ κλαδοῦσιν
ἢ θερίζουσιν.
 11. <Τρίτοι> δέ εἰσιν Ἰνδοῖσιν οἱ νομέες, οἱ
τε καὶ βουκόλοι, καὶ οὗτοι οὔτε κατὰ πόλιας οὔτε ἐν
τῇσι κώμῃσιν οἰκέουσι. Νομάδες τέ εἰσι καὶ ἀνὰ τὰ
35.45
οὔρεα βιοτεύουσι· φόρον δὲ καὶ οὗτοι ἀπὸ τῶν κτηνέων
ἀποφέρουσι· καὶ θηρεύουσιν οὗτοι ἀνὰ τὴν χώρην ὄρνι-
θάς τε καὶ ἄγρια θηρία.
 Cap. XII. <Τέταρτον> δέ ἐστι τὸ δημιουργικόν τε
καὶ καπηλικὸν γένος. Καὶ οὗτοι λειτουργοί εἰσι, καὶ
35.50
φόρον ἀποφέρουσιν ἀπὸ τῶν ἔργων τῶν σφετέρων, πλήν
γε δὴ ὅσοι τὰ ἀρήια ὅπλα ποιέουσιν· οὗτοι δὲ καὶ μισθὸν
ἐκ τοῦ κοινοῦ προσλαμβάνουσιν. Ἐν δὲ τούτῳ τῷ γένεϊ
οἵ τε ναυπηγοὶ καὶ οἱ ναῦταί εἰσιν, ὅσοι κατὰ τοὺς πο-
ταμοὺς πλώουσι.
35.55
 2. <Πέμπτον> δὲ γένος ἐστὶν Ἰνδοῖσιν οἱ πολεμισταὶ,
πλήθεϊ μὲν δεύτερον μετὰ τοὺς γεωργοὺς, πλείστῃ δὲ 


ἐλευθερίῃ τε καὶ εὐθυμίῃ ἐπιχρεόμενον· καὶ οὗτοι ἀσκη-
ταὶ μούνων τῶν πολεμικῶν ἔργων εἰσί. (3) Τὰ δὲ ὅπλα
ἄλλοι αὐτοῖσι ποιέουσι, καὶ ἵππους ἄλλοι παρέχουσι· καὶ
35.60
διακονέουσιν ἐπὶ στρατοπέδου ἄλλοι, οἳ τούς τε ἵππους
αὐτοῖσι θεραπεύουσι καὶ τὰ ὅπλα ἐκκαθαίρουσι καὶ τοὺς
ἐλέφαντας ἄγουσι καὶ τὰ ἅρματα κοσμέουσί τε καὶ
ἡνιοχεύουσιν. (4) Αὐτοὶ δὲ, ἔστ' ἂν μὲν πολεμέει
πολεμέουσιν, εἰρήνης δὲ γενομένης εὐθυμέονται· καί σφιν
35.65
μισθὸς ἐκ τοῦ κοινοῦ τοσόσδε ἔρχεται, ὡς καὶ ἄλλους
τρέφειν ἀφ' αὐτοῦ εὐμαρέως.
 5. <Ἕκτοι> δέ εἰσιν Ἰνδοῖσιν οἱ ἐπίσκοποι καλεόμε-
νοι. Οὗτοι ἐφορῶσι τὰ γινόμενα κατά τε τὴν χώρην καὶ
κατὰ τὰς πόλιας· καὶ ταῦτα ἀναγγέλλουσι τῷ βασιλέϊ,
35.70
ἵναπερ βασιλεύονται Ἰνδοὶ, ἢ τοῖσι τέλεσιν, ἵναπερ αὐ-
τόνομοι εἰσί· καὶ τούτοισιν οὐ θέμις ψεῦδος ἀγγεῖλαι οὐ-
δέν· οὐδέ τις Ἰνδῶν αἰτίην ἔσχε ψεύσασθαι.
 6. <Ἕβδομοι> δέ εἰσιν οἱ ὑπὲρ τῶν κοινῶν βουλευό-
μενοι ὁμοῦ τῷ βασιλέϊ, ἢ κατὰ τὰς πόλιας ὅσαι αὐτόνομοι
35.75
σὺν τῆσιν ἀρχῇσι. (7) Πλήθεϊ μὲν ὀλίγον τὸ γένος
ἐστι, σοφίῃ δὲ καὶ δικαιότητι ἐκ πάντων προκεκριμένον·
ἔνθεν οἵ τε ἄρχοντες αὐτοῖσιν ἐπιλέγονται καὶ ὅσοι νο-
μάρχαι καὶ ὕπαρχοι καὶ θησαυροφύλακές τε καὶ στρα-
τοφύλακες, ναύαρχοί τε καὶ ταμίαι, καὶ τῶν κατὰ
35.80
γεωργίην ἔργων ἐπιστάται.
 8. Γαμέειν δὲ ἐξ ἑτέρου γένεος οὐ θέμις, οἷον τοῖσι
γεωργοῖσιν ἐκ τοῦ δημιουργικοῦ, ἢ ἔμπαλιν· οὐδὲ δύο
τέχνας ἐπιτηδεύειν τὸν αὐτὸν, οὐδὲ τοῦτο θέμις· οὐδὲ
ἀμείβειν ἐξ ἑτέρου γένεος εἰς ἕτερον, οἷον γεωργικὸν ἐκ
35.85
νομέος γενέσθαι, ἢ νομέα ἐκ δημιουργικοῦ. (9) Μοῦνον  
σφίσιν ἀνεῖται σοφιστὴν ἐκ παντὸς γένεος γενέσθαι.
ὅτι οὐ μαλθακὰ τοῖσι σοφιστῇσιν εἰσὶ τὰ πρήγματα,
ἀλλὰ πάντων ταλαιπωρότατα.
36.1
 Strabo XV: (1) <Φησὶ> δὴ τὸ τῶν
Ἰνδῶν πλῆθος εἰς ἑπτὰ μέρη διῃρῆσθαι, καὶ πρώτους
μὲν τοὺς φιλοσόφους εἶναι κατὰ τιμὴν, ἐλαχίστους δὲ
κατ' ἀριθμόν· χρῆσθαι δ' αὐτοῖς ἰδίᾳ μὲν ἑκάστῳ τοὺς
36.5
θύοντας ἢ τοὺς ἐναγίζοντας, κοινῇ δὲ τοὺς βασιλέας
κατὰ τὴν μεγάλην λεγομένην σύνοδον, καθ' ἣν τοῦ
νέου ἔτους ἅπαντες οἱ φιλόσοφοι τῷ βασιλεῖ συνελθόντες
ἐπὶ θύρας, ὅ τι ἂν ἕκαστος αὐτῶν συντάξῃ τῶν χρη-
σίμων, ἢ τηρήσῃ πρὸς εὐετηρίαν καρπῶν τε καὶ ζώων
36.10
καὶ περὶ πολιτείας, προφέρει τοῦτο εἰς τὸ μέσον· ὃς
δ' ἂν τρὶς ἐψευσμένος ἁλῷ, νόμος ἐστὶ σιγᾶν διὰ
βίου· τὸν δὲ κατορθώσαντα ἄφορον καὶ ἀτελῆ κρί-
νουσι.
 2. Δεύτερον δὲ μέρος εἶναι τὸ τῶν γεωργῶν, οἳ πλεῖ-
36.15
στοί τέ εἰσι καὶ ἐπιεικέστατοι, οἱ ἐν ἀστρατείᾳ καὶ
ἀδείᾳ τοῦ ἐργάζεσθαι πόλει μὴ προσιόντες, μηδ' ἄλλῃ
χρείᾳ, μηδ' ὀχλήσει κοινῇ· πολλάκις γοῦν ἐν τῷ αὐτῷ
χρόνῳ καὶ τόπῳ, τοῖς μὲν παρατετάχθαι συμβαίνει καὶ
διακινδυνεύειν πρὸς τοὺς πολεμίους· οἱ δ' ἀροῦσιν,
36.20
ἢ σκάπτουσιν ἀκινδύνως, προμάχους ἔχοντες ἐκείνους.
Ἔστι δ' ἡ χώρα βασιλικὴ πᾶσα· μισθοῦ δ' αὐτὴν ἐπὶ
τετάρταις ἐργάζονται τῶν καρπῶν.
 3. Τρίτον τὸ τῶν ποιμένων καὶ θηρευτῶν, οἷς μό-
νοις ἔξεστι θηρεύειν καὶ θρεμματοτροφεῖν, ὤνιά τε πα-
36.25
ρέχειν καὶ μισθοῦ ζεύγη· ἀντὶ δὲ τοῦ τὴν γῆν ἐλευθε-
ροῦν θηρίων καὶ τῶν σπερμολόγων ὀρνέων μετροῦνται
παρὰ τοῦ βασιλέως σῖτον, πλάνητα καὶ σκηνίτην νεμό-
μενοι βίον.
 Περὶ μὲν οὖν τῶν θηρίων τοσαῦτα λέγεται· ἐπανιόν-
36.30
τες δ' ἐπὶ τὸν <Μεγασθένη> λέγωμεν τὰ ἑξῆς, ὧν ἀπε-
λίπομεν.
 4. Μετὰ γὰρ τοὺς θηρευτὰς καὶ τοὺς ποιμένας <τέ-
ταρτόν> φησιν εἶναι μέρος τοὺς ἐργαζομένους τὰς τέχνας
καὶ τοὺς καπηλικοὺς καὶ οἷς ἀπὸ τοῦ σώματος ἡ ἐργα-
36.35
σία· ὧν οἱ μὲν φόρον τελοῦσι καὶ λειτουργίας παρέ-
χονται τακτάς· τοῖς δ' ὁπλοποιοῖς καὶ ναυπηγοῖς μισθοὶ
καὶ τροφαὶ παρὰ βασιλέως ἔκκεινται· μόνῳ γὰρ ἐργά-
ζονται. Παρέχει δὲ τὰ μὲν ὅπλα τοῖς στρατιώταις
ὁ στρατοφύλαξ, τὰς δὲ ναῦς μισθοῦ τοῖς πλέουσιν
36.40
ὁ ναύαρχος καὶ τοῖς ἐμπόροις.
 5. <Πέμπτον> ἐστὶ τὸ τῶν πολεμιστῶν, οἷς τὸν ἄλλον
χρόνον ἐν σχολῇ καὶ πότοις ὁ βίος ἐστὶν, ἐκ τοῦ βασι-
λικοῦ διαιτωμένοις, ὥστε τὰς ἐξόδους, ὅταν εἴη χρεία,
ταχέως ποιεῖσθαι, πλὴν τῶν σωμάτων μηδὲν ἄλλο κο-
36.45
μίζοντας παρ' ἑαυτῶν.
 6. <Ἕκτοι> δ' εἰσὶν οἱ ἔφοροι· τούτοις δ'
δέδοται τὰ πραττόμενα, καὶ ἀναγγέλλειν λάθρα τῷ
βασιλεῖ, συνεργοὺς ποιουμένοις τὰς ἑταίρας, τοῖς μὲν
ἐν τῇ πόλει τὰς ἐν τῇ πόλει, τοῖς δὲ ἐν στρατοπέδῳ
36.50
τὰς αὐτόθι· καθίστανται δ' οἱ ἄριστοι καὶ πιστό-
τατοι.
 7. <Ἕβδομοι> δ' οἱ σύμβουλοι καὶ σύνεδροι τοῦ βασι-
λέως, ἐξ ὧν τὰ ἀρχεῖα καὶ δικαστήρια καὶ ἡ διοίκησις  
τῶν ὅλων. Οὐκ ἔστι δ' οὔτε γαμεῖν ἐξ ἄλλου γένους,
36.55
οὔτ' ἐπιτήδευμα οὔτ' ἐργασίαν μεταλαμβάνειν ἄλλην
ἐξ ἄλλης, οὐδὲ πλείους μεταχειρίζεσθαι τὸν αὐτὸν,
πλὴν εἰ τῶν φιλοσόφων τις εἴη· ἐᾶσθαι γὰρ τοῦτον δι'
ἀρετήν. 

 Strabo XV: Τῶν δ' ἀρχόντων οἱ μέν
εἰσιν ἀγορανόμοι, οἱ δ' ἀστυνόμοι, οἱ δ' ἐπὶ τῶν στρατιω-
τῶν· ὧν οἱ μὲν ποταμοὺς ἐξεργάζονται, καὶ ἀναμετροῦσι
τὴν γῆν, ὡς ἐν Αἰγύπτῳ, καὶ τὰς κλειστὰς διώρυχας
36a.5
ἀφ' ὧν εἰς τὰς ὀχετείας ταμιεύεται τὸ ὕδωρ, ἐπισκο-
ποῦσιν ὅπως ἐξ ἴσης πᾶσιν ἡ τῶν ὑδάτων παρείη χρῆσις.
Οἱ δ' αὐτοὶ καὶ τῶν θηρευτῶν ἐπιμελοῦνται, καὶ τιμῆς
καὶ κολάσεως εἰσὶ κύριοι τοῖς ἐπαξίοις· καὶ φορολογοῦσι
δὲ, καὶ τὰς τέχνας τὰς περὶ τὴν γῆν ἐπιβλέπουσιν,
36a.10
ὑλοτόμων, τεκτόνων, χαλκέων, μεταλλευτῶν· ὁδοποιοῦσι
δὲ, καὶ κατὰ δέκα στάδια στήλην τιθέασι, τὰς ἐκτροπὰς
καὶ τὰ διαστήματα δηλοῦσαν. Οἱ δ' ἀστυνόμοι εἰς ἓξ
πεντάδας διῄρηνται· καὶ οἱ μὲν τὰ δημιουργικὰ σκοποῦ-
σιν, οἱ δὲ ξενοδοχοῦσιν· καὶ γὰρ καταγωγὰς νέμουσι,
36a.15
καὶ τοῖς βίοις παρακολουθοῦσι, παρέδρους δόντες·
καὶ προπέμπουσιν ἢ αὐτοὺς, ἢ τὰ χρήματα τῶν ἀποθα-
νόντων· νοσούντων τε ἐπιμελοῦνται, καὶ ἀποθανόν-
τας θάπτουσι. Τρίτοι δ' εἰσὶν, οἳ τὰς γενέσεις καὶ
θανάτους ἐξετάζουσι, πότε καὶ πῶς, τῶν τε φόρων
36a.20
χάριν, καὶ ὅπως μὴ ἀφανεῖς εἶεν αἱ κρείττους καὶ χεί-
ρους γοναὶ, καὶ θάνατοι. Τέταρτοι οἱ περὶ τὰς καπη-
λείας καὶ μεταβολάς· οἷς μέτρων μέλει καὶ τῶν ὡραίων,
ὅπως ἀπὸ συσσήμου πωλοῖτο. Οὐκ ἔστι δὲ πλείω τὸν
αὐτὸν μεταβάλλεσθαι, πλὴν εἰ διττοὺς ὑποτελοίη φό-
36a.25
ρους. Πέμπτοι δ' οἱ προεστῶτες τῶν δημιουργουμένων,
καὶ πωλοῦντες ταῦτα ἀπὸ συσσήμου, χωρὶς μὲν τὰ
καινὰ, χωρὶς δὲ τὰ παλαιά· τῷ μιγνύντι δὲ ζημία.
Ἕκτοι δὲ καὶ ὕστατοι οἱ τὰς δεκάτας ἐκλέγοντες τῶν
πωλουμένων· θάνατος δὲ τῷ κλέψαντι τὸ τέλος. Ἰδίᾳ
36a.30
μὲν ἕκαστοι ταῦτα, κοινῇ δ' ἐπιμελοῦνται τῶν τε
ἰδίων καὶ τῶν πολιτικῶν καὶ τῆς τῶν δημοσίων
ἐπισκευῆς τιμῶν τε καὶ ἀγορᾶς καὶ λιμένων, καὶ ἱερῶν.
 Μετὰ δὲ τοὺς ἀστυνόμους τρίτη ἐστὶ συναρχία <ἡ περὶ
τὰ στρατιωτικὰ>, καὶ αὕτη ταῖς πεντάσιν ἑξαχῇ
36a.35
διωρισμένη· ὧν <τὴν μὲν> μετὰ τοῦ ναυάρχου τάττουσι,
<τὴν δὲ> μετὰ τοῦ ἐπὶ τῶν βοϊκῶν ζευγῶν, δι' ὧν ὄρ-
γανα κομίζεται καὶ τροφὴ αὐτοῖς τε καὶ τοῖς κτήνεσι
καὶ τὰ χρήσιμα τῆς στρατείας. Οὗτοι δὲ καὶ τοὺς δια-
κόνους παρέχουσι τυμπανιστὰς καὶ (τοὺς) κωδωνοφό-
36a.40
ρους, ἔτι δὲ καὶ ἱπποκόμους καὶ μηχανοποιοὺς καὶ
τοὺς τούτων ὑπηρέτας· ἐκπέμπουσί τε πρὸς κώδωνας
τοὺς χορτολόγους, τιμῇ καὶ κολάσει τὸ τάχος κα-
τασκευαζόμενοι καὶ τὴν ἀσφάλειαν. <Τρίτοι> δέ εἰσιν
οἱ τῶν πεζῶν ἐπιμελούμενοι· <τέταρτοι> δ' οἱ τῶν  
36a.45
ἵππων· <πέμπτοι> δ' ἁρμάτων· <ἕκτοι> δὲ ἐλεφάν-
των. Βασιλικοί τε σταθμοὶ καὶ ἵπποις καὶ θηρίοις· βα-
σιλικὸν δὲ καὶ ὁπλοφυλάκιον· παραδίδωσι γὰρ ὁ στρα-
τιώτης τήν τε σκευὴν εἰς τὸ ὁπλοφυλάκιον καὶ τὸν
ἵππον εἰς τὸν ἱππῶνα καὶ τὸ θηρίον ὁμοίως. Χρῶνται
36a.50
δ' ἀχαλινώτοις· τὰ δ' ἅρματα ἐν ταῖς ὁδοῖς βόες ἕλκουσιν·
οἱ δὲ ἵπποι ἀπὸ φορβιᾶς ἄγονται τοῦ μὴ παρεμπίπρα-
σθαι τὰ σκέλη μηδὲ τὸ πρόθυμον αὐτῶν ὑπὸ τοῖς ἅρ-
μασιν ἀμβλύνεσθαι. Δύο δ' εἰσὶν ἐπὶ τῷ ἅρματι παρα-
βάται πρὸς ἡνιόχῳ· ὁ δὲ τοῦ ἐλέφαντος ἡνίοχος τέταρτος,
36a.55
τρεῖς δ' οἱ ἀπ' αὐτοῦ τοξεύοντες.
36b.1
 Aelianus XIII, 9:
 Ἵππον δὲ ἄρα Ἰνδὸν
κατασχεῖν καὶ ἀνακροῦσαι προπηδῶντα καὶ ἐκθέοντα
οὐ παντὸς ἦν, ἀλλὰ τῶν ἐκ παιδὸς ἱππείαν πεπαιδευ-
36b.5
μένων. Τοῦτο γὰρ αὐτοῖς ἔστιν ἐν ἔθει, χαλινῷ ἄρχειν
αὐτῶν καὶ ῥυθμίζειν αὐτοὺς καὶ ἰθύνειν· κημοῖς δὲ ἄρα
κεντρωτοῖς ἀκόλαστόν τε ἔχουσι τὴν γλῶτταν, καὶ τὴν
ὑπερῴαν ἀβασάνιστον· ἀναγκάζουσι δὲ αὐτοὺς ὅμως
οἵδε οἱ τὴν ἱππείαν σοφισταὶ περικυκλεῖν καὶ περιδι-
36b.10
νεῖσθαι ἐς ταὐτὸν στρεφομένους, καὶ ᾗπερ εἶδον ἀστό-
μους. Δεῖ δὲ ἄρα τῷ τοῦτο δράσοντι καὶ ῥώμης χειρῶν,
καὶ ἐπιστήμης εὖ μάλα ἱππικῆς. Πειρῶνται δὲ οἱ προή-
κοντες εἰς ἄκρον τῆσδε τῆς σοφίας καὶ ἅρμα οὕτως πε-
ρικυκλεῖν καὶ περιάγειν· εἴη δ' ἂν ἆθλος οὐκ εὐκαταφρό-
36b.15
νητος, ἀδηφάγων ἵππων τέτρωρον περιστρέφειν ῥᾳδίως.
Φέρει δὲ τὸ ἅρμα παραβάτας καὶ δύο. Ὁ δὲ στρατιώτης
ἐλέφας ἐπὶ τοῦ καλουμένου θωρακίου, ἢ καὶ νὴ Δία τοῦ
νώτου γυμνοῦ καὶ ἐλευθέρου, πολεμιστὰς μὲν τρεῖς,
παρ' ἑκάτερα βάλλοντας, καὶ τὸν τρίτον κατόπιν· τέ-
36b.20
ταρτον δὲ τὸν τὴν ἅρπην κατέχοντα διὰ χειρῶν, καὶ
ἐκείνῃ τὸν θῆρα ἰθύνοντα, ὡς οἴακι ναῦν κυβερνητικὸν
ἄνδρα καὶ ἐπιστάτην τῆς νεώς.
37.1
 Strabo XV, 704:
Ἵππον δὲ καὶ ἐλέφαντα τρέφειν οὐκ ἔξεστιν
ἰδιώτῃ· βασιλικὸν δ' ἑκάτερον νενόμισται τὸ κτῆμα, καὶ
εἰσὶν αὐτῶν ἐπιμεληταί.
37.5
 Θήρα δὲ τῶν θηρίων τούτων τοιάδε. Χωρίον ψιλὸν
ὅσον τεττάρων ἢ πέντε σταδίων τάφρῳ περιχαράξαντες
βαθείᾳ, γεφυροῦσι τὴν εἴσοδον στενωτάτῃ γεφύρᾳ· εἶτ'
εἰσαφιᾶσι θηλείας τὰς ἡμερωτάτας τρεῖς ἢ τέτταρας·
αὐτοὶ δ' ἐν καλυβίοις κρυπτοῖς ὑποκάθηνται λοχῶντες·
37.10
ἡμέρας μὲν οὖν οὐ προσίασιν οἱ ἄγριοι· νύκτωρ δ' ἐφ'
ἕνα ποιοῦνται τὴν εἴσοδον· εἰσιόντων δὲ, κλείουσι τὴν
εἴσοδον λάθρα· εἶτα τῶν ἡμέρων ἀθλητῶν τοὺς ἀλκιμω-
τάτους εἰσάγοντες, διαμάχονται πρὸς αὐτοὺς, ἅμα καὶ
λιμῷ καταπονοῦντες· ἤδη δὲ καμνόντων οἱ εὐθαρσέστατοι 

37.15
τῶν ἡνιόχων λάθρα καταβαίνοντες, ὑποδύνουσιν ἕκαστος
τῇ γαστρὶ τοῦ οἰκείου ὀχήματος· ὁρμώμενος δ' ἐνθένδε,
ὑποδύνει τῷ ἀγρίῳ, καὶ σύμποδα δεσμεῖ· γενομένου δὲ
τούτου, κελεύουσι τοῖς τιθασοῖς, τύπτειν τοὺς συμπο-
δισθέντας, ἕως ἂν πέσωσιν εἰς τὴν γῆν· πεσόντων
37.20
δ' ὠμοβοΐνοις ἱμᾶσι προσλαμβάνονται τοὺς αὐχένας
αὐτῶν πρὸς τοὺς τῶν τιθασῶν· ἵνα δὲ μὴ σειόμενοι τοὺς
ἀναβαίνειν ἐπ' αὐτοὺς ἐπιχειροῦντας ἀποσείοιντο, τοῖς
τραχήλοις αὐτῶν ἐμβάλλονται κύκλῳ τομὰς, καὶ κατ'
αὐτὰς τοὺς ἱμάντας περιτιθέασιν, ὥσθ' ὑπ' ἀλγηδόνων
37.25
εἴκειν τοῖς δεσμοῖς, καὶ ἡσυχάζειν· τῶν δ' ἁλόντων ἀπο-
λέξαντες τοὺς πρεσβυτέρους ἢ νεωτέρους τῆς χρείας,
τοὺς λοιποὺς ἀπάγουσιν εἰς τοὺς σταθμούς· δήσαντες
δὲ τοὺς μὲν πόδας πρὸς ἀλλήλους, τοὺς δὲ αὐχένας  
πρὸς κίονα εὖ πεπηγότα, δαμάζουσι λιμῷ· ἔπειτα χλόῃ
37.30
καλάμου καὶ πόας ἀναλαμβάνουσι· μετὰ δὲ ταῦτα πει-
θαρχεῖν διδάσκουσι, τοὺς μὲν διὰ λόγου, τοὺς δὲ με-
λισμῷ τινι καὶ τυμπανισμῷ κηλοῦντες· σπάνιοι δ' οἱ
δυστιθάσευτοι· φύσει γὰρ διάκεινται πρᾴως καὶ ἡμέ-
ρως, ὥστ' ἐγγὺς εἶναι λογικῷ ζῴῳ· τινὲς γὰρ καὶ
37.35
ἐξαίμους τοὺς ἡνιόχους ἐν τοῖς ἀγῶσι πεσόντας ἀνελό-
μενοι σώζουσιν ἒκ τῆς μάχης· οἱ δὲ καὶ ὑποδύντας με-
ταξὺ τῶν ἐμπροσθίων ποδῶν ὑπερμαχόμενοι διέσωσαν·
τῶν δὲ χορτοφόρων καὶ διδασκάλων εἴ τινα παρὰ θυμὸν
ἀπέκτειναν, οὕτως ἐπιποθοῦσιν, ὥσθ' ὑπ' ἀνίας ἀπέ-
37.40
χεσθαι τροφῆς· ἔστι δ' ὅτε καὶ ἀποκαρτερεῖν.
 Βιβάζονται δὲ καὶ τίκτουσιν ὡς ἵπποι τοῦ ἔαρος μά-
λιστα· καιρὸς δ' ἐστὶ τῷ μὲν ἄρρενι, ἐπειδὰν οἴστρῳ
κατέχηται καὶ ἀγριαίνῃ· τότε δὴ καὶ λίπους τι διὰ τῆς
ἀναπνοῆς ἀνίησιν, ἣν ἔχει παρὰ τοὺς κροτάφους· ταῖς
37.45
δὲ θηλείαις, ὅταν ὁ αὐτὸς οὗτος πόρος ἀνεωγὼς τυγχάνῃ.
Κύουσι δὲ τοὺς μὲν πλείστους ὀκτωκαίδεκα μῆνας,
ἐλαχίστους δ' ἑκκαίδεκα· τρέφει δ' ἡ μήτηρ ἓξ ἔτη.
Ζῶσι δ' ὅσον μακροβιώτατοι ἄνθρωποι οἱ πολλοὶ, τινὲς
δὲ καὶ ἐπὶ διακόσια διατείνουσιν ἔτη, πολύνοσοι δὲ
37.50
καὶ δυσίατοι. Ἄκος δὲ πρὸς ὀφθαλμίαν μὲν βόειον γάλα
προσκλυζόμενον, τοῖς πλείστοις δὲ τῶν νοσημάτων
ὁ μέλας οἶνος πινόμενος, τραύμασι δὲ ποτὸν μὲν βούτυ-
ρον, ἐξάγει γὰρ τὰ σιδήρια· τὰ δ' ἕλκη σαρξὶν ὑείαις
πυριῶσιν.
38a.1
 Arrianus Ind. c. 13: Θηρῶσι
δὲ Ἰνδοὶ τὰ μὲν ἄλλα ἄγρια θηρία, κατάπερ καὶ Ἕλ-
ληνες· ἡ δὲ τῶν ἐλεφάντων σφὶν θήρη οὐδέν τι ἄλλῃ
ἔοικεν, ὅτι καὶ ταῦτα τὰ θηρία οὐδαμοῖσιν ἄλλοισι θη-
38a.5
ρίοισιν ἐπέοικεν. (2) Ἀλλὰ τόπον γὰρ ἐπιλεξάμενοι
καὶ καυματώδεα ἐν κύκλῳ τάφρον ὀρύσσουσιν, ὅσον
μεγάλῳ στρατοπέδῳ ἐπαυλίσασθαι. Τῆς δὲ τάφρου τὸ
εὖρος ἐς πέντε ὀργυιὰς ποιέονται, βάθος τε ἐς τέσσαρας.
(3) Τὸν δὲ χόον ὅντινα ἐκβάλλουσιν ἐκ τοῦ ὀρύγματος,
38a.10
ἐπὶ τὰ χείλεα ἑκάτερα τῆς τάφρου ἐπιφορήσαντες, ἀντὶ
τείχεος διαχρέονται· (4) αὐτοὶ δὲ ἐπὶ τῷ χώματι τοῦ
ἐπιχειλέος τοῦ ἔξω τῆς τάφρου σκηνάς σφιν ὀρυκτὰς
ποιέονται, καὶ διὰ τουτέων ὀπὰς ὑπολείπονται· δι' ὧν
φῶς τε αὐτοῖσι συνεισέρχεται καὶ τὰ θηρία προσάγοντα
38a.15
καὶ ἐσελαύνοντα ἐς τὸ ἕρκος σκέπτονται. (5) Ἐνταῦθα
ἐντὸς τοῦ ἕρκεος καταστήσαντες τῶν τινας θηλέων τρεῖς
ἢ τέσσαρας, ὅσαι μάλιστα τὸν θυμὸν χειροήθεες, μίαν
εἴσοδον ἀπολιμπάνουσι κατὰ τὴν τάφρον, γεφυρώσαντες
τὴν τάφρον· καὶ ταύτῃ χόον τε καὶ ποίην πολλὴν ἐπιφέ-
38a.20
ρουσι τοῦ μὴ ἀρίδηλον εἶναι τοῖσι θηρίοισι τὴν γέφυραν,
μή τινα δόλον ὀϊσθῶσιν. (6) Αὐτοὶ μὲν οὖν ἐκποδὼν
σφᾶς ἔχουσι κατὰ τῶν σκηνέων τῶν ὑπὸ τῇ τάφρῳ δε-
δυκότες. Οἱ δὲ ἄγριοι ἐλέφαντες ἡμέρης μὲν οὐ πελά-
ζουσι τοῖσιν οἰκεομένοισι, νύκτωρ δὲ πλανῶνταί τε
38a.25
πάντη καὶ ἀγεληδὸν νέμονται τῷ μεγίστῳ καὶ γενναιο-
τάτῳ σφῶν ἑπόμενοι, κατάπερ αἱ βόες τοῖσι ταύροισιν.
(7) Ἐπεὰν ὦν τῷ ἕρκεϊ πελάσωσι, τήν τε φωνὴν
τῶν θηλέων καὶ τῇ ὀδμῇ αἰσθανόμενοι, δρόμῳ ἵενται
ὡς ἐπὶ τὸν χῶρον τὸν πεφραγμένον· ἐκπεριελθόντες δὲ
38a.30
τῆς τάφρου τὰ χείλεα εὖτ' ἂν τῇ γεφύρῃ ἐπιτύχωσι,
κατὰ ταύτην ἐς τὸ ἕρκος ὠθέονται. (8) Οἱ δὲ ἄνθρωποι
αἰσθόμενοι τὴν ἔσοδον τῶν ἐλεφάντων τῶν ἀγρίων, οἱ  
μὲν αὐτῶν τὴν γέφυραν ὀξέως ἀφεῖλον, οἱ δὲ ἐπὶ τὰς
πέλας κώμας ἀποδραμόντες ἀγγέλλουσι τοὺς ἐλέφαντας
38a.35
ὅτι ἐν τῷ ἕρκεϊ ἔχονται· (9) οἱ δὲ ἀκούσαντες ἐπιβαί-
νουσι τῶν κρατίστων τε τὸν θυμὸν καὶ τῶν χειροηθε-
στάτων ἐλεφάντων, ἐπιβάντες δὲ ἐλαύνουσιν ὡς ἐπὶ τὸ
ἕρκος, ἐλάσαντες δὲ οὐκ αὐτίκα μάχης ἅπτονται, ἀλλ'
ἐῶσι γὰρ λιμῷ τε ταλαιπωρηθῆναι τοὺς ἀγρίους ἐλέ-
38a.40
φαντας καὶ ὑπὸ τῷ δίψεϊ δουλωθῆναι. (10) Εὖτ' ἂν δὲ
σφίσι κακῶς ἔχειν δοκέωσι, τηνικαῦτα ἐπιστήσαντες
αὖθις τὴν γέφυραν ἐλαύνουσί τε ὡς ἐς τὸ ἕρκος, καὶ τὰ
μὲν πρῶτα μάχη ἵσταται κρατερὴ τοῖσιν ἡμέροισι τῶν
ἐλεφάντων πρὸς τοὺς ἑαλωκότας· ἔπειτα κρατέονται
38a.45
μὲν κατὰ τὸ εἰκὸς οἱ ἄγριοι ὑπό τε τῇ ἀθυμίῃ καὶ τῷ
λιμῷ ταλαιπωρεύμενοι. (11) Οἱ δὲ ἀπὸ τῶν ἐλεφάντων
καταβάντες παρειμένοισιν ἤδη τοῖσιν ἀγρίοισι τοὺς
πόδας ἄκρους συνδέουσιν· ἔπειτα ἐγκελεύονται τοῖσιν
ἡμέροισι πληγῇσι σφᾶς κολάζειν πολλῇσι, ἔστ' ἂν
38a.50
ἐκεῖνοι ταλαιπωρεύμενοι ἐς γῆν πέσωσι· παραστάντες 

δὲ βρόχους περιβάλλουσιν αὐτοῖσι κατὰ τοὺς αὐχένας,
καὶ αὐτοὶ ἐπιβαίνουσι κειμένοισι. (12) Τοῦ δὲ μὴ ἀπο-
σείεσθαι τοὺς ἀμβάτας μηδέ τι ἄλλο ἀτάσθαλον ἐργά-
ζεσθαι, τὸν τράχηλον αὐτοῖσιν ἐν κύκλῳ μαχαιρίῳ ὀξέϊ
38a.55
ἐπιτέμνουσι, καὶ τὸν βρόχον κατὰ τὴν τομὴν περι-
δέουσιν, ὡς ἀτρέμα ἔχειν τὴν κεφαλήν τε καὶ τὸν τρά-
χηλον ὑπὸ τοῦ ἕλκεος. (13) Εἰ γὰρ περιστρέφοιντο ὑπὸ
ἀτασθαλίης, τρίβεται αὐτοῖσι τὸ ἕλκος ὑπὸ τῷ κάλῳ·
οὕτω μὲν ὦν ἀτρέμα ἴσχουσι, καὶ αὐτοὶ γνωσιμαχέοντες
38a.60
ἤδη ἄγονται κατὰ τὸν δεσμὸν πρὸς τῶν ἡμέρων.
 Cap. XIV. Ὅσοι δὲ νήπιοι αὐτῶν ἢ διὰ κακότητα
οὐκ ἄξιοι ἐκτῆσθαι, τούτους ἐῶσιν ἀπαλλάττεσθαι ἐς
τὰ σφέτερα ἤθεα. (2) Ἄγοντες δὲ εἰς τὰς κώμας τοὺς
ἁλόντας τοῦ τε χλωροῦ καλάμου καὶ τῆς ποίης τὰ πρῶτα
38a.65
ἐμφαγεῖν ἔδοσαν· (3) οἱ δὲ ὑπὸ ἀθυμίης οὐκ ἐθέλουσιν
οὐδὲν σιτέεσθαι, τοὺς δὲ περιϊστάμενοι οἱ Ἰνδοὶ ᾠδαῖσί
τε καὶ τυμπάνοισι καὶ κυμβάλοισιν ἐν κύκλῳ κρούοντές
τε καὶ ἐπᾴδοντες κατευνάζουσι. (4) Θυμόσοφον γὰρ
εἴπερ τι ἄλλο θηρίον ὁ ἐλέφας· καί τινες ἤδη αὐτῶν τοὺς
38a.70
ἀμβάτας σφῶν ἐν πολέμῳ ἀποθανόντας ἄραντες αὐτοὶ
ἐξήνεγκαν ἐς ταφὴν, οἱ δὲ καὶ ὑπερήσπισαν κειμένους,
οἱ δὲ καὶ πεσόντων προεκινδύνευσαν· ὁ δέ τις πρὸς ὀρ-
γὴν ἀποκτείνας τὸν ἀμβάτην ὑπὸ μετανοίης τε καὶ
ἀθυμίης ἀπέθανεν. (5) [Εἶδον δὲ ἔγωγε καὶ
38a.75
ἤδη ἐλέφαντα καὶ ἄλλους ὀρχεομένους, κυμβάλοιν τῷ
κυμβαλίζοντι πρὸς τοῖν σκελοῖν τοῖν ἔμπροσθεν προς-
ηρτημένοιν, καὶ πρὸς τῇ προβοσκίδι καλεομένῃ ἄλλο
κύμβαλον· (6) ὁ δὲ ἐν μέρεϊ τῇ προβοσκίδι ἔκρουε τὸ
κύμβαλον ἐν ῥυθμῷ πρὸς ἑκατέροιν τοῖν σκελοῖν· οἱ δὲ
38a.80
ὀρχεόμενοι ἐν κύκλῳ τε ἐχόρευον, καὶ ἐπαίροντές τε καὶ
ἐπικάμπτοντες τὰ ἔμπροσθεν σκέλεα ἐν τῷ μέρεϊ ἐν
ῥυθμῷ καὶ οὗτοι ἔβαινον, κατότι ὁ κυμβαλίζων σφίσιν
ὑφηγέετο]. (7) Βαίνεται δὲ ἐλέφας ἦρος ὥρῃ, κατάπερ
βοῦς ἢ ἵππος, ἐπεὰν τῇσι θηλέῃσιν αἱ παρὰ τοῖσι κρο-
38a.85
τάφοισιν ἀναπνοαὶ ἀνοιχθεῖσαι ἐκπνέωσι· κύει δὲ τοὺς
ἐλαχίστους μὲν ἑκκαίδεκα μῆνας, τοὺς πλείστους δὲ
ὀκτωκαίδεκα· τίκτει δὲ ἓν, κατάπερ ἵππος· καὶ τοῦτο
ἐκτρέφει τῷ γάλακτι ἐς ἔτος ὄγδοον. (8) Ζῶσι δὲ
φάντων οἱ πλεῖστα ἔτεα ζῶντες ἐς διηκόσια· πολλοὶ δὲ
38a.90
νούσῳ προτελευτῶσιν· γήραϊ δὲ ἐς τόσον ἔρχονται. (9)
ἔστιν αὐτοῖσι τῶν μὲν ὀφθαλμῶν ἴημα τὸ βόειον γάλα
ἐγχεόμενον, πρὸς δὲ τὰς ἄλλας νούσους ὁ μέλας οἶνος
πινόμενος, ἐπὶ δὲ τοῖσιν ἕλκεσι τὰ ὕεια κρέα ὀπτώμενα
καὶ καταπασσόμενα. Ταῦτα παρ' Ἰνδοῖσίν ἐστιν αὐτοῖσιν
38a.95
ἰήματα.  
38b.1
 Aelianus N. A. XII, 44: Ἐν Ἰνδοῖς ἂν ἁλῷ τέλειος
ἐλέφας, ἡμερωθῆναι χαλεπός ἐστι, καὶ τὴν ἐλευθερίαν
ποθῶν φονᾷ· ἐὰν δὲ αὐτὸν καὶ δεσμοῖς διαλάβῃς, ἔτι καὶ
μᾶλλον ἐς τὸν θυμὸν ἐξάπτεται, καὶ δεσπότην οὐχ ὑπο-
38b.5
νέμει. Ἀλλ' οἱ Ἰνδοὶ καὶ ταῖς τροφαῖς κολακεύουσιν αὐ-
τὸν, καὶ ποικίλοις καὶ ἐφολκοῖς δελέασι πραΰνειν πει-
ρῶνται, παρατιθέντες, ὡς πληροῦν τὴν γαστέρα καὶ
θέλγειν τὸν θυμόν· ὁ δὲ ἄχθεται αὐτοῖς, καὶ ὑπερορᾷ·
Τί οὖν ἐκεῖνοι κατασοφίζονται καὶ δρῶσι; Μοῦσαν αὐ-
38b.10
τοῖς προσάγουσιν ἐπιχώριον, καὶ κατᾴδουσιν αὐτοὺς
ὀργάνῳ τινὶ καὶ τούτῳ συνήθει· καλεῖται δὲ σκινδαψὸς
τὸ ὄργανον· ὁ δὲ ὑπέχει τὰ ὦτα καὶ θέλγεται, καὶ ἡ
μὲν ὀργὴ πραΰνεται, ὁ δὲ θυμὸς ὑποστέλλεταί τε καὶ
θόρνυται, κατὰ μικρὰ δὲ καὶ ἐς τὴν τροφὴν ὁρᾷ· εἶτα
38b.15
ἀφεῖται μὲν τῶν δεσμῶν, μένει δὲ τῇ μούσῃ δεδεμένος,
καὶ δειπνεῖ προθύμως ἁβρὸς δαιτυμὼν καταδεδεμένος·
πόθῳ γὰρ τοῦ μέλους οὐκ ἂν ἔτι ἀποσταίη.
38c.1
 Idem XIII, 6: Τῶν τεθηραμένων ἐλεφάντων ἰῶν-
ται τὰ τραύματα οἱ Ἰνδοὶ τὸν τρόπον τοῦτον. Καταιο-
νοῦσι μὲν αὐτὰ ὕδατι χλιαρῷ, ὥσπερ οὖν τὸ τοῦ Εὐρυ-
πύλου παρὰ τῷ καλῷ Ὁμήρῳ ὁ Πάτροκλος· εἶτα μέν-
38c.5
τοι διαχρίουσι τῷ βουτύρῳ αὐτά· ἐὰν δὲ ᾖ βαθέα, τὴν
φλεγμονὴν πραΰνουσιν, ὕεια κρέα, θερμὰ μὲν, ἔναιμα
δὲ ἔτι, προσφέροντες καὶ ἐντιθέντες. Τὰς δὲ ὀφθαλμίας
θεραπεύουσιν αὐτῶν, βόειον γάλα ἀλεαίνοντες, εἶτα αὐ-
τοῖς ἐγχέοντες· οἱ δὲ ἀνοίγουσι τὰ βλέφαρα καὶ ὠφε-
38c.10
λούμενοι ἥδονταί τε καὶ αἰσθάνονται ὥσπερ ἄνθρωποι·
καὶ εἰς τοσοῦτον ἐπικλύζουσιν, εἰς ὅσον ἂν ὑποπαύσων-
ται λημῶντες· μαρτύριον δὲ τοῦ παύσασθαι τὴν ὀφθαλ-
μίαν τοῦτό ἐστι. Τὰ δὲ νοσήματα ὅσα αὐτοῖς προσπί-
πτει ἄλλως, ὁ μέλας οἶνός ἐστιν αὐτοῖς ἄκος· εἰ δὲ μὴ
38c.15
γένοιτο ἐξάντης τοῦ κακοῦ τῷ φαρμάκῳ τῷδε, ἄσωστά
οἱ ἐστιν.
39a.1
 Strabo XV:
Μεγασθένης δὲ περὶ τῶν μυρμήκων οὕτω
φησὶν, ὅτι ἐν Δέρδαις, ἔθνει μεγάλῳ τῶν προσεῴων καὶ
ὀρεινῶν Ἰνδῶν, ὀροπέδιον εἴη τρισχιλίων πως τὸν κύ-
39a.5
κλον σταδίων· ὑποκειμένων δὲ τούτῳ χρυσωρυχείων, οἱ
μεταλλεύοντες εἶεν μύρμηκες, θηρίων ἀλωπέκων οὐκ
ἐλάττους, τάχος ὑπερφυὲς ἔχοντες, καὶ ζῶντες ἀπὸ θή-
ρας. Ὀρύττουσι δὲ χειμῶνι τὴν γῆν· σωρεύουσί τε
πρὸς τοῖς στομίοις, καθάπερ οἱ ἀσπάλακες· ψῆγμα
39a.10
δ' ἐστὶ χρυσοῦ μικρᾶς ἑψήσεως δεόμενον· τοῦθ' ἱπποζυ-
γίοις μετίασιν οἱ πλησιόχωροι λάθρα· φανερῶς γὰρ δια-
μάχονται, καὶ διώκουσι φεύγοντας· καταλαβόντες δὲ 


διαχρῶνται καὶ αὐτοὺς καὶ τὰ ὑποζύγια. Πρὸς δὲ τὸ
λαθεῖν κρέα θήρεια προστιθέασι κατὰ μέρη· περισπας-
39a.15
θέντων δ' ἀναιροῦνται τὸ ψῆγμα, καὶ τῷ τυχόντι τῶν
ἐμπόρων ἀργὸν διατίθενται, χωνεύειν οὐκ εἰδότες.
39b.1
 Arrianus Ind. c. 5, 4: Μεγασθένης δὲ καὶ ἀτρεκέας
εἶναι ὑπὲρ τῶν μυρμήκων τὸν λόγον ἱστορέει, τούτους
εἶναι τοὺς τὸν χρυσὸν ὀρύσσοντας, οὐκ αὐτοῦ τοῦ χρυ-
σοῦ ἕνεκα, ἀλλὰ φύσι γὰρ κατὰ τῆς γῆς ὀρύσσουσιν, ἵνα
39b.5
φωλεύοιεν· κατάπερ οἱ ἡμέτεροι οἱ σμικροὶ μύρμηκες ὀλί-
γον τῆς γῆς ὀρύσσουσιν· (6) ἐκείνους δὲ, εἶναι γὰρ
πέκων μέζονας, πρὸς λόγον τοῦ μεγέθεος σφῶν καὶ τὴν
γῆν ὀρύσσειν· τὴν δὲ γῆν χρυσῖτιν εἶναι, καὶ ἀπὸ ταύ-
της γίνεσθαι Ἰνδοῖσι τὸν χρυσόν. (7) Ἀλλὰ Μεγασθένης
39b.10
ἀκοὴν ἀπηγέεται.  
40.1
 Strabo XV.: Περὶ δὲ τῶν φιλοσόφων
λέγων (sc. <Μεγασθένης>) τοὺς μὲν ὀρεινοὺς αὐτῶν
φησιν ὑμνητὰς εἶναι τοῦ Διονύσου, δεικνύντας τεκμήρια
τὴν ἀγρίαν ἄμπελον παρὰ μόνοις αὐτοῖς φυομένην καὶ
40.5
κιττὸν καὶ δάφνην καὶ μυρρίνην καὶ πύξον καὶ ἄλλα
τῶν ἀειθαλῶν, ὧν μηδὲν εἶναι πέραν τοῦ Εὐφράτου,
πλὴν ἐν παραδείσοις σπάνια καὶ μετὰ πολλῆς ἐπιμε-
λείας σωζόμενα. Διονυσιακὸν δὲ καὶ τὸ σινδονοφορεῖν
καὶ τὸ μιτροῦσθαι καὶ μυροῦσθαι καὶ βάπτεσθαι ἄνθινα
40.10
καὶ τοὺς βασιλέας κωδωνοφορεῖσθαι καὶ τυμπανίζεσθαι
κατὰ τὰς ἐξόδους. Τοὺς δὲ πεδιασίους τὸν Ἡρακλέα
τιμᾶν. [Ταῦτα μὲν οὖν μυθώδη καὶ ὑπὸ πολλῶν ἐλεγ-
χόμενα, καὶ μάλιστα [τὰ] περὶ τῆς ἀμπέλου καὶ τοῦ οἴνου,
πέραν γὰρ τοῦ Εὐφράτου καὶ τῆς Ἀρμενίας ἐστὶ πολλή·
40.15
καὶ ἡ Μεσοποταμία ὅλη καὶ ἡ Μηδία ἑξῆς μέχρι καὶ
Περσίδος καὶ Καρμανίας· τούτων δὲ τῶν ἐθνῶν ἑκάστου
πολὺ μέρος εὐάμπελον καὶ εὔοινον λέγεται.]
 Ἄλλην δὲ διαίρεσιν ποιεῖται περὶ τῶν φιλοσόφων, δύο
γένη φάσκων, ὧν τοὺς μὲν Βραχμᾶνας καλεῖ, τοὺς δὲ
40.20
Σαρμάνας. Τοὺς μὲν οὖν Βραχμᾶ-
νας εὐδοκιμεῖν μᾶλλον, μᾶλλον γὰρ καὶ ὁμολογεῖν ἐν τοῖς
δόγμασιν· ἤδη δ' εὐθὺς καὶ κυομένους ἔχειν ἐπιμελητὰς,
λογίους ἄνδρας· οὓς προσιόντας λόγον μὲν ἐπᾴδειν δο-
κεῖν καὶ τὴν μητέρα καὶ τὸν κυόμενον εἰς εὐτεκνίαν· τὸ
40.25
δ' ἀληθὲς, σωφρονικάς τινας παραινέσεις καὶ ὑποθήκας  
διδόναι· τὰς δ' ἥδιστα ἀκροωμένας, μᾶλλον εὐτέκνους
εἶναι νομίζεσθαι. Μετὰ δὲ τὴν γένεσιν ἄλλους καὶ ἄλλους
διαδέχεσθαι τὴν ἐπιμέλειαν, ἀεὶ τῆς μείζονος ἡλικίας
χαριεστέρων τυγχανούσης διδασκάλων. Διατρίβειν δὲ
40.30
τοὺς φιλοσόφους ἐν ἄλσει πρὸ τῆς πόλεως, ὑπὸ περιβόλῳ
συμμέτρῳ, λιτῶς ζῶντας ἐν στιβάσι καὶ δοραῖς, ἀπε-
χομένους ἐμψύχων καὶ ἀφροδισίων, ἀκροωμένους λόγων
σπουδαίων, μεταδιδόντας καὶ τοῖς ἐθέλουσι· τὸν δ' ἀκρο-
ώμενον οὔτε λαλῆσαι θέμις, οὔτε χρέμψασθαι, ἀλλ'
40.35
οὐδὲ πτύσαι· ἢ ἐκβάλλεσθαι τῆς συνουσίας τὴν ἡμέραν
ἐκείνην, ὡς ἀκολασταίνοντα. Ἔτη δ' ἑπτὰ καὶ τριά-
κοντα οὕτως ζήσαντα ἀναχωρεῖν εἰς τὴν ἑαυτοῦ κτῆσιν
ἕκαστον, καὶ ζῆν ἀδεῶς καὶ ἀνειμένως μᾶλλον, σινδονο-
φοροῦντα καὶ χρυσοφοροῦντα μετρίως ἐν ταῖς χερσὶ
40.40
καὶ τοῖς ὠσὶ, προσφερόμενον σάρκας τῶν μὴ πρὸς τὴν
χρείαν συνεργῶν ζῴων, δριμέων καὶ ἀρτυτῶν ἀπεχόμε-
νον· γαμεῖν δ' ὅτι πλείστας εἰς πολυτεκνίαν· ἐκ πολλῶν
γὰρ καὶ τὰ σπουδαῖα πλείω γίνεσθαι ἄν· ἀδούλοις οὖσί
τε τὴν ἐκ τέκνων ὑπηρεσίαν ἐγγυτάτω οὖσαν πλείω
40.45
δεῖν παρασκευάζεσθαι. Ταῖς δὲ γυναιξὶ ταῖς γαμεταῖς
μὴ συμφιλοσοφεῖν τοὺς Βραχμᾶνας· εἰ μὲν μοχθηραὶ
γένοιντο, ἵνα μή τι τῶν οὐ θεμιτῶν ἐκφέροιεν εἰς τοὺς
βεβήλους, εἰ δὲ σπουδαῖαι, μὴ καταλείποιεν αὐτούς·
οὐδένα γὰρ ἡδονῆς καὶ πόνου καταφρονοῦντα, ὡς δ' αὕ-
40.50
τως ζωῆς καὶ θανάτου, ἐθέλειν ὑφ' ἑτέρῳ εἶναι· τοιοῦτον
δ' εἶναι τὸν σπουδαῖον καὶ τὴν σπουδαίαν.
 Πλείστους δ' αὐτοῖς εἶναι λόγους περὶ θανάτου· νο-
μίζειν γὰρ δὴ τὸν μὲν ἐνθάδε βίον ὡς ἂν ἀκμὴν κυομέ-
νων εἶναι· τὸν δὲ θάνατον γένεσιν εἰς τὸν ὄντως βίον καὶ
40.55
τὸν εὐδαίμονα τοῖς φιλοσοφήσασι· διὸ τῇ ἀσκήσει πλεί-
στῃ χρῆσθαι πρὸς τὸ ἑτοιμοθάνατον· ἀγαθὸν δὲ ἢ κα-
κὸν μηδὲν εἶναι τῶν συμβαινόντων ἀνθρώποις· οὐ γὰρ
ἂν τοῖς αὐτοῖς τοὺς μὲν ἄχθεσθαι, τοὺς δὲ χαίρειν, ἐνυ-
πνιώδεις ὑπολήψεις ἔχοντας, καὶ τοὺς αὐτοὺς τοῖς αὐτοῖς
40.60
τοτὲ μὲν ἄχθεσθαι, τοτὲ δ' αὖ χαίρειν μεταβαλλομέ-
νους. Τὰ δὲ περὶ φύσιν, τὰ μὲν εὐήθειαν ἐμφαίνειν
φησίν· ἐν ἔργοις γὰρ αὐτοὺς κρείττους ἢ λόγοις εἶναι,
διὰ μύθων τὰ πολλὰ πιστουμένους· περὶ πολλῶν δὲ τοῖς
Ἕλλησιν ὁμοδοξεῖν· ὅτι γὰρ γενητὸς ὁ κό-
40.65
σμος καὶ φθαρτὸς, λέγειν κἀκείνους, καὶ ὅτι σφαιροειδής·
ὅ τε διοικῶν αὐτὸν καὶ ποιῶν θεὸς δι' ὅλου διαπεφοί-
τηκεν αὐτοῦ· ἀρχαὶ δὲ τῶν μὲν συμπάντων ἕτεραι, τῆς
δὲ κοσμοποιίας τὸ ὕδωρ· πρὸς δὲ τοῖς τέτταρσι στοι-
χείοις πέμπτη τίς ἐστι φύσις, ἐξ ἧς ὁ οὐρανὸς καὶ τὰ
40.70
ἄστρα· γῆ δ' ἐν μέσῳ ἵδρυται τοῦ παντός· καὶ περὶ
σπέρματος δὲ καὶ ψυχῆς ὅμοια λέγεται, καὶ ἄλλα πλείω·
παραπλέκουσι δὲ καὶ μύθους, ὥσπερ καὶ Πλάτων περί
τε ἀφθαρσίας ψυχῆς, καὶ τῶν καθ' ᾅδου κρίσεων, καὶ
ἄλλα τοιαῦτα. Περὶ μὲν τῶν Βραχμάνων ταῦτα λέγει.
40.75
 Τοὺς δὲ Σαρμάνας, τοὺς μὲν ἐντιμοτάτους Ὑλοβίους
φησὶν ὀνομάζεσθαι, ζῶντας ἐν ταῖς ὕλαις ἀπὸ φύλλων
καὶ καρπῶν ἀγρίων, ἐσθῆτας δ' ἔχειν ἀπὸ φλοιῶν δεν-


δρείων, ἀφροδισίων χωρὶς καὶ οἴνου· τοῖς δὲ βασιλεῦσι
συνεῖναι, δι' ἀγγέλων πυνθανομένοις περὶ τῶν αἰτίων,
40.80
καὶ δι' ἐκείνων θεραπεύουσι καὶ λιτανεύουσι τὸ θεῖον.
Μετὰ δὲ τοὺς Ὑλοβίους δευτερεύειν κατὰ τιμὴν τοὺς
ἰατρικοὺς, καὶ ὡς περὶ τὸν ἄνθρωπον φιλοσόφους, λι-
τοὺς μὲν, μὴ ἀγραύλους δὲ, ὀρύζῃ καὶ ἀλφίτοις τρεφο-
μένους, ἃ παρέχειν αὐτοῖς πάντα τὸν αἰτηθέντα καὶ
40.85
ὑποδεξάμενον ξενίᾳ· δύνασθαι δὲ καὶ πολυγόνους ποιεῖν,
καὶ ἀρρενογόνους, καὶ θηλυγόνους διὰ φαρμακευτικῆς·
τὴν δὲ ἰατρείαν διὰ σιτίων τὸ πλέον, οὐ διὰ φαρμάκων
ἐπιτελεῖσθαι· τῶν φαρμάκων δὲ μάλιστα εὐδοκιμεῖν
τὰ ἐπίχριστα καὶ τὰ καταπλάσματα· τἄλλα δὲ κα-
40.90
κουργίας πολὺ μετέχειν. Ἀσκεῖν δὲ καὶ τούτους κἀ-
κείνους καρτερίαν, τήν τε ἐν πόνοις καὶ τὴν ἐν ταῖς
ὑπομοναῖς, ὥστ' ἐφ' ἑνὸς σχήματος ἀκίνητον διατελέσαι
τὴν ἡμέραν ὅλην. Ἄλλους δ' εἶναι τοὺς μὲν μαντικοὺς  
καὶ ἐπῳδοὺς καὶ τῶν περὶ τοὺς κατοιχομένους λόγων
40.95
καὶ νομίμων ἐμπείρους, ἐπαιτοῦντας καὶ κατὰ κώμας
καὶ πόλεις· τοὺς δὲ χαριεστέρους τῶν καθ' ᾅδου θρυ-
λουμένων, ὅσα δοκεῖ πρὸς εὐσέβειαν καὶ ὁσιότητα·
συμφιλοσοφεῖν δ' ἐνίοις καὶ γυναῖκας, ἀπεχομένας καὶ
αὐτὰς ἀφροδισίων.
41a.1
 Clem. Alex. Strom. I:
Μεγασθένης ὁ συγγραφεὺς ὁ Σελεύκῳ τῷ Νικάτορι
συμβεβιωκὼς ἐν τῇ τρίτῃ τῶν Ἰνδικῶν ὧδε γράφει·
»Ἅπαντα μέντοι τὰ περὶ φύσεως εἰρημένα παρὰ τοῖς
41a.5
ἀρχαίοις λέγεται καὶ παρὰ τοῖς ἔξω τῆς Ἑλλάδος φιλο-
σοφοῦσι, τὰ μὲν παρ' Ἰνδοῖς ὑπὸ τῶν Βραχμάνων, τὰ
δὲ ἐν τῇ Συρίᾳ ὑπὸ τῶν καλουμένων Ἰουδαίων.»
41b.1
 Clemens l. l.: Διττὸν δὲ τούτων
τὸ γένος· οἱ μὲν Σαρμᾶναι αὐτῶν, οἱ δὲ Βραχμᾶναι
καλούμενοι· καὶ τῶν Σαρμανῶν οἱ Ὑλόβιοι προσαγο-
ρευόμενοι οὔτε πόλεις οἰκοῦσιν οὔτε στέγας ἔχουσιν,
41b.5
δένδρων δὲ ἀμφιέννυνται φλοιοῖς, καὶ ἀκρόδρυα σιτοῦν-
ται καὶ ὕδωρ ταῖς χερσὶ πίνουσιν· οὐ γάμον, οὐ παιδο-
ποιίαν ἴσασιν, [ὥσπερ οἱ νῦν Ἐγκρατηταὶ καλούμενοι,
εἰσὶ δὲ τῶν Ἰνδῶν οἱ τοῖς Βούττα πειθόμενοι παραγ-
γέλμασιν, ὃν δι' ὑπερβολὴν σεμνότητος ὡς θεὸν τετιμή-
41b.10
κασι.]  
42.1
 Strabo XV: <Μεγασθένης> δ' ἐν μὲν τοῖς
φιλοσόφοις οὐκ εἶναι δόγμα φησὶν ἑαυτοὺς ἐξάγειν· τοὺς
δὲ ποιοῦντας τοῦτο, νεανικοὺς κρίνεσθαι, τοὺς μὲν
σκληροὺς τῇ φύσει φερομένους ἐπὶ πληγὴν ἢ κρημνὸν,
42.5
τοὺς δ' ἀπόνους ἐπὶ βυθὸν, τοὺς δὲ πολυπόνους ἀπαγχο-
μένους, τοὺς δὲ πυρώδεις εἰς πῦρ ὠθουμένους, οἷος ἦν
καὶ ὁ Κάλανος, ἀκόλαστος ἄνθρωπος, καὶ ταῖς Ἀλε-
ξάνδρου τραπέζαις δεδουλωμένος· τοῦτον μὲν οὖν ψέ-
γεσθαι, τὸν δὲ Μάνδανιν ἐπαινεῖσθαι, ὃς τῶν τοῦ Ἀλε-
42.10
ξάνδρου ἀγγέλων καλούντων πρὸς τὸν Διὸς υἱὸν,
πειθομένῳ τε δῶρα ἔσεσθαι ὑπισχνουμένων, ἀπει-
θοῦντι δὲ κόλασιν· μήτε ἐκεῖνον φαίη Διὸς υἱὸν, ὅς γε
ἄρχει μηδὲ πολλοστοῦ μέρους τῆς γῆς· μηδὲ αὐτῷ
δεῖν τῶν παρ' ἐκείνου δωρεῶν, ᾧ οὐδεὶς κόρος· μήτε
42.15
δὲ ἀπειλῆς εἶναι φόβον, ᾧ ζῶντι μὲν ἀρκοῦσα εἴη τρο-
φὸς ἡ Ἰνδικὴ, ἀποθανόντι δὲ ἀπαλλάξαιτο τῆς σαρκὸς
ἀπὸ γήρως τετρυχωμένης, μεταστὰς εἰς βελτίω καὶ
καθαρώτερον βίον· ὥστ' ἐπαινέσαι τὸν Ἀλέξανδρον καὶ
συγχωρῆσαι.
43.1
 Arrian. Exp. Alex. VII, 2, 4:
Οὐκοῦν οὐδὲ Ἀλέξανδρον ἐπιχειρῆσαι βιάσασθαι, γνόντα
ἐλεύθερον ὄντα τὸν ἄνδρα· ἀλλὰ Κάλανον γὰρ ἀναπει-
σθῆναι τῶν ταύτῃ σοφιστῶν, ὅντινα μάλιστα δὴ αὑτοῦ
43.5
ἀκράτορα Μεγασθένης ἀνέγραψεν· αὐτούς τε τοὺς σο-
φιστὰς λέγειν κακίζοντας τὸν Κάλανον, ὅτι ἀπολιπὼν
τὴν παρὰ σφίσιν εὐδαιμονίαν, ὁ δὲ δεσπότην ἄλλον
ἢ τὸν θεὸν ἐθεράπευε.  
43.8
\end{greek}

\section{Hipparchus}

\subsection{About Hipparchus}

\blockquote[From Wikipedia]{Hipparchus/hɪˈpɑːrkəs/ of Nicaea, or more correctly Hipparchos (Greek: Ἵππαρχος, Hipparkhos; c. 190 BC – c. 120 BC), was a Greek astronomer, geographer, and mathematician of the Hellenistic period. He is considered the founder of trigonometry but is most famous for his incidental discovery of precession of the equinoxes.\footnote{\url{http://en.wikipedia.org/wiki/Hipparchus}.}}

\subsection{Fragmenta geographica}
Hipparchus Astron., Geogr., Fragmenta geographica (1431: 002)
“The geographical fragments of Hipparchus”, Ed. Dicks, D.R.
London: Athlone Press, 1960.
Fragment 13, line 3

\begin{greek}

Strabo, 69 – 70 
ἔτι φησὶν ὁ Ἵππαρχος ἐν τῷ δευτέρῳ ὑπομνήματι αὐτὸν τὸν 
Ἐρατοσθένη διαβάλλειν τὴν τοῦ Πατροκλέους πίστιν ἐκ τῆς πρὸς 
Μεγασθένη διαφωνίας περὶ τοῦ μήκους τῆς Ἰνδικῆς τοῦ κατὰ τὸ 
βόρειον πλευρόν, τοῦ μὲν Μεγασθένους λέγοντος σταδίων μυρίων 
ἑξακισχιλίων, τοῦ δὲ Πατροκλέους χιλίοις λείπειν φαμένου· ἀπὸ γάρ 
τινος ἀναγραφῆς σταθμῶν ὁρμηθέντα τοῖς μὲν ἀπιστεῖν διὰ τὴν 
διαφωνίαν, ἐκείνῃ δὲ προσέχειν. 



Hipparchus Astron., Geogr., Fragmenta geographica 
Fragment 13, line 11

                                      εἰ οὖν διὰ τὴν διαφωνίαν ἐνταῦθα 
ἄπιστος ὁ Πατροκλῆς, καίτοι παρὰ χιλίους σταδίους τῆς διαφορᾶς 
οὔσης, πόσῳ χρὴ μᾶλλον ἀπιστεῖν ἐν οἷς παρὰ ὀκτακισχιλίους ἡ 
διαφορά ἐστιν, πρὸς δύο καὶ ταῦτα ἄνδρας συμφωνοῦντας ἀλλήλοις, 
τῶν μὲν λεγόντων τὸ τῆς Ἰνδικῆς πλάτος δισμυρίων σταδίων, τοῦ 
δὲ μυρίων καὶ δισχιλίων; 



Hipparchus Astron., Geogr., Fragmenta geographica 
Fragment 14, line 3

Strabo, 71 
ὥστ' οὐδ' ἐκεῖνο εὖ λέγει τὸ ἐπειδὴ οὐκ ἔχομεν λέγειν οὔθ' ἡμέρας 
μεγίστης πρὸς τὴν βραχυτάτην λόγον οὔτε γνώμονος πρὸς σκιὰν ἐπὶ 
τῇ παρωρείᾳ τῇ ἀπὸ Κιλικίας μέχρι Ἰνδῶν, οὐδ' εἰ ἐπὶ παραλλήλου 
γραμμῆς ἐστιν ἡ λόξωσις, ἔχομεν εἰπεῖν, ἀλλ' ἐᾶν ἀδιόρθωτον, λοξὴν 
φυλάξαντες, ὡς οἱ ἀρχαῖοι πίνακες παρέχουσι. 



Hipparchus Astron., Geogr., Fragmenta geographica 
Fragment 15, line 2

                                                      ὅρα γάρ, εἰ τοῦτο μὲν 
μὴ κινοίη τις τὸ τὰ ἄκρα τῆς Ἰνδικῆς τὰ μεσημβρινὰ ἀνταίρειν τοῖς 
κατὰ Μερόην, μηδὲ τὸ διάστημα τὸ ἀπὸ Μερόης ἐπὶ τὸ στόμα τὸ 
κατὰ τὸ Βυζάντιον, ὅτι ἐστὶ περὶ μυρίους σταδίους καὶ ὀκτακισχι-
λίους, ποιοίη δὲ τρισμυρίων τὸ ἀπὸ τῶν μεσημβρινῶν Ἰνδῶν μέχρι 
τῶν ὀρῶν, ὅσα ἂν συμβαίη ἄτοπα. 



Hipparchus Astron., Geogr., Fragmenta geographica 
Fragment 17, line 1

Strabo, 77 
νυνὶ μὲν οὖν ὑποθέμενοι τὰ νοτιώτατα τῆς Ἰνδικῆς ἀνταίρειν τοῖς 
κατὰ Μερόην, ὅπερ εἰρήκασι πολλοὶ καὶ πεπιστεύκασιν, ἐπεδείξαμεν 
τὰ συμβαίνοντα ἄτοπα. 



Hipparchus Astron., Geogr., Fragmenta geographica 
Fragment 17, line 14

                                             τὸ μὲν οὖν κατὰ Μερόην 
κλίμα Φίλωνά τε τὸν συγγράψαντα τὸν εἰς Αἰθιοπίαν πλοῦν ἱστορεῖν, 
ὅτι πρὸ πέντε καὶ τεσσαράκοντα ἡμερῶν τῆς θερινῆς τροπῆς κατὰ 
κορυφὴν γίνεται ὁ ἥλιος, λέγειν δὲ καὶ τοὺς λόγους τοῦ γνώμονος 
πρός τε τὰς τροπικὰς σκιὰς καὶ τὰς ἰσημερινάς, αὐτόν τε Ἐρατοσθένη 
συμφωνεῖν ἔγγιστα τῷ Φίλωνι, τὸ δ' ἐν τῇ Ἰνδικῇ κλίμα μηδένα 
ἱστορεῖν, μηδ' αὐτὸν Ἐρατοσθένη. 



Hipparchus Astron., Geogr., Fragmenta geographica 
Fragment 17, line 18

                                        εἰ δὲ δὴ καὶ αἱ ἄρκτοι ἐκεῖ 
ἀμφότεραι, ὡς οἴονται, ἀποκρύπτονται, πιστεύοντες τοῖς περὶ 
Νέαρχον, μὴ δυνατὸν εἶναι ἐπὶ ταὐτοῦ παραλλήλου κεῖσθαι τήν τε 
Μερόην καὶ τὰ ἄκρα τῆς Ἰνδικῆς. 



Hipparchus Astron., Geogr., Fragmenta geographica 
Fragment 21, line 4

                         βουλόμενος γὰρ βεβαιοῦν τὸ ἐξ ἀρχῆς, ὅτι οὐ 
μεταθετέον τὴν Ἰνδικὴν ἐπὶ τὰ νοτιώτερα, ὥσπερ Ἐρατοσθένης 
ἀξιοῖ, σαφὲς ἂν γενέσθαι τοῦτο μάλιστά φησιν ἐξ ὧν αὐτὸς ἐκεῖνος 
προφέρεται· τὴν γὰρ τρίτην μερίδα κατὰ τὴν βόρειον πλευρὰν εἰπόντα 
ἀφορίζεσθαι ὑπὸ τῆς ἀπὸ Κασπίων πυλῶν ἐπὶ τὸν Εὐφράτην γραμμῆς 
σταδίων μυρίων οὔσης, μετὰ ταῦτα ἐπιφέρειν ὅτι τὸ νότιον πλευρὸν 
τὸ ἀπὸ Βαβυλῶνος εἰς τοὺς ὅρους τῆς Καρμανίας μικρῷ πλειόνων 
ἐστὶν ἢ ἐννακισχιλίων, τὸ δὲ πρὸς δύσει πλευρὸν ἀπὸ Θαψάκου παρὰ 
τὸν Εὐφράτην ἐστὶν εἰς Βαβυλῶνα τετρακισχίλιοι ὀκτακόσιοι στάδιοι, 
καὶ ἑξῆς ἐπὶ τὰς ἐκβολὰς τρισχίλιοι, τὰ δὲ πρὸς ἄρκτον ἀπὸ Θαψάκου, 
τὸ μὲν ἀπομεμέτρηται μέχρι χιλίων ἑκατόν, τὸ λοιπὸν δ' οὐκέτι. 



Hipparchus Astron., Geogr., Fragmenta geographica 
Fragment 24, line 17

ὑποθέσεις ταύτας τὴν διὰ Κασπίων πυλῶν μεσημβρινὴν γραμμὴν ἐπὶ 
τοῦ διὰ Βαβυλῶνος καὶ Σούσων παραλλήλου δυσμικωτέραν ἔχειν τὴν 
κοινὴν τομὴν τῆς κοινῆς τομῆς τοῦ αὐτοῦ παραλλήλου καὶ τῆς ἀπὸ 
Κασπίων πυλῶν καθηκούσης εὐθείας ἐπὶ τοὺς ὅρους τοὺς τῆς Καρ-
μανίας καὶ τῆς Περσίδος πλείοσι τῶν τετρακισχιλίων καὶ τετρα-
κοσίων· σχεδὸν δή τι πρὸς τὴν διὰ Κασπίων πυλῶν μεσημβρινὴν 
γραμμὴν ἡμίσειαν ὀρθῆς ποιεῖν γωνίαν τὴν διὰ Κασπίων πυλῶν καὶ 
τῶν ὅρων τῆς τε Καρμανίας καὶ τῆς Περσίδος, καὶ νεύειν αὐτὴν ἐπὶ 
τὰ μέσα τῆς τε μεσημβρίας καὶ τῆς ἰσημερινῆς ἀνατολῆς· ταύτῃ δ' 
εἶναι παράλληλον τὸν Ἰνδὸν ποταμόν, ὥστε καὶ τοῦτον ἀπὸ τῶν 
ὀρῶν οὐκ ἐπὶ μεσημβρίαν ῥεῖν, ὥς φησιν Ἐρατοσθένης, ἀλλὰ μεταξὺ 
ταύτης καὶ τῆς ἰσημερινῆς ἀνατολῆς, καθάπερ ἐν τοῖς ἀρχαίοις 
πίναξι καταγέγραπται. 



Hipparchus Astron., Geogr., Fragmenta geographica 
Fragment 25, line 2

Strabo, 87 
χωρὶς δὲ τούτων κἀκεῖνος εἴρηκεν, [φησίν], ὅτι ῥομβοειδές ἐστι τὸ 
σχῆμα τῆς Ἰνδικῆς· καὶ καθάπερ ἡ ἑωθινὴ πλευρὰ παρέσπασται 
πολὺ πρὸς ἕω, καὶ μάλιστα τῷ ἐσχάτῳ ἀκρωτηρίῳ, ὃ καὶ πρὸς 
μεσημβρίαν προπίπτει πλέον παρὰ τὴν ἄλλην ἠιόνα, οὕτω καὶ ἡ παρὰ 
τὸν Ἰνδὸν πλευρά. 

\end{greek}

\section{Nicander}

\blockquote[From Wikipedia\footnote{\url{http://en.wikipedia.org/wiki/Nicander}.}]{Nicander of Colophon (Νίκανδρος ὁ Κολοφώνιος, 2nd century BC), Greek poet, physician and grammarian, was born at Claros, (Ahmetbeyli, Izmir in modern Turkey), near Colophon, where his family held the hereditary priesthood of Apollo. He flourished under Attalus III of Pergamum.}

\begin{greek}

Nicander Epic., Theriaca (0022: 001)
“Nicander. The poems and poetical fragments”, Ed. Gow, A.S.F., Scholfield, A.F.
Cambridge: Cambridge University Press, 1953.

Nicander Epic., Theriaca 
Line 890

εἰ δέ, σύ γ' ἐκ ποίης ἀβληχρέος ἔγχλοα ῥίζαν 
θηρὸς ἰσαζομένην τμήξαις ἰοειδέι κέντρῳ 
σκορπίου, ἠὲ σίδας Ψαμαθηίδας, ἅς τε Τράφεια 
Κῶπαί τε λιμναῖον ὑπεθρέψαντο παρ' ὕδωρ, 
ᾗπερ Σχοινῆός τε ῥόος Κνώποιό τε βάλλει, 
ὅσσα θ' ὑπ' Ἰνδὸν χεῦμα πολυφλοίσβοιο Χοάσπεω 
πιστάκι' ἀκρεμόνεσσιν ἀμυγδαλόεντα πέφανται· 
καυκαλίδας, σὺν δ' αἰθὰ βάλοις φιμώδεα μύρτα,   
κάρφεά θ' ὁρμίνοιο καὶ ἐκ μαράθου βρυόεντος, 
εἰρύσιμόν τε καὶ ἀγροτέρου σπερμεῖ' ἐρεβίνθου 
σὺν χλοεροῖς θάμνοισι βαλὼν βαρυώδεα ποίην. 

\end{greek}

\section{Aristophanes of Byzantium}

\blockquote[From Wikipedia\footnote{\url{http://en.wikipedia.org/wiki/Aristophanes_of_Byzantium}.}]{Aristophanes (Greek: Ἀριστοφάνης) of Byzantium (c. 257 BC – c. 185–180 BC) was a Greek scholar, critic and grammarian, particularly renowned for his work in Homeric scholarship, but also for work on other classical authors such as Pindar and Hesiod. Born in Byzantium about 257 BC, he soon moved to Alexandria and studied under Zenodotus, Callimachus, and Dionysius Iambus. He succeeded Eratosthenes as head librarian of the Library of Alexandria at the age of sixty.}

\begin{greek}

Aristophanes Gramm., Aristophanis historiae animalium epitome subjunctis Aeliani Timothei aliorumque eclogis (0644: 001)
“Excerptorum Constantini de natura animalium libri duo. Aristophanis historiae animalium epitome”, Ed. Lambros, S.P.
Berlin: Reimer, 1885; Commentaria in Aristotelem Graeca, suppl. 1.1.
Chapter 2, section 46, line 6

Ὅτι τῶν στενῶν ἐπέκεινα, φησί, τῶν συγκλειόντων τὴν Ἀραβίαν 
καὶ τὴν ἀπέναντι χώραν νῆσοι κεῖνται σποράδες, ταπειναὶ πᾶσαι, μικραὶ 
τῷ μεγέθει, τὸ πλῆθος ἀμύθητοι, καρπὸν οὐδένα γεννῶσαι πρὸς τὸν βίον, 
οὔτε ἥμερον οὔτε ἄγριον, ἀπέχουσαι μὲν τῆς εἰρημένης ἠπείρου σταδίους 
ὡς ἑβδομήκοντα, τετραμμέναι δὲ πρὸς τὸ δοκοῦν πέλαγος παρεκτείνειν τὴν 
Ἰνδικὴν καὶ Γεδρωσίαν. 



Aristophanes Gramm., Aristophanis historiae animalium epitome subjunctis Aeliani Timothei aliorumque eclogis 
Chapter 2, section 59, line 5

                   οὗτοι κομῆται μέν εἰσι καὶ πώγωνας φέρουσιν ἐξαισίους, 
κύνας δὲ τρέφουσι πολλοὺς καὶ μεγάλους, ὁμοίως τοῖς Ὑρκανοῖς, καὶ τοὺς 
ἐπιφοιτῶντας αὐτῶν τῇ χώρᾳ Ἰνδικοὺς βόας δι' αὐτῶν θηρῶσι, πλῆθος 
ἐπιφαινομένους ἀμύθητον ἀπὸ τροπῶν χειμερινῶν ἕως μέσου χειμῶνος· 
εἶτα καὶ τὰς τῶν κυνῶν θηλείας ἀμέλγοντες τῷ γάλακτι τρέφονται, καὶ ἐξ 
ἄλλων δὲ θήρας ζῴων. 



Aristophanes Gramm., Aristophanis historiae animalium epitome subjunctis Aeliani Timothei aliorumque eclogis 
Chapter 2, section 67, line 1

Εἰσὶ δὲ ἐν τοῖς Ἰνδικοῖς οἳ Πυγμαῖοι καλοῦνται. 



Aristophanes Gramm., Aristophanis historiae animalium epitome subjunctis Aeliani Timothei aliorumque eclogis 
Chapter 2, section 67, line 2

                                                               χώρα δ' αὐτῶν 
πολλὴ μέσῃ ἐν τῇ Ἰνδικῇ καὶ ἄνθρωποί εἰσι μέλανες, ὥσπερ οἱ ἄλλοι 
Ἰνδοὶ καὶ ὁμόγλωσσοι ἐκείνοις, μικροὶ δὲ κάρτα, καὶ οἱ μακρότατοι αὐτῶν   
εἰσι πήχεων δύο, οἱ δὲ πλεῖστοι πήχεος <ἑνὸς> καὶ ἡμίσεος, ἄνδρες καὶ 
γυναῖκες. 



Aristophanes Gramm., Aristophanis historiae animalium epitome subjunctis Aeliani Timothei aliorumque eclogis 
Chapter 2, section 67, line 16

                                                                   αὐτοὶ δέ εἰσι 
σιμοί τε καὶ αἰσχροὶ καὶ οὐδὲν ἐοικότες τοῖς ἄλλοις Ἰνδοῖς. 



Aristophanes Gramm., Aristophanis historiae animalium epitome subjunctis Aeliani Timothei aliorumque eclogis 
Chapter 2, section 91, line 1

                                                    πλὴν ἐν Ἰνδοῖς 
ἐὰν ἁλῷ τέλειος ἐλέφας χαλεπός ἐστιν ἡμερωθῆναι, καὶ τὴν ἐλευθερίαν 
ποθῶν φονᾷ. 



Aristophanes Gramm., Aristophanis historiae animalium epitome subjunctis Aeliani Timothei aliorumque eclogis 
Chapter 2, section 91, line 4

               εἰ δὲ καὶ δεσμοῖς αὐτὸν διαλάβῃς, ἔτι καὶ μᾶλλον εἰς τὸν 
θυμὸν ἐξάπτεται <καὶ> δουλοσύνην οὐχ ὑπομένει. ἀλλ' Ἰνδοὶ καὶ τροφαῖς 
κολακεύουσιν αὐτὸν καὶ ποικίλοις δελέασι πραΰνειν πειρῶνται. 



Aristophanes Gramm., Aristophanis historiae animalium epitome subjunctis Aeliani Timothei aliorumque eclogis 
Chapter 2, section 92, line 1

Διώκονται οὖν παρ' Ἰνδῶν ἕνεκεν ὀδόντων, διὸ τοὺς νωδοὺς πρὸ 
τῆς φάλαγγος ἱστᾶσιν, ὥσπερ ἀπατῶντες ὅτι οὐκ ἔχουσι τὸ ζητούμενον. 



Aristophanes Gramm., Aristophanis historiae animalium epitome subjunctis Aeliani Timothei aliorumque eclogis 
Chapter 2, section 100, line 1

Ἀποτρέπονται δὲ τὴν ἐν Ἰνδοῖς ἄρουραν <τὴν> κεκλημένην Φα-
λάκραν καὶ οὐκ ἐσθίουσι. 



Aristophanes Gramm., Aristophanis historiae animalium epitome subjunctis Aeliani Timothei aliorumque eclogis 
Chapter 2, section 101, line 1

Ὅταν μέντοι ὑπὸ Ἰνδῶν ἀναγκάζωνται αὐτόρριζα δένδρα ἐκ-
σπάσαι, οὐ πρότερον ἐπιχειροῦσι πρὶν διασεῖσαι, εἰ ἄρα ἀνατραπῆναι οἷόν 
τ' ἐστὶν ἢ μή. 



Aristophanes Gramm., Aristophanis historiae animalium epitome subjunctis Aeliani Timothei aliorumque eclogis 
Chapter 2, section 110, line 10

                                      εἰ μὲν οὖν ἔμελλον τὴν ἐν Ἰνδοῖς αὐτῶν 
εὐπείθειαν καὶ εὐμαθίαν ἢ τὴν ἐν Αἰθιοπίᾳ ἢ τὴν ἐν Λιβύῃ γράφειν, ἴσως 
καὶ μῦθον ἐδόκουν τινὰ συμπλάσας κομπάζειν, εἶτα ἐπὶ φήμῃ τοῦ θηρίου 
τῆς φύσεως καταψεύδεσθαι· ὅπερ ἐχρῆν δρᾶν φιλοσοφοῦντα ἄνδρα ἥκιστα 
καὶ ἀληθείας ἐραστὴν διάπυρον. 



Aristophanes Gramm., Aristophanis historiae animalium epitome subjunctis Aeliani Timothei aliorumque eclogis 
Chapter 2, section 111, line 1

Οἱ Ἰνδοὶ τέλειον μὲν ἐλέφαντα συλλαβεῖν ῥᾳδίως ἀδυνατοῦσιν   
(οὔτε γὰρ τοσαῦτα δράσουσιν οὔτε τοσοίδε παρέσονται), εἰς δὲ τὰ ἕλη 
φοιτῶντες τὰ γειτνιῶντα τῷ ποταμῷ, εἶτα μέντοι λαμβάνουσιν αὐτῶν τὰ 
βρέφη. 



Aristophanes Gramm., Aristophanis historiae animalium epitome subjunctis Aeliani Timothei aliorumque eclogis 
Chapter 2, section 113, line 1

Ἐν Ἰνδοῖς, ὡς ἀκούω, ἐλέφας καὶ δράκων ἐστὶν ἔχθιστα. 



Aristophanes Gramm., Aristophanis historiae animalium epitome subjunctis Aeliani Timothei aliorumque eclogis 
Chapter 2, section 115, line 6

οἱ τοίνυν πηρωθέντες τὸν ἕτερον ἐπὶ μετώπου ἑστᾶσι, τῶν λοιπῶν προβαλ-
λομένων αὐτούς, ἵνα οἱ μὲν ὑποδέχωνται τὴν πρώτην ὁρμήν, οἱ δὲ ἀμύνωσιν 
ἀκεραίῳ τῇ τῶν ὀδόντων ῥώμῃ καὶ ἰσοπαλεῖ, ἴσως δὲ τῶν Ἰνδῶν καὶ κατα-
φιλοσοφοῦντες καὶ ἐπιδεικνύντες αὐτοῖς ὅτι ἄρα οὐχ ὑπὲρ μεγάλου τοῦ ἄθλου 
κινδυνεύοντες ἥκουσι. 



Aristophanes Gramm., Aristophanis historiae animalium epitome subjunctis Aeliani Timothei aliorumque eclogis 
Chapter 2, section 117, line 1

Πώρου τοῦ Ἰνδῶν βασιλέως ὁ ἐλέφας ἐν τῇ πρὸς Ἀλέξανδρον 
μάχῃ τετρωμένου πολλὰ ἡσυχῆ καὶ μετὰ φειδοῦς τῇ προβοσκίδι ἐξῄρει τὰ 
ἀκόντια, καὶ μέντοι καὶ αὐτὸς τετρωμένος πολλὰ οὐ πρότερον εἶξε, πρὶν ἢ 
συνιέναι, ὅτι ἄρα ὁ δεσπότης αὐτῷ [ὅτι] διὰ τὴν ῥοὴν τοῦ αἵματος τὴν 
πολλὴν παρεῖται καὶ ἐκθνήσκει. 



Aristophanes Gramm., Aristophanis historiae animalium epitome subjunctis Aeliani Timothei aliorumque eclogis 
Chapter 2, section 124, line 7

                          ταύτῃ τοίνυν ἡ τοῦ τρέφοντος αὐτὸν γυνὴ παιδίον 
ἔτυχε τεκοῦσα πρὸ ἡμερῶν τριάκοντα καὶ παρακατέθετο φωνῇ τῇ Ἰνδῶν, 
ἧς ἀκούουσιν ἐλέφαντες. 



Aristophanes Gramm., Aristophanis historiae animalium epitome subjunctis Aeliani Timothei aliorumque eclogis 
Chapter 2, section 125, line 6

           καὶ τοῦτο μὲν Ἰνδικὸν τὸ ἔργον, ἐκεῖθεν δὲ ἐξεφοίτησε δεῦρο· 
ἀκούω δὲ καὶ ἐπὶ Τόπου Ῥωμαίων βασιλεύοντος, ἀνδρὸς καλοῦ καὶ ἀγαθοῦ, 
ἐν τῇ Ῥώμῃ ταὐτὸ γεγονέναι. 



Aristophanes Gramm., Aristophanis historiae animalium epitome subjunctis Aeliani Timothei aliorumque eclogis 
Chapter 2, section 126, line 1

Τῶν τεθηραμένων ἐλεφάντων ἰῶνται τὰ τραύματα οἱ Ἰνδοὶ τὸν 
τρόπον τοῦτον. 



Aristophanes Gramm., Aristophanis historiae animalium epitome subjunctis Aeliani Timothei aliorumque eclogis 
Chapter 2, section 127, line 14

                                                                      κατασπείρει δὲ 
καὶ τοῦ χώρου ἔνθα αὐλίζεται τῶν ἀνθέων πολλά, ἡδυσμένον αἱρεῖσθαι 
γλιχόμενος ὕπνον. Ἰνδοὶ δὲ ἐλέφαντες ἦσαν ἄρα πηχῶν ἐννέα τὸ ὕψος, 
πέντε δὲ τὸ εὖρος. 



Aristophanes Gramm., Aristophanis historiae animalium epitome subjunctis Aeliani Timothei aliorumque eclogis 
Chapter 2, section 128, line 1

Τὸν Ἰνδῶν βασιλέα προϊόντα ἐπὶ δίκας προσκυνεῖ ὁ ἐλέφας 
πρῶτος, δεδιδαγμένος τοῦτο, καὶ μάλα γε δρῶν μνημόνως τε καὶ εὐπειθῶς 
αὐτό. 



Aristophanes Gramm., Aristophanis historiae animalium epitome subjunctis Aeliani Timothei aliorumque eclogis 
Chapter 2, section 128, line 10

                                                                 τέσσαρες δὲ καὶ 
εἴκοσι τῷ βασιλεῖ φρουροὶ παραμένουσιν ἐλέφαντες ἐκ διαδοχῆς, ὥσπερ οὖν 
οἱ φύλακες οἱ λαμπροί, καὶ αὐτοῖς παίδευμα τὴν φρουρὰν οὐ κατανυστάζειν· 
διδάσκονται γὰρ σοφίᾳ τινὶ Ἰνδικῇ καὶ τοῦτο. 



Aristophanes Gramm., Aristophanis historiae animalium epitome subjunctis Aeliani Timothei aliorumque eclogis 
Chapter 2, section 132, line 1

Ὅτι ἡ τῶν Ἰνδῶν γῆ τοὺς ἐλέφαντας ἔχει τοὺς μεγίστους καὶ 
ὑπερφυεστάτους, ὡς ἐκ τῶν ὀδόντων δῆλον τῶν ἐκεῖθεν κομιζομένων, καὶ 
δὴ καὶ τοὺς ταυρελέφαντας λεγομένους. 



Aristophanes Gramm., Aristophanis historiae animalium epitome subjunctis Aeliani Timothei aliorumque eclogis 
Chapter 2, section 156, line 1

Λέοντες δὲ ἐν Ἰνδοῖς οὐ γίνονται μέγιστοι. 



Aristophanes Gramm., Aristophanis historiae animalium epitome subjunctis Aeliani Timothei aliorumque eclogis 
Chapter 2, section 196, line 3

   αἱ δὲ Ὑρκανοὶ κύνες λεοντό-
ποδές εἰσιν, εὔστερνοι καὶ εὐτράχηλοι, βρυχοειδὲς ὑλακτοῦσαι καὶ πρὸς 
λέοντας μαχόμεναι· ἀπὸ δὲ Ἰνδῶν εἰς Ὑρκανίαν τὸ γένος ἦλθεν. 



Aristophanes Gramm., Aristophanis historiae animalium epitome subjunctis Aeliani Timothei aliorumque eclogis 
Chapter 2, section 205, line 1

Ἐν δὲ τοῖς Ἰνδικοῖς κύνες εἰσὶ μέγιστοι πάσης γῆς· καὶ πρὸς 
λέοντας ἕτεροι μὲν κύνες οὐ τολμῶσι προσιέναι, οἱ δὲ Ἰνδικοὶ κύνες 
ὁμόσε χωροῦσι, καὶ μάχονται αὐτοῖς, καὶ πολλὰ κακὰ κύων Ἰνδικὸς λέοντι 
παρασχὼν ἡσσᾶται ὑπὸ λέοντος. 



Aristophanes Gramm., Aristophanis historiae animalium epitome subjunctis Aeliani Timothei aliorumque eclogis 
Chapter 2, section 205, line 5

                                    ἤδη μέντοι τινὰ εἶδον λέοντα ὑπὸ κυνὸς 
Ἰνδικοῦ διαφθαρέντα ἐν θήρᾳ. 



Aristophanes Gramm., Aristophanis historiae animalium epitome subjunctis Aeliani Timothei aliorumque eclogis 
Chapter 2, section 269, line 1

Εἰσί γε μὴν κατὰ τὴν Ἰνδίαν παρδάλεις ξανθαί τε καὶ κυαναῖ 
καὶ μέλαιναι καὶ λευκαί, πάσας δὲ γραμμαί τινες εὔκυκλοί τε καὶ μέλαιναι 
περιβάλλουσιν. 



Aristophanes Gramm., Aristophanis historiae animalium epitome subjunctis Aeliani Timothei aliorumque eclogis 
Chapter 2, section 271, line 2

   ἀνὴρ 
γὰρ Ἰνδὸς διὰ Γάζης, φησί, τῆς ἐμῆς διῆλθε, δύο θῆρας τοιούσδε κομίζων 
δῶρον τῷ βασιλεῖ· Ἀναστάσιος δὲ οὗτος ἦν· καὶ ἦσαν μέγεθος μὲν κατὰ 
κάμηλον, τὸ δέρμα δὲ ἀτεχνῶς πάρδαλις· εὐρεῖα χηλή, [πόδες, μέγεθος 
μὲν κατὰ κάμηλον] πόδες μακροί, ἥσσονες δὲ οἱ ὀπίσθιοι καὶ ὑποκλάζειν 
ἄρα δοκοῦντες. 



Aristophanes Gramm., Aristophanis historiae animalium epitome subjunctis Aeliani Timothei aliorumque eclogis 
Chapter 2, section 273, line 8

                                                                           καὶ κυνὶ 
δὲ Ἰνδῷ τίγρις καὶ ἵπποι τόνδε τὸν τρόπον ὄνοις συνῆλθον. 



Aristophanes Gramm., Aristophanis historiae animalium epitome subjunctis Aeliani Timothei aliorumque eclogis 
Chapter 2, section 278, line 1

Ἡ καμηλοπάρδαλις παρ' Ἰνδοῖς ἐστι μάλιστα γινομένη· ἔστι 
δὲ τὰ μὲν ἄλλα ἔλαφος μεγίστη <εἰς> καμήλου ὕψος ἀφικνουμένη, διαφέρει 
δὲ τῷ τε ἄκερως εἶναι καὶ τὸν αὐχένα μήκιστον καὶ ὑπὲρ τὴν ἀναλογίαν 
τοῦ λοιποῦ σώματος ἔχειν εἰς ὕψος ἀνορθούμενον, καὶ τὴν δορὰν ἅπασαν 
ἀπὸ κεφαλῆς ἄκρας ἕως ποδῶν ἐσχάτων παρδάλει μάλιστα τῇ ποικιλίᾳ 
παρεμφερεστάτην, καὶ τοὺς ἔμπροσθεν πόδας τῶν ὀπισθίων ὑψηλοτέρους. 



Aristophanes Gramm., Aristophanis historiae animalium epitome subjunctis Aeliani Timothei aliorumque eclogis 
Chapter 2, section 280, line 1

Ὅτι τοὺς πάνθηρας ἐν Ἰνδίᾳ θηρεύουσι τίγρεις. 



Aristophanes Gramm., Aristophanis historiae animalium epitome subjunctis Aeliani Timothei aliorumque eclogis 
Chapter 2, section 282, line 1

Καὶ ἄλλος δὲ πάνθηρ ἐστὶν ἐν Ἰνδοῖς μύρου κατάπλεως· ὅς γε 
ἡνίκα αὐτὸν ὁ λιμὸς αἱρεῖ, τοῦ φωλεοῦ πρόεισι, καὶ ξυνέρχεται τοῖς 
θηρίοις· τὰ δὲ τῇ ὀσμῇ κηλούμενα ἡδέως τούτῳ συνομιλεῖ· πρὸς δὲ καὶ 
ἠρέμα θέλγων ἄγει πρὸς τὴν εὐνήν, καὶ αὐτὰ συλλαμβάνων ἐσθίει. 



Aristophanes Gramm., Aristophanis historiae animalium epitome subjunctis Aeliani Timothei aliorumque eclogis 
Chapter 2, section 365, line 2

   καὶ ἄλλο δέ τι 
μυῶν γένος τῇ τῶν Ἰνδῶν ἐντρέφεται γῇ, ὧν ἡδίστη μὲν πνοὴ τοῦ σώματος 
ἔξεισι, τὸ χρῆμα δὲ περιμάχητόν ἐστι τοῖς πολίταις· τούτου γὰρ ἕνεκα καὶ 
νεκρὸν ἂν ἴδοις μῦν ἐσθῆτι † πλομίου φιλοτίμως καθαπτόμενον. 



Aristophanes Gramm., Aristophanis historiae animalium epitome subjunctis Aeliani Timothei aliorumque eclogis 
Chapter 2, section 467, line 3

              ἀμέλει τοι καὶ Ἰνδοὶ μακρὰν ἀνύοντες τρίβον καθάπερ ὁδηγοῖς 
ταύταις ἑαυτοὺς ἐπιτρέψαντες εἶτα τὴν χρυσῖτιν ἐκεῖθεν τῶν μυρμήκων 
ἀποσυλῶσι γῆν. 



Aristophanes Gramm., Aristophanis historiae animalium epitome subjunctis Aeliani Timothei aliorumque eclogis 
Chapter 2, section 474, line 1

Ἐν Ἰνδοῖς ταπίδες γίνονται κάλλισται καὶ ποικιλώταται ἐκ 
τῶν ἐρίων ὧν αἱ κάμηλοι φέρουσιν [ἐξ αὐτῶν] ἐπί <τε> τῆς κεφαλῆς καὶ ἐπὶ 
τῶν μηρῶν ἐπί τε τοῦ ἄλλου σώματος παντός. 



Aristophanes Gramm., Aristophanis historiae animalium epitome subjunctis Aeliani Timothei aliorumque eclogis 
Chapter 2, section 474, line 4

                                                    καὶ οὐ βέβαπται· παντο-
δαπὰ δὲ ἄνθη τῶν ἐρίων φύουσιν αἱ κ<άμη>λοι αἱ Ἰνδικαὶ πλὴν πορφυροειδοῦς 
καὶ πρασίνου καὶ φοι<νι>κίνου, τὰ δ' ἄλλα ἔχουσιν ἐρίων χρώματα αὐτοφυῆ αἱ 
κάμ<ηλοι.> αἱ μὲν γὰρ σφόδρα λευκὰ φέρουσι τὰ ἔρια αὐτόθι, αἱ δὲ σφόδ<ρα 
μέ>λανα καὶ μαλακά, αἱ δὲ κυάνεα, αἱ δὲ ὥσπερ ὄνου τὴν χρ<όαν,> αἱ δὲ 
πέλια, αἱ δὲ πυρρά, αἱ δὲ κροκοειδῆ, αἱ δὲ ἀτρέμας ὑπόχλωρα· ὡς δὲ τὸ 
σύμπαν εἰπεῖν, ἔρια παντοδαπὰ φύουσιν Ἰνδικαὶ κάμηλοι πλὴν τῶν εἰρημένων 
χρωμάτων. 



Aristophanes Gramm., Aristophanis historiae animalium epitome subjunctis Aeliani Timothei aliorumque eclogis 
Chapter 2, section 474, line 15

                                                     οἶμαι δὲ ὅτι οὐδαμοῦ 
γῆς γίνονται παρομοίας φύσεως, οὐδὲ ζῷα ποικιλώτερα καὶ ὀρθότερα ἔχουσι 
ταπίδες † ἢ ἐν Ἰνδοῖς τὰ ἐκ τῶν καμήλων ποιοῦσιν ἔρια. 



Aristophanes Gramm., Aristophanis historiae animalium epitome subjunctis Aeliani Timothei aliorumque eclogis 
Chapter 2, section 474, line 17

                                                                     ἄγουσι δὲ καὶ 
εἰς Πέρσας τούτων τῶν ταπίδων οἱ ἔμποροι [καὶ] οἱ Βάκτριοι καὶ οἱ ἄλλοι 
οἱ ἐμπορευόμενοι εἰς τὴν Ἰνδικὴν γῆν ἐπὶ τῶν καμήλων, καὶ πωλοῦσι τὰς 
τοιαύτας διαγεγραμμένας ταπίδας μάλα τιμίας, καὶ περὶ πολλοῦ αὐτὰς ποι-
οῦνται οἱ Πέρσαι, καὶ ὁ Ἰνδῶν βασιλεὺς δῶρα πέμπει τῷ Περσῶν βασιλεῖ 
τούτων τῶν ταπίδων. 



Aristophanes Gramm., Aristophanis historiae animalium epitome subjunctis Aeliani Timothei aliorumque eclogis 
Chapter 2, section 474, line 20

                       εἰσὶ δὲ καὶ ἀπὸ τῶν προβάτων ταπίδες ἐν τοῖς Ἰνδοῖς 
γινόμεναι, οἷαί περ ἐν Αἰγύπτῳ καὶ Σάρδεσι ποιοῦνται, καὶ βάπτουσι κἀκεῖνα 
καὶ ποικίλλουσιν, ἀλλ' οὐδὲν ὅμοιά εἰσι ταῖς ἐκ τῶν καμήλων ταπίσι γι-
νομέναις. 



Aristophanes Gramm., Aristophanis historiae animalium epitome subjunctis Aeliani Timothei aliorumque eclogis 
Chapter 2, section 503, line 3

                  ἐπεπίστευτο δὲ πρὸ τοῦδε τοῦ ζῴου τῆς Ἰνδῶν μόνης 
φωνῆς ἐπαΐειν τοὺς ἐλάφους. 



Aristophanes Gramm., Aristophanis historiae animalium epitome subjunctis Aeliani Timothei aliorumque eclogis 
Chapter 2, section 518, line 1

              Ὅτι τὰ ἐν Ἰνδοῖς πρόβατα οὐρὰς πήχεως λέγει 
Κτησίας ἔχειν. 



Aristophanes Gramm., Aristophanis historiae animalium epitome subjunctis Aeliani Timothei aliorumque eclogis 
Chapter 2, section 539, line 2

   ἐν 
Ἰνδοῖς δὲ Κωϋθά τις κώμη λέγεται, καὶ ταῖς αἰξὶ ταῖς <ἐπιχω>ρί<ο>ις οἱ 
νομεῖς ἰχθύας ξηροὺς παραβάλλουσιν ὡς χιλόν. 



Aristophanes Gramm., Aristophanis historiae animalium epitome subjunctis Aeliani Timothei aliorumque eclogis 
Chapter 2, section 544, line 1

                                  Προβατεῖαι <δὲ Ἰ>νδῶν ὁποῖαι 
μαθεῖν ἄξιον. 



Aristophanes Gramm., Aristophanis historiae animalium epitome subjunctis Aeliani Timothei aliorumque eclogis 
Chapter 2, section 544, line 4

                 τὰς αἶγας [δὲ] καὶ τὰς ὄϊς ὄνων <τῶν μ>εγίστων μείζονας 
ἀκούω καὶ ἀποκύειν τέτταρα ἑκάστην· μεί<ω γε μὴν> τῶν τριῶν οὔτε αἲξ 
Ἰνδικὴ οὔτ' ἂν ὄϊς ποτὲ τέκοι. 



Aristophanes Gramm., Aristophanis historiae animalium epitome subjunctis Aeliani Timothei aliorumque eclogis 
Chapter 2, section 553, line 1

Ὕεται Ἰνδῶν ἡ γῆ δ<ιὰ τοῦ ἦρος μέλιτι> ὑγρῷ καὶ ἔτι πλέον 
ἡ Πραισίων χώρα, ὅπερ οὖν ἐμπ<ῖπ>τον <ταῖς πό>αις καὶ ταῖς τῶν ἑλείων 
καλάμων κόμαις, νομὰς τοῖς <βουσὶ> καὶ τοῖς προβάτοις παρέχει θαυμαστάς, 
καὶ τὰ μὲν ζ<ῷα ἑστιᾶ>ται τήνδε τὴν ἡδίστην τροφήν· μάλιστα γὰρ ἐνταῦθα 
οἱ νο<μεῖς ἄγους>ιν αὐτά, ἔνθα καὶ μᾶλλον ἡ δρόσος ἡ γλυκεῖα κάθηται πε-
σοῦσα, ἀντεφεστιᾷ δὲ καὶ τὰ ζῷα τοὺς νομέας· ἀμέλγουσι γὰρ περιγλύκιστον 
γάλα, καὶ οὐ δέονται ἀναμίξαι αὐτῷ μέλι, ὅπερ δρῶσιν Ἕλληνες. 



Aristophanes Gramm., Aristophanis historiae animalium epitome subjunctis Aeliani Timothei aliorumque eclogis 
Chapter 2, section 556, line 1

                      Τὰ πρόβατα τῶν Ἰνδῶν αἵ τε αἶγες μεί-
ζονες ὄνων τῶν μεγίστων εἰσὶ καὶ <τίκτει ........ ἑκά>στη ὄϊς καὶ αἲξ ὡς 
ἐπὶ τὸ πολύ· τριῶν δὲ <οὔτε τις ὄϊς οὔτε αἲξ> ἐλάσσω τίκτει· αἱ δὲ πλεῖσται 
τές<σαρα. 



Aristophanes Gramm., Aristophanis historiae animalium epitome subjunctis Aeliani Timothei aliorumque eclogis 
Chapter 2, section 572, line 2

                      Ὗς οὔτε ἥμερός ἐστιν οὔτε ἄγριος ἐν τῇ 
Ἰνδικῇ ὅλως γῇ, οὐδ' ἂν φάγοι Ἰνδῶν οὐδεὶς ὑὸς κρέας οὐδέν περ μᾶλλον 
ἢ ἀνθρώπου. 



Aristophanes Gramm., Aristophanis historiae animalium epitome subjunctis Aeliani Timothei aliorumque eclogis 
Chapter 2, section 597, line 1

Οἱ δὲ Ἄραβες οἱ πρὸς τῷ ὄρει τῆς Ἰνδικῆς ὄντες ὑψαύχενοί 
εἰσι καὶ ποδώκεις, ὀξεῖς καὶ κούφως πηδῶντες, γοργοὶ τὸ βλέμμα, πρὸς τὸ 
καῦμα μὴ ἀπαγορεύοντες, ἱπποτυφίας γέμοντες ἀληθῶς. 



Aristophanes Gramm., Aristophanis historiae animalium epitome subjunctis Aeliani Timothei aliorumque eclogis 
Chapter 2, section 612, line 1

Ἵππους μονοκέρους γῆ Ἰνδικὴ τίκτει. 



Aristophanes Gramm., Aristophanis historiae animalium epitome subjunctis Aeliani Timothei aliorumque eclogis 
Chapter 2, section 623, line 1

Ἵππον δὲ ἄρα Ἰνδὸν κατασχεῖν καὶ ἀνακροῦσαι προπηδῶντα 
καὶ ἐκθέοντα οὐ παντὸς ἦν, ἀλλὰ τῶν ἐκ παιδὸς ἱππείαις πεπαιδευμένων. 

\end{greek}

\section{Callisthenes}
\blockquote[From Wikipedia\footnote{\url{http://en.wikipedia.org/wiki/Callisthenes}}]{Callisthenes of Olynthus (in Greek Καλλισθένης; ca. 360-328 BC) was a Greek historian. He was the son of Hero (niece of Aristotle), the daughter of Proxenus of Atarneus and Arimneste, which made him the great nephew of Aristotle by his sister Arimneste. They first met when Aristotle tutored Alexander the Great. Through his great-uncle's influence, he was later appointed to attend Alexander the Great on his Asiatic expedition as a professional historian.

Fate of Callisthenes

During the first years of Alexander's campaign in Asia, Callisthenes showered praises upon the Macedonian conqueror. As the king and army penetrated further into Asia, however, Callisthenes' tone began to change. He began to sharply criticize Alexander's adoption of Persian customs, with special scorn for Alexander's growing desire that those who presented themselves before him perform the servile ceremony of proskynesis. Callisthenes was later implicated in a treasonous conspiracy and thrown into prison, where he died from torture or disease.

His death was commemorated in a special treatise (Callisthenes or a Treatise on Grief) by his friend Theophrastus, whose acquaintance he made during a visit to Athens. There are nevertheless several different accounts of how he was executed. Crucifixion is the method suggested by Ptolemy, but Chares of Mytilene and Aristobulos agree that he died of natural causes while in prison.[1]
Writings of Callisthenes

Callisthenes wrote an account of Alexander's expedition up to the time of his own execution, a history of Greece from the Peace of Antalcidas (387) to the Phocian war (357), a history of the Phocian war, and other works, all of which have perished. However, his account of Alexander's expedition was preserved long enough to be mined as a direct or indirect source for other histories that have survived. Polybius scolds Callisthenes for his poor descriptions of the battles of Alexander.[2]

A quantity of the more legendary material coalesced into a text known as the Alexander Romance, the basis of all the Alexander legends of the Middle Ages, originated during the time of the Ptolemies, but in its present form belongs to the 3rd century AD. Its author is usually known as pseudo-Callisthenes, although in the Latin translation by Julius Valerius Alexander Polemius (beginning of the 4th century) it is ascribed to a certain Aesopus; Aristotle, Antisthenes, Onesicritus and Arrian have also been credited with the authorship.

There are also Syrian, Armenian and Slavonic versions, in addition to four Greek versions (two in prose and two in verse) in the Middle Ages (see Krumbacher, Geschichte der byzantinischen Literatur, 1897, p. 849). Valerius's translation was completely superseded by that of Leo, arch-priest of Naples in the 10th century, the so-called Historia de Preliis.}

\begin{greek}

Callisthenes Hist., Testimonia (0534: 001)
“FGrH \#124”.
Volume-Jacobyʹ-T 2b,124,T, fragment 7, line 74

                                                        ... 
 ἀποθανεῖν δὲ αὐτὸν οἱ μὲν ὑπ' Ἀλεξάνδρου κρεμασθέντα λέγουσιν 
(Ptolem. 138 F 17), οἱ δὲ ἐν πέδαις δεδεμένον καὶ νοσήσαντα (Aristob. 139 
F 33)· Χάρης (125 F 15) δὲ μετὰ τὴν σύλληψιν ἑπτὰ μῆνας φυλάττεσθαι 
δεδεμένον, ὡς ἐν τῶι συνεδρίωι κριθείη παρόντος Ἀριστοτέλους, ἐν αἷς δὲ 
ἡμέραις Ἀλέξανδρος ἐν Μαλλοῖς Ὀξυδράκαις ἐτρώθη περὶ τὴν Ἰνδίαν, 
ἀποθανεῖν ὑπέρπαχυν γενόμενον καὶ φθειριάσαντα. 

\end{greek}


\section{Duris of Samos}
\blockquote[From Wikipedia\footnote{\url{http://en.wikipedia.org/wiki/Duris}}]{

Duris of Samos (Greek Δοῦρις); c. 350 BC – after 281 BC) was a Greek historian and was at some period tyrant of Samos.

Writings

Duris was the author of a narrative history of events in Greece and Macedonia from the battle of Leuctra (371 BC) down to the death of Lysimachus. This work, like all his others, is lost; over thirty fragments are known through quotations by other authors, including Plutarch. It was continued in the Histories of Phylarchus. Other works by Duris included a life of Agathocles of Syracuse, which was a source for books 19-21 of the Historical Library of Diodorus Siculus. Duris also wrote historical annals of Samos arranged according to the lists of the priests of Hera; and a number of treatises on literary and artistic subjects.
List of Works

For the surviving fragments see the editions by Müller and Jacoby.

    Histories (also listed as Macedonica and Hellenica; 33 fragments)
    On Agathocles (also listed as Libyca; 13 fragments)
    Annals of Samos (22 fragments)
    On Laws (2 fragments)
    On Games (4 fragments)
    On Tragedy (and perhaps On Euripides and Sophocles; 2 fragments)
    On Painters (2 fragments)
    On Sculpture (1 fragment)
}
\begin{greek}

Duris Hist., Fragmenta (1339: 005)
“FHG 2”, Ed. Müller, K.
Paris: Didot, 1841–1870.
Fragment 71, line 2

Etym. M. v. Θώραξ: Δοῦρις ὁ Σάμιος ἐν τῷ Περὶ 
νόμων φησὶν, ὅτι Διόνυσος ἐπιστρατεύσας Ἰνδοῖς καὶ 
μὴ δυνάμενος αὐτοὺς χειρώσασθαι κρατῆρα 
οἴνου πληρώσας πρὸ τῆς χώρας αὐτῶν 
ἔθηκεν· οἱ δὲ ἐμφορηθέντες τοῦ πόματος, ἀσυνήθεις 
ὄντες, οὕτως ἐχειρώθησαν μεθυσθέντες. 
\end{greek}


\chapter{Roman empire, Greek and Latin sources}%rom
\minitoc


\section{Dionysius Periegetes}
\begin{greek}

Paraphrases In Dionysium Periegetam, In Dionysii periegetae orbis descriptionem (4174: 001)
“Geographi Graeci minores, vol. 2”, Ed. Müller, K.
Paris: Didot, 1861, Repr. 1965.
Section 36-42, line 2

Ὅπου δὲ τὸ πρῶτον ἐν τῇ ἀνατολῇ ἀνατέλλει 
τοῖς ἀνθρώποις τὸ πρὸς ἕω μέρος, ἠῷον καὶ Ἰνδικὸν κα-
λεῖται τὸ κῦμα ἐκεῖνο τῆς θαλάσσης, τουτέστι τὸ μέρος 
ἐκεῖνο τοῦ ὠκεανοῦ. 



Paraphrases In Dionysium Periegetam, In Dionysii periegetae orbis descriptionem 
Section 570-579, line 17

                                      Οὐδαμῶς γὰρ 
οὕτως ἐπὶ τοῖς αἰγιαλοῖς τῆς Θρᾳκικῆς Ἀψύνθου οἱ 
Βιστονίδες, τουτέστιν οἱ Θρᾷκες, τὸν ἐρίβρομον 
Εἰραφιώτην, τουτέστι τὸν Βάκχον τὸν ῥαφθέντα 
ποτὲ ἐν τῷ τοῦ Διὸς μηρῷ, ἢ ἀπὸ πόλεως Εἰραφίας 
καλουμένης οὕτως λεγόμενον καλοῦσιν, ἐν τοῖς μυ-
στηρίοις αὐτοῦ μέγαν ἦχον διεγείροντες· οὐδὲ οὕτω 
σὺν τοῖς παισὶν αὐτῶν κατὰ τὸν Γάγγην ποταμὸν 
τὸν μελαίνας στροφὰς ἔχοντα τῷ μεγαλοήχῳ Διονύσῳ 
οἱ Ἰνδοὶ τὸν κῶμον ἄγουσιν, ὡς κατὰ ταύτην τὴν χώ-
ραν αἱ γυναῖκες τῶν Ἀμνιτῶν τὸ εὐοῖ Βάκχε, τουτέστι   
τὸν ὕμνον τὸν εἰς τὰ Διονύσια τελούμενον, λέγουσιν. 



Paraphrases In Dionysium Periegetam, In Dionysii periegetae orbis descriptionem 
Section 580-588, line 13

                                                  Τηνι-
καῦτα γὰρ λοξότερος ὁ ἥλιος φέρεται καὶ τὸν ὁρίζοντα 
ἐκκλινόμενος, κατὰ εὐθεῖαν τῶν ἀκτίνων αὐτοῦ ἐρχο-
μένων· μυλοειδῶς γὰρ αὐτὴν περιστρέφεται καὶ εἰς 
αὐτὴν ἐφορᾷ καθ' ὅλον τὸν κύκλον, ἕως ὅτου πάλιν 
ἔλθῃ ἐπὶ τοὺς τὰ νότια μέρη, ἤγουν τὸ ἀντικείμενον 
νότιον ἡμικύκλιον, κατοικοῦντας Ἰνδοὺς καὶ Αἰθίο-
πας. 



Paraphrases In Dionysium Periegetam, In Dionysii periegetae orbis descriptionem 
Section 620-626, line 11

Τὸ δὲ σχῆμα καὶ ἡ μορφὴ τῆς Ἀσίας ἐστὶν 
ὁ ῥυσμὸς, τουτέστιν ἡ ἔκτασις καὶ ἡ περιγραφὴ, τῶν 
ἀμφοτέρων ἠπείρων, τῆς τε Εὐρώπης καὶ Λιβύης, 
ὅμοιον δέ ἐστι τὸ σχῆμα καὶ παραπλήσιον κώνῳ, ὅπερ 
ἐστὶ σχῆμα στροβιλοειδὲς, κατὰ τὸ ἔτερον μέρος ἢ τὸ 
ἐναντίον ἑλκόμενον κατὰ μικρὸν, καὶ ἀποξυνόμενον 
πρὸς τὸ ἔσχατον πέρας τῆς ἀνατολῆς, τῶν ἑτέρων δύο 
ἠπείρων πρὸς τὴν δύσιν κειμένων· ὅπου καὶ τοῦ ἐν 
Θήβῃ γεννηθέντος Διονύσου αἱ στῆλαι περὶ τὰ ἔσχατα 
τοῦ ὠκεανοῦ ἵστανται ἐν τοῖς ἐσχάτοις μέρεσι τῶν 
Ἰνδῶν, ὅπου ὁ Γάγγης τὸ Νυσσαῖον λευκὸν ὕδωρ ἢ τὸ 
ἀπὸ Νύσσης κατερχόμενον ἐπὶ τὸν πλαταμῶνα κυλίει. 



Paraphrases In Dionysium Periegetam, In Dionysii periegetae orbis descriptionem 
Section 636-651, line 5

Ἀμφοτέρων δὲ τῶν δύο τούτων κόλπων 
ἄπειρος ἰσθμὸς μέσον διαχωρίζει, ἔνθα κἀκεῖσε εἰς 
πολλὰ μέρη τοῖς πεδίοις περιτεινόμενος, εἰς τὰ μέσα 
δὲ ἁπάσης τῆς Ἀσίας ὄρος ἐπιβέβηκεν ἀρξάμενον ἀπὸ 
τῆς Παμφυλίας μέχρι τῶν Ἰνδῶν, ἄλλοτε μὲν λοξὸν 
καὶ κυρτὸν, ἄλλοτε δὲ ὀρθότερον ἰχνῶν, Ταῦρον δὲ κα-
λοῦσιν αὐτὸ, διότι ταυροφανές ἐστι καὶ ὀξυκέφαλον, 
ὥστε λοφοῦσθαι εἰς ὕψος πολυσχιδὲς, ὃ ἔστι εἰς πολλὰ 
μέρη διασχιζόμενον· εἰς γὰρ τὰς ἐξοχὰς αὐτοῦ κέρατα 
ταύρου μιμεῖται ἔνθα κἀκεῖσε τοῖς ἐκτεταμένοις ὄρεσιν. 



Paraphrases In Dionysium Periegetam, In Dionysii periegetae orbis descriptionem 
Section 695-705, line 9

             Ἐπὶ δὲ τούτῳ τῷ ἰσθμῷ τὸ ἀνατολικὸν 
ἔθνος κατοικεῖ τῶν Ἰβήρων (οἵτινές ποτε Ἴβηρες ἀπὸ 
τῆς Πυρήνης, ἤγουν τῆς ἑσπερίας ἢ δυτικῆς Ἰβηρίας 
ἐπὶ τὴν ἀνατολὴν παρεγένοντο, τοῖς Ὑρκανίοις ἄν-  
δρασι μεγάλην ἔχθραν συμβαλόντες), καὶ τὸ μέγα 
ἔθνος τῶν Καμαριτῶν, οἵτινές ποτε τὸν Διόνυσον ἀπὸ 
τοῦ πολέμου τῶν Ἰνδῶν προσδεξάμενοι ἐξενοδόχησαν, 
καὶ ἱερὸν χορὸν, ὃ ἔστι μέγαν, μετὰ τῶν Βακχῶν ἔ-
στησαν, ζώματα καὶ νεβρῖδας ἐπὶ τοῖς στήθεσιν αὐτῶν 
περιβαλόμενοι, εὖ οἱ εὐὰν λέγοντες. 



Paraphrases In Dionysium Periegetam, In Dionysii periegetae orbis descriptionem 
Section 887-893, line 6

      Ἐπὶ γὰρ τοῖς προλεχθεῖσι παρ' ἐμοῦ εἰπόντος 
ἀκούσας ἐπίστασαι Ταῦρον τὸ ὄρος πᾶσαν τὴν Ἀσίαν 
διαχωρίζειν ἢ μέσην διατέμνειν μέχρι τῶν Ἰνδῶν. 



Paraphrases In Dionysium Periegetam, In Dionysii periegetae orbis descriptionem 
Section 887-893, line 10

Καὶ τοῦτο μὲν τὸ ὄρος ἐν τῷ τετραπλεύρῳ τὸ βορειό-
τερον ἔστω· ὁ δὲ Νεῖλος ποταμὸς τὸ ἑσπέριον ἢ δυτι-
κώτερον πλεῦρον ὑπαρχέτω· τὸ δὲ ἑῷον, ἤτοι τὸ ἀνα-
τολικώτερον, ὁ Ἰνδικὸς ὠκεανός· τὸ δὲ νότιον ἡ 
Ἐρυθρὰ θάλασσα ἢ τὸ κῦμα τῆς Ἐρυθρᾶς θαλάσσης. 



Paraphrases In Dionysium Periegetam, In Dionysii periegetae orbis descriptionem 
Section 1063-1079, line 15

                                 Καὶ ταύτην μὲν πολλοὶ 
ποταμοὶ ἄρδουσι καὶ περισσῶς πιαίνουσιν, ἔνθα κἀκεῖσε 
ταῖς σκολιαῖς προχύσεσιν ἐλαυνόμενοι· χωρὶς μέν ἐστιν 
ὁ μέγας Κόρος, χωρὶς δὲ ὁ Χόασπις, ὅς ἐστιν ἀπόρροια 
τοῦ Ἰνδοῦ ποταμοῦ, Ἰνδικὸν ὕδωρ ἕλκων καὶ τὴν τῶν 
Σούσων γῆν παραρρέων. 



Paraphrases In Dionysium Periegetam, In Dionysii periegetae orbis descriptionem 
Section 1086-1106, line 4

                               Ἐφ' οἷστισιν ἔθνεσι πρὸς 
τὴν ἀνατολὴν παρὰ τὸν Ἰνδὸν ποταμὸν οἱ νότιοι Σκύ-
θαι, οἱ καὶ Ἰνδοσκύθαι καλούμενοι, κατοικοῦσιν, ὅστις   
Ἰνδὸς κατεναντίον τῆς Ἐρυθρᾶς θαλάσσης κατέρχεται, 
ταχὺν ῥοῦν ὀρθὸν ἐπὶ νότον ἐλαύνων, ἀπὸ τοῦ ὑψηλοῦ 
ὄρους Καυκάσου πρῶτον ἀρξάμενος. 



Paraphrases In Dionysium Periegetam, In Dionysii periegetae orbis descriptionem 
Section 1107-1113, line 1

Πρὸς ἀνατολὰς δὲ ἡ τῶν Ἰνδῶν ἐπέραστος 
γῆ ἐκτείνεται πασῶν ἐσχάτη περὶ τὰ χείλη, ἤτοι τὸν 
αἰγιαλὸν, τοῦ ὠκεανοῦ, καθ' ἣν ὁ ἥλιος ἐπὶ τὰ τῶν θεῶν 
καὶ ἀνθρώπων ἔργα ἀνερχόμενος, τουτέστιν ἀνατέλλων, 
πρώταις ἀκτῖσιν ἐπιλάμπει. 



Paraphrases In Dionysium Periegetam, In Dionysii periegetae orbis descriptionem 
Section 1114-1127, line 1

Τῶν Ἰνδῶν δὲ οἱ μὲν τοῦ χρυσοῦ μέ-
ταλλα ἐργάζονται, εὐκαμπέσι μακέλλαις τὴν γῆν 
σκάπτοντες, οἱ δὲ τοὺς ἐκ τοῦ λίνου κατασκευασθέντας 
ἱστοὺς ὑφαίνουσιν, οἱ δὲ τοὺς λευκοὺς ὀδόντας τῶν ἐλε-
φάντων πρισθέντας ἀποξύουσιν, ἄλλοι δὲ ἐπὶ ταῖς προό-
δοις ἢ ἐκβολαῖς τῶν ξηροποτάμων ἢ τὴν γλαυκὸν λίθον 
τοῦ βηρύλλου ἢ τὸν τιμιώτατον καὶ διαυγῆ ἀδάμαντα 
ἰχνεύουσιν, ἢ τὴν χλωρῶς διαυγάζουσιν ἴασπιν, ἢ καὶ 
τὸν γλαυκὸν λίθον τοῦ θαλασσοειδοῦς τοπάζου, καὶ τὴν 
γλυκερὰν ἀμέθυστον ἡσύχως πως καὶ μετρίως

πορφυ-



Paraphrases In Dionysium Periegetam, In Dionysii periegetae orbis descriptionem 
Section 1128-1140, line 3

                         Καὶ αὐτὴ μὲν ἡ Ἰνδία ἐπὶ ταῖς 
τρισσαῖς πλευραῖς ἁπάσαις οὔσαις λοξαῖς ἥρμοσεν, 
ὁμοία τῷ εἴδει ῥόμβου· ἀλλὰ τὰ μὲν ἑσπέρια ὕδατα ὁ 
Ἰνδὸς ποταμὸς γείτων ὑπάρχων διαχωρίζει, καὶ τὸ   
κῦμα τῆς Ἐρυθρᾶς θαλάσσης τὸ νότιον, ὁ δὲ Γάγγης εἰς 
τὴν ἀνατολὴν, καὶ ὁ Καύκασος τὰ ἀρκτῶα τοῦ οὐρανοῦ 
μέρη. 



Paraphrases In Dionysium Periegetam, In Dionysii periegetae orbis descriptionem 
Section 1128-1140, line 12

                                   Οἱ μὲν Δαρδανέες 
ἔγγιστα τοῦ μεγίστου καὶ ἀπείρου Ἰνδοῦ ποταμοῦ, ὅθεν 
ὁ Ὑδάσπης ναυσίπορος ποταμὸς τὸν Ἀκεσίνην ποταμὸν 
ἀπὸ τῶν ὑψηλῶν σκοπέλων συρόμενον εἰσδέχεται. 



Paraphrases In Dionysium Periegetam, In Dionysii periegetae orbis descriptionem 
Section 1152-1166, line 11

                       Τούτου χάριν τὴν μὲν ὁδὸν Νυς-
σαίαν ἐκάλεσαν, κοσμίως δὲ καὶ ἐπιστημόνως σὺν τοῖς 
υἱοῖς αὐτῶν πάντα ἐτέλεσαν τὰ μυστήρια, ὁπότε αὐτὸς 
τὰ ἔθνη τῶν μελάνων Ἰνδῶν ὤλεσε καὶ τῶν Ἠμωδῶν 
ὀρῶν ἐπέβη, ἀφ' ὧντινων ὀρῶν βάσιν ὁ μέγας ῥοῦς 
τοῦ ἀνατολικοῦ ὠκεανοῦ φέρεται. 



Paraphrases In Dionysium Periegetam, In Dionysii periegetae orbis descriptionem (4174: 002)
“Aristarchs Homerische Textkritik nach den Fragmenten des Didymos, vol. 2”, Ed. Ludwich, A.
Leipzig: Teubner, 1885, Repr. 1971.
Section 37, line 1

   ὅπου δὲ 
τὸ πρῶτον ἐν τῇ ἀνατολῇ ἀνατέλλει τοῖς ἀνθρώποις τὸ πρὸς 
ἕω μέρος, 
         ἠῷον καὶ Ἰνδικὸν καλοῦσι τὸ κῦμα τῆς θαλάτ-
της ἐκείνης – τουτέστι τὸ μέρος ἐκεῖνο τοῦ ὠκεανοῦ – . 



Paraphrases In Dionysium Periegetam, In Dionysii periegetae orbis descriptionem 
Section 1073, line 2

χωρὶς μέν ἐστιν ὁ μέγας Κόρος, χωρὶς δὲ ὁ Χόασπις –  
ὅς ἐστιν ἀπόρροια ἀπὸ τοῦ Ἰνδοῦ ποταμοῦ – 
         Ἰνδικὸν 
ἕλκων ὕδωρ παραρρέων τε τὴν γῆν τῶν Σούσων – οὐδεὶς δὲ 
ἐξ αὐτοῦ, ὡς μυθεύεται, τοῦ Χοάσπιδος πίνει, εἰ μὴ μόνος ὁ 
βασιλεύς – . 



Paraphrases In Dionysium Periegetam, In Dionysii periegetae orbis descriptionem 
Section 1088, line 2

πρὸς δὲ τὴν ἀνατολὴν τούτων τῶν Γεδρωσῶν – ἅπερ 
εἰσὶν ἔθνη – ἡ γῆ ἕλκεται 
         τοῦ μεγακήτεος ὠκεανοῦ 
γείτων, ἐφ' οἷστισιν ἔθνεσι πρὸς τὰς αὐγὰς 
         παρὰ τῷ 
Ἰνδῷ ποταμῷ οἱ νότιοι Σκύθαι – οἱ καὶ Ἰνδοσκύθαι καλού-
μενοι – κατοικοῦσιν, 
         ὅστις Ἰνδικὸς κατέναντι τῆς ἐρυ-
θρᾶς θαλάσσης κατέρχεται 
         ταχὺν ῥοῦν ἐπὶ τὸν νότον 
ὀρθὸν ἐλαύνων, 
         ἀπὸ τοῦ ἠνεμόεντος, ὅ ἐστιν ὑψηλοῦ, 
ὄρους Καυκάσου τὸ πρῶτον ἀρξάμενος – ὅθεν καὶ ὁ Εὐφράτης 
κατέρχεται. 



Paraphrases In Dionysium Periegetam, In Dionysii periegetae orbis descriptionem 
Section 1093, line 2

   δύο δὲ στόματα αὐτοῦ ὑπάρχει· μέσῃ δὲ   
τῇ ἐρυθρᾷ θαλάσσῃ – ὡς ἀπὸ τῶν στομάτων καὶ τοῦ ὠκεανοῦ 
– ποιῶν νῆσον κατέδραμεν, 
         ἣν καὶ Παταλήνην καλοῦσιν 
οἱ ἄνθρωποι – οὕτως καὶ γὰρ οἱ Ἰνδοὶ Πάταλα ὀνομάζουσιν – . 



Paraphrases In Dionysium Periegetam, In Dionysii periegetae orbis descriptionem 
Section 1096, line 2

ἐκεῖνος δὲ ὁ ποταμὸς πολλῶν ἀνδρῶν ἔθνη ἀποτέμνεται· 
δύνοντος μὲν ἐπὶ τὴν κλίσιν, ἤτοι τὴν δύσιν, τοῦ ἡλίου 
τούς τε Ὠρείτας καὶ Ἄριβας – Ἄριβες δὲ καὶ Ὠρεῖται 
οἱ αὐτοί, οἱ καὶ δυτικώτεροι ὄντες τοῦ Ἰνδοῦ – καὶ τοὺς λινᾶς 
χλαίνας ἔχοντας Ἀραχώτας – οὗτοι δὲ μᾶλλον βορειότεροι τῶν 
Ὠρειτῶν εἰσι –  
         καὶ τοὺς Σατραΐδας καὶ ὅσους παρὰ 
τῇ ἐξοχῇ τοῦ Παρνασσοῦ, 
         κοινῶς δ' ὁμοῦ μάλα πάντας 
ἐπωνύμως Ἀριηνοὺς λέγουσιν – εἶπε καὶ γὰρ ἄνωθεν ἐρυθραίων 
Ἀριηνῶν – , 
         οὐκ ἐπέραστον δὲ καὶ καλὴν γῆν κατοι-
κοῦντες, ἀλλ' ὑπὸ τὴν πλήθουσαν λεπτῇ 




Paraphrases In Dionysium Periegetam, In Dionysii periegetae orbis descriptionem 
Section 1107, line 1

πρὸς δ' ἀνατολὰς ἡ ἐπέραστος τῶν Ἰνδῶν ἐπίκειται γῆ, 
πασῶν ἐσχάτη, παρὰ τοῖς χείλεσι τοῦ ὠκεανοῦ, 
         καθ' 
ἣν ἐπὶ τῶν μακαρίων ἀνδρῶν τὰ ἔργα ὁ ἥλιος ἀνερχόμενος 
ταῖς πρώταις ἀκτῖσιν αὐτοῦ ἐπιφλέγει. 



Paraphrases In Dionysium Periegetam, In Dionysii periegetae orbis descriptionem 
Section 1132, line 3

   ἀλλὰ τοῖς ἑσπερίοις 
μὲν ὕδασι τὴν γῆν ἀποτμήγει, ὅ ἐστι διαχωρίζει, ὁμούριος, ἀντὶ 
τοῦ γείτων, ὁ Ἰνδὸς ὑπάρχων, 
         καὶ τὸ κῦμα τῆς ἐρυθρᾶς 
θαλάσσης τὸ νότιον, 
         ὁ δὲ Γάγγης εἰς τὰς αὐγάς, καὶ 
ὁ Καύκασος ἐπὶ τοῖς ἀρκτώοις μέρεσι. 



Paraphrases In Dionysium Periegetam, In Dionysii periegetae orbis descriptionem 
Section 1137, line 2

   καὶ ταύτην μὲν 
πολλοὶ καὶ πλούσιοι ἄνδρες κατοικοῦσιν, 
         οὐχ ὁμώνυμοι 
καὶ ὁμοῦ ὄντες, ἀλλ' ἀμφοτέρωθεν 
         κεχωρισμένοι, οἱ μὲν 
Δαρδανέες ἔγγιστα τοῦ ἀπείρου Ἰνδοῦ ποταμοῦ, 
         ὅθεν 
ἀπὸ τῶν σκοπέλων – δηλονότι τῶν ὑψηλῶν τόπων, ἀφ' ὧν 
ἔστι κατασκοπήσασθαι – ὁ ναυσίπορος Ὑδάσπης λοξὸν συρό-
μενον τὸν Ἀκεσίνην εἰσδέχεται. 



Paraphrases In Dionysium Periegetam, In Dionysii periegetae orbis descriptionem 
Section 1161, line 2

   τούτου χάριν τὴν μὲν 
ὁδὸν Νυσσαίαν ἐκάλεσαν, 
         κοσμίως δὲ καὶ ἐπιστημόνως 
σὺν τοῖς υἱοῖς αὐτῶν πάντα ἐτέλεσαν τὰ μυστήρια, 
         ὁπότε 
αὐτὸς τὰ ἔθνη τῶν μελάνων Ἰνδῶν ὤλεσε 
         καὶ τῶν 
Ἠμωδῶν ὀρέων ἐπέβη, ὑφ' ὧντινων ὀρῶν βάσιν 
         τοῦ 
ἀνατολικοῦ ὠκεανοῦ ὁ μέγας ῥοῦς φέρεται. 


\end{greek}


\section{Galenus}
\blockquote[From Wikipedia\footnote{\url{}}]{}
\begin{greek}

Galenus Med., De temperamentis libri iii (0057: 009)
“Galeni de temperamentis libri iii”, Ed. Helmreich, G.
Leipzig: Teubner, 1904, Repr. 1969.
Kühn volume 1, page 618, line 6

                                                 Αἰγύπτιοι μὲν 
οὖν καὶ Ἄραβες καὶ Ἰνδοὶ καὶ πᾶν τὸ ξηρὰν καὶ θερμὴν 
χώραν ἐποικοῦν ἔθνος μελαίνας τε καὶ δυσαυξεῖς καὶ ξηρὰς 
καὶ οὔλας καὶ κραύρας ἔχουσι τὰς τρίχας. 



Galenus Med., De anatomicis administrationibus libri ix (0057: 011)
“Claudii Galeni opera omnia, vol. 2”, Ed. Kühn, C.G.
Leipzig: Knobloch, 1821, Repr. 1964.
Volume 2, page 544, line 13

       ἐὰν οὖν ποτ' ἴδῃς ζῶον, εἴτ' ἐξ Ἰνδίας ἢ καὶ Λι-
βύης, εἴτε Σκυθίας ἧκον, ὃ μὴ πρόσθεν ἐθεάσω, προσφε-
ρόμενον τροφὴν φρυγανώδη, μεγάλην τούτῳ καὶ τραχεῖαν 
ἴσθι παρεσκευασμένην κοιλίαν· εἰ δὲ καὶ τοὺς ἄνωθεν 
ὀδόντας οὐκ ἔχοι, γαστέρας ἐξ ἀνάγκης τούτῳ πλείους 
ὑπάρχειν, ὥστ' εἰς μὲν τὴν πρώτην εὐθέως καταπίνειν   
τὴν νομὴν, αὖθις δ' ἐκ ταύτης ἐπανεμοῦν κατεργάζεσθαι 
τῷ στόματι, καὶ μετὰ ταῦτα καταπίνειν αὖθις εἰς ἑτέραν 
γαστέρα, κᾄπειτ' ἐκ ταύτης πάλιν εἰς ἑτέραν μεταβάλλειν, 
εἶτ' ἐξ ἐκείνης αὖθις εἰς ἄλλην. 



Galenus Med., De usu partium (0057: 017)
“Galeni de usu partium libri xvii”, Ed. Helmreich, G.
Leipzig: Teubner, 1:1907; 2:1909, Repr. 1968.
Kühn volume 3, page 30, line 7

                                     οὕτως οὖν καὶ μά-
χαιραν ἀρίστην εἶναί φαμεν οὐ τὴν ἐκ κραύρου σιδή-
ρου γεγενημένην, οἷος ὁ παρὰ τοῖς Ἰνδοῖς ἐστι μάλιστα, 
καίτοι τάχιστά γε τέμνουσαν, ἀλλὰ τὴν εἰς τοσοῦτον   
ἥκουσαν σκληρότητος, ὡς μήτ' αὐτὴν θραύεσθαι ῥᾳδίως 
καὶ τέμνειν ἑτοίμως. 



Galenus Med., Thrasybulus sive utrum medicinae sit an gymnasticae hygieine (0057: 033)
“Claudii Galeni Pergameni scripta minora, vol. 3”, Ed. Marquardt, J., Müller, I., Helmreich, G.
Leipzig: Teubner, 1893, Repr. 1967.
Kühn volume 5, page 867, line 16

                                   φημὶ δή σοι καθόλου 
περὶ τῶν ὀνομάτων οὐ μόνον τούτων ἀλλὰ καὶ τῶν 
ἄλλων ἁπάντων οὐδὲν ἔχειν εἰπεῖν σοφόν, ἀλλ' εἴτ' ἐκ 
τῆς τῶν Ἀσσυρίων εἴη φωνῆς τοὔνομα, παρὰ τῶν   
Ἀσσυρίων αὐτῶν μανθάνειν χρῆναι τὸ πρᾶγμα, καθ' 
οὗ τὸ ὄνομα λέγουσιν, εἴτ' ἐκ τῆς τῶν Περσῶν ἢ 
Ἰνδῶν ἢ Ἀράβων ἢ Αἰθιόπων ἢ ὅλως ὡντινωνοῦν, 
ἐκείνων πυνθάνεσθαι. 



Galenus Med., De sanitate tuenda libri vi (0057: 036)
“Galeni de sanitate tuenda libri vi”, Ed. Koch, K.
Leipzig: Teubner, 1923; Corpus medicorum Graecorum, vol. 5.4.2.
Kühn volume 6, page 90, line 6

ὅπως οὖν ταῦτα γίνοιτο, χρὴ προθερμάναντα μετρίως ἀνατρίψαντά τε 
ςινδόνι τὸ σύμπαν σῶμα κἄπειτα δι' ἐλαίου τρίβειν. 



Galenus Med., De sanitate tuenda libri vi 
Kühn volume 6, page 95, line 4

                                                 ἐπειδὰν δὲ τὰς μὲν τῷ μετὰ 
πλείονος ἐλαίου, τὰς δὲ τῷ μετὰ ἐλάττονος ἢ παντάπασιν ἐλαίου χω-
ρίς, ἤτοι διὰ τῶν χειρῶν μόνον ἢ μετὰ κόνεως ἢ διὰ ςινδόνων, καὶ 
τούτων ἤτοι σκληρῶν ἢ μαλακῶν γίνεσθαι διαφέρειν ἀλλήλων λέγωσι 
τὰς τρίψεις, αἰτίων καταρίθμησιν ποιοῦνται τῶν ἤτοι σκληρὰν ἢ μαλα-
κὴν ἀπεργαζομένων τὴν τρίψιν. 



Galenus Med., De sanitate tuenda libri vi 
Kühn volume 6, page 187, line 1

                            πρῶτον μὲν | οὖν ἀνατριβέσθω ςινδόσιν ἐπι-
πλέον ἢ πρόσθεν· ἔστωσαν δὲ καὶ σφοδρότεραι νῦν μᾶλλον ἢ πρό-
σθεν αἱ τρίψεις καὶ διὰ σκληροτέρων ὀθονίων· εἰ δὲ καὶ χειρῖδας 
ῥαπτὰς περιθέμενοι ταῖς χερσὶν οἱ προγυμνασταὶ τρίβοιεν, ὡς ὁμαλω-
τέραν γενέσθαι τὴν ἐνέργειαν, οὐδὲν ἂν εἴη χεῖρον. 



Galenus Med., De sanitate tuenda libri vi 
Kühn volume 6, page 230, line 12

                                                                            τὰς δ' 
ἐπὶ πλέοσι ποτοῖς ὑγρότητας τρίψεις ξηραὶ μόναι θεραπεύουσι διά τε   
ςινδόνων ἢ χειριδίων ἐπιτελούμεναι καὶ αὐτῶν μόνων ἐνίοτε τῶν 
χειρῶν ἢ χωρὶς λίπους παντὸς ἢ σὺν ἐλαίῳ ἐλαχίστῳ τινί. 



Galenus Med., De sanitate tuenda libri vi 
Kühn volume 6, page 418, line 9

                                       ἐγὼ γοῦν παχύν τινα ἱκανῶς 
ὀλίγῳ χρόνῳ συμμέτρως εὔσαρκον εἰργασάμην ὀξεῖ δρόμῳ χρῆσθαι 
καταναγκάζων, εἶτ' ἀπομάττων μὲν τὸν ἱδρῶτα ςινδόσιν ἤτοι λίαν 
μαλακαῖς ἢ λίαν τραχείαις, ἐφεξῆς δὲ τρίβων ἐπὶ πλεῖστον ἀλείμμασι   
διαφορητικοῖς, ἃ καλοῦσιν ἤδη συνήθως οἱ νεώτεροι τῶν ἰατρῶν 
»ἄκοπα». 



Galenus Med., De sanitate tuenda libri vi 
Kühn volume 6, page 448, line 7

         παραφυλάττειν ἔφην χρῆναι τοὺς τῇ τοιαύτῃ κατασκευῇ σώ-
ματος ἐνοχλουμένους, ἡνίκα μάλιστα φαίνονται πλῆθος ἠθροικέναι σπέρ-
ματος ἀποκρίσεως δεόμενον, ἐν ἡμέρᾳ τινὶ διαιτηθέντας εὐχύμως τε 
καὶ μετρίως χρῆσθαι μὲν ἐπὶ τῷ δείπνῳ τρεπομένους εἰς ὕπνον τῇ 
συνουσίᾳ· κατὰ δὲ τὴν ἑξῆς ἡμέραν, ὅταν αὐτάρκως ἔχωσιν ὕπνου, δι-
αναστάντας ἀνατρίψασθαι ςινδόνι, μέχρις ἂν ἔρευθός τι σχῇ τὸ δέρμα· 
κἄπειτά τινι δι' ἐλαίου τρίψει συμμέτρως χρησαμένους, εἶτα μὴ πολὺ 
διαλείποντας ἄρτον ἄζυμον κριβανίτην καθαρὸν ἐξ οἴνου κεκραμένου 
προσενεγκαμένους, οὕτως ἐπὶ τὰς συνήθεις ἔρχεσθαι πράξεις· ἐν δὲ τῷ 
μεταξὺ τῆς δι' ἐλαίου τρίψεως καὶ τῆς δι' ἄρτου προσφορᾶς, εἰ χωρίον 
ἔχει τι πλησίον ἐπιτήδειον, ἐμπεριπατῆσαι τούτῳ, πλὴν εἰ κρύος εἴη 
χειμέριον· ἄμεινον γὰρ ἔνδον μένειν τηνικαῦτα. 



Galenus Med., Ad Glauconem de medendi methodo libri ii (0057: 067)
“Claudii Galeni opera omnia, vol. 11”, Ed. Kühn, C.G.
Leipzig: Knobloch, 1826, Repr. 1965.
Volume 11, page 110, line 5

                                           ἦν δ' ἐκεῖνα τὰ δι' 
ἀψινθίου κόμης καὶ μυροβαλάνου πιέσματος καὶ νάρδων ἀμφο-
τέρων Ἰνδικῆς τε καὶ Κελτικῆς, ἔτι δὲ κρόκου καὶ οἰνάνθης καὶ 
μαστίχης Χίας καὶ μύρων τῶν διὰ ναρδοστάχυος σκευαζομένων, 
ἔτι δὲ μαστιχίνου καὶ σχινίνου καὶ μηλίνου καὶ οἰνανθίνου. 



Galenus Med., De purgantium medicamentorum facultate (0057: 072)
“Galeni de purgantium medicamentorum facultate”, Ed. Ehlert, J., 1959; Diss. Göttingen.
Kühn volume 11, page 331, line 3

        ταῦτα εἰ μὲν ἐν Ἰνδοῖς τις ὢν γράφων προσεκα-
λεῖτο, χαλεπὸν ἦν ἐξελέγχειν αὐτόν. 



Galenus Med., De simplicium medicamentorum temperamentis ac facultatibus libri xi (0057: 075)
“Claudii Galeni opera omnia, vols. 11–12”, Ed. Kühn, C.G.
Leipzig: Knobloch, 1826, Repr. 1965.
Volume 11, page 405, line 6

                                            τὴν ἀρχὴν γὰρ, οὐδὲ 
μεταβάλλεται πρὸς τῆς ἐν ἡμῖν θερμασίας, ἵνα ἀντιθερμάνῃ 
διὰ τὸ μὴ δύνασθαι καταθραυσθῆναι χνοωδῶς· ἐπεὶ ὅ γε 
κάλαμος ὁ ἐξ Ἰνδίας, ὃν καὶ ἀρωματίτην ὀνομάζουσιν, τῷ 
κοπτεσθαί τε καὶ διασσᾶσθαι καὶ ὅλως τῷ καταθραύεσθαι, 
μᾶλλον τοῦ παρ' ἡμῖν ἐναργῶς φαίνεται θερμαίνων, ὡς καὶ 
τοῖς πρεσβυτέροις ἡμῶν ἰατροῖς ὡμολόγηται πᾶσιν. 



Galenus Med., De simplicium medicamentorum temperamentis ac facultatibus libri xi 
Volume 11, page 821, line 18

                                                ἀρίστη δὲ καὶ ἡ 
κατὰ τὴν Ἰνδίαν, ἧς ὀπός ἐστιν τὸ κομιζόμενον ἐνταυθοῖ 
τοῦτο φάρμακον ἡ ἀλόη προσαγορευομένη, χρείαν παμπόλ-  
λην παρεχόμενον ἐκ τοῦ ξηραίνειν ἀδήκτως. 



Galenus Med., De simplicium medicamentorum temperamentis ac facultatibus libri xi 
Volume 12, page 64, line 11

                                                    τὸ δ' ἕτερον, 
τὸ Ἰνδικὸν, ἰσχυρότερόν ἐστιν εἰς ἅπαν. 



Galenus Med., De simplicium medicamentorum temperamentis ac facultatibus libri xi 
Volume 12, page 66, line 6

                        Μάκερ φλοιός ἐστιν ἐκ 
τῆς Ἰνδικῆς κομιζόμενος, ἐν μὲν τῷ γεύεσθαι στρυφνὸς ἱκα-
νῶς, μετά τινος βραχείας δριμύτητος ἀρωματιζούσης· ὀσμώ-
μενος δὲ ἡδὺς ὁμοίως τοῖς πλείστοις ἀρώμασι τοῖς Ἰνδικοῖς. 



Galenus Med., De simplicium medicamentorum temperamentis ac facultatibus libri xi 
Volume 12, page 71, line 3

                                                           καὶ τὸ 
σάκχαρ δὲ καλούμενον, ὅπερ ἐξ Ἰνδίας τε καὶ τῆς εὐδαίμο-
νος Ἀραβίας κομίζεται, περιπήγνυται μὲν, ὥς φασι, καλά-
μοις, ἔστι δέ τι καὶ αὐτὸ μέλιτος εἶδος. 



Galenus Med., De simplicium medicamentorum temperamentis ac facultatibus libri xi 
Volume 12, page 85, line 2

                                       ἰσχυροτέρα δ' ἐστὶν ἡ 
Ἰνδικὴ προσαγορευομένη, μελαντέρα τῆς Συριακῆς καλουμέ-
νης ὑπάρχουσα. 



Galenus Med., De simplicium medicamentorum temperamentis ac facultatibus libri xi 
Volume 12, page 152, line 12

                           ἀρωματώδης πώς ἐστιν ἡ τῆσδε 
τῆς πόας ῥίζα, νάρδῳ παραπλησία τὴν δύναμιν, ἀλλ' εἰς 
μὲν τὰ πλεῖστα καταδεεστέρα, προτρέπει δ' οὖρα τῆς μὲν 
Ἰνδικῆς καὶ Συριακῆς νάρδου μᾶλλον, ὁμοίως δὲ τῇ Κελτικῇ. 



Galenus Med., De simplicium medicamentorum temperamentis ac facultatibus libri xi 
Volume 12, page 207, line 2

       Περὶ χλωροῦ ἰάσπεώς τε καὶ ὀμφατίτεως καὶ ἱε-
ρακίτου καὶ Ἰνδικοῦ. 



Galenus Med., De simplicium medicamentorum temperamentis ac facultatibus libri xi 
Volume 12, page 207, line 16

                                   ἀλλ' ἔξω τῆς κατὰ μέθοδον 
χρήσεως αἱ τοιαῦται δυνάμεις εἰσὶν, ὥσπερ γε καὶ τοῦ ἱερα-
κίτου τε καὶ Ἰνδικοῦ τὸ ἐκ τῶν αἱμοῤῥοΐδων ἱστῶντος αἷμα. 



Galenus Med., De simplicium medicamentorum temperamentis ac facultatibus libri xi 
Volume 12, page 215, line 18

                                                       προσθήσω 
δή τι τῷ κατ' αὐτὸ λόγῳ χρήσιμον οὐ περὶ διφρυγοῦς μό-
νον γινώσκειν, ἀλλὰ καὶ περὶ Λημνίας σφραγῖδος καὶ πομ-
φόλυγος καὶ ὀποβαλσάμου καὶ λυκίου τοῦ Ἰνδικοῦ. 



Galenus Med., De simplicium medicamentorum temperamentis ac facultatibus libri xi 
Volume 12, page 216, line 8

                                        διὰ τοῦτο τοιγαροῦν ἔς 
τε Λῆμνον καὶ Κύπρον καὶ τὴν Παλαιστίνην Συρίαν ἐσπού-
δασα πορευθεὶς ἑκάστου τῶν φαρμάκων τούτων πολὺ πλῆ-
θος εἰς ὅλον ἐμαυτοῦ παραθέσθαι τὸν βίον, ἀλλὰ καὶ τὸ 
λύκιον τὸ Ἰνδικὸν ἀρτίως ἐνηνεγμένον ἐς Φοινίκην ἅμα 
τῇ Ἰνδικῇ ἀλόῃ κατ' ἐκεῖνον τὸν χρόνον εὐτύχησα λαβεῖν, 
ἡνίκα τὴν ἀπὸ τῆς Παλαιστίνης ὁδὸν ἐπανῄειν, αὐτῷ τε 
τῷ κεκομίσθαι διὰ τῶν καμήλων, σὺν τῷ παντὶ φορτίῳ, 
πεισθεὶς Ἰνδικὸν ὑπάρχειν αὐτὸ καὶ τῷ τὸ νοθευόμενον οὐ 
δύνασθαι πρὸς τῶν κομιζόντων γινώσκεσθαι, τῆς ὕλης ἐξ ἧς 
σκευάζεται κατὰ τοὺς τόπους ἐκείνους μὴ γεννωμένης. 



Galenus Med., De compositione medicamentorum secundum locos libri x (0057: 076)
“Claudii Galeni opera omnia, vols. 12–13”, Ed. Kühn, C.G.
Leipzig: Knobloch, 12:1826; 13:1827, Repr. 1965.
Volume 12, page 458, line 3

                                      καὶ εἰ πλουσίοις σκευά-
ζοις, ἐμβαλεῖς στάχυος νάρδου τῆς τε Ἰνδικῆς καὶ τῆς Κελτι-
κῆς καὶ τῆς ὀρείας καὶ γῆς Ἐρετριάδος, ἐμβαλεῖς δὲ καὶ 
μαλαβάθρου καὶ φύλλου μαλαβάθρου καὶ ἀμώμου καὶ σμύρ-
νης καὶ κρόκου καὶ κόστου καὶ ἡδυχρόου. 



Galenus Med., De compositione medicamentorum secundum locos libri x 
Volume 12, page 589, line 11

                                   λυκίου Ἰνδικοῦ 𐅻 δʹ. 



Galenus Med., De compositione medicamentorum secundum locos libri x 
Volume 12, page 632, line 7

                                ♃ νάρδου Ἰνδικῆς 𐅻 αʹ. 



Galenus Med., De compositione medicamentorum secundum locos libri x 
Volume 12, page 636, line 14

                                                   λυκίου Ἰνδικοῦ 
𐅻 δʹ. 



Galenus Med., De compositione medicamentorum secundum locos libri x 
Volume 12, page 636, line 18

               λυκίου Ἰνδικοῦ, κυτίνων ῥοιᾶς, ῥόδων ἄνθους   
ἀνὰ 𐅻 βʹ. 



Galenus Med., De compositione medicamentorum secundum locos libri x 
Volume 12, page 660, line 8

                        λυκίου Ἰνδικοῦ γο αʹ. 



Galenus Med., De compositione medicamentorum secundum locos libri x 
Volume 12, page 680, line 7

                                             ἐὰν δὲ ᾖ παλαιὰ καὶ 
ὄξος παλαιὸν πρόσμισγε ἢ θεῖον λεῖον, καὶ δριμὺ ὄξος 
εἰς στενόστομον ἀγγεῖον βαλὼν, προστίθει τῇ ἀναπνοῇ αὐ-
τῶν ἐφ' ἱκανὸν, εἶτα μίσυος κυπρίου καὶ νάρδου Ἰνδικῆς τὸ 
ἥμισυ μετὰ χίου οἴνου λεάνας, μέχρι μελιτῶδες γένηται, πτερῷ 
διάχριε τοὺς πόρους καὶ ἐρίῳ ἔμφρασσε. 



Galenus Med., De compositione medicamentorum secundum locos libri x 
Volume 12, page 713, line 15

                                    τοιαῦτα δέ ἐστι κρόκος καὶ 
σμύρνα καὶ λύκιον Ἰνδικὸν, καὶ καστόριον δὲ καὶ λιβανωτὸς 
ἄνευ τοῦ στύφειν πέττει τε ἅμα καὶ διαφορεῖ. 



Galenus Med., De compositione medicamentorum secundum locos libri x 
Volume 12, page 719, line 15

                                                       εὔδηλον δ' ὅτι 
λύκιον λέγω τὸ Ἰνδικὸν, οὐ τουτὶ τὸ καὶ τοῖς παρ' ἡμῖν 
ἔθνεσι γινόμενον. 



Galenus Med., De compositione medicamentorum secundum locos libri x 
Volume 12, page 723, line 1

                  ῥυπτικὸν δέ τι καὶ τὸ καλούμενον Ἀρμένιον   
ἔχει, καθάπερ καὶ τὸ καλούμενον Ἰνδικὸν μέλαν, καὶ διὰ 
τοῦτο τοῖς ἀφλεγμάντοις ἕλκεσιν ἀλύπως ὁμιλεῖ. 



Galenus Med., De compositione medicamentorum secundum locos libri x 
Volume 12, page 730, line 14

νάρδου Ἰνδικῆς 𐅻 αʹ. 



Galenus Med., De compositione medicamentorum secundum locos libri x 
Volume 12, page 731, line 9

                                            νάρδου Ἰνδικῆς δρα-
χμὰς βʹ. 



Galenus Med., De compositione medicamentorum secundum locos libri x 
Volume 12, page 733, line 15

νάρδου Ἰνδικῆς 𐅻 βʹ. 



Galenus Med., De compositione medicamentorum secundum locos libri x 
Volume 12, page 734, line 15

                  νάρδου Ἰνδικῆς, λιβάνου ἄῤῥενος, πεπέρεως 
λευκοῦ ἀνὰ δραχμὴν μίαν, φοινικοβαλάνων ὀστέα δέκα, πάντα 
βαλὼν εἰς ἄγγος κεραμεοῦν ὄπτα φιλοπόνως, ἔπειτα εἰς 
θυίαν κατεράσας καὶ τρίψας ἐπίβαλε ὀποβαλσάμου κοχλιά-  
ρια δύο, ἔπειτα ἀνακόψας καὶ ξηνάνας χρῶ. 



Galenus Med., De compositione medicamentorum secundum locos libri x 
Volume 12, page 735, line 7

                                                        νάρδου Ἰν-
δικῆς 𐅻 αʹ. 



Galenus Med., De compositione medicamentorum secundum locos libri x 
Volume 12, page 735, line 13

νάρδου Ἰνδικῆς, πεπέρεως λευκοῦ ἀνὰ δραχμὴν μίαν, μέλιτος 
Ἀττικοῦ κύαθον αʹ. 



Galenus Med., De compositione medicamentorum secundum locos libri x 
Volume 12, page 743, line 11

                       ♃ Ἀλόης, λυκίου Ἰνδικοῦ, ῥόδων χλω-
ρῶν, κρόκου, ὀπίου, σμύρνης, ἑκάστου τὸ ἴσον, οἴνῳ φυ-
ράσας ἀνάπλασσε τροχίσκους καὶ ξήραινε ἐν σκιᾷ. 



Galenus Med., De compositione medicamentorum secundum locos libri x 
Volume 12, page 747, line 16

                                                       λυκίου Ἰνδι-
κοῦ 𐅻 δʹ. 



Galenus Med., De compositione medicamentorum secundum locos libri x 
Volume 12, page 748, line 7

                                                               λυ-
κίου Ἰνδικοῦ 𐅻 ιβʹ. 



Galenus Med., De compositione medicamentorum secundum locos libri x 
Volume 12, page 753, line 8

                              ἀλόης Ἰνδικῆς 𐅻 ηʹ. 



Galenus Med., De compositione medicamentorum secundum locos libri x 
Volume 12, page 753, line 17

                     σμύρνης δραχμὴν μίαν, νάρδου Ἰνδικῆς 
𐅻 αʹ. 



Galenus Med., De compositione medicamentorum secundum locos libri x 
Volume 12, page 754, line 15

          λυκίου Ἰνδικοῦ δραχμὰς βʹ. 



Galenus Med., De compositione medicamentorum secundum locos libri x 
Volume 12, page 755, line 5

νάρδου Ἰνδικῆς δραχμὰς δʹ. 



Galenus Med., De compositione medicamentorum secundum locos libri x 
Volume 12, page 755, line 5

                                                 λυκίου Ἰνδικοῦ 
δραχμὰς δʹ. 



Galenus Med., De compositione medicamentorum secundum locos libri x 
Volume 12, page 756, line 3

                 νάρδου Ἰνδικῆς 𐅻 αʹ 𐅶ʹʹ. 



Galenus Med., De compositione medicamentorum secundum locos libri x 
Volume 12, page 756, line 8

                     λυκίου Ἰνδικοῦ δραχμὰς βʹ. 



Galenus Med., De compositione medicamentorum secundum locos libri x 
Volume 12, page 756, line 9

                                   νάρδου Ἰνδικῆς 𐅻 βʹ. 



Galenus Med., De compositione medicamentorum secundum locos libri x 
Volume 12, page 757, line 3

λυκίου Ἰνδικοῦ 𐅻 βʹ 𐅶ʹʹ. 



Galenus Med., De compositione medicamentorum secundum locos libri x 
Volume 12, page 757, line 3

                                                          νάρδου Ἰνδι-
κῆς δραχμὰς βʹ. 



Galenus Med., De compositione medicamentorum secundum locos libri x 
Volume 12, page 765, line 4

                                                        νάρδου 
Ἰνδικῆς 𐅻 στʹ. 



Galenus Med., De compositione medicamentorum secundum locos libri x 
Volume 12, page 765, line 17

                     νάρδου Ἰνδικῆς ὀβολὸν αʹ. 



Galenus Med., De compositione medicamentorum secundum locos libri x 
Volume 12, page 766, line 4

                                                 νάρδου Ἰνδικῆς 
ὀβολὸν ἕνα, κόμμεως δραχμὰς γʹ. 



Galenus Med., De compositione medicamentorum secundum locos libri x 
Volume 12, page 767, line 2

          ἀκακίας δραχμὴν μίαν, νάρδου Ἰνδικῆς 𐅻 αʹ. 



Galenus Med., De compositione medicamentorum secundum locos libri x 
Volume 12, page 768, line 2

                                    νάρδου Ἰνδικῆς δραχμὰς βʹ. 



Galenus Med., De compositione medicamentorum secundum locos libri x 
Volume 12, page 772, line 13

                                    νάρδου Ἰνδικῆς 𐅻 δʹ. 



Galenus Med., De compositione medicamentorum secundum locos libri x 
Volume 12, page 774, line 7

                         νάρδου Ἰνδικῆς 𐅻 στʹ. 



Galenus Med., De compositione medicamentorum secundum locos libri x 
Volume 12, page 780, line 18

[Κολλύριον Ἰνδικὸν ἀέρινον ἐπιγραφόμενον προκατα-  
ληπτικὸν ἁπάσης ὀφθαλμίας, ποιεῖ πρὸς ἀμβλυωπίας καὶ 
ψωρώδεις διαθέσεις καὶ πρὸς βεβρωμένους κανθοὺς καὶ οὐ-
λὰς ἀποσμήχει. 



Galenus Med., De compositione medicamentorum secundum locos libri x 
Volume 12, page 781, line 6

          μέλανος Ἰνδικοῦ δραχμὰς ηʹ. 



Galenus Med., De compositione medicamentorum secundum locos libri x 
Volume 12, page 781, line 13

               μέλανος Ἰνδικοῦ 𐅻 ιστʹ. 



Galenus Med., De compositione medicamentorum secundum locos libri x 
Volume 12, page 781, line 18

                                     μέλανος Ἰνδικοῦ 𐅻 ιστʹ. 



Galenus Med., De compositione medicamentorum secundum locos libri x 
Volume 12, page 782, line 6

                      τὸ ὑγρὸν χυλῷ μαράθρου. 
 [Ἰνδικὸν βασιλικὸν ἐπιγραφόμενον, ποιεῖ πρὸς 
ἀρχὰς ὑποχύσεως καὶ πᾶσαν ἀμβλυωπίαν καὶ οὐλὰς ἀπο-
σμήχει. 



Galenus Med., De compositione medicamentorum secundum locos libri x 
Volume 12, page 782, line 9

       μέλανος Ἰνδικοῦ δραχμὰς στʹ. 



Galenus Med., De compositione medicamentorum secundum locos libri x 
Volume 12, page 787, line 8

                                                            νάρδου 
Ἰνδικῆς 𐅻 στʹ. 



Galenus Med., De compositione medicamentorum secundum locos libri x 
Volume 12, page 790, line 3

                                νάρδου Ἰνδικῆς 𐅻 ιβʹ. 



Galenus Med., De compositione medicamentorum secundum locos libri x 
Volume 12, page 825, line 18

                             κόστον Ἰνδικὸν κόψας καὶ σήσας   
λεπτοτάτῳ κοσκίνῳ, ἀναλάμβανε αἰγείᾳ χολῇ καὶ ποιήσας 
κηρωτῆς τὸ πάχος, ἐπίχριε τοὺς ἰόνθους περὶ ἑσπέραν, καὶ 
πρωῒ ἀναστὰς κέλευε γάλακτι τῷ παθόντι προσκλύζεσθαι 
τὸ πρόσωπον. 



Galenus Med., De compositione medicamentorum secundum locos libri x 
Volume 12, page 826, line 13

                        ♃ ὠκίμου σπέρματος, νίτρου ἐρυθροῦ, 
γλήχωνος χλωροῦ, σχιστῆς, κόστου Ἰνδικοῦ, κόψας, σήσας, ὁμοῦ 
ἀναλαβὼν στέατι χηνείῳ ἢ ὀρνιθείῳ, ὥστε κηρωτῶδες ἔχειν τὸ 
πάχος, ἐκ τούτου ἔμπλασσε εἰς ὀθόνιον καὶ ἐπιτίθει μέχρι παν-
τελοῦς ἀπαλλαγῆς. 



Galenus Med., De compositione medicamentorum secundum locos libri x 
Volume 12, page 923, line 15

                          καὶ μέντοι κᾂν πολυτελέστερον ἐθέ-
λῃς ποιῆσαι τὸ φάρμακον, ἐμβαλεῖς καὶ κασσίας φλοιὸν ἢ 
νάρδον Ἰνδικὴν ἢ τὸ τοῦ μαλαβάθρου φύλλον. 



Galenus Med., De compositione medicamentorum secundum locos libri x 
Volume 12, page 924, line 12

              ὀνομάζεται δὲ οὕτω φλοιός τις ἀρωματικὸς 
ἐκ τῆς Ἰνδίας κομιζόμενος, ἱκανῶς θερμαίνων τε καὶ στύ-
φων, ἀλλὰ καὶ ὁ Κυρηναῖος ὀπὸς εἰς διαφόρησιν τῶν σκλη-
ρυνομένων φλεγμονῶν ἐπιτήδειος, ὥσπερ γε καὶ ὁ Μηδικὸς 
ἤ τις τῶν μετ' αὐτούς. 



Galenus Med., De compositione medicamentorum secundum locos libri x 
Volume 12, page 934, line 1

             ὀμφακίου τριώβολον, κρόκου τριώβολον, νάρδου   
Ἰνδικῆς τριώβολον, τούτοις πρόσμιγε μέλιτος κοτύλην μίαν, 
καὶ οὕτω προσεμβαλὼν τὸν χυλὸν, ἐπὶ μαλθακοῦ πυρὸς κι-
νῶν, ἕψε ἕως συστραφῇ, καὶ χρῶ. 



Galenus Med., De compositione medicamentorum secundum locos libri x 
Volume 12, page 936, line 6

                                                         νάρδου Ἰν-
δικῆς 𐅻 αʹ. 



Galenus Med., De compositione medicamentorum secundum locos libri x 
Volume 12, page 938, line 14

                                         νάρδου Ἰνδικῆς, ἀμώμου 
ἀνὰ γο 𐅶ʹʹ. 



Galenus Med., De compositione medicamentorum secundum locos libri x 
Volume 12, page 942, line 13

                                  νάρδου Ἰνδικῆς 𐅻 αʹ. 



Galenus Med., De compositione medicamentorum secundum locos libri x 
Volume 12, page 945, line 8

λυκίου Ἰνδικοῦ 𐅻 αʹ. 



Galenus Med., De compositione medicamentorum secundum locos libri x 
Volume 12, page 946, line 11

                     νάρδου Ἰνδικῆς 𐅻 αʹ. 



Galenus Med., De compositione medicamentorum secundum locos libri x 
Volume 12, page 946, line 18

                                                  νάρδου Ἰνδικῆς 𐅻 αʹ. 



Galenus Med., De compositione medicamentorum secundum locos libri x 
Volume 12, page 962, line 11

           καὶ τούτων δὲ αὐτῶν μέτρια μέν ἐστι τὰ ξηρὰ 
ῥόδα καὶ τὸ ἄνθος αὐτῶν ἀκρέμονές τε ξηρανθέντες, ὧν 
ἄρτι διῆλθον φυτῶν, ὑποκυστίς τε καὶ ῥῆον, ὅ τε τῶν σκυ-
τέων λίθος, ᾧ λαμπρύνουσι τὰ τῶν γυναικῶν ὑποδήματα, 
καλεῖται δὲ ἀγήρατος, ἀλόη τε καὶ κύπρος, ἄνθος τε σχίνου 
καὶ ῥίζα νάρδου τῆς Ἰνδικῆς καὶ τῆς παρ' ἡμῖν ὀρείας. 



Galenus Med., De compositione medicamentorum secundum locos libri x 
Volume 12, page 991, line 8

                     πρὸς ἄφθας λύκιον Ἰνδικὸν σὺν οἴνῳ 
λεανθὲν περιχριόμενον, ποιεῖ καὶ ἐρείκης καρπὸς λεῖος ἢ ῥόδα 
ξηρὰ σὺν μέλιτι. 



Galenus Med., De compositione medicamentorum secundum locos libri x 
Volume 13, page 17, line 8

                                                           νάρ-
δου Ἰνδικῆς δραχμὰς γʹ. 



Galenus Med., De compositione medicamentorum secundum locos libri x 
Volume 13, page 22, line 18

                                                 ♃ νάρδου Ἰν-
δικῆς δραχμὰς δʹ. 



Galenus Med., De compositione medicamentorum secundum locos libri x 
Volume 13, page 27, line 9

                                                    νάρδου Ἰν-
δικῆς 𐅻 δʹ. 



Galenus Med., De compositione medicamentorum secundum locos libri x 
Volume 13, page 52, line 16

                                                      νάρδου Ἰνδι-
κῆς, τερμινθίνης ἀνὰ 𐅻 βʹ. 



Galenus Med., De compositione medicamentorum secundum locos libri x 
Volume 13, page 53, line 17

                                                        νάρδου 
Ἰνδικῆς, ῥόδων ξηρῶν ἀνὰ 𐅻 δʹ. 



Galenus Med., De compositione medicamentorum secundum locos libri x 
Volume 13, page 57, line 9

                        νάρδου Ἰνδικῆς δραχμὰς στʹ. 



Galenus Med., De compositione medicamentorum secundum locos libri x 
Volume 13, page 76, line 14

                    λυκίου Ἰνδικοῦ 𐅻 δʹ. 



Galenus Med., De compositione medicamentorum secundum locos libri x 
Volume 13, page 76, line 17

                                 ♃ κρόκου Κιλικίου, σμύρνης, 
ἴρεως, νάρδου Ἰνδικῆς, λιβάνου, ἀκακίας ἀνὰ γο αʹ. 



Galenus Med., De compositione medicamentorum secundum locos libri x 
Volume 13, page 77, line 9

                                                   νάρδου Ἰνδι-
κῆς γο αʹ. 



Galenus Med., De compositione medicamentorum secundum locos libri x 
Volume 13, page 80, line 3

                                                         λυκίου 
Ἰνδικοῦ γο αʹ. 



Galenus Med., De compositione medicamentorum secundum locos libri x 
Volume 13, page 90, line 14

νάρδου Ἰνδικῆς 𐅻 βʹ. 



Galenus Med., De compositione medicamentorum secundum locos libri x 
Volume 13, page 92, line 15

νάρδου Ἰνδικῆς τετρώβολον, ὕδατι ἀνάπλαττε τροχίσκους 
δυοβολιαίους. 



Galenus Med., De compositione medicamentorum secundum locos libri x 
Volume 13, page 98, line 16

                                   πίσσης ὑγρᾶς βρυτίας τὸ ἴσον, 
νάρδου Ἰνδικῆς, κρόκου, σμύρνης, λιβάνου, πεπέρεως λευκοῦ, 
ἑκάστου ἀνὰ 𐅻 ηʹ. 



Galenus Med., De compositione medicamentorum secundum locos libri x 
Volume 13, page 127, line 3

               ♃ σχοίνου ἄνθους, ξυλοβαλσάμου, μα-
στίχης, κρόκου, νάρδου Ἰνδικῆς, ἀσάρου, κινναμώμου 
ἀνὰ 𐅻 στʹ. 



Galenus Med., De compositione medicamentorum secundum locos libri x 
Volume 13, page 133, line 9

                                                    καὶ ἕψε ἕως 
τὸ ἥμισυ λειφθῇ καὶ διήθησον καὶ λαβὼν ἀλόης Ἰν-
δικῆς λίτραν αʹ. 



Galenus Med., De compositione medicamentorum secundum locos libri x 
Volume 13, page 137, line 15

                                                          ἀλόης Ἰν-
δικῆς 𐅻 αʹ. 



Galenus Med., De compositione medicamentorum secundum locos libri x 
Volume 13, page 145, line 16

                                      νάρδου Ἰνδικῆς 𐅻 δʹ. 



Galenus Med., De compositione medicamentorum secundum locos libri x 
Volume 13, page 159, line 2

           νάρδου Ἰνδικῆς γο αʹ. 



Galenus Med., De compositione medicamentorum secundum locos libri x 
Volume 13, page 166, line 10

                                          νάρδου Ἰνδικῆς 𐅻 δʹ. 



Galenus Med., De compositione medicamentorum secundum locos libri x 
Volume 13, page 170, line 12

                         ἢ οἰνάνθην ὁμοίως τρίψας μετὰ νάρ-
δου Ἰνδικῆς δὸς πιεῖν. 



Galenus Med., De compositione medicamentorum secundum locos libri x 
Volume 13, page 185, line 3

μελιλώτου, ἀμώμου, νάρδου Ἰνδικῆς, κρόκου, σμύρνης, λιβά-
νου, ξυλοκινναμώμου ἀνὰ 𐅻 κεʹ. 



Galenus Med., De compositione medicamentorum secundum locos libri x 
Volume 13, page 185, line 17

φύλλων μαλαβάθρου, κασσίας ῥοδιζούσης, πάνακος, ὀποβαλ-
σάμου, νάρδου Ἰνδικῆς ἀνὰ γο γʹ. 



Galenus Med., De compositione medicamentorum secundum locos libri x 
Volume 13, page 187, line 9

                                                      νάρδου 
Ἰνδικῆς, κρόκου, σμύρνης ἀνὰ μνᾶς δʹ. 



Galenus Med., De compositione medicamentorum secundum locos libri x 
Volume 13, page 202, line 18

                                             νάρδου Ἰνδικῆς 𐅻 αʹ. 



Galenus Med., De compositione medicamentorum secundum locos libri x 
Volume 13, page 204, line 3

                                                νάρδου Ἰνδικῆς 
καὶ Κελτικῆς ἀνὰ 𐅻 στʹ. 



Galenus Med., De compositione medicamentorum secundum locos libri x 
Volume 13, page 205, line 12

                                ♃ νάρδου Ἰνδικῆς 𐅻 βʹ. 



Galenus Med., De compositione medicamentorum secundum locos libri x 
Volume 13, page 213, line 15

                                        ῥήου Ποντικοῦ τῆς ῥίζης, 
σχοίνου ἄνθους, βαλσάμου καρποῦ, κεδρίδων, ἀνίσου, νάρ-
δου Ἰνδικῆς, κρόκου, κινναμώμου, κασσίας, φοῦ, ἀσάρου, πε-
τροσελίνου, χαμαιπίτυος ἀνὰ 𐅻 αʹ 𐅶ʹʹ. 



Galenus Med., De compositione medicamentorum secundum locos libri x 
Volume 13, page 214, line 8

                                                               ♃ 
Χαμαιπίτυος, πρασίου, πετροσελίνου σπέρματος, γεντιανῆς, 
ἄγνου σπέρματος, ἄρκτου χολῆς, νάπυος, σικύου σπέρματος, 
σκολοπενδρίου, πάνακος, μίλτου Λημνίας, ἐρυθροδάνου, κράμ-
βης σπέρματος, ἀριστολοχίας, πεπέρεως, νάρδου Ἰνδικῆς, 
κόστου, σελίνου σπέρματος, εὐζώμου σπέρματος, ἠρυγγίου, 
πολίου, ἐχίου, εὐπατορίου, ἀρκευθίδων ἀνὰ 𐅻 αʹ. 



Galenus Med., De compositione medicamentorum secundum locos libri x 
Volume 13, page 233, line 5

               νάρδου Ἰνδικῆς, ἀσάρου ἀνὰ 𐅻 αʹ. 



Galenus Med., De compositione medicamentorum secundum locos libri x 
Volume 13, page 241, line 2

                                           νάρδου Ἰνδικῆς 𐅻 δʹ. 



Galenus Med., De compositione medicamentorum secundum locos libri x 
Volume 13, page 266, line 5

                                                      ♃ Κρό-
κου, ὀπίου, ὀποπάνακος, σμύρνης, ἴρεως, ἀκόρου, μανδρα-
γόρου, ἀσάρου, κόστου, νάρδου Ἰνδικῆς, δαύκου Κρητικοῦ, 
μήου ἀνὰ 𐅻 ηʹ. 



Galenus Med., De compositione medicamentorum secundum locos libri x 
Volume 13, page 272, line 4

                              ἔνιοι δὲ οὐ τὴν Κρητικὴν χώραν, 
ἀλλὰ τὴν Ἰνδικὴν εἰρῆσθαί φασιν, ἐξ ἧς ὁ ἐλέφας φέρεται. 



Galenus Med., De compositione medicamentorum secundum locos libri x 
Volume 13, page 275, line 5

                     ἔστι μὲν οὖν ἡ Κελτικὴ νάρδος ἀγαθὸν 
φάρμακον εἰς ὅσα περ ἂν καὶ ἡ Ἰνδικὴ, λείπεται δὲ αὐτῆς 
πάμπολυ, καθάπερ γε καὶ ταύτης ἡ ὄρειος. 



Galenus Med., De compositione medicamentorum secundum locos libri x 
Volume 13, page 276, line 18

          νάρδου Ἰνδικῆς ἴσον, καστορίου 𐅻 στʹ. 



Galenus Med., De compositione medicamentorum secundum locos libri x 
Volume 13, page 278, line 10

                                                     πετροσελίνου, 
νάρδου Ἰνδικῆς, πεπέρεως λευκοῦ καὶ μακροῦ ἀνὰ 𐅻 γʹ. 



Galenus Med., De compositione medicamentorum secundum locos libri x 
Volume 13, page 280, line 18

         καστορίου, κόστου, ὀπίου, μανδραγόρου φλοιοῦ, νάρ-
δου Ἰνδικῆς, ἀσάρου, Ἰλλυρίδος, ἀκόρου, σμύρνης ἀνὰ 𐅻 στʹ. 



Galenus Med., De compositione medicamentorum secundum locos libri x 
Volume 13, page 287, line 11

                                          ♃ Κράμβης ἀγρίας σπέρ-
ματος, φοῦ, κινναμώμου, ἴρεως Ἰλλυρικῆς, ἀκόρου, ἀσάρου, 
ὀποπάνακος, σμύρνης, νάρδου Ἰνδικῆς, ὀπίου, κόστου, μαν-
δραγόρου χυλοῦ, καστορίου ἀνὰ 𐅻 στʹ. 



Galenus Med., De compositione medicamentorum secundum locos libri x 
Volume 13, page 288, line 4

                                            ♃ κρόκου, ὀπίου, ὀπο-
πάνακος, σμύρνης, ἴρεως, ἀκόρου, μανδραγόρου, ἀσάρου, κό-
στου, νάρδου Ἰνδικῆς, δαύκου Κρητικοῦ, μήου ἀνὰ 𐅻 στʹ. 



Galenus Med., De compositione medicamentorum secundum locos libri x 
Volume 13, page 289, line 3

                                                       ἀρνογλώσσου 
σπέρματος ἴσον καὶ τοῦ χυλοῦ ἴσον, λυκίου Ἰνδικοῦ 𐅻 αʹ. 



Galenus Med., De compositione medicamentorum secundum locos libri x 
Volume 13, page 290, line 9

                                                                  λυκίου 
Ἰνδικοῦ 𐅻 δʹ. 



Galenus Med., De compositione medicamentorum secundum locos libri x 
Volume 13, page 290, line 15

                       σμύρνης, ὑποκυστίδος χυλοῦ, ἀλόης, ὀπίου, 
τραγακάνθης, λυκίου Ἰνδικοῦ, κηκίδος, ἀνίσου, ἀκακίας, πε-
πέρεως, ῥήου Ποντικοῦ ἀνὰ 𐅻 αʹ. 



Galenus Med., De compositione medicamentorum secundum locos libri x 
Volume 13, page 292, line 9

                       ♃ Σμύρνης, λιβάνου, ἀλόης, κρόκου, 
ὀπίου, ῥοῦ Συριακοῦ καὶ τοῦ βυρσοδεψικοῦ, λυκίου Ἰνδι-
κοῦ, ἀκακίας, σιδίων, ὑποκυστίδος χυλοῦ, κηκίδος, βαλαυ-
στίων, ἑκάστου τὸ ἴσον. 



Galenus Med., De compositione medicamentorum secundum locos libri x 
Volume 13, page 299, line 8

                                                                   λυ-
κίου Ἰνδικοῦ ἴσον, ὀποῦ μήκωνος 𐅻 δʹ. 



Galenus Med., De compositione medicamentorum secundum locos libri x 
Volume 13, page 303, line 13

                                  νάρδου Ἰνδικῆς, ἀνίσου, ἑκά-
στου τὸ ἴσον, σμύρνης 𐅻 βʹ. 



Galenus Med., De compositione medicamentorum secundum locos libri x 
Volume 13, page 303, line 14

                                   ἀλόης Ἰνδικῆς, ὑποκυστίδος 
χυλοῦ, λυκίου Ἰνδικοῦ, ἀκακίας χυλοῦ, ὀποῦ μήκωνος, κηκί-
δων ὀμφακιτίδων, τραγακάνθης, πεπέρεως λευκοῦ, ἑκάστου 
τοσόνδε, οἴνῳ ἀναλάμβανε, ἀνάπλασσε τροχίσκους, δίδου τρι-
ώβολον. 



Galenus Med., De compositione medicamentorum secundum locos libri x 
Volume 13, page 305, line 13

                              σανδαράχης, ἀρσενικοῦ, λεπίδος 
χαλκοῦ, σχιστῆς, ὀμφακίου, λυκίου Ἰνδικοῦ, ἀκακίας χυλοῦ, 
ὑποκυστίδος χυλοῦ, ἀσβέστου ἀνὰ 𐅻 εʹ. 



Galenus Med., De compositione medicamentorum secundum locos libri x 
Volume 13, page 308, line 4

                                                     ♃ Λυκίου Ἰν-
δικοῦ γο βʹ. 



Galenus Med., De compositione medicamentorum secundum locos libri x 
Volume 13, page 308, line 9

       λυκίου Ἰνδικοῦ γο βʹ. 



Galenus Med., De compositione medicamentorum secundum locos libri x 
Volume 13, page 324, line 8

               νάρδου Ἰνδικῆς 𐅻 ιστʹ. 



Galenus Med., De compositione medicamentorum secundum locos libri x 
Volume 13, page 330, line 14

                      λιβανωτίδος ὀρεινῆς, νάρδου Ἰνδικῆς, 
κρόκου Κιλικίου, πετροσελίνου, πολίου, πηγάνου ἀγρίου σπέρ-
ματος, δικτάμνου Κρητικοῦ, ἑκάστου τὸ ἴσον, γλυκυῤῥίζης, 
λίθου Συριακοῦ ἄῤῥενος ἀνὰ 𐅻 ιστʹ. 



Galenus Med., De compositione medicamentorum per genera libri vii (0057: 077)
“Claudii Galeni opera omnia, vol. 13”, Ed. Kühn, C.G.
Leipzig: Knobloch, 1827, Repr. 1965.
Volume 13, page 494, line 2

                        ἀλόης Ἰνδικῆς 𐅻 ιβʹ. 



Galenus Med., De compositione medicamentorum per genera libri vii 
Volume 13, page 650, line 18

                   ἀλόης Ἰνδικῆς 𐅻 βʹ. 



Galenus Med., De compositione medicamentorum per genera libri vii 
Volume 13, page 741, line 13

                              ἡ Ἰνδὴ Θαρσέου χειρουργοῦ πρὸς 
τὰς προειρημένας διαθέσεις. 



Galenus Med., De compositione medicamentorum per genera libri vii 
Volume 13, page 806, line 15

                    ἀλόης Ἰνδικῆς οὐγγίας δʹ. 



Galenus Med., De compositione medicamentorum per genera libri vii 
Volume 13, page 808, line 4

                                                              ἀλόης Ἰν-
δικῆς 𐅻 ιβʹ. 



Galenus Med., De compositione medicamentorum per genera libri vii 
Volume 13, page 822, line 6

Ἰοῦ δὲ χαλκοῦ καὶ ἀλόης Ἰνδικῆς, 
 Χαλκοῦ τε κεκαυμένου καὶ διφρυγοῦς καλοῦ 
 Ἑκκαίδεχ' ἥμισυ ἐξ ἑκάστου μίγματος, 
 Εἴκοσι δραχμὰς δὲ χαλβάνης καὶ δʹ. 



Galenus Med., De compositione medicamentorum per genera libri vii 
Volume 13, page 826, line 9

                          ♃ ἀριστολοχίας Κρητικῆς στρογγύλης, 
κυτίνων, χαλκίτεως ὀπτῆς, λιβάνου, σμύρνης, χαλκάνθου, κρο-
κομάγματος, ἀλόης Ἰνδικῆς, χολῆς ταυρείας ἀνὰ 𐅻 ηʹ. 



Galenus Med., De compositione medicamentorum per genera libri vii 
Volume 13, page 831, line 11

                                        ♃ λυκίου Ἰνδικοῦ, 
ἀλόης Ἰνδικῆς, σμύρνης, χαλκοῦ, στυπτηρίας σχιστῆς, χαλ-
κάνθου, κυτίνων, λιβάνων ἀνὰ 𐅻 δʹ. 



Galenus Med., De compositione medicamentorum per genera libri vii 
Volume 13, page 907, line 9

                                                             ἀλόης 
Ἰνδικῆς 𐅻 βʹ. 



Galenus Med., De compositione medicamentorum per genera libri vii 
Volume 13, page 981, line 11

                                                            κας-
σίας, κινναμώμου, ἀμμωνιακοῦ θυμιάματος, νάρδου Ἰνδικῆς, 
κρόκου, σμύρνης ἀνὰ μνᾶς δʹ. 



Galenus Med., De compositione medicamentorum per genera libri vii 
Volume 13, page 983, line 6

                                                       ἀλόης Ἰνδικῆς 
𐅻 ιʹ. 



Galenus Med., De compositione medicamentorum per genera libri vii 
Volume 13, page 986, line 7

            κινναμώμου, κρόκου Κιλικίου, σμύρνης, κασσίας, 
νάρδου Ἰνδικῆς, σχοίνου ἄνθους, καλάμου ἀρωματικοῦ, ἀμώ-
μου, πεπέρεως, ῥοὸς χυλοῦ, χελώνης θαλασσίας αἵματος, ἀν-
δροσαίμου βοτάνης, πυρέθρου, ἑκάστου ἀνὰ 𐅻 λεʹ. 



Galenus Med., De compositione medicamentorum per genera libri vii 
Volume 13, page 1039, line 16

                                                              νάρδου 
Ἰνδικῆς λίτρας βʹ. 



Galenus Med., De antidotis libri ii (0057: 078)
“Claudii Galeni opera omnia, vol. 14”, Ed. Kühn, C.G.
Leipzig: Knobloch, 1827, Repr. 1965.
Volume 14, page 8, line 10

                                        καθάπερ γε τὸ κα-
λούμενον Ἰνδικὸν Λύκιον, ἕτερά τε πολλὰ δυσκολωτάτην 
ἔχοντα διάγνωσιν ἀπὸ τῶν νενοθευμένων, ἃ χρὴ παρὰ 
τῶν εἰς τὰ χωρία πορευομένων, ἐπιτρόπων τε καὶ συγκλητι-
κῶν ἀρξάντων τῆς χώρας ἀθροίζειν, ἢ καὶ τῶν κατοικούν-
των ἐν αὐτοῖς φίλων, ὥσπερ ἐμὲ ποιοῦντα τεθέασθε. 



Galenus Med., De antidotis libri ii 
Volume 14, page 40, line 3

Τοῖς δ' ἐπὶ κιννάμωμον ἰσάζεο, μηδέ σε λήθῃ 
  Ἀγαρικὸν τούτοις ἰσοβαρὲς θέμεναι,   
 Ἢ ἔτι καὶ σμύρνης καὶ εὐόδμου κόστοιο 
  Καὶ κρόκου, ὃν ἄντρον θρέψατο Κωρύκιον, 
 Καὶ κασίην, Ἰνδήν τε βάλοις εὐώδεα νάρδον, 
  Καὶ σχοῖνον νομάδων θαῦμα φέροις Ἀράβων, 
 Καὶ λίβανον μίσγοιο, καὶ ἀγλαϊὴν στήσαιο 
  Ἄμμιγα κυανέῳ κατθέμενος πεπέρει, 
 Δικτάμνου τε κλῶνας, ἠδὲ χλοεροῦ πρασίοιο, 
  Καὶ ῥῆον. 



Galenus Med., De antidotis libri ii 
Volume 14, page 43, line 8

          σμύρνης, κρόκου, ζιγγιβέρεως, ῥήου Ποντικοῦ, πεν-
ταφύλλου ῥίζης, νεπέτου, οὕτως οἱ Ῥωμαῖοι τὴν καλαμίν-
θην ὀνομάζουσι, πρασίου, πετροσελίνου, στοιχάδος, κόστου, 
πεπέρεως λευκοῦ, καὶ μακροῦ, δικτάμνου Κρητικοῦ, σχοίνου 
ἄνθους, λιβάνου, τερμινθίνης, κασσίας σύριγγος, νάρδου 
Ἰνδικῆς, ἀνὰ 𐅻 στʹ. 



Galenus Med., De antidotis libri ii 
Volume 14, page 52, line 11

Σμύρνης θ' ἓξ δραχμὰς, καὶ φύλλου μαλαβάθροιο, 
 Ἴσον δ' αὖ Ἰνδῆς νάρδου ξανθοῦ τε κρόκοιο. 



Galenus Med., De antidotis libri ii 
Volume 14, page 73, line 16

                  ἐφεξῆς δὲ τῆς προγεγραμμένης ὁ Ἀνδρό-
μαχος Ἰνδικὴν νάρδον κελεύει βαλεῖν, ἥνπερ καὶ στάχυν 
ὀνομάζομεν νάρδου, καίτοι ῥίζαν οὖσαν, ἀπὸ τῆς πρὸς τοὺς 
ἀστάχυας ὁμοιότητος κατὰ τὴν μορφὴν, ἐφ' ἧς φυλάττεσθαι   
χρὴ, μή πως ἀποδῷ τις ἡμῖν τὴν ἔκπλυτον ὀνομαζομένην. 



Galenus Med., De antidotis libri ii 
Volume 14, page 83, line 17

                                                  αἱ ῥίζαι δὲ τῶν-
δε, ζιγγιβέρεως, ἴρεως, ῥήου, πενταφύλλου, κόστου, νάρδου Ἰν-
δικῆς, νάρδου Κελτικῆς, γεντιανῆς, μήου, ἀκόρου, φοῦ, ἀριστο-
λοχίας λεπτῆς. 



Galenus Med., De antidotis libri ii 
Volume 14, page 96, line 17

Καὶ δύ' ἐπὶ ταύταις, καὶ ῥόδων ξηρῶν ἴσας, 
 Καὶ κινναμώμου, καὶ ὀποβαλσάμου καλοῦ, 
 Ἧς τ' Ἰλλυριοὶ πέμπουσιν ἴσας ἴρεως, 
 Λευκοῦ τ' ἀγαρικοῦ, βουνιάδος τοῦ σπέρματος, 
 Τούτων ἑκάστου τὰς ἴσας δυώδεκα, 
 Καὶ Κρητικὸν σκόρδιον ἑνὶ τούτων ἴσον, 
 Σμύρνης, κρόκου, ῥᾶ, πετροσελίνου, στοιχάδος, 
 Ξηρᾶς καλαμίνθης, πενταφύλλου ῥιζίων, 
 Καὶ τοῦ πρασίου σπέρματος τῶν σφαιρίων, 
 Λευκοῦ πεπέρεως καὶ μακροῦ, τῆς Ἰνδικῆς 
 Νάρδου στάχυος, λιβάνου τε καὶ τερμινθίνης,   
 Κόστου τε λευκοῦ, καὶ ἔτι δικτάμνου πόας, 
 Σχοίνου τε, μὴ τῶν καρφίων, τοῦ δ' ἄνθεος, 
 Ξηροῦ τ' ἀβρέκτου ζιγγιβέρεως τὰς ἴσας, 
 Κασίας μελαίνης τοῦ φλοιοῦ, τούτων πάλιν 
 Δὶς τρεῖς ἑκάστου μίγματος δραχμῆς σταθμῷ, 
 Πολίου τε καὶ φοῦ Ποντικοῦ, καὶ θλάσπεως, 
 Χαμαίδρυός τε καὶ στύρακος, καὶ κόμμεως, 
 Καὶ γεντιανῆς, καὶ ἀκόρου, μήου τ' ἔτι 




Galenus Med., De antidotis libri ii 
Volume 14, page 101, line 10

Τερμίνθου δ' ἀπὸ ῥητίνην, νάρδοιό τε ῥίζαν 
 Ἰνδῆς, καὶ κασίης σῦριγξ, λιβάνοιό τ' ἀμήτου 
 Χόνδροι, σπέρμα τε λεπτὸν ἐν οὔρεσιν ἀλδήσκοντος, 
 Πετροσελίνου μίσγε δραχμὰς στʹ. 



Galenus Med., De antidotis libri ii 
Volume 14, page 102, line 10

Σμύρνης δ' ἓξ δραχμὰς καὶ φύλλα μαλαβάθροιο, 
 Ἴσον δ' αὖτ' Ἰνδῆς νάρδου, ξανθοῦ τε κρόκοιο. 



Galenus Med., De antidotis libri ii 
Volume 14, page 107, line 17

                                            νάρδου Ἰνδικῆς 𐅻 δʹ. 



Galenus Med., De antidotis libri ii 
Volume 14, page 109, line 3

                         νάρδου Ἰνδικῆς 𐅻 δʹ. 



Galenus Med., De antidotis libri ii 
Volume 14, page 109, line 14

[Ἀντίδοτος ἡ Ὀρβανοῦ λεγομένη τοῦ Ἰνδοῦ, πρὸς τὸ 
τὰ ἐντὸς βρέφη ἐκβάλλειν. 



Galenus Med., De antidotis libri ii 
Volume 14, page 109, line 16

           νάρδου Ἰνδικῆς 𐅻 ιστʹ. 



Galenus Med., De antidotis libri ii 
Volume 14, page 111, line 14

                            νάρδου Ἰνδικῆς 𐅻 ιαʹ. 



Galenus Med., De antidotis libri ii 
Volume 14, page 128, line 4

                   στάχυος Ἰνδικῆς δραχμὰς, 
 Σμύρνης τε χρηστῆς, καὶ ὀπίου, μηκωνείου, 
 Ὑποκιστίδος τε χυλοῦ, καὶ ὀποβαλσάμου, 
 Ὀποπάνακός τε τοῦ καθαροῦ καὶ προσφάτου, 
 Καὶ καστορίου καὶ λιβάνου πάντων ἴσα, 
 Ἀνὰ ιβʹ. 



Galenus Med., De antidotis libri ii 
Volume 14, page 148, line 13

                                              κόστου, μήου, ἀσάρου, 
ἀκόρου, δαύκου Κρητικοῦ σπέρματος, πετροσελίνου, νάρδου 
Ἰνδικῆς, λίθου νηστωρίτου ἀνὰ 𐅻 δʹ. 



Galenus Med., De antidotis libri ii 
Volume 14, page 150, line 17

                                                νάρδου Ἰνδικῆς, 
σμύρνης Τρωγλοδύτιδος, σχοίνου ἄνθους, κινναμώμου, χαρα-  
κίου, κόστου προσφάτου, πεπέρεως μακροῦ, ὑποκυστίδος, 
πολίου, πεπέρεως περεατικοῦ, σκορδίου, μήου Κρητικοῦ, 
καρδαμώμου ἀνὰ 𐅻 δʹ. 



Galenus Med., De antidotis libri ii 
Volume 14, page 158, line 16

                               νάρδου Ἰνδικῆς 𐅻 δʹ. 



Galenus Med., De antidotis libri ii 
Volume 14, page 168, line 4

         ♃ Λυκίου Ἰνδικοῦ ὡς καλλίστου δίδου ὀβολοὺς γʹ. 



Galenus Med., De antidotis libri ii 
Volume 14, page 180, line 12

                                    ♃ Ὀποῦ Κυρηναϊ-
κοῦ, δαύκου Κρητικοῦ σπέρματος, ἡδυόσμου ξηροῦ, νάρδου 
Ἰνδικῆς, ἑκάστου τὸ ἴσον, ἀνάπλασσε δι' ὄξους, καὶ δίδου 
𐅻 αʹ. 



Galenus Med., De antidotis libri ii 
Volume 14, page 188, line 15

                                                      κάγχρυος 
καὶ σμύρνης, πεπέρεως, σεσέλεως Μασσαλεωτικοῦ, νάρδου 
Ἰνδικῆς, καστορίου, κινναμώμου μέλανος, ὡς λεπτοτάτου, 
πεπέρεως μακροῦ, Κυρηναϊκοῦ ὀποῦ, ἴριδος, πηγάνου 
ἀγρίου σπέρματος, ἀνὰ 𐅻 βʹ. 



Galenus Med., De antidotis libri ii 
Volume 14, page 197, line 5

Λυκίου κρατίστου Παταρικοῦ τριώβολον, 
 Τοῦ δ' Ἰνδικοῦ βέλτιστον ἂν ᾖ σοι παρὸν, 
 Διδόναι προσέταττεν ἡμέρας δυσὶν εἴκοσι, 
 Νήστεσι δι' ὕδατος, ὥσπερ εἶπον ἀρτίως. 



Galenus Med., De antidotis libri ii 
Volume 14, page 208, line 6

λυκίου Ἰνδικοῦ 𐅻 δʹ. 



Galenus Med., De theriaca ad Pisonem (0057: 079)
“Claudii Galeni opera omnia, vol. 14”, Ed. Kühn, C.G.
Leipzig: Knobloch, 1827, Repr. 1965.
Volume 14, page 263, line 1

                                                 νάρδου Ἰνδικῆς 
𐅻 ιστʹ. 



Galenus Med., De theriaca ad Pisonem 
Volume 14, page 292, line 15

                           μαλαβάθρου φύλλων Ἰνδικοῦ γο. 



Galenus Med., In Hippocratis de victu acutorum commentaria iv (0057: 087)
“Galeni in Hippocratis de victu acutorum commentaria iv”, Ed. Helmreich, G.
Leipzig: Teubner, 1914; Corpus medicorum Graecorum, vol. 5.9.1.
Kühn volume 15, page 862, line 7

έχουσιν, ᾗ οἱ μέγιστοι σύνδεσμοι καταπεφυκότες εἰς πόδας 
ἀποτελευτῶσι, τῷ τοιῷδε, ἢν μὴ πυρετὸς καὶ ὕπνος ἐπι-
γένηται καὶ τὰ ἑπόμενα οὖρα πέψιν ἔχοντα ἔλθῃ καὶ ἱδρῶ- |
τες κριτικοί, πίνειν οἶνον κιρρὸν οἰνώδεα καὶ ἄλητον ἑφθὸν 
ἐσθίειν καὶ κηρωτῇ ἀλείφειν καὶ ἐγχρίειν τά τε σκέλεα περι-
ελίσσειν ἕως τῶν ποδῶν, θερμῷ προβρέχων ἐν σκάφῃ, καὶ 
βραχίονας ἕως δακτύλων κατελίσσειν καὶ ὀσφὺν ἀπὸ τοῦ 
τραχήλου ἕως τῶν ἰσχίων, σίαλον ἐγκηρώσας, ὅπως καὶ τὰ 
ἔμπροσθεν περιέξει, καὶ διαλιπὼν πυρία τοῖσιν ἀσκίοισι, θερ-
μὸν ὕδωρ ἐγχέων, καὶ περιτείνων ςινδόνιον ἐπανάκλινε αὐ-  
τόν. 



Galenus Med., In Hippocratis librum vi epidemiarum commentarii vi (0057: 091)
“Galeni in Hippocratis sextum librum epidemiarum commentaria i–vi”, Ed. Wenkebach, E.
Leipzig: Teubner, 1940; Corpus medicorum Graecorum, vol. 5.10.2.2.
Kühn volume 17a, page 886, line 3

                                                                                τὰ 
δὲ ἐντὸς οὕτως ἐκαίετο, ὥστε μήτε τῶν πάνυ λεπτῶν ἱματίων καὶ   
ςινδονίων τὰς ἐπιβολάς, μηδ' ἄλλο τι ἢ γυμνοὶ ἀνέχεσθαι. 



Galenus Med., In Hippocratis librum vi epidemiarum commentarii vi 
Kühn volume 17a, page 902, line 12

                                                                                τοὺς 
μὲν γὰρ ὕποιδόν τε καὶ ὑδερικὴν ἔχοντας ὅλην τὴν ἕξιν τοῦ σώματος 
<σκληρύνομεν> διά τε ςινδόνων προανατρίβοντες καὶ σκληρᾷ τρίψει 
χρώμενοι καὶ γυμνάζοντες ἐν κόνει. 



Galenus Med., In Hippocratis librum vi epidemiarum commentarii vi 
Kühn volume 17b, page 196, line 11

                                                                     τῆς γὰρ 
<ψυχρᾶς κοίτης> σημαινούσης οὐ μόνον ὑποστορέσματα ψυχρὰ καὶ 
περιβλήματα, ἅπερ ἐκ ςινδόνων πυκνῶν καὶ τριβάκων ἱματίων τε τοι-
ούτων γίνεται, <ἀλλὰ καὶ χωρίον ψυχρόν>, προσέρχεταί τι μέγιστον 
ἕτερον ἐπανόρθωμα τῆς τε κατὰ τὴν ὥραν καὶ τὴν φυσικὴν κρᾶσιν 
ἀμετρίας <τὸ> διὰ τῆς εἰσπνοῆς ψυχρᾶς οὔσης ἐμψύχεσθαι τὸ σῶμα. 



Galenus Med., Linguarum seu dictionum exoletarum Hippocratis explicatio (0057: 106)
“Claudii Galeni opera omnia, vol. 19”, Ed. Kühn, C.G.
Leipzig: Knobloch, 1830, Repr. 1965.
Volume 19, page 105, line 13

ἱμερωθεῖσαι: ἀνδρὶ μιγεῖσαι, ἀντὶ τοῦ τῆς ἐπιθυμίας τυ-
 χοῦσαι· ἀνδρὶ μιγεῖσαι. 
Ἰνδικόν: οἱ μὲν γράψαντες τὰς ὀνομασίας τῶν φαρμάκων, 
 καθάπερ Μενεσθεύς τε καὶ Ἀνδρέας ὁ τοῦ Χρύσαρος 
 καὶ Ξενοκράτης καὶ Διοσκουρίδης ὁ Ἀλεξανδρεὺς Ἰνδικὸν 
 ὀνομάζουσι τὸ ζιγγίβερι, πλανηθέντες ἐκ τοῦ τινας οἴε-
 σθαι ῥίζαν αὐτὸ τοῦ πεπέρεως ὑπάρχειν· ἀλλὰ Διοσκου-  
 ρίδης ὁ Ἀναζαρβεὺς σαφῶς διώρισέ τε καὶ ἀπεφήνατο 
 περὶ ζιγγιβέρεως καὶ πεπέρεως· ὁ τοίνυν Διοσκουρίδης 
 ὁ νεώτερος ὁ γλωττογράφος, φυτὸν εἶναί φησιν ἐν Ἰνδίᾳ 
 παραπλήσιον τῷ τοῦ πεπέρεως, οὗ ὁ καρπὸς ὀνομάζεται 
 μυρτίδανον ὅτι μύρτῳ ἔοικεν. 



Galenus Med., Linguarum seu dictionum exoletarum Hippocratis explicatio 
Volume 19, page 105, line 15

Ἰνδικόν: οἱ μὲν γράψαντες τὰς ὀνομασίας τῶν φαρμάκων, 
 καθάπερ Μενεσθεύς τε καὶ Ἀνδρέας ὁ τοῦ Χρύσαρος 
 καὶ Ξενοκράτης καὶ Διοσκουρίδης ὁ Ἀλεξανδρεὺς Ἰνδικὸν 
 ὀνομάζουσι τὸ ζιγγίβερι, πλανηθέντες ἐκ τοῦ τινας οἴε-
 σθαι ῥίζαν αὐτὸ τοῦ πεπέρεως ὑπάρχειν· ἀλλὰ Διοσκου-  
 ρίδης ὁ Ἀναζαρβεὺς σαφῶς διώρισέ τε καὶ ἀπεφήνατο 
 περὶ ζιγγιβέρεως καὶ πεπέρεως· ὁ τοίνυν Διοσκουρίδης 
 ὁ νεώτερος ὁ γλωττογράφος, φυτὸν εἶναί φησιν ἐν Ἰνδίᾳ 
 παραπλήσιον τῷ τοῦ πεπέρεως, οὗ ὁ καρπὸς ὀνομάζεται 
 μυρτίδανον ὅτι μύρτῳ ἔοικεν. 



Galenus Med., Linguarum seu dictionum exoletarum Hippocratis explicatio 
Volume 19, page 141, line 14

στρογγύλου: τοῦ Μυρτιδάνου, αὐτὸς γὰρ οὕτως γράφει καὶ 
 τὸ Ἰνδικὸν ὃ καλοῦσι Πέρσαι πέπερι. 

\end{greek}


\section{Arrian}
\blockquote[From Wikipedia\footnote{\url{http://en.wikipedia.org/wiki/Arrian}}]{Arrian of Nicomedia (/ˈæriən/; Latin: Lucius Flavius Arrianus Xenophon; Greek: Ἀρριανός c. AD 86 – c.160) was a Roman (ethnic Greek)[3] historian, public servant, military commander and philosopher of the 2nd-century Roman period. As with other authors of the Second Sophistic, Arrian wrote primarily in Attic (Indica is in Herodotus' Ionic dialect, his philosophical works in Koine Greek) .

The Anabasis of Alexander is perhaps his best-known work, and is generally considered one of the best sources for the campaigns of Alexander the Great. (It is not to be confused with Anabasis, the best-known work of the Athenian military leader and author Xenophon from the 5th-4th century BC.) Arrian is also considered as one of the founders of a primarily military-based focus on history. His other works include Discourses of Epictetus and Indica.}
\begin{greek}

Flavius Arrianus Hist., Phil., Alexandri anabasis (0074: 001)
“Flavii Arriani quae exstant omnia, vol. 1”, Ed. Roos, A.G., Wirth, G.
Leipzig: Teubner, 1967 (1st edn. corr.).
Book 3, chapter 8, section 3, line 1

Βεβοηθήκεσαν γὰρ Δαρείῳ Ἰνδῶν τε ὅσοι Βακτρίοις 
ὅμοροι καὶ αὐτοὶ Βάκτριοι καὶ Σογδιανοί· τούτων μὲν 
πάντων ἡγεῖτο Βῆσσος ὁ τῆς Βακτρίων χώρας σατρά-
πης. 



Flavius Arrianus Hist., Phil., Alexandri anabasis 
Book 3, chapter 8, section 4, line 2

            Βαρσαέ<ν>της δὲ Ἀραχωτῶν σατράπης Ἀρα-
χωτούς τε ἦγε καὶ τοὺς ὀρείους Ἰνδοὺς καλουμένους. 



Flavius Arrianus Hist., Phil., Alexandri anabasis 
Book 3, chapter 8, section 6, line 6

                                                      ἐλέ-
γετο δὲ ἡ πᾶσα στρατιὰ ἡ Δαρείου ἱππεῖς μὲν ἐς τε-
τρακισμυρίους, πεζοὶ δὲ ἐς ἑκατὸν μυριάδας, καὶ ἅρ-
ματα δρεπανηφόρα διακόσια, ἐλέφαντες δὲ οὐ πολλοί, 
ἀλλὰ ἐς πεντεκαίδεκα μάλιστα Ἰνδοῖς τοῖς ἐπὶ τάδε 
τοῦ Ἰνδοῦ ἦσαν. 



Flavius Arrianus Hist., Phil., Alexandri anabasis 
Book 3, chapter 11, section 5, line 3

                  κατὰ τὸ μέσον δέ, ἵνα ἦν βασιλεὺς 
Δαρεῖος, οἵ τε συγγενεῖς οἱ βασιλέως ἐτετάχατο καὶ οἱ 
μηλοφόροι Πέρσαι καὶ Ἰνδοὶ καὶ Κᾶρες οἱ ἀνάσπαστοι 
καλούμενοι καὶ οἱ Μάρδοι τοξόται· Οὔξιοι δὲ καὶ Βα-
βυλώνιοι καὶ οἱ πρὸς τῇ ἐρυθρᾷ θαλάσσῃ καὶ Σιττα-
κηνοὶ εἰς βάθος ἐπιτεταγμένοι ἦσαν. 



Flavius Arrianus Hist., Phil., Alexandri anabasis 
Book 3, chapter 13, section 1, line 3

Ὡς δὲ ὁμοῦ ἤδη τὰ στρατόπεδα ἐγίγνετο, ὤφθη 
Δαρεῖός τε καὶ οἱ ἀμφ' αὐτόν, οἵ τε μηλοφόροι Πέρ-
σαι καὶ Ἰνδοὶ καὶ Ἀλβανοὶ καὶ Κᾶρες οἱ ἀνάσπαστοι 
καὶ οἱ Μάρδοι τοξόται, κατ' αὐτὸν Ἀλέξανδρον τεταγ-
μένοι καὶ τὴν ἴλην τὴν βασιλικήν. 



Flavius Arrianus Hist., Phil., Alexandri anabasis 
Book 3, chapter 14, section 5, line 2

καὶ ταύτῃ παραρραγείσης αὐτοῖς τῆς τάξεως κατὰ τὸ 
διέχον διεκπαίουσι τῶν τε Ἰνδῶν τινες καὶ τῆς Περ-
σικῆς ἵππου ὡς ἐπὶ τὰ σκευοφόρα τῶν Μακεδόνων· 
καὶ τὸ ἔργον ἐκεῖ καρτερὸν ἐγίγνετο. 



Flavius Arrianus Hist., Phil., Alexandri anabasis 
Book 3, chapter 15, section 1, line 8

         καὶ πρῶτα μὲν τοῖς φεύγουσι τῶν πολεμίων 
ἱππεῦσι, τοῖς τε Παρθυαίοις καὶ τῶν Ἰνδῶν ἔστιν οἷς 
καὶ Πέρσαις τοῖς πλείστοις καὶ κρατίστοις ἐμβάλλει. 



Flavius Arrianus Hist., Phil., Alexandri anabasis 
Book 3, chapter 25, section 8, line 6

                                  Βαρσαέντης δέ, ὃς τότε 
κατεῖχε τὴν χώραν, εἷς ὢν τῶν ξυνεπιθεμένων Δαρείῳ 
ἐν τῇ φυγῇ, προσιόντα Ἀλέξανδρον μαθὼν ἐς Ἰνδοὺς 
τοὺς ἐπὶ τάδε τοῦ Ἰνδοῦ ποταμοῦ ἔφυγε. 



Flavius Arrianus Hist., Phil., Alexandri anabasis 
Book 3, chapter 25, section 8, line 8

                                                  ξυλλαβόντες 
δὲ αὐτὸν οἱ Ἰνδοὶ παρ' Ἀλέξανδρον ἀπέστειλαν, καὶ ἀπο-
θνήσκει πρὸς Ἀλεξάνδρου τῆς ἐς Δαρεῖον ἀδικίας ἕνεκα. 



Flavius Arrianus Hist., Phil., Alexandri anabasis 
Book 3, chapter 28, section 1, line 5

                                                   ἐπῆλθε 
δὲ καὶ τῶν Ἰνδῶν τοὺς προσχώρους Ἀραχώταις. 



Flavius Arrianus Hist., Phil., Alexandri anabasis 
Book 3, chapter 29, section 2, line 5

                                                      ὁ δὲ 
Ὄξος ῥέει μὲν ἐκ τοῦ ὄρους τοῦ Καυκάσου, ἔστι δὲ 
ποταμῶν μέγιστος τῶν ἐν τῇ Ἀσίᾳ, ὅσους γε δὴ Ἀλέξ-
ανδρος καὶ οἱ ξὺν Ἀλεξάνδρῳ ἐπῆλθον, πλὴν τῶν Ἰν-
δῶν ποταμῶν· οἱ δὲ Ἰνδοὶ πάντων ποταμῶν μέγιστοί 
εἰσιν. 



Flavius Arrianus Hist., Phil., Alexandri anabasis 
Book 4, chapter 15, section 6, line 2

                                                    αὑτῷ 
δὲ τὰ Ἰνδῶν ἔφη ἐν τῷ τότε μέλειν. 



Flavius Arrianus Hist., Phil., Alexandri anabasis 
Book 4, chapter 22, section 3, line 2

Ἐκ Βάκτρων δὲ ἐξήκοντος ἤδη τοῦ ἦρος ἀναλαβὼν 
τὴν στρατιὰν προὐχώρει ὡς ἐπ' Ἰνδούς, Ἀμύνταν 
ἀπολιπὼν ἐν τῇ χώρᾳ τῶν Βακτρίων καὶ ξὺν αὐτῷ 
ἱππέας μὲν τρισχιλίους καὶ πεντακοσίους, πεζοὺς δὲ 
μυρίους. 



Flavius Arrianus Hist., Phil., Alexandri anabasis 
Book 4, chapter 22, section 6, line 4

                                                       ἀφ-
ικόμενος δὲ ἐς Νίκαιαν πόλιν καὶ τῇ Ἀθηνᾷ θύσας 
προὐχώρει ὡς ἐπὶ τὸν Κωφῆνα, προπέμψας κήρυκα 
ὡς Ταξίλην τε καὶ τοὺς ἐπὶ τάδε τοῦ Ἰνδοῦ ποταμοῦ, 
κελεύσας ἀπαντᾶν ὅπως ἂν ἑκάστοις προχωρῇ. 



Flavius Arrianus Hist., Phil., Alexandri anabasis 
Book 4, chapter 22, section 6, line 7

                                                      καὶ 
Ταξίλης τε καὶ οἱ ἄλλοι ὕπαρχοι ἀπήντων, δῶρα τὰ 
μέγιστα παρ' Ἰνδοῖς νομιζόμενα κομίζοντες, καὶ τοὺς 
ἐλέφαντας δώσειν ἔφασκον τοὺς παρὰ σφίσιν ὄντας, 
ἀριθμὸν ἐς πέντε καὶ εἴκοσιν. 



Flavius Arrianus Hist., Phil., Alexandri anabasis 
Book 4, chapter 22, section 7, line 3

Ἔνθα δὴ διελὼν τὴν στρατιὰν Ἡφαιστίωνα μὲν 
καὶ Περδίκκαν ἐκπέμπει ἐς τὴν Πευκελαῶτιν χώραν 
ὡς ἐπὶ τὸν Ἰνδὸν ποταμόν, ἔχοντας τήν τε Γοργίου 
τάξιν καὶ Κλείτου καὶ Μελεάγρου καὶ τῶν ἑταίρων 
ἱππέων τοὺς ἡμίσεας καὶ τοὺς μισθοφόρους ἱππέας 
ξύμπαντας, προστάξας τά τε κατὰ τὴν ὁδὸν χωρία ἢ 
βίᾳ ἐξαιρεῖν ἢ ὁμολογίᾳ παρίστασθαι καὶ ἐπὶ τὸν 
Ἰνδὸν ποταμὸν ἀφικομένους παρασκευάζειν ὅσα ἐς τὴν 
διάβασιν τοῦ ποταμοῦ ξύμφορα. 



Flavius Arrianus Hist., Phil., Alexandri anabasis 
Book 4, chapter 22, section 8, line 2

                                                  καὶ οὗτοι 
ὡς ἀφίκοντο πρὸς τὸν Ἰνδὸν ποταμόν, ἔπρασσον ὅσα 
ἐξ Ἀλεξάνδρου ἦν τεταγμένα. 



Flavius Arrianus Hist., Phil., Alexandri anabasis 
Book 4, chapter 24, section 3, line 1

Τὸν δὲ ἡγεμόνα αὐτὸν τῶν ταύτῃ Ἰνδῶν Πτολε-
μαῖος ὁ Λάγου πρός τινι ἤδη γηλόφῳ ὄντα κατιδὼν 
καὶ τῶν ὑπασπιστῶν ἔστιν οὓς ἀμφ' αὐτὸν ξὺν πολὺ 
ἐλάττοσιν αὐτὸς ὢν ὅμως ἐδίωκεν ἔτι ἐκ τοῦ ἵππου· 
ὡς δὲ χαλεπὸς ὁ γήλοφος τῷ ἵππῳ ἀναδραμεῖν ἦν, 
τοῦτον μὲν αὐτοῦ καταλείπει παραδούς τινι τῶν 
ὑπασπιστῶν ἄγειν, αὐτὸς δὲ ὡς εἶχε πεζὸς τῷ Ἰνδῷ 
εἵπετο. 



Flavius Arrianus Hist., Phil., Alexandri anabasis 
Book 4, chapter 24, section 4, line 3

              καὶ ὁ μὲν Ἰνδὸς τοῦ Πτολεμαίου διὰ τοῦ 
θώρακος παίει ἐκ χειρὸς ἐς τὸ στῆθος ξυστῷ μακρῷ, 
καὶ ὁ θώραξ ἔσχε τὴν πληγήν· Πτολεμαῖος δὲ τὸν 
μηρὸν διαμπὰξ βαλὼν τοῦ Ἰνδοῦ καταβάλλει τε καὶ 
σκυλεύει αὐτόν. 



Flavius Arrianus Hist., Phil., Alexandri anabasis 
Book 4, chapter 24, section 5, line 8

καὶ οὗτοι ἐπιγενόμενοι μόγις ἐξέωσαν τοὺς Ἰνδοὺς ἐς 
τὰ ὄρη καὶ τοῦ νεκροῦ ἐκράτησαν. 



Flavius Arrianus Hist., Phil., Alexandri anabasis 
Book 4, chapter 25, section 3, line 3

                          καὶ γίγνεται καὶ τούτοις μάχη 
καρτερὰ τοῦ χωρίου τῇ χαλεπότητι καὶ ὅτι οὐ κατὰ τοὺς 
ἄλλους τοὺς ταύτῃ βαρβάρους οἱ Ἰνδοί, ἀλλὰ πολὺ δή 
τι ἀλκιμώτατοι τῶν προσχώρων εἰσίν. 



Flavius Arrianus Hist., Phil., Alexandri anabasis 
Book 4, chapter 26, section 1, line 4

                                ὡς δὲ προσῆγεν ἤδη 
τοῖς τείχεσι, θαρρήσαντες οἱ βάρβαροι τοῖς μισθοφόροις 
τοῖς ἐκ τῶν πρόσω Ἰνδῶν, ἦσαν γὰρ οὗτοι ἐς ἑπτα-
κισχιλίους, ὡς στρατοπεδευομένους εἶδον τοὺς Μακε-
δόνας, δρόμῳ ἐπ' αὐτοὺς ᾔεσαν. 



Flavius Arrianus Hist., Phil., Alexandri anabasis 
Book 4, chapter 26, section 4, line 4

                                                         οἱ   
δὲ Ἰνδοὶ τῷ τε παραλόγῳ ἐκπλαγέντες καὶ ἅμα ἐν 
χερσὶ γεγενημένης τῆς μάχης ἐγκλίναντες ἔφευγον ἐς 
τὴν πόλιν. 



Flavius Arrianus Hist., Phil., Alexandri anabasis 
Book 4, chapter 26, section 5, line 4

              ἐπαγαγὼν δὲ τὰς μηχανὰς τῇ ὑστεραίᾳ 
τῶν μὲν τειχῶν τι εὐμαρῶς κατέσεισε, βιαζομένους δὲ 
ταύτῃ τοὺς Μακεδόνας ᾗ παρέρρηκτο τοῦ τείχους οὐκ 
ἀτόλμως οἱ Ἰνδοὶ ἠμύνοντο, ὥστε ταύτῃ μὲν τῇ ἡμέρᾳ 
ἀνεκαλέσατο τὴν στρατιάν. 



Flavius Arrianus Hist., Phil., Alexandri anabasis 
Book 4, chapter 26, section 5, line 9

                              τῇ δὲ ὑστεραίᾳ τῶν τε 
Μακεδόνων αὐτῶν ἡ προσβολὴ καρτερωτέρα ἐγίγνετο 
καὶ πύργος ἐπήχθη ξύλινος τοῖς τείχεσιν, ὅθεν ἐκ-
τοξεύοντες οἱ τοξόται καὶ βέλη ἀπὸ μηχανῶν ἀφιέμενα 
ἀνέστελλεν ἐπὶ πολὺ τοὺς Ἰνδούς. 



Flavius Arrianus Hist., Phil., Alexandri anabasis 
Book 4, chapter 27, section 2, line 1

Καὶ οἱ Ἰνδοί, ἕως μὲν αὐτοῖς ὁ ἡγεμὼν τοῦ 
χωρίου περιῆν, ἀπεμάχοντο καρτερῶς· ὡς δὲ βέλει 
ἀπὸ μηχανῆς τυπεὶς ἀποθνήσκει ἐκεῖνος, αὐτῶν τε 
οἱ μέν τινες πεπτωκότες ἐν τῇ ξυνεχεῖ πολιορκίᾳ, οἱ 
πολλοὶ δὲ τραυματίαι τε καὶ ἀπόμαχοι ἦσαν, ἐπεκηρυ-
κεύοντο πρὸς Ἀλέξανδρον. 



Flavius Arrianus Hist., Phil., Alexandri anabasis 
Book 4, chapter 27, section 3, line 3

                              τῷ δὲ ἀσμένῳ γίνεται 
ἄνδρας ἀγαθοὺς διασῶσαι· καὶ ξυμβαίνει ἐπὶ τῷδε 
Ἀλέξανδρος τοῖς μισθοφόροις Ἰνδοῖς ὡς καταταχθέντας 
ἐς τὴν ἄλλην στρατιὰν ξὺν αὑτῷ στρατεύεσθαι. 



Flavius Arrianus Hist., Phil., Alexandri anabasis 
Book 4, chapter 27, section 3, line 9

                                    νυκτὸς δὲ ἐπενόουν 
δρασμῷ διαχρησάμενοι ἐς τὰ σφέτερα ἤθη ἀπαναστῆναι 
οὐκ ἐθέλοντες ἐναντία αἴρεσθαι τοῖς ἄλλοις Ἰνδοῖς 
ὅπλα. 



Flavius Arrianus Hist., Phil., Alexandri anabasis 
Book 4, chapter 27, section 4, line 3

        καὶ ταῦτα ὡς ἐξηγγέλθη Ἀλεξάνδρῳ, περιστήσας 
τῆς νυκτὸς τῷ γηλόφῳ τὴν στρατιὰν πᾶσαν κατα-
κόπτει τοὺς Ἰνδοὺς ἐν μέσῳ ἀπολαβών, τήν τε πόλιν 
αἱρεῖ κατὰ κράτος ἐρημωθεῖσαν τῶν προμαχομένων, 
καὶ τὴν μητέρα τὴν Ἀσσακάνου καὶ τὴν παῖδα ἔλαβεν. 



Flavius Arrianus Hist., Phil., Alexandri anabasis 
Book 4, chapter 28, section 2, line 1

              εἰ μὲν δὴ καὶ ἐς Ἰνδοὺς ἀφίκετο ὁ Ἡρακλῆς 
ὁ Θηβαῖος ἢ ὁ Τύριος ἢ ὁ Αἰγύπτιος ἐς οὐδέτερα ἔχω 
ἰσχυρίσασθαι· μᾶλλον δὲ δοκῶ ὅτι οὐκ ἀφίκετο, ἀλλὰ 
πάντα γὰρ ὅσα χαλεπὰ οἱ ἄνθρωποι ἐς τοσόνδε ἄρα 
αὔξουσιν αὐτῶν τὴν χαλεπότητα, ὡς καὶ τῷ Ἡρακλεῖ 
ἂν ἄπορα γενέσθαι μυθεύειν. 



Flavius Arrianus Hist., Phil., Alexandri anabasis 
Book 4, chapter 28, section 5, line 4

                                           καὶ οἱ ἀμφὶ Ἡφαι-
στίωνά τε καὶ Περδίκκαν αὐτῷ ἄλλην πόλιν ἐκτειχί-
σαντες, Ὀροβάτις ὄνομα τῇ πόλει ἦν, καὶ φρουρὰν 
καταλιπόντες ὡς ἐπὶ τὸν Ἰνδὸν ποταμὸν ᾔεσαν· ὡς 
δὲ ἀφίκοντο, ἔπρασσον ἤδη ὅσα ἐς τὸ ζεῦξαι τὸν Ἰνδὸν 
ὑπὸ Ἀλεξάνδρου ἐτέτακτο. 



Flavius Arrianus Hist., Phil., Alexandri anabasis 
Book 4, chapter 28, section 6, line 2

Ἀλέξανδρος δὲ τῆς μὲν χώρας τῆς ἐπὶ τάδε τοῦ 
Ἰνδοῦ ποταμοῦ σατράπην κατέστησε Νικάνορα τῶν 
ἑταίρων. 



Flavius Arrianus Hist., Phil., Alexandri anabasis 
Book 4, chapter 28, section 6, line 3

           αὐτὸς δὲ τὰ μὲν πρῶτα ὡς ἐπὶ τὸν Ἰνδὸν 
ποταμὸν ἦγε, καὶ πόλιν τε Πευκελαῶτιν οὐ πόρρω 
τοῦ Ἰνδοῦ ᾠκισμένην ὁμολογίᾳ παρεστήσατο καὶ ἐν 
αὐτῇ φρουρὰν καταστήσας τῶν Μακεδόνων καὶ Φίλιππον 
ἐπὶ τῇ φρουρᾷ ἡγεμόνα, ὁ δὲ καὶ ἄλλα προσηγάγετο 
μικρὰ πολίσματα πρὸς τῷ Ἰνδῷ ποταμῷ ᾠκισμένα. 



Flavius Arrianus Hist., Phil., Alexandri anabasis 
Book 4, chapter 29, section 3, line 5

ὡς δὲ Ἀλεξάνδρῳ ἄπορον τὴν προσβολὴν κατέμαθον 
οἱ βάρβαροι, ἀναστρέψαντες τοῖς ἀμφὶ Πτολεμαῖον 
αὐτοὶ προσέβαλλον· καὶ γίγνεται αὐτῶν τε καὶ τῶν 
Μακεδόνων μάχη καρτερά, τῶν μὲν διασπάσαι τὸν 
χάρακα σπουδὴν ποιουμένων, τῶν Ἰνδῶν, Πτολεμαίου 
δὲ διαφυλάξαι τὸ χωρίον· καὶ μεῖον σχόντες οἱ βάρ-
βαροι ἐν τῷ ἀκροβολισμῷ νυκτὸς ἐπιγενομένης ἀπ-
εχώρησαν. 



Flavius Arrianus Hist., Phil., Alexandri anabasis 
Book 4, chapter 29, section 4, line 1

Ἀλέξανδρος δὲ τῶν Ἰνδῶν τινα τῶν αὐτομόλων 
πιστόν τε ἄλλως καὶ τῶν χωρίων δαήμονα ἐπιλεξά-
μενος πέμπει παρὰ Πτολεμαῖον τῆς νυκτὸς, γράμ-
ματα φέροντα τὸν Ἰνδόν, ἵνα ἐνεγέγραπτο, ἐπειδὰν 
αὐτὸς προσβάλλῃ τῇ πέτρᾳ, τὸν δὲ ἐπιέναι τοῖς βαρ-
βάροις κατὰ τὸ ὄρος μηδὲ ἀγαπᾶν ἐν φυλακῇ ἔχοντα 
τὸ χωρίον, ὡς ἀμφοτέρωθεν βαλλομένους τοὺς Ἰνδοὺς 
ἀμφιβόλους γίγνεσθαι. 



Flavius Arrianus Hist., Phil., Alexandri anabasis 
Book 4, chapter 29, section 6, line 2

                      ἔστε μὲν γὰρ ἐπὶ μεσημβρίαν 
ξυνειστήκει καρτερὰ μάχη τοῖς τε Ἰνδοῖς καὶ τοῖς 
Μακεδόσιν, τῶν μὲν ἐκβιαζομένων ἐς τὴν πρόσβασιν, 
τῶν δὲ βαλλόντων ἀνιόντας· ὡς δὲ οὐκ ἀνίεσαν οἱ 
Μακεδόνες, ἄλλοι ἐπ' ἄλλοις ἐπιόντες, οἱ δὲ πρόσθεν 
ἀναπαυόμενοι, μόγις δὴ ἀμφὶ δείλην ἐκράτησαν τῆς   
παρόδου καὶ ξυνέμιξαν τοῖς ξὺν Πτολεμαίῳ. 



Flavius Arrianus Hist., Phil., Alexandri anabasis 
Book 4, chapter 30, section 1, line 3

                     ἐς δὲ τὴν ὑστεραίαν οἵ τε σφεν-
δονῆται σφενδονῶντες ἐς τοὺς Ἰνδοὺς ἐκ τοῦ ἤδη 
κεχωσμένου καὶ ἀπὸ τῶν μηχανῶν βέλη ἀφιέμενα 
ἀνέστελλε τῶν Ἰνδῶν τὰς ἐκδρομὰς τὰς ἐπὶ τοὺς 
χωννύοντας. 



Flavius Arrianus Hist., Phil., Alexandri anabasis 
Book 4, chapter 30, section 2, line 1

Οἱ δὲ Ἰνδοὶ πρός τε τὴν ἀδιήγητον τόλμαν τῶν 
ἐς τὸν γήλοφον βιασαμένων Μακεδόνων ἐκπλαγέντες   
καὶ τὸ χῶμα ξυνάπτον ἤδη ὁρῶντες, τοῦ μὲν ἀπο-
μάχεσθαι ἔτι ἀπείχοντο, πέμψαντες δὲ κήρυκας σφῶν 
παρὰ Ἀλέξανδρον ἐθέλειν ἔφασκον ἐνδοῦναι τὴν πέτραν, 
εἴ σφισι σπένδοιτο. 



Flavius Arrianus Hist., Phil., Alexandri anabasis 
Book 4, chapter 30, section 4, line 8

                             εἴχετό τε Ἀλεξάνδρῳ ἡ 
πέτρα ἡ τῷ Ἡρακλεῖ ἄπορος γενομένη καὶ ἔθυεν ἐπ' 
αὐτῇ Ἀλέξανδρος καὶ κατεσκεύασε φρούριον, παραδοὺς 
Σισικόττῳ ἐπιμελεῖσθαι τῆς φρουρᾶς, ὃς ἐξ Ἰνδῶν 
μὲν πάλαι ηὐτομολήκει ἐς Βάκτρα παρὰ Βῆσσον, Ἀλεξ-
άνδρου δὲ κατασχόντος τὴν χώραν τὴν Βακτρίαν 
ξυνεστράτευέ τε αὐτῷ καὶ πιστὸς ἐς τὰ μάλιστα 
ἐφαίνετο. 



Flavius Arrianus Hist., Phil., Alexandri anabasis 
Book 4, chapter 30, section 7, line 1

Αὐτὸς δὲ ὡς ἐπὶ τὸν Ἰνδὸν ποταμὸν ἤδη ἦγε, καὶ 
ἡ στρατιὰ αὐτῷ ὡδοποίει τὸ πρόσω ἰοῦσα ἄπορα 
ἄλλως ὄντα τὰ ταύτῃ χωρία. 



Flavius Arrianus Hist., Phil., Alexandri anabasis 
Book 4, chapter 30, section 7, line 5

                                   ἐνταῦθα ξυλλαμβάνει 
ὀλίγους τῶν βαρβάρων, καὶ παρὰ τούτων ἔμαθεν, ὅτι 
οἱ μὲν ἐν τῇ χώρᾳ Ἰνδοὶ παρὰ Ἀβισάρῃ ἀποπεφευγότες 
εἶεν, τοὺς δὲ ἐλέφαντας ὅτι αὐτοῦ κατέλιπον νέμεσθαι 
πρὸς τῷ ποταμῷ τῷ Ἰνδῷ· καὶ τούτους ἡγήσασθαί οἱ 
τὴν ὁδὸν ἐκέλευσεν ὡς ἐπὶ τοὺς ἐλέφαντας. 



Flavius Arrianus Hist., Phil., Alexandri anabasis 
Book 4, chapter 30, section 8, line 2

                                                     εἰσὶ δὲ 
Ἰνδῶν πολλοὶ κυνηγέται τῶν ἐλεφάντων, καὶ τούτους 
σπουδῇ ἀμφ' αὑτὸν εἶχεν Ἀλέξανδρος, καὶ τότε ἐθήρα 
ξὺν τούτοις τοὺς ἐλέφαντας· καὶ δύο μὲν αὐτῶν 
ἀπόλλυνται κατὰ κρημνοῦ σφᾶς ῥίψαντες ἐν τῇ διώξει,   
οἱ δὲ ἄλλοι ξυλληφθέντες ἔφερόν τε τοὺς ἀμβάτας 
καὶ τῇ στρατιᾷ ξυνετάσσοντο. 



Flavius Arrianus Hist., Phil., Alexandri anabasis 
Book 4, chapter 30, section 9, line 4

                                                      καὶ 
αὗται κατὰ τὸν Ἰνδὸν ποταμὸν ἤγοντο ὡς ἐπὶ τὴν 
γέφυραν, ἥντινα Ἡφαιστίων καὶ Περδίκκας αὐτῷ ἐξ-
ῳκοδομηκότες πάλαι ἦσαν. 



Flavius Arrianus Hist., Phil., Alexandri anabasis 
Book 5, chapter 1, section 1, line 2

ΑΡΡΙΑΝΟΥ 
ΑΛΕΞΑΝΔΡΟΥ ΑΝΑΒΑΣΕΩΣ 
ΒΙΒΛΙΟΝ ΠΕΜΠΤΟΝ


 Ἐν δὲ τῇ χώρᾳ ταύτῃ, ἥντινα μεταξὺ τοῦ τε 
Κωφῆνος καὶ τοῦ Ἰνδοῦ ποταμοῦ ἐπῆλθεν Ἀλέξαν-
δρος, καὶ Νῦσαν πόλιν ᾠκίσθαι λέγουσι· τὸ δὲ κτίσμα 
εἶναι Διονύσου· Διόνυσον δὲ κτίσαι τὴν Νῦσαν ἐπεί 
τε Ἰνδοὺς ἐχειρώσατο, ὅστις δὴ οὗτος ὁ Διόνυσος καὶ 
ὁπότε ἢ ὅθεν ἐπ' Ἰνδοὺς ἐστράτευσεν· οὐ γὰρ ἔχω 
συμβαλεῖν εἰ ὁ Θηβαῖος Διόνυσος [ὃς] ἐκ Θηβῶν ἢ 
ἐκ Τμώλου τοῦ Λυδίου ὁρμηθεὶς ἐπὶ Ἰνδοὺς ἧκε 
στρατιὰν ἄγων, τοσαῦτα μὲν ἔθνη μάχιμα καὶ ἄγνωστα 
τοῖς τότε Ἕλλησιν ἐπελθών, οὐδὲν δὲ αὐτῶν ἄλλο ὅτι 
μὴ τὸ Ἰνδῶν βίᾳ χειρωσάμενος· πλήν γε δὴ ὅτι οὐκ 




Flavius Arrianus Hist., Phil., Alexandri anabasis 
Book 5, chapter 1, section 2, line 3

ΑΛΕΞΑΝΔΡΟΥ ΑΝΑΒΑΣΕΩΣ 
ΒΙΒΛΙΟΝ ΠΕΜΠΤΟΝ


 Ἐν δὲ τῇ χώρᾳ ταύτῃ, ἥντινα μεταξὺ τοῦ τε 
Κωφῆνος καὶ τοῦ Ἰνδοῦ ποταμοῦ ἐπῆλθεν Ἀλέξαν-
δρος, καὶ Νῦσαν πόλιν ᾠκίσθαι λέγουσι· τὸ δὲ κτίσμα 
εἶναι Διονύσου· Διόνυσον δὲ κτίσαι τὴν Νῦσαν ἐπεί 
τε Ἰνδοὺς ἐχειρώσατο, ὅστις δὴ οὗτος ὁ Διόνυσος καὶ 
ὁπότε ἢ ὅθεν ἐπ' Ἰνδοὺς ἐστράτευσεν· οὐ γὰρ ἔχω 
συμβαλεῖν εἰ ὁ Θηβαῖος Διόνυσος [ὃς] ἐκ Θηβῶν ἢ 
ἐκ Τμώλου τοῦ Λυδίου ὁρμηθεὶς ἐπὶ Ἰνδοὺς ἧκε 
στρατιὰν ἄγων, τοσαῦτα μὲν ἔθνη μάχιμα καὶ ἄγνωστα 
τοῖς τότε Ἕλλησιν ἐπελθών, οὐδὲν δὲ αὐτῶν ἄλλο ὅτι 
μὴ τὸ Ἰνδῶν βίᾳ χειρωσάμενος· πλήν γε δὴ ὅτι οὐκ 
ἀκριβῆ ἐξεταστὴν χρὴ εἶναι τῶν ὑπὲρ τοῦ θείου ἐκ 
παλαιοῦ μεμυθευμένων. 



Flavius Arrianus Hist., Phil., Alexandri anabasis 
Book 5, chapter 1, section 2, line 6

Κωφῆνος καὶ τοῦ Ἰνδοῦ ποταμοῦ ἐπῆλθεν Ἀλέξαν-
δρος, καὶ Νῦσαν πόλιν ᾠκίσθαι λέγουσι· τὸ δὲ κτίσμα 
εἶναι Διονύσου· Διόνυσον δὲ κτίσαι τὴν Νῦσαν ἐπεί 
τε Ἰνδοὺς ἐχειρώσατο, ὅστις δὴ οὗτος ὁ Διόνυσος καὶ 
ὁπότε ἢ ὅθεν ἐπ' Ἰνδοὺς ἐστράτευσεν· οὐ γὰρ ἔχω 
συμβαλεῖν εἰ ὁ Θηβαῖος Διόνυσος [ὃς] ἐκ Θηβῶν ἢ 
ἐκ Τμώλου τοῦ Λυδίου ὁρμηθεὶς ἐπὶ Ἰνδοὺς ἧκε 
στρατιὰν ἄγων, τοσαῦτα μὲν ἔθνη μάχιμα καὶ ἄγνωστα 
τοῖς τότε Ἕλλησιν ἐπελθών, οὐδὲν δὲ αὐτῶν ἄλλο ὅτι 
μὴ τὸ Ἰνδῶν βίᾳ χειρωσάμενος· πλήν γε δὴ ὅτι οὐκ 
ἀκριβῆ ἐξεταστὴν χρὴ εἶναι τῶν ὑπὲρ τοῦ θείου ἐκ 
παλαιοῦ μεμυθευμένων. 



Flavius Arrianus Hist., Phil., Alexandri anabasis 
Book 5, chapter 1, section 5, line 3

                                                      Διό-
νυσος γὰρ ἐπειδὴ χειρωσάμενος τὸ Ἰνδῶν ἔθνος ἐπὶ 
θάλασσαν ὀπίσω κατῄει τὴν Ἑλληνικήν, ἐκ τῶν ἀπο-
μάχων στρατιωτῶν, οἳ δὴ αὐτῷ καὶ βάκχοι ἦσαν, κτίζει 
τὴν πόλιν τήνδε μνημόσυνον τῆς αὑτοῦ πλάνης τε καὶ 
νίκης τοῖς ἔπειτα ἐσόμενον, καθάπερ οὖν καὶ σὺ αὐτὸς 
Ἀλεξάνδρειάν τε ἔκτισας τὴν πρὸς Καυκάσῳ ὄρει καὶ 
ἄλλην Ἀλεξάνδρειαν ἐν τῇ Αἰγυπτίων γῇ, καὶ ἄλλας 
πολλὰς τὰς μὲν ἔκτικας ἤδη, τὰς δὲ καὶ κτίσεις ἀνὰ 
χρόνον, οἷα δὴ πλείονα Διονύσου ἔργα ἀποδειξάμενος. 



Flavius Arrianus Hist., Phil., Alexandri anabasis 
Book 5, chapter 1, section 6, line 9

                                      καὶ ἐκ τούτου ἐλευ-
θέραν τε οἰκοῦμεν τὴν Νῦσαν καὶ αὐτοὶ αὐτόνομοι   
καὶ ἐν κόσμῳ πολιτεύοντες· τῆς δὲ ἐκ Διονύσου 
οἰκίσεως καὶ τόδε σοι γενέσθω τεκμήριον· κιττὸς γὰρ 
οὐκ ἄλλῃ τῆς Ἰνδῶν γῆς φυόμενος παρ' ἡμῖν φύεται. 



Flavius Arrianus Hist., Phil., Alexandri anabasis 
Book 5, chapter 2, section 6, line 3

                             καὶ τοὺς Μακεδόνας ἡδέως 
τὸν κισσὸν ἰδόντας, οἷα δὴ διὰ μακροῦ ὀφθέντα (οὐ 
γὰρ εἶναι ἐν τῇ Ἰνδῶν χώρᾳ κισσόν, οὐδὲ ἵναπερ 
αὐτοῖς ἄμπελοι ἦσαν) στεφάνους σπουδῇ ἀπ' αὐτοῦ 
ποιεῖσθαι, ὡς καὶ στεφανώσασθαι εἶχον, ἐφυμνοῦντας 
τὸν Διόνυσόν τε καὶ τὰς ἐπωνυμίας τοῦ θεοῦ ἀνα-
καλοῦντας. 



Flavius Arrianus Hist., Phil., Alexandri anabasis 
Book 5, chapter 3, section 3, line 3

        τὸν δὲ Καύκασον τὸ ὄρος ἐκ τοῦ Πόντου ἐς 
τὰ πρὸς ἕω μέρη τῆς γῆς καὶ τὴν Παραπαμισαδῶν 
χώραν ὡς ἐπὶ Ἰνδοὺς μετάγειν τῷ λόγῳ τοὺς Μακε-
δόνας, Παραπάμισον ὄντα τὸ ὄρος αὐτοὺς καλοῦντας 
Καύκασον τῆς Ἀλεξάνδρου ἕνεκα δόξης, ὡς ὑπὲρ τὸν 
Καύκασον ἄρα ἐλθόντα Ἀλέξανδρον. 



Flavius Arrianus Hist., Phil., Alexandri anabasis 
Book 5, chapter 3, section 4, line 2

                                          ἔν τε αὐτῇ τῇ 
Ἰνδῶν γῇ βοῦς ἰδόντας ἐγκεκαυμένας ῥόπαλον τεκ-
μηριοῦσθαι ἐπὶ τῷδε, ὅτι Ἡρακλῆς ἐς Ἰνδοὺς ἀφίκετο. 



Flavius Arrianus Hist., Phil., Alexandri anabasis 
Book 5, chapter 3, section 5, line 1

Ἀλέξανδρος δὲ ὡς ἀφίκετο ἐπὶ τὸν Ἰνδὸν ποταμόν, 
καταλαμβάνει γέφυράν τε ἐπ' αὐτῷ πεποιημένην πρὸς 
Ἡφαιστίωνος καὶ πλοῖα πολλὰ μὲν σμικρότερα, δύο 
δὲ τριακοντόρους, καὶ παρὰ Ταξίλου τοῦ Ἰνδοῦ δῶρα 
ἥκοντα ἀργυρίου μὲν τάλαντα ἐς διακόσια, ἱερεῖα δὲ 
βοῦς μὲν τρισχιλίας, πρόβατα δὲ ὑπὲρ μύρια, ἐλέφαν-
τας δὲ ἐς τριάκοντα· καὶ ἱππεῖς δὲ ἑπτακόσιοι αὐτῷ 
Ἰνδῶν ἐς ξυμμαχίαν παρὰ Ταξίλου ἧκον· καὶ τὴν πόλιν 
Τάξιλα, τὴν μεγίστην μεταξὺ Ἰνδοῦ τε ποταμοῦ καὶ 
Ὑδάσπου, ὅτι αὐτῷ Ταξίλης ἐνδίδωσιν. 



Flavius Arrianus Hist., Phil., Alexandri anabasis 
Book 5, chapter 4, section 1, line 1

Ὁ δὲ Ἰνδὸς ποταμὸς ὅτι μέγιστος ποταμῶν ἐστι 
τῶν κατὰ τὴν Ἀσίαν τε καὶ τὴν Εὐρώπην, πλὴν 
Γάγγου, καὶ τούτου Ἰνδοῦ ποταμοῦ, καὶ ὅτι αἱ πηγαί 
εἰσιν αὐτῷ ἐπὶ τάδε τοῦ ὄρους τοῦ Παραπαμίσου ἢ 
Καυκάσου, καὶ ὅτι ἐκδίδωσιν ἐς τὴν μεγάλην θάλασσαν 
τὴν κατὰ Ἰνδοὺς ὡς ἐπὶ νότον ἄνεμον, καὶ ὅτι 
δίστομός ἐστιν ὁ Ἰνδὸς καὶ αἱ ἐκβολαὶ αὐτοῦ ἀμφότεραι 
τεναγώδεις, καθάπερ αἱ πέντε τοῦ Ἴστρου, καὶ ὅτι 
Δέλτα ποιεῖ καὶ αὐτὸς ἐν τῇ Ἰνδῶν γῇ τῷ Αἰγυπτίῳ 
Δέλτα παραπλήσιον καὶ τοῦτο Πάταλα καλεῖται τῇ 




Flavius Arrianus Hist., Phil., Alexandri anabasis 
Book 5, chapter 4, section 1, line 11

τῶν κατὰ τὴν Ἀσίαν τε καὶ τὴν Εὐρώπην, πλὴν 
Γάγγου, καὶ τούτου Ἰνδοῦ ποταμοῦ, καὶ ὅτι αἱ πηγαί 
εἰσιν αὐτῷ ἐπὶ τάδε τοῦ ὄρους τοῦ Παραπαμίσου ἢ 
Καυκάσου, καὶ ὅτι ἐκδίδωσιν ἐς τὴν μεγάλην θάλασσαν 
τὴν κατὰ Ἰνδοὺς ὡς ἐπὶ νότον ἄνεμον, καὶ ὅτι 
δίστομός ἐστιν ὁ Ἰνδὸς καὶ αἱ ἐκβολαὶ αὐτοῦ ἀμφότεραι 
τεναγώδεις, καθάπερ αἱ πέντε τοῦ Ἴστρου, καὶ ὅτι 
Δέλτα ποιεῖ καὶ αὐτὸς ἐν τῇ Ἰνδῶν γῇ τῷ Αἰγυπτίῳ 
Δέλτα παραπλήσιον καὶ τοῦτο Πάταλα καλεῖται τῇ 
Ἰνδῶν φωνῇ, ταῦτα μὲν ὑπὲρ τοῦ Ἰνδοῦ τὰ μάλιστα 
οὐκ ἀμφίλογα καὶ ἐμοὶ ἀναγεγράφθω. 



Flavius Arrianus Hist., Phil., Alexandri anabasis 
Book 5, chapter 4, section 2, line 3

                                           ἐπεὶ καὶ ὁ   
Ὑδάσπης καὶ Ἀκεσίνης καὶ Ὑδραώτης καὶ Ὕφασις, καὶ 
οὗτοι Ἰνδοὶ ποταμοὶ ὄντες, τῶν μὲν ἄλλων τῶν Ἀσιανῶν 
ποταμῶν πολύ τι κατὰ μέγεθος ὑπερφέρουσι, τοῦ 
Ἰνδοῦ δὲ μείονές εἰσιν καὶ πολὺ δὴ μείονες, ὅπου καὶ 
αὐτὸς ὁ Ἰνδὸς τοῦ Γάγγου. 



Flavius Arrianus Hist., Phil., Alexandri anabasis 
Book 5, chapter 4, section 2, line 8

                                   Κτησίας μὲν δή, εἰ δή 
τῳ ἱκανὸς καὶ Κτησίας ἐς τεκμηρίωσιν, ἵνα μὲν 
στενότατος αὐτὸς αὑτοῦ ὁ Ἰνδός ἐστι, τεσσαράκοντα 
σταδίους <λέγει> ὅτι διέχουσιν αὐτῷ αἱ ὄχθαι, ἵνα δὲ 
πλατύτατος, καὶ ἑκατόν· τὸ πολὺ δὲ εἶναι αὐτοῦ τὸ 
μέσον τούτοιν. 



Flavius Arrianus Hist., Phil., Alexandri anabasis 
Book 5, chapter 4, section 3, line 1

Τοῦτον τὸν ποταμὸν τὸν Ἰνδὸν ὑπὸ τὴν ἕω διέβαινε 
ξὺν τῇ στρατιᾷ Ἀλέξανδρος ἐς τῶν Ἰνδῶν τὴν γῆν· 
ὑπὲρ ὧν ἐγὼ οὔτε οἷστισι νόμοις διαχρῶνται ἐν τῇδε 
τῇ συγγραφῇ ἀνέγραψα, οὔτε ζῷα εἰ δή τινα ἄτοπα 
ἡ χώρα αὐτοῖς ἐκφέρει, οὔτε ἰχθύας ἢ κήτη ὅσα ἢ 
οἷα ὁ Ἰνδὸς ἢ ὁ Ὑδάσπης ἢ ὁ Γάγγης ἢ οἱ ἄλλοι 
Ἰνδῶν ποταμοὶ φέρουσιν, οὐδὲ τοὺς μύρμηκας τοὺς 
τὸν χρυσόν σφισιν ἐργαζομένους, οὐδὲ τοὺς γρῦπας 
τοὺς φύλακας, οὐδὲ ὅσα ἄλλα ἐφ' ἡδονῇ μᾶλλόν τι 
πεποίηται ἢ ἐς ἀφήγησιν τῶν ὄντων, ὡς τά γε κατ' 




Flavius Arrianus Hist., Phil., Alexandri anabasis 
Book 5, chapter 4, section 4, line 4

                 ἀλλὰ Ἀλέξανδρος γὰρ καὶ οἱ ξὺν 
τούτῳ στρατεύσαντες τὰ πολλὰ ἐξήλεγξαν, ὅσα γε μὴ   
καὶ αὐτῶν ἔστιν οἳ ἐψεύσαντο· ἀχρύσους τε εἶναι 
Ἰνδοὺς ἐξήλεγξαν, ὅσους γε δὴ Ἀλέξανδρος ξὺν τῇ 
στρατιᾷ ἐπῆλθε, πολλοὺς δὲ ἐπῆλθε, καὶ ἥκιστα 
χλιδῶντας κατὰ τὴν δίαιταν, ἀλλὰ μεγάλους μὲν τὰ 
σώματα, οἵους μεγίστους τῶν κατὰ τὴν Ἀσίαν, πεντα-
πήχεις τοὺς πολλοὺς ἢ ὀλίγον ἀποδέοντας, καὶ μελαν-
τέρους τῶν ἄλλων ἀνθρώπων, πλὴν Αἰθιόπων, καὶ τὰ 
πολέμια πολύ τι γενναιοτάτους τῶν γε δὴ τότε 
ἐποίκων τῆς Ἀσίας. 



Flavius Arrianus Hist., Phil., Alexandri anabasis 
Book 5, chapter 4, section 5, line 5

                         τὸ γὰρ Περσῶν τῶν πάλαι, ξὺν 
οἷς ὁρμηθεὶς Κῦρος ὁ Καμβύσου Μήδους τε τὴν 
ἀρχὴν τῆς Ἀσίας ἀφείλετο καὶ ἄλλα ἔθνη τὰ μὲν 
κατεστρέψατο, τὰ δὲ προσχωρήσαντά οἱ ἑκόντα κατέσχεν, 
οὐκ ἔχω ἀτρεκῶς ὥς γε δὴ πρὸς τὰ Ἰνδῶν ξυμβαλεῖν. 



Flavius Arrianus Hist., Phil., Alexandri anabasis 
Book 5, chapter 5, section 1, line 1

Ἀλλὰ ὑπὲρ Ἰνδῶν ἰδίᾳ μοι γεγράψεται ὅσα πιστό-
τατα ἐς ἀφήγησιν οἵ τε ξὺν Ἀλεξάνδρῳ στρατεύσαντες 
καὶ ὁ ἐκπεριπλεύσας τῆς μεγάλης θαλάσσης τὸ κατ' 
Ἰνδοὺς Νέαρχος, ἐπὶ δὲ ὅσα Μεγασθένης τε καὶ 
Ἐρατοσθένης, δοκίμω ἄνδρε, ξυνεγραψάτην, καὶ νόμιμα 
ἅττα Ἰνδοῖς ἐστι καὶ εἰ δή τινα ἄτοπα ζῷα αὐτόθι 
φύεται καὶ τὸν παράπλουν αὐτὸν τῆς ἔξω θαλάσσης. 



Flavius Arrianus Hist., Phil., Alexandri anabasis 
Book 5, chapter 5, section 4, line 2

ὃ δὴ Καύκασον ἐκάλουν οἱ Ἀλεξάνδρῳ ξυστρατεύσαντες 
Μακεδόνες, ὡς μὲν λέγεται τὰ Ἀλεξάνδρου αὔξοντες, 
ὅτι δὴ καὶ ἐπέκεινα ἄρα τοῦ Καυκάσου κρατῶν τοῖς 
ὅπλοις ἦλθεν Ἀλέξανδρος· τυχὸν δὲ καὶ ξυνεχὲς 
τυγχάνει ὂν τοῦτο τὸ ὄρος τῷ ἄλλῳ τῷ Σκυθικῷ 
Καυκάσῳ, καθάπερ οὖν αὐτῷ τούτῳ ὁ Ταῦρος· καὶ 
ἐμοὶ αὐτῷ πρότερόν ποτε ἐπὶ τῷδε λέλεκται Καύκασος 
τὸ ὄρος τοῦτο καὶ ὕστερον τῷδε τῷ ὀνόματι κλη-
θήσεται· τὸν δὲ Καύκασον τοῦτον καθήκειν ἔστε ἐπὶ 
<τὴν> μεγάλην τὴν πρὸς ἕω τε καὶ Ἰνδοὺς θάλασσαν. 



Flavius Arrianus Hist., Phil., Alexandri anabasis 
Book 5, chapter 5, section 5, line 2

τοὺς οὖν ποταμούς, ὅσοι κατὰ τὴν Ἀσίαν λόγου ἄξιοι, 
ἐκ τοῦ Ταύρου τε καὶ τοῦ Καυκάσου ἀνίσχοντας τοὺς 
μὲν ὡς ἐπ' ἄρκτον τετραμμένον ἔχειν τὸ ὕδωρ, καὶ 
τούτων τοὺς μὲν ἐς τὴν λίμνην ἐκδιδόναι τὴν 
Μαιῶτιν, τοὺς δὲ ἐς τὴν Ὑρκανίαν καλουμένην θάλας-
σαν, καὶ ταύτην κόλπον οὖσαν τῆς μεγάλης θαλάσσης, 
τοὺς δὲ ὡς ἐπὶ νότον ἄνεμον τὸν Εὐφράτην τε εἶναι 
καὶ τὸν Τίγρητα καὶ τὸν Ἰνδόν τε καὶ τὸν Ὑδάσπην 
καὶ Ἀκεσίνην καὶ Ὑδραώτην καὶ Ὕφασιν καὶ ὅσοι 
ἐν μέσῳ τούτων τε καὶ τοῦ Γάγγου ποταμοῦ ἐς   
θάλασσαν καὶ οὗτοι ἐσβάλλουσιν ἢ εἰς τενάγη ἀναχε-
όμενοι ἀφανίζονται, καθάπερ ὁ Εὐφράτης ποταμὸς 
ἀφανίζεται. 



Flavius Arrianus Hist., Phil., Alexandri anabasis 
Book 5, chapter 6, section 2, line 3

                                   τῆς δὲ ὡς ἐπὶ νότον 
Ἀσίας τετραχῇ αὖ τεμνομένης μεγίστην μὲν μοῖραν 
τὴν Ἰνδῶν γῆν ποιεῖ Ἐρατοσθένης τε καὶ Μεγασθένης, 
ὃς ξυνῆν μὲν Σιβυρτίῳ τῷ σατράπῃ τῆς Ἀραχωσίας, 
πολλάκις δὲ λέγει ἀφικέσθαι παρὰ Σανδράκοττον τὸν 
Ἰνδῶν βασιλέα, ἐλαχίστην δὲ ὅσην ὁ Εὐφράτης 
ποταμὸς ἀπείργει ὡς πρὸς τὴν ἐντὸς τὴν ἡμετέραν 
θάλασσαν. 



Flavius Arrianus Hist., Phil., Alexandri anabasis 
Book 5, chapter 6, section 2, line 9

           δύο δὲ αἱ μεταξὺ Εὐφράτου τε ποταμοῦ 
καὶ τοῦ Ἰνδοῦ ἀπειργόμεναι αἱ δύο ξυντεθεῖσαι μόλις 
ἄξιαι τῇ Ἰνδῶν γῇ ξυμβαλεῖν. 



Flavius Arrianus Hist., Phil., Alexandri anabasis 
Book 5, chapter 6, section 3, line 2

                                       ἀπείργεσθαι δὲ τὴν 
Ἰνδῶν χώραν πρὸς μὲν ἕω τε καὶ ἀπηλιώτην ἄνεμον 
ἔστε ἐπὶ μεσημβρίαν τῇ μεγάλῃ θαλάσσῃ· τὸ πρὸς 
βορρᾶν δὲ αὐτῆς ἀπείργειν τὸν Καύκασον τὸ ὄρος 
ἔστε ἐπὶ τοῦ Ταύρου τὴν ξυμβολήν· τὴν δὲ ὡς πρὸς 
ἑσπέραν τε καὶ ἄνεμον Ἰάπυγα ἔστε ἐπὶ τὴν μεγάλην 
θάλασσαν ὁ Ἰνδὸς ποταμὸς ἀποτέμνεται. 



Flavius Arrianus Hist., Phil., Alexandri anabasis 
Book 5, chapter 6, section 6, line 6

                                             εἰ δὴ οὖν 
εἷς τε ποταμὸς παρ' ἑκάστοις καὶ οὐ μεγάλοι οὗτοι   
ποταμοὶ ἱκανοὶ γῆν πολλὴν ποιῆσαι ἐς θάλασσαν 
προχεόμενοι, ὁπότε ἰλὺν καταφέροιεν καὶ πηλὸν ἐκ 
τῶν ἄνω τόπων ἔνθενπερ αὐτοῖς αἱ πηγαί εἰσιν, οὐδὲ 
ὑπὲρ τῆς Ἰνδῶν ἄρα χώρας ἐς ἀπιστίαν ἰέναι ἄξιον, 
ὅπως πεδίον τε ἡ πολλή ἐστι καὶ ἐκ τῶν ποταμῶν τὸ 
πεδίον ἔχει προσκεχωσμένον. 



Flavius Arrianus Hist., Phil., Alexandri anabasis 
Book 5, chapter 6, section 7, line 4

                                Ἕρμον μὲν γὰρ καὶ 
Κάϋστρον καὶ Κάϊκόν τε καὶ Μαίανδρον ἢ ὅσοι πολλοὶ 
ποταμοὶ τῆς Ἀσίας ἐς τήνδε τὴν ἐντὸς θάλασσαν 
ἐκδιδοῦσιν οὐδὲ σύμπαντας ξυντεθέντας ἑνὶ τῶν Ἰνδῶν 
ποταμῶν ἄξιον ξυμβαλεῖν πλήθους ἕνεκα τοῦ ὕδατος, 
μὴ ὅτι τῷ Γάγγῃ τῷ μεγίστῳ, ὅτῳ οὔτε <τὸ> τοῦ 
Νείλου ὕδωρ τοῦ Αἰγυπτίου οὔτε ὁ Ἴστρος ὁ κατὰ τὴν 
Εὐρώπην ῥέων ἄξιοι ξυμβαλεῖν, ἀλλ' οὐδὲ τῷ Ἰνδῷ 
ποταμῷ ἐκεῖνοί γε πάντες ξυμμιχθέντες ἐς ἴσον ἔρχονται, 
ὃς μέγας τε εὐθὺς ἀπὸ τῶν πηγῶν ἀνίσχει καὶ πεντε-
καίδεκα ποταμοὺς πάντας τῶν Ἀσιανῶν μείζονας παρα-
λαβὼν καὶ τῇ ἐπωνυμίᾳ κρατήσας οὕτως ἐκδιδοῖ ἐς 
θάλασσαν. 



Flavius Arrianus Hist., Phil., Alexandri anabasis 
Book 5, chapter 6, section 8, line 6

           ταῦτά μοι ἐν τῷ παρόντι περὶ Ἰνδῶν τῆς 
χώρας λελέχθω· τὰ δὲ ἄλλα ἀποκείσθω ἐς τὴν Ἰνδικὴν 
ξυγγραφήν. 



Flavius Arrianus Hist., Phil., Alexandri anabasis 
Book 5, chapter 7, section 1, line 1

Τὸ δὲ ζεῦγμα τὸ ἐπὶ τοῦ Ἰνδοῦ ποταμοῦ ὅπως 
μὲν ἐποιήθη Ἀλεξάνδρῳ οὔτε Ἀριστόβουλος οὔτε 
Πτολεμαῖος, οἷς μάλιστα ἐγὼ ἕπομαι, λέγουσιν· οὐδὲ 
αὐτὸς ἔχω ἀτρεκῶς εἰκάσαι, πότερα πλοίοις ἐζεύχθη ὁ 
πόρος, καθάπερ οὖν ὁ Ἑλλήσποντός τε πρὸς Ξέρξου 
καὶ ὁ Βόσπορός τε καὶ ὁ Ἴστρος πρὸς Δαρείου, ἢ 
γέφυρα κατὰ τοῦ ποταμοῦ διηνεκὴς ἐποιήθη αὐτῷ·   
δοκεῖ δ' ἔμοιγε πλοίοις μᾶλλον ζευχθῆναι· οὐ γὰρ 
ἂν δέξασθαι γέφυραν τὸ βάθος τοῦ ὕδατος, οὐδ' ἂν 
ἐν τοσῷδε χρόνῳ ἔργον οὕτως ἄτοπον ξυντελεσθῆναι. 



Flavius Arrianus Hist., Phil., Alexandri anabasis 
Book 5, chapter 8, section 1, line 2

Ῥωμαίοις μὲν δὴ οὕτω ταῦτα ἐκ παλαιοῦ ἐπήσκηται· 
Ἀλεξάνδρῳ δὲ ὅπως ἐζεύχθη ὁ Ἰνδὸς ποταμὸς οὐκ 
ἔχω εἰπεῖν, ὅτι μηδὲ οἱ συστρατεύσαντες αὐτῷ εἶπον. 



Flavius Arrianus Hist., Phil., Alexandri anabasis 
Book 5, chapter 8, section 2, line 2

                                              ὡς δὲ διέβη πέραν 
τοῦ Ἰνδοῦ ποταμοῦ, καὶ ἐνταῦθα αὖ θύει κατὰ νόμον 
Ἀλέξανδρος. 



Flavius Arrianus Hist., Phil., Alexandri anabasis 
Book 5, chapter 8, section 2, line 3

               ἄρας δὲ ἀπὸ τοῦ Ἰνδοῦ ἐς Τάξιλα 
ἀφίκετο, πόλιν μεγάλην καὶ εὐδαίμονα, τὴν μεγίστην 
τῶν μεταξὺ Ἰνδοῦ τε ποταμοῦ καὶ Ὑδάσπου. 



Flavius Arrianus Hist., Phil., Alexandri anabasis 
Book 5, chapter 8, section 2, line 7

                                                   καὶ 
ἐδέχετο αὐτὸν Ταξίλης ὁ ὕπαρχος τῆς πόλεως καὶ 
αὐτοὶ οἱ τῇδε Ἰνδοὶ φιλίως. 



Flavius Arrianus Hist., Phil., Alexandri anabasis 
Book 5, chapter 8, section 3, line 3

                                                         ἧκον 
δὲ ἐνταῦθα παρ' αὐτὸν καὶ παρὰ Ἀβισάρου πρέσβεις 
τοῦ τῶν ὀρείων Ἰνδῶν βασιλέως ὅ τε ἀδελφὸς τοῦ 
Ἀβισάρου καὶ ἄλλοι ξὺν αὐτῷ οἱ δοκιμώτατοι, καὶ 
παρὰ Δοξάρεως νομάρχου ἄλλοι, δῶρα φέροντες. 



Flavius Arrianus Hist., Phil., Alexandri anabasis 
Book 5, chapter 8, section 3, line 8

                                                 καὶ ἀπο-
δείξας σατράπην τῶν ταύτῃ Ἰνδῶν Φίλιππον τὸν   
Μαχάτα φρουράν τε ἀπολείπει ἐν Ταξίλοις καὶ τοὺς 
ἀπομάχους τῶν στρατιωτῶν διὰ νόσον· αὐτὸς δὲ ἦγεν 
ὡς ἐπὶ τὸν Ὑδάσπην ποταμόν. 



Flavius Arrianus Hist., Phil., Alexandri anabasis 
Book 5, chapter 8, section 4, line 5

                                           ταῦτα ὡς ἔγνω 
Ἀλέξανδρος, Κοῖνον μὲν τὸν Πολεμοκράτους πέμψας 
ὀπίσω ἐπὶ τὸν Ἰνδὸν ποταμὸν τὰ πλοῖα ὅσα παρ-
εσκεύαστο αὐτῷ ἐπὶ τοῦ πόρου τοῦ Ἰνδοῦ ξυντεμόντα 
κελεύει φέρειν ὡς ἐπὶ τὸν Ὑδάσπην ποταμόν. 



Flavius Arrianus Hist., Phil., Alexandri anabasis 
Book 5, chapter 8, section 5, line 8

                                                         αὐτὸς 
δὲ ἀναλαβὼν ἥν τε δύναμιν ἔχων ἧκεν ἐς Τάξιλα 
καὶ πεντακισχιλίους τῶν Ἰνδῶν, οὓς Ταξίλης τε καὶ 
οἱ ταύτῃ ὕπαρχοι ἦγον, ᾔει ὡς ἐπὶ τὸν Ὑδάσπην 
ποταμόν. 



Flavius Arrianus Hist., Phil., Alexandri anabasis 
Book 5, chapter 9, section 4, line 2

                                     ἄλλως τε ἐν μὲν 
τῷ τότε οἱ ποταμοὶ πάντες οἱ Ἰνδικοὶ πολλοῦ τε 
ὕδατος καὶ θολεροῦ ἔρρεον καὶ ὀξέος τοῦ ῥεύματος· 
ἦν γὰρ ὥρα ἔτους ᾗ μετὰ τροπὰς μάλιστα <τὰς> ἐν 
θέρει τρέπεται ὁ ἥλιος· ταύτῃ δὲ τῇ ὥρᾳ ὕδατά τε 
ἐξ οὐρανοῦ ἀθρόα τε καταφέρεται ἐς τὴν γῆν τὴν   
Ἰνδικὴν καὶ αἱ χιόνες αἱ τοῦ Καυκάσου, ἔνθενπερ 
τῶν πολλῶν ποταμῶν αἱ πηγαί εἰσι, κατατηκόμεναι 
αὔξουσιν αὐτοῖς τὸ ὕδωρ ἐπὶ μέγα· χειμῶνος δὲ 
ἔμπαλιν ἴσχουσιν ὀλίγοι τε γίγνονται καὶ καθαροὶ 
ἰδεῖν καὶ ἔστιν ὅπου περάσιμοι, πλήν γε δὴ τοῦ 




Flavius Arrianus Hist., Phil., Alexandri anabasis 
Book 5, chapter 9, section 4, line 12

τῷ τότε οἱ ποταμοὶ πάντες οἱ Ἰνδικοὶ πολλοῦ τε 
ὕδατος καὶ θολεροῦ ἔρρεον καὶ ὀξέος τοῦ ῥεύματος· 
ἦν γὰρ ὥρα ἔτους ᾗ μετὰ τροπὰς μάλιστα <τὰς> ἐν 
θέρει τρέπεται ὁ ἥλιος· ταύτῃ δὲ τῇ ὥρᾳ ὕδατά τε 
ἐξ οὐρανοῦ ἀθρόα τε καταφέρεται ἐς τὴν γῆν τὴν   
Ἰνδικὴν καὶ αἱ χιόνες αἱ τοῦ Καυκάσου, ἔνθενπερ 
τῶν πολλῶν ποταμῶν αἱ πηγαί εἰσι, κατατηκόμεναι 
αὔξουσιν αὐτοῖς τὸ ὕδωρ ἐπὶ μέγα· χειμῶνος δὲ 
ἔμπαλιν ἴσχουσιν ὀλίγοι τε γίγνονται καὶ καθαροὶ 
ἰδεῖν καὶ ἔστιν ὅπου περάσιμοι, πλήν γε δὴ τοῦ 
Ἰνδοῦ καὶ Γάγγου καὶ τυχὸν καὶ ἄλλου του· ἀλλ' ὅ 
γε Ὑδάσπης περατὸς γίνεται. 



Flavius Arrianus Hist., Phil., Alexandri anabasis 
Book 5, chapter 11, section 3, line 8

                καὶ Κρατερὸς ὑπελέλειπτο ἐπὶ στρατο-  
πέδου τήν τε αὑτοῦ ἔχων ἱππαρχίαν καὶ τοὺς ἐξ 
Ἀραχωτῶν καὶ Παραπαμισαδῶν ἱππέας καὶ τῆς φάλαγ-
γος τῶν Μακεδόνων τήν τε Ἀλκέτου καὶ τὴν Πολυ-
πέρχοντος τάξιν καὶ τοὺς νομάρχας τῶν ἐπὶ τάδε 
Ἰνδῶν καὶ τοὺς ἅμα τούτοις τοὺς πεντακισχιλίους. 



Flavius Arrianus Hist., Phil., Alexandri anabasis 
Book 5, chapter 12, section 1, line 7

                                       ἐν μέσῳ δὲ τῆς 
νήσου τε καὶ τοῦ μεγάλου στρατοπέδου, ἵνα αὐτῷ 
Κρατερὸς ὑπελέλειπτο, Μελέαγρός τε καὶ Ἄτταλος καὶ 
Γοργίας ξὺν τοῖς μισθοφόροις ἱππεῦσί τε καὶ πεζοῖς 
ἐτετάχατο· καὶ τούτοις διαβαίνειν παρηγγέλλετο κατὰ 
μέρος, διελόντας τὸν στρατόν, ὁπότε ξυνεχομένους ἤδη 
ἐν τῇ μάχῃ τοὺς Ἰνδοὺς ἴδοιεν. 



Flavius Arrianus Hist., Phil., Alexandri anabasis 
Book 5, chapter 14, section 2, line 6

γνώμην δὲ ἐπεποίητο, ὡς εἰ μὲν προσμίξειαν αὐτῷ οἱ 
ἀμφὶ τὸν Πῶρον ξὺν τῇ δυνάμει ἁπάσῃ, ἢ κρατήσειν 
αὐτῶν οὐ χαλεπῶς τῇ ἵππῳ προσβαλὼν ἢ ἀπομαχεῖσθαί 
γε ἔστε τοὺς πεζοὺς ἐν τῷ ἔργῳ ἐπιγενέσθαι· εἰ δὲ 
πρὸς τὴν τόλμαν τῆς διαβάσεως ἄτοπον γενομένην οἱ 
Ἰνδοὶ ἐκπλαγέντες φεύγοιεν, οὐ πόρρωθεν ἕξεσθαι 
αὐτῶν κατὰ τὴν φυγήν, ὡς πλείονα ἐν τῇ ἀποχωρήσει τὸν 
φόνον γενόμενον ὀλίγον ἔτι ὑπολείπεσθαι αὐτῷ τὸ ἔργον. 



Flavius Arrianus Hist., Phil., Alexandri anabasis 
Book 5, chapter 14, section 3, line 6

Ἀριστόβουλος δὲ λέγει τὸν Πώρου παῖδα φθάσαι 
ἀφικόμενον σὺν ἅρμασιν ὡς ἑξήκοντα πρὶν τὸ ὕστε-
ρον ἐκ τῆς νήσου τῆς μικρᾶς περᾶσαι Ἀλέξανδρον· 
καὶ τοῦτον δυνηθῆναι ἂν εἶρξαι Ἀλέξανδρον τῆς 
διαβάσεως χαλεπῶς καὶ μηδενὸς εἴργοντος περαιω-
θέντα, εἴπερ οὖν καταπηδήσαντες οἱ Ἰνδοὶ ἐκ τῶν 
ἁρμάτων προσέκειντο τοῖς πρώτοις τῶν ἐκβαινόντων·   
ἀλλὰ παραλλάξαι γὰρ ξὺν τοῖς ἅρμασι καὶ ἀκίνδυνον 
ποιῆσαι Ἀλεξάνδρῳ τὴν διάβασιν· καὶ ἐπὶ τούτους 
ἀφεῖναι Ἀλέξανδρον τοὺς ἱπποτοξότας, καὶ τραπῆναι 
αὐτοὺς οὐ χαλεπῶς, πληγὰς λαμβάνοντας. 



Flavius Arrianus Hist., Phil., Alexandri anabasis 
Book 5, chapter 14, section 4, line 2

                                             οἱ δὲ καὶ 
μάχην λέγουσιν ἐν τῇ ἐκβάσει γενέσθαι τῶν Ἰνδῶν 
τῶν ξὺν τῷ παιδὶ τῷ Πώρου ἀφιγμένων πρὸς Ἀλέξ-
ανδρόν τε καὶ τοὺς ξὺν αὐτῷ ἱππέας. 



Flavius Arrianus Hist., Phil., Alexandri anabasis 
Book 5, chapter 15, section 2, line 1

ὡς δὲ κατέμαθεν ἀτρεκῶς τὸ πλῆθος τὸ τῶν Ἰνδῶν, 
ἐνταῦθα δὴ ὀξέως ἐπιπεσεῖν αὐτοῖς ξὺν τῇ ἀμφ' αὑτὸν 
ἵππῳ· τοὺς δὲ ἐγκλῖναι, ὡς Ἀλέξανδρόν τε αὐτὸν 
κατεῖδον καὶ τὸ στῖφος ἀμφ' αὐτὸν τῶν ἱππέων οὐκ 
ἐπὶ μετώπου, ἀλλὰ κατ' ἴλας ἐμβεβληκός. 



Flavius Arrianus Hist., Phil., Alexandri anabasis 
Book 5, chapter 16, section 1, line 2

                                      Ἀλέξανδρος δὲ ὡς 
ἤδη καθεώρα τοὺς Ἰνδοὺς ἐκτασσομένους, ἐπέστησε 
τοὺς ἱππέας τοῦ πρόσω, ὡς ἀναλαμβάνειν τῶν πεζῶν 
τοὺς ἀεὶ προσάγοντας. 



Flavius Arrianus Hist., Phil., Alexandri anabasis 
Book 5, chapter 16, section 2, line 1

             ὡς δὲ τὴν τάξιν κατεῖδε τῶν Ἰνδῶν, κατὰ 
μέσον μέν, ἵνα οἱ ἐλέφαντες προεβέβληντο καὶ πυκνὴ 
ἡ φάλαγξ κατὰ τὰ διαλείποντα αὐτῶν ἐπετέτακτο, οὐκ 
ἔγνω προάγειν, αὐτὰ ἐκεῖνα ὀκνήσας ἅπερ ὁ Πῶρος 
τῷ λογισμῷ ξυνθεὶς ταύτῃ ἔταξεν· ἀλλὰ αὐτὸς μὲν 
ἅτε ἱπποκρατῶν τὴν πολλὴν τῆς ἵππου ἀναλαβὼν ἐπὶ 
τὸ εὐώνυμον κέρας τῶν πολεμίων παρήλαυνεν, ὡς 
ταύτῃ ἐπιθησόμενος. 



Flavius Arrianus Hist., Phil., Alexandri anabasis 
Book 5, chapter 16, section 4, line 2

Ἤδη τε ἐντὸς βέλους ἐγίγνετο καὶ ἐφῆκεν ἐπὶ τὸ 
κέρας τὸ εὐώνυμον τῶν Ἰνδῶν τοὺς ἱπποτοξότας, 
ὄντας ἐς χιλίους, ὡς ταράξαι τοὺς ταύτῃ ἐφεστηκότας 
τῶν πολεμίων τῇ πυκνότητί τε τῶν τοξευμάτων καὶ 
τῶν ἵππων τῇ ἐπελάσει. 



Flavius Arrianus Hist., Phil., Alexandri anabasis 
Book 5, chapter 17, section 1, line 1

Ἐν τούτῳ δὲ οἵ τε Ἰνδοὶ τοὺς ἱππέας πάντοθεν 
ξυναλίσαντες παρίππευον Ἀλεξάνδρῳ ἀντιπαρεξάγοντες 
τῇ ἐλάσει, καὶ οἱ περὶ Κοῖνον, ὡς παρήγγελτο, κατόπιν   
αὐτοῖς ἐπεφαίνοντο. 



Flavius Arrianus Hist., Phil., Alexandri anabasis 
Book 5, chapter 17, section 1, line 4

                        ταῦτα ξυνιδόντες οἱ Ἰνδοὶ ἀμφί-
στομον ἠναγκάσθησαν ποιῆσαι τὴν τάξιν τῆς ἵππου, 
τὴν μὲν ὡς ἐπ' Ἀλέξανδρον τὴν πολλήν τε καὶ 
κρατίστην, οἱ δὲ ἐπὶ Κοῖνόν τε καὶ τοὺς ἅμα τούτῳ 
ἐπέστρεφον. 



Flavius Arrianus Hist., Phil., Alexandri anabasis 
Book 5, chapter 17, section 2, line 2

              τοῦτό τε οὖν εὐθὺς ἐτάραξε τὰς τάξεις 
τε καὶ τὰς γνώμας τῶν Ἰνδῶν καὶ Ἀλέξανδρος ἰδὼν 
τὸν καιρὸν ἐν αὐτῇ τῇ ἐπὶ θάτερα ἐπιστροφῇ τῆς 
ἵππου ἐπιτίθεται τοῖς καθ' αὑτόν, ὥστε οὐδὲ τὴν 
ἐμβολὴν ἐδέξαντο τῶν ἀμφ' Ἀλέξανδρον ἱππέων οἱ 
Ἰνδοί, ἀλλὰ κατηρ[ρ]άχθησαν ὥσπερ εἰς τεῖχός τι φίλιον 
τοὺς ἐλέφαντας. 



Flavius Arrianus Hist., Phil., Alexandri anabasis 
Book 5, chapter 17, section 3, line 9

                                         καὶ ἦν τὸ 
ἔργον οὐδενὶ τῶν πρόσθεν ἀγώνων ἐοικός· τά τε γὰρ 
θηρία ἐπεκθέοντα ἐς τὰς τάξεις τῶν πεζῶν, ὅπῃ 
ἐπιστρέψειεν, ἐκεράϊζε καίπερ πυκνὴν οὖσαν τὴν τῶν 
Μακεδόνων φάλαγγα, καὶ οἱ ἱππεῖς οἱ τῶν Ἰνδῶν 
τοῖς πεζοῖς ἰδόντες ξυνεστηκὸς τὸ ἔργον ἐπιστρέψαντες 
αὖθις καὶ αὐτοὶ ἐπήλαυνον τῇ ἵππῳ. 



Flavius Arrianus Hist., Phil., Alexandri anabasis 
Book 5, chapter 17, section 4, line 7

                       καὶ ἐν τούτῳ πᾶσα ἡ ἵππος 
Ἀλεξάνδρῳ ἐς μίαν ἴλην ἤδη ξυνηγμένη, οὐκ ἐκ παρ-
αγγέλματος, ἀλλὰ ἐν τῷ ἀγῶνι αὐτῷ ἐς τήνδε τὴν 
τάξιν καταστᾶσα, ὅπῃ προσπέσοι τῶν Ἰνδῶν ταῖς 
τάξεσι, ξὺν πολλῷ φόνῳ ἀπελύοντο. 



Flavius Arrianus Hist., Phil., Alexandri anabasis 
Book 5, chapter 17, section 6, line 7

          ἀλλ' οἱ μὲν Μακεδόνες, ἅτε ἐν εὐρυχωρίᾳ 
τε καὶ κατὰ γνώμην τὴν σφῶν προσφερόμενοι τοῖς 
θηρίοις, ὅπῃ μὲν ἐπιφέροιντο εἶκον, ἀποστραφέντων 
δὲ εἴχοντο ἐσακοντίζοντες· οἱ δὲ Ἰνδοὶ ἐν αὐτοῖς 
ἀναστρεφόμενοι τὰ πλείω ἤδη πρὸς ἐκείνων ἐβλάπτοντο. 



Flavius Arrianus Hist., Phil., Alexandri anabasis 
Book 5, chapter 17, section 7, line 8

                         καὶ οὕτως οἱ μὲν ἱππεῖς τῶν 
Ἰνδῶν πλὴν ὀλίγων κατεκόπησαν ἐν τῷ ἔργῳ· ἐκό-
πτοντο δὲ καὶ οἱ πεζοὶ πανταχόθεν ἤδη προσκειμένων 
σφίσι τῶν Μακεδόνων. 



Flavius Arrianus Hist., Phil., Alexandri anabasis 
Book 5, chapter 18, section 1, line 6

             καὶ οὗτοι οὐ μείονα τὸν φόνον ἐν τῇ 
ἀποχωρήσει τῶν Ἰνδῶν ἐποίησαν, ἀκμῆτες ἀντὶ κεκμη-
κότων τῶν ἀμφ' Ἀλέξανδρον ἐπιγενόμενοι τῇ διώξει. 



Flavius Arrianus Hist., Phil., Alexandri anabasis 
Book 5, chapter 18, section 2, line 1

Ἀπέθανον δὲ τῶν Ἰνδῶν πεζοὶ μὲν ὀλίγον ἀπο-
δέοντες τῶν δισμυρίων, ἱππεῖς δὲ ἐς τρισχιλίους, τὰ 
δὲ ἅρματα ξύμπαντα κατεκόπη· καὶ Πώρου δύο παῖδες 
ἀπέθανον καὶ Σπιτάκης ὁ νομάρχης τῶν ταύτῃ Ἰνδῶν 
καὶ τῶν ἐλεφάντων καὶ ἁρμάτων οἱ ἡγεμόνες καὶ οἱ 
ἱππάρχαι καὶ οἱ στρατηγοὶ τῆς στρατιᾶς τῆς Πώρου 
ξύμπαντες . 



Flavius Arrianus Hist., Phil., Alexandri anabasis 
Book 5, chapter 18, section 5, line 1

Πῶρος δὲ μεγάλα ἔργα ἐν τῇ μάχῃ ἀποδειξάμενος 
μὴ ὅτι στρατηγοῦ, ἀλλὰ καὶ στρατιώτου γενναίου, ὡς 
τῶν τε ἱππέων τὸν φόνον κατεῖδε καὶ τῶν ἐλεφάντων 
τοὺς μὲν αὐτοῦ πεπτωκότας, τοὺς δὲ ἐρήμους τῶν 
ἡγεμόνων λυπηροὺς πλανωμένους, τῶν δὲ πεζῶν αὐτῷ 
οἱ πλείους ἀπολώλεσαν, οὐχ ᾗπερ Δαρεῖος ὁ μέγας 
βασιλεὺς ἐξάρχων τοῖς ἀμφ' αὑτὸν τῆς φυγῆς ἀπεχώρει, 
ἀλλὰ ἔστε γὰρ ὑπέμενέ τι τῶν Ἰνδῶν ἐν τῇ μάχῃ   
ξυνεστηκός, ἐς τοσόνδε ἀγωνισάμενος, τετρωμένος δὲ τὸν 
δεξιὸν ὦμον, ὃν δὴ γυμνὸν μόνον ἔχων ἐν τῇ μάχῃ ἀν-
εστρέφετο (ἀπὸ γὰρ τοῦ ἄλλου σώματος ἤρκει αὐτῷ τὰ 
βέλη ὁ θώραξ περιττὸς ὢν κατά τε τὴν ἰσχὺν καὶ τὴν 
ἁρμονίαν, ὡς ὕστερον καταμαθεῖν θεωμένοις ἦν), τότε 
δὴ καὶ αὐτὸς ἀπεχώρει ἐπιστρέψας τὸν ἐλέφαντα. 



Flavius Arrianus Hist., Phil., Alexandri anabasis 
Book 5, chapter 18, section 6, line 4

                                    πέμπει δὴ παρ' αὐτὸν 
πρῶτα μὲν Ταξίλην τὸν Ἰνδόν· καὶ Ταξίλης προς-
ιππεύσας ἐφ' ὅσον οἱ ἀσφαλὲς ἐφαίνετο τῷ ἐλέφαντι 
ὃς ἔφερε τὸν Πῶρον ἐπιστῆσαί τε ἠξίου τὸ θηρίον, 
οὐ γὰρ εἶναί οἱ ἔτι φεύγειν, καὶ ἀκοῦσαι τῶν παρ' 
Ἀλεξάνδρου λόγων. 



Flavius Arrianus Hist., Phil., Alexandri anabasis 
Book 5, chapter 18, section 7, line 7

Ἀλέξανδρος δὲ οὐδὲ ἐπὶ τῷδε τῷ Πώρῳ χαλεπὸς ἐγέ-
νετο, ἀλλ' ἄλλους τε ἐν μέρει ἔπεμπε καὶ δὴ καὶ 
Μερόην ἄνδρα Ἰνδόν, ὅτι φίλον εἶναι ἐκ παλαιοῦ τῷ 
Πώρῳ τὸν Μερόην ἔμαθεν. 



Flavius Arrianus Hist., Phil., Alexandri anabasis 
Book 5, chapter 19, section 3, line 2

           καὶ Ἀλέξανδρος τούτῳ ἔτι μᾶλλον τῷ λόγῳ 
ἡσθεὶς τήν τε ἀρχὴν τῷ Πώρῳ τῶν τε αὐτῶν Ἰνδῶν 
ἔδωκεν καὶ ἄλλην ἔτι χώραν πρὸς τῇ πάλαι οὔσῃ 
πλείονα τῆς πρόσθεν προσέθηκεν· καὶ οὕτως αὐτός 
τε βασιλικῶς κεχρημένος ἦν ἀνδρὶ ἀγαθῷ καὶ ἐκείνῳ 
ἐκ τούτου ἐς ἅπαντα πιστῷ ἐχρήσατο. 



Flavius Arrianus Hist., Phil., Alexandri anabasis 
Book 5, chapter 19, section 3, line 8

                                             τοῦτο τὸ τέλος 
τῇ μάχῃ τῇ πρὸς Πῶρόν τε καὶ τοὺς ἐπέκεινα τοῦ 
Ὑδάσπου ποταμοῦ Ἰνδοὺς Ἀλεξάνδρῳ ἐγένετο ἐπ' 
ἄρχοντος Ἀθηναίοις Ἡγήμονος μηνὸς Μουνυχιῶνος. 



Flavius Arrianus Hist., Phil., Alexandri anabasis 
Book 5, chapter 19, section 4, line 3

καὶ τὴν μὲν Νίκαιαν τῆς νίκης τῆς κατ' Ἰνδῶν ἐπώ-
νυμον ὠνόμασε, τὴν δὲ Βουκεφάλαν ἐς τοῦ ἵππου τοῦ   
Βουκεφάλα τὴν μνήμην, ὃς ἀπέθανεν αὐτοῦ, οὐ βληθεὶς 
πρὸς οὐδενός, ἀλλὰ ὑπὸ καύματος τε καὶ ἡλικίας (ἦν 
γὰρ ἀμφὶ τὰ τριάκοντα ἔτη) καματηρὸς γενόμενος, 
πολλὰ δὲ πρόσθεν ξυγκαμών τε καὶ συγκινδυνεύσας 
Ἀλεξάνδρῳ, ἀναβαινόμενός τε πρὸς μόνου Ἀλεξάνδρου 
[ὁ Βουκεφάλας οὗτος], ὅτι τοὺς ἄλλους πάντας ἀπηξίου 
ἀμβάτας, καὶ μεγέθει μέγας καὶ τῷ θυμῷ γενναῖος. 



Flavius Arrianus Hist., Phil., Alexandri anabasis 
Book 5, chapter 20, section 2, line 5

                                                 Κρατερὸν 
μὲν δὴ ξὺν μέρει τῆς στρατιᾶς ὑπελείπετο τὰς πόλεις 
ἅστινας ταύτῃ ἔκτιζεν ἀναστήσοντά τε καὶ ἐκτειχιοῦντα· 
αὐτὸς δὲ ἤλαυνεν ὡς ἐπὶ τοὺς προσχώρους τῇ Πώρου 
ἀρχῇ Ἰνδούς. 



Flavius Arrianus Hist., Phil., Alexandri anabasis 
Book 5, chapter 20, section 6, line 2

                                    ἧκον δὲ καὶ παρὰ 
τῶν αὐτονόμων Ἰνδῶν πρέσβεις παρ' Ἀλέξανδρον καὶ 
παρὰ Πώρου ἄλλου του ὑπάρχου Ἰνδῶν. 



Flavius Arrianus Hist., Phil., Alexandri anabasis 
Book 5, chapter 20, section 8, line 3

        τούτου τοῦ Ἀκεσίνου τὸ μέγεθος μόνου τῶν 
Ἰνδῶν ποταμῶν Πτολεμαῖος ὁ Λάγου ἀνέγραψεν· 
εἶναι γὰρ ἵνα ἐπέρασεν αὐτὸν Ἀλέξανδρος ἐπὶ τῶν 
πλοίων τε καὶ τῶν διφθερῶν ξὺν τῇ στρατιᾷ τὸ μὲν 
ῥεῦμα ὀξὺ τοῦ Ἀκεσίνου πέτραις μεγάλαις καὶ ὀξείαις, 
καθ' ὧν φερόμενον βίᾳ τὸ ὕδωρ κυμαίνεσθαί τε καὶ 
καχλάζειν, τὸ δὲ εὖρος σταδίους ἐπέχειν πεντεκαίδεκα. 



Flavius Arrianus Hist., Phil., Alexandri anabasis 
Book 5, chapter 20, section 10, line 3

                           εἴη ἂν οὖν ἐκ τοῦδε τοῦ 
λόγου ξυντιθέντι τεκμηριοῦσθαι, ὅτι οὐ πόρρω τοῦ 
ἀληθοῦς ἀναγέγραπται τοῦ Ἰνδοῦ ποταμοῦ τὸ μέγεθος, 
ὅσοις ἐς τεσσαράκοντα σταδίους δοκεῖ τοῦ Ἰνδοῦ εἶναι   
τὸ εὖρος, ἵνα μέσως ἔχει αὐτὸς αὑτοῦ ὁ Ἰνδός· ἵνα δὲ 
στενότατός τε καὶ διὰ στενότητα βαθύτατος ἐς τοὺς 
πεντεκαίδεκα ξυνάγεσθαι· καὶ ταῦτα πολλαχῇ εἶναι τοῦ 
Ἰνδοῦ. 



Flavius Arrianus Hist., Phil., Alexandri anabasis 
Book 5, chapter 21, section 1, line 5

Περάσας δὲ τὸν ποταμὸν Κοῖνον μὲν ξὺν τῇ αὑτοῦ 
τάξει ἀπολείπει αὐτοῦ ἐπὶ τῇ ὄχθῃ προστάξας ἐπι-
μελεῖσθαι τῆς ὑπολελειμμένης στρατιᾶς τῆς διαβάσεως, 
οἳ τόν τε σῖτον αὐτῷ τὸν ἐκ τῆς ἤδη ὑπηκόου τῶν 
Ἰνδῶν χώρας καὶ τὰ ἄλλα ἐπιτήδεια παρακομίζειν 
ἔμελλον. 



Flavius Arrianus Hist., Phil., Alexandri anabasis 
Book 5, chapter 21, section 2, line 2

           Πῶρον δὲ ἐς τὰ αὑτοῦ ἤθη ἀποπέμπει, 
κελεύσας Ἰνδῶν τε τοὺς μαχιμωτάτους ἐπιλεξάμενον 
καὶ εἴ τινας παρ' αὑτῷ ἔχοι ἐλέφαντας, τούτους δὲ 
ἀναλαβόντα[ς] ἰέναι παρ' αὑτόν. 



Flavius Arrianus Hist., Phil., Alexandri anabasis 
Book 5, chapter 21, section 4, line 2

Ἐπὶ τοῦτον ἐλαύνων Ἀλέξανδρος ἀφικνεῖται ἐπὶ 
τὸν Ὑδραώτην ποταμόν, ἄλλον αὖ τοῦτον Ἰνδὸν   
ποταμόν, τὸ μὲν εὖρος οὐ μείονα τοῦ Ἀκεσίνου, ὀξύτητι 
δὲ τοῦ ῥοῦ μείονα. 



Flavius Arrianus Hist., Phil., Alexandri anabasis 
Book 5, chapter 21, section 5, line 7

                ἐνταῦθα Ἡφαιστίωνα μὲν ἐκπέμπει 
δοὺς αὐτῷ μέρος τῆς στρατιᾶς, πεζῶν μὲν φάλαγγας 
δύο, ἱππέων δὲ τήν τε αὑτοῦ καὶ τὴν Δημητρίου 
ἱππαρχίαν καὶ τῶν τοξοτῶν τοὺς ἡμίσεας, ἐς τὴν 
Πώρου τοῦ ἀφεστηκότος χώραν, κελεύσας παραδιδόναι 
ταύτην Πώρῳ τῷ ἄλλῳ, καὶ εἰ δή τινα πρὸς ταῖς 
ὄχθαις τοῦ Ὑδραώτου ποταμοῦ αὐτόνομα ἔθνη Ἰνδῶν 
νέμεται, καὶ ταῦτα προσαγαγόμενον τῷ Πώρῳ ἄρχειν 
ἐγχειρίσαι. 



Flavius Arrianus Hist., Phil., Alexandri anabasis 
Book 5, chapter 22, section 1, line 2

Ἐν τούτῳ δὲ ἐξαγγέλλεται Ἀλεξάνδρῳ τῶν αὐτο-
νόμων Ἰνδῶν ἄλλους τέ τινας καὶ τοὺς καλουμένους 
Καθαίους αὐτούς τε παρασκευάζεσθαι ὡς πρὸς μάχην, 
εἰ προσάγοι τῇ χώρᾳ αὐτῶν Ἀλέξανδρος, καὶ ὅσα ὅμορά 
σφισιν <ἔθνη> ὡσαύτως αὐτόνομα, καὶ ταῦτα παρα-
καλεῖν ἐς τὸ ἔργον· εἶναι δὲ τήν τε πόλιν ὀχυρὰν 
πρὸς ᾗ ἐπενόουν ἀγωνίσασθαι, Σάγγαλα ἦν τῇ πόλει 
ὄνομα, καὶ αὐτοὶ οἱ Καθαῖοι εὐτολμότατοί τε καὶ τὰ 
πολέμια κράτιστοι ἐνομίζοντο, καὶ τούτοις κατὰ τὰ   
αὐτὰ Ὀξυδράκαι, ἄλλο Ἰνδῶν ἔθνος, καὶ Μαλλοί, ἄλλο 
καὶ τοῦτο· ἐπεὶ καὶ ὀλίγῳ πρόσθεν στρατεύσαντας ἐπ' 




Flavius Arrianus Hist., Phil., Alexandri anabasis 
Book 5, chapter 22, section 2, line 8

εἰ προσάγοι τῇ χώρᾳ αὐτῶν Ἀλέξανδρος, καὶ ὅσα ὅμορά 
σφισιν <ἔθνη> ὡσαύτως αὐτόνομα, καὶ ταῦτα παρα-
καλεῖν ἐς τὸ ἔργον· εἶναι δὲ τήν τε πόλιν ὀχυρὰν 
πρὸς ᾗ ἐπενόουν ἀγωνίσασθαι, Σάγγαλα ἦν τῇ πόλει 
ὄνομα, καὶ αὐτοὶ οἱ Καθαῖοι εὐτολμότατοί τε καὶ τὰ 
πολέμια κράτιστοι ἐνομίζοντο, καὶ τούτοις κατὰ τὰ   
αὐτὰ Ὀξυδράκαι, ἄλλο Ἰνδῶν ἔθνος, καὶ Μαλλοί, ἄλλο 
καὶ τοῦτο· ἐπεὶ καὶ ὀλίγῳ πρόσθεν στρατεύσαντας ἐπ' 
αὐτοὺς Πῶρόν τε καὶ Ἀνισάρην ξύν τε τῇ σφετέρᾳ 
δυνάμει καὶ πολλὰ ἄλλα ἔθνη τῶν αὐτονόμων Ἰνδῶν 
ἀναστήσαντας οὐδὲν πράξαντας τῆς παρασκευῆς ἄξιον 
ξυνέβη ἀπελθεῖν. 



Flavius Arrianus Hist., Phil., Alexandri anabasis 
Book 5, chapter 22, section 3, line 4

                            καὶ δευτεραῖος μὲν ἀπὸ τοῦ 
ποταμοῦ τοῦ Ὑδραώτου πρὸς πόλιν ἧκεν ᾗ ὄνομα 
Πίμπραμα· τὸ δὲ ἔθνος τοῦτο τῶν Ἰνδῶν Ἀδραϊσταὶ 
ἐκαλοῦντο. 



Flavius Arrianus Hist., Phil., Alexandri anabasis 
Book 5, chapter 22, section 5, line 6

          Ἀλέξανδρος δὲ τό τε πλῆθος κατιδὼν τῶν 
βαρβάρων καὶ τοῦ χωρίου τὴν φύσιν, ὡς μάλιστα πρὸς 
τὰ παρόντα ἐν καιρῷ οἱ ἐφαίνετο παρετάσσετο· καὶ 
τοὺς μὲν ἱπποτοξότας εὐθὺς ὡς εἶχεν ἐκπέμπει ἐπ' 
αὐτούς, ἀκροβολίζεσθαι κελεύσας παριππεύοντας, ὡς 
μήτε ἐκδρομήν τινα ποιήσασθαι τοὺς Ἰνδοὺς πρὶν 
ξυνταχθῆναι αὐτῷ τὴν στρατιὰν καὶ ὡς πληγὰς γίγνε-
σθαι αὐτοῖς καὶ πρὸ τῆς μάχης ἐντὸς τοῦ ὀχυρώματος. 



Flavius Arrianus Hist., Phil., Alexandri anabasis 
Book 5, chapter 22, section 7, line 8

                         καὶ τούτων τοὺς μὲν ἱππέας ἐπὶ 
τὰ κέρατα διελὼν παρήγαγεν, ἀπὸ δὲ τῶν πεζῶν τῶν 
προσγενομένων πυκνοτέραν τὴν ξύγκλεισιν τῆς φά-
λαγγος ποιήσας αὐτὸς ἀναλαβὼν τὴν ἵππον τὴν ἐπὶ 
τοῦ δεξιοῦ τεταγμένην παρήγαγεν ἐπὶ τὰς κατὰ τὸ 
εὐώνυμον τῶν Ἰνδῶν ἁμάξας. 



Flavius Arrianus Hist., Phil., Alexandri anabasis 
Book 5, chapter 23, section 1, line 2

Ὡς δὲ ἐπὶ τὴν ἵππον προσαγαγοῦσαν οὐκ ἐξέδραμον 
οἱ Ἰνδοὶ ἔξω τῶν ἁμαξῶν, ἀλλ' ἐπιβεβηκότες αὐτῶν 
ἀφ' ὑψηλοῦ ἠκροβολίζοντο, γνοὺς Ἀλέξανδρος ὅτι οὐκ 
εἴη τῶν ἱππέων τὸ ἔργον καταπηδήσας ἀπὸ τοῦ ἵππου 
πεζὸς ἐπῆγε τῶν πεζῶν τὴν φάλαγγα. 



Flavius Arrianus Hist., Phil., Alexandri anabasis 
Book 5, chapter 23, section 2, line 3

                                          καὶ ἀπὸ μὲν 
τῶν πρώτων ἁμαξῶν οὐ χαλεπῶς ἐβιάσαντο οἱ Μακε-
δόνες τοὺς Ἰνδούς· πρὸ δὲ τῶν δευτέρων οἱ Ἰνδοὶ 
παραταξάμενοι ῥᾷον ἀπεμάχοντο, οἷα δὴ πυκνότεροί 
τε ἐφεστηκότες ἐλάττονι τῷ κύκλῳ καὶ τῶν Μακεδόνων 
οὐ κατ' εὐρυχωρίαν ὡσαύτως προσαγόντων σφίσιν, ἐν 
ᾧ τάς τε πρώτας ἁμάξας ὑπεξῆγον καὶ κατὰ τὰ δια-
λείμματα αὐτῶν ὡς ἑκάστοις προὐχώρει ἀτάκτως   
προσέβαλλον· ἀλλὰ καὶ ἀπὸ τούτων ὅμως ἐξώσθησαν 
οἱ Ἰνδοὶ βιασθέντες πρὸς τῆς φάλαγγος. 



Flavius Arrianus Hist., Phil., Alexandri anabasis 
Book 5, chapter 23, section 3, line 1

τῶν πρώτων ἁμαξῶν οὐ χαλεπῶς ἐβιάσαντο οἱ Μακε-
δόνες τοὺς Ἰνδούς· πρὸ δὲ τῶν δευτέρων οἱ Ἰνδοὶ 
παραταξάμενοι ῥᾷον ἀπεμάχοντο, οἷα δὴ πυκνότεροί 
τε ἐφεστηκότες ἐλάττονι τῷ κύκλῳ καὶ τῶν Μακεδόνων 
οὐ κατ' εὐρυχωρίαν ὡσαύτως προσαγόντων σφίσιν, ἐν 
ᾧ τάς τε πρώτας ἁμάξας ὑπεξῆγον καὶ κατὰ τὰ δια-
λείμματα αὐτῶν ὡς ἑκάστοις προὐχώρει ἀτάκτως   
προσέβαλλον· ἀλλὰ καὶ ἀπὸ τούτων ὅμως ἐξώσθησαν 
οἱ Ἰνδοὶ βιασθέντες πρὸς τῆς φάλαγγος. 



Flavius Arrianus Hist., Phil., Alexandri anabasis 
Book 5, chapter 23, section 4, line 5

                                  καὶ Ἀλέξανδρος ταύτην 
μὲν τὴν ἡμέραν περιεστρατοπέδευσε τοῖς πεζοῖς τὴν 
πόλιν ὅσα γε ἠδυνήθη αὐτῷ περιβαλεῖν ἡ φάλαγξ· 
ἐπὶ πολὺ γὰρ ἐπέχον τὸ τεῖχος τῷ στρατοπέδῳ κυκλώ-
σασθαι οὐ δυνατὸς ἐγένετο· κατὰ δὲ τὰ διαλείποντα 
αὐτοῦ, ἵνα καὶ λίμνη οὐ μακρὰν τοῦ τείχους ἦν, τοὺς 
ἱππέας ἐπέταξεν ἐν κύκλῳ τῆς λίμνης, γνοὺς οὐ βαθεῖαν 
οὖσαν τὴν λίμνην καὶ ἅμα εἰκάσας ὅτι φοβεροὶ γενό-
μενοι οἱ Ἰνδοὶ ἀπὸ τῆς προτέρας ἥττης ἀπολείψουσι 
τῆς νυκτὸς τὴν πόλιν. 



Flavius Arrianus Hist., Phil., Alexandri anabasis 
Book 5, chapter 23, section 6, line 7

αὐτομολήσαντες δὲ αὐτῷ τῶν ἐκ τῆς πόλεώς τινες 
φράζουσιν, ὅτι ἐν νῷ ἔχοιεν αὐτῆς ἐκείνης τῆς νυκτὸς 
ἐκπίπτειν ἐκ τῆς πόλεως οἱ Ἰνδοὶ κατὰ τὴν λίμνην, 
ἵναπερ τὸ ἐκλιπὲς ἦν τοῦ χάρακος. 



Flavius Arrianus Hist., Phil., Alexandri anabasis 
Book 5, chapter 24, section 4, line 2

Ἐν τούτῳ δὲ καὶ Πῶρος ἀφίκετο τούς τε ὑπολοί-
πους ἐλέφαντας ἅμα οἷ ἄγων καὶ τῶν Ἰνδῶν ἐς πεντα-
κισχιλίους, αἵ τε μηχαναὶ Ἀλεξάνδρῳ ξυμπεπηγμέναι 
ἦσαν καὶ προσήγοντο ἤδη τῷ τείχει. 



Flavius Arrianus Hist., Phil., Alexandri anabasis 
Book 5, chapter 24, section 5, line 2

                     καὶ ἀποθνήσκουσι μὲν ἐν τῇ 
καταλήψει τῶν Ἰνδῶν ἐς μυρίους καὶ ἑπτακισχιλίους, 
ἑάλωσαν δὲ ὑπὲρ τὰς ἑπτὰ μυριάδας καὶ ἅρματα τρια-
κόσια καὶ ἵπποι πεντακόσιοι. 



Flavius Arrianus Hist., Phil., Alexandri anabasis 
Book 5, chapter 24, section 6, line 8

Θάψας δὲ ὡς νόμος αὐτῷ τοὺς τελευτήσαντας 
Εὐμενῆ τὸν γραμματέα ἐκπέμπει ἐς τὰς δύο πόλεις 
τὰς ξυναφεστώσας τοῖς Σαγγάλοις δοὺς αὐτῷ τῶν 
ἱππέων ἐς τριακοσίους, φράσοντα[ς] τοῖς ἔχουσι τὰς 
πόλεις τῶν τε Σαγγάλων τὴν ἅλωσιν καὶ ὅτι αὐτοῖς 
οὐδὲν ἔσται χαλεπὸν <ἐξ> Ἀλεξάνδρου ὑπομένουσί τε   
καὶ δεχομένοις φιλίως Ἀλέξανδρον· οὐδὲ γὰρ οὐδὲ 
ἄλλοις τισὶ γενέσθαι τῶν αὐτονόμων Ἰνδῶν ὅσοι 
ἑκόντες σφᾶς ἐνέδοσαν. 



Flavius Arrianus Hist., Phil., Alexandri anabasis 
Book 5, chapter 24, section 8, line 4

                                       ὡς δὲ ἀπέγνω 
διώκειν τοῦ πρόσω τοὺς φεύγοντας, ἐπανελθὼν ἐς τὰ 
Σάγγαλα τὴν πόλιν μὲν κατέσκαψε, τὴν χώραν δὲ 
τῶν Ἰνδῶν τοῖς πάλαι μὲν αὐτονόμοις, τότε δὲ ἑκουσίως 
προσχωρήσασι προσέθηκεν. 



Flavius Arrianus Hist., Phil., Alexandri anabasis 
Book 5, chapter 24, section 8, line 9

                           καὶ Πῶρον μὲν ξὺν τῇ 
δυνάμει τῇ ἀμφ' αὐτὸν ἐκπέμπει ἐπὶ τὰς πόλεις αἳ 
προσκεχωρήκεσαν, φρουρὰς εἰσάξοντα εἰς αὐτάς, αὐτὸς 
δὲ ξὺν τῇ στρατιᾷ ἐπὶ τὸν Ὕφασιν ποταμὸν προὐχώρει, 
ὡς καὶ τοὺς ἐπέκεινα Ἰνδοὺς καταστρέψαιτο. 



Flavius Arrianus Hist., Phil., Alexandri anabasis 
Book 5, chapter 25, section 2, line 1

Τὰ δὲ δὴ πέραν τοῦ Ὑφάσιος εὐδαίμονά τε τὴν 
χώραν εἶναι ἐξηγγέλλετο καὶ ἀνθρώπους ἀγαθοὺς μὲν 
γῆς ἐργάτας, γενναίους δὲ τὰ πολέμια καὶ ἐς τὰ ἴδια 
δὲ σφῶν ἐν κόσμῳ πολιτεύοντας (πρὸς γὰρ τῶν 
ἀρίστων ἄρχεσθαι τοὺς πολλούς, τοὺς δὲ οὐδὲν ἔξω   
τοῦ ἐπιεικοῦς ἐξηγεῖσθαι), πλῆθός τε ἐλεφάντων εἶναι 
τοῖς ταύτῃ ἀνθρώποις πολύ τι ὑπὲρ τοὺς ἄλλους 
Ἰνδοὺς, καὶ μεγέθει μεγίστους καὶ ἀνδρείᾳ. 



Flavius Arrianus Hist., Phil., Alexandri anabasis 
Book 5, chapter 25, section 5, line 7

καὶ Αἴγυπτος ξὺν τῇ Λιβύῃ τῇ Ἑλληνικῇ καὶ Ἀραβίας 
ἔστιν ἃ καὶ Συρία ἥ τε κοίλη καὶ ἡ μέση τῶν ποταμῶν, 
καὶ Βαβυλὼν δὲ ἔχεται καὶ τὸ Σουσίων ἔθνος καὶ 
Πέρσαι καὶ Μῆδοι καὶ ὅσων Πέρσαι καὶ Μῆδοι   
ἐπῆρχον, καὶ ὅσων δὲ οὐκ ἦρχον, τὰ ὑπὲρ τὰς Κασπίας 
πύλας, τὰ ἐπέκεινα τοῦ Καυκάσου, ὁ Τάναϊς, τὰ πρόσω 
ἔτι τοῦ Τανάϊδος, Βακτριανοί, Ὑρκάνιοι, ἡ θάλασσα 
ἡ Ὑρκανία, Σκύθας τε ἀνεστείλαμεν ἔστε ἐπὶ τὴν 
ἔρημον, ἐπὶ τούτοις μέντοι καὶ ὁ Ἰνδὸς ποταμὸς διὰ 
τῆς ἡμετέρας ῥεῖ, ὁ Ὑδάσπης διὰ τῆς ἡμετέρας, ὁ 
Ἀκεσίνης, ὁ Ὑδραώτης, τί ὀκνεῖτε καὶ τὸν Ὕφασιν καὶ 
τὰ ἐπέκεινα τοῦ Ὑφάσιος γένη προσθεῖναι τῇ ἡμετέρᾳ 
Μακεδόνων τε ἀρχῇ; 



Flavius Arrianus Hist., Phil., Alexandri anabasis 
Book 5, chapter 26, section 2, line 3

                                                     καὶ 
ἐγὼ ἐπιδείξω Μακεδόσι τε καὶ τοῖς ξυμμάχοις τὸν μὲν 
Ἰνδικὸν κόλπον ξύρρουν ὄντα τῷ Περσικῷ, τὴν δὲ 
Ὑρκανίαν <θάλασσαν> τῷ Ἰνδικῷ· ἀπὸ δὲ τοῦ Περ-
σικοῦ εἰς Λιβύην περιπλευσθήσεται στόλῳ ἡμετέρῳ 
τὰ μέχρι Ἡρακλέους Στηλῶν· ἀπὸ δὲ Στηλῶν ἡ ἐντὸς 
Λιβύη πᾶσα ἡμετέρα γίγνεται καὶ ἡ Ἀσία δὴ οὕτω 
πᾶσα, καὶ ὅροι τῆς ταύτῃ ἀρχῆς οὕσπερ καὶ τῆς γῆς 
ὅρους ὁ θεὸς ἐποίησε. 



Flavius Arrianus Hist., Phil., Alexandri anabasis 
Book 5, chapter 27, section 7, line 9

                                          σὺ δὲ νῦν μὴ 
ἄγειν ἄκοντας· οὐδὲ γὰρ ὁμοίοις ἔτι χρήσῃ ἐς τοὺς 
κινδύνους, οἷς τὸ ἑκούσιον ἐν τοῖς ἀγῶσιν ἀπέσται· 
ἐπανελθὼν δὲ αὐτός [τε], εἰ δοκεῖ, ἐς τὴν οἰκ<ε>ίαν 
καὶ τὴν μητέρα τὴν σαυτοῦ ἰδὼν καὶ τὰ τῶν Ἑλλήνων 
καταστησάμενος καὶ τὰς νίκας ταύτας τὰς πολλὰς καὶ 
μεγάλας ἐς τὸν πατρῷον οἶκον κομίσας οὕτω δὴ ἐξ 
ἀρχῆς ἄλλον στόλον στέλλεσθαι, εἰ μὲν βούλει, ἐπ' 
αὐτὰ ταῦτα τὰ πρὸς τὴν ἕω ᾠκισμένα Ἰνδῶν γένη, 
εἰ δὲ βούλει, ἐς τὸν Εὔξεινον πόντον, εἰ δέ, ἐπὶ 
Καρχηδόνα καὶ τὰ ἐπέκεινα Καρχηδονίων τῆς Λιβύης. 



Flavius Arrianus Hist., Phil., Alexandri anabasis 
Book 5, chapter 29, section 4, line 4

Ἐν τούτῳ δὲ ἀφίκοντο πρὸς αὐτὸν Ἀρσάκης τε ὁ 
τῆς ὁμόρου Ἀβισάρῃ χώρας ὕπαρχος καὶ ὁ ἀδελφὸς 
Ἀβισάρου καὶ οἱ ἄλλοι οἰκεῖοι, δῶρά τε κομίζοντες ἃ 
μέγιστα παρ' Ἰνδοῖς καὶ τοὺς παρ' Ἀβισάρου ἐλέφαντας,   
ἀριθμὸν ἐς τριάκοντα· Ἀβισάρην γὰρ νόσῳ ἀδύνατον 
γενέσθαι ἐλθεῖν. 



Flavius Arrianus Hist., Phil., Alexandri anabasis 
Book 6, chapter 1, section 2, line 1

     πρότερον μέν γε ἐν τῷ Ἰνδῷ ποταμῷ κροκοδείλους 
ἰδών, μόνῳ τῶν ἄλλων ποταμῶν πλὴν Νείλου, πρὸς 
δὲ ταῖς ὄχθαις τοῦ Ἀκεσίνου κυάμους πεφυκότας 
ὁποίους ἡ γῆ ἐκφέρει ἡ Αἰγυπτία, καὶ [ὁ] ἀκούσας 
ὅτι ὁ Ἀκεσίνης ἐμβάλλει ἐς τὸν Ἰνδόν, ἔδοξεν ἐξευρη-
κέναι τοῦ Νείλου τὰς ἀρχάς, ὡς τὸν Νεῖλον ἐνθένδε 
ποθὲν ἐξ Ἰνδῶν ἀνίσχοντα καὶ δι' ἐρήμου πολλῆς γῆς 
ῥέοντα καὶ ταύτῃ ἀπολλύοντα τὸν Ἰνδὸν τὸ ὄνομα, 
ἔπειτα, ὁπόθεν ἄρχεται διὰ τῆς οἰκουμένης χώρας ῥεῖν, 
Νεῖλον ἤδη πρὸς Αἰθιόπων τε τῶν ταύτῃ καὶ

Αἰγυ-



Flavius Arrianus Hist., Phil., Alexandri anabasis 
Book 6, chapter 1, section 4, line 2

                   καὶ δὴ καὶ πρὸς Ὀλυμπιάδα γράφοντα   
ὑπὲρ τῶν Ἰνδῶν τῆς γῆς ἄλλα τε γράψαι καὶ ὅτι 
δοκοίη αὑτῷ ἐξευρηκέναι τοῦ Νείλου τὰς πηγάς, μικροῖς 
δή τισι καὶ φαύλοις ὑπὲρ τῶν τηλικούτων τεκμαιρό-
μενον. 



Flavius Arrianus Hist., Phil., Alexandri anabasis 
Book 6, chapter 1, section 5, line 2

       ἐπεὶ μέντοι ἀτρεκέστερον ἐξήλεγξε τὰ ἀμφὶ 
τῷ ποταμῷ τῷ Ἰνδῷ, οὕτω δὴ μαθεῖν παρὰ τῶν ἐπι-
χωρίων τὸν μὲν Ὑδάσπην τῷ Ἀκεσίνῃ, τὸν Ἀκεσίνην 
δὲ τῷ Ἰνδῷ τό τε ὕδωρ ξυμβάλλοντας καὶ τῷ ὀνόματι 
ξυγχωροῦντας, τὸν Ἰνδὸν δὲ ἐκδιδόντα ἤδη ἐς τὴν 
μεγάλην θάλασσαν, δίστομον τὸν Ἰνδὸν ὄντα, οὐδέ<ν> τι 
αὐτῷ προσῆκον τῆς γῆς τῆς Αἰγυπτίας· τηνικαῦτα δὲ 
τῆς ἐπιστολῆς τῆς πρὸς τὴν μητέρα τοῦτο <τὸ> ἀμφὶ τῷ 
Νείλῳ γραφὲν ἀφελεῖν. 



Flavius Arrianus Hist., Phil., Alexandri anabasis 
Book 6, chapter 2, section 1, line 4

                                          αὐτὸς δὲ 
ξυναγαγὼν τούς τε ἑταίρους καὶ ὅσοι Ἰνδῶν πρέσβεις 
παρ' αὐτὸν ἀφιγμένοι ἦσαν βασιλέα μὲν τῆς ἑαλωκυίας 
ἤδη Ἰνδῶν γῆς ἀπέδειξε Πῶρον, ἑπτὰ μὲν ἐθνῶν τῶν 
ξυμπάντων, πόλεων δὲ ἐν τοῖς ἔθνεσιν ὑπὲρ τὰς 
δισχιλίας. 



Flavius Arrianus Hist., Phil., Alexandri anabasis 
Book 6, chapter 2, section 3, line 2

                                           Φιλίππῳ δὲ τῷ 
σατράπῃ τῆς ἐπέκεινα τοῦ Ἰνδοῦ ὡς ἐπὶ Βακτρίους 
γῆς διαλιπόντι τρεῖς ἡμέρας παρήγγελτο ἕπεσθαι ξὺν 
τοῖς ἀμφ' αὐτόν. 



Flavius Arrianus Hist., Phil., Alexandri anabasis 
Book 6, chapter 3, section 1, line 9

                          καὶ ἐπιβὰς τῆς νεὼς ἀπὸ τῆς 
πρώρας ἐκ χρυσῆς φιάλης ἔσπενδεν ἐς τὸν ποταμόν, 
τόν τε Ἀκεσίνην ξυνεπικαλούμενος τῷ Ὑδάσπῃ, ὅντινα 
μέγιστον αὖ τῶν ἄλλων ποταμῶν ξυμβάλλειν τῷ 
Ὑδάσπῃ ἐπέπυστο καὶ οὐ πόρρω αὐτῶν εἶναι τὰς ξυμ-
βολάς, καὶ τὸν Ἰνδόν, ἐς ὅντινα ὁ Ἀκεσίνης ξὺν τῷ 
Ὑδάσπῃ ἐμβάλλει. 



Flavius Arrianus Hist., Phil., Alexandri anabasis 
Book 6, chapter 3, section 4, line 3

ἐνδιδόντων τὰς ἀρχάς τε καὶ ἀναπαύλας τῇ εἰρεσίᾳ 
καὶ τῶν ἐρετῶν ὁπότε ἀθρόοι ἐμπίπτοντες τῷ ῥοθίῳ 
ἐπαλαλάξειαν· αἵ τε ὄχθαι, ὑψηλότεραι τῶν νεῶν 
πολλαχῇ οὖσαι, ἐς στενόν τε τὴν βοὴν ξυνάγουσαι 
καὶ τῇ ξυναγωγῇ αὐτῇ ἐπὶ μέγα ηὐξημένην ἐς ἀλλήλας 
ἀντέπεμπον, καί που καὶ νάπαι ἑκατέρωθεν τοῦ 
ποταμοῦ τῇ τε ἐρημίᾳ καὶ τῇ ἀντιπέμψει τοῦ κτύπου 
καὶ αὗται ξυνεπελάμβανον· οἵ τε ἵπποι διαφαινόμενοι 
διὰ τῶν ἱππαγωγῶν πλοίων, οὐ πρόσθεν ἵπποι ἐπὶ 
νεῶν ὀφθέντες ἐν τῇ Ἰνδῶν γῇ (καὶ γὰρ καὶ τὸν 
Διονύσου ἐπ' Ἰνδοὺς στόλον οὐκ ἐμέμνηντο γενέσθαι 
ναυτικόν), ἔκπληξιν παρεῖχον τοῖς θεωμένοις τῶν   
βαρβάρων, ὥστε οἱ μὲν αὐτόθεν τῇ ἀναγωγῇ παρα-
γενόμενοι ἐπὶ πολὺ ἐφωμάρτουν, ἐς ὅσους δὲ τῶν 
ἤδη Ἀλεξάνδρῳ προσκεχωρηκότων Ἰνδῶν ἡ βοὴ τῶν 
ἐρετῶν ἢ ὁ κτύπος τῆς εἰρεσίας ἐξίκετο, καὶ οὗτοι 
ἐπὶ τῇ ὄχθῃ κατέθεον καὶ ξυνείποντο ἐπᾴδοντες βαρβα-
ρικῶς. 



Flavius Arrianus Hist., Phil., Alexandri anabasis 
Book 6, chapter 3, section 4, line 4

καὶ τῶν ἐρετῶν ὁπότε ἀθρόοι ἐμπίπτοντες τῷ ῥοθίῳ 
ἐπαλαλάξειαν· αἵ τε ὄχθαι, ὑψηλότεραι τῶν νεῶν 
πολλαχῇ οὖσαι, ἐς στενόν τε τὴν βοὴν ξυνάγουσαι 
καὶ τῇ ξυναγωγῇ αὐτῇ ἐπὶ μέγα ηὐξημένην ἐς ἀλλήλας 
ἀντέπεμπον, καί που καὶ νάπαι ἑκατέρωθεν τοῦ 
ποταμοῦ τῇ τε ἐρημίᾳ καὶ τῇ ἀντιπέμψει τοῦ κτύπου 
καὶ αὗται ξυνεπελάμβανον· οἵ τε ἵπποι διαφαινόμενοι 
διὰ τῶν ἱππαγωγῶν πλοίων, οὐ πρόσθεν ἵπποι ἐπὶ 
νεῶν ὀφθέντες ἐν τῇ Ἰνδῶν γῇ (καὶ γὰρ καὶ τὸν 
Διονύσου ἐπ' Ἰνδοὺς στόλον οὐκ ἐμέμνηντο γενέσθαι 
ναυτικόν), ἔκπληξιν παρεῖχον τοῖς θεωμένοις τῶν   
βαρβάρων, ὥστε οἱ μὲν αὐτόθεν τῇ ἀναγωγῇ παρα-
γενόμενοι ἐπὶ πολὺ ἐφωμάρτουν, ἐς ὅσους δὲ τῶν 
ἤδη Ἀλεξάνδρῳ προσκεχωρηκότων Ἰνδῶν ἡ βοὴ τῶν 
ἐρετῶν ἢ ὁ κτύπος τῆς εἰρεσίας ἐξίκετο, καὶ οὗτοι 
ἐπὶ τῇ ὄχθῃ κατέθεον καὶ ξυνείποντο ἐπᾴδοντες βαρβα-
ρικῶς. 



Flavius Arrianus Hist., Phil., Alexandri anabasis 
Book 6, chapter 3, section 5, line 2

ἀντέπεμπον, καί που καὶ νάπαι ἑκατέρωθεν τοῦ 
ποταμοῦ τῇ τε ἐρημίᾳ καὶ τῇ ἀντιπέμψει τοῦ κτύπου 
καὶ αὗται ξυνεπελάμβανον· οἵ τε ἵπποι διαφαινόμενοι 
διὰ τῶν ἱππαγωγῶν πλοίων, οὐ πρόσθεν ἵπποι ἐπὶ 
νεῶν ὀφθέντες ἐν τῇ Ἰνδῶν γῇ (καὶ γὰρ καὶ τὸν 
Διονύσου ἐπ' Ἰνδοὺς στόλον οὐκ ἐμέμνηντο γενέσθαι 
ναυτικόν), ἔκπληξιν παρεῖχον τοῖς θεωμένοις τῶν   
βαρβάρων, ὥστε οἱ μὲν αὐτόθεν τῇ ἀναγωγῇ παρα-
γενόμενοι ἐπὶ πολὺ ἐφωμάρτουν, ἐς ὅσους δὲ τῶν 
ἤδη Ἀλεξάνδρῳ προσκεχωρηκότων Ἰνδῶν ἡ βοὴ τῶν 
ἐρετῶν ἢ ὁ κτύπος τῆς εἰρεσίας ἐξίκετο, καὶ οὗτοι 
ἐπὶ τῇ ὄχθῃ κατέθεον καὶ ξυνείποντο ἐπᾴδοντες βαρβα-
ρικῶς. 



Flavius Arrianus Hist., Phil., Alexandri anabasis 
Book 6, chapter 3, section 5, line 5

        φιλῳδοὶ γάρ, εἴπερ τινὲς ἄλλοι, Ἰνδοὶ καὶ 
φιλορχήμονες ἀπὸ Διονύσου ἔτι καὶ τῶν ἅμα Διονύσῳ 
βακχευσάντων κατὰ τὴν Ἰνδῶν γῆν. 



Flavius Arrianus Hist., Phil., Alexandri anabasis 
Book 6, chapter 4, section 2, line 5

                                                  προσορμιζό-
μενος δὲ ὅπῃ τύχοι ταῖς ὄχθαις τοὺς προσοικοῦντας 
τῷ Ὑδάσπῃ Ἰνδοὺς τοὺς μὲν ἐνδιδόντας σφᾶς ὁμο-
λογίαις παρελάμβανεν, ἤδη δέ τινας καὶ ἐς ἀλκὴν 
χωρήσαντας βίᾳ κατεστρέψατο. 



Flavius Arrianus Hist., Phil., Alexandri anabasis 
Book 6, chapter 4, section 3, line 3

                                 αὐτὸς δὲ ὡς ἐπὶ τὴν 
Μαλλῶν τε καὶ Ὀξυδρακῶν γῆν σπουδῇ ἔπλει, πλεί-
στους τε καὶ μαχιμωτάτους τῶν ταύτῃ Ἰνδῶν πυνθανό-
μενος καὶ ὅτι ἐξηγγέλλοντο αὐτῷ παῖδας μὲν καὶ 
γυναῖκας ἀποτεθεῖσθαι εἰς τὰς ὀχυρωτάτας τῶν πόλεων,   
αὐτοὶ δὲ ἐγνωκέναι διὰ μάχης ἰέναι πρὸς αὐτόν. 



Flavius Arrianus Hist., Phil., Alexandri anabasis 
Book 6, chapter 6, section 1, line 5

Αὐτὸς δὲ ἀναλαβὼν τοὺς ὑπασπιστάς τε καὶ τοὺς 
τοξότας καὶ τοὺς Ἀγριᾶνας καὶ τῶν πεζεταίρων καλου-
μένων τὴν Πείθωνος τάξιν καὶ τοὺς ἱπποτοξότας τε 
πάντας καὶ τῶν ἱππέων τῶν ἑταίρων τοὺς ἡμίσεας 
διὰ γῆς ἀνύδρου ὡς ἐπὶ Μαλλοὺς ἦγεν, ἔθνος Ἰνδικὸν 
Ἰνδῶν τῶν αὐτονόμων. 



Flavius Arrianus Hist., Phil., Alexandri anabasis 
Book 6, chapter 6, section 4, line 5

                         ὡς δὲ τάχιστα οἱ πεζοὶ ἀφίκοντο, 
Περδίκκαν μὲν τήν τε αὑτοῦ ἱππαρχίαν ἔχοντα καὶ 
τὴν Κλείτου καὶ τοὺς Ἀγριᾶνας πρὸς ἄλλην πόλιν 
ἐκπέμπει τῶν Μαλλῶν, οἷ ξυμπεφευγότες ἦσαν πολλοὶ 
τῶν ταύτῃ Ἰνδῶν, φυλάσσειν τοὺς ἐν τῇ πόλει κελεύ-
σας, ἔργου δὲ μὴ ἔχεσθαι ἔστ' ἂν ἀφίκηται αὐτός, ὡς 
μηδὲ ἀπὸ ταύτης τῆς πόλεως διαφυγόντας τινὰς 
αὐτῶν ἀγγέλους γενέσθαι τοῖς ἄλλοις βαρβάροις ὅτι 
προσάγει ἤδη Ἀλέξανδρος· αὐτὸς δὲ προσέβαλλεν τῷ 
τείχει. 



Flavius Arrianus Hist., Phil., Alexandri anabasis 
Book 6, chapter 7, section 6, line 3

                                           εἴχετό τε ἤδη ἡ ἄκρα, 
καὶ τῶν Ἰνδῶν οἱ μὲν τὰς οἰκίας ἐνεπίμπρασαν καὶ ἐν 
αὐταῖς ἐγκαταλαμβανόμενοι ἀπέθνησκον, οἱ πολλοὶ δὲ 
μαχόμενοι αὐτῶν. 



Flavius Arrianus Hist., Phil., Alexandri anabasis 
Book 6, chapter 8, section 4, line 4

                                   ἀλλὰ καὶ ταύτην ἐξέλι-
πον οἱ Ἰνδοὶ ὡς προσάγοντα Ἀλέξανδρον ἔμαθον. 



Flavius Arrianus Hist., Phil., Alexandri anabasis 
Book 6, chapter 8, section 6, line 5

                       ὡς δὲ κατεῖδον ἱππέας μόνους, ἐπι-
στρέψαντες οἱ Ἰνδοὶ καρτερῶς ἐμάχοντο πλῆθος ὄντες 
ἐς πέντε μυριάδας. 



Flavius Arrianus Hist., Phil., Alexandri anabasis 
Book 6, chapter 8, section 7, line 1

                      καὶ Ἀλέξανδρος ὡς τήν τε φάλαγγα   
αὐτῶν πυκνὴν κατεῖδε καὶ αὐτῷ οἱ πεζοὶ ἀπῆσαν, 
προσβολὰς μὲν ἐποιεῖτο ἐς κύκλους παριππεύων, ἐς 
χεῖρας δὲ οὐκ ᾔει τοῖς Ἰνδοῖς. 



Flavius Arrianus Hist., Phil., Alexandri anabasis 
Book 6, chapter 8, section 7, line 5

                                                         καὶ 
οἱ Ἰνδοὶ ὁμοῦ σφισι πάντων τῶν δεινῶν προσκειμένων 
ἀποστρέψαντες ἤδη προτροπάδην ἔφευγον ἐς πόλιν 
ὀχυρωτάτην τῶν πλησίον. 



Flavius Arrianus Hist., Phil., Alexandri anabasis 
Book 6, chapter 9, section 1, line 4

                               καὶ ἐν τούτῳ οὐ δεξά-
μενοι οἱ Ἰνδοὶ τῶν Μακεδόνων τὴν ὁρμὴν τὰ μὲν 
τείχη τῆς πόλεως λείπουσιν, αὐτοὶ δὲ ἐς τὴν ἄκραν 
ξυνέφευγον. 



Flavius Arrianus Hist., Phil., Alexandri anabasis 
Book 6, chapter 9, section 4, line 3

                    ἤδη τε πρὸς τῇ ἐπάλξει τοῦ 
τείχους ὁ βασιλεὺς ἦν καὶ ἐρείσας ἐπ' αὐτῇ τὴν ἀσπίδα 
τοὺς μὲν ὤθει εἴσω τοῦ τείχους τῶν Ἰνδῶν, τοὺς δὲ 
καὶ αὐτοῦ τῷ ξίφει ἀποκτείνας γεγυμνώκει τὸ ταύτῃ 
τεῖχος· καὶ οἱ ὑπασπισταὶ ὑπέρφοβοι γενόμενοι ὑπὲρ 
τοῦ βασιλέως σπουδῇ ὠθούμενοι κατὰ τὴν αὐτὴν κλί-
μακα συντρίβουσιν αὐτήν, ὥστε οἱ μὲν ἤδη ἀνιόντες 
αὐτῶν κάτω ἔπεσον, τοῖς δὲ ἄλλοις ἄπορον ἐποίησαν 
τὴν ἄνοδον. 



Flavius Arrianus Hist., Phil., Alexandri anabasis 
Book 6, chapter 9, section 5, line 3

Ἀλέξανδρος δὲ ὡς ἐπὶ τοῦ τείχους στὰς κύκλῳ τε 
ἀπὸ τῶν πλησίον πύργων ἐβάλλετο, οὐ γὰρ πελάσαι 
γε ἐτόλμα τις αὐτῷ τῶν Ἰνδῶν, καὶ ὑπὸ τῶν ἐκ τῆς 
πόλεως, οὐδὲ πόρρω τούτων γε ἐσακοντιζόντων (ἔτυχε   
γάρ τι καὶ προσκεχωσμένον ταύτῃ πρὸς τὸ τεῖχος), 
δῆλος μὲν ἦν Ἀλέξανδρος ὢν τῶν τε ὅπλων τῇ λαμ-
πρότητι καὶ τῷ ἀτόπῳ τῆς τόλμης, ἔγνω δὲ ὅτι αὐτοῦ 
μὲν μένων κινδυνεύσει μηδὲν ὅ τι καὶ λόγου ἄξιον 
ἀποδεικνύμενος, καταπηδήσας δὲ εἴσω τοῦ τείχους 
τυχὸν μὲν αὐτῷ τούτῳ ἐκπλήξει τοὺς Ἰνδούς, εἰ δὲ 
μή, καὶ κινδυνεύειν δέοι, μεγάλα ἔργα καὶ τοῖς ἔπειτα 
πυθέσθαι ἄξια ἐργασάμενος οὐκ ἀσπουδεὶ

ἀποθανεῖ-



Flavius Arrianus Hist., Phil., Alexandri anabasis 
Book 6, chapter 9, section 6, line 3

         ἔνθα δὴ ἐρεισθεὶς πρὸς τῷ τείχει τοὺς μέν 
τινας ἐς χεῖρας ἐλθόντας καὶ τόν γε ἡγεμόνα τῶν 
Ἰνδῶν προσφερόμενόν οἱ θρασύτερον παίσας τῷ ξίφει 
ἀποκτείνει· ἄλλον δὲ πελάζοντα λίθῳ βαλὼν ἔσχε καὶ 
ἄλλον λίθῳ, τὸν δὲ ἐγγυτέρω προσάγοντα τῷ ξίφει 
αὖθις. 



Flavius Arrianus Hist., Phil., Alexandri anabasis 
Book 6, chapter 11, section 1, line 1

Ἐν τούτῳ δὲ οἱ μὲν ἔκτεινον τοὺς Ἰνδούς, καὶ 
ἀπέκτεινάν γε πάντας οὐδὲ γυναῖκα ἢ παῖδα ὑπ-
ελείποντο, οἱ δὲ ἐξέφερον τὸν βασιλέα ἐπὶ τῆς ἀσπί-
δος κακῶς ἔχοντα, οὔπω γιγνώσκοντες βιώσιμον ὄντα. 



Flavius Arrianus Hist., Phil., Alexandri anabasis 
Book 6, chapter 11, section 3, line 3

Αὐτίκα ἐν Ὀξυδράκαις τὸ πάθημα τοῦτο γενέσθαι 
Ἀλεξάνδρῳ ὁ πᾶς λόγος κατέχει· τὸ δὲ ἐν Μαλλοῖς 
ἔθνει αὐτονόμῳ Ἰνδικῷ ξυνέβη, καὶ ἥ τε πόλις Μαλλῶν 
ἦν καὶ οἱ βαλόντες Ἀλέξανδρον Μαλλοί, οἳ δὴ ἐγνώ-
κεσαν μὲν ξυμμίξαντες τοῖς Ὀξυδράκαις οὕτω δια-
γωνίζεσθαι, ἔφθη δὲ διὰ τῆς ἀνύδρου ἐπ' αὐτοὺς 
ἐλάσας πρίν τινα ὠφέλειαν αὐτοῖς παρὰ τῶν Ὀξυδρα-  
κῶν γενέσθαι ἢ αὐτοὺς ἐκείνοις τι ἐπωφελῆσαι. 



Flavius Arrianus Hist., Phil., Alexandri anabasis 
Book 6, chapter 13, section 3, line 10

           οἱ δὲ ἐπέλαζον ἄλλος ἄλλοθεν, οἱ μὲν χειρῶν, 
οἱ δὲ γονάτων, οἱ δὲ τῆς ἐσθῆτος αὐτῆς ἁπτόμενοι, 
οἱ δὲ καὶ ἰδεῖν ἐγγύθεν καί τι καὶ ἐπευφημήσαντες 
ἀπελθεῖν· οἱ δὲ ταινίαις ἔβαλλον, οἱ δὲ ἄνθεσιν, ὅσα 
ἐν τῷ τότε ἡ Ἰνδῶν γῆ παρεῖχε. 



Flavius Arrianus Hist., Phil., Alexandri anabasis 
Book 6, chapter 14, section 1, line 6

Ἐν τούτῳ δὲ ἀφίκοντο παρὰ Ἀλέξανδρον τῶν 
Μαλλῶν τῶν ὑπολειπομένων πρέσβεις ἐνδιδόντες τὸ 
ἔθνος, καὶ παρὰ Ὀξυδρακῶν οἵ τε ἡγεμόνες τῶν πόλεων 
καὶ οἱ νομάρχαι αὐτοὶ καὶ ἄλλοι ἅμα τούτοις ἑκατὸν 
καὶ πεντήκοντα οἱ γνωριμώτατοι αὐτοκράτορες περὶ 
σπονδῶν δῶρά τε ὅσα μέγιστα παρ' Ἰνδοῖς κομίζοντες 
καὶ τὸ ἔθνος καὶ οὗτοι ἐνδιδόντες. 



Flavius Arrianus Hist., Phil., Alexandri anabasis 
Book 6, chapter 14, section 2, line 5

                                            συγγνωστὰ δὲ 
ἁμαρτεῖν ἔφασαν οὐ πάλαι παρ' αὐτὸν πρεσβευσά-
μενοι· ἐπιθυμεῖν γάρ, ὥσπερ τινὲς ἄλλοι, ἔτι μᾶλλον 
αὐτοὶ ἐλευθερίας τε καὶ αὐτόνομοι εἶναι, ἥντινα ἐλευ-
θερίαν ἐξ ὅτου Διόνυσος ἐς Ἰνδοὺς ἧκε σώαν σφίσιν 
εἶναι ἐς Ἀλέξανδρον· εἰ δὲ Ἀλεξάνδρῳ δοκοῦν ἐστιν, 
ὅτι καὶ Ἀλέξανδρον ἀπὸ θεοῦ γενέσθαι λόγος κατέχει, 
σατράπην τε ἀναδέξεσθαι, ὅντινα τάττοι Ἀλέξανδρος, 
καὶ φόρους ἀποίσειν τοὺς Ἀλεξάνδρῳ δόξαντας· διδόναι 
δὲ καὶ ὁμήρους ἐθέλειν ὅσους ἂν αἰτῇ Ἀλέξανδρος. 



Flavius Arrianus Hist., Phil., Alexandri anabasis 
Book 6, chapter 14, section 3, line 4

ὁ δὲ χιλίους ᾔτησε τοὺς κρατιστεύοντας τοῦ ἔθνους, 
οὕς, εἰ μὲν βούλοιτο, ἀντὶ ὁμήρων καθέξειν, εἰ δὲ 
μή, ξυστρατεύοντας ἕξειν ἔστ' ἂν διαπολεμηθῇ αὐτῷ 
πρὸς τοὺς ἄλλους Ἰνδούς. 



Flavius Arrianus Hist., Phil., Alexandri anabasis 
Book 6, chapter 14, section 5, line 1

Ὡς δὲ ταῦτα αὐτῷ κεκόσμητο καὶ πλοῖα ἐπὶ τῇ 
διατριβῇ τῇ ἐκ τοῦ τραύματος πολλὰ προσενεναυπή-
γητο, ἀναβιβάσας ἐς τὰς ναῦς τῶν μὲν ἑταίρων ἱππέας 
ἑπτακοσίους καὶ χιλίους, τῶν ψιλῶν δὲ ὅσουσπερ καὶ 
πρότερον, πεζοὺς δὲ ἐς μυρίους, ὀλίγον μέν τι τῷ 
Ὑδραώτῃ ποταμῷ κατέπλευσεν, ὡς δὲ συνέμιξεν ὁ 
Ὑδραώτης τῷ Ἀκεσίνῃ, ὅτι ὁ Ἀκεσίνης κρατεῖ τοῦ 
Ὑδραώτου [ἐν] τῇ ἐπωνυμίᾳ, κατὰ τὸν Ἀκεσίνην αὖ 
ἔπλει, ἔστε ἐπὶ τὴν ξυμβολὴν τοῦ Ἀκεσίνου καὶ τοῦ 
Ἰνδοῦ ἧκεν. 



Flavius Arrianus Hist., Phil., Alexandri anabasis 
Book 6, chapter 14, section 5, line 2

                 τέσσαρες γὰρ οὗτοι μεγάλοι ποταμοὶ καὶ 
ναυσίποροι οἱ τέσσαρες εἰς τὸν Ἰνδὸν ποταμὸν τὸ 
ὕδωρ ξυμβάλλουσιν, οὐ ξὺν τῇ σφετέρᾳ ἕκαστος ἐπω-
νυμίᾳ, ἀλλὰ ὁ Ὑδάσπης μὲν ἐς τὸν Ἀκεσίνην ἐμβάλλει, 
ἐμβαλὼν δὲ τὸ πᾶν ὕδωρ Ἀκεσίνην παρέχεται καλού-
μενον· αὖθις δὲ ὁ Ἀκεσίνης οὗτος ξυμβάλλει τῷ 
Ὑδραώτῃ, καὶ παραλαβὼν τοῦτον ἔτι Ἀκεσίνης ἐστί·   
καὶ τὸν Ὕφασιν ἐπὶ τούτῳ ὁ Ἀκεσίνης παραλαβὼν τῷ 
αὑτοῦ δὴ ὀνόματι ἐς τὸν Ἰνδὸν ἐμβάλλει· ξυμβαλὼν 
δὲ ξυγχωρεῖ δὴ τῷ Ἰνδῷ. 



Flavius Arrianus Hist., Phil., Alexandri anabasis 
Book 6, chapter 14, section 5, line 10

                                 ἔνθεν δὴ ὁ Ἰνδὸς πρὶν ἐς 
τὸ Δέλτα σχισθῆναι οὐκ ἀπιστῶ ὅτι καὶ ἐς ἑκατὸν 
σταδίους ἔρχεται καὶ ὑπὲρ τοὺς ἑκατὸν τυχὸν, ἵναπερ 
λιμνάζει μᾶλλον. 



Flavius Arrianus Hist., Phil., Alexandri anabasis 
Book 6, chapter 15, section 1, line 2

Ἐνταῦθα ἐπὶ ταῖς ξυμβολαῖς τοῦ Ἀκεσίνου καὶ 
Ἰνδοῦ προσέμενεν ἔστε ἀφίκετο αὐτῷ ξὺν τῇ στρατιᾷ 
Περδίκκας καταστρεψάμενος ἐν παρόδῳ τὸ Ἀβαστανῶν 
ἔθνος αὐτόνομον. 



Flavius Arrianus Hist., Phil., Alexandri anabasis 
Book 6, chapter 15, section 1, line 7

                     ἐν τούτῳ δὲ ἄλλαι τε προσγίγνονται 
Ἀλεξάνδρῳ τριακόντοροι καὶ πλοῖα στρογγύλα ἄλλα, ἃ δὴ 
ἐν Ξάθροις ἐναυπηγήθη αὐτῷ, καὶ <Σόγδοι> ἄλλο ἔθνος 
Ἰνδῶν αὐτόνομον προσεχώρησαν. 



Flavius Arrianus Hist., Phil., Alexandri anabasis 
Book 6, chapter 15, section 1, line 8

                                    καὶ παρὰ Ὀσσαδίων, 
καὶ τούτου γένους αὐτονόμου Ἰνδικοῦ, πρέσβεις ἧκον, 
ἐνδιδόντες καὶ οὗτοι τοὺς Ὀσσαδίους. 



Flavius Arrianus Hist., Phil., Alexandri anabasis 
Book 6, chapter 15, section 2, line 3

                                              Φιλίππῳ μὲν δὴ 
τῆς σατραπείας ὅρους ἔταξε τὰς συμβολὰς τοῦ τε Ἀκεσί-
νου καὶ Ἰνδοῦ καὶ ἀπολείπει ξὺν αὐτῷ τούς τε Θρᾷκας 
πάντας καὶ ἐκ τῶν τάξεων ὅσοι ἐς φυλακὴν τῆς χώρας 
ἱκανοὶ ἐφαίνοντο, πόλιν τε ἐνταῦθα κτίσαι ἐκέλευσεν ἐπ' 
αὐτῇ τῇ ξυμβολῇ τοῖν ποταμοῖν, ἐλπίσας μεγάλην τε 
ἔσεσθαι καὶ ἐπιφανῆ ἐς ἀνθρώπους, καὶ νεωσοίκους 
ποιηθῆναι. 



Flavius Arrianus Hist., Phil., Alexandri anabasis 
Book 6, chapter 15, section 4, line 3

Ἔνθα δὴ διαβιβάσας Κρατερόν τε καὶ τῆς στρα-
τιᾶς τὴν πολλὴν καὶ τοὺς ἐλέφαντας ἐπ' ἀριστερὰ τοῦ 
Ἰνδοῦ ποταμοῦ, ὅτι εὐπορώτερά τε ταύτῃ τὰ παρὰ 
τὸν ποταμὸν στρατιᾷ βαρείᾳ ἐφαίνετο καὶ τὰ ἔθνη τὰ 
προσοικοῦντα οὐ πάντῃ φίλια ἦν, αὐτὸς κατέπλει ἐς 
τῶν Σόγδων τὸ βασίλειον. 



Flavius Arrianus Hist., Phil., Alexandri anabasis 
Book 6, chapter 15, section 4, line 9

                                              τῆς δὲ ἀπὸ 
τῶν ξυμβολῶν τοῦ τε Ἰνδοῦ καὶ Ἀκεσίνου χώρας ἔστε 
ἐπὶ θάλασσαν σατράπην ἀπέδειξε[ν Ὀξυάρτην καὶ] 
Πείθωνα ξὺν τῇ παραλίᾳ πάσῃ τῆς Ἰνδῶν γῆς. 



Flavius Arrianus Hist., Phil., Alexandri anabasis 
Book 6, chapter 15, section 5, line 4

Καὶ Κρατερὸν μὲν ἐκπέμπει αὖθις ξὺν τῇ στρατιᾷ 
[διὰ τῆς Ἀραχωτῶν καὶ Δραγγῶν γῆς], αὐτὸς δὲ 
κατέπλει ἐς τὴν Μουσικανοῦ ἐπικράτειαν, ἥντινα 
εὐδαιμονεστάτην τῆς Ἰνδῶν γῆς εἶναι ἐξηγγέλλετο, 
ὅτι οὔπω οὔτε ἀπηντήκει αὐτῷ Μουσικανὸς ἐνδιδοὺς 
αὑτόν τε καὶ τὴν χώραν οὔτε πρέσβεις ἐπὶ φιλίᾳ 
ἐκπέμπει, οὐδέ τι οὔτε αὐτὸς ἐπεπόμφει ἃ δὴ μεγάλῳ 
βασιλεῖ εἰκός, οὔτε τι ᾐτήκει ἐξ Ἀλεξάνδρου. 



Flavius Arrianus Hist., Phil., Alexandri anabasis 
Book 6, chapter 15, section 6, line 7

                                                οὕτω δὴ 
ἐκπλαγεὶς κατὰ τάχος ἀπήντα Ἀλεξάνδρῳ, δῶρά τε τὰ 
πλείστου ἄξια παρ' Ἰνδοῖς κομίζων καὶ τοὺς ἐλέφαντας 
ξύμπαντας ἄγων καὶ τὸ ἔθνος τε καὶ αὑτὸν ἐνδιδοὺς 
καὶ ὁμολογῶν ἀδικεῖν, ὅπερ μέγιστον παρ' Ἀλεξάνδρῳ 
ἦν ἐς τὸ τυχεῖν ὧν τις δέοιτο. 



Flavius Arrianus Hist., Phil., Alexandri anabasis 
Book 6, chapter 16, section 2, line 7

                                 ὁ δὲ τὴν μὲν λείαν τῇ 
στρατιᾷ δίδωσι, τοὺς ἐλέφαντας δὲ ἅμα οἷ ἦγε· καὶ <αἱ> 
ἄλλαι δὲ πόλεις αὐτῷ αἱ ἐν τῇ αὐτῇ χώρᾳ ἐνεδίδοντο 
ἐπιόντι οὐδέ τις ἐτρέπετο ἐς ἀλκήν· οὕτω καὶ Ἰνδοὶ 
πάντες ἐδεδούλωντο ἤδη τῇ γνώμῃ πρὸς Ἀλεξάνδρου 
τε καὶ τῆς Ἀλεξάνδρου τύχης. 



Flavius Arrianus Hist., Phil., Alexandri anabasis 
Book 6, chapter 16, section 3, line 1

Ὁ δὲ ἐπὶ Σάμβον αὖ ἦγε τῶν ὀρείων Ἰνδῶν 
σατράπην ὑπ' αὐτοῦ κατασταθέντα, ὃς πεφευγέναι 
αὐτῷ ἐξηγγέλλετο ὅτι Μουσικανὸν ἀφειμένον πρὸς 
Ἀλεξάνδρου ἐπύθετο καὶ τῆς χώρας τῆς ἑαυτοῦ ἄρχοντα· 
τὰ γὰρ πρὸς Μουσικανὸν αὐτῷ πολέμια ἦν. 



Flavius Arrianus Hist., Phil., Alexandri anabasis 
Book 6, chapter 16, section 5, line 2

ὁ δὲ καὶ ἄλλην πόλιν ἐν τούτῳ ἀποστᾶσαν εἷλεν καὶ 
τῶν Βραχμάνων, οἳ δὴ σοφισταὶ τοῖς Ἰνδοῖς εἰσιν, 
ὅσοι αἴτιοι τῆς ἀποστάσεως ἐγένοντο ἀπέκτεινεν. 



Flavius Arrianus Hist., Phil., Alexandri anabasis 
Book 6, chapter 16, section 5, line 4

                                                           ὑπὲρ 
ὧν ἐγὼ τῆς σοφίας, εἰ δή τίς ἐστιν, ἐν τῇ Ἰνδικῇ 
ξυγγραφῇ δηλώσω. 



Flavius Arrianus Hist., Phil., Alexandri anabasis 
Book 6, chapter 17, section 2, line 7

ἀφίκετο δὲ αὐτῷ καὶ ὁ τῶν Πατάλων τῆς χώρας 
ἄρχων, ὃ δὴ τὸ Δέλτα ἔφην εἶναι τὸ πρὸς τοῦ ποταμοῦ 
τοῦ Ἰνδοῦ ποιούμενον, μεῖζον ἔτι τοῦ Δέλτα τοῦ 
Αἰγυπτίου, καὶ οὗτος τήν τε χώραν αὐτῷ ἐνεδίδου 
πᾶσαν καὶ αὑτόν τε καὶ τὰ αὑτοῦ ἐπέτρεψεν. 



Flavius Arrianus Hist., Phil., Alexandri anabasis 
Book 6, chapter 17, section 4, line 5

            .... Ἡφαιστίων ἐπετάχθη, Πείθωνα δὲ 
τούς τε ἱππακοντιστὰς ἄγοντα καὶ τοὺς Ἀγριᾶνας ἐς 
τὴν ἐπέκεινα ὄχθην τοῦ Ἰνδοῦ διαβιβάσας, οὐχ ᾗπερ 
Ἡφαιστίων τὴν στρατιὰν ἄγειν ἤμελλε, τάς τε ἐκ-
τετειχισμένας ἤδη πόλεις ξυνοικίσαι ἐκέλευσε καὶ εἰ 
δή τινα νεωτερίζοιτο πρὸς τῶν ταύτῃ Ἰνδῶν καὶ 
ταῦτα ἐς κόσμον καταστήσαντα ξυμβάλλειν οἱ ἐς τὰ 
Πάταλα. 



Flavius Arrianus Hist., Phil., Alexandri anabasis 
Book 6, chapter 18, section 2, line 1

Περὶ δὲ τοῖς Πατάλοις σχίζεται τοῦ Ἰνδοῦ τὸ 
ὕδωρ ἐς <δύο> ποταμοὺς μεγάλους, καὶ οὗτοι ἀμφό-
τεροι σώζουσι τοῦ Ἰνδοῦ τὸ ὄνομα ἔστε ἐπὶ τὴν 
θάλασσαν. 



Flavius Arrianus Hist., Phil., Alexandri anabasis 
Book 6, chapter 18, section 4, line 2

                                      οὐκ ἔχοντι δὲ αὐτῷ 
ἡγεμόνα τοῦ πλοῦ, ὅτι πεφεύγεσαν οἱ ταύτῃ Ἰνδοί, 
ἀπορώτερα τὰ τοῦ κατάπλου ἦν· χειμών τε ἐπιγίγνεται 
ἐς τὴν ὑστεραίαν ἀπὸ τῆς ἀναγωγῆς καὶ ὁ ἄνεμος τῷ 
ῥόῳ πνέων ὑπεναντίος κοῖλόν τε ἐποίει τὸν ποταμὸν 
καὶ τὰ σκάφη διέσειεν, ὥστε ἐπόνησαν αὐτῷ αἱ 
πλεῖσται τῶν νεῶν, τῶν δὲ τριακοντόρων ἔστιν αἳ καὶ 
πάντῃ διελύθησαν. 



Flavius Arrianus Hist., Phil., Alexandri anabasis 
Book 6, chapter 18, section 5, line 4

               καὶ τῶν ψιλῶν τοὺς κουφοτάτους 
ἐκπέμψας ἐς τὴν προσωτέρω τῆς ὄχθης χώραν ξυλ-
λαμβάνει τινὰς τῶν Ἰνδῶν, καὶ οὗτοι τὸ ἀπὸ τοῦδε 
ἐξηγοῦντο αὐτῷ τὸν πόρον. 



Flavius Arrianus Hist., Phil., Alexandri anabasis 
Book 6, chapter 19, section 5, line 1

αὐτὸς δὲ ὑπερβαλὼν τοῦ Ἰνδοῦ ποταμοῦ τὰς ἐκβολὰς 
ἐς τὸ πέλαγος ἀνέπλει, ὡς μὲν ἔλεγεν, ἀπιδεῖν εἴ πού 
τις χώρα πλησίον ἀνίσχει ἐν τῷ πόντῳ, ἐμοὶ δὲ δοκεῖ, 
οὐχ ἥκιστα ὡς πεπλευκέναι τὴν μεγάλην τὴν ἔξω 
Ἰνδῶν θάλασσαν. 



Flavius Arrianus Hist., Phil., Alexandri anabasis 
Book 6, chapter 20, section 1, line 9

                              Ἡφαιστίων μὲν δὴ ἐτάχθη 
παρασκευάζειν τὰ πρὸς τὸν ἐκτειχισμόν τε τοῦ ναυ-
στάθμου καὶ τῶν νεωσοίκων τὴν κατασκευήν· καὶ γὰρ 
καὶ ἐνταῦθα ἐπενόει στόλον ὑπολείπεσθαι νεῶν οὐκ 
ὀλίγων πρὸς τῇ πόλει τοῖς Πατάλοις, ἵναπερ ἐσχίζετο 
ὁ ποταμὸς ὁ Ἰνδός. 



Flavius Arrianus Hist., Phil., Alexandri anabasis 
Book 6, chapter 20, section 2, line 1

Αὐτὸς δὲ κατὰ τὸ ἕτερον στόμα τοῦ Ἰνδοῦ κατ-
έπλει αὖθις ἐς τὴν μεγάλην θάλασσαν, ὡς καταμαθεῖν, 
ὅπῃ εὐπορωτέρα ἡ ἐκβολὴ τοῦ Ἰνδοῦ ἐς τὸν πόντον   
γίγνεται· ἀπέχει δὲ ἀλλήλων τὰ στόματα τοῦ ποταμοῦ 
τοῦ Ἰνδοῦ ἐς σταδίους μάλιστα ὀκτακοσίους καὶ 
χιλίους. 



Flavius Arrianus Hist., Phil., Alexandri anabasis 
Book 6, chapter 20, section 4, line 3

                           προσορμισθεὶς οὖν κατὰ τὴν 
λίμνην ἵναπερ οἱ καθηγεμόνες ἐξηγοῦντο, τῶν μὲν 
στρατιωτῶν τοὺς πολλοὺς καταλείπει σὺν Λεοννάτῳ 
αὐτοῦ καὶ τοὺς κερκούρους ξύμπαντας, αὐτὸς δὲ ταῖς 
τριακοντόροις τε καὶ ἡμιολίαις ὑπερβαλὼν τὴν ἐκβολὴν 
τοῦ Ἰνδοῦ καὶ προελθὼν καὶ ταύτῃ ἐς τὴν θάλασσαν 
εὐπορωτέραν τε κατέμαθεν τὴν ἐπὶ τάδε τοῦ Ἰνδοῦ 
ἐκβολὴν καὶ αὐτὸς προσορμισθεὶς τῷ αἰγιαλῷ καὶ τῶν 
ἱππέων τινὰς ἅμα οἷ ἔχων παρὰ θάλασσαν ᾔει στα-
θμοὺς τρεῖς, τήν τε χώραν ὁποία τίς ἐστιν ἡ ἐν 
τῷ παράπλῳ ἐπισκεπτόμενος καὶ φρέατα ὀρύσσεσθαι 
κελεύων, ὅπως ἔχοιεν ὑδρεύεσθαι οἱ πλέοντες. 



Flavius Arrianus Hist., Phil., Alexandri anabasis 
Book 6, chapter 21, section 3, line 11

                    ἐκεῖθεν δὲ ἀναλαβὼν τῶν ὑπασπι-
στῶν τε καὶ τῶν τοξοτῶν τοὺς ἡμίσεας καὶ τῶν πεζε-
ταίρων καλουμένων τὰς τάξεις καὶ τῆς ἵππου τῆς 
ἑταιρικῆς τό τε ἄγημα καὶ ἴλην ἀφ' ἑκάστης ἱππαρχίας 
καὶ τοὺς ἱπποτοξότας ξύμπαντας ὡς ἐπὶ τὴν θάλασσαν 
ἐς ἀριστερὰ ἐτράπετο, ὕδατά τε ὀρύσσειν, ὡς κατὰ τὸν 
παράπλουν ἄφθονα εἴη τῇ στρατιᾷ τῇ παραπλεούσῃ, 
καὶ ἅμα ὡς τοῖς Ὠρείταις τοῖς ταύτῃ Ἰνδοῖς αὐτονόμοις   
ἐκ πολλοῦ οὖσιν ἄφνω ἐπιπεσεῖν, ὅτι μηδὲν φίλιον 
αὐτοῖς ἐς αὐτόν τε καὶ τὴν στρατιὰν ἐπέπρακτο. 



Flavius Arrianus Hist., Phil., Alexandri anabasis 
Book 6, chapter 24, section 2, line 5

                                         οὐ μὴν ἀγνοήσαντα 
Ἀλέξανδρον τῆς ὁδοῦ τὴν χαλεπότητα ταύτῃ ἐλθεῖν, 
τοῦτο μὲν μόνος Νέαρχος λέγει ὧδε, ἀλλὰ ἀκούσαντα 
γὰρ ὅτι οὔπω τις πρόσθεν διελθὼν ταύτῃ ξὺν στρατιᾷ 
ἀπεσώθη, ὅτι μὴ Σεμίραμις ἐξ Ἰνδῶν ἔφυγε. 



Flavius Arrianus Hist., Phil., Alexandri anabasis 
Book 6, chapter 24, section 3, line 2

                             ἐλθεῖν γὰρ δὴ καὶ Κῦρον 
ἐς τοὺς χώρους τούτους ὡς ἐσβαλοῦντα ἐς τὴν Ἰνδῶν 
γῆν, φθάσαι δὲ ὑπὸ τῆς ἐρημίας τε καὶ ἀπορίας τῆς 
ὁδοῦ ταύτης ἀπολέσαντα τὴν πολλὴν τῆς στρατιᾶς. 



Flavius Arrianus Hist., Phil., Alexandri anabasis 
Book 6, chapter 25, section 4, line 4

            ὕεται γὰρ ἡ Γαδρωσίων γῆ ὑπ' ἀνέμων 
τῶν ἐτησίων, καθάπερ οὖν καὶ ἡ Ἰνδῶν γῆ, οὐ τὰ 
πεδία τῶν Γαδρωσίων, ἀλλὰ τὰ ὄρη, ἵναπερ προς-
φέρονταί τε αἱ νεφέλαι ἐκ τοῦ πνεύματος καὶ ἀνα-
χέονται, οὐχ ὑπερβάλλουσαι τῶν ὀρῶν τὰς κορυφάς. 



Flavius Arrianus Hist., Phil., Alexandri anabasis 
Book 6, chapter 27, section 2, line 3

                          ἤδη τε ἐπὶ Καρμανίας προὐ-
χώρει ὁ βασιλεὺς καὶ ἀγγέλλεται αὐτῷ Φίλιππον τὸν 
σατράπην τῆς Ἰνδῶν γῆς ἐπιβουλευθέντα πρὸς τῶν 
μισθοφόρων δόλῳ ἀποθανεῖν, τοὺς δὲ ἀποκτείναντας 
ὅτι οἱ σωματοφύλακες τοῦ Φιλίππου οἱ Μακεδόνες 
τοὺς μὲν ἐν αὐτῷ τῷ ἔργῳ, τοὺς δὲ καὶ ὕστερον 
λαβόντες ἀπέκτειναν. 



Flavius Arrianus Hist., Phil., Alexandri anabasis 
Book 6, chapter 27, section 2, line 8

                        ταῦτα δὲ ὡς ἔγνω, ἐκπέμπει 
γράμματα ἐς Ἰνδοὺς παρὰ Εὔδαμόν τε καὶ Ταξίλην 
ἐπιμελεῖσθαι τῆς χώρας τῆς πρόσθεν ὑπὸ Φιλίππῳ 
τεταγμένης ἔστ' ἂν αὐτὸς σατράπην ἐκπέμψῃ ἐπ' 
αὐτῆς. 



Flavius Arrianus Hist., Phil., Alexandri anabasis 
Book 6, chapter 28, section 2, line 2

Ἤδη δέ τινες καὶ τοιάδε ἀνέγραψαν, οὐ πιστὰ 
ἐμοὶ λέγοντες, ὡς συζεύξας δύο ἁρμαμάξας κατακεί-
μενος ξὺν τοῖς ἑταίροις καταυλούμενος τὴν διὰ Καρ-
μανίας ἦγεν, ἡ στρατιὰ δὲ αὐτῷ ἐστεφανωμένη τε καὶ 
παίζουσα εἵπετο, προὔκειτο δὲ αὐτῇ σῖτά τε καὶ ὅσα 
ἄλλα ἐς τρυφὴν παρὰ τὰς ὁδοὺς συγκεκομισμένα πρὸς 
τῶν Καρμανίων, καὶ ταῦτα πρὸς μίμησιν τῆς Διονύσου 
βακχείας ἀπεικάσθη Ἀλεξάνδρῳ, ὅτι καὶ ὑπὲρ ἐκείνου 
λόγος ἐλέγετο καταστρεψάμενον Ἰνδοὺς Διόνυσον οὕτω 
τὴν πολλὴν τῆς Ἀσίας ἐπελθεῖν, καὶ Θρίαμβόν τε 
αὐτὸν ἐπικληθῆναι τὸν Διόνυσον καὶ τὰς ἐπὶ ταῖς 
νίκαις ταῖς ἐκ πολέμου πομπὰς ἐπὶ τῷ αὐτῷ τούτῳ 
θριάμβους. 



Flavius Arrianus Hist., Phil., Alexandri anabasis 
Book 6, chapter 28, section 3, line 3

             ἀλλὰ ἐκεῖνα ἤδη Ἀριστοβούλῳ ἑπόμενος 
ξυγγράφω, θῦσαι ἐν Καρμανίᾳ Ἀλέξανδρον χαριστήρια 
τῆς κατ' Ἰνδῶν νίκης καὶ ὑπὲρ τῆς στρατιᾶς, ὅτι 
ἀπεσώθη ἐκ Γαδρωσίων, καὶ ἀγῶνα διαθεῖναι μουσικόν 
τε καὶ γυμνικόν· καταλέξαι δὲ καὶ Πευκέσταν ἐς τοὺς 
σωματοφύλακας, ἤδη μὲν ἐγνωκότα σατράπην κατα-
στῆσαι τῆς Περσίδος, ἐθέλοντα δὲ πρὸ τῆς σατρα-
πείας μηδὲ ταύτης τῆς τιμῆς καὶ πίστεως ἀπείρατον 
εἶναι ἐπὶ τῷ ἐν Μαλλοῖς ἔργῳ· εἶναι δὲ αὐτῷ ἑπτὰ 
εἰς τότε σωματοφύλακας, Λεοννάτον Ἀντέου, Ἡφαι-
στίωνα τὸν Ἀμύντορος, Λυσίμαχον Ἀγαθοκλέους, Ἀρι-
στόνουν Πεισαίου, τούτους μὲν Πελλαίους, Περδίκκαν 




Flavius Arrianus Hist., Phil., Alexandri anabasis 
Book 6, chapter 28, section 6, line 4

           τοῦτον μὲν δὴ καταπέμπει αὖθις ἐκπερι-
πλεύσοντα ἔστε ἐπὶ τὴν Σουσιανῶν τε γῆν καὶ τοῦ 
Τίγρητος ποταμοῦ τὰς ἐκβολάς· ὅπως δὲ ἐπλεύσθη 
αὐτῷ τὰ ἀπὸ τοῦ Ἰνδοῦ ποταμοῦ ἐπὶ τὴν θάλασσαν 
τὴν Περσικὴν καὶ τὸ στόμα τοῦ Τίγρητος, ταῦτα ἰδίᾳ 
ἀναγράψω αὐτῷ Νεάρχῳ ἑπόμενος, ὡς καὶ τήνδε εἶναι   
ὑπὲρ Ἀλεξάνδρου Ἑλληνικὴν τὴν συγγραφήν. 



Flavius Arrianus Hist., Phil., Alexandri anabasis 
Book 6, chapter 29, section 2, line 3

ὡς δὲ ἐπὶ τοῖς ὅροις ἦν τῆς Περσίδος, Φρασαόρτην 
μὲν οὐ κατέλαβε σατραπεύοντα ἔτι (νόσῳ γὰρ τε-
τελευτηκὼς ἐτύγχανεν ἐν Ἰνδοῖς ἔτι Ἀλεξάνδρου ὄντος), 
Ὀρξίνης δὲ ἐπεμέλετο τῆς Περσίδος, οὐ πρὸς Ἀλεξ-
άνδρου κατασταθείς, ἀλλ' ὅτι οὐκ ἀπηξίωσεν αὑτὸν 
ἐν κόσμῳ Πέρσας διαφυλάξαι Ἀλεξάνδρῳ οὐκ ὄντος 
ἄλλου ἄρχοντος. 



Flavius Arrianus Hist., Phil., Alexandri anabasis 
Book 7, chapter 1, section 2, line 1

ΑΡΡΙΑΝΟΥ 
ΑΛΕΞΑΝΔΡΟΥ ΑΝΑΒΑΣΕΩΣ 
ΒΙΒΛΙΟΝ ΕΒΔΟΜΟΝ


 Ὡς δὲ ἐς Πασαργάδας τε καὶ ἐς Περσέπολιν 
ἀφίκετο Ἀλέξανδρος, πόθος λαμβάνει αὐτὸν κατα-
πλεῦσαι κατὰ τὸν Εὐφράτην τε καὶ κατὰ τὸν Τίγρητα 
ἐπὶ τὴν θάλασσαν τὴν Περσικὴν καὶ τῶν τε ποταμῶν 
ἰδεῖν τὰς ἐκβολὰς τὰς ἐς τὸν πόντον, καθάπερ τοῦ 
Ἰνδοῦ, καὶ τὴν ταύτῃ θάλασσαν. 



Flavius Arrianus Hist., Phil., Alexandri anabasis 
Book 7, chapter 1, section 5, line 2

                               καὶ ἐπὶ τῷδε ἐπαινῶ τοὺς 
σοφιστὰς τῶν Ἰνδῶν, ὧν λέγουσιν ἔστιν οὓς κατα-
ληφθέντας ὑπ' Ἀλεξάνδρου ὑπαιθρίους ἐν λειμῶνι, 
ἵναπερ αὐτοῖς διατριβαὶ ἦσαν, ἄλλο μὲν οὐδὲν ποιῆσαι 
πρὸς τὴν ὄψιν αὐτοῦ τε καὶ τῆς στρατιᾶς, κρούειν δὲ 
τοῖς ποσὶ τὴν γῆν ἐφ' ἧς βεβηκότες ἦσαν. 



Flavius Arrianus Hist., Phil., Alexandri anabasis 
Book 7, chapter 2, section 2, line 4

                                                          ἐπεὶ 
καὶ ἐς Τάξιλα αὐτῷ ἀφικομένῳ καὶ ἰδόντι τῶν σοφι-
στῶν <τῶν]2 Ἰνδῶν τοὺς γυμνοὺς πόθος ἐγένετο ξυν-
εῖναί τινα οἱ τῶν ἀνδρῶν τούτων, ὅτι τὴν καρτερίαν 
αὐτῶν ἐθαύμασε· καὶ ὁ μὲν πρεσβύτατος τῶν σοφιστῶν, 
ὅτου ὁμιληταὶ οἱ ἄλλοι ἦσαν, Δάνδαμις ὄνομα, οὔτε 
αὐτὸς ἔφη παρ' Ἀλέξανδρον ἥξειν οὔτε τοὺς ἄλλους 
εἴα, ἀλλὰ ὑποκρίνασθαι γὰρ λέγεται ὡς Διὸς υἱὸς καὶ 
αὐτὸς εἴη, εἴπερ οὖν καὶ Ἀλέξανδρος, καὶ ὅτι οὔτε 
δέοιτό του τῶν παρ' Ἀλεξάνδρου, ἔχει<ν> γάρ οἱ εὖ 
τὰ παρόντα, καὶ ἅμα ὁρᾶν τοὺς ξὺν αὐτῷ πλανωμένους 
τοσαύτην γῆν καὶ θάλασσαν ἐπ' ἀγαθῷ οὐδενί, μηδὲ 




Flavius Arrianus Hist., Phil., Alexandri anabasis 
Book 7, chapter 2, section 4, line 1

                                                    οὔτ' 
οὖν ποθεῖν τι αὐτὸς ὅτου κύριος ἦν Ἀλέξανδρος 
δοῦναι, οὔτε αὖ δεδιέναι, ὅτου κρατοίη ἐκεῖνος, ἔστιν   
οὗ εἴργεσθαι· ζῶντι μὲν γάρ οἱ τὴν Ἰνδῶν γῆν ἐξ-
αρκεῖν φέρουσαν τὰ ὡραῖα, ἀποθανόντα δὲ ἀπαλ-
λαγήσεσθαι οὐκ ἐπιεικοῦς ξυνοίκου τοῦ σώματος. 



Flavius Arrianus Hist., Phil., Alexandri anabasis 
Book 7, chapter 3, section 3, line 5

                                                   αὐτῷ δὲ 
παρασκευασθῆναι μὲν ἵππον, ὅτι βαδίσαι ἀδυνάτως   
εἶχεν ὑπὸ τῆς νόσου· οὐ μὴν δυνηθῆναί γε οὐδὲ τοῦ 
ἵππου ἐπιβῆναι, ἀλλὰ ἐπὶ κλίνης γὰρ κομισθῆναι 
φερόμενον, ἐστεφανωμένον τε τῷ Ἰνδῶν νόμῳ καὶ 
ᾄδοντα τῇ Ἰνδῶν γλώσσῃ. 



Flavius Arrianus Hist., Phil., Alexandri anabasis 
Book 7, chapter 3, section 3, line 6

                                  οἱ δὲ Ἰνδοὶ λέγουσιν ὅτι 
ὕμνοι θεῶν ἦσαν καὶ αὐτῶν ἔπαινοι. 



Flavius Arrianus Hist., Phil., Alexandri anabasis 
Book 7, chapter 3, section 6, line 8

                                      ταῦτα καὶ τοιαῦτα 
ὑπὲρ Καλάνου τοῦ Ἰνδοῦ ἱκανοὶ ἀναγεγράφασιν, οὐκ 
ἀχρεῖα πάντα ἐς ἀνθρώπους, ὅτῳ γνῶναι ἐπιμελές, 
[ὅτι] ὡς καρτερόν τέ ἐστι καὶ ἀνίκητον γνώμη ἀνθρω-
πίνη ὅ τι περ ἐθέλοι ἐξεργάσασθαι. 



Flavius Arrianus Hist., Phil., Alexandri anabasis 
Book 7, chapter 4, section 2, line 4

πολλὰ μὲν δὴ ἐπεπλημμέλητο ἐκ τῶν κατεχόντων τὰς 
χώρας ὅσαι δορίκτητοι πρὸς Ἀλεξάνδρου ἐγένοντο ἔς 
τε τὰ ἱερὰ καὶ τάφους καὶ αὐτοὺς τοὺς ὑπηκόους, 
ὅτι χρόνιος ὁ εἰς Ἰνδοὺς στόλος ἐγεγένητο τῷ βασιλεῖ 
καὶ οὐ πιστὸν ἐφαίνετο ἀπονοστήσειν αὐτὸν ἐκ 
τοσῶνδε ἐθνῶν καὶ τοσῶνδε ἐλεφάντων, ὑπὲρ τὸν 
Ἰνδόν τε καὶ Ὑδάσπην καὶ τὸν Ἀκεσίνην καὶ Ὕφασιν 
φθειρόμενον. 



Flavius Arrianus Hist., Phil., Alexandri anabasis 
Book 7, chapter 5, section 5, line 2

                             καὶ ἐστεφάνωσε χρυσοῖς 
στεφάνοις τοὺς ἀνδραγαθίᾳ διαπρέποντας, πρῶτον μὲν 
Πευκέσταν τὸν ὑπερασπίσαντα, ἔπειτα Λεοννάτον, καὶ 
τοῦτον ὑπερασπίσαντα, καὶ διὰ τοὺς ἐν Ἰνδοῖς κινδύ-
νους καὶ τὴν ἐν Ὤροις νίκην γενομένην, ὅτι παραταξά-
μενος σὺν τῇ ὑπολειφθείσῃ δυνάμει πρὸς τοὺς νεωτερί-
ζοντας τῶν τε Ὠρειτῶν καὶ τῶν πλησίον τούτων   
ᾠκισμένων τῇ τε μάχῃ ἐκράτησε καὶ τὰ ἄλλα καλῶς 
ἔδοξε τὰ ἐν Ὤροις κοσμῆσαι. 



Flavius Arrianus Hist., Phil., Alexandri anabasis 
Book 7, chapter 5, section 6, line 2

                                    ἐπὶ τούτοις δὲ Νέαρχον 
ἐπὶ τῷ περίπλῳ τῷ ἐκ τῆς Ἰνδῶν γῆς κατὰ τὴν μεγά-
λην θάλασσαν ἐστεφάνωσε· καὶ γὰρ καὶ οὗτος ἤδη 
ἀφιγμένος ἐς Σοῦσα ἦν· ἐπὶ τούτοις δὲ Ὀνησίκριτον 
τὸν κυβερνήτην τῆς νεὼς τῆς βασιλικῆς· ἔτι δὲ 
Ἡφαιστίωνα καὶ τοὺς ἄλλους τοὺς σωματοφύλακας. 



Flavius Arrianus Hist., Phil., Alexandri anabasis 
Book 7, chapter 9, section 8, line 6

σατράπας τοὺς Δαρείου τήν τε Ἰωνίαν πᾶσαν τῇ 
ὑμετέρᾳ ἀρχῇ προσέθηκα καὶ τὴν Αἰολίδα πᾶσαν καὶ 
Φρύγας ἀμφοτέρους καὶ Λυδούς, καὶ Μίλητον εἷλον 
πολιορκίᾳ· τὰ δὲ ἄλλα πάντα ἑκόντα προσχωρήσαντα 
λαβὼν ὑμῖν καρποῦσθαι ἔδωκα· καὶ τὰ ἐξ Αἰγύπτου 
καὶ Κυρήνης ἀγαθά, ὅσα ἀμαχεὶ ἐκτησάμην, ὑμῖν 
ἔρχεται, ἥ τε κοίλη Συρία καὶ ἡ Παλαιστίνη καὶ ἡ 
μέση τῶν ποταμῶν ὑμέτερον κτῆμά εἰσι, καὶ Βαβυλὼν 
καὶ Βάκτρα καὶ Σοῦσα ὑμέτερα, καὶ ὁ Λυδῶν πλοῦτος   
καὶ οἱ Περσῶν θησαυροὶ καὶ τὰ Ἰνδῶν ἀγαθὰ καὶ ἡ 
ἔξω θάλασσα ὑμέτερα· ὑμεῖς σατράπαι, ὑμεῖς στρατηγοί, 
ὑμεῖς ταξιάρχαι. 



Flavius Arrianus Hist., Phil., Alexandri anabasis 
Book 7, chapter 10, section 6, line 6

βούλεσθε, ἄπιτε πάντες, καὶ ἀπελθόντες οἴκοι ἀπαγ-
γείλατε ὅτι τὸν βασιλέα ὑμῶν Ἀλέξανδρον, νικῶντα 
μὲν Πέρσας καὶ Μήδους καὶ Βακτρίους καὶ Σάκας, 
καταστρεψάμενον δὲ Οὐξίους τε καὶ Ἀραχωτοὺς καὶ 
Δράγγας, κεκτημένον δὲ καὶ Παρθυαίους καὶ Χορας-
μίους καὶ Ὑρκανίους ἔστε ἐπὶ τὴν θάλασσαν τὴν 
Κασπίαν, ὑπερβάντα δὲ τὸν Καύκασον ὑπὲρ τὰς 
Κασπίας πύλας, καὶ περάσαντα Ὄξον τε ποταμὸν καὶ 
Τάναϊν, ἔτι δὲ τὸν Ἰνδὸν ποταμόν, οὐδενὶ ἄλλῳ ὅτι 
μὴ Διονύσῳ περαθέντα, καὶ τὸν Ὑδάσπην καὶ τὸν 
Ἀκεσίνην καὶ τὸν Ὑδραώτην, καὶ τὸν Ὕφασιν δια-
περάσαντα ἄν, εἰ μὴ ὑμεῖς ἀπωκνήσατε, καὶ ἐς τὴν   
μεγάλην θάλασσαν κατ' ἀμφότερα τοῦ Ἰνδοῦ τὰ 
στόματα ἐμβαλόντα, καὶ διὰ τῆς Γαδρωσίας τῆς ἐρήμου 
ἐλθόντα, ᾗ οὐδείς πω πρόσθεν σὺν στρατιᾷ ἦλθε, καὶ 
Καρμανίαν ἐν παρόδῳ προσκτησάμενον καὶ τὴν Ὠρει-
τῶν γῆν, περιπεπλευκότος δὲ ἤδη αὐτῷ τοῦ ναυτικοῦ 
τὴν ἀπ' Ἰνδῶν γῆς εἰς Πέρσας θάλασσαν, ὡς εἰς 




Flavius Arrianus Hist., Phil., Alexandri anabasis 
Book 7, chapter 10, section 7, line 3


Δράγγας, κεκτημένον δὲ καὶ Παρθυαίους καὶ Χορας-
μίους καὶ Ὑρκανίους ἔστε ἐπὶ τὴν θάλασσαν τὴν 
Κασπίαν, ὑπερβάντα δὲ τὸν Καύκασον ὑπὲρ τὰς 
Κασπίας πύλας, καὶ περάσαντα Ὄξον τε ποταμὸν καὶ 
Τάναϊν, ἔτι δὲ τὸν Ἰνδὸν ποταμόν, οὐδενὶ ἄλλῳ ὅτι 
μὴ Διονύσῳ περαθέντα, καὶ τὸν Ὑδάσπην καὶ τὸν 
Ἀκεσίνην καὶ τὸν Ὑδραώτην, καὶ τὸν Ὕφασιν δια-
περάσαντα ἄν, εἰ μὴ ὑμεῖς ἀπωκνήσατε, καὶ ἐς τὴν   
μεγάλην θάλασσαν κατ' ἀμφότερα τοῦ Ἰνδοῦ τὰ 
στόματα ἐμβαλόντα, καὶ διὰ τῆς Γαδρωσίας τῆς ἐρήμου 
ἐλθόντα, ᾗ οὐδείς πω πρόσθεν σὺν στρατιᾷ ἦλθε, καὶ 
Καρμανίαν ἐν παρόδῳ προσκτησάμενον καὶ τὴν Ὠρει-
τῶν γῆν, περιπεπλευκότος δὲ ἤδη αὐτῷ τοῦ ναυτικοῦ 
τὴν ἀπ' Ἰνδῶν γῆς εἰς Πέρσας θάλασσαν, ὡς εἰς 
Σοῦσα ἐπανηγάγετε, ἀπολιπόντες οἴχεσθε, παραδόντες 
φυλάσσειν τοῖς νενικημένοις βαρβάροις. 



Flavius Arrianus Hist., Phil., Alexandri anabasis 
Book 7, chapter 10, section 7, line 8

Τάναϊν, ἔτι δὲ τὸν Ἰνδὸν ποταμόν, οὐδενὶ ἄλλῳ ὅτι 
μὴ Διονύσῳ περαθέντα, καὶ τὸν Ὑδάσπην καὶ τὸν 
Ἀκεσίνην καὶ τὸν Ὑδραώτην, καὶ τὸν Ὕφασιν δια-
περάσαντα ἄν, εἰ μὴ ὑμεῖς ἀπωκνήσατε, καὶ ἐς τὴν   
μεγάλην θάλασσαν κατ' ἀμφότερα τοῦ Ἰνδοῦ τὰ 
στόματα ἐμβαλόντα, καὶ διὰ τῆς Γαδρωσίας τῆς ἐρήμου 
ἐλθόντα, ᾗ οὐδείς πω πρόσθεν σὺν στρατιᾷ ἦλθε, καὶ 
Καρμανίαν ἐν παρόδῳ προσκτησάμενον καὶ τὴν Ὠρει-
τῶν γῆν, περιπεπλευκότος δὲ ἤδη αὐτῷ τοῦ ναυτικοῦ 
τὴν ἀπ' Ἰνδῶν γῆς εἰς Πέρσας θάλασσαν, ὡς εἰς 
Σοῦσα ἐπανηγάγετε, ἀπολιπόντες οἴχεσθε, παραδόντες 
φυλάσσειν τοῖς νενικημένοις βαρβάροις. 



Flavius Arrianus Hist., Phil., Alexandri anabasis 
Book 7, chapter 16, section 2, line 5

                           πόθος γὰρ εἶχεν αὐτὸν καὶ 
ταύτην ἐκμαθεῖν τὴν θάλασσαν τὴν Κασπίαν τε καὶ 
Ὑρκανίαν καλουμένην ποίᾳ τινὶ ξυμβάλλει θαλάσσῃ, 
πότερα τῇ τοῦ πόντου τοῦ Εὐξείνου ἢ ἀπὸ τῆς ἑῴας 
τῆς κατ' Ἰνδοὺς ἐκπεριερχομένη ἡ μεγάλη θάλασσα 
ἀναχεῖται εἰς κόλπον τὸν Ὑρκάνιον, καθάπερ οὖν καὶ 
τὸν Περσικὸν ἐξεῦρε, τὴν Ἐρυθρὰν δὴ καλουμένην 
θάλασσαν, κόλπον οὖσαν τῆς μεγάλης θαλάσσης. 



Flavius Arrianus Hist., Phil., Alexandri anabasis 
Book 7, chapter 16, section 3, line 6

                                                    οὐ 
γάρ πω ἐξεύρηντο αἱ ἀρχαὶ τῆς Κασπίας θαλάσσης, 
καίτοι ἐθνῶν τε αὐτὴν <περι>οικούντων οὐκ ὀλίγων 
καὶ ποταμῶν πλοΐμων ἐμβαλλόντων ἐς αὐτήν· ἐκ 
Βάκτρων μὲν Ὄξος, μέγιστος τῶν Ἀσιανῶν ποταμῶν, 
πλήν γε δὴ τῶν Ἰνδῶν, ἐξίησιν ἐς ταύτην τὴν θάλας-
σαν, διὰ Σκυθῶν δὲ Ἰαξάρτης· καὶ τὸν Ἀράξην δὲ   
τὸν ἐξ Ἀρμενίων ῥέοντα ἐς ταύτην ἐσβάλλειν ὁ πλείων 
λόγος κατέχει. 



Flavius Arrianus Hist., Phil., Alexandri anabasis 
Book 7, chapter 18, section 1, line 5

Ἐπεὶ καὶ τοῖόνδε τινὰ λόγον Ἀριστόβουλος ἀνα-
γέγραφεν, Ἀπολλόδωρον τὸν Ἀμφιπολίτην τῶν ἑταίρων 
τῶν Ἀλεξάνδρου, στρατηγὸν τῆς στρατιᾶς ἣν παρὰ 
Μαζαίῳ τῷ Βαβυλῶνος σατράπῃ ἀπέλιπεν Ἀλέξανδρος, 
ἐπειδὴ συνέμιξεν ἐπανιόντι αὐτῷ ἐξ Ἰνδῶν, ὁρῶντα 
πικρῶς τιμωρούμενον τοὺς σατράπας ὅσοι ἐπ' ἄλλῃ καὶ 
ἄλλῃ χώρᾳ τεταγμένοι ἦσαν, ἐπιστεῖλαι Πειθαγόρᾳ τῷ 
ἀδελφῷ, μάντιν γὰρ εἶναι τὸν Πειθαγόραν τῆς ἀπὸ 
σπλάγχνων μαντείας, μαντεύσασθαι καὶ ὑπὲρ αὐτοῦ 
τῆς σωτηρίας. 



Flavius Arrianus Hist., Phil., Alexandri anabasis 
Book 7, chapter 18, section 6, line 1

καὶ μὲν δὴ καὶ ὑπὲρ Καλάνου τοῦ σοφιστοῦ τοῦ Ἰνδοῦ 
τοῖόσδε τις ἀναγέγραπται λόγος, ὁπότε ἐπὶ τὴν πυρὰν 
ᾔει ἀποθανούμενος, τότε τοὺς μὲν ἄλλους ἑταίρους 
ἀσπάζεσθαι αὐτόν, Ἀλεξάνδρῳ δὲ οὐκ ἐθελῆσαι προς-
ελθεῖν ἀσπασόμενον, ἀλλὰ φάναι γὰρ ὅτι ἐν Βαβυλῶνι 
αὐτῷ ἐντυχὼν ἀσπάσεται. 



Flavius Arrianus Hist., Phil., Alexandri anabasis 
Book 7, chapter 19, section 1, line 5

Παρελθόντι δ' αὐτῷ ἐς Βαβυλῶνα πρεσβεῖαι παρὰ 
τῶν Ἑλλήνων ἐνέτυχον, ὑπὲρ ὅτων μὲν ἕκαστοι πρες-
βευόμενοι οὐκ ἀναγέγραπται, δοκεῖν δ' ἔμοιγε αἱ 
πολλαὶ στεφανούντων τε αὐτὸν ἦσαν καὶ ἐπαινούντων 
ἐπὶ ταῖς νίκαις ταῖς τε ἄλλαις καὶ μάλιστα ταῖς Ἰνδι-
καῖς, καὶ ὅτι σῶος ἐξ Ἰνδῶν ἐπανήκει χαίρειν φα-
σκόντων. 



Flavius Arrianus Hist., Phil., Alexandri anabasis 
Book 7, chapter 20, section 1, line 6

Λόγος δὲ κατέχει ὅτι ἤκουεν Ἄραβας δύο μόνον 
τιμᾶν θεούς, τὸν Οὐρανόν τε καὶ τὸν Διόνυσον, τὸν 
μὲν Οὐρανὸν αὐτόν τε ὁρώμενον καὶ τὰ ἄστρα ἐν οἷ 
ἔχοντα τά τε ἄλλα καὶ τὸν ἥλιον, ἀφ' ὅτου μεγίστη 
καὶ φανοτάτη ὠφέλεια ἐς πάντα ἥκει τὰ ἀνθρώπεια, 
Διόνυσον δὲ κατὰ δόξαν τῆς ἐς Ἰνδοὺς στρατιᾶς. 



Flavius Arrianus Hist., Phil., Alexandri anabasis 
Book 7, chapter 20, section 1, line 10

οὔκουν ἀπαξιοῦν καὶ αὐτὸν τρίτον ἂν νομισθῆναι 
πρὸς Ἀράβων θεόν, οὐ φαυλότερα ἔργα Διονύσου 
ἀποδειξάμενον, εἴπερ οὖν καὶ Ἀράβων κρατήσας ἐπι-
τρέψειεν αὐτοῖς, καθάπερ Ἰνδοῖς, πολιτεύειν κατὰ τὰ 
σφῶν νόμιμα. 



Flavius Arrianus Hist., Phil., Alexandri anabasis 
Book 7, chapter 20, section 2, line 7

               τῆς τε χώρας ἡ εὐδαιμονία ὑπεκίνει 
αὐτόν, ὅτι ἤκουεν ἐκ μὲν τῶν λιμνῶν τὴν κασίαν 
γίγνεσθαι αὐτοῖς, ἀπὸ δὲ τῶν δένδρων τὴν σμύρναν 
τε καὶ τὸν λιβανωτόν, ἐκ δὲ τῶν θάμνων τὸ κιννάμω-
μον τέμνεσθαι, οἱ λειμῶνες δὲ ὅτι νάρδον αὐτόματοι   
ἐκφέρουσι· τό <τε> μέγεθος τῆς χώρας, ὅτι οὐκ ἐλάτ-
των ἡ παράλιος τῆς Ἀραβίας ἤπερ ἡ τῆς Ἰνδικῆς αὐτῷ 
ἐξηγγέλλετο, καὶ νῆσοι αὐτῇ προσκεῖσθαι πολλαί, καὶ 
λιμένες πανταχοῦ τῆς χώρας ἐνεῖναι, οἷοι παρασχεῖν 
μὲν ὅρμους τῷ ναυτικῷ, παρασχεῖν δὲ καὶ πόλεις 
ἐνοικισθῆναι καὶ ταύτας γενέσθαι εὐδαίμονας. 



Flavius Arrianus Hist., Phil., Alexandri anabasis 
Book 7, chapter 20, section 8, line 8

                                          ἦν μὲν γὰρ 
αὐτῷ προστεταγμένον περιπλεῦσαι τὴν χερρόνησον τὴν 
Ἀράβων πᾶσαν ἔστε ἐπὶ τὸν κόλπον τὸν πρὸς Αἰγύπτῳ 
τὸν Ἀράβιον τὸν καθ' Ἡρώων πόλιν· οὐ μὴν ἐτόλμησέ 
γε τὸ πρόσω ἐλθεῖν, καίτοι ἐπὶ τὸ πολὺ παραπλεύσας 
τὴν Ἀράβων γῆν· ἀλλ' ἀναστρέψας γὰρ παρ' Ἀλέξαν-
δρον ἐξήγγειλεν τὸ μέγεθός τε τῆς χερρονήσου θαυ-
μαστόν τι εἶναι καὶ ὅσον οὐ πολὺ ἀποδέον τῆς Ἰνδῶν 
γῆς, ἄκραν τε ἀνέχειν ἐπὶ πολὺ τῆς μεγάλης θαλάσσης· 
ἣν δὴ καὶ τοὺς σὺν Νεάρχῳ ἀπὸ τῆς Ἰνδικῆς πλέοντας, 
πρὶν ἐπικάμψαι ἐς τὸν κόλπον τὸν Περσικόν, οὐ πόρρω 
ἀνατείνουσαν ἰδεῖν τε καὶ παρ' ὀλίγον ἐλθεῖν διαβαλεῖν   
ἐς αὐτήν, καὶ Ὀνησικρίτῳ τῷ κυβερνήτῃ ταύτῃ δοκοῦν· 
ἀλλὰ Νέαρχος λέγει ὅτι αὐτὸς διεκώλυσεν, ὡς ἐκπερι-
πλεύσας τὸν κόλπον τὸν Περσικὸν ἔχοι ἀπαγγεῖλαι 
Ἀλεξάνδρῳ ἐφ' οἷστισι πρὸς αὐτοῦ ἐστάλη· οὐ γὰρ 
ἐπὶ τῷ πλεῦσαι τὴν μεγάλην θάλασσαν ἐστάλθαι, ἀλλ' 




Flavius Arrianus Hist., Phil., Historia Indica (0074: 002)
“Flavii Arriani quae exstant omnia, vol. 2”, Ed. Roos, A.G., Wirth, G.
Leipzig: Teubner, 1968 (1st edn. corr.).
Chapter t, section 1, line 1

                                                          ἐπεὶ 
καὶ αὐτὸς ἐμεμψάμην ἔστιν ἃ ἐν τῇ ξυγγραφῇ τῶν 
Ἀλεξάνδρου ἔργων, ἀλλὰ αὐτόν γε Ἀλέξανδρον οὐκ 
αἰσχύνομαι θαυμάζων· τὰ δὲ ἔργα ἐκεῖνα ἐκάκισα 
ἀληθείας τε ἕνεκα τῆς ἐμῆς καὶ ἅμα ὠφελείας τῆς ἐς 
ἀνθρώπους· ἐφ' ὅτῳ ὡρμήθην οὐδὲ αὐτὸς ἄνευ θεοῦ 
ἐς τήνδε τὴν ξυγγραφήν.   
ΙΝΔ*ιΚΗ


 Τὰ ἔξω Ἰνδοῦ ποταμοῦ τὰ πρὸς ἑσπέρην ἔστε ἐπὶ πο-
ταμὸν Κωφῆνα Ἀστακηνοὶ καὶ Ἀσσακηνοί, ἔθνεα Ἰνδικά, 
ἐποικέουσιν, ἀλλ' οὔτε μεγάλοι τὰ σώματα, καθάπερ οἱ 
ἐντὸς τοῦ Ἰνδοῦ ᾠκισμένοι, οὔτε ἀγαθοὶ ὡσαύτως τὸν 
θυμὸν οὐδὲ μέλανες ὡσαύτως τοῖς πολλοῖς Ἰνδοῖσιν. 



Flavius Arrianus Hist., Phil., Historia Indica 
Chapter 1, section 1, line 1

ΙΝΔ*ιΚΗ


 Τὰ ἔξω Ἰνδοῦ ποταμοῦ τὰ πρὸς ἑσπέρην ἔστε ἐπὶ πο-
ταμὸν Κωφῆνα Ἀστακηνοὶ καὶ Ἀσσακηνοί, ἔθνεα Ἰνδικά, 
ἐποικέουσιν, ἀλλ' οὔτε μεγάλοι τὰ σώματα, καθάπερ οἱ 
ἐντὸς τοῦ Ἰνδοῦ ᾠκισμένοι, οὔτε ἀγαθοὶ ὡσαύτως τὸν 
θυμὸν οὐδὲ μέλανες ὡσαύτως τοῖς πολλοῖς Ἰνδοῖσιν. 



Flavius Arrianus Hist., Phil., Historia Indica 
Chapter 1, section 4, line 1

                 Νυσαῖοι δὲ οὐκ Ἰνδικὸν γένος ἐστίν, ἀλλὰ 
τῶν ἅμα Διονύσῳ ἐλθόντων ἐς τὴν γῆν τὴν Ἰνδῶν, 
τυχὸν μὲν [καὶ] Ἑλλήνων, ὅσοι ἀπόμαχοι αὐτῶν ἐγέ-
νοντο ἐν τοῖς πολέμοις οὕστινας πρὸς Ἰνδοὺς Διόνυσος 
ἐπολέμησε, τυχὸν δὲ καὶ τῶν ἐπιχωρίων τοὺς ἐθέλοντας   
τοῖς Ἕλλησι συνῴκισε, τήν τε χώρην Νυσαίην ὠνόμασεν 
ἀπὸ τῆς τροφοῦ τῆς Νύσης Διόνυσος καὶ τὴν πόλιν 
αὐτὴν Νῦσαν. 



Flavius Arrianus Hist., Phil., Historia Indica 
Chapter 1, section 8, line 4

                                                       ταῦτα μὲν 
οἱ ποιηταὶ ἐπὶ Διονύσῳ ἐποίησαν, καὶ ἐξηγείσθων αὐτὰ 
ὅσοι λόγιοι Ἑλλήνων ἢ βαρβάρων· ἐν Ἀσσακηνοῖσι δὲ 
Μάσσακα, πόλις μεγάλη, ἵναπερ καὶ τὸ κράτος τῆς γῆς 
ἐστι τῆς Ἀσσακίης· καὶ ἄλλη πόλις Πευκελαῗτις, μεγάλη 
καὶ αὐτή, οὐ μακρὰν τοῦ Ἰνδοῦ. 



Flavius Arrianus Hist., Phil., Historia Indica 
Chapter 1, section 8, line 5

                                        ταῦτα μὲν ἔξω τοῦ 
Ἰνδοῦ ποταμοῦ ᾤκισται πρὸς ἑσπέρην ἔστε ἐπὶ τὸν 
Κωφῆνα· τὰ δὲ ἀπὸ τοῦ Ἰνδοῦ πρὸς ἕω, τοῦτό μοι ἔστω 
ἡ Ἰνδῶν γῆ καὶ Ἰνδοὶ οὗτοι ἔστωσαν. 



Flavius Arrianus Hist., Phil., Historia Indica 
Chapter 2, section 1, line 3

ὅροι δὲ τῆς Ἰνδῶν γῆς πρὸς μὲν βορέου ἀνέμου ὁ 
Ταῦρος τὸ ὄρος. 



Flavius Arrianus Hist., Phil., Historia Indica 
Chapter 2, section 5, line 2

                                                         τὰ 
πρὸς ἑσπέρην δὲ τῆς Ἰνδῶν γῆς ὁ ποταμὸς ὁ Ἰνδὸς 
ἀπείργει ἔστε ἐπὶ τὴν μεγάλην θάλασσαν, ἵναπερ αὐτὸς 
κατὰ δύο στόματα ἐκδιδοῖ, οὐ συνεχέα ἀλλήλοισι τὰ 
στόματα, κατάπερ τὰ πέντε τοῦ Ἴστρου ἐστὶ συνεχέα, 
ἀλλ' ὡς τὰ τοῦ Νείλου, ὑπ' ὅτων τὸ Δέλτα ποιέεται τὸ 
Αἰγύπτιον, ὧδέ τι καὶ τῆς Ἰνδῶν γῆς Δέλτα ποιέει ὁ 
Ἰνδὸς ποταμός, οὐ μεῖον τοῦ Αἰγυπτίου, καὶ τοῦτο 
Πάταλα τῇ Ἰνδῶν γλώσσῃ καλέεται. 



Flavius Arrianus Hist., Phil., Historia Indica 
Chapter 2, section 7, line 3

                                           τὸ δὲ πρὸς νότου 
τε ἀνέμου καὶ μεσαμβρίης αὐτὴ ἡ μεγάλη θάλασσα 
ἀπείργει τὴν Ἰνδῶν γῆν, καὶ τὰ πρὸς ἕω ἡ αὐτὴ θά-
λασσα ἀπείργει. 



Flavius Arrianus Hist., Phil., Historia Indica 
Chapter 2, section 8, line 2

                  τὰ μὲν πρὸς μεσημβρίης κατὰ Πάταλά 
τε καὶ τοῦ Ἰνδοῦ τὰς ἐκβολὰς ὤφθη πρός τε Ἀλεξάνδρου 
καὶ Μακεδόνων καὶ πολλῶν Ἑλλήνων· τὰ δὲ πρὸς ἕω   
Ἀλέξανδρος μὲν οὐκ ἐπῆλθε τὰ [δὲ] πρόσω ποταμοῦ 
Ὑφάσιος, ὀλίγοι δὲ ἀνέγραψαν τὰ μέχρι ποταμοῦ Γάγ-
γεω καὶ ἵνα τοῦ Γάγγεω αἱ ἐκβολαὶ καὶ πόλις Παλίμ-
βοθρα μεγίστη Ἰνδῶν πρὸς τῶν Γάγγῃ. 



Flavius Arrianus Hist., Phil., Historia Indica 
Chapter 3, section 2, line 2

                                         οὗτος ἀπὸ τοῦ ὄρεος 
τοῦ Ταύρου, ἵνα τοῦ Ἰνδοῦ αἱ πηγαί, παρ' αὐτὸν <τὸν]2 
Ἰνδὸν ποταμὸν ἰόντι ἔστε ἐπὶ τὴν μεγάλην θάλασσαν 
καὶ τοῦ Ἰνδοῦ τὰς ἐκβολὰς μυρίους σταδίους καὶ τρισχι-
λίους τὴν πλευρὴν λέγει ἐπέχειν τῆς γῆς τῆς Ἰνδῶν. 



Flavius Arrianus Hist., Phil., Historia Indica 
Chapter 3, section 3, line 6

                                                               ταυ-
τησὶ δὲ ἀντίπορον πλευρὴν ποιέει τὴν ἀπὸ τοῦ αὐτοῦ 
ὄρεος παρὰ τὴν ἑῴην θάλασσαν, οὐκέτι ταύτῃ τῇ πλευρῇ 
ἴσην, ἀλλὰ ἄκρην γὰρ ἀνέχειν ἐπὶ μέγα εἴσω εἰς τὸ πέ-
λαγος, ἐς τρισχιλίους σταδίους μάλιστα ἀνατείνουσαν τὴν 
ἄκρην· εἴη ἂν ὦν αὐτῷ ἡ πλευρὴ τῆς Ἰνδῶν γῆς <ἡ> 
πρὸς ἕω μυρίους καὶ ἑξακισχιλίους σταδίους ἐπέχουσα. 



Flavius Arrianus Hist., Phil., Historia Indica 
Chapter 3, section 4, line 1

τοῦτο μὲν αὐτῷ πλάτος τῆς Ἰνδῶν γῆς συμβαίνει, μῆκος 
δὲ τὸ ἀπ' ἑσπέρης ἐπὶ ἕω ἔστε μὲν ἐπὶ πόλιν Παλίμ-
βοθρα μεμετρημένον σχοίνοισι λέγει ἀναγράφειν καὶ 
– εἶναι γὰρ ὁδὸν βασιληίην – τοῦτο ἐπέχειν ἐς μυ-  
ρίους σταδίους· τὰ δὲ ἐπέκεινα οὐκέτι ὡσαύτως ἀτρεκέα· 
φήμας δὲ ὅσοι ἀνέγραψαν, ξὺν τῇ ἄκρῃ τῇ ἀνεχούσῃ ἐς 
τὸ πέλαγος ἐς μυρίους σταδίους μάλιστα ἐπέχειν λέγου-
σιν· εἶναι δὲ ἂν ὦν τὸ μῆκος τῆς Ἰνδῶν γῆς σταδίων 
μάλιστα δισμυρίων. 



Flavius Arrianus Hist., Phil., Historia Indica 
Chapter 3, section 6, line 1

                     <Κτησίης> δὲ ὁ Κνίδιος τὴν Ἰνδῶν 
γῆν ἴσην τῇ ἄλλῃ Ἀσίῃ λέγει, οὐδὲν λέγων, οὐδὲ <Ὀνη-
σίκριτος>, τρίτην μοῖραν τῆς πάσης γῆς. 



Flavius Arrianus Hist., Phil., Historia Indica 
Chapter 3, section 7, line 1

                                                <Νέαρχος> δὲ 
μηνῶν τεσσάρων ὁδὸν τὴν δι' αὐτοῦ τοῦ πεδίου τῆς 
Ἰνδῶν γῆς. 



Flavius Arrianus Hist., Phil., Historia Indica 
Chapter 3, section 7, line 2

               <Μεγασθένει> δὲ τὸ ἀπὸ ἀνατολῶν ἐς ἑσπέ-
ρην πλάτος ἐστὶ τῆς Ἰνδῶν γῆς ὅ τι περ οἱ ἄλλοι μῆκος 
ποιέουσι· καὶ λέγει <Μεγασθένης> μυρίων καὶ ἑξακισχι-
λίων σταδίων εἶναι ἵναπερ τὸ βραχύτατον αὐτοῦ. 



Flavius Arrianus Hist., Phil., Historia Indica 
Chapter 3, section 9, line 1

ποταμοὶ δὲ τοσοίδε εἰσὶν ἐν τῇ Ἰνδῶν γῇ ὅσοι οὐδὲ 
ἐν τῇ πάσῃ Ἀσίῃ. 



Flavius Arrianus Hist., Phil., Historia Indica 
Chapter 3, section 9, line 3

                          μέγιστοι μὲν ὁ Γάγγης τε καὶ ὁ Ἰν-
δός, ὅτου καὶ ἡ γῆ ἐπώνυμος, ἄμφω τοῦ τε Νείλου τοῦ 
Αἰγυπτίου καὶ τοῦ Ἴστρου τοῦ Σκυθικοῦ, καὶ εἰ ἐς 
ταὐτὸ συνέλθοι αὐτοῖσι τὸ ὕδωρ, μέζονες. 



Flavius Arrianus Hist., Phil., Historia Indica 
Chapter 3, section 10, line 4

                                                  δοκέειν δὲ 
ἔμοιγε, καὶ ὁ Ἀκεσίνης μέζων ἐστὶ τοῦ τε Ἴστρου καὶ τοῦ 
Νείλου, ἵναπερ παραλαβὼν ἅμα τόν τε Ὑδάσπεα καὶ τὸν 
Ὑδραώτεα καὶ τὸν Ὕφασιν ἐμβάλλει ἐς τὸν Ἰνδόν, ὡς 
καὶ τριάκοντα αὐτῷ στάδια τὸ πλάτος ταύτῃ εἶναι· καὶ   
τυχὸν καὶ ἄλλοι πολλοὶ μέζονες ποταμοὶ ἐν τῇ Ἰνδῶν γῇ 
ῥέουσιν. 



Flavius Arrianus Hist., Phil., Historia Indica 
Chapter 4, section 2, line 2

               αὐτοῖν δὲ τοῖν μεγίστοιν ποταμοῖν τοῦ τε 
Γάγγεω καὶ τοῦ Ἰνδοῦ τὸν Γάγγεα μεγέθει πολύ τι 
ὑπερφέρειν <Μεγασθένης> ἀνέγραψε, καὶ ὅσοι ἄλλοι 
μνήμην τοῦ Γάγγεω ἔχουσιν· αὐτόν τε γὰρ μέγαν ἀνίς-
χειν ἐκ τῶν πηγέων, δέχεσθαί τε ἐς ἑωυτὸν τόν τε 
Καϊνὰν ποταμὸν καὶ τὸν Ἐραννοβόαν καὶ τὸν Κοσσό-
ανον, πάντας πλωτούς, ἔτι δὲ Σῶνόν τε ποταμὸν καὶ 
Σιττόκατιν καὶ Σολόματιν, καὶ τούτους πλωτούς, ἐπὶ 
δὲ Κονδοχάτην τε καὶ Σάμβον καὶ Μάγωνα καὶ Ἀγό-
ρανιν καὶ Ὤμαλιν. 



Flavius Arrianus Hist., Phil., Historia Indica 
Chapter 4, section 5, line 1

                      ἐμβάλλουσι δὲ ἐς αὐτὸν Κομμινά-
σης τε μέγας ποταμὸς καὶ Κάκουθις καὶ Ἀνδώματις ἐξ 
ἔθνεος Ἰνδικοῦ τοῦ Μαδυανδινῶν ῥέων, καὶ ἐπὶ τού-
τοισιν Ἄμυστις παρὰ πόλιν Καταδούπην, καὶ Ὀξύμαγις 
ἐπὶ <τοῖσι> Παζάλαις καλουμένοισι· καὶ Ἐρέννεσις ἐν 
Μάθαις, ἔθνει Ἰνδικῷ, συμβάλλει τῷ Γάγγῃ. 



Flavius Arrianus Hist., Phil., Historia Indica 
Chapter 4, section 8, line 1

                                              τῷ δὲ Ἰνδῷ ἐς 
ταὐτὸν ἔρχεται. 



Flavius Arrianus Hist., Phil., Historia Indica 
Chapter 4, section 10, line 2

                     ὁ δὲ Ἀκεσίνης ἐν Μαλλοῖς ξυμβάλλει 
τῷ Ἰνδῷ· καὶ Τούταπος δὲ μέγας ποταμὸς ἐς τὸν Ἀκε-
σίνην ἐκδιδοῖ. 



Flavius Arrianus Hist., Phil., Historia Indica 
Chapter 4, section 11, line 1

                  τούτων ὁ Ἀκεσίνης ἐμπλησθεὶς καὶ τῇ 
ἐπικλήσει ἐκνικήσας αὐτὸς τῷ ἑωυτοῦ ἤδη ὀνόματι ἐς-
βάλλει ἐς τὸν Ἰνδόν. 



Flavius Arrianus Hist., Phil., Historia Indica 
Chapter 4, section 12, line 1

                           Κωφὴν δὲ ἐν Πευκελαΐτιδι, ἅμα 
οἷ ἄγων Μαλάμαντόν τε καὶ Σόαστον καὶ Γαροίαν, ἐκ-
διδοῖ ἐς τὸν Ἰνδόν. 



Flavius Arrianus Hist., Phil., Historia Indica 
Chapter 4, section 12, line 3

                          καθύπερθε δὲ τουτέων Πάρεννος 
καὶ Σάπαρνος, οὐ πολὺ διέχοντες, ἐμβάλλουσιν ἐς τὸν 
Ἰνδόν. 



Flavius Arrianus Hist., Phil., Historia Indica 
Chapter 4, section 13, line 2

                                                         οὔκουν 
ἀπιστίαν χρὴ ἔχειν ὑπέρ τε τοῦ Ἰνδοῦ καὶ τοῦ Γάγγεω μηδὲ 
συμβλητοὺς εἶναι αὐτοῖσι τόν τε Ἴστρον καὶ τοῦ Νείλου 
τὸ ὕδωρ. 



Flavius Arrianus Hist., Phil., Historia Indica 
Chapter 4, section 15, line 3

            ἐς μέν γε τὸν Νεῖλον οὐδένα ποταμὸν ἐκδι-
δόντα ἴσμεν, ἀλλ' ἀπ' αὐτοῦ διώρυχας τετμημένας κατὰ 
τὴν χώρην τὴν Αἰγυπτίην· ὁ δὲ Ἴστρος ὀλίγος μὲν 
ἀνίσχει ἀπὸ τῶν πηγέων, δέχεται δὲ πολλοὺς ποταμούς, 
ἀλλὰ οὔτε πλήθει ἴσους τοῖς Ἰνδῶν ποταμοῖσιν, οἳ 
ἐς τὸν Ἰνδὸν καὶ τὸν Γάγγην ἐκδιδοῦσιν, πλωτοὺς δὲ 
δὴ καὶ κάρτα ὀλίγους, ὧν τοὺς μὲν αὐτὸς ἰδὼν οἶδα, τὸν 
Ἔνον τε καὶ τὸν Σάον. 



Flavius Arrianus Hist., Phil., Historia Indica 
Chapter 5, section 1, line 2

τὸ δὲ αἴτιον ὅστις ἐθέλει φράζειν τοῦ πλήθεός τε 
καὶ μεγέθεος τῶν Ἰνδῶν ποταμῶν, φραζέτω· ἐμοὶ δὲ καὶ 
ταῦτα ὡς ἀκοὴ ἀναγεγράφθω. 



Flavius Arrianus Hist., Phil., Historia Indica 
Chapter 5, section 2, line 3

                                 ἐπεὶ καὶ ἄλλων πολλῶν 
ποταμῶν οὐνόματα <Μεγασθένης> ἀνέγραψεν, οἳ ἔξω 
τοῦ Γάγγεώ τε καὶ τοῦ Ἰνδοῦ ἐκδιδοῦσιν ἐς τὸν ἑῷόν 
τε καὶ μεσημβρινὸν τὸν ἔξω πόντον, ὥστε τοὺς πάντας 
ὀκτὼ καὶ πεντήκοντα λέγει ὅτι εἰσὶν Ἰνδοὶ ποταμοί,   
ναυσίποροι πάντες. 



Flavius Arrianus Hist., Phil., Historia Indica 
Chapter 5, section 3, line 2

                     ἀλλ' οὐδὲ <Μεγασθένης> πολλὴν 
δοκέει μοι ἐπελθεῖν τῆς Ἰνδῶν χώρης, πλήν γε <δὴ> ὅτι 
πλεῦνα ἢ οἱ ξὺν Ἀλεξάνδρῳ τῷ Φιλίππου ἐπελθόντες· 
συγγενέσθαι γὰρ Σανδροκόττῳ λέγει, τῷ μεγίστῳ βασιλεῖ 
Ἰνδῶν, καὶ Πώρου ἔτι τούτῳ μείζονι. 



Flavius Arrianus Hist., Phil., Historia Indica 
Chapter 5, section 4, line 2

                                               οὗτος ὦν ὁ <Μεγα-
σθένης> λέγει, οὔτε Ἰνδοὺς ἐπιστρατεῦσαι οὐδαμοῖσιν ἀν-
θρώποισιν, οὔτε Ἰνδοῖσιν ἄλλους ἀνθρώπους, ἀλλὰ 
Σέσωστριν μὲν τὸν Αἰγύπτιον, τῆς Ἀσίας καταστρεψά-
μενον τὴν πολλήν, ἔστε ἐπὶ τὴν Εὐρώπην σὺν στρατιῇ 
ἐλάσαντα ὀπίσω ἀπονοστῆσαι, Ἰδάνθυρσον δὲ τὸν Σκύ-
θεα ἐκ Σκυθίης ὁρμηθέντα πολλὰ μὲν τῆς Ἀσίης ἔθνεα 
καταστρέψασθαι, ἐπελθεῖν δὲ καὶ τὴν Αἰγυπτίων γῆν 
κρατέοντα. 



Flavius Arrianus Hist., Phil., Historia Indica 
Chapter 5, section 7, line 2

            Σεμίραμιν δὲ τὴν Ἀσσυρίην ἐπιχειρέειν μὲν 
στέλλεσθαι εἰς Ἰνδούς, ἀποθανεῖν δὲ πρὶν τέλος ἐπιθεῖναι 
τοῖς βουλεύμασιν. 



Flavius Arrianus Hist., Phil., Historia Indica 
Chapter 5, section 8, line 1

                    ἀλλὰ Ἀλέξανδρον γὰρ στρατεῦσαι ἐπ' 
Ἰνδοὺς μοῦνον. 



Flavius Arrianus Hist., Phil., Historia Indica 
Chapter 5, section 8, line 3

                   καὶ πρὸ Ἀλεξάνδρου Διονύσου μὲν πέρι 
πολλὸς λόγος κατέχει ὡς καὶ τούτου στρατεύσαντος ἐς   
Ἰνδοὺς καὶ καταστρεψαμένου Ἰνδούς, Ἡρακλέος δὲ πέρι 
οὐ πολλός. 



Flavius Arrianus Hist., Phil., Historia Indica 
Chapter 5, section 9, line 3

             Διονύσου μέν γε καὶ Νῦσα πόλις μνῆμα οὐ 
φαῦλον τῆς στρατηλασίης, καὶ ὁ Μηρὸς τὸ ὄρος, καὶ ὁ 
κισσὸς ὅτι ἐν τῷ ὄρει τούτῳ φύεται, καὶ αὐτοὶ οἱ Ἰνδοὶ 
ὑπὸ τυμπάνων τε καὶ κυμβάλων στελλόμενοι ἐς τὰς 
μάχας, καὶ ἐσθὴς αὐτοῖσι κατάστικτος ἐοῦσα, κατάπερ 
τοῦ Διονύσου τοῖσι βάκχοισιν· Ἡρακλέος δὲ οὐ πολλὰ 
ὑπομνήματα. 



Flavius Arrianus Hist., Phil., Historia Indica 
Chapter 5, section 12, line 1

καὶ δὴ καὶ ἐν Σίβαισιν, Ἰνδικῷ γένει, ὅτι δορὰς ἀμπε-
χομένους εἶδον τοὺς Σίβας, ἀπὸ τῆς Ἡρακλέους στρα-
τηλασίης ἔφασκον τοὺς ὑπολειφθέντας εἶναι τοὺς Σίβας· 
καὶ γὰρ καὶ σκυτάλην φορέουσί τε οἱ Σίβαι καὶ τῇσι 
βουσὶν αὐτῶν ῥόπαλον ἐπικέκαυται, καὶ τοῦτο ἐς μνήμην 
ἀνέφερον τοῦ ῥοπάλου τοῦ Ἡρακλέους. 



Flavius Arrianus Hist., Phil., Historia Indica 
Chapter 5, section 13, line 4

                                             εἰ δέ τῳ πιστὰ 
ταῦτα, ἄλλος ἂν οὗτος Ἡρακλέης εἴη, οὐχ ὁ Θηβαῖος ἢ   
ὁ Τύριος [οὗτος] ἢ ὁ Αἰγύπτιος, ἤ τις καὶ κατὰ τὴν 
ἄνω χώρην οὐ πόρρω τῆς Ἰνδῶν γῆς ᾠκισμένος μέγας 
βασιλεύς. 



Flavius Arrianus Hist., Phil., Historia Indica 
Chapter 6, section 1, line 3

ταῦτα μέν μοι ἐκβολὴ ἔστω τοῦ λόγου ἐς τὸ μὴ πι-
στὰ φαίνεσθαι ὅσα ὑπὲρ τῶν ἐπέκεινα τοῦ Ὑφάσιος πο-
ταμοῦ Ἰνδῶν μετεξέτεροι ἀνέγραψαν· (ἔστε γὰρ ἐπὶ τὸν 
Ὕφασιν οἱ τῆς Ἀλεξάνδρου στρατηλασίης μετασχόντες 
οὐ πάντη ἄπιστοί εἰσιν)· ἐπεὶ καὶ τόδε λέγει <Μεγα-
σθένης> ὑπὲρ ποταμοῦ Ἰνδικοῦ, Σίλαν μὲν εἶναί οἱ 
ὄνομα, ῥέειν δὲ ἀπὸ κρήνης ἐπωνύμου τοῦ ποταμοῦ διὰ 
τῆς χώρης τῆς Σιλαίων, καὶ τούτων ἐπωνύμων τοῦ πο-
ταμοῦ τε καὶ τῆς κρήνης, τὸ δὲ ὕδωρ παρέχεσθαι τοι-
όνδε. 



Flavius Arrianus Hist., Phil., Historia Indica 
Chapter 6, section 4, line 1

ὕεται δὲ ἡ Ἰνδῶν γῆ τοῦ θέρεος, μάλιστα μὲν τὰ ὄρεα, 
Παραπάμισός τε καὶ ὁ Ἠμωδὸς καὶ τὸ Ἰμαϊκὸν ὄρος, 
καὶ ἀπὸ τουτέων μεγάλοι καὶ θολεροὶ οἱ ποταμοὶ ῥέουσιν. 



Flavius Arrianus Hist., Phil., Historia Indica 
Chapter 6, section 5, line 1

ὕεται δὲ τοῦ θέρους καὶ τὰ πεδία τῶν Ἰνδῶν, ὥστε 
λιμνάζει τὰ πολλὰ αὐτέων. 



Flavius Arrianus Hist., Phil., Historia Indica 
Chapter 6, section 8, line 2

                                                               ὕεσθαι 
δὲ κατάπερ τὰ Ἰνδῶν οὐκ ἔξω ἐστὶ τοῦ εἰκότος, ἐπεὶ 
καὶ τἄλλα <ἡ]2 Ἰνδῶν γῆ οὐκ ἀπέοικε τῆς Αἰθιοπίης καὶ 
οἱ ποταμοὶ οἱ Ἰνδοὶ ὁμοίως τῷ Νείλῳ τῷ Αἰθιοπηίῳ τε 
καὶ Αἰγυπτίῳ κροκοδείλους τε φέρουσιν, ἔστιν δὲ οἳ 
αὐτῶν καὶ ἰχθύας καὶ ἄλλα κήτεα ὅσα ὁ Νεῖλος πλὴν 
ἵππου τοῦ ποταμίου,  – <Ὀνησίκριτος> δὲ καὶ τοὺς 
ἵππους τοὺς ποταμίους λέγει ὅτι φέρουσι – τῶν τε ἀν-
θρώπων αἱ ἰδέαι οὐ πάντη ἀπᾴδουσιν αἱ Ἰνδῶν τε καὶ 
Αἰθιόπων. 



Flavius Arrianus Hist., Phil., Historia Indica 
Chapter 6, section 9, line 3

             οἱ μὲν πρὸς νότου ἀνέμου Ἰνδοὶ τοῖς Αἰθίοψι 
μᾶλλόν τι ἐοίκασι μέλανές τε ἰδέσθαι εἰσὶ καὶ ἡ κόμη 
αὐτοῖς μέλαινα, πλήν γε δὴ ὅτι σιμοὶ οὐχ ὡσαύτως οὐδὲ 
οὐλόκρανοι ὡς Αἰθίοπες. 



Flavius Arrianus Hist., Phil., Historia Indica 
Chapter 7, section 1, line 1

ἔθνεα δὲ Ἰνδικὰ εἴκοσι καὶ ἑκατὸν τὰ ἅπαντα λέγει   
<Μεγασθένης>, δυοῖν δέοντα. 



Flavius Arrianus Hist., Phil., Historia Indica 
Chapter 7, section 1, line 3

                                    καὶ πολλὰ μὲν εἶναι ἔθνεα 
Ἰνδικὰ καὶ αὐτὸς συμφέρομαι <Μεγασθένει>, τὸ δὲ 
ἀτρεκὲς οὐκ ἔχω εἰκάσαι ὅπως ἐκμαθὼν ἀνέγραψεν, οὐδὲ 
πολλοστὸν μέρος τῆς Ἰνδῶν γῆς ἐπελθών, οὐδὲ ἐπιμι-
ξίης πᾶσι τοῖς γένεσιν ἐούσης ἐς ἀλλήλους. 



Flavius Arrianus Hist., Phil., Historia Indica 
Chapter 7, section 2, line 2

                                                    πάλαι μὲν 
δὴ νομάδας εἶναι Ἰνδούς, καθάπερ Σκυθέων τοὺς οὐκ 
ἀροτῆρας, οἳ ἐπὶ τῇσιν ἁμάξῃσι πλανώμενοι ἄλλοτε ἄλ-
λην τῆς Σκυθίης ἀμείβουσιν, οὔτε πόληας οἰκέοντες οὔτε 
ἱερὰ θεῶν σέβοντες. 



Flavius Arrianus Hist., Phil., Historia Indica 
Chapter 7, section 3, line 1

                        οὕτω μηδὲ Ἰνδοῖσι πόληας εἶναι 
μηδὲ ἱερὰ θεῶν δεδομημένα, ἀλλ' ἀμπίσχεσθαι μὲν δο-
ρὰς θηρίων ὅσων κατακάνοιεν, σιτέεσθαι δὲ τῶν δεν-
δρέων τὸν φλοιόν. 



Flavius Arrianus Hist., Phil., Historia Indica 
Chapter 7, section 3, line 5

                     καλέεσθαι δὲ τὰ δένδρεα ταῦτα τῇ 
Ἰνδῶν φωνῆ τάλα, καὶ φύεσθαι ἐπ' αὐτῶν, κατάπερ τῶν 
φοινίκων ἐπὶ τῇσι κορυφῇσιν, οἷά περ τολύπας. 



Flavius Arrianus Hist., Phil., Historia Indica 
Chapter 7, section 5, line 1

                                                         σιτέεσθαι 
δὲ καὶ τῶν θηρίων ὅσα ἕλοιεν ὠμοφαγέοντας, πρίν γε 
δὴ Διόνυσον ἐλθεῖν ἐς τὴν χώρην τῶν Ἰνδῶν. 



Flavius Arrianus Hist., Phil., Historia Indica 
Chapter 7, section 5, line 2

                                                       Διόνυσον 
δὲ ἐλθόντα, ὡς καρτερὸς ἐγένετο Ἰνδῶν, πόληάς τε 
οἰκίσαι καὶ νόμους θέσθαι τῇσι πόλεσιν, οἴνου τε δο-
τῆρα Ἰνδοῖς γενέσθαι κατάπερ Ἕλλησι, καὶ σπείρειν δι-
δάξαι τὴν γῆν διδόντα αὐτὸν σπέρματα, ἢ οὐκ ἐλάσαντος 
ταύτῃ Τριπτολέμου, ὅτε περ ἐκ Δήμητρος ἐστάλη σπεί-
ρειν τὴν γῆν πᾶσαν, ἢ πρὸ Τριπτολέμου τις οὗτος Διό-
νυσος ἐπελθὼν τὴν Ἰνδῶν γῆν σπέρματά σφισιν 
ἔδωκε καρποῦ τοῦ ἡμέρου. 



Flavius Arrianus Hist., Phil., Historia Indica 
Chapter 7, section 7, line 3

                               βόας τε ὑπ' ἄροτρον ζεῦξαι   
Διόνυσον πρῶτον καὶ ἀροτῆρας ἀντὶ νομάδων ποιῆσαι 
Ἰνδῶν τοὺς πολλοὺς καὶ ὁπλίσαι ὅπλοισι τοῖσιν ἀρηίοισι. 



Flavius Arrianus Hist., Phil., Historia Indica 
Chapter 7, section 9, line 1

καὶ θεοὺς σέβειν ὅτι ἐδίδαξε Διόνυσος ἄλλους τε καὶ 
μάλιστα δὴ ἑωυτὸν κυμβαλίζοντας καὶ τυμπανίζοντας· 
καὶ ὄρχησιν δὲ ἐκδιδάξαι τὴν σατυρικήν, τὸν κόρδακα 
παρ' Ἕλλησι καλούμενον, καὶ κομᾶν [Ἰνδοὺς] τῷ θεῷ 
μιτρηφορέειν τε ἀναδεῖξαι καὶ μύρων ἀλοιφὰς ἐκδιδάξαι, 
ὥστε καὶ εἰς Ἀλέξανδρον ἔτι ὑπὸ κυμβάλων τε καὶ 
τυμπάνων ἐς τὰς μάχας Ἰνδοὶ καθίσταντο. 



Flavius Arrianus Hist., Phil., Historia Indica 
Chapter 8, section 1, line 1

ἀπιόντα δὲ ἐκ τῆς Ἰνδῶν γῆς, ὥς οἱ ταῦτα κεκοσμέατο, 
καταστῆσαι βασιλέα τῆς χώρης Σπατέμβαν, τῶν ἑταίρων 
ἕνα τὸν βακχωδέστατον· τελευτήσαντος δὲ Σπατέμβα 
τὴν βασιληίην ἐκδέξασθαι Βουδύαν τὸν τούτου παῖδα. 



Flavius Arrianus Hist., Phil., Historia Indica 
Chapter 8, section 2, line 1

καὶ τὸν μὲν πεντήκοντα καὶ δύο ἔτεα βασιλεῦσαι Ἰνδῶν, 
τὸν πατέρα, τὸν δὲ παῖδα εἴκοσιν ἔτεα. 



Flavius Arrianus Hist., Phil., Historia Indica 
Chapter 8, section 3, line 4

                                                καὶ τούτου 
παῖδα ἐκδέξασθαι τὴν βασιληίην Κραδεύαν, καὶ τὸ ἀπὸ 
τοῦδε τὸ πολὺ μὲν κατὰ γένος ἀμείβειν τὴν βασιληίην, 
παῖδα παρὰ πατρὸς ἐκδεχόμενον· εἰ δὲ ἐκλείποι τὸ γένος, 
οὕτω δὴ ἀριστίνδην καθίστασθαι Ἰνδοῖσι βασιλέας. 



Flavius Arrianus Hist., Phil., Historia Indica 
Chapter 8, section 4, line 1

Ἡρακλέα δέ, ὅντινα ἐς Ἰνδοὺς ἀφικέσθαι λόγος κατέχει, 
παρ' αὐτοῖσιν Ἰνδοῖσι γηγενέα λέγεσθαι. 



Flavius Arrianus Hist., Phil., Historia Indica 
Chapter 8, section 5, line 3

                                               τοῦτον τὸν 
Ἡρακλέα μάλιστα πρὸς Σουρασηνῶν γεραίρεσθαι, Ἰν-
δικοῦ ἔθνεος, ἵνα δύο πόληες μεγάλαι, Μέθορά τε καὶ 
Κλεισόβορα· καὶ ποταμὸς Ἰωμάνης πλωτὸς διαρρεῖ τὴν   
χώρην αὐτῶν· τὴν σκευὴν δὲ οὗτος ὁ Ἡρακλέης ἥντινα 
ἐφόρεε <Μεγασθένης> λέγει ὅτι ὁμοίην τῷ Θηβαίῳ 
Ἡρακλεῖ, ὡς αὐτοὶ Ἰνδοὶ ἀπηγέονται. 



Flavius Arrianus Hist., Phil., Historia Indica 
Chapter 8, section 6, line 4

                                               καὶ τούτῳ ἄρσε-
νας μὲν παῖδας πολλοὺς κάρτα γενέσθαι ἐν τῇ Ἰνδῶν 
γῇ – πολλῇσι γὰρ δὴ γυναιξὶν ἐς γάμον ἐλθεῖν καὶ 
τοῦτον τὸν Ἡρακλέα – , θυγατέρα δὲ μουνογενέην. 



Flavius Arrianus Hist., Phil., Historia Indica 
Chapter 8, section 8, line 1

                                  καὶ τάδε μετεξέτεροι Ἰνδῶν 
περὶ Ἡρακλέους λέγουσιν, ἐπελθόντα αὐτὸν πᾶσαν γῆν 
καὶ θάλασσαν καὶ καθήραντα ὅ τι περ κακόν, καινὸν 
εἶδος ἐξευρεῖν ἐν τῇ θαλάσσῃ κόσμου γυναικηίου, ὅντινα 
καὶ εἰς τοῦτο ἔτι οἵ τε ἐξ Ἰνδῶν τῆς χώρης τὰ ἀγώγιμα 
παρ' ἡμέας ἀγινέοντες σπουδῇ ὠνεόμενοι ἐκκομίζουσι, 
καὶ Ἑλλήνων δὲ πάλαι καὶ Ῥωμαίων νῦν ὅσοι πολυ-
κτέανοι καὶ εὐδαίμονες μέζονι ἔτι σπουδῆ ὠνέονται, 
τὸν μαργαρίτην δὴ τὸν θαλάσσιον οὕτω τῇ Ἰνδῶν 
γλώσσῃ καλεόμενον. 



Flavius Arrianus Hist., Phil., Historia Indica 
Chapter 8, section 10, line 2

                      τὸν γὰρ Ἡρακλέα, ὡς καλόν οἱ 
ἐφάνη τὸ φόρημα, ἐκ πάσης τῆς θαλάσσης ἐς τὴν Ἰνδῶν 
γῆν συναγινέειν τὸν μαργαρίτην δὴ τοῦτον, τῇ θυγατρὶ   
τῇ ἑωυτοῦ εἶναι κόσμον. 



Flavius Arrianus Hist., Phil., Historia Indica 
Chapter 8, section 13, line 2

                                                     καὶ εἶναι 
γὰρ καὶ παρ' Ἰνδοῖσι τὸν μαργαρίτην τριστάσιον κατὰ 
τιμὴν πρὸς χρυσίον τὸ ἄπεφθον, καὶ τοῦτο ἐν τῇ Ἰνδῶν 
γῇ ὀρυσσόμενον. 



Flavius Arrianus Hist., Phil., Historia Indica 
Chapter 9, section 2, line 2

                          καὶ ὑπὲρ τούτου λεγόμενον 
λόγον εἶναι παρὰ Ἰνδοῖσιν. 



Flavius Arrianus Hist., Phil., Historia Indica 
Chapter 9, section 2, line 6

                                  Ἡρακλέα, ὀψιγόνου οἱ γε-
νομένης τῆς παιδός, ἐπεί τε δὴ ἐγγὺς ἔμαθεν ἑαυτῷ 
ἐοῦσαν τὴν τελευτήν, οὐκ ἔχοντα ὅτῳ ἀνδρὶ ἐκδῷ τὴν 
παῖδα ἑωυτοῦ ἐπαξίῳ, αὐτὸν μιγῆναι τῇ παιδὶ ἑπταέτεϊ 
ἐούσῃ, ὡς γένος ἐξ οὗ τε κἀκείνης ὑπολείπεσθαι Ἰνδῶν 
βασιλέας. 



Flavius Arrianus Hist., Phil., Historia Indica 
Chapter 9, section 9, line 1

ἀπὸ μὲν δὴ Διονύσου βασιλέας ἠρίθμεον Ἰνδοὶ ἐς 
Σανδρόκοττον τρεῖς καὶ πεντήκοντα καὶ ἑκατόν, ἔτεα δὲ 
δύο καὶ τεσσαράκοντα καὶ ἑξακισχίλια· ἐν δὲ τούτοισι 
τρὶς τὸ πᾶν εἰς ἐλευθερίην ***, τὴν δὲ καὶ ἐς τριακό-  
σια, τὴν δὲ εἴκοσίν τε ἐτέων καὶ ἑκατόν. 



Flavius Arrianus Hist., Phil., Historia Indica 
Chapter 9, section 10, line 2

                                                   πρεσβύτερόν 
τε Διόνυσον Ἡρακλέος δέκα καὶ πέντε γενεῇσιν Ἰνδοὶ 
λέγουσιν· ἄλλον δὲ οὐδένα ἐμβαλεῖν ἐς γῆν τὴν Ἰνδῶν 
ἐπὶ πολέμῳ, οὐδὲ Κῦρον τὸν Καμβύσεω, καίτοι ἐπὶ 
Σκύθας ἐλάσαντα καὶ τἄλλα πολυπραγμονέστατον δὴ 
τῶν κατὰ τὴν Ἀσίαν βασιλέων γενόμενον τὸν Κῦρον. 



Flavius Arrianus Hist., Phil., Historia Indica 
Chapter 9, section 12, line 1

                                  οὐ μὲν δὴ οὐδὲ Ἰνδῶν τινὰ 
ἔξω τῆς οἰκείης σταλῆναι ἐπὶ πολέμῳ διὰ δικαιότητα. 



Flavius Arrianus Hist., Phil., Historia Indica 
Chapter 10, section 1, line 1

λέγεται δὲ καὶ τάδε, μνημεῖα ὅτι Ἰνδοὶ τοῖς τελευ-
τήσασιν οὐ ποιέουσιν, ἀλλὰ τὰς ἀρετὰς γὰρ τῶν ἀνδρῶν 
ἱκανὰς ἐς μνήμην τίθενται τοῖσιν ἀποθανοῦσι καὶ τὰς 
ᾠδὰς αἳ αὐτοῖσιν ἐπᾴδονται. 



Flavius Arrianus Hist., Phil., Historia Indica 
Chapter 10, section 2, line 2

                                      πόλεων δὲ καὶ ἀριθμὸν οὐκ 
εἶναι ἂν ἀτρεκὲς ἀναγράψαι τῶν Ἰνδικῶν ὑπὸ πλήθεος· 
ἀλλὰ γὰρ ὅσαι παραποτάμιαι αὐτέων ἢ παραθαλάσσιαι, 
ταύτας μὲν ξυλίνας ποιέεσθαι· οὐ γὰρ ἂν ἐκ πλίνθου 
ποιεομένας διαρκέσαι ἐπὶ χρόνον τοῦ τε ὕδατος ἕνεκα 
τοῦ ἐξ οὐρανοῦ καὶ ὅτι οἱ ποταμοὶ αὐτοῖσιν ὑπερβάλ-
λοντες ὑπὲρ τὰς ὄχθας ἐμπιμπλᾶσι τοῦ ὕδατος τὰ πεδία. 



Flavius Arrianus Hist., Phil., Historia Indica 
Chapter 10, section 5, line 1

                           μεγίστην δὲ πόλιν Ἰνδοῖσιν 
εἶναι <τὴν> Παλίμβοθρα καλεομένην, ἐν τῇ Πρασίων 
γῇ, ἵνα αἱ συμβολαί εἰσι τοῦ τε Ἐραννοβόα ποταμοῦ καὶ   
τοῦ Γάγγεω· τοῦ μὲν Γάγγεω, τοῦ μεγίστου ποταμῶν· 
ὁ δὲ Ἐραννοβόας τρίτος μὲν ἂν εἴη τῶν Ἰνδῶν ποτα-
μῶν, μέζων δὲ τῶν ἄλλῃ καὶ οὗτος, ἀλλὰ ξυγχωρέει 
αὐτὸς τῷ Γάγγῃ, ἐπειδὰν ἐμβάλῃ ἐς αὐτὸν τὸ ὕδωρ. 



Flavius Arrianus Hist., Phil., Historia Indica 
Chapter 10, section 8, line 2

                                       εἶναι δὲ καὶ τόδε 
μέγα ἐν τῇ Ἰνδῶν γῇ, πάντας Ἰνδοὺς εἶναι ἐλευθέρους, 
οὐδέ τινα δοῦλον εἶναι Ἰνδόν. 



Flavius Arrianus Hist., Phil., Historia Indica 
Chapter 10, section 9, line 2

                                      τοῦτο μὲν Λακεδαιμονίοι-
σιν ἐς ταὐτὸ συμβαίνει καὶ Ἰνδοῖσι. 



Flavius Arrianus Hist., Phil., Historia Indica 
Chapter 10, section 9, line 4

                                            Λακεδαιμονίοις μέν 
γε οἱ εἵλωτες δοῦλοί εἰσιν καὶ τὰ δούλων ἐργάζονται, 
Ἰνδοῖσι δὲ οὐδὲ ἄλλος δοῦλός ἐστι, μήτι γε Ἰνδῶν τις. 



Flavius Arrianus Hist., Phil., Historia Indica 
Chapter 11, section 1, line 1

νενέμηνται δὲ οἱ πάντες Ἰνδοὶ ἐς ἑπτὰ μάλιστα γένεα. 



Flavius Arrianus Hist., Phil., Historia Indica 
Chapter 11, section 3, line 1

                                                    οὐδέ τι 
ἄλλο ἀνάγκης ἁπλῶς ἐπεῖναι τοῖς σοφιστῇσιν, ὅτι μὴ 
θύειν τὰς θυσίας τοῖσι θεοῖσιν ὑπὲρ τοῦ κοινοῦ <τῶν]2 
Ἰνδῶν· καὶ ὅστις δὲ ἰδίᾳ θύει, ἐξηγητὴς αὐτῷ τῆς θυ-
σίης τῶν τις σοφιστῶν τούτων γίνεται, ὡς οὐκ ἂν ἄλλως 
κεχαρισμένα τοῖς θεοῖς θύσαντας. 



Flavius Arrianus Hist., Phil., Historia Indica 
Chapter 11, section 4, line 2

                                     εἰσὶ δὲ καὶ μαντικῆς 
οὗτοι μοῦνοι Ἰνδῶν δαήμονες, οὐδὲ ἐφεῖται ἄλλῳ μαν-
τεύεσθαι ὅτι μὴ σοφιστῇ ἀνδρί. 



Flavius Arrianus Hist., Phil., Historia Indica 
Chapter 11, section 9, line 2

δεύτεροι δ' ἐπὶ τούτοισιν οἱ γεωργοί εἰσιν, οὗτοι πλή-
θει πλεῖστοι Ἰνδῶν ἐόντες. 



Flavius Arrianus Hist., Phil., Historia Indica 
Chapter 11, section 10, line 1

       καὶ εἰ πόλεμος ἐς ἀλλήλους τοῖσιν Ἰνδοῖσι τύχοι, 
τῶν ἐργαζομένων τὴν γῆν οὐ θέμις σφιν ἅπτεσθαι οὐδὲ 
αὐτὴν τὴν γῆν τέμνειν, ἀλλὰ οἳ μὲν πολεμοῦσι καὶ κα-
τακαίνουσιν ἀλλήλους ὅπως τύχοιεν, οἳ δὲ πλησίον 
αὐτῶν κατ' ἡσυχίαν ἀροῦσιν ἢ τρυγῶσιν ἢ κλαδῶσιν ἢ 
θερίζουσιν. 



Flavius Arrianus Hist., Phil., Historia Indica 
Chapter 11, section 11, line 1

τρίτοι δέ εἰσιν Ἰνδοῖσιν οἱ νομέες, οἱ ποιμένες τε 
καὶ βουκόλοι. 



Flavius Arrianus Hist., Phil., Historia Indica 
Chapter 12, section 2, line 1

πέμπτον δὲ γένος ἐστὶν Ἰνδοῖσιν οἱ πολεμισταί, 
πλήθει μὲν δεύτερον μετὰ τοὺς γεωργούς, πλείστῃ δὲ 
ἐλευθερίῃ τε καὶ εὐθυμίῃ ἐπιχρεόμενον. 



Flavius Arrianus Hist., Phil., Historia Indica 
Chapter 12, section 5, line 1

ἕκτοι δέ εἰσιν Ἰνδοῖσιν οἱ ἐπίσκοποι καλεόμενοι. 



Flavius Arrianus Hist., Phil., Historia Indica 
Chapter 12, section 5, line 4

                                                             οὗτοι 
ἐφορῶσι τὰ γινόμενα κατά τε τὴν χώρην καὶ κατὰ τὰς 
πόληας, καὶ ταῦτα ἀναγγέλλουσι τῷ βασιλεῖ, ἵναπερ βα-
σιλεύονται Ἰνδοί, ἢ τοῖς τέλεσιν, ἵναπερ αὐτόνομοί εἰσι. 



Flavius Arrianus Hist., Phil., Historia Indica 
Chapter 12, section 5, line 6

καὶ τούτοις οὐ θέμις ψεῦδος ἀγγεῖλαι οὐδέν, οὐδέ τις 
Ἰνδῶν αἰτίην ἔσχε ψεύσασθαι. 



Flavius Arrianus Hist., Phil., Historia Indica 
Chapter 13, section 1, line 1

θηρῶσι δὲ Ἰνδοὶ τὰ μὲν ἄλλα ἄγρια θηρία κατάπερ 
καὶ Ἕλληνες, ἡ δὲ τῶν ἐλεφάντων σφιν θήρα οὐδέν τι 
ἄλλῃ ἔοικεν, ὅτι καὶ ταῦτα τὰ θηρία οὐδαμοῖσιν ἄλλοισι 
θηρίοις ἐπέοικεν. 



Flavius Arrianus Hist., Phil., Historia Indica 
Chapter 14, section 3, line 3

                                                      ἀγ<αγ>όν-
τες δὲ εἰς τὰς κώμας τοὺς ἁλόντας τοῦ τε χλωροῦ καλά-
μου καὶ τῆς πόας τὰ πρῶτα ἐμφαγεῖν ἔδοσαν, οἳ δὲ ὑπὸ 
ἀθυμίης οὐκ ἐθέλουσιν οὐδὲν σιτέεσθαι, τοὺς δὲ περιι-
στάμενοι οἱ Ἰνδοὶ ᾠδαῖσι τε καὶ τυμπάνοισι καὶ κυμβά-
λοισιν ἐν κύκλῳ κρούοντές τε καὶ ἐπᾴδοντες κατευνά-
ζουσι. 



Flavius Arrianus Hist., Phil., Historia Indica 
Chapter 14, section 9, line 5

                           καὶ ἔστιν αὐτοῖσι τῶν μὲν ὀφθαλ-
μῶν ἴαμα τὸ βόειον γάλα ἐγχεόμενον, πρὸς δὲ τὰς ἄλλας 
νόσους ὁ μέλας οἶνος πινόμενος, ἐπὶ δὲ τοῖσιν ἕλκεσι   
τὰ ὕεια κρέα ὀπτώμενα καὶ καταπλασσόμενα· ταῦτα παρ' 
Ἰνδοῖσίν ἐστιν αὐτοῖσι ἰάματα. 



Flavius Arrianus Hist., Phil., Historia Indica 
Chapter 15, section 1, line 2

τοῦ δὲ ἐλέφαντος τὴν τίγριν πολλόν τι ἀλκιμωτέρην 
Ἰνδοὶ ἄγουσι. 



Flavius Arrianus Hist., Phil., Historia Indica 
Chapter 15, section 1, line 3

                   τίγριος δὲ δορὴν μὲν ἰδεῖν λέγει <Νέαρ-
χος>, αὐτὴν δὲ τίγριν οὐκ ἰδεῖν· ἀλλὰ τοὺς Ἰνδοὺς γὰρ 
ἀπηγέεσθαι, τίγριν εἶναι μέγεθος μὲν ἡλίκον τὸν μέγι-
στον ἵππον, τὴν δὲ ὠκύτητα καὶ ἀλκὴν οἵην οὐδενὶ ἄλλῳ 
εἰκάσαι· τίγριν γὰρ ἐπεὰν ὁμοῦ ἔλθῃ ἐλέφαντι, ἐπιπη-
δᾶν τε ἐπὶ τὴν κεφαλὴν τοῦ ἐλέφαντος καὶ ἄγχειν 
εὐπετέως. 



Flavius Arrianus Hist., Phil., Historia Indica 
Chapter 15, section 4, line 3

                      ἐπεὶ καὶ ὑπὲρ τῶν μυρμήκων λέγει 
<Νέαρχος> μύρμηκα μὲν αὐτὸς οὐκ ἰδέειν, ὁποῖον δή τινα 
μετεξέτεροι διέγραψαν γίνεσθαι ἐν τῇ Ἰνδῶν γῇ, δορὰς 
δὲ καὶ τούτων ἰδεῖν πολλὰς ἐς τὸ στρατόπεδον κατακο-
μισθείσας τὸ Μακεδονικόν. 



Flavius Arrianus Hist., Phil., Historia Indica 
Chapter 15, section 7, line 1

                                 ἐκείνους δέ – εἶναι γὰρ 
ἀλωπεκέων μέζονας – πρὸς λόγον τοῦ μεγέθεος σφῶν 
καὶ τὴν γῆν ὀρύσσειν· τὴν δὲ γῆν χρυσῖτιν εἶναι, καὶ 
ἀπὸ ταύτης γίνεσθαι Ἰνδοῖσι τὸν χρυσόν. 



Flavius Arrianus Hist., Phil., Historia Indica 
Chapter 15, section 8, line 2

                       σιττακοὺς δὲ <Νέαρχος> μὲν ὡς 
δή τι θαῦμα ἀπηγέεται ὅτι γίνονται ἐν τῇ Ἰνδῶν γῇ, 
καὶ ὁποῖος ὄρνις ἐστὶν ὁ σιττακός, καὶ ὅπως φωνὴν ἵει 
ἀνθρωπίνην. 



Flavius Arrianus Hist., Phil., Historia Indica 
Chapter 15, section 9, line 4

              ἐγὼ δὲ ὅτι αὐτός τε πολλοὺς ὀπώπεα καὶ 
ἄλλους ἐπισταμένους ᾔδεα τὸν ὄρνιθα, οὐδὲν ὡς <ὑπὲρ> 
ἀτόπου δῆθεν ἀπηγήσομαι· οὐδὲ ὑπὲρ τῶν πιθήκων 
τοῦ μεγέθεος, ἢ ὅτι καλοὶ παρ' Ἰνδοῖς πίθηκοί εἰσιν, 
οὐδὲ ὅπως θηρῶνται ἐρέω. 



Flavius Arrianus Hist., Phil., Historia Indica 
Chapter 15, section 10, line 4

                               αὐτοὺς δὲ τοὺς Ἰνδοὺς πολὺ 
μείζονας τούτων λέγειν εἶναι τοὺς μεγίστους ὄφεας. 



Flavius Arrianus Hist., Phil., Historia Indica 
Chapter 15, section 11, line 2

ὅσοι δὲ ἰητροὶ Ἕλληνες, τούτοισιν οὐδὲν ἄκος ἐξεύρητο   
ὅστις ὑπὸ ὄφεως δηχθείη Ἰνδικοῦ· ἀλλ' αὐτοὶ γὰρ οἱ 
Ἰνδοὶ ἰῶντο τοὺς πληγέντας. 



Flavius Arrianus Hist., Phil., Historia Indica 
Chapter 15, section 11, line 5

                                   καὶ ἐπὶ τῷδε <Νέαρχος> λέγει 
<ὅτι> συλλελεγμένους ἀμφ' αὑτὸν εἶχεν Ἀλέξανδρος 
Ἰνδῶν ὅσοι ἰητρικὴν σοφώτατοι, καὶ κεκήρυκτο ἀνὰ τὸ 
στρατόπεδον, ὅστις δηχθείη, ἐπὶ τὴν σκηνὴν φοιτᾶν τὴν 
βασιλέως. 



Flavius Arrianus Hist., Phil., Historia Indica 
Chapter 15, section 12, line 2

                              οὐ πολλὰ δὲ ἐν Ἰνδοῖσι πάθεα 
γίνεται, ὅτι αἱ ὧραι σύμμετροί εἰσιν αὐτόθι· εἰ δέ τι 
μεῖζον καταλαμβάνοι, τοῖσι σοφιστῇσιν ἀνεκοινοῦντο· 
καὶ ἐκεῖνοι οὐκ ἄνευ θεοῦ ἐδόκεον ἰῆσθαι ὅ τι περ 
ἰήσιμον. 



Flavius Arrianus Hist., Phil., Historia Indica 
Chapter 16, section 1, line 1

ἐσθῆτι δὲ Ἰνδοὶ λινέῃ χρέονται, κατάπερ λέγει <Νέ-
αρχος>, λίνου τοῦ ἀπὸ τῶν δενδρέων, ὑπὲρ ὅτων μοι 
ἤδη λέλεκται. 



Flavius Arrianus Hist., Phil., Historia Indica 
Chapter 16, section 3, line 1

                                 καὶ ἐνώτια Ἰνδοὶ φορέου-
σιν ἐλέφαντος ὅσοι κάρτα εὐδαίμονες· οὐ γὰρ πάντες 
Ἰνδοὶ φορέουσι. 



Flavius Arrianus Hist., Phil., Historia Indica 
Chapter 16, section 4, line 2

                    τοὺς δὲ πώγωνας λέγει <Νέαρχος> ὅτι 
βάπτονται Ἰνδοί, χροιὴν δὲ ἄλλην καὶ ἄλλην <βάπτον-  
ται>, οἳ μὲν ὡς λευκοὺς φαίνεσθαι οἵους λευκοτάτους, 
οἳ δὲ κυανέους, τοῖς δὲ φοινικέους εἶναι, τοῖς δὲ καὶ 
πορφυρέους, ἄλλοις πρασοειδέας· καὶ σκιάδια ὅτι προ-
βάλλονται τοῦ θέρεος ὅσοι οὐκ ἠμελημένοι Ἰνδῶν. 



Flavius Arrianus Hist., Phil., Historia Indica 
Chapter 16, section 6, line 1

ὁπλίσιος δὲ τῆς Ἰνδῶν οὐκ ὡυτὸς εἷς τρόπος ἀλλ' 
οἱ μὲν πεζοὶ αὐτοῖσι τόξον τε ἔχουσι, ἰσόμηκες τῷ φο-
ρέοντι τὸ τόξον, καὶ τοῦτο κάτω ἐπὶ τὴν γῆν θέντες καὶ 
τῷ ποδὶ τῷ ἀριστερῷ ἀντιβάντες, οὕτως ἐκτοξεύουσι, 
τὴν νευρὴν ἐπὶ μέγα ὀπίσω ἀπαγαγόντες· ὁ γὰρ ὀιστὸς 
αὐτοῖσιν ὀλίγον ἀποδέων τριπήχεος, οὐδέ τι ἀντέχει 
τοξευθὲν πρὸς Ἰνδοῦ ἀνδρὸς τοξικοῦ, οὔτε ἀσπὶς οὔτε 
θώρηξ οὔτε <εἴ> τι <τὸ κάρτα> καρτερὸν ἐγένετο. 



Flavius Arrianus Hist., Phil., Historia Indica 
Chapter 16, section 9, line 4

                                                         μάχαιραν 
δὲ πάντες φορέουσι, πλατείην δὲ καὶ τὸ μῆκος οὐ μείω 
τριπήχεος· καὶ ταύτην, ἐπεὰν συστάδην καταστῇ αὐτοῖ-
σιν ἡ μάχη – τὸ δὲ οὐκ εὐμαρέως Ἰνδοῖσιν ἐς ἀλλή-
λους γίνεται – ἀμφοῖν τοῖν χεροῖν καταφέρουσιν ἐς 
τὴν πληγήν, τοῦ καρτερὴν τὴν πληγὴν γενέσθαι. 



Flavius Arrianus Hist., Phil., Historia Indica 
Chapter 17, section 1, line 1

τὰ δὲ σώματα ἰσχνοί τέ εἰσιν Ἰνδοὶ καὶ εὐμήκεες, καὶ 
κοῦφοι πολλόν τι ὑπὲρ τοὺς ἄλλους ἀνθρώπους. 



Flavius Arrianus Hist., Phil., Historia Indica 
Chapter 17, section 1, line 3

                                                      ὀχή-
ματα δὲ τοῖς μὲν πολλοῖς Ἰνδῶν κάμηλοί εἰσιν καὶ ἵπποι 
καὶ ὄνοι, τοῖς δὲ εὐδαίμοσιν ἐλέφαντες. 



Flavius Arrianus Hist., Phil., Historia Indica 
Chapter 17, section 2, line 2

                                                 βασιλικὸν γὰρ 
ὄχημα ἐλέφας παρ' Ἰνδοῖς ἐστι, δεύτερον δὲ τιμῇ ἐπὶ 
τούτῳ τὰ τέθριππα, τρίτον δὲ αἱ κάμηλοι. 



Flavius Arrianus Hist., Phil., Historia Indica 
Chapter 17, section 3, line 4

                                    αἱ γυναῖκες δὲ αὐτοῖσιν, 
ὅσαι κάρτα σώφρονες, ἐπὶ μὲν ἄλλῳ μισθῷ οὐκ ἄν τι 
διαμάρτοιεν, ἐλέφαντα δὲ λαβοῦσα γυνὴ μίσγεται τῷ 
δόντι· οὐδὲ αἰσχρὸν Ἰνδοὶ ἄγουσι τὸ ἐπὶ ἐλέφαντι μι-
γῆναι, ἀλλὰ καὶ σεμνὸν δοκέει τῇσι γυναιξὶν ἀξίην τὸ 
κάλλος φανῆναι ἐλέφαντος. 



Flavius Arrianus Hist., Phil., Historia Indica 
Chapter 17, section 5, line 2

                                       σιτοφάγοι δὲ καὶ   
ἀροτῆρες Ἰνδοί εἰσιν, ὅσοι γε μὴ ὄρειοι αὐτῶν· οὗτοι 
δὲ τὰ θήρεια κρέα σιτέονται. 



Flavius Arrianus Hist., Phil., Historia Indica 
Chapter 17, section 6, line 1

ταῦτά μοι ἀπόχρη δεδηλῶσθαι ὑπὲρ Ἰνδῶν, ὅσα γνω-
ριμώτατα <Νέαρχός> τε καὶ <Μεγασθένης>, δοκίμω ἄνδρε, 
ἀνεγραψάτην, ἐπεὶ οὐδὲ ἡ ὑπόθεσίς μοι τῆσδε τῆς συγ-
γραφῆς τὰ Ἰνδῶν νόμιμα ἀναγράψαι ἦν, ἀλλ' ὅπως γὰρ 
παρεκομίσθη Ἀλεξάνδρῳ ἐς Πέρσας ἐξ Ἰνδῶν ὁ στόλος· 
ταῦτα δὲ ἐκβολή μοι ἔστω τοῦ λόγου. 



Flavius Arrianus Hist., Phil., Historia Indica 
Chapter 18, section 11, line 6

                                                               ὡς 
δὲ ταῦτα ἐκεκόσμητο Ἀλεξάνδρῳ, ἔθυε τοῖς θεοῖσιν ὅσοι 
τε πάτριοι ἢ μαντευτοὶ αὐτῷ καὶ Ποσειδῶνι καὶ Ἀμφι-
τρίτῃ καὶ Νηρηίσι καὶ αὐτῷ τῷ Ὠκεανῷ, καὶ τῷ Ὑδάσπῃ 
ποταμῷ, ἀπ' ὅτου ὡρμᾶτο, καὶ τῷ Ἀκεσίνῃ, ἐς ὅντινα 
ἐκδιδοῖ ὁ Ὑδάσπης, καὶ τῷ Ἰνδῷ, ἐς ὅντινα ἄμφω ἐκδι-
δοῦσιν· ἀγῶνές τε αὐτῷ μουσικοὶ καὶ γυμνικοὶ ἐποιεῦντο, 
καὶ ἱερεῖα τῇ στρατιῇ πάσῃ κατὰ τέλεα ἐδίδοτο. 



Flavius Arrianus Hist., Phil., Historia Indica 
Chapter 19, section 9, line 3

                                    ὁ δὲ λόγος ὅδε τοῦ παρά-
πλου μοι ἀφήγησίς ἐστιν, ὃν Νέαρχος σὺν τῷ στόλῳ 
παρέπλευσεν ἀπὸ τοῦ Ἰνδοῦ τῶν ἐκβολέων ὁρμηθεὶς 
κατὰ τὴν θάλασσαν τὴν μεγάλην ἔστε ἐπὶ τὸν κόλπον 
τὸν Περσικόν, ἣν δὴ Ἐρυθρὴν θάλασσαν μετεξέτεροι 
καλέουσι. 



Flavius Arrianus Hist., Phil., Historia Indica 
Chapter 20, section 2, line 1

                                                               πό-
θον μὲν εἶναι Ἀλεξάνδρῳ ἐκπεριπλῶσαι τὴν θάλασσαν 
τὴν ἀπὸ Ἰνδῶν ἔστε ἐπὶ τὴν Περσικήν, ὀκνέειν δὲ αὐτὸν 
τοῦ τε πλόου τὸ μῆκος καὶ μή τινι ἄρα χώρῃ ἐρήμῳ 
ἐγκύρσαντες ἢ ὅρμων ἀπόρῳ ἢ οὐ ξυμμέτρως ἐχούσῃ 
τῶν ὡραίων, οὕτω δὴ διαφθαρῇ αὐτῷ ὁ στόλος, καὶ οὐ 
φαύλη κηλὶς αὕτη τοῖς ἔργοισιν αὐτοῦ τοῖσι μεγάλοισιν   
ἐπιγενομένη τὴν πᾶσαν εὐτυχίην αὐτῷ ἀφανίσῃ· ἀλλὰ 
ἐκνικῆσαι γὰρ αὐτῷ τὴν ἐπιθυμίην τοῦ καινόν τι αἰεὶ 
καὶ ἄτοπον ἐργάζεσθαι. 



Flavius Arrianus Hist., Phil., Historia Indica 
Chapter 20, section 10, line 3

                 πολὺ δὲ δὴ συνεπιλαβέσθαι ἐς εὐθυμίην 
τῇ στρατιῇ τὸ δὴ αὐτὸν Ἀλέξανδρον ὁρμηθέντα κατὰ 
τοῦ Ἰνδοῦ τὰ στόματα ἀμφότερα ἐκπλῶσαι ἐς τὸν πόν-
τον σφάγιά τε τῷ Ποσειδῶνι ἐντεμεῖν καὶ ὅσοι ἄλλοι 
θεοὶ θαλάσσιοι, καὶ δῶρα μεγαλοπρεπέα τῇ θαλάσσῃ 
χαρίσασθαι. 



Flavius Arrianus Hist., Phil., Historia Indica 
Chapter 21, section 2, line 3

                                        ἄραντες δὲ ἀπὸ τοῦ 
ναυστάθμου τῇ πρώτῃ ἡμέρῃ κατὰ τὸν Ἰνδὸν ποταμὸν 
ὁρμίζονται πρὸς διώρυχι μεγάλῃ, καὶ μένουσιν αὐτοῦ 
δύο ἡμέρας· Στοῦρα δὲ ὄνομα ἦν τῷ χώρῳ· στάδιοι ἀπὸ 
τοῦ ναυστάθμου ἐς ἑκατόν. 



Flavius Arrianus Hist., Phil., Historia Indica 
Chapter 21, section 5, line 3

                     ἐνθένδε ὁρμηθέντες ἔπλεον οὐκ ἐπὶ 
πολλόν· ἕρμα γὰρ ἐφάνη αὐτοῖσι κατὰ τὴν ἐκβολὴν τὴν 
ταύτῃ τοῦ Ἰνδοῦ καὶ τὰ κύματα ἐρρόχθει πρὸς τῇ ἠιόνι, 
καὶ ἡ ἠιὼν αὕτη τραχεῖα ἦν. 



Flavius Arrianus Hist., Phil., Historia Indica 
Chapter 21, section 8, line 1

                                         προσοικέει δὲ ταύτῃ   
ἔθνος Ἰνδικόν, οἱ Ἀράβιες καλεόμενοι, ὧν καὶ ἐν τῇ 
μέζονι ξυγγραφῇ μνήμην ἔσχον, καὶ ὅτι εἰσὶν ἐπώνυμοι 
ποταμοῦ Ἀράβιος, ὃς διὰ τῆς γῆς αὐτῶν ῥέων ἐκδιδοῖ 
ἐς θάλασσαν, ὁρίζων τούτων τε τὴν χώρην καὶ τὴν 
Ὠρειτέων. 



Flavius Arrianus Hist., Phil., Historia Indica 
Chapter 22, section 10, line 4

                            μέχρι μὲν τοῦδε Ἀράβιες, 
ἔσχατοι Ἰνδῶν ταύτῃ ᾠκισμένοι, τὰ δὲ ἀπὸ τοῦδε 
Ὠρεῖται ἐπεῖχον. 



Flavius Arrianus Hist., Phil., Historia Indica 
Chapter 25, section 2, line 2

                Ὠρεῖται δὲ ὅσοι ἄνω ἀπὸ θαλάσσης οἰκέου-
σιν, ἐσταλμένοι μὲν κατάπερ Ἰνδοί εἰσι, καὶ τὰ ἐς πό-
λεμον ὡσαύτως παραρτέονται· γλῶσσα δὲ ἄλλη αὐτοῖσι 
καὶ ἄλλα νόμαια. 



Flavius Arrianus Hist., Phil., Historia Indica 
Chapter 25, section 4, line 1

         παραπλεόντων δὲ τὴν Ἰνδῶν γῆν (τὸ ἐντεῦθεν 
γὰρ οὐκέτι Ἰνδοί εἰσι) λέγει <Νέαρχος> ὅτι αἱ σκιαὶ 
αὐτοῖσιν οὐ ταὐτὸ ἐποίεον· ἀλλὰ ὅπου μὲν ἐπὶ πολὺ τοῦ 
πόντου ὡς πρὸς μεσημβρίαν προχωρήσειαν, αἳ δὲ καὶ 
αὐταὶ [αἱ σκιαὶ] πρὸς μεσημβρίην τετραμμέναι ἐφαίνοντο· 
ὁπότε δὲ τὸ μέσον τῆς ἡμέρης ἐπέχοι ὁ ἥλιος, ἤδη δὲ 
καὶ ἔρημα σκιῆς πάντα ὤφθη αὐτοῖσι. 



Flavius Arrianus Hist., Phil., Historia Indica 
Chapter 25, section 8, line 1

                             εἰκὸς ὦν καὶ ἐν Ἰνδοῖσιν, ἅτε πρὸς 
μεσαμβρίην ᾠκισμένοισι, τὰ αὐτὰ δὴ πάθεα ἐπέχειν, καὶ 
μάλιστα δὴ κατὰ τὸν πόντον τὸν Ἰνδικόν, ὅσῳ μᾶλλον 
αὐτοῖσιν ἡ θάλασσα πρὸς μεσαμβρίην κέκλιται. 



Flavius Arrianus Hist., Phil., Historia Indica 
Chapter 38, section 3, line 4

                              ὑπὸ δὲ τὴν ἕω ἐς ἄλλην 
νῆσον πλεύσαντες ὁρμίζονται οἰκουμένην, ἵνα καὶ μαρ-
γαρίτην θηρᾶσθαι λέγει <Νέαρχος> κατάπερ ἐν τῇ Ἰν-
δῶν θαλάσσῃ. 



Flavius Arrianus Hist., Phil., Historia Indica 
Chapter 42, section 10, line 2

                           οὕτω μὲν ἀπεσώθη Ἀλεξάν-
δρῳ ἐκ τοῦ Ἰνδοῦ τῶν ἐκβολέων ὁρμηθεὶς ὁ στρατός. 



Flavius Arrianus Hist., Phil., Tactica (0074: 005)
“Flavii Arriani quae exstant omnia, vol. 2”, Ed. Roos, A.G., Wirth, G.
Leipzig: Teubner, 1968 (1st edn. corr.).
Chapter 2, section 2, line 4

                                             καὶ τῆς δευτέρας 
ἰδέας τὸ μὲν ἱππικόν, ὅπερ ἵπποις χρῆται, τὸ δ' ἐπὶ 
ἐλεφάντων, καθάπερ τὰ Ἰνδῶν στρατόπεδα καὶ τὰ Αἰ-
θιοπικά, χρόνῳ δ' ὕστερον καὶ Μακεδόνες καὶ Καρχη-
δόνιοι καί που καὶ Ῥωμαῖοι. 



Flavius Arrianus Hist., Phil., Tactica 
Chapter 19, section 6, line 3

                ἀλλὰ ξύμπαντα ταῦτα τὰ ἀσκήματα ἐκλέ-
λειπται, καὶ ἡ τῶν ἐλεφάντων δὲ χρεία ἐς τοὺς πο-
λέμους ὅτι μὴ παρ' Ἰνδοῖς τυχὸν ἢ τοῖς ἄνω Αἰθίοψιν 
καὶ αὕτη ἐκλέλειπται. 



Flavius Arrianus Hist., Phil., Historia successorum Alexandri (fragmenta ap. Photium, Bibl. cod. 92) (0074: 011)
“Flavii Arriani quae exstant omnia, vol. 2”, Ed. Roos, A.G., Wirth, G.
Leipzig: Teubner, 1968 (1st edn. corr.).
Fragment 1,36, line 5

                                   Ἀρείων δὲ καὶ τῆς Δραγ-
γηνῶν χώρας Στάσανδρον καθίστη ἡγεμόνα, τῆς δὲ 
Βακτριανῆς καὶ Σογδιανῆς Στασάνορα τὸν Σόλιον, Ἀρα-
χωτῶν δὲ Σιβύρτιον· καὶ Παραπαμισάδας Ὀξυάρτῃ τῷ 
Ῥωξάνης πατρί· τῆς δὲ Ἰνδῶν γῆς τὰ μὲν ξύνορα Πα-
ραπαμισάδαις Πείθωνι τῷ Ἀγήνορος, τὰς δὲ ἐχομένας 
ξατραπείας τὴν μὲν παρὰ τὸν Ἰνδὸν ποταμὸν καὶ Πά-
ταλα τῶν ἐκείνῃ Ἰνδῶν πόλεων τὴν μεγίστην Πώρῳ τῷ 
βασιλεῖ ἐπεχώρησε, τὴν δὲ παρὰ τὸν Ὑδάσπην ποταμὸν 
Ταξίλῃ, καὶ τούτῳ Ἰνδῷ, ἐπεὶ μηδὲ ῥᾴδιον μετακινῆσαι 
αὐτοὺς ἐξ Ἀλεξάνδρου τε ἐπιτετραμμένους τὴν ἀρχὴν καὶ   
δύναμιν ἱκανὴν ἔχοντας. 



Flavius Arrianus Hist., Phil., Historia successorum Alexandri (fragmenta ap. Photium, Bibl. cod. 82) (0074: 016)
“Flavii Arriani quae exstant omnia, vol. 2”, Ed. Roos, A.G., Wirth, G.
Leipzig: Teubner, 1968 (1st edn. corr.).
Section 5, line 2

         ἦσαν δὲ ἄρχοντες   
Ἰνδῶν μὲν ἁπάντων Πῶ-
ρος καὶ Ταξίλης, ἀλλ' ὁ 
μὲν Πῶρος οἳ ἐν μέσω Ἰν-
δοῦ ποταμοῦ καὶ Ὑδάσπου 
νέμονται, Ταξίλης δὲ τῶν 
λοιπῶν· Πείθων δέ τις 
τῶν τούτοις ὁμόρων ἡγεῖτο, 
πλὴν Παραπαμισαδῶν. 



Flavius Arrianus Hist., Phil., Historia successorum Alexandri (fragmenta ap. Photium, Bibl. cod. 82) 
Section 5, line 10

                       οἱ 
δὲ συνάπτοντες Ἰνδοῖς, 
ὅσοι ὑπὸ τοῖς Καυκασίοις 
ὄρεσι νέμονται, Ὀξυάρτῃ 
τῷ Βακτρίῳ, ὃς ἦν Ῥωξά-
νης πατήρ, εἰς ἀρχὴν ἀπε-
νεμήθησαν – ἧς ἐτέχθη 
παῖς μετὰ τὸν τοῦ πατρὸς 
Ἀλεξάνδρου θάνατον, ᾧ τὸ 
Μακεδόνων πλῆθος τοῦ 
πατρὸς τὴν προσηγορίαν 




Flavius Arrianus Hist., Phil., Fragmenta (0074: 018)
“FGrH \#156”.
Volume-Jacobyʹ-F 2b,156,F, fragment 9, line 132

                                                                    (36) Ἀρείων 
δὲ καὶ τῆς Δραγγηνῶν χώρας Στάσανδρον καθίστη ἡγεμόνα· τῆς δὲ Βα-
κτριανῆς καὶ Σογδιανῆς Στασάνορα τὸν Σόλιον· Ἀραχώτων δὲ Σιβύρτιον· 
καὶ Παραπαμισάδας Ὀξυάρτηι τῶι Ῥωξάνης πατρί· τῆς δὲ Ἰνδῶν γῆς τὰ 
μὲν ξύνορα Παραπαμισάδαις Πίθωνι τῶι Ἀγήνορος· τὰς δὲ ἐχομένας 
ξατραπείας, τὴν μὲν παρὰ τὸν Ἰνδὸν ποταμὸν καὶ Πάταλα τῶν ἐκείνηι 
Ἰνδῶν πόλεων τὴν μεγίστην Πώρωι τῶι βασιλεῖ ἐπεχώρησε· τὴν δὲ παρὰ 
τὸν Ὑδάσπην ποταμὸν Ταξίληι, καὶ τούτωι Ἰνδῶι, ἐπεὶ μηδὲ ῥάιδιον μετα-
κινῆσαι αὐτοὺς ἐξ Ἀλεξάνδρου τε ἐπιτετραμμένους τὴν ἀρχὴν καὶ δύναμιν 
ἱκανὴν ἔχοντας. 



Flavius Arrianus Hist., Phil., Fragmenta 
Volume-Jacobyʹ-F 2b,156,F, fragment 175b, line 17

                                                     (12) Βραχμᾶνες καὶ αὐτοὶ Ἰνδοὶ 
Ἀλεξάνδρωι τῶι βασιλεῖ ταῦτα λέγουσιν· ‘ὀρεγόμενος σοφίας ἦλθες πρὸς ἡμᾶς 
Ἀλέξανδρε, ὅπερ προθυμότερον ἀποδεχόμεθα Βραχμᾶνες· ὅτι ἐστὶ βασιλικώτερον 
ἐν τῶι βίωι ἡμῶν, τοῦτο γὰρ ἠθέλησας μαθεῖν, βασιλεῦ Ἀλέξανδρε. 

\end{greek}

\section{Origen}
\blockquote[From Wikipedia\footnote{\url{http://en.wikipedia.org/wiki/Origen}}]{Origen (/ˈɒrɪdʒən/; Greek: Ὠριγένης Ōrigénēs), or Origen Adamantius (184/185 – 253/254),[1] was a scholar, early Christian theologian and Church Father,[2] who was born and spent the first half of his career in Alexandria. He was a prolific writer in multiple branches of theology, including textual criticism, biblical exegesis and hermeneutics, philosophical theology, preaching, and spirituality. Some of his reputed teachings, such as the pre-existence of souls, the final reconciliation of all creatures, including perhaps even the devil (the apokatastasis),[3] and the subordination of the Son of God to God the Father, later became controversial among Christian theologians. Origen was declared anathema in 553 AD by the Second Ecumenical Council of Constantinople and by three subsequent ecumenical councils. For this reason Origen was and is not called a saint in either the Catholic or Orthodox churches.}
\begin{greek}


Origenes Theol., Contra Celsum (2042: 001)
“Origène. Contre Celse, 4 vols.”, Ed. Borret, M.
Paris: Cerf, 1:1967; 2:1968; 3–4:1969; Sources chrétiennes 132, 136, 147, 150.
Book 1, section 12, line 36

                                                  Τὸ δ' αὐτὸ καὶ 
περὶ Σύρων καὶ Ἰνδῶν καὶ τῶν ὅσοι καὶ μύθους καὶ γράμματα 
ἔχουσι λεκτέον. 



Origenes Theol., Contra Celsum 
Book 1, section 14, line 30

                                        Ἄκουε γὰρ λέγοντος 
τοῦ Κέλσου ὅτι ἔστιν ἀρχαῖος ἄνωθεν λόγος, περὶ ὃν δὴ 
ἀεὶ καὶ τὰ ἔθνη τὰ σοφώτατα καὶ πόλεις καὶ ἄνδρες σοφοὶ 
κατεγένοντο. Καὶ οὐκ ἐβουλήθη ἔθνος σοφώτατον εἰπεῖν 
κἂν παραπλησίως Αἰγυπτίοις καὶ Ἀσσυρίοις καὶ Ἰνδοῖς καὶ 
Πέρσαις καὶ Ὀδρύσαις καὶ Σαμόθρᾳξι καὶ Ἐλευσινίοις τοὺς 
Ἰουδαίους. 



Origenes Theol., Contra Celsum 
Book 1, section 24, line 7

οὐδὲ ὅλον, ἐπεὶ τὸ ὅλον ἐκ μερῶν ἐστι· καὶ οὐχ αἱρεῖ λόγος 
παραδέξασθαι τὸν ἐπὶ πᾶσι θεὸν εἶναι ἐκ μερῶν, ὧν ἕκαστον 
οὐ δύναται ὅπερ τὰ ἄλλα μέρη.⌋ 
 ⌈Μετὰ ταῦτά φησιν ὅτι οἱ αἰπόλοι καὶ ποιμένες ἕνα 
ἐνόμισαν θεόν, εἴτε Ὕψιστον εἴτ' Ἀδωναῖον εἴτ' Οὐράνιον 
εἴτε Σαβαώθ, εἴτε καὶ ὅπῃ καὶ ὅπως χαίρουσιν ὀνομάζοντες 
τόνδε τὸν κόσμον· καὶ πλεῖον οὐδὲν ἔγνωσαν. Καὶ ἐν τοῖς 
ἑξῆς δέ φησι ⌊μηδὲν διαφέρειν τῷ παρ' Ἕλλησι φερομένῳ 
ὀνόματι τὸν ἐπὶ πᾶσι θεὸν καλεῖν⌋ Δία ἢ τῷ ⌊δεῖνα, φέρ'   
εἰπεῖν, παρ' Ἰνδοῖς ἢ τῷ δεῖνα παρ' Αἰγυπτίοις.⌋ Λεκτέον 
δὲ καὶ πρὸς τοῦτο ὅτι ⌊ἐμπίπτει εἰς τὸ προκείμενον λόγος 
βαθὺς καὶ ἀπόρρητος, ὁ περὶ φύσεως ὀνομάτων· πότερον, 
ὡς οἴεται Ἀριστοτέλης, θέσει εἰσὶ τὰ ὀνόματα ἤ, ὡς νομί-
ζουσιν οἱ ἀπὸ τῆς Στοᾶς, φύσει, μιμουμένων τῶν πρώτων 
φωνῶν τὰ πράγματα, καθ' ὧν τὰ ὀνόματα, καθὸ καὶ στοιχεῖά 
τινα τῆς ἐτυμολογίας εἰσάγουσιν, ἤ, ὡς διδάσκει Ἐπίκουρος, 
ἑτέρως ἢ ὡς οἴονται οἱ ἀπὸ τῆς Στοᾶς, φύσει ἐστὶ τὰ ὀνόματα, 
ἀπορρηξάντων τῶν πρώτων ἀνθρώπων τινὰς φωνὰς κατὰ 
τῶν πραγμάτων. 



Origenes Theol., Contra Celsum 
Book 1, section 24, line 19

                 Ἐὰν τοίνυν δυνηθῶμεν ἐν προηγουμένῳ 
λόγῳ παραστῆσαι φύσιν ὀνομάτων ἐνεργῶν, ὧν τισι χρῶνται 
Αἰγυπτίων οἱ σοφοὶ ἢ τῶν παρὰ Πέρσαις μάγων οἱ λόγιοι 
ἢ τῶν παρ' Ἰνδοῖς φιλοσοφούντων Βραχμᾶναι⌋ ἢ Σαμαναῖοι, 
καὶ οὕτω καθ' ἕκαστον τῶν ἐθνῶν, ⌊καὶ κατασκευάσαι⌋ οἷοί 
τε γενώμεθα ⌊ὅτι καὶ ἡ καλουμένη μαγεία οὐχ, ὡς οἴονται 
οἱ ἀπὸ Ἐπικούρου καὶ Ἀριστοτέλους, πρᾶγμά ἐστιν ἀσύς-
τατον πάντῃ ἀλλ' ὡς οἱ περὶ ταῦτα δεινοὶ ἀποδεικνύουσι,   
συνεστὸς μὲν λόγους δ' ἔχον σφόδρα ὀλίγοις γινωσκομένους· 
τότ' ἐροῦμεν ὅτι τὸ μὲν Σαβαὼθ ὄνομα καὶ τὸ Ἀδωναῒ καὶ 
ἄλλα παρ' Ἑβραίοις μετὰ πολλῆς σεμνολογίας παραδιδόμενα 
οὐκ ἐπὶ τῶν τυχόντων καὶ γενητῶν κεῖται πραγμάτων ἀλλ' 
ἐπί τινος θεολογίας ἀπορρήτου, ἀναφερομένης εἰς τὸν τῶν 




Origenes Theol., Contra Celsum 
Book 5, section 34, line 27

                                Σκύθαις γε μὴν καὶ ἀνθρώπους 
δαίνυσθαι καλόν· Ἰνδῶν δέ εἰσιν οἳ καὶ τοὺς πατέρας 
ἐσθίοντες ὅσια δρᾶν νομίζουσι. 



Origenes Theol., Contra Celsum 
Book 5, section 34, line 42

                                                               Δαρεῖος 
δὲ μετὰ ταῦτα καλέσας Ἰνδῶν τοὺς καλεομένους Καλλατίας, 
οἳ τοὺς γονέας κατεσθίουσιν, εἴρετο, παρεόντων τῶν Ἑλλήνων 
καὶ δι' ἑρμηνέος μανθανόντων τὰ λεγόμενα, ἐπὶ τίνι χρήματι 
δεξαίατ' ἂν τελευτῶντας τοὺς πατέρας κατακαίειν πυρί· 
οἱ δὲ ἀμβώσαντες μέγα εὐφημέειν μιν ἐκέλευον. 



Origenes Theol., Contra Celsum 
Book 5, section 36, line 25

                                Καὶ Ἰνδῶν δὲ οἱ τοὺς πατέρας 
ἐσθίοντες ὅσια δρᾶν νομίζουσι, καὶ κατὰ τὸν Κέλσον, ἢ οὐκ 
ἄδικά γε. 



Origenes Theol., Contra Celsum 
Book 5, section 36, line 30

             Ἐκτίθεται γοῦν Ἡροδότου λέξιν συναγορεύουσαν 
τὸ ἕκαστον τοῖς πατρίοις νόμοις καθηκόντως χρῆσθαι, καὶ 
ἔοικεν ἀποδεχομένῳ τοὺς ἐπὶ Δαρείου Καλλατίας καλου-
μένους Ἰνδοὺς τοὺς γονεῖς κατεσθίοντας, ἐπεὶ πρὸς τὸν 
Δαρεῖον πυνθανόμενον, ἐπὶ πόσῳ χρήματι ἀποθέσθαι τοῦτον 
τὸν νόμον βούλονται, ἀναβοήσαντες μέγα εὐφημεῖν αὐτὸν 
ἐκέλευον. 



Origenes Theol., Contra Celsum 
Book 6, section 39, line 20

                             Οὕτω δὲ οὐδὲ περὶ τῶν λοιπῶν 
ταὐτόν τις ἐρεῖ· ἀπ' ἄλλων γὰρ ὁρμώμενοι Ἕλληνες πραγμά-
των καὶ ἐτυμολογιῶν οὕτως ὠνόμασαν τοὺς παρ' ἑαυτοῖς 
νομιζομένους θεούς, ἀπ' ἄλλων δὲ Σκύθαι, οὕτω δὲ καὶ 
ἀπ' ἄλλων μὲν Πέρσαι ἀπ' ἄλλων δὲ Ἰνδοὶ ἢ Αἰθίοπες ἢ 
Λίβυες, ἢ ὅπως φίλον ἑκάστοις ὀνομάζειν, μὴ μείνασιν ἐπὶ 
τῆς πρώτης καὶ καθαρᾶς ἐννοίας τοῦ τῶν ὅλων δημιουργοῦ. 



Origenes Theol., Contra Celsum 
Book 6, section 80, line 15

               Καὶ Πέρσαι δὲ οἱ τὰς μητέρας γαμοῦντες 
καὶ θυγατράσι μιγνύμενοι ἔνθεον ἔθνος εἶναι τῷ Κέλσῳ 
δοκοῦσιν, ἀλλὰ καὶ Ἰνδοί, ὧν τινας ἐν τοῖς προειρημένοις 
ἔλεγε καὶ ἀνθρωπείων γεγεῦσθαι σαρκῶν. 



Origenes Theol., De principiis (2042: 002)
“Origenes vier Bücher von den Prinzipien”, Ed. Görgemanns, H., Karpp, H.
Darmstadt: Wissenschaftliche Buchgesellschaft, 1976.
Book 4, chapter 3, section 1, line 24

                                                                 τίς γὰρ   
οὐκ ἂν τῶν μὴ παρέργως ἀναγινωσκόντων τὰ τοιαῦτα καταγινώσκοι 
τῶν οἰομένων τῷ τῆς σαρκὸς ὀφθαλμῷ, δεηθέντι ὕψους ὑπὲρ τοῦ 
κατανοηθῆναι δύνασθαι τὰ κατωτέρω καὶ ὑποκείμενα, ἑωρᾶσθαι τὴν 
Περσῶν καὶ Σκυθῶν καὶ Ἰνδῶν καὶ Παρθυαίων βασιλείαν, καὶ ὡς 
δοξάζονται παρὰ ἀνθρώποις οἱ βασιλεύοντες; 



Origenes Theol., Homiliae in Lucam (2042: 016)
“Origenes Werke, vol. 9, 2nd edn.”, Ed. Rauer, M.
Berlin: Akademie–Verlag, 1959; Die griechischen christlichen Schriftsteller 49 (35).
Homily 30, page 172, line 22

                 Οὐκ ἐδείκνυε δὲ αὐ-
τῷ τὰς βασιλείας τοῦ κόσμου, οἷον 
φέρε εἰπεῖν τὴν Περσῶν οἰκονομίαν 
ἢ τὴν Ἰνδῶν, ἀλλ' «ἐδείκνυεν αὐτῷ   
τὰς βασιλείας τοῦ κόσμου», τίνα τρό-
πον ἰσχύει αὐτῶν βασιλεύειν, ἵν' αὐτὸν 
προτρέψηται ποιῆσαι, ὃ ἐνόμιζε ποιή-
σαντος περιγενήσεσθαι τοῦ Χριστοῦ. 



Origenes Theol., Fragmenta in Lucam (in catenis) (2042: 017)
“Origenes Werke, vol. 9, 2nd edn.”, Ed. Rauer, M.
Berlin: Akademie–Verlag, 1959; Die griechischen christlichen Schriftsteller 49 (35).
Fragment 98, line 1

οἷον φέρε εἰπεῖν τὴν Περσῶν ἡγεμονίαν ἢ τὴν Ἰνδῶν, τοῦ κόσμου οὖν 
παντὸς τὴν δόξαν ἐδείκνυε καὶ πάσας «τὰς βασιλείας», ἀλλ' «ἐδείκνυεν αὐτῷ 
τὰς βασιλείας τοῦ κόσμου», τίνα τρόπον ἰσχύει αὐτῶν βασιλεύειν, ὅθεν 
ὡς ψιλὸν ἄνθρωπον προέτρεπεν ἐπὶ τοῦτο, βουλόμενος καὶ αὐτοῦ ὡς καὶ 
τῶν λοιπῶν περιγενέσθαι. 



Origenes Theol., Philocalia sive Ecloga de operibus Origenis a Basilio et Gregorio Nazianzeno facta (cap. 1–27) (2042: 019)
“The philocalia of Origen”, Ed. Robinson, J.A.
Cambridge: Cambridge University Press, 1893.
Chapter p, section c, line 41

      Πρὸς τοὺς λέγοντας τῶν φιλοσόφων μηδὲν διαφέρειν τῷ παρ' 
Ἕλλησι φερομένῳ ὀνόματι τὸν ἐπὶ πᾶσι θεὸν καλεῖν Δία· ἢ τῷ δεῖνα, 
φέρε εἰπεῖν, παρ' Ἰνδοῖς· ἢ τῷ δεῖνα παρ' Αἰγυπτίοις. 



Origenes Theol., Philocalia sive Ecloga de operibus Origenis a Basilio et Gregorio Nazianzeno facta (cap. 1-27) 
Chapter 1, section 17, line 29

                                                          τίς 
γὰρ οὐκ ἂν τῶν μὴ παρέργως ἀναγινωσκόντων τὰ τοιαῦτα 
καταγινώσκοι τῶν οἰομένων τῷ τῆς σαρκὸς ὀφθαλμῷ, 
δεηθέντι ὕψους ὑπὲρ τοῦ κατανοηθῆναι δύνασθαι τὰ κατω-
τέρω καὶ ὑποκείμενα, ἑωρᾶσθαι τὴν Περσῶν καὶ Σκυθῶν 
καὶ Ἰνδῶν καὶ Παρθυαίων βασιλείαν, καὶ ὡς δοξάζονται 
παρὰ ἀνθρώποις οἱ βασιλεύοντες; 



Origenes Theol., Philocalia sive Ecloga de operibus Origenis a Basilio et Gregorio Nazianzeno facta (cap. 1-27) 
Chapter 17, section 1n, line 4

8Πρὸς τοὺς λέγοντας τῶν φιλοσόφων μηδὲν δια-
φέρειν τῷ παρ' Ἕλλησι φερομένῳ ὀνόματι τὸν ἐπὶ 
πᾶσι θεὸν καλεῖν Δία, ἢ τῷ δεῖνα φέρ' εἰπεῖν παρ' 
Ἰνδοῖς, ἢ τῷ δεῖνα παρ' Αἰγυπτίοις. 



Origenes Theol., Philocalia sive Ecloga de operibus Origenis a Basilio et Gregorio Nazianzeno facta (cap. 1-27) 
Chapter 17, section 1, line 7

καὶ ἐν τοῖς ἑξῆς δέ φησι ‘μηδὲν διαφέρειν τῷ παρ' Ἕλλησι 
φερομένῳ ὀνόματι τὸν ἐπὶ πᾶσι θεὸν καλεῖν Δία, ἢ τῷ 
δεῖνα φέρ' εἰπεῖν παρ' Ἰνδοῖς, ἢ τῷ δεῖνα παρ' Αἰγυπτίοις. 



Origenes Theol., Philocalia sive Ecloga de operibus Origenis a Basilio et Gregorio Nazianzeno facta (cap. 1-27) 
Chapter 17, section 1, line 19

                                     ἐὰν τοίνυν δυνηθῶμεν 
ἐν προηγουμένῳ λόγῳ παραστῆσαι φύσιν ὀνομάτων ἐνερ-
γῶν, ὧν τισὶ χρῶνται Αἰγυπτίων οἱ σοφοὶ, ἢ τῶν παρὰ 
Πέρσαις μάγων οἱ λόγιοι, ἢ τῶν παρ' Ἰνδοῖς φιλοσοφούν-
των Βράχμαναι, ἢ Σαμαναῖοι· καὶ οὕτω καθ' ἕκαστον τῶν 
ἐθνῶν· καὶ κατασκευάσαι οἷοί τε γενώμεθα ὅτι καὶ ἡ καλου-
μένη μαγεία οὐχ, ὡς οἴονται οἱ ἀπὸ Ἐπικούρου καὶ Ἀριστο-
τέλους, πρᾶγμά ἐστιν ἀσύστατον πάντη, ἀλλ', ὡς οἱ περὶ 
ταῦτα δεινοὶ ἀποδεικνύουσι, συνεστὼς μὲν, λόγους δ' ἔχον 
σφόδρα ὀλίγοις γιγνωσκομένους· τότ' ἐροῦμεν ὅτι τὸ μὲν 
Σαβαὼθ ὄνομα καὶ τὸ Ἀδωναῒ, καὶ ὅσα ἄλλα παρ' Ἑβραίοις 
μετὰ πολλῆς σεμνολογίας παραδεδομένα, οὐκ ἐπὶ τῶν 
τυχόντων καὶ γενητῶν κεῖται πραγμάτων, ἀλλ' ἐπί τινος 




Origenes Theol., Philocalia sive Ecloga de operibus Origenis a Basilio et Gregorio Nazianzeno facta (cap. 1-27) 
Chapter 18, section 6, line 33

                                             τὸ δ' αὐτὸ καὶ 
περὶ Σύρων καὶ Ἰνδῶν καὶ τῶν ὅσοι καὶ μύθους καὶ γράμ-
ματα ἔχουσι λεκτέον. 



Origenes Theol., Commentarium in evangelium Matthaei (lib. 10–11) (2042: 029)
“Origène. Commentaire sur l'évangile selon Matthieu, vol. 1”, Ed. Girod, R.
Paris: Cerf, 1970; Sources chrétiennes 162.
Book 10, section 7, line 15

                                          Καὶ οἱ μὲν χερσαῖοι 
παρ' Ἰνδοῖς μόνοις γίνονται πρέποντες σφραγῖσι καὶ σφενδό-
ναις καὶ ὅρμοις. 



Origenes Theol., Commentarium in evangelium Matthaei (lib. 10-11) 
Book 10, section 7, line 17

                    Οἱ δὲ θαλάττιοι οἱ μὲν διαφέροντες παρὰ 
τοῖς αὐτοῖς Ἰνδοῖς εὑρίσκονται, οἵτινές εἰσι καὶ ἄριστοι ἐν 
τῇ ἐρυθρᾷ θαλάσσῃ γινόμενοι. 



Origenes Theol., Commentarium in evangelium Matthaei (lib. 10-11) 
Book 10, section 7, line 23

Ἔτι δὲ ταῦτα ἐλέγετο περὶ τοῦ Ἰνδικοῦ μαργαρίτου 
ὅτι ἐν κόγχοις γίνεται προσεοικόσι τὴν φύσιν εὐμεγέθεσι 
στρόμβοις. 



Origenes Theol., Commentarium in evangelium Matthaei (lib. 10-11) 
Book 10, section 7, line 30

                                Ἱστόρηται δὲ καὶ περὶ τῆς 
θήρας τῶν διαφερόντων, τουτέστι τῶν ἐν Ἰνδίᾳ, τοιοῦτον· 
ὅτι περιλαμβάνοντες οἱ ἐπιχώριοι δικτύοις κύκλον αἰγιαλοῦ 
μέγαν κατακολυμβῶσιν, ἕνα ἐξ ἁπάντων τὸν προηγούμενον 
ἐπιτηδεύοντες λαβεῖν, τούτου γὰρ ἁλόντος φασὶν ἄμοχθον 
γενέσθαι τὴν θήραν τῆς ὑπὸ τούτῳ ἀγέλης, οὐδενὸς ἔτι 
ἀτρεμοῦντος τῶν ἀπ' αὐτῆς, ἀλλὰ οἷον δεδεμένου ἱμάντι καὶ 
ἑπομένου τῷ ἀγελάρχῃ. 



Origenes Theol., Commentarium in evangelium Matthaei (lib. 10-11) 
Book 10, section 7, line 37

Λέγεται δὲ καὶ ἡ γένεσις τῶν ἐν Ἰνδίᾳ μαργαριτῶν χρόνοις 
συνίστασθαι, τροπὰς λαμβάνοντος τοῦ ζῴου πλείονας καὶ 
μεταβολάς, ἕως τελειωθῇ. 



Origenes Theol., Commentarium in evangelium Matthaei (lib. 10-11) 
Book 10, section 7, line 51

                                                              Ἔτι 
δὲ καὶ τοῦτο ἔχει ὁ Ἰνδικὸς μαργαρίτης παρὰ τοὺς ἄλλους· 
λευκός ἐστι τὴν χρόαν, ἀργύρῳ διαφανεῖ προσφερὴς αὐγήν 
τε ὑποχλωρίζουσαν ἠρέμα διαλάμπει, ὡς ἐπίπαν δὲ σχῆμα 
ἔχει στρογγύλον. 



Origenes Theol., Commentarium in evangelium Matthaei (lib. 10-11) 
Book 10, section 7, line 60

           Ταῦτα μὲν οὖν περὶ τοῦ Ἰνδικοῦ. 



Origenes Theol., Selecta in Psalmos [Dub.] (fragmenta e catenis) (2042: 058); MPG 12.
Volume 12, page 1524, line 43

             Σαβὰ δὲ πόλις τῆς Ἰνδίας, ἀφ' ἧς ἦλθεν ἡ 
βασίλισσα νότου πρὸς Σαλομῶντα. 



Origenes Theol., Homiliae in Job (fragmenta in catenis, typus II) (e codd. Marc. gr. 21, 538) (2042: 073); MPG 17.
Volume 17, page 89, line 51

Τὰ ἐσώτατα περὶ τὴν Ἰνδίαν, ἢ τὰ περὶ τὴν 
ἐπέκεινα χώραν, ἔνθα εἰσὶ τίμιοι λίθοι. 



Origenes Theol., Scholia in Matthaeum (2042: 077); MPG 17.
Volume 17, page 296, line 25

Οἵ γε μὴν περὶ μαργαριτῶν γράψαντες, φασὶν 
ἓξ εἶναι τὰς τούτων διαφορὰς, ὧν κρείττους οἱ κατ' 
Ἰνδίαν ἐν κόγχαις γενόμενοι ἐκ δρόσου τῆς οὐρανίας. 



Origenes Theol., Homiliae in Job (fragmenta in catenis, typus I+II) (e codd. Vat.) (2042: 086)
“Analecta sacra spicilegio Solesmensi parata, vol. 2”, Ed. Pitra, J.B.
Paris: Tusculum, 1884, Repr. 1966.
Page 391, line 4

Συριακὴν νῦν τὴν Ἑβραίων διάλεκτον κα-  
λεῖ, ἐπειδὴ καὶ Συρίαν τὴν Ἰουδαῖαν, καὶ 
Σύρους οἱ πολλοὶ τοὺς Παλαιστινοὺς ὀνομά-
ζουσιν· καὶ Ἡρόδοτος ὁ ἱστοριογράφος φησί· 
Περιτέμνονται δὲ Ἰνδοὶ, καὶ Αἰγύπτιοι, καὶ 
Ἄραβες, καὶ οἱ ἐν Παλαιστίνῃ Σύροι· τοὺς 
Ἰουδαίους οὕτω καλοῦσιν. 

\end{greek}



\section{Pseudo-Scymnus}
\blockquote[From Wikipedia\footnote{\url{http://en.wikipedia.org/wiki/Pseudo-Scymnus}}]{Pseudo-Scymnus is the name given by Augustus Meineke to the unknown author of a work on geography written in Classical Greek, the Periodos to Nicomedes. It is a an account of the world (periegesis) in 'comic' iambic trimeters which is dedicated to a King Nicomedes of Bithynia. This is either Nicomedes II Epiphanes who reigned from 149 BC for an unknown number of years or his son, Nicomedes III Euergetes.[1]}
\begin{greek}

Pseudo-Scymnus Geogr., Ad Nicomedem regem, vv. 1–980 (sub titulo Orbis descriptio) (0068: 001)
“Geographi Graeci minores, vol. 1”, Ed. Müller, K.
Paris: Didot, 1855, Repr. 1965.
Line 171

Τὴν μὲν γὰρ ἐντὸς ἀνατολῶν πᾶσαν σχεδόν 
οἰκοῦσιν Ἰνδοὶ, τὴν δὲ πρὸς μεσημβρίαν 
Αἰθίοπες ἐγγὺς κείμενοι νότου πνοῆς· 
τὸν ἀπὸ ζεφύρου Κελτοὶ δὲ μέχρι δυσμῶν τόπον 
θερινῶν ἔχουσιν, τὸν δὲ πρὸς βορρᾶν Σκύθαι. 



Pseudo-Scymnus Geogr., Ad Nicomedem regem, vv. 722–1026 (0068: 002)
“The tradition of the minor Greek geographers”, Ed. Diller, A.
Lancaster, Pennsylvania: American Philological Association, 1952.
Line 933

                            εἰσιόντι δὲ 
ἀριστερὰ τοῦ Φάσιδος παρακειμένη 
Μιλησίων πόλις <ἐστὶ> Φᾶσις λεγομένη 
Ἑλληνίς· εἰς ταύτην δὲ καταβαίνειν λόγος 
φωναῖς διαφόροις χρώμεν' ἐξήκοντ' ἔθνη, 
ἐν οἷς τινας λέγουσιν ἀπὸ τῆς Ἰνδικῆς 
καὶ Βακτριανῆς <γῆς> συναφικνεῖσθαι βαρβάρους. 

\end{greek}



\section{Cassius Dio}
\blockquote[From Wikipedia\footnote{\url{http://en.wikipedia.org/wiki/Cassius_Dio}}]{Lucius (or Claudius) Cassius Dio (alleged to have the cognomen Cocceianus),[1][2] (Ancient Greek: Δίων ὁ Κάσσιος, c. AD 155 – 235,[3][4] known in English as Cassius Dio, Dio Cassius, or Dio (Dione. lib), was a Roman consul and noted historian who wrote in Greek. Dio published a history of Rome in 80 volumes, beginning with the legendary arrival of Aeneas in Italy; the volumes then documented the subsequent founding of Rome (753 BC), the formation of the Republic (509 BC), and the creation of the Empire (31 BC), up until AD 229. The entire period covered by Dio's work is approximately 1,400 years. Of the 80 books, written over 22 years, many survive into the modern age, intact, or as fragments, providing modern scholars with a detailed perspective on Roman history.}
\begin{greek}

Cassius Dio Hist., Historiae Romanae (0385: 001)
“Cassii Dionis Cocceiani historiarum Romanarum quae supersunt, 3 vols.”, Ed. Boissevain, U.P.
Berlin: Weidmann, 1:1895; 2:1898; 3:1901, Repr. 1955.
Book 49, chapter 41, section 3, line 6

                  ἐκείνοις μὲν δὴ ταῦτ' ἔνειμε, τοῖς δὲ δὴ αὑτοῦ 
παισὶ τοῖς ἐκ τῆς Κλεοπάτρας οἱ γεγονόσι, Πτολεμαίῳ μὲν τήν 
τε Συρίαν καὶ τὰ ἐντὸς τοῦ Εὐφράτου μέχρι τοῦ Ἑλλησπόντου 
πάντα, Κλεοπάτρᾳ δὲ τὴν Λιβύην τὴν περὶ Κυρήνην, τῷ τε ἀδελφῷ 
αὐτῶν Ἀλεξάνδρῳ τήν τε Ἀρμενίαν καὶ τἆλλα τὰ πέραν τοῦ Εὐ-
φράτου μέχρις Ἰνδῶν δώσειν ὑπέσχετο· καὶ γὰρ ἐκεῖνα ὡς ἔχων 
ἤδη ἐχαρίζετο. 



Cassius Dio Hist., Historiae Romanae 
Book 54, chapter 9, section 8, line 2

                                        πάμπολλαι γὰρ δὴ πρεσβεῖαι 
πρὸς αὐτὸν ἀφίκοντο, καὶ οἱ Ἰνδοὶ προκηρυκευσάμενοι πρότερον 
φιλίαν τότε ἐσπείσαντο, δῶρα πέμψαντες ἄλλα τε καὶ τίγρεις, 
πρῶτον τότε τοῖς Ῥωμαίοις, νομίζω δ' ὅτι καὶ τοῖς Ἕλλησιν, 
ὀφθείσας. 



Cassius Dio Hist., Historiae Romanae 
Book 54, chapter 9, section 10, line 2

                                                                         εἷς 
δ' οὖν τῶν Ἰνδῶν Ζάρμαρος, εἴτε δὴ τοῦ τῶν σοφιστῶν γένους ὤν, 
καὶ κατὰ τοῦτο ὑπὸ φιλοτιμίας, εἴτε καὶ ὑπὸ τοῦ γήρως κατὰ τὸν 
πάτριον νόμον, εἴτε καὶ ἐς ἐπίδειξιν τοῦ τε Αὐγούστου καὶ τῶν 
Ἀθηναίων (καὶ γὰρ ἐκεῖσε ἦλθεν) ἀποθανεῖν ἐθελήσας ἐμυήθη τε 
τὰ τοῖν θεοῖν, τῶν μυστηρίων καίπερ οὐκ ἐν τῷ καθήκοντι καιρῷ, 
ὥς φασι, διὰ τὸν Αὔγουστον καὶ <αὐτὸν> μεμυημένον γενομένων, 
καὶ πυρὶ ἑαυτὸν ζῶντα ἐξέδωκεν. 



Cassius Dio Hist., Historiae Romanae 
Book 59, chapter 17, section 3, line 5

                                                    ἐπειδή τε ἕτοιμα ἦν, 
τόν τε θώρακα τὸν Ἀλεξάνδρου, ὥς γε ἔλεγε, καὶ ἐπ' αὐτῷ χλαμύδα 
σηρικὴν ἁλουργῆ, πολὺ μὲν χρυσίον πολλοὺς δὲ καὶ λίθους Ἰνδικοὺς 
ἔχουσαν, ἐπενέδυ, ξίφος τε παρεζώσατο καὶ ἀσπίδα ἔλαβε καὶ δρυῒ 
ἐστεφανώσατο, κἀκ τούτου τῷ τε Ποσειδῶνι καὶ ἄλλοις τισὶ θεοῖς 
Φθόνῳ τε θύσας, μὴ καὶ βασκανία τις αὐτῷ, ὡς ἔφασκε, γένηται, 
ἔς τε τὸ ζεῦγμα ἀπὸ τῶν Βαύλων ἐσέβαλε, παμπληθεῖς καὶ ἱππέας 
καὶ πεζοὺς ὡπλισμένους ἐπαγόμενος, καὶ σπουδῇ καθάπερ ἐπὶ πο-
λεμίους τινὰς ἐς τὴν πόλιν ἐσέπεσε. 



Cassius Dio Hist., Historiae Romanae 
Book 68, chapter 15, section 1, line 2

                              Xiph. 232, 28 – 234, 16 R. St. 
 πρὸς <δὲ> τὸν Τραϊανὸν ἐς τὴν Ῥώμην ἐλθόντα πλεῖσται ὅσαι 
πρεσβεῖαι παρὰ βαρβάρων ἄλλων τε καὶ Ἰνδῶν ἀφίκοντο. 



Cassius Dio Hist., Historiae Romanae 
Book 68, chapter 29, section 1, line 2

Exc. Val. 292 (p. 713) et Xiph. 239, 14 – 16 R. St. 
 κἀντεῦθεν ἐπ' αὐτὸν τὸν ὠκεανὸν ἐλθών, τήν τε φύσιν αὐτοῦ 
καταμαθὼν καὶ πλοῖόν τι ἐς Ἰνδίαν πλέον ἰδών, εἶπεν ὅτι “πάν-
τως ἂν καὶ ἐπὶ τοὺς Ἰνδούς, εἰ νέος ἔτι ἦν, ἐπεραιώθην”. 



Cassius Dio Hist., Historiae Romanae 
Book 69, chapter 16, section 1, line 2

                       Exc. UG9 55 (p. 407).   
 Ἁδριανὸς δὲ τό τε Ὀλύμπιον τὸ ἐν ταῖς Ἀθήναις, ἐν ᾧ καὶ 
αὐτὸς ἵδρυται, ἐξεποίησε, καὶ δράκοντα ἐς αὐτὸ ἀπὸ Ἰνδίας κομι-
σθέντα ἀνέθηκε· τά τε Διονύσια, τὴν μεγίστην παρ' αὐτοῖς ἀρχὴν 
ἄρξας, ἐν τῇ ἐσθῆτι τῇ ἐπιχωρίῳ λαμπρῶς ἐπετέλεσε. 



Cassius Dio Hist., Historiae Romanae 
Book 72, chapter 17, section 3, line 5

              ἐνέδυνε δέ, πρὶν μὲν ἐς τὸ θέατρον ἐσιέναι, χιτῶνα 
χειριδωτὸν σηρικὸν λευκὸν διάχρυσον (καὶ ἐν τούτῳ γε αὐτὸν τῷ 
σχήματι ὄντα ἠσπαζόμεθα), ἐσιὼν δὲ ὁλοπόρφυρον χρυσῷ κατά-
παστον, χλαμύδα τε ὁμοίαν τὸν Ἑλληνικὸν τρόπον λαμβάνων, καὶ 
στέφανον ἔκ τε λίθων Ἰνδικῶν καὶ ἐκ χρυσοῦ πεποιημένον, κηρύ-
κειόν τε τοιοῦτον φέρων ὁποῖον ὁ Ἑρμῆς. 



Cassius Dio Hist., Historiae Romanae 
Book 74, chapter 5, section 1, line 1

                                  εἶτ' εἰκόνες ἧκον ἀνδρῶν ἄλλων, οἷς 
τι ἔργον ἢ ἐξεύρημα ἢ καὶ ἐπιτήδευμα λαμπρὸν ἐπέπρακτο, καὶ 
μετ' αὐτοὺς οἵ τε ἱππεῖς καὶ οἱ πεζοὶ ὡπλισμένοι οἵ τε ἀθληταὶ 
ἵπποι καὶ τὰ ἐντάφια, ὅσα ὅ τε αὐτοκράτωρ καὶ ἡμεῖς αἵ τε γυ-
ναῖκες ἡμῶν καὶ οἱ ἱππεῖς οἱ ἐλλόγιμοι οἵ τε δῆμοι καὶ τὰ ἐν τῇ 
πόλει συστήματα ἐπέμψαμεν· καὶ αὐτοῖς βωμὸς περίχρυσος, ἐλέ-
φαντί τε καὶ λίθοις Ἰνδικοῖς ἠσκημένος, ἠκολούθει. 



Cassius Dio Hist., Historiae Romanae 
Book 76, chapter 1, section 4, line 1

                                                   ἐν ταύταις ταῖς θέαις καὶ 
σύες τοῦ Πλαυτιανοῦ ἑξήκοντα ἄγριοι ἐπάλαισαν ἀλλήλοις ὑπὸ 
παραγγέλματος, ἐσφάγησαν δὲ ἄλλα τε πολλὰ θηρία καὶ ἐλέφας 
καὶ κοροκότας· τὸ δὲ ζῷον τοῦτο Ἰνδικόν τέ ἐστι, καὶ τότε πρῶτον 
ἐς τὴν Ῥώμην, ὅσα καὶ ἐγὼ ἐπίσταμαι, ἐσήχθη, ἔχει δὲ χροιὰν μὲν 
λεαίνης τίγριδι μεμιγμένης, εἶδος δὲ ἐκείνων τε καὶ κυνὸς καὶ ἀλώ-
πεκος ἰδίως πως συγκεκραμένον. 



Cassius Dio Hist., Historiae Romanae (versio 1 in volumine 1) (0385: 002)
“Cassii Dionis Cocceiani historiarum Romanarum quae supersunt, vol. 1”, Ed. Boissevain, U.P.
Berlin: Weidmann, 1895, Repr. 1955.



Cassius Dio Hist., Historiae Romanae (Xiphilini epitome) (0385: 010)
“Cassii Dionis Cocceiani historiarum Romanarum quae supersunt, vol. 3”, Ed. Boissevain, U.P.
Berlin: Weidmann, 1901, Repr. 1955.
Dindorf-Stephanus page 70, line 32

                                                    Ἀντώνιος δὲ τὸν 
βασιλέα τῶν Ἀρμενίων δόλῳ καὶ ἀπάτῃ ἑλών, ὅτι μὴ συνεμάχησέν 
οἱ κατὰ τῶν Πάρθων, ἀργυραῖς ἁλύσεσι περιῆγεν, εἶτα καὶ χρυσαῖς 
τῇ Κλεοπάτρᾳ προσῆγε, καὶ τοῦ λοιποῦ τὰ ὅπλα ῥίψας συνετρύφα 
αὐτῇ, τοὺς παῖδας αὐτῆς βασιλέων τε βασιλέας προσαγορεύων, καὶ 
χώρας αὐτοῖς οὐχ ὅτι τὴν Ἀρμενίων καὶ ἃς εἶχεν, ἀλλὰ καὶ 
τὴν Ἰνδῶν καὶ τὴν Πάρθων ἀπονέμων. 



Cassius Dio Hist., Historiae Romanae (Xiphilini epitome) 
Dindorf-Stephanus page 92, line 25

            καὶ γὰρ καὶ Ἰνδοὶ προκηρυκευσάμενοι πρότερον φιλίαν 
τότε ἐσπείσαντο, δῶρα πέμψαντες ἀλλά τε καὶ τίγρεις, πρῶτον τότε 
Ῥωμαίοις ὀφθείσας, καί τι καὶ μειράκιον ἄνευ ὤμων αὐτῷ ἐδωρήσαντο. 



Cassius Dio Hist., Historiae Romanae (Xiphilini epitome) 
Dindorf-Stephanus page 93, line 1

                                               τότε καὶ ὁ σοφιστὴς Ζά-
μαρκος ὁ Ἰνδός, εἴτε ὑπὸ φιλοτιμίας εἴτε ὑπὸ τοῦ γήρως, κατὰ τὸν 
πάτριον νόμον πυρὶ ἑαυτὸν ζῶντα ἐξέδωκεν. 



Cassius Dio Hist., Historiae Romanae (Xiphilini epitome) 
Dindorf-Stephanus page 162, line 26

     ἐπειδὴ δὲ ἕτοιμα ἦν, τόν τε θώρακα τοῦ Ἀλεξάνδρου, ὡς 
ἔλεγε, καὶ ἐπ' αὐτῷ χλαμύδα σηρικὴν ἁλουργῆ πολὺ μὲν χρυσίον 
πολλοὺς δὲ καὶ λίθους Ἰνδικοὺς ἔχουσαν ἐπενέδυ, ξίφος τε παρεζώ-
σατο καὶ ἀσπίδα ἔλαβε καὶ δρυῒ ἐστεφάνωτο. 



Cassius Dio Hist., Historiae Romanae (Xiphilini epitome) 
Dindorf-Stephanus page S234, line 18

      παρὰ τὸν Τραϊανὸν ἐς τὴν Ῥώμην ἐλθόντα πλεῖσται πρεσβεῖαι 
ἄλλων τε βαρβάρων καὶ Ἰνδῶν ἀφίκοντο. 



Cassius Dio Hist., Historiae Romanae (Xiphilini epitome) 
Dindorf-Stephanus page S239, line 18

                                                      κἀντεῦθεν ἐπ' αὐτὸν 
τὸν ὠκεανὸν ἐλθών, τήν τε φύσιν αὐτοῦ καταμαθὼν καὶ πλοῖόν τι 
ἐς Ἰνδίαν πλέον ἰδών, εἶπεν ὅτι “πάντως ἂν καὶ ἐπὶ τοὺς Ἰνδούς, 
εἰ νέος ἔτι ἦν, ἐπεραιώθην. 



Cassius Dio Hist., Historiae Romanae (Xiphilini epitome) 
Dindorf-Stephanus page S252, line 3

                                                       Ἀδριανὸς δὲ τό 
τε Ὀλύμπιον τὸ ἐν ταῖς Ἀθήναις, ἐν ᾧ καὶ αὐτὸς ἵδρυται, ἐξεποίησε, 
καὶ δράκοντα ἐς αὐτὸ ἀπὸ Ἰνδίας κομισθέντα ἀνέθηκε· τά τε Διο-
νύσια, τὴν μεγίστην παρ' αὐτοῖς ἀρχὴν ἄρξας, ἐν τῇ ἐσθῆτι τῇ ἐπι-
χωρίῳ λαμπρῶς ἐπετέλεσε. 



Cassius Dio Hist., Historiae Romanae (Xiphilini epitome) 
Dindorf-Stephanus page S277, line 23

                                             ἐνέδυνε δέ, πρὶν μὲν εἰς τὸ 
θέατρον εἰσιέναι, χιτῶνα χειριδωτὸν σηρικὸν λευκὸν διάχρυσον (καὶ 
ἐν τούτῳ γε αὐτὸν τῷ σχήματι ὄντα ἠσπαζόμεθα), ἐσιὼν δὲ ὁλο-
πόρφυρον χρυσῷ κατάπαστον, χλαμύδα τε ὁμοίαν τὸν Ἑλληνικὸν 
τρόπον λαμβάνων, καὶ στέφανον ἔκ τε λίθων Ἰνδικῶν καὶ ἐκ χρυσοῦ 
πεποιημένον, κηρύκειόν τε τοιοῦτον φέρων ὁποῖον ὁ Ἑρμῆς. 



Cassius Dio Hist., Historiae Romanae (Xiphilini epitome) 
Dindorf-Stephanus page S296, line 9

                                                              εἶτ' εἰκόνες 
ἧκον ἀνδρῶν ἄλλων, οἷς τι ἔργον ἢ ἐξεύρημα ἢ καὶ ἐπιτήδευμα 
λαμπρὸν ἐπέπρακτο, καὶ μετ' αὐτοὺς οἵ τε ἱππεῖς καὶ οἱ πεζοὶ 
ὡπλισμένοι οἵ τε ἀθληταὶ ἵπποι καὶ τὰ ἐντάφια, ὅσα ὅ τε αὐτοκρά-
τωρ καὶ ἡμεῖς αἵ τε γυναῖκες ἡμῶν καὶ οἱ ἱππεῖς οἱ ἐλλόγιμοι οἵ τε 
δῆμοι καὶ τὰ ἐν τῇ πόλει συστήματα ἐπέμψαμεν· καὶ αὐτοῖς βωμὸς 
περίχρυσος, ἐλέφαντί τε καὶ λίθοις Ἰνδικοῖς ἠσκημένος, ἠκολούθει. 



Cassius Dio Hist., Historiae Romanae (Xiphilini epitome) 
Dindorf-Stephanus page S314, line 30

                                  ἐν ταύταις ταῖς θέαις καὶ σύες τοῦ 
Πλαυτιανοῦ ἑξήκοντα ἄγριοι ἐπάλαισαν ἀλλήλοις ὑπὸ παραγγέλματος, 
ἐσφάγησαν δὲ ἄλλα τε πολλὰ θηρία καὶ ἐλέφας καὶ κοροκότας· τὸ 
δὲ ζῷον τοῦτο Ἰνδικόν ἐστι, καὶ τότε πρῶτον ἐς τὴν Ῥώμην, ὅσα 
καὶ ἐγὼ ἐπίσταμαι, εἰσήχθη, ἔχει δὲ χροιὰν μὲν λεαίνης τίγριδι μεμι-
γμένης, εἶδος δὲ ἐκείνων τε καὶ κυνὸς καὶ ἀλώπεκος ἰδίως πως 
συγκεκραμένον. 

\end{greek}

\section{Hippolytus Scr. Eccl}
\blockquote[From Wikipedia\footnote{\url{http://en.wikipedia.org/wiki/Hippolytus_of_Rome}}]{Hippolytus of Rome (170–235) was the most important 3rd-century theologian in the Christian Church in Rome,[2] where he was probably born.}
\begin{greek}

Hippolytus Scr. Eccl., Chronicon (2115: 036)
“Hippolytus Werke, vol. 4, 2nd edn.”, Ed. Helm, R. (post A. Bauer)
Berlin: Akademie–Verlag, 1955; Die griechischen christlichen Schriftsteller 46.
Section 47, line 2

   τῶν τριῶν ἀδελφῶν αἱ φυλαὶ διε-
μερίσθησαν, 
         καὶ τῷ μὲν Σὴμ τῷ πρωτοτόκῳ 
ἀπὸ Περσίδος καὶ Βάκτρων ἕως Ἰνδικῆς τὸ μῆ-
κος, πλάτος δὲ ἀπὸ τῆς Ἰνδικῆς ἕως Ῥινοκορού-
ρων, 
         Χὰμ δὲ τῷ δευτέρῳ ἀπὸ Ῥινοκορού-
ρων ἕως Γαδείρων τὰ πρὸς νότον, 
         Ἰάφεθ 
δὲ τῷ τρίτῳ ἀπὸ Μηδίας ἕως Γαδείρων τὰ πρὸς 
βορρᾶν. 



Hippolytus Scr. Eccl., Chronicon 
Section 84, line 4

Αἱ δὲ χῶραί εἰσιν αὗται· (1) Μηδία (2) Ἀλ-
βανία (3) Ἀμαζονίς (4) Ἀρμενία μικρὰ καὶ με-
γάλη (5) Καππαδοκία (6) Παφλαγονία (7) Γα-
λατία (8) Κολχίς (9) Ἰνδικὴ Ἀχαΐα (10) Βοσπο-
ρινή (11) Μαιῶτις (12) Δέρρης (13) Σαρματίς 
(14) Ταυριανή (15) Βασταρνίς (16) Σκυθία (17) 
Θρᾴκη (18) Μακεδονία (19) Δελματία (20) Μολ-
χίς (21) Θεσσαλία (22) Λωκρίς (23) Βοιωτία   
(24) Αἰτωλία (25) Ἀττική <(26) Ἀχαία> (27) Πε-
λοπόννησος <(28) Ἀκαρνία> (29) Ἠπειρώτης 
(30) Ἰλλυρίς (31) ἡ Λυχνῖτις (32) Ἀδριακή, ἀφ' 
ἧς τὸ Ἀδριακὸν πέλαγος (33) Γαλλία (34) Θου-
σκηνή (35) Λυσιτανία (36) Μεσαλία (37) Ἰταλία 




Hippolytus Scr. Eccl., Chronicon 
Section 139, line 1

(2) Αἰθιοπία ἡ βλέπουσα κατὰ Ἰνδούς,   
 (3) καὶ ἑτέρα Αἰθιοπία, ὅθεν ἐκπορεύεται 
Γηὼν ὁ <τῶν Αἰθιόπων> ποταμὸς ὁ καλούμενος 
Νεῖλος, 
         (4) Ἐρυθρὰ ἡ βλέπουσα κατὰ 
ἀνατολάς, 
         (5) Θηβαῒς ὅλη, 
         (6) Λι-
βύη ἡ παρεκτείνουσα μέχρι Κορκυρίνης, 
         (7) 




Hippolytus Scr. Eccl., Chronicon 
Section 176, line 2

Ἰεκτὰν δὲ [ὁ ἀδελφὸς Φάλεχ] ἐγέννησε (15) τὸν 
Ἐλμωδὰδ, ὅθεν γεννῶνται οἱ Ἰνδοί, 
         καὶ 
(16) τὸν Σαλέφ, ὅθεν οἱ Βακτριανοί, 
         καὶ 
(17) τὸν Ἀράμ, ὅθεν οἱ Ἀράβ<ι>ες, 
         καὶ 
(18) Ἰ<δ>ουράμ, ὅθεν Καρμήλιοι, 
         καὶ (19) 
Αἰθήλ, ὅθεν οἱ Ἀρειανοί, 
         καὶ (20)

Ἀβι-



Hippolytus Scr. Eccl., Chronicon 
Section 188, line 5

   Πάντων δὲ τῶν υἱῶν τοῦ Σήμ 
ἐστιν ἡ κατοικία ἀπὸ Βάκτρων ἕως Ῥινοκορούρων 
τῆς ὁριζούσης Συρίαν καὶ Αἴγυπτον καὶ τὴν 
ἐρυθρὰν θάλασσαν ἀπὸ στόματος τοῦ κατὰ τὸν 
Ἀρσινοΐτην τῆς Ἰνδικῆς. 



Hippolytus Scr. Eccl., Chronicon 
Section 190, line 3

Ταῦτα δὲ τὰ ἐξ αὐτῶν γενόμενα ἔθνη· 
(1) Ἑβραῖοι <οἱ> καὶ Ἰουδαῖοι (2) Πέρσαι 
(3) Μῆδοι (4) Παίονες (5) Ἀρειανοί <(6) Ἀσσύ-  
ριοι> (7) Ὑρκάνιοι (8) Ἰνδοί (9) Μαγαρδοί 
(10) Πάρθοι (11) Γερμανοί (12) Ἐλυμαῖοι 
(13) Κοσσαῖοι (14) Ἄραβες [οἱ] πρῶτοι οἱ καλού-
μενοι Κεδρούσιοι (15) Ἄραβες δεύτεροι [οἱ καλού-
μενοι] (16) Γυμνοσοφισταί. 



Hippolytus Scr. Eccl., Chronicon 
Section 192, line 4

   Οἱ δὲ 
ἐπιστάμενοι αὐτῶν γράμματα οὗτοί εἰσιν· 
(1) Ἑβραῖοι οἱ καὶ Ἰουδαῖοι (2) Πέρσαι (3) Μῆδοι 
(4) Χαλδαῖοι (5) Ἰνδοί (6) Ἀσσύριοι. 



Hippolytus Scr. Eccl., Chronicon 
Section 195, line 3

   Ἐστὶ 
δὲ ἡ κατοικία τῶν υἱῶν τοῦ Σὴμ τοῦ πρωτοτόκου 
υἱοῦ Νῶε μῆκος μὲν ἀπὸ τῆς Ἰνδικῆς ἕως Ῥινοκο-
ρούρων, πλάτος δὲ ἀπὸ τῆς Περσίδος καὶ Βάκτρων 
ἕως τῆς Ἰνδικῆς. 



Hippolytus Scr. Eccl., Chronicon 
Section 194, line 5

Τὰ δὲ ὀνόματα τῶν χωρῶν τῶν υἱῶν τοῦ 
Σήμ ἐστι ταῦτα· 
         (1) Περσὶς σὺν τοῖς ἐπικει-
μένοις αὐτῇ ἔθνεσιν (2) Βακτριανή (3) Ὑρκανία 
(4) Βαβυλωνία (5) Κορδυλία (6) Ἀσσυρία (7) Μεσο-
ποταμία (8) Ἀραβία ἡ ἀρχαία (9) Ἐλυμαΐς 
(10) Ἰνδική (11) Ἀραβία ἡ εὐδαίμων <(12) Κοίλη 
Συρία> (13) Κομμαγηνή (14) καὶ ἡ Φοινίκη ἣπερ 
ἐστὶ τῶν υἱῶν τοῦ Σήμ. 



Hippolytus Scr. Eccl., Chronicon 
Section 200, line 6


Φαλὲκ καὶ Ἰεκτὰν τῶν δύο ἀδελφῶν κατὰ τὰς 
ἰδίας γλώσσας αὐτῶν ἐν τῇ πυργοποιίᾳ, ὅτε 
συνεχύθησαν αἱ γλῶσσαι αὐτῶν, ἐστὶ ταῦτα· 
(1) Ἑβραῖοι οἱ καὶ Ἰουδαῖοι (2) Ἀσσύριοι   
(3) Χαλδαῖοι (4) Μῆδοι (5) Πέρσαι <(6) Ἄραβες 
πρῶτοι καὶ δεύτεροι> (7) Μαδιηναῖοι πρῶτοι καὶ 
δεύτεροι (8) Ἀδιαβηνοί (9) Ταιηνοί (10) Σαλα-
μοσηνοί (11) Σαρακηνοί (12) Μάγοι (13) Κάσπιοι 
(14) Ἀλβανοί (15) Ἰνδοὶ πρῶτοι καὶ βʹ (16) Αἰ-
θίοπες πρῶτοι καὶ δεύτεροι (17) Αἰγύπτιοι καὶ 
Θηβαῖοι (18) Λίβυες [πρῶτοι καὶ βʹ] (19) Χετ-
ταῖοι (20) Χαναναῖοι (21) Φερεζαῖοι (22) Εὐαῖοι 
(23) Ἀμορραῖοι (24) Γεργεσαῖοι (25) Ἰεβουσσαῖοι 
(26) Ἰδουμαῖοι (27) Σαμαρρεῖοι (28) Φοίνικες 
(29) Σύροι (30) Κίλικες οἱ καὶ Θαρσεῖς (31) Καπ-
πάδοκες (32) Ἀρμένιοι (33) Ἴβηρες (34) Βιβρανοί 
(35) Σκύθαι (36) Κόλχοι (37) Σαῦνοι (38) Βος-
πορανοί (39) Ἀσιανοί (40) Ἰσαυροί (41) Λυκάονες   




Hippolytus Scr. Eccl., Chronicon 
Section 237, line 3

   Ποταμοὶ 
οὖν εἰσιν ὀνομαστοὶ τεσσαράκοντα ἐν τῇ γῇ οὗτοι· 
(1) Ἰνδὸς ὁ καλούμενος Φισών (2) Νεῖλος ὁ 
καλούμενος Γηών (3) Τίγρις (4) Εὐφράτης 
(5) Ἰορδάνης (6) Κηφισσός (7) Τάναϊς (8) Ἰς-
μηνός (9) Ἐρύμανθος (10) Ἅλυς (11) Αἰσωπός 
(12) Θερμώδων (13) Ἐρασῖνος (14) Ῥεῖος 
(15) Βορυσθένης (16) Ἀλφειός (17) Ταῦρος 
(18) Εὐρώτας (19) Μέανδρος (20) Ἄξιος (21) Πύ-  
ραμος (22) Ὀρέντης (23) Ἔβρων (24) Σαγγάριος 
(25) Ἀχελῷος (26) Πινειός (27) Εὔηνος (28) Σπερ-
χιός (29) Κάϋστρος (30) Σιμόεις (31) Σκάμανδρος 




Hippolytus Scr. Eccl., Refutatio omnium haeresium (= Philosophumena) (2115: 060)
“Hippolytus. Refutatio omnium haeresium”, Ed. Marcovich, M.
Berlin: De Gruyter, 1986; Patristische Texte und Studien 25.
Book 1, chapter pinax, section 6, line 3

Βραχμᾶνες οἱ ἐν Ἰνδοῖς, Δρυΐδαι οἱ ἐν Κελτοῖς, καὶ Ἡσίοδος. 



Hippolytus Scr. Eccl., Refutatio omnium haeresium (= Philosophumena) 
Book 1, chapter 13, section 1, line 2

Δημόκριτος δὲ Λευκίππου γίνεται γνώριμος· Δημόκριτος Δαμασίππου 
Ἀβδηρίτης, πολλοῖς συμβαλών, γυμνοσοφισταῖς ἐν Ἰνδοῖς καὶ ἱερεῦσιν 
ἐν Αἰγύπτῳ καὶ ἀστρολόγοις καὶ μάγοις ἐν Βαβυλῶνι. 



Hippolytus Scr. Eccl., Refutatio omnium haeresium (= Philosophumena) 
Book 1, chapter 24, section 1, line 1

Ἔστι δὲ καὶ παρὰ Ἰνδοῖς αἵρεσις φιλοσοφουμένων ἐν τοῖς Βραχμά-
ναις. 



Hippolytus Scr. Eccl., Refutatio omnium haeresium (= Philosophumena) 
Book 4, chapter 28, section 13, line 2

                                                                   αἱματώδη 
δὲ φλόγα ποιεῖ τὸ Ἰνδικὸν μέλαν ἐνθεὶς τῷ λιβανωτῷ, καθὼς προείπα-
μεν. 



Hippolytus Scr. Eccl., Refutatio omnium haeresium (= Philosophumena) 
Book 7, chapter 25, section 6, line 3

                              ἀλλὰ γὰρ καθάπερ ὁ νάφθας ὁ Ἰνδικός, 
ὀφθεὶς μόνον ἀπὸ πάνυ πολλοῦ διαστήματος, συνάπτει πῦρ, οὕτω κάτωθεν, 
ἀπὸ τῆς ἀμορφίας τοῦ σωροῦ, διήκουσιν, <φησίν,> αἱ δυνάμεις ἄνω 
μέχρι τῆς υἱότητος. 



Hippolytus Scr. Eccl., Refutatio omnium haeresium (= Philosophumena) 
Book 7, chapter 25, section 7, line 2

                        ἅπτει μὲν γὰρ καὶ λαμβάνει τὰ νοήματα κατὰ τὸν 
<ν>άφθαν τὸν Ἰνδικόν – οἷον ἄφθας<τός> τις ὤν – ὁ τοῦ μεγάλου 
τῆς Ὀγδοάδος ἄρχοντος υἱὸς ἀπὸ τῆς μετὰ τὸ μεθόριον <Πνεῦμα> 
μακαρίας υἱότητος· ἡ γὰρ ἐν μέσῳ τοῦ ἁγίου Πνεύματος, ἐν τῷ μεθορίῳ 
<μένουσα> τῆς υἱότητος δύναμις ῥέοντα καὶ φερόμενα τὰ νοήματα 
<ἀπὸ> τῆς <μακαρίας> υἱότητος μεταδίδωσι τῷ υἱῷ τοῦ μεγάλου 
ἄρχοντος. 



Hippolytus Scr. Eccl., Refutatio omnium haeresium (= Philosophumena) 
Book 8, chapter pinax, section 7, line 2

Τίς ἡ τῶν Ἐγκρατιτῶν κενοδοξία, καὶ ὅτι οὐκ ἐξ ἁγίων γραφῶν τὰ 
δόγματα αὐτῶν συνέστηκεν, ἀλλ' ἐξ αὑτῶν καὶ ἐκ τῶν παρ' Ἰνδοῖς 
γυμνοσοφιστῶν. 



Hippolytus Scr. Eccl., Refutatio omnium haeresium (= Philosophumena) 
Book 10, chapter 34, section 1, line 3

Τοιοῦτος ὁ περὶ τὸ θεῖον ἀληθὴς λόγος, ὦ ἄνθρωποι Ἔλληνές τε 
καὶ βάρβαροι, Χαλδαῖοί τε καὶ Ἀσσύριοι, Αἰγύπτιοί τε καὶ Λίβυες, 
Ἰνδοί τε καὶ Αἰθίοπες, Κελτοί τε καὶ οἱ στρατηγοῦντες Λατῖνοι, πάντες 
τε οἱ τὴν Εὐρώπην, Ἀσίαν τε καὶ Λιβύην κατοικοῦντες. 


\end{greek}



\section{Nicolaus of Damascus}
\blockquote[From Wikipedia\footnote{\url{http://en.wikipedia.org/wiki/Nicolaus_of_Damascus}}]{igation, search

Nicolaus of Damascus (Greek: Νικόλαος Δαμασκηνός, Nikolāos Damaskēnos) was a Greek[1] historian and philosopher who lived during the Augustan age of the Roman Empire. His name is derived from that of his birthplace, Damascus. He was born around 64 BC.[2]

He was an intimate friend of Herod the Great, whom he survived by a number of years. He was also the tutor of the children of Antony and Cleopatra (born in 40 BC), according to Sophronius.[3] He went to Rome with Herod Archelaus.[4]

His output was vast, but is nearly all lost. His chief work was a universal history in 144 books. He also wrote an autobiography, a life of Augustus, a life of Herod, some philosophical works, and some tragedies and comedies.

There is an article on him in the Suda.[5]}
\begin{greek}

Nicolaus Hist., Fragmenta (0577: 003)
“FHG 3”, Ed. Müller, K.
Paris: Didot, 1841–1870.
Fragment 7, line 1

Exc. De ins: Ὅτι μετὰ τὸν Ἰνδικὸν πόλεμον Σε-
μίραμις ἐπεὶ ὁδοιποροῦσα ἐγένετο ἐν Μήδοις, ἀναβᾶσα 
ἐπί τι ὑψηλὸν ὄρος, πάντοθεν πλὴν καθ' ἓν μέρος πε-
ριερρωγὸς καὶ ἄβατον λισσάδι καὶ ἀποτόμῳ πέτρᾳ, 
ἐθεᾶτο τὴν στρατιὰν (1) ἀπό τινος ἐξέδρας, ἣν παρα-
χρῆμα ᾠκοδομήσατο. 



Nicolaus Hist., Fragmenta 
Fragment 74, line 12

                       Μάρτυς δὲ τούτων ἡμῖν ἐστι καὶ 
Νικόλαος ὁ Δαμασκηνὸς, οὕτως ἱστορῶν· «Τρόπαιον δὲ 
στήσας Ἀντίοχος ἐπὶ τῷ Λύκῳ ποταμῷ, νικήσας Ἰν-
δάτην τὸν Πάρθων στρατηγὸν, αὐτόθι ἔμεινεν ἡμέρας 
δύο, δεηθέντος Ὑρκανοῦ τοῦ Ἰουδαίου διά τινα ἑορτὴν 
πάτριον, ἐν ᾗ τοῖς Ἰουδαίοις οὐκ ἦν νόμιμον ἐξοδεύειν. 



Nicolaus Hist., Fragmenta 
Fragment 91, line 4

                                             Φησὶ γὰρ 
οὗτος ἐν Ἀντιοχείᾳ τῇ ἐπὶ Δάφνῃ παρατυχεῖν τοῖς Ἰν-
δῶν πρέσβεσιν, ἀφιγμένοις παρὰ Καίσαρα τὸν Σεβα-
στόν. 



Nicolaus Hist., Fragmenta 
Fragment 91, line 28

        Ἐπιγεγράφθαι δὲ τῷ τάφῳ· 
ΖΑΡΜΑΝΟΣ ΧΗΓΑΝ 
ΙΝΔ*οΣ ΑΠΟ ΒΑΡΓΟΣΗΣ 
ΚΑΤΑ ΤΑ ΠΑΤΡΙΑ ΙΝΔ*ωΝ ΕΘΗ 
ΕΑΥΤΟΝ ΑΠΑΘΑΝΑΤΙΣΑΣ ΚΕΙΤΑΙ. 



Nicolaus Hist., Fragmenta 
Fragment 101, line 568

XXVI. Ὁ δὲ νεκρὸς ἐκεῖ (1) ἔκειτο ἔνθα ἔπεσεν 
ἀτίμως πεφυρμένος αἵματι, ἀνδρὸς ἐλάσαντος μὲν πρὸς 
ἑσπέραν ἄχρι Βρεττανῶν τε καὶ Ὠκεανοῦ, διανοουμέ-
νου δ' ἐλαύνειν πρὸς ἕω ἐπὶ τὰ Πάρθων ἀρχεῖα καὶ 
Ἰνδῶν, ὡς ἂν, κἀκείνων ὑπηκόων γενομένων, εἰς μίαν 
ἀρχὴν κεφαλαιωθείη γῆς πάσης καὶ θαλάττης τὰ κράτη· 
τότε δ' οὖν ἔκειτο, μηδενὸς τολμῶντος ὑπομένειν (2) 
καὶ τὸν νεκρὸν ἀναιρεῖσθαι. 



Nicolaus Hist., Fragmenta 
Fragment t143, line 1

                         Ἄθυροι δ' αὐτῶν αἱ οἰκίαι, 
καὶ ἐν ταῖς ὁδοῖς κειμένων πολλῶν οὐδὲ εἷς κλέπτει. 
ΙΝΔ*οΙ. 




Nicolaus Hist., Fragmenta 
Fragment 143, line 1

CXXIII, 12: Ἰνδοὶ συγκατακαίουσιν ὅταν τελευ-
τήσωσι τῶν γυναικῶν τὴν προσφιλεστάτην. 



Nicolaus Hist., Fragmenta 
Fragment 143, line 7

XLIV, 41: Παρ' Ἰνδοῖς ἐάν τις ἀποστερηθῇ δανείου 
ἢ παρακαταθήκης, οὐκ ἔστι κρίσις, ἀλλ' αὑτὸν αἰτιᾶ-
ται ὁ πιστεύσας. 



Nicolaus Hist., Fragmenta 
Fragment 145, line 1

IX, 52: Ἐν Παδαίοις, Ἰνδικῷ ἔθνει, οὐχ ὁ θύων, 
ἀλλ' ὁ συνετώτατος τῶν παρόντων κατάρχεται τῶν 
ἱερῶν. 

\end{greek}


\section{Philo of Alexandria}
\blockquote[From Wikipedia\footnote{\url{http://en.wikipedia.org/wiki/Philo}}]{Philo of Alexandria (Greek: Φίλων, Philōn; c. 20 BCE – c. 50 CE), also called Philo Judaeus, was a Hellenistic Jewish philosopher who lived in Alexandria, Egypt, during the Roman Empire.

Philo used philosophical allegory to attempt to fuse and harmonize Greek philosophy with Jewish philosophy. His method followed the practices of both Jewish exegesis and Stoic philosophy. His allegorical exegesis was important for several Christian Church Fathers, but he has barely any reception history within Judaism. He believed that literal interpretations of the Hebrew Bible would stifle humanity's view and perception of a God too complex and marvelous to be understood in literal human terms.}
\begin{greek}

Philo Judaeus Phil., De somniis (lib. i–ii) (0018: 019)
“Philonis Alexandrini opera quae supersunt, vol. 3”, Ed. Wendland, P.
Berlin: Reimer, 1898, Repr. 1962.
Book 2, section 56, line 3

                                          καὶ μὴν πρός γε ὕπνον μαλακὸν 
μὲν ἔδαφος αὔταρκες ἦν – ἐπεὶ καὶ μέχρι νῦν τοὺς Γυμνοσοφιστὰς παρ' 
Ἰνδοῖς χαμευνεῖν ἐκ παλαιῶν ἐθῶν κατέχει λόγος – , εἰ δὲ μή, στιβὰς 
γοῦν ἐκ λίθων λογάδων ἢ ξύλων εὐτελῶν πεποιημένη κλίνη. 



Philo Judaeus Phil., De somniis (lib. i-ii) 
Book 2, section 60, line 1

                                                                                ἀλλὰ 
γὰρ ἐπετειχίσθη τοῖς ὠφελοῦσι τὰ ἡδέα τῆς κενῆς δόξης ἀλείμματα, εἰς 
ἃ καὶ μυρεψοὶ πονοῦσι καὶ χῶραι μεγάλαι συντελοῦσι, Συρία, Βαβυλών, 
Ἰνδοί, Σκύθαι, παρ' οἷς αἱ τῶν ἀρωμάτων γενέσεις εἰσί. 



Philo Judaeus Phil., De Abrahamo (0018: 020)
“Philonis Alexandrini opera quae supersunt, vol. 4”, Ed. Cohn, L.
Berlin: Reimer, 1902, Repr. 1962.
Section 182, line 1

νοσημάτων λοιμικῶν γενησομένους, τοὺς δ' ὑπὲρ νενομισμένης εὐσεβείας, 
εἰ καὶ μὴ πρὸς ἀλήθειαν οὔσης· Ἑλλήνων μέν γε τοὺς δοκιμωτάτους, 
οὐκ ἰδιώτας μόνον ἀλλὰ καὶ βασιλεῖς, ὀλίγα φροντίσαντας ὧν ἐγέν-
νησαν διὰ τῆς τούτων ἀναιρέσεως δυνάμεις στρατευμάτων μεγάλας καὶ   
πολυανθρώπους ἐν μὲν τῇ συμμαχίᾳ τεταγμένας διασῶσαι, ἐν δὲ τῇ 
μερίδι τῶν ἐχθρῶν αὐτοβοεὶ διαφθεῖραι· βαρβαρικὰ δὲ ἔθνη μέχρι 
πολλοῦ παιδοκτονίαν ὡς ὅσιον ἔργον καὶ θεοφιλὲς προσέσθαι, ὧν μεμνῆ-
σθαι τοῦ ἄγους καὶ τὸν ἱερώτατον Μωυσῆν· αἰτιώμενος γὰρ αὐτοὺς 
τοῦ μιάσματος φάσκει, ὅτι “τοὺς υἱοὺς αὐτῶν καὶ τὰς θυγατέρας κατα-
καίουσι τοῖς θεοῖς αὐτῶν” (Deut. 12, 31)· Ἰνδῶν δὲ | τοὺς γυμνο-
σοφιστὰς ἄχρι νῦν, ἐπειδὰν ἄρχηται καταλαμβάνειν ἡ μακρὰ καὶ ἀνίατος 
νόσος, τὸ γῆρας, πρὶν βεβαίως κρατηθῆναι, πυρὰν νήσαντας ἑαυτοὺς 
ἐμπιπράναι, δυναμένους ἔτι πρὸς πολυετίαν ἴσως ἀντισχεῖν· ἤδη δὲ καὶ 
γύναια προαποθανόντων ἀνδρῶν ὁρμῆσαι γεγηθότα πρὸς τὴν αὐτὴν 
πυρὰν καὶ ζῶντα τοῖς ἐκείνων σώμασιν ὑπομεῖναι συγκαταφλεχθῆναι· 
ταῦτα μὲν εἰκότως ἄν τις τῆς εὐτολμίας θαυμάσειεν ἐκ πολλοῦ τοῦ 
περιόντος καταφρονητικῶς ἔχοντα θανάτου καὶ ὡς ἐπ' ἀθανασίαν 
αὐτὸν ἱέμενα καὶ ἀπνευστὶ θέοντα· τὸν δὲ τί προσῆκεν ἐπαινεῖν 
ὡς ἐγχειρητὴν κεκαινουργημένης πράξεως, ἣν καὶ ἰδιῶται καὶ βασιλεῖς 




Philo Judaeus Phil., De praemiis et poenis + De exsecrationibus (0018: 026)
“Philonis Alexandrini opera quae supersunt, vol. 5”, Ed. Cohn, L.
Berlin: Reimer, 1906, Repr. 1962.
Section 89, line 1

τότε μοι δοκοῦσιν ἄρκτοι καὶ λέοντες καὶ παρδάλεις καὶ τὰ παρ' Ἰνδοῖς, 
ἐλέφαντές τε καὶ τίγρεις, καὶ ὅσα ἄλλα τὰς ἀλκὰς καὶ τὰς δυνάμεις 
ἀήττητα μεταβαλεῖν ἐκ τοῦ μονωτικοῦ τε καὶ μονοτρόπου πρὸς τὸ 
σύννομον· κἀκ τοῦ πρὸς ὀλίγον μιμήσει τῶν ἀγελαίων ἡμερωθήσεται 
πρὸς τὴν ἀνθρώπου φαντασίαν, μηκέτι ὡς πρότερον ἀνερεθισθέντα, κατα-
πλαγέντα δ' ὡς ἄρχοντα καὶ φύσει δεσπότην εὐλαβῶς ἕξει, ἔνια δὲ καὶ 
τοῦ χειροήθους ἅμα καὶ φιλοδεσπότου τῇ παραζηλώσει, καθάπερ τὰ 
Μελιταῖα τῶν κυνιδίων ταῖς κέρκοις μεθ' ἱλαρωτέρας κινήσεως προσσαί-
νοντα. 



Philo Judaeus Phil., Quod omnis probus liber sit (0018: 027)
“Philonis Alexandrini opera quae supersunt, vol. 6”, Ed. Cohn, L., Reiter, S.
Berlin: Reimer, 1915, Repr. 1962.
Section 74, line 5

                      κατὰ δὲ τὴν βάρβαρον, ἐν ᾗ πρεσβευταὶ λόγων καὶ 
ἔργων, πολυανθρωπότατα στίφη καλῶν καὶ ἀγαθῶν ἐστιν ἀνδρῶν· ἐν 
Πέρσαις μὲν τὸ μάγων, οἳ τὰ φύσεως ἔργα διερευνώμενοι πρὸς ἐπίγνωσιν 
τῆς ἀληθείας καθ' ἡσυχίαν τὰς θείας ἀρετὰς τρανοτέραις ἐμφάσεσιν 
ἱεροφαντοῦνταί τε καὶ ἱεροφαντοῦσιν· ἐν Ἰνδοῖς δὲ τὸ γυμνοσοφιστῶν, οἳ 
πρὸς τῇ φυσικῇ καὶ τὴν ἠθικὴν φιλοσοφίαν | διαπονοῦντες ὅλον ἐπίδειξιν 
ἀρετῆς πεποίηνται τὸν βίον. 



Philo Judaeus Phil., Quod omnis probus liber sit 
Section 93, line 1

                                    Κάλανος ἦν Ἰνδὸς γένος τῶν γυμνοσο-
φιστῶν· οὗτος καρτερικώτατος τῶν κατ' αὐτὸν ἁπάντων | νομισθεὶς οὐ 
μόνον ὑπὸ τῶν ἐγχωρίων ἀλλὰ καὶ πρὸς ἀλλοφύλων, ὃ δὴ σπανιώτατόν 
ἐστιν, ἐχθρῶν βασιλέων ἐθαυμάσθη, λόγοις ἐπαινετοῖς σπουδαῖα ἔργα 
συνυφήνας. 



Philo Judaeus Phil., Quod omnis probus liber sit 
Section 96, line 3

Φίλοι πείθουσί σε χεῖρας καὶ ἀνάγκην προσφέρειν Ἰνδῶν φιλοσόφοις 
οὐδ' ἐν ὕπνοις ἑωρακότες ἡμέτερα ἔργα. 



Philo Judaeus Phil., De aeternitate mundi (0018: 029)
“Philonis Alexandrini opera quae supersunt, vol. 6”, Ed. Cohn, L., Reiter, S.
Berlin: Reimer, 1915, Repr. 1962.
Section 128, line 2

                                                τὸ παραπλήσιον μέντοι καὶ 
τοὺς κατὰ τὴν Ἰνδικὴν δράκοντάς φασι πάσχειν· ἀνέρποντας γὰρ ἐπὶ τὰ   
μέγιστα τῶν ζῴων, ἐλέφαντας, περὶ νῶτα καὶ νηδὺν ἅπασαν εἱλεῖσθαι, 
φλέβα δ' ἣν ἂν τύχῃ διελόντας ἐμπίνειν τοῦ αἵματος, ἀπλήστως ἐπισπω-
μένους βιαίῳ πνεύματι καὶ συντόνῳ ῥοίζῳ· μέχρι μὲν οὖν τινος ἐξανα-
λουμένους ἐκείνους ἀντέχειν ὑπ' ἀμηχανίας ἀνασκιρτῶντας καὶ τῇ προνο-
μαίᾳ τὴν πλευρὰν τύπτοντας ὡς καθιξομένους τῶν δρακόντων, εἶτα ἀεὶ 
κενουμένου τοῦ ζωτικοῦ, πηδᾶν μὲν μηκέτι δύνασθαι, κραδαινομένους δ' 
ἑστάναι, μικρὸν δ' ὕστερον καὶ τῶν σκελῶν ἐξασθενησάντων, κατασει-
σθέντας ὑπὸ λιφαιμίας ἀποψύχειν, πεσόντας δὲ τοὺς αἰτίους τοῦ θανάτου 
συναπολλύναι τρόπῳ τοιῷδε· μηκέτ' ἔχοντες τροφὴν οἱ δράκοντες ὃν 




Philo Judaeus Phil., De ebrietate ii (0018: 036)
“Neu entdeckte Fragmente Philos nebst einer Untersuchung über die ursprüngliche Gestalt der Schrift de sacrificiis Abelis et Caini”, Ed. Wendland, P.
Berlin: Reimer, 1891.
Fragment 5, line 4

Τίς οὐκ οἶδεν ὅτι κριθαῖς μὲν ἔθνη τρέφεται μεγάλα καθάπερ τὰ 
πολλὰ τῶν Λιβύην νεμομένων, ὀλύραις δὲ ἕτερα, καὶ τὸ πολυανθρωπότατον 
Ἰνδῶν ὀρύζῃ; 

\end{greek}



\section{De bello Parthico}
\blockquote[From Wikipedia\footnote{\url{}}]{}
\begin{greek}

Anonymi Historici (FGrH), De bello Parthico (1139: 013)
“FGrH \#203”.
Volume-Jacobyʹ-T+F 2b,203,F, fragment 7b, line 2

   τὰ δ' ἐν 
Ἰνδοῖς πραχθησόμενα ὑπέσχετο ἤδη συγγράψειν καὶ τὸν περίπλουν τῆς 
ἔξω θαλάσσης. 



Anonymi Historici (FGrH), De bello Parthico 
Volume-Jacobyʹ-T+F 2b,203,F, fragment 7b, line 4

                 καὶ οὐχ ὑπόσχεσις ταῦτα μόνον, ἀλλὰ καὶ τὸ προοίμιον 
τῆς Ἰνδικῆς ἤδη συντέτακται, καὶ τὸ τρίτον τάγμα καὶ οἱ Κελτοὶ καὶ 
Μαύρων μοῖρα ὀλίγη σὺν Κασσίωι πάντες οὗτοι ἐπεραιώθησαν τὸν Ἰνδὸν 
ποταμόν· ὅ τι δὲ καὶ πράξουσιν ἢ πῶς δέξονται τὴν τῶν ἐλεφάντων ἐπέ-
λασιν, οὐκ εἰς μακρὰν ἡμῖν ὁ θαυμαστὸς συγγραφεὺς ἀπὸ Μουζίριδος ἢ ἀπ' 
Ὀξυδρακῶν ἐπιστελεῖ. 

\end{greek}


\section{Clemens Alexandrinus}
\blockquote[From Wikipedia\footnote{\url{http://en.wikipedia.org/wiki/Clemens_Alexandrinus}}]{Titus Flavius Clemens (Greek: Κλήμης ὁ Ἀλεξανδρεύς; c. 150 – c. 215), known as Clement of Alexandria, was a Christian theologian who taught at the Catechetical School of Alexandria. A convert to Christianity, he was an educated man who was familiar with classical Greek philosophy and literature. As his three major works demonstrate, Clement was influenced by Hellenistic philosophy to a greater extent than any other Christian thinker of his time, and in particular by Plato and the Stoics.[1] His secret works, which exist only in fragments, suggest that he was also familiar with pre-Christian Jewish esotericism and Gnosticism. In one of his works he argued that Greek philosophy had its origin among non-Greeks, claiming that both Plato and Pythagoras were taught by Egyptian scholars.[2] Among his pupils were Origen and Alexander of Jerusalem.

The Protrepticus is, as its title suggests, an exhortation to the pagans of Greece to adopt Christianity, and within it Clement demonstrates his extensive knowledge of pagan mythology and theology. It is chiefly important due to Clement's exposition of religion as an anthropological phenomenon.[14] After a short philosophical discussion, it opens with a history of Greek religion in seven stages.[15] Clement suggests that at first, men mistakenly believed the Sun, the Moon and other heavenly bodies to be gods. The next development was the worship of the products of agriculture, from which he contends the cults of Demeter and Dionysus arose.[16] Man then paid reverence to revenge, and deified human feelings of love and fear, among others. In the following stage, the poets Hesiod and Homer attempt to enumerate the Gods; Hesiod's Theogony giving the number of twelve. Finally, men proclaimed other men, such as Asclepius and Heracles, deities.[16] Discussing idolatry, Clement contends that the objects of primitive religion were unshaped wood and stone, and idols thus arose when such natural items were carved.[17] Following Plato, Clement is critical of all forms of visual art, suggesting that artworks are but illusions and "deadly toys".[17]}

\begin{greek}


Clemens Alexandrinus Theol., Protrepticus (0555: 001)
“Clément d'Alexandrie. Le protreptique, 2nd edn.”, Ed. Mondésert, C.
Paris: Cerf, 1949; Sources chrétiennes 2.
Chapter 2, section 26, subsection 2, line 1

Οἳ μὲν γὰρ εὐθέως ἀμφὶ τὴν οὐρανοῦ θέαν ἀπατώ-
μενοι καὶ ὄψει μόνῃ πεπιστευκότες τῶν ἀστέρων τὰς κινήσεις   
ἐπιθεώμενοι ἐθαύμασάν τε καὶ ἐξεθείασαν, θεοὺς ἐκ τοῦ θεῖν 
ὀνομάσαντες τοὺς ἀστέρας, καὶ προσεκύνησαν ἥλιον, ὡς 
Ἰνδοί, καὶ σελήνην, ὡς Φρύγες· οἳ δὲ τῶν ἐκ γῆς φυομένων 
τοὺς ἡμέρους δρεπόμενοι καρποὺς Δηὼ τὸν σῖτον, ὡς 
Ἀθηναῖοι, καὶ Διόνυσον τὴν ἄμπελον, ὡς Θηβαῖοι, προση-
γόρευσαν. 



Clemens Alexandrinus Theol., Protrepticus 
Chapter 4, section 47, subsection 6, line 6

                                                  Πολλοὶ δ' ἂν 
τάχα που θαυμάσειαν, εἰ μάθοιεν τὸ Παλλάδιον τὸ διοπετὲς 
καλούμενον, ὃ Διομήδης καὶ Ὀδυσσεὺς ἱστοροῦνται μὲν 
ὑφελέσθαι ἀπὸ Ἰλίου, παρακαταθέσθαι δὲ Δημοφῶντι, ἐκ 
τῶν Πέλοπος ὀστῶν κατεσκευάσθαι, καθάπερ τὸν Ὀλύμπιον 
ἐξ ἄλλων ὀστῶν Ἰνδικοῦ θηρίου. 





Clemens Alexandrinus Theol., Paedagogus (0555: 002)
“Clément d'Alexandrie. Le pédagogue, 3 vols.”, Ed. Marrou, H.–I., Harl, M., Mondésert, C., Matray, C.
Paris: Cerf, 1:1960; 2:1965; 3:1970; Sources chrétiennes 70, 108, 158.
Book 2, chapter 2, subchapter 30, section 3, line 3

Εἰ μή τι καὶ τὸ ὕδωρ ἐποίσονται, ὡς οἱ βασιλεῖς οἱ ἀνόητοι [Χόασπις 
ποταμὸς οὕτω λεγόμενος τῆς Ἰνδικῆς, οὗ κάλλιστον ὕδωρ εἰς πόσιν] 
τὸ Χοάσπειον, καθάπερ καὶ τοὺς φίλους, οὕτω δὲ καὶ τὸ ὕδωρ 
ἐπαγόμενοι. 



Clemens Alexandrinus Theol., Paedagogus 
Book 2, chapter 3, subchapter 37, section 2, line 3

Ἢ ἐπὶ τὴν μοῖραν τοῦ κρέως Ἰνδικὸν σίδηρον χαλκευτέον, καθάπερ 
συμμαχικόν τι παρακαλοῦντας; 



Clemens Alexandrinus Theol., Paedagogus 
Book 2, chapter 10bis, subchapter 107, section 3, line 4

        Εἰ δὲ συμπεριφέρεσθαι χρή, ὀλίγον ἐνδοτέον αὐταῖς μαλα-
κωτέροις χρῆσθαι τοῖς ὑφάσμασιν μόνον τὰς μεμωρημένας λεπτουρ-
γίας καὶ τὰς ἐν ταῖς ὑφαῖς περιέργους πλοκὰς ἐκποδὼν μεθιστάντας, 
νῆμα χρυσοῦ καὶ σῆρας Ἰνδικοὺς καὶ τοὺς περιέργους βόμβυκας   
χαίρειν ἐῶντας. 



Clemens Alexandrinus Theol., Paedagogus 
Book 3, chapter 2, subchapter 4, section 2, line 6

Αἳ δέ, τὴν ἐπιφάνειαν καλλωπιζόμεναι καὶ τὰ βάθη 
χερσούμεναι, λελήθασι σφᾶς αὐτὰς κατὰ τοὺς Αἰγυπτίων 
κοσμοῦσαι ναούς· προπύλαια παρ' αὐτοῖς καὶ προτεμε-
νίσματα ἐξήσκηται, ἄλση τε καὶ ὀργάδες, κίοσίν τε 
παμπόλλοις ἐστεφάνωνται αἱ αὐλαί· τοῖχοι δὲ ἀποστίλβουσι 
ξενικοῖς λίθοις καὶ γραφῆς ἐντέχνου [οἷς] ἐνδεῖ οὐδὲ ἕν· 
χρυσῷ δὲ καὶ ἀργύρῳ καὶ ἠλέκτρῳ παραστίλβουσιν οἱ 
ναοί, καὶ τοῖς ἀπὸ Ἰνδίας καὶ Αἰθιοπίας πεποικιλμένοι μαρ-
μαίρουσι λιθιδίοις, τὰ δὲ ἄδυτα χρυσοπάστοις ἐπισκιάζεται 
πέπλοις· ἀλλ' ἢν παρεισέλθῃς <εἰς> τὸ βάθος τοῦ περιβόλου 
σπεύδων ἐπὶ τὴν θέαν τοῦ κρείττονος, ζητήσῃς <δὲ> τὸ 
ἄγαλμα τὸ ἔνοικον τοῦ νεώ, παστοφόρος [δὲ] ἤ τις ἄλλος 
τῶν ἱεροποιούντων περὶ τὸ τέμενος σεμνὸν δεδορκώς, 
παιᾶνα τῇ Αἰγυπτίων ᾄδων γλώττῃ, ὀλίγον ἐπαναστείλας 
τοῦ καταπετάσματος, ὡς δείξων τὸν θεόν, πλατὺν ἡμῖν 
ἐνδίδωσι γέλωτα τοῦ σεβάσματος. 



Clemens Alexandrinus Theol., Paedagogus 
Book 3, chapter 2, subchapter 10, section 1, line 7

                        Τράπεζα πλήρης καὶ κύλικες ἐπάλ-
ληλοι ἱκαναὶ πληρῶσαι τὴν λαιμαργίαν· τοῖς δὲ φιλοχρύσοις 
καὶ φιλοπορφύροις καὶ φιλολίθοις οὔτε ὁ ὑπὲρ γῆς καὶ 
ὑπὸ γῆν χρυσὸς αὐτάρκης οὔτε ἡ Τυρίων θάλαττα οὔτε 
μὴν ὁ φόρτος ὁ ἀπ' Ἰνδῶν καὶ Αἰθιόπων, ἀλλ' οὐδὲ ὁ 
Πακτωλὸς ὁ ῥέων τὸν πλοῦτον. 



Clemens Alexandrinus Theol., Paedagogus 
Book 3, chapter 4, subchapter 30, section 1, line 1

Ἀλλ' αἵ γε ἀστειότεραι τούτων ὄρνεις Ἰνδικοὺς καὶ 
ταῶνας Μηδικοὺς ἐκτρέφουσιν καὶ συνανακλίνονται τοῖς 
φοξοῖς παίζουσαι, σικίννοις τέρασι γανύμεναι· καὶ τὸν 
μὲν Θερσίτην ἀκούουσαι γελῶσιν, αὐταὶ δὲ πολυτιμήτους 
ὠνούμεναι Θερσίτας οὐκ ἐπ' ἀνδράσιν ὁμοζύγοις, ἀλλ' 
ἐπ' ἐκείνοις αὐχοῦσιν, ἃ δὴ ἄχθος ἐστὶ γῆς· καὶ χήραν μὲν 
παρορῶσι σωφρονοῦσαν Μελιταίου πολλῷ διαφέρουσαν 
κυνιδίου, καὶ πρεσβύτην παραβλέπουσι δίκαιον, εὐπρεπές-
τερον, οἶμαι, τέρατος ἀργυρωνήτου· παιδίον δὲ οὐδὲ 
προσίενται ὀρφανὸν αἱ τοὺς ψιττακοὺς καὶ τοὺς χαραδριοὺς 




Clemens Alexandrinus Theol., Paedagogus 
Book 3, chapter 4, subchapter 30, section 4, line 4

                                                            Αἳ 
δὲ ἔμπαλιν ἀπαιδευσίαν προτετιμήκασι σωφροσύνης, τὰς 
ἑαυτῶν οὐσίας ἀπολιθοῦσαι εἰς τοὺς μαργαρίτας καὶ τὰς 
σμαράγδους τὰς Ἰνδικάς· ναὶ μὴν καὶ εἰς τὰς ἐξιτήλους 
βαφὰς καὶ εἰς τὰ ἀργυρώνητα ἀνδράποδα σπαθῶσι καὶ 
διαρρίπτουσι τὰ χρήματα, δίκην ὀρνίθων κεκορεσμένων 
τὰ τοῦ βίου σκαλεύουσαι κόπρια. 



Clemens Alexandrinus Theol., Stromata (0555: 004)
“Clemens Alexandrinus, vols. 2, 3rd edn. and 3, 2nd edn.”, Ed. Stählin, O., Früchtel, L., Treu, U.
Berlin: Akademie–Verlag, 2:1960; 3:1970; Die griechischen christlichen Schriftsteller 52(15), 17.
Book 1, chapter 15, section 71, subsection 4, line 5

           προέστησαν δ' αὐτῆς Αἰγυπτίων τε οἱ προφῆται καὶ 
Ἀσσυρίων οἱ Χαλδαῖοι καὶ Γαλατῶν οἱ Δρυΐδαι καὶ Σαμαναῖοι Βάκ-
τρων καὶ Κελτῶν οἱ φιλοσοφήσαντες καὶ Περσῶν οἱ Μάγοι (οἳ μα-
γείᾳ καὶ τοῦ σωτῆρος προεμήνυσαν τὴν γένεσιν, ἀστέρος αὐτοῖς καθη-
γουμένου εἰς τὴν Ἰουδαίαν ἀφικνούμενοι γῆν]1 Ἰνδῶν τε οἱ γυμνοσο-
φισταί, ἄλλοι γε φιλόσοφοι βάρβαροι. 



Clemens Alexandrinus Theol., Stromata 
Book 1, chapter 15, section 71, subsection 6, line 1

                                                      εἰσὶ δὲ τῶν Ἰνδῶν οἱ 
τοῖς Βούττα πειθόμενοι παραγγέλμασιν. 



Clemens Alexandrinus Theol., Stromata 
Book 1, chapter 15, section 72, subsection 5, line 3

                                          φανερώτατα δὲ Μεγασθένης ὁ 
συγγραφεὺς ὁ Σελεύκῳ τῷ Νικάτορι συμβεβιωκὼς ἐν τῇ τρίτῃ τῶν 
Ἰνδικῶν ὧδε γράφει· «ἅπαντα μέντοι τὰ περὶ φύσεως εἰρημένα παρὰ 
τοῖς ἀρχαίοις λέγεται καὶ παρὰ τοῖς ἔξω τῆς Ἑλλάδος φιλοσοφοῦσι, 
τὰ μὲν παρ' Ἰνδοῖς ὑπὸ τῶν Βραχμάνων, τὰ δὲ ἐν τῇ Συρίᾳ ὑπὸ 
τῶν καλουμένων Ἰουδαίων. 



Clemens Alexandrinus Theol., Stromata 
Book 2, chapter 20, section 125, subsection 1, line 1

Καλῶς ὁ Ζήνων ἐπὶ τῶν Ἰνδῶν ἔλεγεν ἕνα Ἰνδὸν παροπτώμε-
νον ἐθέλειν <ἂν> ἰδεῖν ἢ πάσας τὰς περὶ πόνου ἀποδείξεις μαθεῖν. 



Clemens Alexandrinus Theol., Stromata 
Book 3, chapter 7, section 60, subsection 2, line 4

         Βραχμᾶναι γοῦν οὔτε ἔμψυχον ἐσθίουσιν οὔτε οἶνον πίνου-
σιν· ἀλλ' οἳ μὲν αὐτῶν καθ' ἑκάστην ἡμέραν ὡς ἡμεῖς τὴν τροφὴν 
προσίενται, ἔνιοι δ' αὐτῶν διὰ τριῶν ἡμερῶν, ὥς φησιν Ἀλέξανδρος 
ὁ Πολυΐστωρ ἐν τοῖς Ἰνδικοῖς· καταφρονοῦσι δὲ θανάτου καὶ παρ'   
οὐδὲν ἡγοῦνται τὸ ζῆν· πείθονται γὰρ εἶναι παλιγγενεσίαν, θεοὺς δὲ 
σέβουσιν Ἡρακλέα καὶ Πᾶνα. 



Clemens Alexandrinus Theol., Stromata 
Book 3, chapter 7, section 60, subsection 3, line 1

                                  οἱ καλούμενοι δὲ Σεμνοὶ τῶν Ἰνδῶν 
γυμνοὶ διαιτῶνται τὸν πάντα βίον· οὗτοι τὴν ἀλήθειαν ἀσκοῦσι καὶ 
περὶ τῶν μελλόντων προμηνύουσι καὶ σέβουσί τινα πυραμίδα, ὑφ' 
ἣν ὀστέα τινὸς θεοῦ νομίζουσιν ἀποκεῖσθαι. 



Clemens Alexandrinus Theol., Stromata 
Book 4, chapter 4, section 17, subsection 3, line 3

                                       οὐ γὰρ τὸν χαρακτῆρα σῴζουσι 
τοῦ μαρτυρίου τοῦ πιστοῦ, τὸν ὄντως θεὸν μὴ γνωρίσαντες, θανάτῳ 
δὲ ἑαυτοὺς ἐπιδιδόασι κενῷ, καθάπερ καὶ οἱ τῶν Ἰνδῶν γυμνοσο-
φισταὶ ματαίῳ πυρί. 



Clemens Alexandrinus Theol., Stromata 
Book 4, chapter 7, section 50, subsection 1, line 1

               εἴη δ' ἂν ὁ τοιοῦτος ὁ κολλώμενος τῷ ἀγαθῷ κατὰ 
τὸν ἀπόστολον, ἀποστυγῶν τὸ πονηρόν, ἀγάπην ἔχων ἀνυπόκριτον· 
»ὁ γὰρ ἀγαπῶν τὸν ἕτερον νόμον πεπλήρωκεν·» εἰ δὲ «ὁ θεὸς τῆς 
ἐλπίδος» οὗτός ἐστιν ᾧ μαρτυροῦμεν, ὥσπερ οὖν ἐστι, τὴν ἐλπίδα 
ἡμῶν ὁμολογοῦμεν εἰς τὴν ἐλπίδα σπεύδοντες· οἱ δὲ «μεστοὶ τῆς 
ἀγαθωσύνης πεπληρωμένοι», φησί, «πάσης τῆς γνώσεως.» 
 Ἰνδῶν οἱ φιλόσοφοι Ἀλεξάνδρῳ λέγουσι τῷ Μακεδόνι· «σώματα 
μὲν μετάξεις ἐκ τόπου εἰς τόπον, ψυχὰς δ' ἡμετέρας οὐκ ἀναγκάσεις 
ποιεῖν ἃ μὴ βουλόμεθα. 



Clemens Alexandrinus Theol., Stromata 
Book 6, chapter 4, section 38, subsection 1, line 1

Καὶ τὰ μὲν Αἰγυπτίων ὡς ἐν βραχεῖ φάναι τοιαῦτα· Ἰνδῶν δὲ 
ἡ φιλοσοφία καὶ αὐτῶν διαβεβόηται. 



Clemens Alexandrinus Theol., Stromata 
Book 6, chapter 4, section 38, subsection 2, line 1

Ἀλέξανδρος γοῦν ὁ Μακεδὼν δέκα λαβὼν Ἰνδῶν γυμνοσοφιστὰς 
τοὺς δοκοῦντας ἀρίστους εἶναι καὶ βραχυλογωτάτους προβλήματα 
αὐτοῖς προὔθηκε, τὸν μὴ ἀποκρινόμενον εὐστόχως ἀνελεῖν ἀπειλήσας, 
ἕνα <δὲ> τὸν πρεσβύτατον αὐτῶν ἐπικρίνειν κελεύσας. 



Clemens Alexandrinus Theol., Stromata 
Book 6, chapter 7, section 57, subsection 3, line 5

                                   Κλεάνθης μὲν γὰρ Ζήνωνα ἐπιγρά-
φεται καὶ Θεόφραστος Ἀριστοτέλη Μητρόδωρός τε Ἐπίκουρον καὶ 
Πλάτων Σωκράτην· ἀλλὰ κἂν ἐπὶ Πυθαγόραν ἔλθω καὶ Φερεκύδην 
καὶ Θάλητα καὶ τοὺς πρώτους σοφούς, ἵσταμαι τὸν τούτων διδάσκα-
λον ζητῶν κἂν Αἰγυπτίους εἴπῃς κἂν Ἰνδοὺς κἂν Βαβυλωνίους κἂν 
τοὺς Μάγους αὐτούς, οὐ παύσομαι τὸν τούτων διδάσκαλον ἀπαιτῶν, 
ἀνάγω δέ σε καὶ ἐπὶ τὴν πρώτην γένεσιν ἀνθρώπων, κἀκεῖθεν ἄρ-
χομαι ζητεῖν, τίς ὁ διδάσκαλος; 


\end{greek}


\section{Eutropius / Paeanii translatio}
\blockquote[From Wikipedia\footnote{\url{http://en.wikipedia.org/wiki/Eutropius_\%28historian\%29}}]{Flavius Eutropius was an Ancient Roman historian who flourished in the latter half of the 4th century. He held the office of secretary (magister memoriae) at Constantinople, accompanied the Emperor Julian (361–363) on his expedition against the Persians (363), and was alive during the reign of Valens (364–378), to whom he dedicates his Breviarium historiae Romanae and where his history ends.

The Breviarium historiae Romanae is a complete compendium, in ten books, of Roman history from the foundation of the city to the accession of Valens. It was compiled with considerable care from the best accessible authorities, and is written generally with impartiality, and in a clear and simple style. Although the Latin in some instances differs from that of the purest models, the work was for a long time a favorite elementary school-book. Its independent value is small, but it sometimes fills a gap left by the more authoritative records. For the early parts of his work, Eutropius depended upon an epitome of Livy; for the later parts, he used the now lost Enmannsche Kaisergeschichte. The Breviarium was enlarged and continued down to the time of Justinian by Paulus Diaconus; the work of the latter was in turn enlarged by Landolfus Sagax (c. 1000), and taken down to the time of the emperor Leo the Armenian (813–820) in the Historia Miscella.

Of the Greek translations by Paeanius (around 380) and Capito Lycius (6th century), the version of the former is extant in an almost complete state. The best edition of Eutropius is by H. Droysen (1879), containing the Greek version and the enlarged editions of Paulus Diaconus and Landolfus. There are numerous English editions and translations.

Paeanius was a late Roman Empire historian, author of a translation into Greek language of the Latin Breviarium historiae Romanae, the historical work of Eutropius.

The Breviarium was completed by Eutropius within 369: Paeanius' translation appeared before 380. The Breviarium was a compendium of ancient Roman history, used both as a textbook in schools and as a fast course on Roman history for the higher social classes (it was dedicated to Emperor Valens): Paeanius' translation allowed Greek-speaking people to have a graceful version of this compendium.
}
\begin{greek}

Eutropius Hist., Breviarium ab urbe condita (Paeanii translatio) (2236: 001)
“”Παιανίου μετάφρασις εἰς τὴν τοῦ Εὐτροπίου Ῥωμαϊκὴν ἱστορίαν””, Ed. Lambros, S.P., 1912; Νέος Ἑλληνομνήμων 9.
Book 7, chapter 10, line 1

Πρὸς αὐτὸν καὶ Ἰνδοὶ πρεσβείαν ἔστειλαν, οὔπω πρότερον 
τὴν ῥωμαϊκὴν βασιλείαν οὐδὲ ἀκοῇ δεδεγμένοι. 



Eutropius Hist., Breviarium ab urbe condita (Paeanii translatio) 
Book 8, chapter 3, line 9

      Μεσσηνίους τε μάχῃ νικήσας κτῆμά τε καὶ τούτους αὐτῷ 
ποιησάμενος, ἄχρις αὐτῶν ἐχώρησεν Ἰνδῶν καὶ τῆς Ἐρυθρᾶς κα-
λουμένης θαλάσσης. 



Eutropius Hist., Breviarium ab urbe condita (Paeanii translatio) 
Book 8, chapter 3, line 14

                                    Πρῶτος καὶ τὴν Ἀραβίαν ἐπαρχίαν 
ἐποίησε στόλον τε ἐγκατέστησε κατὰ τὴν Ἐρυθρὰν θάλασσαν, ὥςτε 
αὐτῷ καὶ διὰ θαλάττης ἐξεῖναι πολιορκεῖν Ἰνδούς. 

\end{greek}


\section{Longus}
\blockquote[From Wikipedia\footnote{\url{http://en.wikipedia.org/wiki/Longus}}]{Longus, sometimes Longos (Greek: Λόγγος), was the author of an ancient Greek novel or romance, Daphnis and Chloe. Very little is known of his life, and it is assumed that he lived on the isle of Lesbos (setting for Daphnis and Chloe) during the 2nd century AD

It has been suggested[by whom?] that the name Longus is merely a misinterpretation of the last word of Daphnis and Chloe's title Λεσβιακῶν ἐρωτικῶν λόγοι ("story of a Lesbian romance", "Lesbian" for "from Lesbos island") in the Florentine manuscript; Seiler[citation needed] also observes that the best manuscript begins and ends with λόγου (not λόγγου) ποιμενικῶν. If his name was really Longus, he was probably a freedman of some Roman family which bore that name as a cognomen.}
\begin{greek}

Longus Scr. Erot., Daphnis et Chloe (0561: 001)
“Longus. Pastorales (Daphnis et Chloé)”, Ed. Dalmeyda, G.
Paris: Les Belles Lettres, 1934, Repr. 1971.
Book 4, chapter 3, section 2, line 3

Εἶχε δὲ καὶ ἔνδοθεν ὁ νεὼς Διονυσιακὰς γραφάς· 
Σεμέλην τίκτουσαν, Ἀριάδνην καθεύδουσαν, Λυκοῦργον 
δεδεμένον, Πενθέα διαιρούμενον· ἐπῆσαν καὶ Ἰνδοὶ νικώ-
μενοι καὶ Τυρρηνοὶ μεταμορφούμενοι· πανταχοῦ Σάτυροι 
<πατοῦντες>, πανταχοῦ Βάκχαι χορεύουσαι· οὐδὲ ὁ Πὰν 
ἠμέλητο· ἐκαθέζετο δὲ καὶ αὐτὸς συρίζων ἐπὶ πέτρας, 
ὅμοιος ἐνδιδόντι κοινὸν μέλος καὶ τοῖς πατοῦσι καὶ ταῖς 
χορευούσαις. 

\end{greek}


\section{Pseudo-Apollodorus}
\blockquote[From Wikipedia\footnote{\url{http://en.wikipedia.org/wiki/Pseudo-Apollodorus}}]{The Bibliotheca (Greek: Βιβλιοθήκη, Bibliothēkē, "library") is an ancient Greek compendium of myths and heroic legends, arranged in three books. It was known traditionally as the Library of Apollodorus, but the attribution is now regarded as false. The Bibliotheca has been called "the most valuable mythographical work that has come down from ancient times".[1] An epigram recorded by Photius expressed its purpose:

    It has the following not ungraceful epigram: 'Draw your knowledge of the past from me and read the ancient tales of learned lore. Look neither at the page of Homer, nor of elegy, nor tragic muse, nor epic strain. Seek not the vaunted verse of the cycle; but look in me and you will find in me all that the world contains'.[2]

The brief and unadorned accounts of myth in the Bibliotheca have led some commentators to suggest that even its complete sections are an epitome of a lost work.

Pseudo-Apollodorus

A certain "Apollodorus" is indicated as author on some surviving manuscripts (Diller 1983). This Apollodorus has been mistakenly identified with Apollodorus of Athens (born c. 180 BC), a student of Aristarchus of Samothrace, mainly as it is known—from references in the minor scholia on Homer—that Apollodorus of Athens did leave a similar comprehensive repertory on mythology, in the form of a verse chronicle. The text that we[who?] possess, however, cites a Roman author: Castor the Annalist, a contemporary of Cicero in the 1st century BC. The mistaken attribution was made by scholars from Photius onwards.[citation needed] Since for chronological reasons Apollodorus of Athens could not have written the book, the author of the Bibliotheca is conventionally called the "Pseudo-Apollodorus" by those wishing to be scrupulously correct. Traditional references simply instance "the Library and Epitome".

One of his many sources was the Tragodoumena (Subjects of Tragedies) a 4th-century B.C. analysis of the myths in Greek tragedies by Asclepiades of Tragilus,[3] the first known Greek mythographic compilation.[4]}

\begin{greek}

Pseudo-Apollodorus Myth., Bibliotheca (sub nomine Apollodori) (0548: 001)
“Apollodori bibliotheca. Pediasimi libellus de duodecim Herculis laboribus”, Ed. Wagner, R.
Leipzig: Teubner, 1894; Mythographi Graeci 1.
Chapter 1, section 147, line 6

ἀλλ' οὗτος μὲν πολλῶν κρατήσας βαρβάρων τὴν ὑφ' 
ἑαυτὸν χώραν ἅπασαν Μηδίαν ἐκάλεσε, καὶ στρατευό-
μενος ἐπὶ Ἰνδοὺς ἀπέθανε· Μήδεια δὲ εἰς Κόλχους 
ἦλθεν ἄγνωστος, καὶ καταλαβοῦσα Αἰήτην ὑπὸ τοῦ 
ἀδελφοῦ Πέρσου τῆς βασιλείας ἐστερημένον, κτείνασα 
τοῦτον τῷ πατρὶ τὴν βασιλείαν ἀποκατέστησεν. 



Pseudo-Apollodorus Myth., Bibliotheca (sub nomine Apollodori) 
Chapter 3, section 34, line 1

               καὶ τὸ μὲν πρῶτον Πρωτεὺς αὐτὸν ὑπο-
δέχεται βασιλεὺς Αἰγυπτίων, αὖθις δὲ εἰς Κύβελα τῆς 
Φρυγίας ἀφικνεῖται, κἀκεῖ καθαρθεὶς ὑπὸ Ῥέας καὶ τὰς 
τελετὰς ἐκμαθών, καὶ λαβὼν παρ' ἐκείνης τὴν στολήν, 
[ἐπὶ Ἰνδοὺς] διὰ τῆς Θρᾴκης ἠπείγετο. 



Pseudo-Apollodorus Myth., Bibliotheca (sub nomine Apollodori) 
Chapter 3, section 36, line 1

  διελθὼν δὲ Θρᾴκην [καὶ τὴν Ἰνδικὴν ἅπασαν, 
στήλας ἐκεῖ στήσας] ἧκεν εἰς Θήβας, καὶ τὰς γυναῖκας 
ἠνάγκασε καταλιπούσας τὰς οἰκίας βακχεύειν ἐν τῷ 
Κιθαιρῶνι. 


\end{greek}


\section{\emph{Periplus Maris Erythraei}}
\blockquote[From Wikipedia\footnote{\url{http://en.wikipedia.org/wiki/Periplus_Maris_Erythraei}}]{The Periplus of the Erythraean Sea or Periplus of the Red Sea (Greek: Περίπλους τὴς Ἐρυθράς Θαλάσσης, Latin: Periplus Maris Erythraei) is a Greco-Roman periplus, written in Greek, describing navigation and trading opportunities from Roman Egyptian ports like Berenice along the coast of the Red Sea, and others along Northeast Africa and the Indian subcontinent. The text has been ascribed to different dates between the 1st and 3rd centuries AD, but a mid-1st century date is now the most commonly accepted. Although the author is unknown, it is clearly a firsthand description by someone familiar with the area and is nearly unique in providing accurate insights into what the ancient world knew about the lands around the Indian Ocean.

Although Erythraean Sea (Greek: Ἐρυθρά Θάλασσα) literally means "Red Sea", to the Greeks it included the Indian Ocean and the Persian Gulf.

Overview
17th century map depicting the locations of the Periplus of the Erythraean Sea.

The work consists of 66 chapters, most of them about the length of a long paragraph in English. For instance, the short Chapter 9 reads in its entirety:

    "From Malao (Berbera) it is two courses to the mart of Moundou, where ships anchor more safely by an island lying very close to the land. The imports to this are as aforesaid [Chapter 8 mentions iron, gold, silver, drinking cups, etc.], and from it likewise are exported the same goods [Chapter 8 mentions myrrh, douaka, makeir, and slaves], and fragrant gum called mokrotou (cf. Sanskrit makaranda). The inhabitants who trade here are more peaceful."

In many cases, the description of places is sufficiently accurate to identify their present locations; for others, there is considerable debate. For instance, a "Rhapta" is mentioned as the farthest market down the African coast of "Azania", but there are at least five locations matching the description, ranging from Tanga to south of the Rufiji River delta. The description of the Indian coast mentions the Ganges River clearly, yet after that is somewhat garbled, describing China as a "great inland city Thina" that is a source of raw silk.

The Periplus says that a direct sailing route from the Red Sea to India across the open ocean was discovered by Hippalus (1st century BC).

Many trade goods are mentioned in the Periplus, but some of the words naming trade goods are seen nowhere else in ancient literature, and so we can only guess as to what they might be. For example, one trade good mentioned is "lakkos chromatinos". The name lakkos appears nowhere else in ancient Greek or Roman literature. The name re-surfaces in late medieval Latin as lacca, borrowed from medieval Arabic lakk in turn borrowed from Sanskritic lakh, meaning lac i.e. a red-colored resin native to India used as a lacquer and used also as a red colorant.[1] Some other named trade goods remain obscure.

The Periplus text derives from a Byzantine 10th-century manuscript in minuscule hand, contained in the collections of the University Library Heidelberg (CPG 398: 40v-54v), and a copy of it dating from the 14th or 15th century in the British Museum (B.M. Add 19391 9r-12r). In the 10th-century manuscript, the text is attributed to Arrian, probably for no deeper reason than that the manuscript was adjacent to the Periplus Ponti Euxini written by him. The Periplus was edited by Sigmund Gelen (Zikmund Hruby z Jeleni of Prague)[2] and first published in a modern edition by Hieronymus Froben in 1533. This edition was corrupt and full of errors but served for later editions for three centuries until the rediscovery of the 10th century Heidelberg manuscript which was taken to Rome during the Thirty Years War (1618–1648), then to Paris under Napoleon, and finally returned to Heidelberg in 1816.[3]
Date/Authorship

One historical analysis, published by Schoff in 1912, narrowed the date of the text to 60 A. D.[4] Though narrowing the date down, from 1912, to a single year roughly 2000 years earlier might be considered remarkable by modern standards, a date of 60 A. D. nevertheless remains in perfect agreement with present day estimates of in the middle of the 1st century. Schoff additionally provides an historical analysis as to the text's original authorship[5] and arrives at the conclusion that the author must have been a "Greek in Egypt, a Roman subject,"[6] and by Schoff's calculations this would be during the time of Tiberius Claudius Balbilus (who coincidentally also was an Egyptian Greek).

John Hill maintains that the "Periplus can now be confidently dated to between 40 and 70 CE and, probably, between CE 40 and 50."[7]

Schoff continues by noting that the author could not have been "a highly educated man" as "is evident from his frequent confusion of Greek and Latin words and his clumsy and sometimes ungrammatical constructions."[8] Because of "the absence of any account of the journey up the Nile and across the desert from Coptos,"[8] Schoff prefers to pinpoint the author's residence to "Berenice rather than Alexandria."[8] Though Schoff is unclear about which "Berenice" he is referring to and several possibilities exist for "Berenice", it is actually Berenice Troglodytica which is documented, discussed at length and vividly described within the periplus text itself.[9]

One peculiarity noted by Schoff while translating from the original Greek version is that "the text is so vague and uncertain that [the author] seems rather to be quoting from someone else, unless indeed much of this part of the work has been lost in copying."[8]
Opone (Somalia)
Main article: Hafun

Ras Hafun in northern Somalia is believed to be the location of the ancient trade center of Opone. Ancient Egyptian, Roman and Persian Gulf pottery has been recovered from the site by an archaeological team from the University of Michigan. Opone is in the thirteenth entry of the Periplus of the Erythraean Sea, which in part states:

    "And then, after sailing four hundred stadia along a promontory, toward which place the current also draws you, there is another market-town called Opone, into which the same things are imported as those already mentioned, and in it the greatest quantity of cinnamon is produced, (the arebo and moto), and slaves of the better sort, which are brought to Egypt in increasing numbers; and a great quantity of tortoiseshell, better than that found elsewhere."
    —Periplus of the Erythraean Sea, Chap.13[10]

In ancient times, Opone operated as a port of call for merchants from Phoenicia, Egypt, Greece, Persia, Yemen, Nabataea, Azania, the Roman Empire and elsewhere, as it possessed a strategic location along the coastal route from Azania to the Red Sea. Merchants from as far afield as Indonesia and Malaysia passed through Opone, trading spices, silks and other goods, before departing south for Azania or north to Yemen or Egypt on the trade routes that spanned the length of the Indian Ocean's rim. As early as 50 AD, Opone was well-known as a center for the cinnamon trade, along with the trading of cloves and other spices, ivory, exotic animal skins and incense.
Malao (Somalia)
Main article: Berbera

The ancient port city of Malao, situated in present-day Berbera in northwestern Somalia, is also mentioned in the Periplus:

    "After Avalites there is another market-town, better than this, called Malao, distant a sail of about eight hundred stadia. The anchorage is an open roadstead, sheltered by a spit running out from the east. Here the natives are more peaceable. There are imported into this place the things already mentioned, and many tunics, cloaks from Arsinoe, dressed and dyed; drinking-cups, sheets of soft copper in small quantity, iron, and gold and silver coin, not much. There are exported from these places myrrh, a little frankincense, (that known as far-side), the harder cinnamon, duaca, Indian copal and macir, which are imported into Arabia; and slaves, but rarely."
    —Periplus of the Erythraean Sea, Chap.8[10]

Aksum Empire (Eritrea and Ethiopia)
Main article: Kingdom of Aksum
Coins of king Endybis, 227-235 AD. British Museum. The left one reads in Greek "AΧWMITW BACIΛEYC", "King of Axum". The right one reads in Greek: ΕΝΔΥΒΙC ΒΑCΙΛΕΥC, "King Endybis".

Aksum is mentioned in the Periplus as an important market place for ivory, which was exported throughout the ancient world:

    "From that place to the city of the people called Auxumites there is a five days' journey more; to that place all the ivory is brought from the country beyond the Nile through the district called Cyeneum, and thence to Adulis."
    —Periplus of the Erythraean Sea, Chap.4

According to the Periplus, the ruler of Aksum in the 1st century AD was Zoscales, who, besides ruling in Aksum also held under his sway two harbours on the Red Sea: Adulis (near Massawa) and Avalites (Assab). He is also said to have been familiar with Greek literature:

    "These places, from the Calf-Eaters to the other Berber country, are governed by Zoscales; who is miserly in his ways and always striving for more, but otherwise upright, and acquainted with Greek literature."
    —Periplus of the Erythraean Sea, Chap.5[10]

Himyarite kingdom and Saba (Arabia)
Main article: Himyarite Kingdom
Main article: Sheba
Coin of the Himyarite Kingdom, southern coast of the Arabian Peninsula, in which stopped ships between Egypt and India passed. This is an imitation of a coin of Augustus. 1st Century AD.

Ships from Himyar regularly traveled the East African coast. The Periplus of the Erythraean Sea describes the trading empire of Himyar and Saba, regrouped under a single ruler Charibael (Karab Il Watar Yuhan'em II), who is said to have been on friendly terms with Rome:

    "23. And after nine days more there is Saphar, the metropolis, in which lives Charibael, lawful king of two tribes, the Homerites and those living next to them, called the Sabaites; through continual embassies and gifts, he is a friend of the Emperors."
    —Periplus of the Erythraean Sea, Paragraph 23.[10]

Frankincense kingdom (Hadramaut)

The Frankincense kingdom is described further east along the southern coast of the Arabian Peninsula, with the harbour of Cana (South Arabic Qana, modern Bir Ali in Hadramaut). The ruler of this kingdom is named Eleazus, or Eleazar, thought to correspond to King Iliazz Yalit I:

    "27. After Eudaemon Arabia there is a continuous length of coast, and a bay extending two thousand stadia or more, along which there are Nomads and Fish-Eaters living in villages; just beyond the cape projecting from this bay there is another market-town by the shore, Cana, of the Kingdom of Eleazus, the Frankincense Country; and facing it there are two desert islands, one called Island of Birds, the other Dome Island, one hundred and twenty stadia from Cana. Inland from this place lies the metropolis Sabbatha, in which the King lives. All the frankincense produced in the country is brought by camels to that place to be stored, and to Cana on rafts held up by inflated skins after the manner of the country, and in boats. And this place has a trade also with the far-side ports, with Barygaza and Scythia and Ommana and the neighboring coast of Persia."
    —Periplus of the Erythraean Sea, Chap 27

Rhapta (Tanzania - or Mozambique?)

Recent research by the Tanzanian archaeologist Felix Chami has uncovered extensive remains of Roman trade items near the mouth of the Rufiji River and the nearby Mafia island, and makes a strong case that the ancient port of Rhapta was situated on the banks of the Rufiji River just south of Dar es Salaam.

The Periplus informs us that:

    "Two runs beyond this island [Menuthias = Zanzibar?] comes the very last port of trade on the coast of Azania, called Rhapta ["sewn"], a name derived from the aforementioned sewn boats, where there are great quantities of ivory and tortoise shell."[11]

Chami summarizes the evidence for Rhapta's location as follows:

    "The actual location of the Azanian capital, Rhapta, remains unknown. However, archaeological indicators reported above suggest that it was located on the coast of Tanzania, in the region of the Rufiji River and Mafia Island. It is in this region where the concentration of Panchaea/Azanian period settlements has been discovered. If the island of Menuthias mentioned in the Periplus was Zanzibar, a short voyage south would land one in the Rufiji region. The 2nd century geographer, Ptolemy, locates Rhapta at latitude 8º south, which is the exact latitude of the Rufiji Delta and Mafia Island. The metropolis was on the mainland about one degree west of the coast near a large river and a bay with the same name. While the river should be regarded as the modern Rufiji River, the bay should definitely be identified with the calm waters between the island of Mafia and the Rufiji area. The peninsula east of Rhapta would have been the northern tip of Mafia Island. The southern part of the bay is protected from the deep sea by numerous deltaic small islets separated from Mafia Island by shallow and narrow channels. To the north the bay is open to the sea and any sailor entering the waters from that direction would feel as if he were entering a bay. Even today the residents identify these waters as a bay, referring to it as a 'female sea', as opposed to the more violent open sea on the other side of the island of Mafia."[12]

In recent years, Felix Chami has found archaeological evidence for extensive Roman trade on Mafia Island and, not far away, on the mainland, near the mouth of the Rufiji River, which he dated to the first few centuries CE. Furthermore, J. Innes Miller points out that Roman coins have been found on Pemba island, just north of Rhapta.[13]

Nevertheless, Carl Peters has argued that Rhapta was near modern-day Quelimane in Mozambique[14], citing the fact that (according to the Periplus) the coastline there ran down towards the southwest. Peters also suggests that the description of the "Pyralaoi" (i.e., the "Fire people") - "situated at the entry to the [Mozambique] Channel" - indicates that they were the inhabitants of the volcanic Comoro Islands. He also maintains that Menuthias (with its abundance of rivers and crocodiles) cannot have been Zanzibar; i.e., Madagascar seems more likely.

Interestingly, the Periplus informs us that Rhapta, was under the firm control of a governor appointed by Arabian king of Musa,[disambiguation needed] taxes were collected, and it was serviced by "merchant craft that they staff mostly with Arab skippers and agents who, through continual intercourse and intermarriage, are familiar with the area and its language."[11]

The Periplus explicitly states that Azania (which included Rhapta) was subject to Charibaêl, the king of both the Sabaeans and Homerites in the southwest corner of Arabia. The kingdom is known to have been a Roman ally at this period. Charibaêl is stated in the Periplus to be “a friend of the (Roman) emperors, thanks to continuous embassies and gifts” and, therefore, Azania could fairly be described as a vassal or dependency of Rome, just as Zesan is described in the 3rd century Chinese history, the Weilüe.[15][16]
Barygaza (India)
Main article: Bharuch

Trade with the Indian harbour of Barygaza is described extensively in the Periplus. Nahapana, ruler of the Indo-Scythian Western Satraps is mentioned under the name Nambanus,[17] as ruler of the area around Barigaza:

    41. "Beyond the gulf of Baraca is that of Barygaza and the coast of the country of Ariaca, which is the beginning of the Kingdom of Nambanus and of all India. That part of it lying inland and adjoining Scythia is called Abiria, but the coast is called Syrastrene. It is a fertile country, yielding wheat and rice and sesame oil and clarified butter, cotton and the Indian cloths made therefrom, of the coarser sorts. Very many cattle are pastured there, and the men are of great stature and black in color. The metropolis of this country is Minnagara, from which much cotton cloth is brought down to Barygaza."
    —Periplus of the Erythraean Sea, Chap. 41[10]

Under the Western Satraps, Barigaza was one of the main centers of Roman trade with India. The Periplus describes the many goods exchanged:

    49. There are imported into this market-town (Barigaza), wine, Italian preferred, also Laodicea[disambiguation needed]n and Arabian; copper, tin, and lead; coral and topaz; thin clothing and inferior sorts of all kinds; bright-colored girdles a cubit wide; storax, sweet clover, flint glass, realgar, antimony, gold and silver coin, on which there is a profit when exchanged for the money of the country; and ointment, but not very costly and not much. And for the King there are brought into those places very costly vessels of silver, singing boys, beautiful maidens for the harem, fine wines, thin clothing of the finest weaves, and the choicest ointments. There are exported from these places spikenard, costus, bdellium, ivory, agate and carnelian, lycium, cotton cloth of all kinds, silk cloth, mallow cloth, yarn, long pepper and such other things as are brought here from the various market-towns. Those bound for this market-town from Egypt make the voyage favorably about the month of July, that is Epiphi."
    —Periplus of the Erythraean Sea, Chapter 49.[18]

Goods were also brought down in quantity from Ujjain, the capital of the Western Satraps:

    48. Inland from this place and to the east, is the city called Ozene, formerly a royal capital; from this place are brought down all things needed for the welfare of the country about Barygaza, and many things for our trade : agate and carnelian, Indian muslins and mallow cloth, and much ordinary cloth.
    —Periplus of the Erythraean Sea, Chapter 48.[18]

Early Chera, Chola, and early Pandyan kingdoms (India)
See also: Chera Dynasty, Early Pandyan Kingdom, Muziris, and Economy of ancient Tamil country

The lost port city of Muziris (Near present day Cochin) in the Chera kingdom, as well as the Early Pandyan Kingdom are mentioned in the Periplus as major centers of trade, pepper and other spices, metal work and semiprecious stones, between Damirica and the Roman Empire.

According to the Periplus, numerous Greek seamen managed an intense trade with Muziris:

    Then come Naura (Kannur) and Tyndis, the first markets of Damirica or Limyrike, and then Muziris and Nelcynda, which are now of leading importance. Tyndis is of the Kingdom of Cerobothra; it is a village in plain sight by the sea. Muziris, of the same kingdom, abounds in ships sent there with cargoes from Arabia, and by the Greeks; it is located on a river (River Periyar), distant from Tyndis by river and sea five hundred stadia, and up the river from the shore twenty stadia. Nelcynda is distant from Muziris by river and sea about five hundred stadia, and is of another Kingdom, the Pandian. This place also is situated on a river, about one hundred and twenty stadia from the sea...."
    —The Periplus of the Erythraean Sea, 53-54

Damirica or Limyrike is Tamilakkam (Tamil தமிழகம்) – the "Tamil country". Further, this area served as a hub for trade with the interior, in the Gangetic plain:

    Besides this there are ex-ported great quantities of fine pearls, ivory, silk cloth, spikenard from the Ganges, malabathrum from the places in the interior, transparent stones of all kinds, diamonds and sapphires, and tortoise-shell; that from Chryse Island, and that taken among the islands along the coast of Damirica (Limyrike). They make the voyage to this place in a favorable season who set out from Egypt about the month of July, that is Epiphi.
    —The Periplus of the Erythraean Sea, 56

Remains of the Indo-Greek kingdom
The Periplus explains that coins of the Indo-Greek king Menander I were current in Barigaza.

The Periplus describes numerous Greek buildings and fortifications in Barigaza, although mistakenly attributing them to Alexander the Great, who never went this far south. If true, this account would relate to the remains of the southern expansion of the Indo-Greeks into Gujarat:

    "The metropolis of this country is Minnagara, from which much cotton cloth is brought down to Barygaza. In these places there remain even to the present time signs of the expedition of Alexander, such as ancient shrines, walls of forts and great wells."
    —Periplus, Chap. 41

The Periplus further testifies to the circulation of Indo-Greek coinage in the region:

    "To the present day ancient drachmae are current in Barygaza, coming from this country, bearing inscriptions in Greek letters, and the devices of those who reigned after Alexander, Apollodorus [sic] and Menander."
    —Periplus Chap. 47[10]

The Greek city of Alexandria Bucephalous on the Jhelum River is mentioned in the Periplus, as well as in the Roman Peutinger Table:

    "The country inland of Barigaza is inhabited by numerous tribes, such as the Arattii, the Arachosii, the Gandaraei and the people of Poclais, in which is Bucephalus Alexandria"
    —Periplus of the Erythraean Sea, 47[10]

}
\begin{greek}

Periplus Maris Erythraei, Anonymi (Arriani, ut fertur) periplus maris Erythraei (0071: 001)
“Geographi Graeci minores, vol. 1”, Ed. Müller, K.
Paris: Didot, 1855, Repr. 1965.
Section 6, line 18

                                           Ὁμοίως δὲ καὶ 
ἀπὸ τῶν ἔσω τόπων τῆς Ἀραβικῆς σίδηρος Ἰνδικὸς 
καὶ στόμωμα καὶ ὀθόνιον Ἰνδικὸν τὸ πλατύτερον, ἡ λε-  
γομένη μοναχὴ, καὶ σαγματογῆναι καὶ περιζώματα 
καὶ καυνάκαι καὶ μολόχινα καὶ σινδόνες ὀλίγαι καὶ 
λάκκος χρωμάτινος. 



Periplus Maris Erythraei, Anonymi (Arriani, ut fertur) periplus maris Erythraei 
Section 17, line 9

                              Ἐκφέρεται δὲ ἀπὸ   
τῶν τόπων ἐλέφας πλεῖστος (ἥσσων δὲ τοῦ Ἀδουλιτι-
κοῦ) καὶ ῥινόκερως καὶ χελώνη διάφορος μετὰ τὴν Ἰν-
δικὴν καὶ ναύπλιος ὀλίγος. 



Periplus Maris Erythraei, Anonymi (Arriani, ut fertur) periplus maris Erythraei 
Section 26, line 9

                     Εὐδαίμων δ' ἐπεκλήθη, πρότερον 
οὖσα πόλις, ὅτε, μήπω ἀπὸ τῆς Ἰνδικῆς εἰς τὴν Αἴ-
γυπτον ἐρχομένων μηδὲ ἀπὸ [τῆς] Αἰγύπτου τολμών-
των εἰς τοὺς ἔσω τόπους διαίρειν, ἀλλ' ἄχρι ταύτης 
παραγινομένων, τοὺς παρ' ἀμφοτέρων φόρτους ἀπεδέ-
χετο, ὥσπερ Ἀλεξάνδρεια καὶ τῶν ἔξωθεν καὶ τῶν ἀπὸ 
τῆς Αἰγύπτου φερομένων ἀποδέχεται. 



Periplus Maris Erythraei, Anonymi (Arriani, ut fertur) periplus maris Erythraei 
Section 30, line 15

        Οἱ δ' ἐνοικοῦντες αὐτὴν ὀλίγοι κατὰ μίαν πλευ-
ρὰν τῆς νήσου τὴν πρὸς ἀπαρκτίαν οἰκοῦσι, καθ' ὃ μέρος 
ἀποβλέπει τὴν ἤπειρον· εἰσὶ δὲ ἐπίξενοι καὶ ἐπίμικτοι 
Ἀράβων τε καὶ Ἰνδῶν καὶ ἔτι Ἑλλήνων τῶν πρὸς 
ἐργασίαν ἐκπλεόντων. 



Periplus Maris Erythraei, Anonymi (Arriani, ut fertur) periplus maris Erythraei 
Section 30, line 24

                                 Γίνεται δὲ ἐν αὐτῇ 
καὶ κιννάβαρι τὸ λεγόμενον Ἰνδικὸν, ἀπὸ τῶν δέν-
δρων ὡς δάκρυ συναγόμενον. 



Periplus Maris Erythraei, Anonymi (Arriani, ut fertur) periplus maris Erythraei 
Section 31, line 6

                                   Συνεχρήσαντο δὲ 
αὐτῇ καὶ ἀπὸ Μούζα τινὲς καὶ τῶν ἐκπλεόντων ἀπὸ 
Λιμυρικῆς καὶ Βαρυγάζων ὅσοι κατὰ τύχην εἰς αὐτὴν 
ἐπιβάλλοντες ὄρυζάν τε καὶ σῖτον καὶ ὀθόνην Ἰνδικὴν   
ἀντικαταλλασσόμενοι καὶ σώματα θηλυκὰ διὰ σπά-
νιν ἐκεῖ προχωροῦντα, χελώνην ἀντεφορτίζοντο πλεί-
στην· νῦν δὲ ὑπὸ τῶν βασιλέων ἡ νῆσος ἐκμεμίσθωται 
καὶ παραφυλάσσεται. 



Periplus Maris Erythraei, Anonymi (Arriani, ut fertur) periplus maris Erythraei 
Section 36, line 12

                                                 Εἰς-
φέρεται δὲ ἀπὸ ἑκατέρων τῶν ἐμπορίων εἴς τε Βαρύ-
γαζα καὶ εἰς Ἀραβίαν πινικὸν, πολὺ μὲν, χεῖρον δὲ   
τοῦ Ἰνδικοῦ, καὶ πορφύρα καὶ ἱματισμὸς ἐντόπιος καὶ 
οἶνος καὶ φοῖνιξ πολὺς καὶ χρυσὸς καὶ σώματα. 



Periplus Maris Erythraei, Anonymi (Arriani, ut fertur) periplus maris Erythraei 
Section 39, line 10

                        Ἀντιφορτίζεται δὲ κόστος, βδέλλα,   
λύκιον, νάρδος καὶ καλλεανὸς λίθος καὶ σάπφειρος καὶ 
Σηρικὰ δέρματα καὶ ὀθόνιον καὶ νῆμα Σηρικὸν καὶ 
Ἰνδικὸν μέλαν. 



Periplus Maris Erythraei, Anonymi (Arriani, ut fertur) periplus maris Erythraei 
Section 39, line 11

                   Ἀνάγονται δὲ καὶ αὐτοὶ οἱ πλέοντες 
μετὰ τῶν Ἰνδικῶν * περὶ τὸν Ἰούλιον μῆνα, ὅς ἐστιν 
Ἐπιφί· δυσεπίβολος μὲν, ἐπιφορώτερος δὲ ἐκείνων καὶ 
συντομώτερος ὁ πλοῦς. 



Periplus Maris Erythraei, Anonymi (Arriani, ut fertur) periplus maris Erythraei 
Section 41, line 3

Μετὰ δὲ τὸν Βαράκην εὐθύς ἐστιν ὁ Βαρυγάζων 
κόλπος καὶ ἡ ἤπειρος τῆς Ἀριακῆς χώρας, τῆς Μαμ-
βάρου βασιλείας ἀρχὴ καὶ τῆς ὅλης Ἰνδικῆς οὖσα. 



Periplus Maris Erythraei, Anonymi (Arriani, ut fertur) periplus maris Erythraei 
Section 41, line 7

                                               Πολυ-
φόρος δὲ ἡ χώρα σίτου καὶ ὀρύζης καὶ ἐλαίου σησαμί-
νου καὶ βουτύρου καὶ καρπάσου καὶ τῶν ἐξ αὐτῆς Ἰνδι-
κῶν ὀθονίων τῶν χυδαίων· βουκόλια δὲ ἐν αὐτῇ πλεῖστα 
καὶ ἄνδρες ὑπερμεγέθεις τῷ σώματι καὶ μέλανες τῇ 
χροιᾷ. 



Periplus Maris Erythraei, Anonymi (Arriani, ut fertur) periplus maris Erythraei 
Section 45, line 1

Πᾶσα μὲν ἡ Ἰνδικὴ χώρα ποταμοὺς ἔχει πλεί-  
στους, ἀμπώτεις τε καὶ πλήμας μεγίστας, συναυξομέ-
νας ὑπὸ τὴν ἀνατολὴν καὶ τὴν πανσέληνον ἄχρι τριῶν 
ἡμερῶν, καὶ τοῖς μεταξὺ καταστήμασι τῆς σελήνης 
ἐλασσουμένας, πολὺ δὲ μᾶλλον ἡ κατὰ Βαρυγάζων, 
ὥστε αἰφνίδιον τόν τε βυθὸν ὁρᾶσθαι, καὶ ** τινα μέρη 
τῆς ἠπείρου, ποτὲ δὲ ξηρὰ τὰ πρὸ μικροῦ πλωϊζόμενα, 
τούς τε ποταμοὺς ὑπὸ τὴν εἰσβολὴν τῆς πλήμης, τοῦ 
πελάγους ὅλου συνωθουμένου, σφοδρότερον ἄνω φέρε-
σθαι τοῦ κατὰ φύσιν ῥεύματος ἐπὶ πλείστους σταδίους. 



Periplus Maris Erythraei, Anonymi (Arriani, ut fertur) periplus maris Erythraei 
Section 47, line 8

                                                          Καὶ 
Ἀλέξανδρος ὁρμηθεὶς ἀπὸ τῶν μερῶν τούτων ἄχρι 
τοῦ Γάγγους διῆλθε, καταλιπὼν τήν τε Λιμυρικὴν 
καὶ τὰ νότια τῆς Ἰνδικῆς· ἀφ' οὗ μέχρι νῦν ἐν Βαρυ-
γάζοις παλαιαὶ προχωροῦσι δραχμαὶ, γράμμασιν Ἑλ-
ληνικοῖς ἐγκεχαραγμέναι ἐπίσημα τῶν μετ' Ἀλέξαν-
δρον βεβασιλευκότων Ἀπολλοδότου καὶ Μενάνδρου. 



Periplus Maris Erythraei, Anonymi (Arriani, ut fertur) periplus maris Erythraei 
Section 48, line 5

Ἔνι δὲ αὐτῇ καὶ ἐξ ἀνατολῆς πόλις λεγομένη 
Ὀζήνη, ἐν ᾗ καὶ τὰ βασίλεια πρότερον ἦν· ἀφ' ἧς 
πάντα τὰ πρὸς εὐθηνίαν τῆς χώρας εἰς Βαρύγαζα κα-
ταφέρεται καὶ τὰ πρὸς ἐμπορίαν τὴν ἡμετέραν, ὀνυχίνη 
λιθία καὶ μουρρίνη καὶ σινδόνες Ἰνδικαὶ καὶ μολόχιναι 
καὶ ἱκανὸν χυδαῖον ὀθόνιον. 



Periplus Maris Erythraei, Anonymi (Arriani, ut fertur) periplus maris Erythraei 
Section 57, line 7

Τοῦτον δὲ ὅλον τὸν εἰρημένον περίπλουν ἀπὸ 
Κανῆς καὶ τῆς Εὐδαίμονος Ἀραβίας οἱ μὲν * μικροτέροις 
πλοίοις περικολπίζοντες ἔπλεον, πρῶτος δὲ Ἵππαλος   
κυβερνήτης, κατανοήσας τὴν θέσιν τῶν ἐμπορίων καὶ 
τὸ σχῆμα τῆς θαλάσσης, τὸν διὰ πελάγους ἐξεῦρε πλοῦν, 
ἀφ' οὗ καὶ τοπικῶς ἐκ τοῦ ὠκεανοῦ φυσώντων, κατὰ 
τὸν καιρὸν τῶν παρ' ἡμῖν, ἐτησίων ἐν τῷ Ἰνδικῷ πε-
λάγει ὁ λιβόνοτος φαίνεται [ἵππαλος] προσονομάζε-
σθαι ⟦ἀπὸ τῆς προσηγορίας τοῦ πρώτως ἐξευρηκότος 
τὸν διάπλουν⟧. 



Periplus Maris Erythraei, Anonymi (Arriani, ut fertur) periplus maris Erythraei 
Section 63, line 6

            Ποταμὸς δέ ἐστι περὶ αὐτὴν ὁ Γάγγης λε-
γόμενος, καὶ αὐτὸς μέγιστος τῶν κατὰ τὴν Ἰνδικὴν, 
ἀπόβασίν τε καὶ ἀνάβασιν τὴν αὐτὴν ἔχων τῷ Νείλῳ, 
καθ' ὃν καὶ ἐμπόριόν ἐστιν ὁμώνυμον τῷ ποταμῷ, 
ὁ Γάγγης, δι' οὗ φέρεται τό τε μαλάβαθρον καὶ ἡ Γαγ-
γητικὴ νάρδος καὶ πινικὸν καὶ σινδόνες αἱ διαφορώ-
ταται, αἱ Γαγγητικαὶ λεγόμεναι. 



Periplus Maris Erythraei, Anonymi (Arriani, ut fertur) periplus maris Erythraei 
Section 65, line 20

         Ἔνθεν τὰ τρία μέρη τοῦ μαλαβάθρου γί-
νεται καὶ τότε φέρεται εἰς τὴν Ἰνδικὴν ὑπὸ τῶν κατ-
εργαζομένων αὐτά. 

\end{greek}


\section{Strabo}
\blockquote[From Wikipedia\footnote{\url{http://en.wikipedia.org/wiki/Strabo}}]{Strabo[1] (play /ˈstreɪboʊ/; Greek: Στράβων Strabōn; 64/63 BCE – ca. 24 CE), was a Greek geographer, philosopher and historian.}
\begin{greek}

Strabo Geogr., Geographica (0099: 001)
“Strabonis geographica, 3 vols.”, Ed. Meineke, A.
Leipzig: Teubner, 1877, Repr. 1969.
Book Cap, chapter 1, section 15, line 1

{ΙΕ} 
 Τὸ πεντεκαιδέκατον περιέχει Ἰνδίαν καὶ Περσίδα. 



Strabo Geogr., Geographica 
Book Cap, chapter 1, section 16, line 4

{Ιϛ} 
 Τὸ ἑκκαιδέκατον περιέχει τὴν Ἀσσυρίων χώραν, ἐν ᾗ Βαβυλὼν 
καὶ Νίσιβις, πόλεις μέγιστα, καὶ τὴν Ἀδιαβηνὴν καὶ Μεσοποταμίαν, 
Συρίαν πᾶσαν, Φοινίκην, Παλαιστίνην, Ἀραβίαν πᾶσαν καὶ ὅσα τῆς 
Ἰνδικῆς τῇ Ἀραβίᾳ συνάπτει, καὶ τὴν Σαρακηνῶν, ἣν Σκηνῆτιν (leg. 
Σκηνῖτιν) καλεῖ, καὶ πᾶσαν τὴν παρακειμένην τῇ τε νεκρᾷ θαλάσσῃ 
καὶ τῇ ἐρυθρᾶ. 



Strabo Geogr., Geographica 
Book 1, chapter 1, section 8, line 7

                                     τὸ μὲν γὰρ ἑωθινὸν 
πλευρόν, τὸ κατὰ τοὺς Ἰνδούς, καὶ τὸ ἑσπέριον, τὸ 
κατὰ τοὺς Ἴβηρας καὶ τοὺς Μαυρουσίους, περιπλεῖ-
ται πᾶν ἐπὶ πολὺ τοῦ τε νοτίου μέρους καὶ τοῦ βο-  
ρείου· τὸ δὲ λειπόμενον ἄπλουν ἡμῖν μέχρι νῦν τῷ μὴ 
συμμῖξαι μηδένας ἀλλήλοις τῶν ἀντιπεριπλεόντων οὐ 
πολύ, εἴ τις συντίθησιν ἐκ τῶν παραλλήλων διαστη-
μάτων τῶν ἐφικτῶν ἡμῖν. 



Strabo Geogr., Geographica 
Book 1, chapter 1, section 13, line 15

           ὁμοίως δὲ καὶ τὸ παρ' Ἰνδοῖς οἰκεῖν ἢ παρ' 
Ἴβηρσιν· ὧν τοὺς μὲν ἑῴους μάλιστα τοὺς δὲ ἑσπε-
ρίους, τρόπον δέ τινα καὶ ἀντίποδας ἀλλήλοις ἴσμεν. 



Strabo Geogr., Geographica 
Book 1, chapter 1, section 16, line 52

         ὥστ' οὐκ ἂν εἴη θαυμαστὸν οὐδ' εἰ ἄλλος μὲν 
Ἰνδοῖς προσήκοι χωρογράφος, ἄλλος δὲ Αἰθίοψιν, ἄλ-
λος δὲ Ἕλλησι καὶ Ῥωμαίοις. 



Strabo Geogr., Geographica 
Book 1, chapter 1, section 16, line 54

                                    τί γὰρ ἂν προσήκοι τῷ παρ'   
Ἰνδοῖς γεωγράφῳ καὶ τὰ κατὰ Βοιωτοὺς οὕτω φράζειν 
ὡς Ὅμηρος “οἵθ' Ὑρίην ἐνέμοντο καὶ Αὐλίδα πετρή-
εσσαν Σχοῖνόν τε Σκῶλόν τε; 



Strabo Geogr., Geographica 
Book 1, chapter 1, section 16, line 57

                                    ἡμῖν δὲ προσήκει, 
τὰ δὲ παρ' Ἰνδοῖς οὕτω καὶ τὰ καθ' ἕκαστα οὐκέτι· 
οὐδὲ γὰρ ἡ χρεία ἐπάγεται· μέτρον δ' αὕτη μάλιστα 
τῆς τοιαύτης ἐμπειρίας. 



Strabo Geogr., Geographica 
Book 1, chapter 2, section 28, line 5

Μηνύει δὲ καὶ Ἔφορος τὴν παλαιὰν περὶ τῆς Αἰ-
θιοπίας δόξαν, ὅς φησιν ἐν τῷ περὶ τῆς Εὐρώπης 
λόγῳ, τῶν περὶ τὸν οὐρανὸν καὶ τὴν γῆν τόπων εἰς 
τέτταρα μέρη διῃρημένων, τὸ πρὸς τὸν ἀπηλιώτην Ἰν-  
δοὺς ἔχειν, πρὸς νότον δὲ Αἰθίοπας, πρὸς δύσιν δὲ 
Κελτούς, πρὸς δὲ βορρᾶν ἄνεμον Σκύθας. 



Strabo Geogr., Geographica 
Book 1, chapter 2, section 31, line 22

     οἱ μὲν δὴ πλεῦσαι φήσαντες εἰς τὴν Αἰθιοπίαν 
οἱ μὲν περίπλουν τῶν διὰ Γαδείρων μέχρι τῆς Ἰνδικῆς 
εἰσάγουσιν, ἅμα καὶ τὸν χρόνον τῇ πλάνῃ συνοικει-
οῦντες, ὅν φησιν ὅτι ὀγδοάτῳ ἔτει ἦλθον, οἱ δὲ διὰ 
τοῦ ἰσθμοῦ τοῦ κατὰ τὸν Ἀράβιον κόλπον, οἱ δὲ διὰ 
τῶν διωρύγων τινός. 



Strabo Geogr., Geographica 
Book 1, chapter 2, section 32, line 11

                                                     νὴ Δία, 
ἀλλ' ἡ Ἀραβία προσῆν καὶ τὰ μέχρι τῆς Ἰνδικῆς· τού-
των δ' ἡ μὲν εὐδαίμων κέκληται μόνη τῶν ἁπασῶν, 
τὴν δέ, εἰ καὶ μὴ ὀνομαστὶ καλοῦσιν, οὕτως ὑπολαμ-
βάνουσί γε καὶ ἱστοροῦσιν ὡς εὐδαιμονεστάτην. 



Strabo Geogr., Geographica 
Book 1, chapter 2, section 32, line 15

                                                      τὴν 
μὲν οὖν Ἰνδικὴν οὐκ οἶδεν Ὅμηρος (εἰδὼς δὲ ἐμέμνη-
το ἄν), τὴν δ' Ἀραβίαν, ἣν εὐδαίμονα προσαγορεύ-  
ουσιν οἱ νῦν, τότε δ' οὐκ ἦν πλουσία, ἀλλὰ καὶ αὐτὴ 
ἄπορος καὶ ἡ πολλὴ αὐτῆς σκηνιτῶν ἀνδρῶν· ὀλίγη 
δ' ἡ ἀρωματοφόρος, δι' ἣν καὶ τοῦτο τοὔνομα εὕρετο 
ἡ χώρα διὰ τὸ καὶ τὸν φόρτον εἶναι τὸν τοιοῦτον ἐν 
τοῖς παρ' ἡμῖν σπάνιον καὶ τίμιον. 



Strabo Geogr., Geographica 
Book 1, chapter 2, section 35, line 39

Θεόπομπος δὲ ἐξομολογεῖται φήσας ὅτι καὶ μύθους 
ἐν ταῖς ἱστορίαις ἐρεῖ, κρεῖττον ἢ ὡς Ἡρόδοτος καὶ 
Κτησίας καὶ Ἑλλάνικος καὶ οἱ τὰ Ἰνδικὰ συγγρά-
ψαντες. 



Strabo Geogr., Geographica 
Book 1, chapter 4, section 5, line 5

                    ὅτι μὲν γὰρ πλέον ἢ διπλάσιον τὸ 
γνώριμον μῆκός ἐστι τοῦ γνωρίμου πλάτους, ὁμολο-
γοῦσι καὶ οἱ ὕστερον καὶ τῶν παλαιῶν οἱ χαριέστατοι·   
λέγω δὲ τὸ ἀπὸ τῶν ἄκρων τῆς Ἰνδικῆς ἐπὶ τὰ ἄκρα 
τῆς Ἰβηρίας, τοῦ ἀπ' Αἰθιόπων ἕως τοῦ κατὰ Ἰέρνην 
κύκλου. 



Strabo Geogr., Geographica 
Book 1, chapter 4, section 5, line 10

                   φησὶ δ' οὖν τὸ μὲν τῆς Ἰνδικῆς μέχρι 
τοῦ Ἰνδοῦ ποταμοῦ τὸ στενώτατον σταδίων μυρίων 
ἑξακισχιλίων (τὸ γὰρ ἐπὶ τὰ ἀκρωτήρια τεῖνον τρισχι-
λίοις εἶναι μεῖζον), τὸ δὲ ἔνθεν ἐπὶ Κασπίους πύλας 
μυρίων τετρακισχιλίων, εἶτ' ἐπὶ τὸν Εὐφράτην μυ-
ρίων, ἐπὶ δὲ τὸν Νεῖλον ἀπὸ τοῦ Εὐφράτου πεντακις-
χιλίων, ἄλλους δὲ χιλίους καὶ τριακοσίους μέχρι Κα-
νωβικοῦ στόματος, εἶτα μέχρι τῆς Καρχηδόνος μυρίους 
τρισχιλίους πεντακοσίους, εἶτα μέχρι στηλῶν ὀκτακις-
χιλίους τοὐλάχιστον· ὑπεραίρειν δὴ τῶν ἑπτὰ

μυριά-



Strabo Geogr., Geographica 
Book 1, chapter 4, section 6, line 8

              .. ὡς οἱ μαθηματικοί φασι, κύκλον 
συνάπτειν, συμβάλλουσαν αὐτὴν ἑαυτῇ, ὥστ' εἰ μὴ 
τὸ μέγεθος τοῦ Ἀτλαντικοῦ πελάγους ἐκώλυε, κἂν 
πλεῖν ἡμᾶς ἐκ τῆς Ἰβηρίας εἰς τὴν Ἰνδικὴν διὰ τοῦ αὐ-
τοῦ παραλλήλου, τὸ λοιπὸν μέρος παρὰ τὸ λεχθὲν διά-
στημα ὑπὲρ τὸ τρίτον μέρος ὂν τοῦ ὅλου κύκλου· 
εἴπερ ὁ δι' Ἀθηνῶν ἐλάττων ἐστὶν εἴκοσι μυριάδων, 
ὅπου πεποιήμεθα τὸν εἰρημένον σταδιασμὸν ἀπὸ τῆς 
Ἰνδικῆς εἰς τὴν Ἰβηρίαν. 



Strabo Geogr., Geographica 
Book 1, chapter 4, section 9, line 8

                                              πολλοὺς γὰρ 
καὶ τῶν Ἑλλήνων εἶναι κακοὺς καὶ τῶν βαρβάρων 
ἀστείους, καθάπερ Ἰνδοὺς καὶ Ἀριανούς, ἔτι δὲ Ῥω-
μαίους καὶ Καρχηδονίους οὕτω θαυμαστῶς πολιτευο-
μένους. 



Strabo Geogr., Geographica 
Book 2, chapter 1, section 1, line 7

Ἐν δὲ τῷ τρίτῳ τῶν γεωγραφικῶν καθιστάμενος 
τὸν τῆς οἰκουμένης πίνακα γραμμῇ τινι διαιρεῖ δίχα 
ἀπὸ δύσεως ἐπ' ἀνατολὴν παραλλήλῳ τῇ ἰσημερινῇ 
γραμμῇ, πέρατα δ' αὐτῆς τίθησι πρὸς δύσει μὲν τὰς 
Ἡρακλείους στήλας, ἐπ' ἀνατολῇ δὲ τὰ ἄκρα καὶ ἔσχατα 
ὄρη τῶν ἀφοριζόντων ὀρῶν τὴν πρὸς ἄρκτον τῆς Ἰν-
δικῆς πλευράν. 



Strabo Geogr., Geographica 
Book 2, chapter 1, section 1, line 16

                                             μέχρι μὲν δὴ δεῦ-
ρο διὰ τῆς θαλάττης φησὶν εἶναι τὴν λεχθεῖσαν γραμ-
μὴν καὶ τῶν παρακειμένων ἠπείρων (καὶ γὰρ αὐτὴν 
ὅλην τὴν καθ' ἡμᾶς θάλατταν οὕτως ἐπὶ μῆκος τετά-
σθαι μέχρι τῆς Κιλικίας), εἶτα ἐπ' εὐθείας πως ἐκβάλ-
λεσθαι παρ' ὅλην τὴν ὀρεινὴν τοῦ Ταύρου μέχρι τῆς 
Ἰνδικῆς· τὸν γὰρ Ταῦρον ἐπ' εὐθείας τῇ ἀπὸ στηλῶν 
θαλάττῃ τεταμένον δίχα τὴν Ἀσίαν διαιρεῖν ὅλην ἐπὶ 
μῆκος, τὸ μὲν αὐτῆς μέρος βόρειον ποιοῦντα τὸ δὲ 
νότιον, ὥσθ' ὁμοίως καὶ αὐτὸν ἐπὶ τοῦ δι' Ἀθηνῶν 
ἱδρῦσθαι παραλλήλου καὶ τὴν ἀπὸ στηλῶν μέχρι δεῦρο 
θάλατταν. 



Strabo Geogr., Geographica 
Book 2, chapter 1, section 2, line 4

Ταῦτα δ' εἰπὼν οἴεται δεῖν διορθῶσαι τὸν ἀρ-
χαῖον γεωγραφικὸν πίνακα· πολὺ γὰρ ἐπὶ τὰς ἄρκτους 
παραλλάττειν τὰ ἑωθινὰ μέρη τῶν ὀρῶν κατ' αὐτόν, 
συνεπισπᾶσθαι δὲ καὶ τὴν Ἰνδικὴν ἀρκτικωτέραν ἢ   
δεῖ γινομένην. 



Strabo Geogr., Geographica 
Book 2, chapter 1, section 2, line 6

                 πίστιν δὲ τούτου φέρει μίαν μὲν ταύ-
την, ὅτι τὰ τῆς Ἰνδικῆς ἄκρα τὰ μεσημβρινώτατα ὁμο-
λογοῦσι πολλοὶ τοῖς κατὰ Μερόην ἀνταίρειν τόποις, 
ἀπό τε τῶν ἀέρων καὶ τῶν οὐρανίων τεκμαιρόμενοι, 
ἐντεῦθεν δ' ἐπὶ τὰ βορειότατα τῆς Ἰνδικῆς τὰ πρὸς 
τοῖς Καυκασίοις ὄρεσι Πατροκλῆς, ὁ μάλιστα πιστεύε-
σθαι δίκαιος διά τε τὸ ἀξίωμα καὶ διὰ τὸ μὴ ἰδιώτης 
εἶναι τῶν γεωγραφικῶν, φησὶ σταδίους μυρίους καὶ 
πεντακισχιλίους· ἀλλὰ μὴν καὶ τὸ ἀπὸ Μερόης ἐπὶ τὸν 
δι' Ἀθηνῶν παράλληλον τοσοῦτόν πώς ἐστιν ὥστε 
τῆς Ἰνδικῆς τὰ προσάρκτια μέρη συνάπτοντα τοῖς 




Strabo Geogr., Geographica 
Book 2, chapter 1, section 2, line 15

την, ὅτι τὰ τῆς Ἰνδικῆς ἄκρα τὰ μεσημβρινώτατα ὁμο-
λογοῦσι πολλοὶ τοῖς κατὰ Μερόην ἀνταίρειν τόποις, 
ἀπό τε τῶν ἀέρων καὶ τῶν οὐρανίων τεκμαιρόμενοι, 
ἐντεῦθεν δ' ἐπὶ τὰ βορειότατα τῆς Ἰνδικῆς τὰ πρὸς 
τοῖς Καυκασίοις ὄρεσι Πατροκλῆς, ὁ μάλιστα πιστεύε-
σθαι δίκαιος διά τε τὸ ἀξίωμα καὶ διὰ τὸ μὴ ἰδιώτης 
εἶναι τῶν γεωγραφικῶν, φησὶ σταδίους μυρίους καὶ 
πεντακισχιλίους· ἀλλὰ μὴν καὶ τὸ ἀπὸ Μερόης ἐπὶ τὸν 
δι' Ἀθηνῶν παράλληλον τοσοῦτόν πώς ἐστιν ὥστε 
τῆς Ἰνδικῆς τὰ προσάρκτια μέρη συνάπτοντα τοῖς 
Καυκασίοις ὄρεσιν εἰς τοῦτον τελευτᾶν τὸν κύκλον. 



Strabo Geogr., Geographica 
Book 2, chapter 1, section 3, line 14

                                 ἀπὸ δὲ Μερόης ἐπὶ τὸν Ἑλ-
λήσποντον οὐ πλείους εἰσὶ τῶν μυρίων καὶ ὀκτακισχι-
λίων σταδίων, ὅσοι καὶ ἀπὸ τοῦ μεσημβρινοῦ πλευροῦ 
τῆς Ἰνδικῆς πρὸς τὰ περὶ τοὺς Βακτρίους μέρη, προς-
τεθέντων τρισχιλίων τοῖς μυρίοις καὶ πεντακισχι-
λίοις, ὧν οἱ μὲν τοῦ πλάτους ἦσαν τῶν ὀρῶν οἱ δὲ 
τῆς Ἰνδικῆς. 



Strabo Geogr., Geographica 
Book 2, chapter 1, section 5, line 5

τίνες οὖν ἦσαν οἱ φάσκοντες τὰ μεσημβρινὰ ἄκρα τῆς 
Ἰνδικῆς ἀνταίρειν τοῖς κατὰ Μερόην; 



Strabo Geogr., Geographica 
Book 2, chapter 1, section 7, line 4

Ἔτι φησὶν ὁ Ἵππαρχος ἐν τῷ δευτέρῳ ὑπομνήματι 
αὐτὸν τὸν Ἐρατοσθένη διαβάλλειν τὴν τοῦ Πατρο-
κλέους πίστιν ἐκ τῆς πρὸς Μεγασθένη διαφωνίας περὶ 
τοῦ μήκους τῆς Ἰνδικῆς τοῦ κατὰ τὸ βόρειον πλευρόν, 
τοῦ μὲν Μεγασθένους λέγοντος σταδίων μυρίων ἑξακις-
χιλίων, τοῦ δὲ Πατροκλέους χιλίοις λείπειν φαμένου· 
ἀπὸ γάρ τινος ἀναγραφῆς σταθμῶν ὁρμηθέντα τοῖς 
μὲν ἀπιστεῖν διὰ τὴν διαφωνίαν, ἐκείνῃ δὲ προσέχειν. 



Strabo Geogr., Geographica 
Book 2, chapter 1, section 7, line 13

                                                                εἰ 
οὖν διὰ τὴν διαφωνίαν ἐνταῦθα ἄπιστος ὁ Πατροκλῆς, 
καίτοι παρὰ χιλίους σταδίους τῆς διαφορᾶς οὔσης, 
πόσῳ χρὴ μᾶλλον ἀπιστεῖν ἐν οἷς παρὰ ὀκτακισχιλίους 
ἡ διαφορά ἐστι, πρὸς δύο καὶ ταῦτα ἄνδρας συμφω-
νοῦντας ἀλλήλοις, τῶν μὲν λεγόντων τὸ τῆς Ἰνδικῆς 
πλάτος δισμυρίων σταδίων, τοῦ δὲ μυρίων καὶ δις-
χιλίων; 



Strabo Geogr., Geographica 
Book 2, chapter 1, section 9, line 1

Ἅπαντες μὲν τοίνυν οἱ περὶ τῆς Ἰνδικῆς γράψαν-
τες ὡς ἐπὶ τὸ πολὺ ψευδολόγοι γεγόνασι, καθ' ὑπερ-
βολὴν δὲ Δηίμαχος, τὰ δὲ δεύτερα λέγει Μεγασθένης, 
Ὀνησίκριτος δὲ καὶ Νέαρχος καὶ ἄλλοι τοιοῦτοι παρα-
ψελλίζοντες ἤδη. 



Strabo Geogr., Geographica 
Book 2, chapter 1, section 11, line 15

                                       ὥστ' οὐδ' ἐκεῖνο 
εὖ λέγει τό “ἐπειδὴ οὐκ ἔχομεν λέγειν οὔθ' ἡμέρας με-
“γίστης πρὸς τὴν βραχυτάτην λόγον οὔτε γνώμονος 
“πρὸς σκιὰν ἐπὶ τῇ παρωρείᾳ τῇ ἀπὸ Κιλικίας μέχρι 
“Ἰνδῶν, οὐδ' εἰ ἐπὶ παραλλήλου γραμμῆς ἐστιν ἡ λό-
“ξωσις ἔχομεν εἰπεῖν, ἀλλ' ἐᾶν ἀδιόρθωτον, λοξὴν φυ-
“λάξαντες, ὡς οἱ ἀρχαῖοι πίνακες παρέχουσι. 



Strabo Geogr., Geographica 
Book 2, chapter 1, section 12, line 3

ὅρα γάρ, εἰ τοῦτο μὲν μὴ κινοίη τις τὸ τὰ ἄκρα τῆς Ἰν-
δικῆς τὰ μεσημβρινὰ ἀνταίρειν τοῖς κατὰ Μερόην, 
μηδὲ τὸ διάστημα τὸ ἀπὸ Μερόης ἐπὶ τὸ στόμα τὸ κατὰ 
τὸ Βυζάντιον, ὅτι ἐστὶ περὶ μυρίους σταδίους καὶ ὀκτα-
κισχιλίους, ποιοίη δὲ τρισμυρίων τὸ ἀπὸ τῶν μεσημ-
βρινῶν Ἰνδῶν μέχρι τῶν ὀρῶν, ὅσα ἂν συμβαίη ἄτοπα. 



Strabo Geogr., Geographica 
Book 2, chapter 1, section 14, line 4

            αὕτη δ' ἐστὶν ἡ περὶ τὴν Ταπροβάνην· ἡ 
δὲ Ταπροβάνη πεπίστευται σφόδρα ὅτι τῆς Ἰνδικῆς   
πρόκειται πελαγία μεγάλη νῆσος πρὸς νότον, μηκύνε-
ται δὲ ἐπὶ τὴν Αἰθιοπίαν πλέον ἢ πεντακισχιλίους 
σταδίους, ὥς φασιν, ἐξ ἧς καὶ ἐλέφαντα κομίζεσθαι 
πολὺν εἰς τὰ τῶν Ἰνδῶν ἐμπόρια καὶ χελώνεια καὶ ἄλ-
λον φόρτον. 



Strabo Geogr., Geographica 
Book 2, chapter 1, section 14, line 11

             ταύτῃ δὴ τῇ νήσῳ πλάτος προστεθὲν τὸ 
ἀνάλογον τῷ μήκει καὶ δίαρμα τὸ ἐπ' αὐτὴν ἐκ τῆς 
Ἰνδικῆς τῶν μὲν τρισχιλίων σταδίων οὐκ ἂν ἔλαττον 
ποιήσειε διάστημα, ὅσον ἦν τὸ ἀπὸ τοῦ ὅρου τῆς οἰ-
κουμένης εἰς Μερόην, εἴπερ μέλλει τὰ ἄκρα τῆς Ἰνδι-
κῆς ἀνταίρειν τῇ Μερόῃ· πιθανώτερον δ' ἐστὶ καὶ 
πλείους τῶν τρισχιλίων τιθέναι. 



Strabo Geogr., Geographica 
Book 2, chapter 1, section 14, line 22

         τίς ἂν οὖν θαρρήσειε ταῦτα λέγειν, ἀκούων 
καὶ τῶν πάλαι καὶ τῶν νῦν τὴν εὐκρασίαν καὶ τὴν εὐ-
καρπίαν λεγόντων πρῶτον μὲν τὴν τῶν προσβόρρων 
Ἰνδῶν, ἔπειτα δὲ καὶ τὴν ἐν τῇ Ὑρκανίᾳ καὶ τῇ Ἀρίᾳ 
καὶ ἐφεξῆς τῇ τε Μαργιανῇ καὶ τῇ Βακτριανῇ; 



Strabo Geogr., Geographica 
Book 2, chapter 1, section 14, line 26

                                                          ἅπασαι 
γὰρ αὗται προσεχεῖς μέν εἰσι τῇ βορείῳ πλευρᾷ τοῦ 
Ταύρου, καὶ ἥ γε Βακτριανὴ καὶ πλησιάζει τῇ εἰς Ἰν-
δοὺς ὑπερθέσει, τοσαύτῃ δ' εὐδαιμονίᾳ κέχρηνται 
ὥστε πάμπολύ τι ἀπέχειν τῆς ἀοικήτου. 



Strabo Geogr., Geographica 
Book 2, chapter 1, section 15, line 14

       καὶ τὸν Ὦξον δὲ τὸν ὁρίζοντα τὴν Βακτριανὴν 
ἀπὸ τῆς Σογδιανῆς οὕτω φασὶν εὔπλουν εἶναι ὥστε 
τὸν Ἰνδικὸν φόρτον ὑπερκομισθέντα εἰς αὐτὸν ῥᾳδίως 
εἰς τὴν Ὑρκανίαν κατάγεσθαι καὶ τοὺς ἐφεξῆς τόπους 
μέχρι τοῦ Πόντου διὰ τῶν ποταμῶν. 



Strabo Geogr., Geographica 
Book 2, chapter 1, section 17, line 22

                                      ἔσται δὲ Βάκτρα καὶ 
τοῦ στόματος τῆς Κασπίας θαλάττης εἴτε Ὑρκανίας 
πάμπολύ τι ἀρκτικώτερα, ὅπερ τοῦ μυχοῦ τῆς Κασπίας 
καὶ τῶν Ἀρμενιακῶν καὶ Μηδικῶν ὀρῶν διέχει περὶ 
ἑξακισχιλίους σταδίους, καὶ δοκεῖ αὐτῆς τῆς παραλίας 
μέχρι τῆς Ἰνδικῆς ἀρκτικώτερον εἶναι σημεῖον καὶ περί-
πλουν ἔχειν ἀπὸ τῆς Ἰνδικῆς δυνατόν, ὥς φησιν ὁ τῶν 
τόπων ἡγησάμενος τούτων Πατροκλῆς. 



Strabo Geogr., Geographica 
Book 2, chapter 1, section 19, line 3

Πάλιν δ' ἐκείνου τὸν Δηίμαχον ἰδιώτην ἐνδείξα-
σθαι βουλομένου καὶ ἄπειρον τῶν τοιούτων· οἴεσθαι 
γὰρ τὴν Ἰνδικὴν μεταξὺ κεῖσθαι τῆς τε φθινοπωρινῆς 
ἰσημερίας καὶ τῶν τροπῶν τῶν χειμερινῶν, Μεγασθέ-
νει τε ἀντιλέγειν φήσαντι ἐν τοῖς νοτίοις μέρεσι τῆς 
Ἰνδικῆς τάς τε ἄρκτους ἀποκρύπτεσθαι καὶ τὰς σκιὰς 
ἀντιπίπτειν· μηδέτερον γὰρ τούτων μηδαμοῦ τῆς Ἰν-
δικῆς συμβαίνειν· ταῦτα δὴ φάσκοντος ἀμαθῶς λέγε-
σθαι· τό τε γὰρ τὴν φθινοπωρινὴν τῆς ἐαρινῆς δια-
φέρειν οἴεσθαι κατὰ τὴν διάστασιν τὴν πρὸς τὰς τρο-
πὰς ἀμαθές, τοῦ τε κύκλου τοῦ αὐτοῦ ὄντος καὶ τῆς 
ἀνατολῆς· τοῦ τε διαστήματος τοῦ ἐπὶ τῆς γῆς

τροπι-



Strabo Geogr., Geographica 
Book 2, chapter 1, section 19, line 13

ἰσημερίας καὶ τῶν τροπῶν τῶν χειμερινῶν, Μεγασθέ-
νει τε ἀντιλέγειν φήσαντι ἐν τοῖς νοτίοις μέρεσι τῆς 
Ἰνδικῆς τάς τε ἄρκτους ἀποκρύπτεσθαι καὶ τὰς σκιὰς 
ἀντιπίπτειν· μηδέτερον γὰρ τούτων μηδαμοῦ τῆς Ἰν-
δικῆς συμβαίνειν· ταῦτα δὴ φάσκοντος ἀμαθῶς λέγε-
σθαι· τό τε γὰρ τὴν φθινοπωρινὴν τῆς ἐαρινῆς δια-
φέρειν οἴεσθαι κατὰ τὴν διάστασιν τὴν πρὸς τὰς τρο-
πὰς ἀμαθές, τοῦ τε κύκλου τοῦ αὐτοῦ ὄντος καὶ τῆς 
ἀνατολῆς· τοῦ τε διαστήματος τοῦ ἐπὶ τῆς γῆς τροπι-
κοῦ ἀπὸ τοῦ ἰσημερινοῦ, ὧν μεταξὺ τίθησι τὴν Ἰνδι-
κὴν ἐκεῖνος, δειχθέντος ἐν τῇ ἀναμετρήσει πολὺ ἐλάτ-
τονος τῶν δισμυρίων σταδίων, συμβῆναι ἂν καὶ κατ' 
αὐτὸν ἐκεῖνον, ὅπερ αὐτὸς νομίζει, οὐχ ὃ ἐκεῖνος· δυεῖν 
μὲν γὰρ ἢ καὶ τριῶν μυριάδων οὖσαν τὴν Ἰνδικὴν οὐδὲ 
πεσεῖν μεταξὺ τοσούτου διαστήματος, ὅσων δ' αὐτὸς 
εἴρηκε, πεσεῖν ἄν· τῆς δ' αὐτῆς ἀγνοίας εἶναι καὶ τὸ 
μηδαμοῦ τῆς Ἰνδικῆς ἀποκρύπτεσθαι φάσκειν τὰς ἄρ-
κτους μηδὲ τὰς σκιὰς ἀντιπίπτειν, ὅτε γε καὶ πεντα-
κισχιλίους προελθόντι ἀπ' Ἀλεξανδρείας εὐθὺς

συμ-



Strabo Geogr., Geographica 
Book 2, chapter 1, section 19, line 17

δικῆς συμβαίνειν· ταῦτα δὴ φάσκοντος ἀμαθῶς λέγε-
σθαι· τό τε γὰρ τὴν φθινοπωρινὴν τῆς ἐαρινῆς δια-
φέρειν οἴεσθαι κατὰ τὴν διάστασιν τὴν πρὸς τὰς τρο-
πὰς ἀμαθές, τοῦ τε κύκλου τοῦ αὐτοῦ ὄντος καὶ τῆς 
ἀνατολῆς· τοῦ τε διαστήματος τοῦ ἐπὶ τῆς γῆς τροπι-
κοῦ ἀπὸ τοῦ ἰσημερινοῦ, ὧν μεταξὺ τίθησι τὴν Ἰνδι-
κὴν ἐκεῖνος, δειχθέντος ἐν τῇ ἀναμετρήσει πολὺ ἐλάτ-
τονος τῶν δισμυρίων σταδίων, συμβῆναι ἂν καὶ κατ' 
αὐτὸν ἐκεῖνον, ὅπερ αὐτὸς νομίζει, οὐχ ὃ ἐκεῖνος· δυεῖν 
μὲν γὰρ ἢ καὶ τριῶν μυριάδων οὖσαν τὴν Ἰνδικὴν οὐδὲ 
πεσεῖν μεταξὺ τοσούτου διαστήματος, ὅσων δ' αὐτὸς 
εἴρηκε, πεσεῖν ἄν· τῆς δ' αὐτῆς ἀγνοίας εἶναι καὶ τὸ 
μηδαμοῦ τῆς Ἰνδικῆς ἀποκρύπτεσθαι φάσκειν τὰς ἄρ-
κτους μηδὲ τὰς σκιὰς ἀντιπίπτειν, ὅτε γε καὶ πεντα-
κισχιλίους προελθόντι ἀπ' Ἀλεξανδρείας εὐθὺς συμ-
βαίνειν ἄρχεται· ταῦτα δὴ εἰπόντος, εὐθύνει πάλιν 
οὐκ εὖ ὁ Ἵππαρχος, πρῶτον ἀντὶ τοῦ χειμερινοῦ τρο-
πικοῦ τὸν θερινὸν δεξάμενος, εἶτ' οὐκ οἰόμενος δεῖν 
μάρτυρι χρῆσθαι τῶν μαθηματικῶν ἀναστρολογήτῳ 




Strabo Geogr., Geographica 
Book 2, chapter 1, section 19, line 20

πὰς ἀμαθές, τοῦ τε κύκλου τοῦ αὐτοῦ ὄντος καὶ τῆς 
ἀνατολῆς· τοῦ τε διαστήματος τοῦ ἐπὶ τῆς γῆς τροπι-
κοῦ ἀπὸ τοῦ ἰσημερινοῦ, ὧν μεταξὺ τίθησι τὴν Ἰνδι-
κὴν ἐκεῖνος, δειχθέντος ἐν τῇ ἀναμετρήσει πολὺ ἐλάτ-
τονος τῶν δισμυρίων σταδίων, συμβῆναι ἂν καὶ κατ' 
αὐτὸν ἐκεῖνον, ὅπερ αὐτὸς νομίζει, οὐχ ὃ ἐκεῖνος· δυεῖν 
μὲν γὰρ ἢ καὶ τριῶν μυριάδων οὖσαν τὴν Ἰνδικὴν οὐδὲ 
πεσεῖν μεταξὺ τοσούτου διαστήματος, ὅσων δ' αὐτὸς 
εἴρηκε, πεσεῖν ἄν· τῆς δ' αὐτῆς ἀγνοίας εἶναι καὶ τὸ 
μηδαμοῦ τῆς Ἰνδικῆς ἀποκρύπτεσθαι φάσκειν τὰς ἄρ-
κτους μηδὲ τὰς σκιὰς ἀντιπίπτειν, ὅτε γε καὶ πεντα-
κισχιλίους προελθόντι ἀπ' Ἀλεξανδρείας εὐθὺς συμ-
βαίνειν ἄρχεται· ταῦτα δὴ εἰπόντος, εὐθύνει πάλιν 
οὐκ εὖ ὁ Ἵππαρχος, πρῶτον ἀντὶ τοῦ χειμερινοῦ τρο-
πικοῦ τὸν θερινὸν δεξάμενος, εἶτ' οὐκ οἰόμενος δεῖν 
μάρτυρι χρῆσθαι τῶν μαθηματικῶν ἀναστρολογήτῳ 
ἀνθρώπῳ, ὥσπερ τοῦ Ἐρατοσθένους προηγουμένως 
τὴν ἐκείνου μαρτυρίαν ἐγκρίνοντος, ἀλλ' οὐ κοινῷ τινι 
ἔθει χρωμένου πρὸς τοὺς ματαιολογοῦντας. 



Strabo Geogr., Geographica 
Book 2, chapter 1, section 20, line 2

Νυνὶ μὲν οὖν ὑποθέμενοι τὰ νοτιώτατα τῆς Ἰν-
δικῆς ἀνταίρειν τοῖς κατὰ Μερόην, ὅπερ εἰρήκασι πολ-
λοὶ καὶ πεπιστεύκασιν, ἐπεδείξαμεν τὰ συμβαίνοντα 
ἄτοπα. 



Strabo Geogr., Geographica 
Book 2, chapter 1, section 20, line 18

                                                       τὸ 
μὲν οὖν κατὰ Μερόην κλίμα Φίλωνά τε τὸν συγγρά-
ψαντα τὸν εἰς Αἰθιοπίαν πλοῦν ἱστορεῖν, ὅτι πρὸ 
πέντε καὶ τετταράκοντα ἡμερῶν τῆς θερινῆς τροπῆς 
κατὰ κορυφὴν γίνεται ὁ ἥλιος, λέγειν δὲ καὶ τοὺς λό-
γους τοῦ γνώμονος πρός τε τὰς τροπικὰς σκιὰς καὶ τὰς 
ἰσημερινάς, αὐτόν τε Ἐρατοσθένη συμφωνεῖν ἔγγιστα 
τῷ Φίλωνι, τὸ δ' ἐν τῇ Ἰνδικῇ κλίμα μηδένα ἱστορεῖν, 
μηδ' αὐτὸν Ἐρατοσθένη. 



Strabo Geogr., Geographica 
Book 2, chapter 1, section 20, line 22

                            εἰ δὲ δὴ καὶ αἱ ἄρκτοι ἐκεῖ 
ἀμφότεραι, ὡς οἴεται, ἀποκρύπτονται, πιστεύων τοῖς 
περὶ Νέαρχον, μὴ δυνατὸν εἶναι ἐπὶ ταὐτοῦ παραλλή-
λου κεῖσθαι τήν τε Μερόην καὶ τὰ ἄκρα τῆς Ἰνδικῆς. 



Strabo Geogr., Geographica 
Book 2, chapter 1, section 20, line 25

εἰ μὲν τοίνυν περὶ τῶν ἄρκτων ἀμφοτέρων ὅτι ἀπο-
κρύπτονται συναποφαίνεται τοῖς εἰποῦσιν Ἐρατοσθέ-
νης, πῶς περὶ τοῦ ἐν τῇ Ἰνδικῇ κλίματος οὐδεὶς ἀπο-
φαίνεται, οὐδ' αὐτὸς Ἐρατοσθένης; 



Strabo Geogr., Geographica 
Book 2, chapter 1, section 20, line 29

                           οὐ συναποφαίνεται δέ γε, ἀλλὰ 
τοῦ Δηιμάχου φήσαντος μηδαμοῦ τῆς Ἰνδικῆς μήτ' 
ἀποκρύπτεσθαι τὰς ἄρκτους μήτ' ἀντιπίπτειν τὰς 
σκιάς, ἅπερ ὑπείληφεν ὁ Μεγασθένης, ἀπειρίαν αὐ-
τοῦ καταγιγνώσκει, τὸ συμπεπλεγμένον νομίζων ψεῦ-  
δος, ἐν ᾧ ὁμολογουμένως καὶ κατ' αὐτὸν τὸν Ἵππαρ-
χον τό γε μὴ ἀντιπίπτειν τὰς σκιὰς ψεῦδος ἐμπέπλε-
κται. 



Strabo Geogr., Geographica 
Book 2, chapter 1, section 20, line 36

      καὶ γὰρ εἰ μὴ τῇ Μερόῃ ἀνταίρει, τῆς γε Συή-
νης νοτιώτερα εἶναι τὰ ἄκρα τῆς Ἰνδικῆς συγχωρῶν 
φαίνεται. 



Strabo Geogr., Geographica 
Book 2, chapter 1, section 22, line 11

καὶ δὴ τοῦ νοτίου μέρους πρώτην εἰπὼν σφραγῖδα τὴν 
Ἰνδικήν, δευτέραν δὲ τὴν Ἀριανήν, ἐχούσας τι εὐπε-
ρίγραφον, ἴσχυσεν ἀμφοτέρων ἀποδοῦναι καὶ μῆκος   
καὶ πλάτος, τρόπον δέ τινα καὶ σχῆμα, ὡς ἂν γεωμε-
τρικός. 



Strabo Geogr., Geographica 
Book 2, chapter 1, section 22, line 14

         τὴν μὲν γὰρ Ἰνδικὴν ῥομβοειδῆ φησι διὰ τὸ 
τῶν πλευρῶν τὰς μὲν θαλάττῃ κλύζεσθαι τῇ τε νοτίῳ 
καὶ τῇ ἑῴᾳ, μὴ πάνυ κολπώδεις ᾐόνας ποιούσαις, τὰς δὲ 
λοιπὰς τὴν μὲν τῷ ὄρει τὴν δὲ τῷ ποταμῷ, κἀνταῦθα 
τοῦ εὐθυγράμμου σχήματος ὑπό τι σωζομένου· τὴν 
δ' Ἀριανὴν ὁρῶν τάς γε τρεῖς πλευρὰς ἔχουσαν εὐφυ-
εῖς πρὸς τὸ ἀποτελέσαι παραλληλόγραμμον σχῆμα, τὴν 
δ' ἑσπέριον οὐκ ἔχων σημείοις ἀφορίσαι διὰ τὸ ἐπαλ-
λάττειν ἀλλήλοις τὰ ἔθνη, γραμμῇ τινι ὅμως δηλοῖ τῇ 




Strabo Geogr., Geographica 
Book 2, chapter 1, section 22, line 26

ἑσπέριον μὲν οὖν καλεῖ τοῦτο τὸ πλευρόν, ἑῷον δὲ τὸ 
παρὰ τὸν Ἰνδόν, παράλληλα δ' οὐ λέγει· οὐδὲ τὰ λοι-
πά, τό τε τῷ ὄρει γραφόμενον καὶ τὸ τῇ θαλάττῃ, ἀλλὰ 
μόνον τὸ μὲν βόρειον τὸ δὲ νότιον. 



Strabo Geogr., Geographica 
Book 2, chapter 1, section 27, line 5

βουλόμενος γὰρ βεβαιοῦν τὸ ἐξ ἀρχῆς, ὅτι οὐ μεταθε-
τέον τὴν Ἰνδικὴν ἐπὶ τὰ νοτιώτερα, ὥσπερ Ἐρατοσθέ-
νης ἀξιοῖ, σαφὲς ἂν γενέσθαι τοῦτο μάλιστά φησιν ἐξ 
ὧν αὐτὸς ἐκεῖνος προφέρεται· τὴν γὰρ τρίτην μερίδα 
κατὰ τὴν βόρειον πλευρὰν εἰπόντα ἀφορίζεσθαι ὑπὸ 
τῆς ἀπὸ Κασπίων πυλῶν ἐπὶ τὸν Εὐφράτην γραμμῆς 
σταδίων μυρίων οὔσης, μετὰ ταῦτα ἐπιφέρειν ὅτι τὸ 
νότιον πλευρὸν τὸ ἀπὸ Βαβυλῶνος εἰς τοὺς ὅρους τῆς 
Καρμανίας μικρῷ πλειόνων ἐστὶν ἢ ἐνακισχιλίων, τὸ 
δὲ πρὸς δύσει πλευρὸν ἀπὸ Θαψάκου παρὰ τὸν Εὐφρά-
την ἐστὶν εἰς Βαβυλῶνα τετρακισχίλιοι ὀκτακόσιοι 




Strabo Geogr., Geographica 
Book 2, chapter 1, section 31, line 3

                                          καὶ τοῦ νοτίου μέ-
ρους ἡ μὲν Ἰνδικὴ περιώρισται πολλοῖς· καὶ γὰρ ὄρει 
καὶ ποταμῷ καὶ θαλάττῃ καὶ ἑνὶ ὀνόματι ὡς ἑνὸς ἔ-
θνους· ὥστε καὶ τετράπλευρος ὀρθῶς λέγεται καὶ ῥομ-
βοειδής. 



Strabo Geogr., Geographica 
Book 2, chapter 1, section 31, line 31

                                  οὐδὲ γὰρ ὑπὸ μεγέθους 
ἀπηναγκάσθαι λέγοι ἄν· καὶ γὰρ τὸ μέχρι θαλάττης 
οὐ μήν πω ἂν ἐξισάζοιτο τῇ Ἰνδικῇ, ἀλλ' οὐδὲ τῇ 
Ἀριανῇ, προσλαβὸν καὶ τὸ μέχρι τῶν ὅρων τῆς εὐδαί-
μονος Ἀραβίας καὶ τῆς Αἰγύπτου· ὥστε πολὺ κρεῖττον 
ἦν μέχρι δεῦρο προελθεῖν, τῆς τρίτης εἰπόντα σφρα-
γῖδος τοσαύτῃ προσθήκῃ τῇ μέχρι τῆς Συριακῆς θα-
λάττης τὸ μὲν νότιον πλευρὸν οὐχ ὥσπερ ἐκεῖνος εἶπεν 
ἔχον, οὐδ' ἐπ' εὐθείας, ἀλλ' ἀπὸ τῆς Καρμανίας εὐ-
θὺς τὴν δεξιὰν παραλίαν εἰσπλέοντι τὸν Περσικὸν 
κόλπον μέχρι τῆς ἐκβολῆς τοῦ Εὐφράτου, καὶ μετὰ 




Strabo Geogr., Geographica 
Book 2, chapter 1, section 34, line 38

παραλλήλου δυσμικωτέραν ἔχειν τὴν κοινὴν τομὴν τῆς 
κοινῆς τομῆς τοῦ αὐτοῦ παραλλήλου καὶ τῆς ἀπὸ Κας-
πίων πυλῶν καθηκούσης εὐθείας ἐπὶ τοὺς ὅρους τοὺς 
τῆς Καρμανίας καὶ τῆς Περσίδος πλείοσι τῶν τετρα-
κισχιλίων καὶ τετρακοσίων· σχεδὸν δή τι πρὸς τὴν διὰ 
Κασπίων πυλῶν μεσημβρινὴν γραμμὴν ἡμίσειαν ὀρθῆς 
ποιεῖν γωνίαν τὴν διὰ Κασπίων πυλῶν καὶ τῶν ὅρων 
τῆς τε Καρμανίας καὶ τῆς Περσίδος, καὶ νεύειν αὐτὴν 
ἐπὶ τὰ μέσα τῆς τε μεσημβρίας καὶ τῆς ἰσημερινῆς ἀνα-
τολῆς· ταύτῃ δ' εἶναι παράλληλον τὸν Ἰνδὸν ποτα-  
μόν, ὥστε καὶ τοῦτον ἀπὸ τῶν ὀρῶν οὐκ ἐπὶ μεσημ-
βρίαν ῥεῖν, ὥς φησιν Ἐρατοσθένης, ἀλλὰ μεταξὺ ταύ-
της καὶ τῆς ἰσημερινῆς ἀνατολῆς, καθάπερ ἐν τοῖς 
ἀρχαίοις πίναξι καταγέγραπται. 



Strabo Geogr., Geographica 
Book 2, chapter 1, section 34, line 47

                                       τίς δὲ τῷ Ἰνδῷ 
παράλληλον τὴν ἀπὸ Κασπίων πυλῶν ἐπὶ τοὺς ὅρους 
τῆς Καρμανίας; 



Strabo Geogr., Geographica 
Book 2, chapter 1, section 34, line 51

χωρὶς δὲ τούτων κἀκεῖνος εἴρηκεν, ὅτι ῥομβοειδές 
ἐστι τὸ σχῆμα τῆς Ἰνδικῆς· καὶ καθάπερ ἡ ἑωθινὴ 
πλευρὰ παρέσπασται πολὺ πρὸς ἕω, καὶ μάλιστα τῷ 
ἐσχάτῳ ἀκρωτηρίῳ, ὃ καὶ πρὸς μεσημβρίαν προπίπτει 
πλέον παρὰ τὴν ἄλλην ᾐόνα, οὕτω καὶ ἡ παρὰ τὸν Ἰν-
δὸν πλευρά. 



Strabo Geogr., Geographica 
Book 2, chapter 3, section 4, line 12

        τυχεῖν δή τινα Ἰνδὸν κομισθέντα ὡς τὸν βα-
σιλέα ὑπὸ τῶν φυλάκων τοῦ Ἀραβίου μυχοῦ, λεγόν-
των εὑρεῖν ἡμιθανῆ καταχθέντα μόνον ἐν νηί, τίς δ' 
εἴη καὶ πόθεν ἀγνοεῖν μὴ συνιέντας τὴν διάλεκτον· 
τὸν δὲ παραδοῦναι τοῖς διδάξουσιν ἑλληνίζειν. 



Strabo Geogr., Geographica 
Book 2, chapter 3, section 4, line 17

                                                     ἐκμα-
θόντα δὲ διηγήσασθαι διότι ἐκ τῆς Ἰνδικῆς πλέων πε-
ριπέσοι πλάνῃ καὶ σωθείη δεῦρο τοὺς σύμπλους ἀπο-
βαλὼν λιμῷ· ὑποληφθέντα δὲ ὑποσχέσθαι τὸν εἰς Ἰν-
δοὺς πλοῦν ἡγήσασθαι τοῖς ὑπὸ τοῦ βασιλέως προχει-
ρισθεῖσι· τούτων δὲ γενέσθαι τὸν Εὔδοξον. 



Strabo Geogr., Geographica 
Book 2, chapter 3, section 4, line 60

                    καὶ πρῶτον μὲν εἰς Δικαιάρχειαν, 
εἶτ' εἰς Μασσαλίαν ἐλθεῖν καὶ τὴν ἑξῆς παραλίαν μέχρι 
Γαδείρων, πανταχοῦ δὲ διακωδωνίζοντα ταῦτα καὶ 
χρηματιζόμενον κατασκευάσασθαι πλοῖον μέγα καὶ 
ἐφόλκια δύο λέμβοις λῃστρικοῖς ὅμοια, ἐμβιβάσαι τε 
μουσικὰ παιδισκάρια καὶ ἰατροὺς καὶ ἄλλους τεχνίτας, 
ἔπειτα πλεῖν ἐπὶ τὴν Ἰνδικὴν μετέωρον ζεφύροις συνε-
χέσι. 



Strabo Geogr., Geographica 
Book 2, chapter 3, section 4, line 72

                                                           ἀφέντα 
δὴ τὸν ἐπὶ Ἰνδοὺς πλοῦν ἀναστρέφειν· ἐν δὲ τῷ παρά-
πλῳ νῆσον εὔυδρον καὶ εὔδενδρον ἐρήμην ἰδόντα ση-
μειώσασθαι· σωθέντα δὲ εἰς τὴν Μαυρουσίαν, διαθέ-
μενον τοὺς λέμβους πεζῇ κομισθῆναι πρὸς τὸν Βόγον 
καὶ συμβουλεύειν αὐτῷ τὴν ναυστολίαν ἐπανελέσθαι 
ταύτην, ἰσχῦσαι δ' εἰς τἀναντία τοὺς φίλους ὑποτεί-
νοντας φόβον μὴ συμβῇ τὴν χώραν εὐεπιβούλευτον γε-
νέσθαι, δειχθείσης παρόδου τοῖς ἔξωθεν ἐπιστρατεύειν 
ἐθέλουσιν. 



Strabo Geogr., Geographica 
Book 2, chapter 3, section 5, line 14

      τίς γὰρ ἡ πιθανότης πρῶτον μὲν τῆς κατὰ τὸν 
Ἰνδὸν περιπετείας; 



Strabo Geogr., Geographica 
Book 2, chapter 3, section 5, line 19

                       ὁ γὰρ Ἀράβιος κόλπος ποταμοῦ 
δίκην στενός ἐστι καὶ μακρὸς [πεντακισχιλίους] ἐπὶ μυ-
ρίοις που σταδίους μέχρι τοῦ στόματος, καὶ τούτου 
στενοῦ παντάπασιν ὄντος· οὐκ εἰκὸς δ' οὔτ' ἔξω που 
τὸν πλοῦν ἔχοντας εἰς τὸν κόλπον παρωσθῆναι τοὺς 
Ἰνδοὺς κατὰ πλάνην (τὰ γὰρ στενὰ ἀπὸ τοῦ στόματος 
δηλώσειν ἔμελλε τὴν πλάνην), οὔτ' εἰς τὸν κόλπον 
ἐπίτηδες καταχθεῖσιν ἔτι πλάνης ἦν πρόφασις καὶ ἀνέ-
μων ἀστάτων. 



Strabo Geogr., Geographica 
Book 2, chapter 3, section 5, line 31

ὁ δὲ δὴ σπονδοφόρος καὶ θεωρὸς τῶν Κυζικηνῶν πῶς 
ἀφεὶς τὴν πόλιν εἰς Ἰνδοὺς ἔπλει; 



Strabo Geogr., Geographica 
Book 2, chapter 3, section 6, line 18

                                             ὑπονοεῖ δὲ   
τὸ τῆς οἰκουμένης μῆκος ἑπτά που μυριάδων σταδίων 
ὑπάρχον ἥμισυ εἶναι τοῦ ὅλον κύκλου καθ' ὃν εἴλη-
πται, ὥστε (φησίν) ἀπὸ τῆς δύσεως εὔρῳ πλέων ἐν το-
σαύταις μυριάσιν ἔλθοι ἂν εἰς Ἰνδούς. 



Strabo Geogr., Geographica 
Book 2, chapter 3, section 7, line 22

                                       ὁ δὲ συγχεῖ ταῦτα· 
ἐπαινῶν δὲ τὴν τοιαύτην διαίρεσιν τῶν ἠπείρων, οἵα 
νῦν ἐστι, παραδείγματι χρῆται τῷ τοὺς Ἰνδοὺς τῶν 
Αἰθιόπων διαφέρειν τῶν ἐν τῇ Λιβύῃ· εὐερνεστέρους 
γὰρ εἶναι καὶ ἧττον ἕψεσθαι τῇ ξηρασίᾳ τοῦ περιέχον-
τος· διὸ καὶ Ὅμηρον πάντας λέγοντα Αἰθίοπας δίχα 
διελεῖν “οἱ μὲν δυσομένου Ὑπερίονος, οἱ δ' ἀνιόντος,” 
Κράτητα δ' εἰσάγοντα τὴν ἑτέραν οἰκουμένην, ἣν οὐκ 
οἶδεν Ὅμηρος, δουλεύειν ὑποθέσει· καὶ ἔδει (φησί)   
μεταγράφειν οὕτως “ἠμὲν ἀπερχομένου Ὑπερίονος,” 
οἷον ἀπὸ τοῦ μεσημβρινοῦ περικλίνοντος. 



Strabo Geogr., Geographica 
Book 2, chapter 3, section 8, line 5

                                                          ἔπειθ' 
Ὅμηρος οὐ διὰ τοῦτο διαιρεῖ τοὺς Αἰθίοπας, ὅτι τοὺς 
Ἰνδοὺς ᾔδει τοιούτους τινὰς τοῖς σώμασιν (οὐδὲ γὰρ 
ἀρχὴν εἰδέναι τοὺς Ἰνδοὺς εἰκὸς Ὅμηρον, ὅπου γε 
οὐδ' ὁ Εὐεργέτης κατὰ τὸν Εὐδόξειον μῦθον ᾔδει τὰ 
κατὰ τὴν Ἰνδικήν, οὐδὲ τὸν πλοῦν τὸν ἐπ' αὐτήν), 
ἀλλὰ μᾶλλον κατὰ τὴν λεχθεῖσαν ὑφ' ἡμῶν πρότερον 
διαίρεσιν. 



Strabo Geogr., Geographica 
Book 2, chapter 5, section 1, line 13

                                                     αὐτὸ 
γὰρ τὸ εἰς ἐπίπεδον γράφειν ἐπιφάνειαν μίαν καὶ τὴν 
αὐτὴν τά τε Ἰβηρικὰ καὶ τὰ Ἰνδικὰ καὶ τὰ μέσα τού-
των, καὶ μηδὲν ἧττον δύσεις καὶ ἀνατολὰς ἀφορίζειν 
καὶ μεσουρανήσεις ὡς ἂν κοινὰς πᾶσι, τῷ μὲν προε-
πινοήσαντι τὴν τοῦ οὐρανοῦ διάθεσίν τε καὶ κίνησιν 
καὶ λαβόντι, ὅτι σφαιρικὴ μέν ἐστιν ἡ κατ' ἀλήθειαν 
τῆς γῆς ἐπιφάνεια, πλάττεται δὲ νῦν ἐπίπεδος πρὸς 
τὴν ὄψιν, γεωγραφικὴν ἔχει τὴν παράδοσιν, τῷ δ' ἄλ-
λως, οὐ γεωγραφικήν. 



Strabo Geogr., Geographica 
Book 2, chapter 5, section 9, line 11

ρυσθένους διαστήματι τὸ ἀπὸ Βορυσθένους ἐπὶ τὰς 
ἄρκτους τῶν τετρακισχιλίων σταδίων διάστημα, γίνε-
ται τὸ πᾶν μύριοι δισχίλιοι ἑπτακόσιοι στάδιοι, τὸ δ' 
ἀπὸ τῆς Ῥοδίας ἐπὶ τὸ νότιον πέρας ἐστὶ τῆς οἰκουμέ-
νης μύριοι ἑξακισχίλιοι ἑξακόσιοι, ὥστε τὸ σύμπαν   
πλάτος τῆς οἰκουμένης εἴη ἂν ἔλαττον τῶν τρισμυρίων 
ἀπὸ νότου πρὸς ἄρκτον· τὸ δέ γε μῆκος περὶ ἑπτὰ μυ-
ριάδας λέγεται, τοῦτο δ' ἐστὶν ἀπὸ δύσεως ἐπὶ τὰς 
ἀνατολὰς τὸ ἀπὸ τῶν ἄκρων τῆς Ἰβηρίας ἐπὶ τὰ ἄκρα 
τῆς Ἰνδικῆς, τὸ μὲν ὁδοῖς τὸ δὲ ταῖς ναυτιλίαις ἀναμε-
μετρημένον. 



Strabo Geogr., Geographica 
Book 2, chapter 5, section 12, line 14

                                               ἀπήγγελται 
δ' ἡμῖν καὶ ὑπὸ τῶν τὰ Παρθικὰ συγγραψάντων, τῶν 
περὶ Ἀπολλόδωρον τὸν Ἀρτεμιτηνόν, ἃ πολλῶν ἐκεῖ-
νοι μᾶλλον ἀφώρισαν, τὰ περὶ τὴν Ὑρκανίαν καὶ τὴν 
Βακτριανήν· τῶν τε Ῥωμαίων καὶ εἰς τὴν εὐδαίμονα 
Ἀραβίαν ἐμβαλόντων μετὰ στρατιᾶς νεωστί, ἧς ἡγεῖτο 
ἀνὴρ φίλος ἡμῖν καὶ ἑταῖρος Αἴλιος Γάλλος, καὶ τῶν 
ἐκ τῆς Ἀλεξανδρείας ἐμπόρων στόλοις ἤδη πλεόντων 
διὰ τοῦ Νείλου καὶ τοῦ Ἀραβίου κόλπου μέχρι τῆς 
Ἰνδικῆς, πολὺ μᾶλλον καὶ ταῦτα ἔγνωσται τοῖς νῦν ἢ 
τοῖς πρὸ ἡμῶν. 



Strabo Geogr., Geographica 
Book 2, chapter 5, section 12, line 19

                   ὅτε γοῦν Γάλλος ἐπῆρχε τῆς Αἰγύ-
πτου, συνόντες αὐτῷ καὶ συναναβάντες μέχρι Συήνης 
καὶ τῶν Αἰθιοπικῶν ὅρων ἱστοροῦμεν, ὅτι καὶ ἑκατὸν   
καὶ εἴκοσι νῆες πλέοιεν ἐκ Μυὸς ὅρμου πρὸς τὴν Ἰν-
δικήν, πρότερον ἐπὶ τῶν Πτολεμαϊκῶν βασιλέων ὀλί-
γων παντάπασι θαρρούντων πλεῖν καὶ τὸν Ἰνδικὸν 
ἐμπορεύεσθαι φόρτον. 



Strabo Geogr., Geographica 
Book 2, chapter 5, section 14, line 10

Ἔστι δή τι χλαμυδοειδὲς σχῆμα τῆς γῆς τῆς οἰκου-
μένης, οὗ τὸ μὲν πλάτος ὑπογράφει τὸ μέγιστον ἡ διὰ 
τοῦ Νείλου γραμμή, λαβοῦσα τὴν ἀρχὴν ἀπὸ τοῦ διὰ 
τῆς Κινναμωμοφόρου παραλλήλου καὶ τῆς τῶν Αἰ-
γυπτίων τῶν φυγάδων νήσου μέχρι τοῦ διὰ τῆς Ἰέρ-
νης παραλλήλου, τὸ δὲ μῆκος ἡ ταύτῃ πρὸς ὀρθὰς ἀπὸ 
τῆς ἑσπέρας διὰ στηλῶν καὶ τοῦ Σικελικοῦ πορθμοῦ 
μέχρι τῆς Ῥοδίας καὶ τοῦ Ἰσσικοῦ κόλπου, παρὰ τὸν 
Ταῦρον ἰοῦσα τὸν διεζωκότα τὴν Ἀσίαν καὶ καταστρέ-
φοντα ἐπὶ τὴν ἑῴαν θάλατταν μεταξὺ Ἰνδῶν καὶ τῶν 
ὑπὲρ τῆς Βακτριανῆς Σκυθῶν. 



Strabo Geogr., Geographica 
Book 2, chapter 5, section 14, line 31

     τῆς τε γὰρ Ἰνδικῆς νοτιωτέραν πολὺ τὴν Ταπρο-
βάνην καλουμένην νῆσον ἀποφαίνουσιν, οἰκουμένην 
ἔτι καὶ ἀνταίρουσαν τῇ τῶν Αἰγυπτίων νήσῳ καὶ τῇ 
τὸ κιννάμωμον φερούσῃ γῇ· τὴν γὰρ κρᾶσιν τῶν ἀέ-
ρων παραπλησίαν εἶναι· τῆς τε μετὰ τοὺς Ἰνδοὺς Σκυ-
θίας τῆς ὑστάτης ἀρκτικώτερά ἐστι τὰ κατὰ τὸ στόμα 
τῆς Ὑρκανίας θαλάττης καὶ ἔτι μᾶλλον τὰ κατὰ τὴν 
Ἰέρνην. 



Strabo Geogr., Geographica 
Book 2, chapter 5, section 31, line 5

διαιρουμένης γὰρ αὐτῆς ὑπὸ ὄρους τοῦ Ταύρου δίχα 
διατείνοντος ἀπὸ τῶν ἄκρων τῆς Παμφυλίας ἐπὶ τὴν 
ἑῴαν θάλατταν κατ' Ἰνδοὺς καὶ τοὺς ταύτῃ Σκύθας, 
τὸ μὲν πρὸς τὰς ἄρκτους νενευκὸς τῆς ἠπείρου μέρος 
καλοῦσιν οἱ Ἕλληνες ἐντὸς τοῦ Ταύρου, τὸ δὲ πρὸς 
μεσημβρίαν ἐκτός· τὰ δὴ συνεχῆ τῇ Μαιώτιδι καὶ τῷ 
Τανάιδι μέρη τὰ ἐντὸς τοῦ Ταύρου ἐστί. 



Strabo Geogr., Geographica 
Book 2, chapter 5, section 31, line 16

        ἔπειτα ἐντὸς τοῦ Ταύρου τὰ ὑπὲρ τῆς Ὑρκα-
νίας μέχρι πρὸς τὴν κατὰ Ἰνδοὺς καὶ Σκύθας τοὺς 
πρὸς τὴν αὐτὴν θάλατταν καὶ τὸ Ἰμάιον ὄρος. 



Strabo Geogr., Geographica 
Book 2, chapter 5, section 31, line 24

                                                        ταῦτα 
δ' ἔχουσι τὰ μὲν οἱ Μαιῶται καὶ οἱ μεταξὺ τῆς Ὑρκα-
νίας καὶ τοῦ Πόντου μέχρι τοῦ Καυκάσου καὶ Ἰβήρων 
καὶ Ἀλβανῶν, Σαυρομάται καὶ Σκύθαι καὶ Ἀχαιοὶ καὶ 
Ζυγοὶ καὶ Ἡνίοχοι, τὰ δ' ὑπὲρ τῆς Ὑρκανίας θαλάτ-
της Σκύθαι καὶ Ὑρκανοὶ καὶ Παρθυαῖοι καὶ Βάκτριοι 
καὶ Σογδιανοὶ καὶ τἆλλα τὰ ὑπερκείμενα μέρη τῶν 
Ἰνδῶν πρὸς ἄρκτον. 



Strabo Geogr., Geographica 
Book 2, chapter 5, section 32, line 5

          πρώτη δ' ἐστὶ τούτων ἡ Ἰνδική, ἔθνος μέγι-
στον τῶν πάντων καὶ εὐδαιμονέστατον, τελευτῶν πρός 
τε τὴν ἑῴαν θάλατταν καὶ τὴν νοτίαν τῆς Ἀτλαντικῆς. 



Strabo Geogr., Geographica 
Book 2, chapter 5, section 32, line 8

ἐν δὲ τῇ νοτίᾳ ταύτῃ θαλάττῃ πρόκειται τῆς Ἰνδικῆς 
νῆσος οὐκ ἐλάττων τῆς Βρεττανικῆς ἡ Ταπροβάνη· 
μετὰ δὲ τὴν Ἰνδικὴν ἐπὶ τὰ ἑσπέρια νεύουσιν, ἐν δεξιᾷ 
δ' ἔχουσι τὰ ὄρη χώρα ἐστὶ συχνή, φαύλως οἰκουμένη 
διὰ λυπρότητα ὑπ' ἀνθρώπων τελέως βαρβάρων οὐχ 
ὁμοεθνῶν· καλοῦσι δ' Ἀριανούς, ἀπὸ τῶν ὀρῶν δια-
τείνοντας μέχρι Γεδρωσίας καὶ Καρμανίας. 



Strabo Geogr., Geographica 
Book 2, chapter 5, section 36, line 7

Τοῖς δὲ κατὰ Μερόην καὶ Πτολεμαΐδα τὴν ἐν τῇ 
Τρωγλοδυτικῇ ἡ μεγίστη ἡμέρα ὡρῶν ἰσημερινῶν 
ἐστι τρισκαίδεκα· ἔστι δ' αὕτη ἡ οἴκησις μέση πως τοῦ 
τε ἰσημερινοῦ καὶ τοῦ δι' Ἀλεξανδρείας παρὰ χιλίους 
καὶ ὀκτακοσίους τοὺς πλεονάζοντας πρὸς τῷ ἰσημε-
ρινῷ· διήκει δ' ὁ διὰ Μερόης παράλληλος τῇ μὲν δι' 
ἀγνωρίστων μερῶν, τῇ δὲ διὰ τῶν ἄκρων τῆς Ἰνδικῆς. 



Strabo Geogr., Geographica 
Book 2, chapter 5, section 36, line 16

                     ὁ δὲ διὰ Συήνης παράλληλος τῇ 
μὲν διὰ τῆς τῶν Ἰχθυοφάγων τῶν κατὰ τὴν Γεδρωσίαν 
καὶ τῆς Ἰνδικῆς διήκει, τῇ δὲ διὰ τῶν νοτιωτέρων Κυ-
ρήνης πεντακισχιλίοις σταδίοις παρὰ μικρόν. 



Strabo Geogr., Geographica 
Book 2, chapter 5, section 38, line 15

                  διήκει δ' ὁ παράλληλος οὗτος τῇ μὲν 
διὰ Κυρήνης καὶ τῶν νοτιωτέρων Καρχηδόνος ἐνακο-
σίοις σταδίοις μέχρι Μαυρουσίας μέσης, τῇ δὲ δι' Αἰ-
γύπτου καὶ Κοίλης Συρίας καὶ τῆς ἄνω Συρίας καὶ 
Βαβυλωνίας καὶ Σουσιάδος Περσίδος Καρμανίας Γε-
δρωσίας τῆς ἄνω μέχρι τῆς Ἰνδικῆς. 



Strabo Geogr., Geographica 
Book 2, chapter 5, section 39, line 14

                                                      .. · 
διήκει δ' ὁ παράλληλος οὗτος κατ' Ἐρατοσθένη διὰ 
Καρίας Λυκαονίας Καταονίας Μηδίας Κασπίων πυ-
λῶν Ἰνδῶν τῶν κατὰ Καύκασον. 



Strabo Geogr., Geographica 
Book 3, chapter 4, section 1, line 15

                        ἐντεῦθεν δ' ἐπὶ τὸν Ἴβηρα ἄλλους 
τοσούτους σχεδόν τι (ταύτην δ' ἔχειν Ἐδητανούς), ἐν-
τὸς δὲ τοῦ Ἴβηρος μέχρι Πυρήνης καὶ τῶν Πομπηίου 
ἀναθημάτων χιλίους καὶ ἑξακοσίους· οἰκεῖν δὲ Ἐδη-
τανῶν τε ὀλίγους καὶ λοιπὸν τοὺς προσαγορευομένους 
Ἰνδικήτας μεμερισμένους τέτραχα. 



Strabo Geogr., Geographica 
Book 3, chapter 4, section 8, line 14

δίπολις δ' ἐστὶ τείχει διωρισμένη, πρότερον τῶν Ἰνδι-
κητῶν τινας προσοίκους ἔχουσα, οἳ καίπερ ἰδίᾳ πολι-
τευόμενοι κοινὸν ὅμως περίβολον ἔχειν ἐβούλοντο πρὸς 
τοὺς Ἕλληνας ἀσφαλείας χάριν, τῷ χρόνῳ δ' εἰς ταὐτὸ 
πολίτευμα συνῆλθον μικτόν τι ἔκ τε βαρβάρων καὶ Ἑλ-
ληνικῶν νομίμων, ὅπερ καὶ ἐπ' ἄλλων πολλῶν συνέβη. 



Strabo Geogr., Geographica 
Book 3, chapter 5, section 5, line 64

                                               Ἀλέξανδρος 
δὲ τῆς Ἰνδικῆς στρατείας ὅρια βωμοὺς ἔθετο ἐν τοῖς 
τόποις εἰς οὓς ὑστάτους ἀφίκετο τῶν πρὸς ταῖς ἀνατο-
λαῖς Ἰνδῶν, μιμούμενος τὸν Ἡρακλέα καὶ τὸν Διόνυ-
σον. 



Strabo Geogr., Geographica 
Book 3, chapter 5, section 6, line 5

              οὐδὲ ἐν τῇ Ἰνδικῇ στήλας φασὶν ὁρα-
θῆναι κειμένας οὔθ' Ἡρακλέους οὔτε Διονύσου· καὶ 
λεγομένων μέντοι καὶ δεικνυμένων τόπων τινῶν οἱ 
Μακεδόνες ἐπίστευον τούτους εἶναι στήλας, ἐν οἷς τι 
σημεῖον εὕρισκον ἢ τῶν περὶ τὸν Διόνυσον ἱστορου-
μένων ἢ τῶν περὶ τὸν Ἡρακλέα. 



Strabo Geogr., Geographica 
Book 3, chapter 5, section 6, line 39

                                                     τὸ δὲ 
ἐπ' αὐτὰς ἀναφέρειν τὰς ἐν τῷ Ἡρακλείῳ στήλας τῷ 
ἐνθάδε ἧττον εὔλογον, ὡς ἐμοὶ φαίνεται· οὐ γὰρ ἐμ-
πόρων ἀλλ' ἡγεμόνων μᾶλλον ἀρξάντων τοῦ ὀνόμα-
τος τούτου, κρατῆσαι πιθανὸν τὴν δόξαν, καθάπερ 
καὶ ἐπὶ τῶν Ἰνδικῶν στηλῶν. 



Strabo Geogr., Geographica 
Book 5, chapter 2, section 6, line 38

τοῦτό τε δὴ παράδοξον ἡ νῆσος ἔχει καὶ τὸ τὰ ὀρύγματα 
ἀναπληροῦσθαι πάλιν τῷ χρόνῳ τὰ μεταλλευθέντα,   
καθάπερ τοὺς πλαταμῶνάς φασι τοὺς ἐν Ῥόδῳ καὶ τὴν 
ἐν Πάρῳ πέτραν τὴν μάρμαρον καὶ τὰς ἐν Ἰνδοῖς 
ἅλας, ἅς φησι Κλείταρχος. 



Strabo Geogr., Geographica 
Book 7, chapter 3, section 8, line 9

                                        πλήρεις δὲ καὶ 
αἱ Περσικαὶ ἐπιστολαὶ τῆς ἁπλότητος ἧς λέγω, καὶ τὰ 
ὑπὸ τῶν Αἰγυπτίων καὶ Βαβυλωνίων καὶ Ἰνδῶν ἀπο-  
μνημονευόμενα. 



Strabo Geogr., Geographica 
Book 10, chapter 3, section 17, line 13

           οἵ τ' ἐπιμεληθέντες τῆς ἀρχαίας μουσικῆς 
Θρᾷκες λέγονται, Ὀρφεύς τε καὶ Μουσαῖος καὶ Θάμυ-
ρις, καὶ τῷ Εὐμόλπῳ δὲ τοὔνομα ἐνθένδε, καὶ οἱ τῷ 
Διονύσῳ τὴν Ἀσίαν ὅλην καθιερώσαντες μέχρι τῆς 
Ἰνδικῆς ἐκεῖθεν καὶ τὴν πολλὴν μουσικὴν μεταφέρου-
σι· καὶ ὁ μέν τίς φησιν “κιθάραν Ἀσιᾶτιν ῥάσσων,” ὁ 
δὲ τοὺς αὐλοὺς Βερεκυντίους καλεῖ καὶ Φρυγίους· καὶ 
τῶν ὀργάνων ἔνια βαρβάρως ὠνόμασται νάβλας καὶ 
σαμβύκη καὶ βάρβιτος καὶ μαγάδις καὶ ἄλλα πλείω. 



Strabo Geogr., Geographica 
Book 11, chapter 1, section 3, line 4

Πλάτος μὲν οὖν ἔχει τὸ ὄρος πολλαχοῦ καὶ τρισχι-
λίων σταδίων, μῆκος δ' ὅσον καὶ τὸ τῆς Ἀσίας, τετ-
τάρων που μυριάδων καὶ πεντακισχιλίων, ἀπὸ τῆς 
Ῥοδίων περαίας ἐπὶ τὰ ἄκρα τῆς Ἰνδικῆς καὶ Σκυθίας 
πρὸς τὰς ἀνατολάς. 



Strabo Geogr., Geographica 
Book 11, chapter 1, section 7, line 3

Δεύτερον δ' ἂν εἴη μέρος τὸ ὑπὲρ τῆς Ὑρκανίας 
θαλάττης, ἣν [καὶ] Κασπίαν καλοῦμεν, μέχρι τῶν κατ' 
Ἰνδοὺς Σκυθῶν. 



Strabo Geogr., Geographica 
Book 11, chapter 1, section 7, line 11

                                           τῶν δὲ ἄλλων 
τῶν ἔξω τοῦ Ταύρου τήν τε Ἰνδικὴν τίθεμεν καὶ τὴν 
Ἀριανὴν μέχρι τῶν ἐθνῶν τῶν καθηκόντων πρός τε 
τὴν κατὰ Πέρσας θάλατταν καὶ τὸν Ἀράβιον κόλπον 
καὶ τὸν Νεῖλον καὶ πρὸς τὸ Αἰγύπτιον πέλαγος καὶ τὸ 
Ἰσσικόν. 



Strabo Geogr., Geographica 
Book 11, chapter 5, section 5, line 4

Καὶ τὰ πρὸς τὸ ἔνδοξον θρυληθέντα οὐκ ἀνωμο-
λόγηται παρὰ πάντων, οἱ δὲ πλάσαντες ἦσαν οἱ κολα-
κείας μᾶλλον ἢ ἀληθείας φροντίζοντες· οἷον τὸ τὸν 
Καύκασον μετενεγκεῖν εἰς τὰ Ἰνδικὰ ὄρη καὶ τὴν πλη-
σιάζουσαν ἐκείνοις ἑῴαν θάλατταν ἀπὸ τῶν ὑπερκει-
μένων τῆς Κολχίδος καὶ τοῦ Εὐξείνου ὀρῶν· ταῦτα 
γὰρ οἱ Ἕλληνες καὶ Καύκασον ὠνόμαζον, διέχοντα τῆς 
Ἰνδικῆς πλείους ἢ τρισμυρίους σταδίους, καὶ ἐνταῦθα 
ἐμύθευσαν τὰ περὶ Προμηθέα καὶ τὸν δεσμὸν αὐτοῦ· 
ταῦτα γὰρ τὰ ὕστατα πρὸς ἕω ἐγνώριζον οἱ τότε. 



Strabo Geogr., Geographica 
Book 11, chapter 5, section 5, line 11

                                                           ἡ δὲ 
ἐπὶ Ἰνδοὺς στρατεία Διονύσου καὶ Ἡρακλέους ὑστε-
ρογενῆ τὴν μυθοποιίαν ἐμφαίνει, ἅτε τοῦ Ἡρακλέ-
ους καὶ τὸν Προμηθέα λῦσαι λεγομένου χιλιάσιν ἐτῶν 
ὕστερον. 



Strabo Geogr., Geographica 
Book 11, chapter 5, section 5, line 15

           καὶ ἦν μὲν ἐνδοξότερον τὸ τὸν Ἀλέξανδρον 
μέχρι τῶν Ἰνδικῶν ὀρῶν καταστρέψασθαι τὴν Ἀσίαν 
ἢ μέχρι τοῦ μυχοῦ τοῦ Εὐξείνου καὶ τοῦ Καυκάσου· 
ἀλλ' ἡ δόξα τοῦ ὄρους καὶ τοὔνομα καὶ τὸ τοὺς περὶ 
Ἰάσονα δοκεῖν μακροτάτην στρατείαν τελέσαι τὴν μέ-
χρι τῶν πλησίον Καυκάσου καὶ τὸ τὸν Προμηθέα πα-
ραδεδόσθαι δεδεμένον ἐπὶ τοῖς ἐσχάτοις τῆς γῆς ἐν τῷ 
Καυκάσῳ . 



Strabo Geogr., Geographica 
Book 11, chapter 5, section 5, line 22

            .. χαριεῖσθαί τι τῷ βασιλεῖ ὑπέλαβον 
τοὔνομα τοῦ ὄρους μετενέγκαντες εἰς τὴν Ἰνδικήν. 



Strabo Geogr., Geographica 
Book 11, chapter 5, section 8, line 11

                                Ἀβέακος μὲν οὖν ὁ τῶν 
Σιράκων βασιλεύς, ἡνίκα Φαρνάκης τὸν Βόσπορον 
εἶχε, δύο μυριάδας ἱππέων ἔστελλε, Σπαδίνης δ' ὁ τῶν 
Ἀόρσων καὶ * εἴκοσιν, οἱ δὲ ἄνω Ἄορσοι καὶ πλείονας· 
καὶ γὰρ ἐπεκράτουν πλείονος γῆς καὶ σχεδόν τι τῆς 
Κασπίων παραλίας τῆς πλείστης ἦρχον, ὥστε καὶ ἐνε-
πορεύοντο καμήλοις τὸν Ἰνδικὸν φόρτον καὶ τὸν Βα-
βυλώνιον παρά τε Ἀρμενίων καὶ Μήδων διαδεχόμε-
νοι· ἐχρυσοφόρουν δὲ διὰ τὴν εὐπορίαν. 



Strabo Geogr., Geographica 
Book 11, chapter 6, section 2, line 6

Εἰσπλέοντι δ' ἐν δεξιᾷ μὲν τοῖς Εὐρωπαίοις οἱ 
συνεχεῖς Σκύθαι νέμονται καὶ Σαρμάται οἱ μεταξὺ τοῦ 
Τανάιδος καὶ τῆς θαλάττης ταύτης, νομάδες οἱ πλεί-
ους, περὶ ὧν εἰρήκαμεν· ἐν ἀριστερᾷ δ' οἱ πρὸς ἕω 
Σκύθαι, νομάδες καὶ οὗτοι, μέχρι τῆς ἑῴας θαλάττης 
καὶ τῆς Ἰνδικῆς παρατείνοντες. 



Strabo Geogr., Geographica 
Book 11, chapter 7, section 2, line 24

                                φησὶ δ' Ἀριστόβουλος 
ὑλώδη οὖσαν τὴν Ὑρκανίαν δρῦν ἔχειν, πεύκην δὲ 
καὶ ἐλάτην καὶ πίτυν μὴ φύειν, τὴν δ' Ἰνδικὴν πλη-
θύειν τούτοις. 



Strabo Geogr., Geographica 
Book 11, chapter 7, section 3, line 6

                            Ἀριστόβουλος δὲ καὶ μέγιστον 
ἀποφαίνει τὸν Ὦξον τῶν ἑωραμένων ὑφ' ἑαυτοῦ κατὰ 
τὴν Ἀσίαν πλὴν τῶν Ἰνδικῶν· φησὶ δὲ καὶ εὔπλουν 
εἶναι καὶ οὗτος καὶ Ἐρατοσθένης παρὰ Πατροκλέους 
λαβών, καὶ πολλὰ τῶν Ἰνδικῶν φορτίων κατάγειν εἰς 
τὴν Ὑρκανίαν θάλατταν, ἐντεῦθεν δ' εἰς τὴν Ἀλβα-
νίαν περαιοῦσθαι καὶ διὰ τοῦ Κύρου καὶ τῶν ἑξῆς τό-
πων εἰς τὸν Εὔξεινον καταφέρεσθαι. 



Strabo Geogr., Geographica 
Book 11, chapter 7, section 4, line 17

                                  Πολύκλειτος δὲ καὶ πίστεις 
προφέρεται περὶ τοῦ λίμνην εἶναι τὴν θάλατταν ταύ-
την, ὄφεις τε γὰρ ἐκτρέφειν καὶ ὑπόγλυκυ εἶναι τὸ 
ὕδωρ· ὅτι δὲ καὶ οὐχ ἑτέρα τῆς Μαιώτιδός ἐστι, τεκμαι-
ρόμενος ἐκ τοῦ τὸν Τάναϊν εἰς αὐτὴν ἐμβάλλειν· ἐκ 
γὰρ τῶν αὐτῶν ὀρῶν τῶν Ἰνδικῶν ἐξ ὧν ὅ τε Ὦχος 
καὶ ὁ Ὦξος καὶ ἄλλοι πλείους φέρεται καὶ ὁ Ἰαξάρτης 
ἐκδίδωσί τε ὁμοίως ἐκείνοις εἰς τὸ Κάσπιον πέλαγος 
πάντων ἀρκτικώτατος. 



Strabo Geogr., Geographica 
Book 11, chapter 7, section 4, line 27

                 Ἐρατοσθένης δέ φησι καὶ ἐν τῇ Ἰνδικῇ 
φύεσθαι ἐλάτην καὶ ἐντεῦθεν ναυπηγήσασθαι τὸν 
στόλον Ἀλέξανδρον· πολλὰ δὲ καὶ ἄλλα τοιαῦτα συγ-
κρούειν Ἐρατοσθένης πειρᾶται, ἡμῖν δ' ἀποχρώντως 
εἰρήσθω περὶ αὐτῶν. 



Strabo Geogr., Geographica 
Book 11, chapter 8, section 1, line 2

Ἀπὸ δὲ τῆς Ὑρκανίας θαλάττης προϊόντι ἐπὶ τὴν 
ἕω δεξιὰ μέν ἐστι τὰ ὄρη μέχρι τῆς Ἰνδικῆς θαλάττης 
παρατείνοντα, ἅπερ οἱ Ἕλληνες ὀνομάζουσι Ταῦρον, 
ἀρξάμενα ἀπὸ τῆς Παμφυλίας καὶ τῆς Κιλικίας καὶ 
μέχρι δεῦρο προϊόντα ἀπὸ τῆς ἑσπέρας συνεχῆ καὶ 
τυγχάνοντα ἄλλων καὶ ἄλλων ὀνομάτων. 



Strabo Geogr., Geographica 
Book 11, chapter 8, section 8, line 10

      φησὶ δ' Ἐρατοσθένης τοὺς Ἀραχωτοὺς καὶ Μας-
σαγέτας τοῖς Βακτρίοις παρακεῖσθαι * πρὸς δύσιν παρὰ 
τὸν Ὦξον, καὶ Σάκας μὲν καὶ Σογδιανοὺς τοῖς ὅλοις 
ἐδάφεσιν ἀντικεῖσθαι τῇ Ἰνδικῇ, Βακτρίους δ' ἐπ' ὀλί-
γον· τὸ γὰρ πλέον τῷ Παροπαμισῷ παρακεῖσθαι· δι-
είργειν δὲ Σάκας μὲν καὶ Σογδιανοὺς τὸν Ἰαξάρτην, 
καὶ Σογδιανοὺς δὲ καὶ Βακτριανοὺς τὸν Ὦξον, μεταξὺ 
δὲ Ὑρκανῶν καὶ Ἀρίων Ταπύρους οἰκεῖν· κύκλῳ δὲ 
περὶ τὴν θάλατταν μετὰ τοὺς Ὑρκανοὺς Ἀμάρδους τε 
καὶ Ἀναριάκας καὶ Καδουσίους καὶ Ἀλβανοὺς καὶ 
Κασπίους καὶ Οὐιτίους, τάχα δὲ καὶ ἑτέρους μέχρι 
Σκυθῶν, ἐπὶ θάτερα δὲ μέρη τῶν Ὑρκανῶν Δέρβικας, 
τοὺς δὲ Καδουσίους συμψαύειν Μήδων καὶ Ματιανῶν 




Strabo Geogr., Geographica 
Book 11, chapter 8, section 9, line 10

                                      λέγει δὲ καὶ οὕτω 
τὰ διαστήματα ἀπὸ Κασπίων πυλῶν εἰς Ἰνδούς· εἰς 
μὲν Ἑκατόμπυλον χιλίους ἐνακοσίους ἑξήκοντά φασιν, 
εἰς δ' Ἀλεξάνδρειαν τὴν ἐν Ἀρίοις τετρακισχιλίους 
πεντακοσίους τριάκοντα, εἶτ' εἰς Προφθασίαν τὴν ἐν 
Δραγγῇ χιλίους ἑξακοσίους, οἱ δὲ πεντακοσίους, εἶτ' 
εἰς Ἀραχωτοὺς τὴν πόλιν τετρακισχιλίους ἑκατὸν εἴκο-
σιν, εἶτ' εἰς Ὀρτόσπανα ἐπὶ τὴν ἐκ Βάκτρων τρίοδον 
δισχιλίους, εἶτ' εἰς τὰ ὅρια τῆς Ἰνδικῆς χιλίους· ὁμοῦ 
μύριοι πεντακισχίλιοι τριακόσιοι. 



Strabo Geogr., Geographica 
Book 11, chapter 8, section 9, line 20

                                     ἐπ' εὐθείας δὲ τῷ 
διαστήματι τούτῳ τὸ συνεχὲς δεῖ νοεῖν, τὸ ἀπὸ τοῦ 
Ἰνδοῦ μέχρι τῆς ἑῴας θαλάττης μῆκος τῆς Ἰνδικῆς. 



Strabo Geogr., Geographica 
Book 11, chapter 10, section 1, line 14

                           συντελὴς δ' ἦν αὐτῇ καὶ ἡ 
Δραγγιανὴ μέχρι Καρμανίας, τὸ μὲν πλέον τοῖς νο-
τίοις μέρεσι τῶν ὀρῶν ὑποπεπτωκυῖα, ἔχουσα μέντοι 
τινὰ τῶν μερῶν καὶ τοῖς ἀρκτικοῖς πλησιάζοντα τοῖς 
κατὰ τὴν Ἀρίαν· καὶ ἡ Ἀραχωσία δὲ οὐ πολὺ ἄπωθέν 
ἐστι, καὶ αὕτη τοῖς νοτίοις μέρεσι τῶν ὀρῶν ὑποπε-
πτωκυῖα καὶ μέχρι τοῦ Ἰνδοῦ ποταμοῦ τεταμένη, μέ-
ρος οὖσα τῆς Ἀριανῆς. 



Strabo Geogr., Geographica 
Book 11, chapter 11, section 1, line 6

                                                τοσοῦτον 
δὲ ἴσχυσαν οἱ ἀποστήσαντες Ἕλληνες αὐτὴν διὰ τὴν 
ἀρετὴν τῆς χώρας ὥστε τῆς τε Ἀριανῆς ἐπεκράτουν 
καὶ τῶν Ἰνδῶν, ὥς φησιν Ἀπολλόδωρος ὁ Ἀρταμιτη-
νός, καὶ πλείω ἔθνη κατεστρέψαντο ἢ Ἀλέξανδρος, 
καὶ μάλιστα Μένανδρος (εἴ γε καὶ τὸν Ὕπανιν διέβη 
πρὸς ἔω καὶ μέχρι τοῦ Ἰμάου προῆλθε) τὰ μὲν αὐτὸς 
τὰ δὲ Δημήτριος ὁ Εὐθυδήμου υἱὸς τοῦ Βακτρίων βα-
σιλέως· οὐ μόνον δὲ τὴν Παταληνὴν κατέσχον ἀλλὰ 
καὶ τῆς ἄλλης παραλίας τήν τε Σαραόστου καλουμέ-
νην καὶ τὴν Σιγέρδιδος βασιλείαν. 



Strabo Geogr., Geographica 
Book 11, chapter 11, section 6, line 13

                                     οὐχ ὁμολογοῦσι δ' 
ὅτι περιέπλευσάν τινες ἀπὸ τῆς Ἰνδικῆς ἐπὶ τὴν Ὑρ-
κανίαν· ὅτι δὲ δυνατόν, Πατροκλῆς εἴρηκε. 



Strabo Geogr., Geographica 
Book 11, chapter 11, section 7, line 2

Λέγεται δὲ διότι τοῦ Ταύρου τὸ τελευταῖον, ὃ   
καλοῦσιν Ἴμαιον, τῇ Ἰνδικῇ θαλάττῃ ξυνάπτον, οὐδὲν 
οὔτε προὔχει πρὸς ἕω τῆς Ἰνδικῆς μᾶλλον οὔτ' εἰσέχει· 
παριόντι δ' εἰς τὸ βόρειον πλευρὸν ἀεί τι τοῦ μή-
κους ὑφαιρεῖ καὶ τοῦ πλάτους ἡ θάλαττα, ὥστ' ἀπο-
φαίνειν μύουρον πρὸς ἕω τὴν νῦν ὑπογραφομένην 
μερίδα τῆς Ἀσίας, ἣν ὁ Ταῦρος ἀπολαμβάνει πρὸς τὸν 
ὠκεανὸν τὸν πληροῦντα τὸ Κάσπιον πέλαγος. 



Strabo Geogr., Geographica 
Book 11, chapter 11, section 7, line 15

εἴρηται γὰρ ὅτι περὶ τετρακισμυρίους σταδίους ἐστὶ 
τὸ ἀπὸ τοῦ Ἰσσικοῦ κόλπου μέχρι τῆς ἑῴας θαλάττης 
τῆς κατὰ Ἰνδούς, ἐπὶ δ' Ἰσσὸν ἀπὸ τῶν ἑσπερίων ἄκρων 
τῶν κατὰ στήλας ἄλλοι τρισμύριοι· ἔστι δὲ ὁ μυχὸς τοῦ 
Ἰσσικοῦ κόλπου μικρὸν ἢ οὐδὲν Ἀμισοῦ ἑωθινώτερος, 
τὸ δὲ ἀπὸ Ἀμισοῦ ἐπὶ τὴν Ὑρκανίαν γῆν περὶ μυρίους 
ἐστὶ σταδίους, παράλληλον ὂν τῷ ἀπὸ τοῦ Ἰσσοῦ λε-
χθέντι ἐπὶ τοὺς Ἰνδούς. 



Strabo Geogr., Geographica 
Book 14, chapter 2, section 29, line 36

                                                       τὰ 
δ' ἐπ' εὐθείας τούτοις μέχρι τῆς Ἰνδικῆς τὰ αὐτὰ κεῖται 
καὶ παρὰ τῷ Ἀρτεμιδώρῳ ἅπερ καὶ παρὰ τῷ Ἐρατο-
σθένει. 



Strabo Geogr., Geographica 
Book 14, chapter 5, section 11, line 22

                                 διὰ δὲ τοῦτ' αὐτὸ καὶ 
τὴν ἐκ τῆς Ῥοδίας γραμμήν, ἣν μέχρι τοῦ Κύδνου κα-
τηγάγομεν, τὴν αὐτὴν ἀποφαίνομεν τῇ μέχρι Ἰσσοῦ,   
οὐδὲν παρὰ τοῦτο ποιούμενοι, καὶ τὸν Ταῦρόν φαμεν 
διήκειν ἐπ' εὐθείας τῇδε τῇ γραμμῇ μέχρι τῆς Ἰνδικῆς. 



Strabo Geogr., Geographica 
Book 15, chapter 1, section 1, line 3

Τὰ περιλειπόμενα τῆς Ἀσίας ἐστὶ τὰ ἐκτὸς τοῦ 
Ταύρου, πλὴν Κιλικίας καὶ Παμφυλίας καὶ Λυκίας, τὰ 
ἀπὸ τῆς Ἰνδικῆς μέχρι Νείλου μεταξὺ τοῦ Ταύρου καὶ 
τῆς ἔξω θαλάττης τῆς νοτίου κείμενα. 



Strabo Geogr., Geographica 
Book 15, chapter 1, section 1, line 6

                                            μετὰ δὲ τὴν 
Ἀσίαν ἡ Λιβύη ἐστί, περὶ ἧς ἐροῦμεν ὕστερον, νῦν δ' 
ἀπὸ τῆς Ἰνδικῆς ἀρκτέον· πρώτη γὰρ ἔκκειται πρὸς 
ταῖς ἀνατολαῖς καὶ μεγίστη. 



Strabo Geogr., Geographica 
Book 15, chapter 1, section 3, line 7

                              Ἀπολλόδωρος γοῦν ὁ τὰ 
Παρθικὰ ποιήσας, μεμνημένος καὶ τῶν τὴν Βακτρια-
νὴν ἀποστησάντων Ἑλλήνων παρὰ τῶν Συριακῶν βα-
σιλέων τῶν ἀπὸ Σελεύκου τοῦ Νικάτορος, φησὶ μὲν 
αὐτοὺς αὐξηθέντας ἐπιθέσθαι καὶ τῇ Ἰνδικῇ· οὐδὲν 
δὲ προσανακαλύπτει * τῶν πρότερον ἐγνωσμένων, ἀλλὰ   
καὶ ἐναντιολογεῖ πλείω τῆς Ἰνδικῆς ἐκείνους ἢ Μακεδό-
νας καταστρέψασθαι λέγων· Εὐκρατίδαν γοῦν πόλεις 
χιλίας ὑφ' ἑαυτῷ ἔχειν· ἐκεῖνοι δέ γε αὐτὰ τὰ μεταξὺ 
ἔθνη τοῦ τε Ὑδάσπου καὶ τοῦ Ὑπάνιος τὸν ἀριθμὸν 
ἐννέα, πόλεις τε σχεῖν πεντακισχιλίας, ὧν μηδεμίαν 
εἶναι Κῶ τῆς Μεροπίδος ἐλάττω· ταύτην δὲ πᾶσαν 
τὴν χώραν καταστρεψάμενον Ἀλέξανδρον παραδοῦναι 
Πώρῳ. 



Strabo Geogr., Geographica 
Book 15, chapter 1, section 4, line 2

Καὶ οἱ νῦν δὲ ἐξ Αἰγύπτου πλέοντες ἐμπορικοὶ 
τῷ Νείλῳ καὶ τῷ Ἀραβίῳ κόλπῳ μέχρι τῆς Ἰνδικῆς 
σπάνιοι μὲν καὶ περιπεπλεύκασι μέχρι τοῦ Γάγγου, 
καὶ οὗτοι δ' ἰδιῶται καὶ οὐδὲν πρὸς ἱστορίαν τῶν τό-
πων χρήσιμοι. 



Strabo Geogr., Geographica 
Book 15, chapter 1, section 4, line 8

               κἀκεῖθεν δὲ ἀφ' ἑνὸς τόπου καὶ παρ' 
ἑνὸς βασιλέως, Πανδίονος κατ' ἄλλους Πώρου, ἧκεν 
ὡς Καίσαρα τὸν Σεβαστὸν δῶρα καὶ πρεσβεῖα καὶ ὁ 
κατακαύσας ἑαυτὸν Ἀθήνησι σοφιστὴς Ἰνδός, καθάπερ 
καὶ ὁ Κάλανος Ἀλεξάνδρῳ τὴν τοιαύτην θέαν ἐπι-
δειξάμενος. 



Strabo Geogr., Geographica 
Book 15, chapter 1, section 5, line 7

                                                         φησὶ 
γοῦν Νέαρχος φιλονεικῆσαι αὐτὸν διὰ τῆς Γεδρωσίας 
ἀγαγεῖν τὴν στρατιάν, πεπυσμένον διότι καὶ Σεμίραμις 
ἐστράτευσεν ἐπὶ Ἰνδοὺς καὶ Κῦρος· ἀλλ' ἡ μὲν ἀνέ-
στρεψε φεύγουσα μετὰ εἴκοσιν ἀνθρώπων, ἐκεῖνος δὲ 
μεθ' ἑπτά· ὡς σεμνὸν τὸ ἐκείνων τοσαῦτα παθόντων 
αὐτὸν καὶ στρατόπεδον διασῶσαι μετὰ νίκης διὰ τῶν 
αὐτῶν ἐθνῶν τε καὶ τόπων. 



Strabo Geogr., Geographica 
Book 15, chapter 1, section 6, line 2

Ἐκεῖνος μὲν δὴ ἐπίστευσεν· ἡμῖν δὲ τίς ἂν δικαία 
γένοιτο πίστις περὶ τῶν Ἰνδικῶν ἐκ τῆς τοιαύτης στρα-
τείας τοῦ Κύρου ἢ τῆς Σεμιράμιδος; 



Strabo Geogr., Geographica 
Book 15, chapter 1, section 6, line 5

                                            συναποφαίνεται   
δέ πως καὶ Μεγασθένης τῷ λόγῳ τούτῳ κελεύων ἀπι-
στεῖν ταῖς ἀρχαίαις περὶ Ἰνδῶν ἱστορίαις· οὔτε γὰρ 
παρ' Ἰνδῶν ἔξω σταλῆναί ποτε στρατιάν, οὔτ' ἐπελθεῖν 
ἔξωθεν καὶ κρατῆσαι πλὴν τῆς μεθ' Ἡρακλέους καὶ 
Διονύσου καὶ τῆς νῦν μετὰ Μακεδόνων. 



Strabo Geogr., Geographica 
Book 15, chapter 1, section 6, line 16

                   μέχρι μὲν δὴ δεῦρο καὶ Τεάρκωνα 
ἀφικέσθαι, ἐκεῖνον δὲ καὶ ἐκ τῆς Ἰβηρίας εἰς τὴν Θρᾴ-
κην καὶ τὸν Πόντον ἀγαγεῖν τὴν στρατιάν· Ἰδάνθυρ-
σον δὲ τὸν Σκύθην ἐπιδραμεῖν τῆς Ἀσίας μέχρι Αἰγύ-
πτου· τῆς δὲ Ἰνδικῆς μηδένα τούτων ἅψασθαι· καὶ 
Σεμίραμιν δ' ἀποθανεῖν πρὸ τῆς ἐπιχειρήσεως· Πέρ-
σας δὲ μισθοφόρους μὲν ἐκ τῆς Ἰνδικῆς μεταπέμψασθαι 
Ὑδράκας, ἐκεῖ δὲ μὴ στρατεῦσαι, ἀλλ' ἐγγὺς ἐλθεῖν 
μόνον ἡνίκα Κῦρος ἤλαυνεν ἐπὶ Μασσαγέτας. 



Strabo Geogr., Geographica 
Book 15, chapter 1, section 8, line 11

προσωνόμασαν καὶ πόλιν παρ' αὐτοῖς Νῦσαν Διονύ-
σου κτίσμα, καὶ ὄρος τὸ ὑπὲρ τῆς πόλεως Μηρόν, αἰ-
τιασάμενοι καὶ τὸν αὐτόθι κισσὸν καὶ ἄμπελον, οὐδὲ 
ταύτην τελεσίκαρπον· ἀπορρεῖ γὰρ ὁ βότρυς πρὶν 
περκάσαι διὰ τοὺς ὄμβρους τοὺς ἄδην· Διονύσου δ' 
ἀπογόνους τοὺς Συδράκας ἀπὸ τῆς ἀμπέλου τῆς παρ' 
αὐτοῖς καὶ τῶν πολυτελῶν ἐξόδων, βακχικῶς τάς τε 
ἐκστρατείας ποιουμένων τῶν βασιλέων καὶ τὰς ἄλλας 
ἐξόδους μετὰ τυμπανισμοῦ καὶ εὐανθοῦς στολῆς· ὅπερ 
ἐπιπολάζει καὶ παρὰ τοῖς ἄλλοις Ἰνδοῖς. 



Strabo Geogr., Geographica 
Book 15, chapter 1, section 8, line 12

                                                  Ἄορνον δέ 
τινα πέτραν, ἧς τὰς ῥίζας ὁ Ἰνδὸς ὑπορρεῖ πλησίον τῶν 
πηγῶν, Ἀλεξάνδρου κατὰ μίαν προσβολὴν ἑλόντος, 
σεμνύνοντες ἔφασαν τὸν Ἡρακλέα τρὶς μὲν προσβα-
λεῖν τῇ πέτρᾳ ταύτῃ τρὶς δ' ἀποκρουσθῆναι. 



Strabo Geogr., Geographica 
Book 15, chapter 1, section 9, line 8

Ὅτι δ' ἐστὶ πλάσματα ταῦτα τῶν κολακευόντων 
Ἀλέξανδρον πρῶτον μὲν ἐκ τοῦ μὴ ὁμολογεῖν ἀλλή-
λοις τοὺς συγγραφέας δῆλον, ἀλλὰ τοὺς μὲν λέγειν 
τοὺς δὲ μηδ' ἁπλῶς μεμνῆσθαι· οὐ γὰρ εἰκὸς τὰ οὕ-
τως ἔνδοξα καὶ τύφου πλήρη μὴ πεπύσθαι, ἢ πεπύ-
σθαι μὲν μὴ ἄξια δὲ μνήμης ὑπολαβεῖν, καὶ ταῦτα 
τοὺς πιστοτάτους αὐτῶν· ἔπειτα ἐκ τοῦ μηδὲ τοὺς με-
ταξύ, δι' ὧν ἐχρῆν τὴν ἐς Ἰνδοὺς ἄφιξιν γενέσθαι 
τοῖς περὶ τὸν Διόνυσον καὶ τὸν Ἡρακλέα, μηδὲν ἔχειν 
τεκμήριον δεικνύναι τῆς ἐκείνων ὁδοῦ διὰ τῆς σφετέ-
ρας γῆς. 



Strabo Geogr., Geographica 
Book 15, chapter 1, section 10, line 9

                       μάλιστα δ' ἐκ τῆς διαίτης ἐδόκει 
τῆς τότε πιστότατα εἶναι τὰ ὑπὸ τοῦ Ἐρατοσθένους ἐν 
τῷ τρίτῳ τῶν γεωγραφικῶν ἐκτεθέντα κεφαλαιωδῶς 
περὶ τῆς τότε νομιζομένης Ἰνδικῆς, ἡνίκα Ἀλέξανδρος 
ἐπῆλθε· καὶ ἦν ὁ Ἰνδὸς ὅριον ταύτης τε καὶ τῆς Ἀρια-
νῆς ἣν ἐφεξῆς πρὸς τῇ ἑσπέρᾳ κειμένην Πέρσαι κα-
τεῖχον· ὕστερον γὰρ δὴ καὶ τῆς Ἀριανῆς πολλὴν ἔσχον 
οἱ Ἰνδοὶ λαβόντες παρὰ τῶν Μακεδόνων. 



Strabo Geogr., Geographica 
Book 15, chapter 1, section 11, line 1

Τὴν Ἰνδικὴν περιώρικεν ἀπὸ μὲν τῶν ἄρκτων 
τοῦ Ταύρου τὰ ἔσχατα ἀπὸ τῆς Ἀριανῆς μέχρι τῆς 
ἑῴας θαλάττης, ἅπερ οἱ ἐπιχώριοι κατὰ μέρος Παρο-
πάμισόν τε καὶ Ἠμωδὸν καὶ Ἴμαον καὶ ἄλλα ὀνομάζου-  
σι, Μακεδόνες δὲ Καύκασον· ἀπὸ δὲ τῆς ἑσπέρας ὁ 
Ἰνδὸς ποταμός· τὸ δὲ νότιον καὶ τὸ προσεῷον πλευ-
ρόν, πολὺ μείζω τῶν ἑτέρων ὄντα, προπέπτωκεν εἰς 
τὸ Ἀτλαντικὸν πέλαγος, καὶ γίνεται ῥομβοειδὲς τὸ τῆς 
χώρας σχῆμα τῶν μειζόνων πλευρῶν ἑκατέρου πλεο-
νεκτοῦντος παρὰ τὸ ἀπεναντίον πλευρὸν καὶ

τρισχι-



Strabo Geogr., Geographica 
Book 15, chapter 1, section 11, line 16

                                                       τῆς μὲν 
οὖν ἑσπερίου πλευρᾶς ἀπὸ τῶν Καυκασίων ὀρῶν ἐπὶ 
τὴν νότιον θάλατταν στάδιοι μάλιστα λέγονται μύριοι 
τρισχίλιοι παρὰ τὸν Ἰνδὸν ποταμὸν μέχρι τῶν ἐκβολῶν 
αὐτοῦ, ὥστ' ἀπεναντίον ἡ ἑωθινὴ προσλαβοῦσα τοὺς 
τῆς ἄκρας τρισχιλίους ἔσται μυρίων καὶ ἑξακισχιλίων 
σταδίων. 



Strabo Geogr., Geographica 
Book 15, chapter 1, section 11, line 34

τούτῳ δὴ πάλιν τῷ διαστήματι προστεθὲν τὸ τῆς ἄκρας 
διάστημα τὸ προπῖπτον ἐπὶ πλέον πρὸς τὰς ἀνατολάς, 
οἱ τρισχίλιοι στάδιοι ποιήσουσι τὸ μέγιστον μῆκος· 
ἔστι δὲ τοῦτο τὸ ἀπὸ τῶν ἐκβολῶν τοῦ Ἰνδοῦ ποταμοῦ   
παρὰ τὴν ἑξῆς ᾐόνα μέχρι τῆς λεχθείσης ἄκρας καὶ 
τῶν ἀνατολικῶν αὐτῆς τερμόνων· οἰκοῦσι δ' ἐνταῦθα 
οἱ Κωνιακοὶ καλούμενοι. 



Strabo Geogr., Geographica 
Book 15, chapter 1, section 12, line 3

Ἐκ δὲ τούτων πάρεστιν ὁρᾶν ὅσον διαφέρουσιν 
αἱ τῶν ἄλλων ἀποφάσεις, Κτησίου μὲν οὐκ ἐλάττω 
τῆς ἄλλης Ἀσίας τὴν Ἰνδικὴν λέγοντος, Ὀνησικρίτου 
δὲ τρίτον μέρος τῆς οἰκουμένης, Νεάρχου δὲ μηνῶν 
ὁδὸν τεττάρων τὴν διὰ τοῦ πεδίου, Μεγασθένους δὲ 
καὶ Δηιμάχου μετριασάντων μᾶλλον· ὑπὲρ γὰρ δισμυ-
ρίους τιθέασι σταδίους τὸ ἀπὸ τῆς νοτίου θαλάττης 
ἐπὶ τὸν Καύκασον, Δηίμαχος δ' ὑπὲρ τοὺς τρισμυρί-
ους κατ' ἐνίους τόπους· πρὸς οὓς ἐν τοῖς πρώτοις λό-
γοις εἴρηται, νῦν δὲ τοσοῦτον εἰπεῖν ἱκανόν, ὅτι καὶ 
ταῦτα συνηγορεῖ τοῖς αἰτουμένοις συγγνώμην, ἐάν τι 
περὶ τῶν Ἰνδικῶν λέγοντες μὴ διισχυρίζωνται. 



Strabo Geogr., Geographica 
Book 15, chapter 1, section 12, line 12

τῆς ἄλλης Ἀσίας τὴν Ἰνδικὴν λέγοντος, Ὀνησικρίτου 
δὲ τρίτον μέρος τῆς οἰκουμένης, Νεάρχου δὲ μηνῶν 
ὁδὸν τεττάρων τὴν διὰ τοῦ πεδίου, Μεγασθένους δὲ 
καὶ Δηιμάχου μετριασάντων μᾶλλον· ὑπὲρ γὰρ δισμυ-
ρίους τιθέασι σταδίους τὸ ἀπὸ τῆς νοτίου θαλάττης 
ἐπὶ τὸν Καύκασον, Δηίμαχος δ' ὑπὲρ τοὺς τρισμυρί-
ους κατ' ἐνίους τόπους· πρὸς οὓς ἐν τοῖς πρώτοις λό-
γοις εἴρηται, νῦν δὲ τοσοῦτον εἰπεῖν ἱκανόν, ὅτι καὶ 
ταῦτα συνηγορεῖ τοῖς αἰτουμένοις συγγνώμην, ἐάν τι 
περὶ τῶν Ἰνδικῶν λέγοντες μὴ διισχυρίζωνται. 



Strabo Geogr., Geographica 
Book 15, chapter 1, section 13, line 1

Ἅπασα δ' ἐστὶ κατάρρυτος ποταμοῖς ἡ Ἰνδική, 
τοῖς μὲν εἰς δύο τοὺς μεγίστους συρρηγνυμένοις τόν 
τε Ἰνδὸν καὶ τὸν Γάγγην, τοῖς δὲ κατ' ἴδια στόματα 
ἐκδιδοῦσιν εἰς τὴν θάλατταν· ἅπαντες δ' ἀπὸ τοῦ 
Καυκάσου τὴν ἀρχὴν ἔχουσι καὶ φέρονται μὲν ἐπὶ τὴν 
μεσημβρίαν τὸ πρῶτον, εἶθ' οἱ μὲν μένουσιν ἐπὶ τῆς αὐ-
τῆς φορᾶς καὶ μάλιστα οἱ εἰς τὸν Ἰνδὸν συμβάλλοντες, 
οἱ δ' ἐπιστρέφονται πρὸς ἕω, καθάπερ καὶ ὁ Γάγγης 
ποταμός. 



Strabo Geogr., Geographica 
Book 15, chapter 1, section 13, line 13

          οὗτος μὲν οὖν καταβὰς ἐκ τῆς ὀρεινῆς, ἐπει-
δὰν ἅψηται τῶν πεδίων ἐπιστρέψας πρὸς ἕω καὶ ῥυεὶς 
παρὰ τὰ Παλίβοθρα μεγίστην πόλιν πρόεισιν ἐπὶ τὴν 
ταύτῃ θάλατταν καὶ μίαν ἐκβολὴν ποιεῖται, μέγιστος 
ὢν τῶν κατὰ τὴν Ἰνδικὴν ποταμῶν· ὁ δὲ Ἰνδὸς δυσὶ 
στόμασιν εἰς τὴν μεσημβρινὴν ἐκπίπτει θάλατταν, ἐμ-
περιλαμβάνων τὴν Παταληνὴν καλουμένην χώραν 
παραπλησίαν τῷ κατ' Αἴγυπτον Δέλτα. 



Strabo Geogr., Geographica 
Book 15, chapter 1, section 13, line 19

                                            ἐκ δὲ τῆς ἀνα-
θυμιάσεως τῶν τοσούτων ποταμῶν καὶ ἐκ τῶν ἐτησί-  
ων, ὡς Ἐρατοσθένης φησί, βρέχεται τοῖς θερινοῖς ὄμ-
βροις ἡ Ἰνδική, καὶ λιμνάζει τὰ πεδία· ἐν μὲν οὖν τού-
τοις τοῖς ὄμβροις λίνον σπείρεται καὶ κέγχρος, πρὸς 
τούτοις σήσαμον ὄρυζα βόσμορον· τοῖς δὲ χειμερινοῖς 
καιροῖς πυροὶ κριθαὶ ὄσπρια καὶ ἄλλοι καρποὶ ἐδώδι-
μοι, ὧν ἡμεῖς ἄπειροι. 



Strabo Geogr., Geographica 
Book 15, chapter 1, section 13, line 24

                       .. τὰ αὐτὰ φύεται καὶ ἐν τῇ Ἰνδικῇ, 
καὶ τῶν ἐν τοῖς ποταμοῖς πλὴν ἵππου ποταμίου τὰ 
ἄλλα φέρουσι καὶ οἱ Ἰνδικοί· Ὀνησίκριτος δὲ καὶ τοὺς 
ἵππους γίνεσθαί φησι. 



Strabo Geogr., Geographica 
Book 15, chapter 1, section 14, line 2

Τὴν δὲ Ταπροβάνην πελαγίαν εἶναί φασι νῆσον 
ἀπέχουσαν τῶν νοτιωτάτων τῆς Ἰνδικῆς τῶν κατὰ τοὺς 
Κωνιακοὺς πρὸς μεσημβρίαν ἡμερῶν ἑπτὰ πλοῦν, μῆ-
κος μὲν ὡς πεντακισχιλίων σταδίων ἐπὶ τὴν Αἰθιοπίαν· 
ἔχειν δὲ καὶ ἐλέφαντας. 



Strabo Geogr., Geographica 
Book 15, chapter 1, section 15, line 10

                            εἶναι δὲ καὶ ἄλλας νήσους 
αὐτῆς μεταξὺ καὶ τῆς Ἰνδικῆς, νοτιωτάτην δ' ἐκεί-
νην. 



Strabo Geogr., Geographica 
Book 15, chapter 1, section 17, line 12

                                                κατανοη-
θῆναι δὲ ταῦτα καὶ ὑφ' ἑαυτοῦ καὶ ὑπὸ τῶν ἄλλων φη-
σίν, ὡρμηκότων μὲν εἰς τὴν Ἰνδικὴν ἀπὸ Παροπαμι-
σαδῶν, μετὰ δὲ δυσμὰς πληιάδων, καὶ διατριψάντων 
κατὰ τὴν ὀρεινὴν ἔν τε τῇ Ὑπασίων καὶ τῇ Ἀσσακα-
νοῦ γῇ τὸν χειμῶνα, τοῦ δ' ἔαρος ἀρχομένου καταβε-
βηκότων εἰς τὰ πεδία καὶ πόλιν Τάξιλα εὐμεγέθη, ἐν-
τεῦθεν δ' ἐπὶ Ὑδάσπην καὶ τὴν Πώρου χώραν· τοῦ 
μὲν οὖν χειμῶνος ὕδωρ οὐκ ἰδεῖν ἀλλὰ χιόνας μόνον· 
ἐν δὲ τοῖς Ταξίλοις πρῶτον ὑσθῆναι, καὶ ἐπειδὴ κατα-
βᾶσιν ἐπὶ τὸν Ὑδάσπην καὶ νικήσασι Πῶρον ὁδὸς ἦν   




Strabo Geogr., Geographica 
Book 15, chapter 1, section 19, line 4

Τὴν δ' ὁμοιότητα τῆς χώρας ταύτης πρός τε τὴν 
Αἴγυπτον καὶ τὴν Αἰθιοπίαν καὶ πάλιν τὴν ἐναντιό-
τητα παραθεὶς ὁ Ἀριστόβουλος, διότι τῷ Νείλῳ μὲν 
ἐκ τῶν νοτίων ὄμβρων ἐστὶν ἡ πλήρωσις τοῖς Ἰνδικοῖς 
δὲ ποταμοῖς ἀπὸ τῶν ἀρκτικῶν, ζητεῖ πῶς οἱ μεταξὺ 
τόποι οὐ κατομβροῦνται· οὔτε γὰρ ἡ Θηβαῒς μέχρι 
Συήνης καὶ τῶν ἐγγὺς Μερόης οὔτε τῆς Ἰνδικῆς τὰ 
ἀπὸ τῆς Παταληνῆς μέχρι τοῦ Ὑδάσπου· τὴν δ' ὑπὲρ 
ταῦτα τὰ μέρη χώραν ἐν ᾗ καὶ ὄμβροι καὶ νιφετοί, πα-
ραπλησίως ἔφη γεωργεῖσθαι τῇ ἄλλῃ τῇ ἔξω τῆς Ἰνδι-
κῆς χώρᾳ· ποτίζεσθαι γὰρ ἐκ τῶν ὄμβρων καὶ χιόνων. 



Strabo Geogr., Geographica 
Book 15, chapter 1, section 19, line 17

        πεμφθεὶς γοῦν ἐπί τινα χρείαν ἰδεῖν φησιν 
ἐρημωθεῖσαν χώραν πλειόνων ἢ χιλίων πόλεων σὺν 
κώμαις, ἐκλιπόντος τοῦ Ἰνδοῦ τὸ οἰκεῖον ῥεῖθρον ἐκ-
τραπομένου δ' εἰς τὸ ἕτερον ἐν ἀριστερᾷ κοιλότερον   
πολύ, καὶ οἷον καταρράξαντος, ὡς τὴν ἀπολειφθεῖσαν 
ἐν δεξιᾷ χώραν μηκέτι ποτίζεσθαι ταῖς ὑπερχύσεσι, 
μετεωροτέραν οὖσαν οὐ τοῦ ῥείθρου τοῦ καινοῦ μό-
νον ἀλλὰ καὶ τῶν ὑπερχύσεων. 



Strabo Geogr., Geographica 
Book 15, chapter 1, section 20, line 7

                              Μεγασθένης δὲ τὴν εὐδαι-
μονίαν τῆς Ἰνδικῆς ἐπισημαίνεται τῷ δίκαρπον εἶναι 
καὶ δίφορον· καθάπερ καὶ Ἐρατοσθένης ἔφη, τὸν μὲν 
εἰπὼν σπόρον χειμερινὸν τὸν δὲ θερινόν, καὶ ὄμβρον 
ὁμοίως· οὐδὲν γὰρ ἔτος εὑρίσκεσθαί φησι πρὸς ἀμφο-
τέρους καιροὺς ἄνομβρον· ὥστ' εὐετηρίαν ἐκ τούτου 
συμβαίνειν ἀφόρου μηδέποτε τῆς γῆς οὔσης· τούς 
τε ξυλίνους καρποὺς γεννᾶσθαι πολλοὺς καὶ τὰς ῥίζας 
τῶν φυτῶν καὶ μάλιστα τῶν μεγάλων καλάμων, γλυ-
κείας καὶ φύσει καὶ ἑψήσει χλιαινομένου τοῦ ὕδατος 
τοῖς ἡλίοις τοῦ τ' ἐκπίπτοντος ἐκ Διὸς καὶ τοῦ

ποτα-



Strabo Geogr., Geographica 
Book 15, chapter 1, section 21, line 1

Πολλὰ γὰρ δὴ δένδρα παράδοξα ἡ Ἰνδικὴ τρέφει, 
ὧν ἐστι καὶ τὸ κάτω νεύοντας ἔχον τοὺς κλάδους τὰ δὲ 
φύλλα ἀσπίδος οὐκ ἐλάττω. 



Strabo Geogr., Geographica 
Book 15, chapter 1, section 21, line 5

                                Ὀνησίκριτος δὲ καὶ περι-
εργότερον τὰ ἐν τῇ Μουσικανοῦ διεξιών, ἅ φησι νο-
τιώτατα εἶναι τῆς Ἰνδικῆς, διηγεῖται μεγάλα δένδρα 
τινά, ὧν τοὺς κλάδους αὐξηθέντας ἐπὶ πήχεις καὶ δώ-
δεκα, ἔπειτα τὴν λοιπὴν αὔξησιν καταφερῆ λαμβάνειν 
ὡς ἂν κατακαμπτομένους, ἕως ἂν ἅψωνται τῆς γῆς· 
ἔπειτα κατὰ γῆς διαδοθέντας ῥιζοῦσθαι ὁμοίως ταῖς 
κατώρυξιν, εἶτ' ἀναδοθέντας στελεχοῦσθαι· ἐξ οὗ πά-
λιν ὁμοίως τῇ αὐξήσει κατακαμφθέντας ἄλλην κατώ-
ρυγα ποιεῖν, εἶτ' ἄλλην, καὶ οὕτως ἐφεξῆς, ὥστ' ἀφ' 
ἑνὸς δένδρου σκιάδιον γίνεσθαι μακρὸν πολυστύλῳ 
σκηνῇ ὅμοιον. 



Strabo Geogr., Geographica 
Book 15, chapter 1, section 22, line 3

Ἐν δὲ τῇ Μουσικανοῦ καὶ σῖτον αὐτοφυῆ λέγει 
πυρῷ παραπλήσιον καὶ ἄμπελον, ὥστ' οἰνοφορεῖν τῶν 
ἄλλων ἄοινον λεγόντων τὴν Ἰνδικήν· ὥστε μηδ' αὐ-
λὸν εἶναι κατὰ τὸν Ἀνάχαρσιν μηδ' ἄλλο τῶν μουσι-
κῶν ὀργάνων μηδὲν πλὴν κυμβάλων καὶ τυμπάνων 
καὶ κροτάλων ἃ τοὺς θαυματοποιοὺς κεκτῆσθαι. 



Strabo Geogr., Geographica 
Book 15, chapter 1, section 22, line 14

                                               ἔχειν δὲ καὶ 
κιννάμωμον καὶ νάρδον καὶ τὰ ἄλλα ἀρώματα τὴν νό-
τιον γῆν τὴν Ἰνδικὴν ὁμοίως ὥσπερ τὴν Ἀραβίαν καὶ 
τὴν Αἰθιοπίαν ἔχουσάν τι ἐμφερὰς ἐκείναις κατὰ τοὺς 
ἡλίους· διαφέρειν δὲ τῷ πλεονασμῷ τῶν ὑδάτων ὥστ' 
ἔνικμον εἶναι τὸν ἀέρα καὶ τροφιμώτερον παρὰ τοῦτο 
καὶ γόνιμον μᾶλλον, ὡς δ' αὕτως καὶ τὴν γῆν καὶ τὸ 
ὕδωρ, ᾗ δὴ καὶ μείζω τά τε χερσαῖα τῶν ζῴων καὶ τὰ 
καθ' ὕδατος τὰ ἐν Ἰνδοῖς τῶν παρ' ἄλλοις εὑρίσκεσθαι. 



Strabo Geogr., Geographica 
Book 15, chapter 1, section 23, line 6

                                ὅσῳ δέ γε φησὶ τὸ μὲν τοῦ 
Νείλου ὕδωρ δι' εὐθείας ἔπεισι πολλὴν χώραν καὶ στε-
νὴν καὶ μεταβάλλει πολλὰ κλίματα καὶ πολλοὺς ἀέρας, 
τὰ δ' Ἰνδικὰ ῥεύματα ἐς πεδία ἀναχεῖται μείζω καὶ 
πλατύτερα ἐνδιατρίβοντα πολὺν χρόνον τοῖς αὐτοῖς 
κλίμασι, τοσῷδε ἐκεῖνα τούτου τροφιμώτερα, διότι 
καὶ τὰ κήτη μείζω τε καὶ πλείω· καὶ ἐκ τῶν νεφῶν δὲ 
ἑφθὸν ἤδη χεῖσθαι τὸ ὕδωρ. 



Strabo Geogr., Geographica 
Book 15, chapter 1, section 24, line 22

                                      βελτίους δὲ οἱ τὸν ἥλιον 
αἰτιώμενοι καὶ τὴν ἐξ αὐτοῦ ἐπίκαυσιν κατ' ἐπίλειψιν 
σφοδρὰν τῆς ἐπιπολῆς ἰκμάδος· καθ' ὃ καὶ τοὺς Ἰν-
δοὺς μὴ οὐλοτριχεῖν φαμεν, μηδ' οὕτως ἀπεφεισμέ-
νως ἐπικεκαῦσθαι τὴν χρόαν, ὅτι ὑγροῦ κοινωνοῦσιν 
ἀέρος. 



Strabo Geogr., Geographica 
Book 15, chapter 1, section 25, line 6

                            Νέαρχος δὲ τὸ ζητούμενον 
πρότερον ἐπὶ τοῦ Νείλου πόθεν ἡ πλήρωσις αὐτοῦ, δι-
δάσκειν ἔφη τοὺς Ἰνδικοὺς ποταμοὺς ὅτι ἐκ τῶν θερι-
νῶν ὄμβρων συμβαίνει· Ἀλέξανδρον δ' ἐν μὲν τῷ 
Ὑδάσπῃ κροκοδείλους ἰδόντα, ἐν δὲ τῷ Ἀκεσίνῃ κυά-
μους Αἰγυπτίους, εὑρηκέναι δόξαι τὰς τοῦ Νείλου πη-
γάς, καὶ παρασκευάζεσθαι στόλον εἰς τὴν Αἴγυπτον 
ὡς τῷ ποταμῷ τούτῳ μέχρι ἐκεῖσε πλευσόμενον· μι-
κρὸν δ' ὕστερον γνῶναι διότι οὐ δύναται ὃ ἤλπισε· 
“μέσσῳ γὰρ μεγάλοι ποταμοὶ καὶ δεινὰ ῥέεθρα, Ὠκεα-
“νὸς μὲν πρῶτον,” εἰς ὃν ἐκδιδόασιν οἱ Ἰνδικοὶ πάντες 
ποταμοί, ἔπειτα ἡ Ἀριανὴ καὶ ὁ Περσικὸς κόλπος καὶ 




Strabo Geogr., Geographica 
Book 15, chapter 1, section 25, line 14

πρότερον ἐπὶ τοῦ Νείλου πόθεν ἡ πλήρωσις αὐτοῦ, δι-
δάσκειν ἔφη τοὺς Ἰνδικοὺς ποταμοὺς ὅτι ἐκ τῶν θερι-
νῶν ὄμβρων συμβαίνει· Ἀλέξανδρον δ' ἐν μὲν τῷ 
Ὑδάσπῃ κροκοδείλους ἰδόντα, ἐν δὲ τῷ Ἀκεσίνῃ κυά-
μους Αἰγυπτίους, εὑρηκέναι δόξαι τὰς τοῦ Νείλου πη-
γάς, καὶ παρασκευάζεσθαι στόλον εἰς τὴν Αἴγυπτον 
ὡς τῷ ποταμῷ τούτῳ μέχρι ἐκεῖσε πλευσόμενον· μι-
κρὸν δ' ὕστερον γνῶναι διότι οὐ δύναται ὃ ἤλπισε· 
“μέσσῳ γὰρ μεγάλοι ποταμοὶ καὶ δεινὰ ῥέεθρα, Ὠκεα-
“νὸς μὲν πρῶτον,” εἰς ὃν ἐκδιδόασιν οἱ Ἰνδικοὶ πάντες 
ποταμοί, ἔπειτα ἡ Ἀριανὴ καὶ ὁ Περσικὸς κόλπος καὶ 
ὁ Ἀράβιος καὶ αὐτὴ ἡ Ἀραβία καὶ ἡ Τρωγλοδυτική. 



Strabo Geogr., Geographica 
Book 15, chapter 1, section 26, line 6

                 ὁ δὲ Νεῖλος καὶ οἱ κατὰ τὴν Ἰνδικὴν 
πλεονέκτημά τι ἔχουσι παρὰ τοὺς ἄλλους διὰ τὸ τὴν 
χώραν ἀοίκητον εἶναι χωρὶς αὐτῶν, πλωτὴν ἅμα καὶ 
γεωργήσιμον οὖσαν, καὶ μήτ' ἐφοδεύεσθαι δυναμένην 
ἄλλως μήτ' οἰκεῖσθαι τὸ παράπαν. 



Strabo Geogr., Geographica 
Book 15, chapter 1, section 26, line 11

                                        τοὺς μὲν οὖν εἰς 
τὸν Ἰνδὸν καταφερομένους ἱστοροῦμεν τοὺς ἀξίους 
μνήμης καὶ τὰς χώρας, δι' ὧν ἡ φορά, τῶν δ' ἄλλων 
ἐστὶν ἄγνοια πλείων ἢ γνῶσις. 



Strabo Geogr., Geographica 
Book 15, chapter 1, section 26, line 17

                       ἧκε μὲν οὖν τῆς Ἰνδικῆς πλησίον 
δι' Ἀριανῶν, ἀφεὶς δ' αὐτὴν ἐν δεξιᾷ ὑπερέβη τὸν 
Παροπάμισον εἰς τὰ προσάρκτια μέρη καὶ τὴν Βακτρι-
ανήν· καταστρεψάμενος δὲ τἀκεῖ πάντα ὅσα ἦν ὑπὸ 
Πέρσαις καὶ ἔτι πλείω, τότ' ἤδη καὶ τῆς Ἰνδικῆς ὠρέ-
χθη, λεγόντων μὲν περὶ αὐτῆς πολλῶν οὐ σαφῶς δέ. 



Strabo Geogr., Geographica 
Book 15, chapter 1, section 26, line 24

ἀνέστρεψε δ' οὖν ὑπερθεὶς τὰ αὐτὰ ὄρη κατ' ἄλλας 
ὁδοὺς ἐπιτομωτέρας ἐν ἀριστερᾷ ἔχων τὴν Ἰνδικήν, 
εἶτ' ἐπέστρεψεν εὐθὺς ἐπ' αὐτὴν καὶ τοὺς ὅρους τοὺς 
ἑσπερίους αὐτῆς καὶ τὸν Κώφην ποταμὸν καὶ τὸν Χο-
άσπην, ὃς εἰς τὸν Κώφην ἐμβάλλει ποταμὸν κατὰ 
Πλημύριον πόλιν, ῥυεὶς παρὰ * Γώρυδι ἄλλην πόλιν, 
καὶ διεξιὼν τήν τε Βανδοβηνὴν καὶ τὴν Γανδαρῖτιν. 



Strabo Geogr., Geographica 
Book 15, chapter 1, section 27, line 1

Ἦν δὲ μετὰ τὸν Κώφην ὁ Ἰνδός, εἶθ' ὁ Ὑδάσπης, 
εἶθ' ὁ Ἀκεσίνης καὶ ὁ Ὑάρωτις, ὕστατος δ' ὁ Ὕπανις. 



Strabo Geogr., Geographica 
Book 15, chapter 1, section 27, line 8


ταῦτ' οὖν ἐγένετο γνώριμα ἡμῖν τῶν ἑωθινῶν τῆς Ἰν-
δικῆς μερῶν, ὅσα ἐντὸς τοῦ Ὑπάνιος, καὶ εἴ τινα προς-
ιστόρησαν οἱ μετ' ἐκεῖνον περαιτέρω τοῦ Ὑπάνιος 
προελθόντες μέχρι τοῦ Γάγγου καὶ Παλιβόθρων. 



Strabo Geogr., Geographica 
Book 15, chapter 1, section 27, line 11

                                                     μετὰ 
μὲν οὖν τὸν Κώφην ὁ Ἰνδὸς ῥεῖ· τὰ δὲ μεταξὺ τούτων 
τῶν δυεῖν ποταμῶν ἔχουσιν Ἀστακηνοί τε καὶ Μασια-
νοὶ καὶ Νυσαῖοι καὶ Ὑπάσιοι· εἶθ' ἡ Ἀσσακανοῦ, ὅπου 
Μασόγα πόλις, τὸ βασίλειον τῆς χώρας. 



Strabo Geogr., Geographica 
Book 15, chapter 1, section 27, line 15

                                             ἤδη δὲ πρὸς 
τῷ Ἰνδῷ πάλιν ἄλλη πόλις Πευκολαῖτις, πρὸς ᾗ 
ζεῦγμα γενηθὲν ἐπεραίωσε τὴν στρατιάν. 



Strabo Geogr., Geographica 
Book 15, chapter 1, section 28, line 1

Μεταξὺ δὲ τοῦ Ἰνδοῦ καὶ τοῦ Ὑδάσπου Τάξιλα 
ἔστι πόλις μεγάλη καὶ εὐνομωτάτη, καὶ ἡ περικειμένη 
χώρα συχνὴ καὶ σφόδρα εὐδαίμων, ἤδη συνάπτουσα 
καὶ τοῖς πεδίοις. 



Strabo Geogr., Geographica 
Book 15, chapter 1, section 28, line 9

                     ἐδέξαντό τε δὴ φιλανθρώπως τὸν 
Ἀλέξανδρον οἱ ἄνθρωποι καὶ ὁ βασιλεὺς αὐτῶν Ταξί-
λης· ἔτυχόν τε πλειόνων ἢ αὐτοὶ παρέσχον, ὥστε φθο-
νεῖν τοὺς Μακεδόνας καὶ λέγειν ὡς οὐκ εἶχεν, ὡς ἔοι-  
κεν, Ἀλέξανδρος οὓς εὐεργετήσει πρὶν ἢ διέβη τὸν Ἰν-
δόν. 



Strabo Geogr., Geographica 
Book 15, chapter 1, section 30, line 16

τατον ἱστορεῖται τὸ περὶ τοῦ κάλλους ὅτι τιμᾶται δια-
φερόντως, ὡς ἵππων καὶ κυνῶν· βασιλέα τε γὰρ τὸν 
κάλλιστον αἱρεῖσθαί φησιν Ὀνησίκριτος, γενόμενόν 
τε παιδίον μετὰ δίμηνον κρίνεσθαι δημοσίᾳ πότερον 
ἔχοι τὴν ἔννομον μορφὴν καὶ τοῦ ζῆν ἀξίαν ἢ οὔ, κρι-
θέντα δ' ὑπὸ τοῦ ἀποδειχθέντος ἄρχοντος ζῆν ἢ θα-
νατοῦσθαι· βάπτεσθαί τε πολλοῖς εὐανθεστάτοις χρώ-
μασι τοὺς πώγωνας αὐτοῦ τούτου χάριν καλλωπιζο-
μένους· τοῦτο δὲ καὶ ἄλλους ποιεῖν ἐπιμελῶς συχνοὺς 
τῶν Ἰνδῶν (καὶ γὰρ δὴ φέρειν τὴν χώραν χρόας θαυ-  
μαστάς) καὶ θριξὶ καὶ ἐσθῆσι· τοὺς δ' ἀνθρώπους τὰ 
ἄλλα μὲν εὐτελεῖς εἶναι φιλοκόσμους δέ. 



Strabo Geogr., Geographica 
Book 15, chapter 1, section 30, line 27

                                  φασὶ δ' ἐν τῇ Σωπείθους 
χώρᾳ ὀρυκτῶν ἁλῶν ὄρος εἶναι ἀρκεῖν δυνάμενον ὅλῃ 
τῇ Ἰνδικῇ· καὶ χρυσεῖα δὲ καὶ ἀργυρεῖα οὐ πολὺ ἄπω-
θεν ἐν ἄλλοις ὄρεσιν ἱστορεῖται καλά, ὡς ἐδήλωσε 
Γόργος ὁ μεταλλευτής. 



Strabo Geogr., Geographica 
Book 15, chapter 1, section 30, line 29

                          οἱ δ' Ἰνδοὶ μεταλλείας καὶ χω-
νείας ἀπείρως ἔχοντες οὐδ' ὧν εὐποροῦσιν ἴσασιν, ἀλλ' 
ἁπλούστερον μεταχειρίζονται τὸ πρᾶγμα. 



Strabo Geogr., Geographica 
Book 15, chapter 1, section 32, line 8

                                                           πάντες 
δ' οἱ λεχθέντες ποταμοὶ συμβάλλουσιν εἰς ἕνα τὸν Ἰν-
δόν, ὕστατος δ' ὁ Ὕπανις· πεντεκαίδεκα δὲ τοὺς σύμ-
παντας συρρεῖν φασι τούς γε ἀξιολόγους· πληρωθεὶς 
δ' ἐκ πάντων ὥστε καὶ ἐφ' ἑκατὸν σταδίους, ὡς οἱ μὴ 
μετριάζοντές φασιν, εὐρύνεσθαι κατά τινας τόπους, 
ὡς δ' οἱ μετριώτεροι, πεντήκοντα τὸ πλεῖστον ἐλάχι-
στον δὲ ἑπτά, ἔπειτα δυσὶ στόμασιν εἰς τὴν νοτίαν ἐκ-
δίδωσι θάλατταν καὶ τὴν Παταληνὴν προσαγορευομέ-
νην ποιεῖ νῆσον. 



Strabo Geogr., Geographica 
Book 15, chapter 1, section 33, line 4

Ἡ μὲν οὖν μεταξὺ τοῦ Ὑπάνιος καὶ τοῦ Ὑδάσπου 
λέγεται ἐννέα ἔχειν ἔθνη, πόλεις δὲ εἰς πεντακισχιλίας 
οὐκ ἐλάττους Κῶ τῆς Μεροπίδος· δοκεῖ δὲ πρὸς ὑπερ-
βολὴν εἰρῆσθαι τὸ πλῆθος· ἡ δὲ μεταξὺ τοῦ Ἰνδοῦ καὶ 
τοῦ Ὑδάσπου εἴρηται σχεδόν τι ὑφ' ὧν οἰκεῖται τῶν 
ἀξίων μνήμης. 



Strabo Geogr., Geographica 
Book 15, chapter 1, section 33, line 14

                                  πρὸς αὐτῇ δ' ἤδη τῇ 
Παταληνῇ τήν τε τοῦ Μουσικανοῦ λέγουσι καὶ τὴν 
Σάβου, τὰ Σινδόμανα, καὶ ἔτι τὴν Πορτικανοῦ καὶ ἄλ-  
λων ὧν ἐκράτησεν ἁπάντων Ἀλέξανδρος, τὴν τοῦ Ἰνδοῦ 
παροικούντων ποταμίαν, ὑστάτης δὲ τῆς Παταληνῆς 
ἣν ὁ Ἰνδὸς ποιεῖ σχισθεὶς εἰς δύο προχοάς. 



Strabo Geogr., Geographica 
Book 15, chapter 1, section 34, line 8

        λέγει δὲ καὶ περὶ τῆς Μουσικανοῦ χώρας ἐπὶ 
πλέον ἐγκωμιάζων αὐτήν, ὧν τινα κοινὰ καὶ ἄλλοις 
Ἰνδοῖς ἱστόρηται, ὡς τὸ μακρόβιον ὥστε καὶ τριάκοντα 
ἐπὶ τοῖς ἑκατὸν προσλαμβάνειν (καὶ γὰρ τοὺς Σῆρας 
ἔτι τούτων μακροβιωτέρους τινές φασι) καὶ τὸ λιτόβιον 
καὶ τὸ ὑγιεινόν, καίπερ τῆς χώρας ἀφθονίαν ἁπάντων 
ἐχούσης. 



Strabo Geogr., Geographica 
Book 15, chapter 1, section 35, line 8

                 ὅτι μὲν γὰρ μέγιστος τῶν μνημονευομέ-
νων κατὰ τὰς τρεῖς ἠπείρους καὶ μετ' αὐτὸν ὁ Ἰνδός, 
τρίτος δὲ καὶ τέταρτος ὁ Ἴστρος καὶ ὁ Νεῖλος, ἱκανῶς 
συμφωνεῖται· τὰ καθ' ἕκαστα δ' ἄλλοι ἄλλως περὶ 
αὐτοῦ λέγουσιν, οἱ μὲν τριάκοντα σταδίων τοὐλάχιστον 
πλάτος οἱ δὲ καὶ τριῶν, Μεγασθένης δέ, ὅταν ᾖ μέτριος, 
καὶ εἰς ἑκατὸν εὐρύνεσθαι, βάθος δὲ εἴκοσιν ὀργυιῶν 
τοὐλάχιστον. 



Strabo Geogr., Geographica 
Book 15, chapter 1, section 39, line 1

Φησὶ δὴ τὸ τῶν Ἰνδῶν πλῆθος εἰς ἑπτὰ μέρη διῃ-
ρῆσθαι, καὶ πρώτους μὲν τοὺς φιλοσόφους εἶναι κατὰ 
τιμήν, ἐλαχίστους δὲ κατ' ἀριθμόν· χρῆσθαι δ' αὐτοῖς 
ἰδίᾳ μὲν ἑκάστῳ τοὺς θύοντας ἢ τοὺς ἐναγίζοντας, 
κοινῇ δὲ τοὺς βασιλέας κατὰ τὴν μεγάλην λεγομένην 
σύνοδον, καθ' ἣν τοῦ νέου ἔτους ἅπαντες οἱ φιλόσοφοι 
τῷ βασιλεῖ συνελθόντες ἐπὶ θύρας ὅ τι ἂν αὐτῶν ἕκα-
στος συντάξῃ τῶν χρησίμων ἢ τηρήσῃ πρὸς εὐετηρίαν 
καρπῶν τε καὶ ζῴων καὶ περὶ πολιτείας, προσφέρει 
τοῦτ' εἰς τὸ μέσον· ὃς δ' ἂν τρὶς ἐψευσμένος ἁλῷ,

νό-



Strabo Geogr., Geographica 
Book 15, chapter 1, section 44, line 4

                                   Μεγασθένης δὲ περὶ 
τῶν μυρμήκων οὕτω φησὶν ὅτι ἐν Δέρδαις, ἔθνει με-
γάλῳ τῶν προσεῴων καὶ ὀρεινῶν Ἰνδῶν, ὀροπέδιον 
εἴη τρισχιλίων πως τὸν κύκλον σταδίων· ὑποκειμένων 
δὲ τούτῳ χρυσωρυχείων οἱ μεταλλεύοντες εἶεν μύρμη-
κες, θηρία ἀλωπέκων οὐκ ἐλάττω, τάχος ὑπερφυὲς 
ἔχοντα καὶ ζῶντα ἀπὸ θήρας· ὀρύττει δὲ χειμῶνι τὴν 
γῆν, σωρεύει τε πρὸς τοῖς στομίοις, καθάπερ οἱ ἀσφά-
λακες· ψῆγμα δ' ἐστὶ χρυσοῦ μικρᾶς ἑψήσεως δεόμε-
νον· τοῦθ' ὑποζυγίοις μετίασιν οἱ πλησιόχωροι λά-
θρᾳ· φανερῶς γὰρ διαμάχονται καὶ διώκουσι φεύγον-
τας, καταλαβόντες δὲ διαχρῶνται καὶ αὐτοὺς καὶ τὰ 




Strabo Geogr., Geographica 
Book 15, chapter 1, section 45, line 28

                          οὐδὲν δὲ τούτων οὕτως ὀχλεῖν 
ὡς τὰ λεπτὰ ὀφείδια οὐ μείζω σπιθαμιαίων· εὑρίσκε-
σθαι γὰρ ἐν σκηναῖς, ἐν σκεύεσιν, ἐν θριγγοῖς ἐγκε-
κρυμμένα· τοὺς δὲ πληγέντας αἱμορροεῖν ἐκ παντὸς 
πόρου μετὰ ἐπωδυνίας, ἔπειτα ἀποθνήσκειν, εἰ μὴ 
βοηθήσει τις εὐθύς· τὴν δὲ βοήθειαν ῥᾳδίαν εἶναι διὰ 
τὴν ἀρετὴν τῶν Ἰνδικῶν ῥιζῶν καὶ φαρμάκων. 



Strabo Geogr., Geographica 
Book 15, chapter 1, section 45, line 30

                                                      κρο-
κοδείλους τε οὔτε πολλοὺς οὔτε βλαπτικοὺς ἀνθρώ-
πων ἐν τῷ Ἰνδῷ φησιν εὑρίσκεσθαι, καὶ τὰ ἄλλα δὲ   
ζῷα τὰ πλεῖστα τὰ αὐτὰ ἅπερ ἐν τῷ Νείλῳ γεννᾶσθαι 
πλὴν ἵππου ποταμίου· Ὀνησίκριτος δὲ καὶ τοῦτόν φησι 
γεννᾶσθαι. 



Strabo Geogr., Geographica 
Book 15, chapter 1, section 45, line 36

            τῶν δ' ἐκ θαλάττης φησὶν ὁ Ἀριστόβουλος 
εἰς μὲν τὸν Νεῖλον ἀνατρέχειν μηδὲν ἔξω θρίσσης καὶ 
κεστρέως καὶ δελφῖνος διὰ τοὺς κροκοδείλους, ἐν δὲ τῷ 
Ἰνδῷ πλῆθος· τῶν δὲ καρίδων τὰς μὲν μικρὰς μέχρι 
* ὄρους ἀναθεῖν, τὰς δὲ μεγάλας μέχρι τῶν συμβολῶν 
τοῦ τε Ἰνδοῦ καὶ τοῦ Ἀκεσίνου. 



Strabo Geogr., Geographica 
Book 15, chapter 1, section 53, line 1

Εὐτελεῖς δὲ κατὰ τὴν δίαιταν Ἰνδοὶ πάντες, μᾶλ-
λον δ' ἐν ταῖς στρατείαις· οὐδ' ὄχλῳ περιττῷ χαίρουσι, 
διόπερ εὐκοσμοῦσι. 



Strabo Geogr., Geographica 
Book 15, chapter 1, section 54, line 21

                                    δούλοις δὲ οὗτος 
μέν φησι μηδένα Ἰνδῶν χρῆσθαι, Ὀνησίκριτος δὲ τῶν 
ἐν τῇ Μουσικανοῦ τοῦτ' ἴδιον ἀποφαίνει καὶ ὡς κατόρ-
θωμά γε· καθάπερ καὶ ἄλλα πολλὰ λέγει τῆς χώρας 
ταύτης κατορθώματα ὡς εὐνομωτάτης. 



Strabo Geogr., Geographica 
Book 15, chapter 1, section 67, line 1

Τὴν δὲ φιλοτεχνίαν τῶν Ἰνδῶν ἐμφανίζων σπόγ-
γους φησὶν ἰδόντας παρὰ τοῖς Μακεδόσι μιμήσασθαι, 
τρίχας καὶ σχοινία λεπτὰ καὶ ἁρπεδόνας διαρράψαντας 
εἰς ἔρια, καὶ μετὰ τὸ πιλῆσαι τὰ μὲν ἐξελκύσαντας τὰ 
δὲ βάψαντας χροιαῖς· στλεγγιδοποιούς τε καὶ ληκυθο-
ποιοὺς ταχὺ γενέσθαι πολλούς· ἐπιστολὰς δὲ γράφειν 
ἐν σινδόσι λίαν κεκροτημέναις, τῶν ἄλλων γράμμασιν 
αὐτοὺς μὴ χρῆσθαι φαμένων· χαλκῷ δὲ χρῆσθαι χυ-
τῷ, τῷ δ' ἐλατῷ μή· τὴν δ' αἰτίαν οὐκ εἶπε, καίτοι 
τὴν ἀτοπίαν εἰπὼν τὴν παρακολουθοῦσαν, ὅτι

θραύ-



Strabo Geogr., Geographica 
Book 15, chapter 1, section 67, line 12

                                          τῶν δὲ περὶ 
τῆς Ἰνδικῆς λεγομένων καὶ τοῦτ' ἐστίν, ὅτι ἀντὶ τοῦ 
προσκυνεῖν προσεύχεσθαι τοῖς βασιλεῦσι καὶ πᾶσι τοῖς 
ἐν ἐξουσίᾳ καὶ ὑπεροχῇ νόμος. 



Strabo Geogr., Geographica 
Book 15, chapter 1, section 68, line 7

Τῆς δ' ἀνομολογίας τῶν συγγραφέων ἔστω παρά-
δειγμα καὶ ὁ περὶ τοῦ Καλάνου λόγος· ὅτι μὲν γὰρ συν-
ῆλθεν Ἀλεξάνδρῳ καὶ ἀπέθανεν ἑκὼν παρ' αὐτῷ διὰ 
πυρὸς ὁμολογοῦσι· τὸν δὲ τρόπον οὐ τὸν αὐτόν φα-
σιν οὐδὲ κατὰ τὰς αὐτὰς αἰτίας, ἀλλ' οἱ μὲν οὕτως εἰ-
ρήκασι· συνακολουθῆσαι γὰρ ὡς ἐγκωμιαστὴν τοῦ 
βασιλέως ἔξω τῶν τῆς Ἰνδικῆς ὅρων παρὰ τὸ κοινὸν 
ἔθος τῶν ἐκεῖ φιλοσόφων· ἐκείνους γὰρ τοῖς αὐτόθι 
συνεῖναι βασιλεῦσιν ὑφηγουμένους τὰ περὶ τοὺς θεούς, 
ὡς τοὺς μάγους τοῖς Πέρσαις. 



Strabo Geogr., Geographica 
Book 15, chapter 1, section 68, line 34

πυρώδεις εἰς· πῦρ ὠθουμένους· οἷος ἦν καὶ ὁ Κάλανος, 
ἀκόλαστος ἄνθρωπος καὶ ταῖς Ἀλεξάνδρου τραπέζαις 
δεδουλωμένος· τοῦτον μὲν οὖν ψέγεσθαι, τὸν δὲ 
Μάνδανιν ἐπαινεῖσθαι, ὃς τῶν τοῦ Ἀλεξάνδρου ἀγγέ-
λων καλούντων πρὸς τὸν Διὸς υἱὸν πειθομένῳ τε δῶ-
ρα ἔσεσθαι ὑπισχνουμένων ἀπειθοῦντι δὲ κόλασιν 
μήτ' ἐκεῖνον φαίη Διὸς υἱὸν ὅν γε ἄρχειν μηδὲ πολλο-
στοῦ μέρους τῆς γῆς, μήτε αὐτῷ δεῖν τῶν παρ' ἐκεί-
νου δωρεῶν ὧν οὐδεὶς κόρος, μήτε δὲ ἀπειλῆς εἶναι 
φόβον ᾧ ζῶντι μὲν ἀρκοῦσα εἴη τροφὸς ἡ Ἰνδική, ἀπο-
θανὼν δὲ ἀπαλλάξαιτο τῆς τετρυχωμένης ὑπὸ γήρως 
σαρκός, μεταστὰς εἰς βελτίω καὶ καθαρώτερον βίον· 
ὥστ' ἐπαινέσαι τὸν Ἀλέξανδρον καὶ συγχωρῆσαι. 



Strabo Geogr., Geographica 
Book 15, chapter 1, section 69, line 2

Λέγεται δὲ καὶ ταῦτα παρὰ τῶν συγγραφέων, ὅτι 
σέβονται μὲν τὸν ὄμβριον Δία Ἰνδοὶ καὶ τὸν Γάγγην 
ποταμὸν καὶ τοὺς ἐγχωρίους δαίμονας. 



Strabo Geogr., Geographica 
Book 15, chapter 1, section 69, line 13

                              τῶν τε μυρμήκων τινὰς 
καὶ πτερωτοὺς λέγουσι τῶν χρυσωρύχων· ψήγματά   
τε χρυσοῦ καταφέρειν τοὺς ποταμούς, καθάπερ τοὺς 
Ἰβηρικούς· ἐν δὲ ταῖς κατὰ τὰς ἑορτὰς πομπαῖς πολλοὶ 
μὲν ἐλέφαντες πέμπονται χρυσῷ κεκοσμημένοι καὶ 
ἀργύρῳ, πολλὰ δὲ τέθριππα καὶ βοϊκὰ ζεύγη· εἶθ' ἡ 
στρατιὰ κεκοσμημένη· καὶ χρυσώματα δὲ τῶν μεγά-
λων λεβήτων καὶ κρατήρων ὀργυιαίων· καὶ τοῦ Ἰνδι-
κοῦ χαλκοῦ τράπεζαί τε καὶ θρόνοι καὶ ἐκπώματα καὶ 
λουτῆρες, λιθοκόλλητα τὰ πλεῖστα σμαράγδοις καὶ βη-
ρύλλοις καὶ ἄνθραξιν Ἰνδικοῖς· καὶ ἐσθὴς δὲ ποικίλη 
χρυσόπαστος, καὶ βόνασοι καὶ παρδάλεις καὶ λέοντες 
τιθασοὶ καὶ τῶν ποικίλων ὀρνέων καὶ εὐφθόγγων πλῆ-
θος. 



Strabo Geogr., Geographica 
Book 15, chapter 1, section 69, line 16

καὶ πτερωτοὺς λέγουσι τῶν χρυσωρύχων· ψήγματά   
τε χρυσοῦ καταφέρειν τοὺς ποταμούς, καθάπερ τοὺς 
Ἰβηρικούς· ἐν δὲ ταῖς κατὰ τὰς ἑορτὰς πομπαῖς πολλοὶ 
μὲν ἐλέφαντες πέμπονται χρυσῷ κεκοσμημένοι καὶ 
ἀργύρῳ, πολλὰ δὲ τέθριππα καὶ βοϊκὰ ζεύγη· εἶθ' ἡ 
στρατιὰ κεκοσμημένη· καὶ χρυσώματα δὲ τῶν μεγά-
λων λεβήτων καὶ κρατήρων ὀργυιαίων· καὶ τοῦ Ἰνδι-
κοῦ χαλκοῦ τράπεζαί τε καὶ θρόνοι καὶ ἐκπώματα καὶ 
λουτῆρες, λιθοκόλλητα τὰ πλεῖστα σμαράγδοις καὶ βη-
ρύλλοις καὶ ἄνθραξιν Ἰνδικοῖς· καὶ ἐσθὴς δὲ ποικίλη 
χρυσόπαστος, καὶ βόνασοι καὶ παρδάλεις καὶ λέοντες 
τιθασοὶ καὶ τῶν ποικίλων ὀρνέων καὶ εὐφθόγγων πλῆ-
θος. 



Strabo Geogr., Geographica 
Book 15, chapter 1, section 71, line 3

     τοὺς δὲ πολιτικοὺς σινδονίτας κατὰ πόλιν ζῆν ἢ   
καὶ κατ' ἀγρούς, καθημμένους νεβρίδας ἢ δορκάδων 
δοράς· ὡς δ' εἰπεῖν, Ἰνδοὺς ἐσθῆτι λευκῇ χρῆσθαι 
καὶ σινδόσι λευκαῖς καὶ καρπάσοις, ὑπεναντίως τοῖς 
εἰποῦσιν εὐανθέστατα αὐτοὺς ἀμπέχεσθαι φορήματα· 
κομᾶν δὲ καὶ πωγωνοτροφεῖν πάντας, ἀναπλεκομέ-
νους δὲ μιτροῦσθαι τὰς κόμας. 



Strabo Geogr., Geographica 
Book 15, chapter 1, section 73, line 2

Φησὶ γὰρ οὗτος ἐν Ἀντιοχείᾳ τῇ ἐπὶ Δάφνῃ πα-
ρατυχεῖν τοῖς Ἰνδῶν πρέσβεσιν ἀφιγμένοις παρὰ Καί-
σαρα τὸν Σεβαστόν· οὓς ἐκ μὲν τῆς ἐπιστολῆς πλείους 
δηλοῦσθαι, σωθῆναι δὲ τρεῖς μόνους, οὓς ἰδεῖν φησι, 
τοὺς δ' ἄλλους ὑπὸ μήκους τῶν ὁδῶν διαφθαρῆναι τὸ 
πλέον· τὴν δ' ἐπιστολὴν ἑλληνίζειν ἐν διφθέρᾳ γε-
γραμμένην, δηλοῦσαν ὅτι Πῶρος εἴη ὁ γράψας, ἑξα-
κοσίων δὲ ἄρχων βασιλέων ὅμως περὶ πολλοῦ ποιοῖτο 
φίλος εἶναι Καίσαρι, καὶ ἕτοιμος εἴη δίοδόν τε παρέ-
χειν ὅπῃ βούλεται καὶ συμπράττειν ὅσα καλῶς ἔχει. 



Strabo Geogr., Geographica 
Book 15, chapter 1, section 73, line 25

                                            συνῆν δέ, ὥς   
φησι, καὶ ὁ Ἀθήνησι κατακαύσας ἑαυτόν· ποιεῖν δὲ 
τοῦτο τοὺς μὲν ἐπὶ κακοπραγίᾳ ζητοῦντας ἀπαλλαγὴν 
τῶν παρόντων, τοὺς δ' ἐπ' εὐπραγίᾳ, καθάπερ τοῦ-
τον· ἅπαντα γὰρ κατὰ γνώμην πράξαντα μέχρι νῦν 
ἀπιέναι δεῖν, μή τι τῶν ἀβουλήτων χρονίζοντι συμπέ-
σοι· καὶ δὴ καὶ γελῶντα ἁλέσθαι γυμνὸν λίπ' ἀλη-
λιμμένον ἐν περιζώματι ἐπὶ τὴν πυράν· ἐπιγεγράφθαι 
δὲ τῷ τάφῳ “Ζαρμανοχηγὰς Ἰνδὸς ἀπὸ Βαργόσης κατὰ 
“τὰ πάτρια Ἰνδῶν ἔθη ἑαυτὸν ἀπαθανατίσας κεῖται. 



Strabo Geogr., Geographica 
Book 15, chapter 2, section 1, line 1

Μετὰ δὲ τὴν Ἰνδικήν ἐστιν ἡ Ἀριανή, μερὶς πρώτη 
τῆς ὑπὸ Πέρσαις τῆς μετὰ τὸν Ἰνδὸν ποταμὸν καὶ 
τῶν ἄνω σατραπειῶν τῶν ἐκτὸς τοῦ Ταύρου, τὰ μὲν 
νότια καὶ τὰ ἀρκτικὰ μέρη τῇ αὐτῇ θαλάττῃ καὶ τοῖς 
αὐτοῖς ὄρεσιν ἀφοριζομένη οἷσπερ καὶ ἡ Ἰνδική, καὶ 
τῷ αὐτῷ ποταμῷ τῷ Ἰνδῷ, μέσον ἔχουσα αὐτὸν ἑαυ-
τῆς τε καὶ τῆς Ἰνδικῆς, ἐντεῦθεν δὲ πρὸς τὴν ἑσπέραν 
ἐκτεινομένη μέχρι τῆς ἀπὸ Κασπίων πυλῶν εἰς Καρ-
μανίαν γραφομένης γραμμῆς, ὥστε εἶναι τετράπλευ-
ρον τὸ σχῆμα. 



Strabo Geogr., Geographica 
Book 15, chapter 2, section 1, line 11

                τὸ μὲν οὖν νότιον πλευρὸν ἀπὸ τῶν 
ἐκβολῶν ἄρχεται τοῦ Ἰνδοῦ καὶ τῆς Παταληνῆς, τε-
λευτᾷ δὲ πρὸς Καρμανίαν καὶ τοῦ Περσικοῦ κόλπου 
τὸ στόμα, ἄκραν ἔχον ἐκκειμένην ἱκανῶς πρὸς νότον· 
εἶτα εἰς τὸν κόλπον λαμβάνει καμπὴν ὡς ἐπὶ τὴν Περ-
σίδα. 



Strabo Geogr., Geographica 
Book 15, chapter 2, section 1, line 18

       οἰκοῦσι δὲ Ἄρβιες πρῶτον, ὁμώνυμοι τῷ πο-
ταμῷ Ἄρβει τῷ ὁρίζοντι αὐτοὺς ἀπὸ τῶν ἑξῆς Ὠριτῶν, 
ὅσον χιλίων σταδίων ἔχοντες παραλίαν, ὥς φησι Νέ-
αρχος· Ἰνδῶν δ' ἐστὶ μερὶς καὶ αὕτη· εἶτ' Ὠρῖται ἔθνος 
αὐτόνομον· τούτων δ' ὁ παράπλους χιλίων ὀκτακο-
σίων, ὁ δὲ τῶν ἑξῆς Ἰχθυοφάγων ἑπτακισχίλιοι τετρα-
κόσιοι, ὁ δὲ τῶν Καρμανίων τρισχίλιοι ἑπτακόσιοι μέ-
χρι Περσίδος· ὥσθ' οἱ σύμπαντες μύριοι τρισχίλιοι 
ἐννακόσιοι. 



Strabo Geogr., Geographica 
Book 15, chapter 2, section 3, line 1

Ὑπέρκειται δὲ τούτων ἡ Γεδρωσία, τῆς μὲν Ἰνδι-
κῆς ἧττον ἔμπυρος τῆς δ' ἄλλης Ἀσίας μᾶλλον, καὶ 
τοῖς καρποῖς καὶ τοῖς ὕδασιν ἐνδεὴς πλὴν θέρους, οὐ 
πολὺ ἀμείνων τῆς τῶν Ἰχθυοφάγων· ἀρωματοφόρος 
δὲ νάρδου μάλιστα καὶ σμύρνης, ὥστε τὴν Ἀλεξάν-
δρου στρατιὰν ὁδεύουσαν ἀντὶ ὀρόφου καὶ στρωμάτων 
τούτοις χρῆσθαι, εὐωδιαζομένην ἅμα καὶ ὑγιεινότε-
ρον τὸν ἀέρα ἔχουσαν παρὰ τοῦτο· γενέσθαι δ' αὐ-
τοῖς θέρους τὴν ἐκ τῆς Ἰνδικῆς ἄφοδον ἐπίτηδες συν-
έβη· τότε γὰρ ὄμβρους ἔχειν τὴν Γεδρωσίαν καὶ 




Strabo Geogr., Geographica 
Book 15, chapter 2, section 5, line 17

                                              φασὶ δὲ φι-
λονεικῆσαι τὸν Ἀλέξανδρον καίπερ εἰδότα τὰς ἀπο-
ρίας πρὸς τὴν κατέχουσαν δόξαν, ὡς Σεμίραμις μὲν 
ἐξ Ἰνδῶν φεύγουσα σωθείη μετὰ ἀνδρῶν ὡς εἴκοσι, 
Κῦρος δὲ ἑπτά, εἰ δύναιτο αὐτὸς τοσοῦτο στράτευμα 
διασῶσαι διὰ τῆς αὐτῆς χώρας, νικῶν καὶ ταῦτα. 



Strabo Geogr., Geographica 
Book 15, chapter 2, section 8, line 9

ὁρίζεσθαι μὲν γάρ φησι τὴν Ἀριανὴν ἐκ μὲν τῶν πρὸς 
ἕω τῷ Ἰνδῷ, πρὸς νότον δὲ τῇ μεγάλῃ θαλάττῃ, πρὸς 
ἄρκτον δὲ τῷ Παροπαμισῷ καὶ τοῖς ἑξῆς ὄρεσι μέχρι 
Κασπίων πυλῶν, τὰ δὲ πρὸς ἑσπέραν τοῖς αὐτοῖς ὅροις 
οἷς ἡ μὲν Παρθυηνὴ πρὸς Μηδίαν ἡ δὲ Καρμανία 
πρὸς τὴν Παραιτακηνὴν καὶ Περσίδα διώρισται· πλά-
τος δὲ τῆς χώρας τὸ τοῦ Ἰνδοῦ μῆκος τὸ ἀπὸ τοῦ Πα-
ροπαμισοῦ μέχρι τῶν ἐκβολῶν μύριοι καὶ δισχίλιοι 
στάδιοι (οἱ δὲ τρισχιλίους φασί)· μῆκος δὲ ἀπὸ Κα-
σπίων πυλῶν, ὡς ἐν τοῖς Ἀσιατικοῖς σταθμοῖς ἀναγέ-
γραπται, διττόν. 



Strabo Geogr., Geographica 
Book 15, chapter 2, section 8, line 25

                  μέχρι μὲν Ἀλεξανδρείας τῆς ἐν Ἀρίοις 
ἀπὸ Κασπίων πυλῶν διὰ τῆς Παρθυαίας μία καὶ ἡ 
αὐτὴ ὁδός· εἶθ' ἡ μὲν ἐπ' εὐθείας διὰ τῆς Βακτριανῆς 
καὶ τῆς ὑπερβάσεως τοῦ ὄρους εἰς Ὀρτόσπανα ἐπὶ τὴν 
ἐκ Βάκτρων τρίοδον ἥτις ἐστὶν ἐν τοῖς Παροπαμισά-
δαις· ἡ δ' ἐκτρέπεται μικρὸν ἀπὸ τῆς Ἀρίας πρὸς νό-
τον εἰς Προφθασίαν τῆς Δραγγιανῆς· εἶτα πάλιν ἡ 
λοιπὴ μέχρι τῶν ὅρων τῆς Ἰνδικῆς καὶ τοῦ Ἰνδοῦ· ὥστε 
μακροτέρα ἐστὶν αὕτη ἡ διὰ τῶν Δραγγῶν καὶ Ἀραχω-
τῶν, σταδίων μυρίων πεντακισχιλίων τριακοσίων ἡ 
πᾶσα. 



Strabo Geogr., Geographica 
Book 15, chapter 2, section 9, line 2

Ἡ δὲ τάξις τῶν ἐθνῶν τοιαύτη· παρὰ μὲν τὸν Ἰν-
δὸν οἱ Παροπαμισάδαι, ὧν ὑπέρκειται ὁ Παροπαμισὸς 
ὄρος, εἶτ' Ἀραχωτοὶ πρὸς νότον, εἶτ' ἐφεξῆς πρὸς νότον 
Γεδρωσηνοὶ σὺν τοῖς ἄλλοις τοῖς τὴν παραλίαν ἔχου-
σιν· ἅπασι δὲ παρὰ τὰ πλάτη τῶν χωρίων παράκειται 
ὁ Ἰνδός. 



Strabo Geogr., Geographica 
Book 15, chapter 2, section 9, line 6

             * τούτων δ' ἐκ μέρους τῶν παρὰ τὸν Ἰνδὸν 
ἔχουσί τινα Ἰνδοὶ πρότερον ὄντα Περσῶν, ἃ ἀφείλετο 
μὲν ὁ Ἀλέξανδρος τῶν Ἀριανῶν καὶ κατοικίας ἰδίας 
συνεστήσατο, ἔδωκε δὲ Σέλευκος ὁ Νικάτωρ Σανδρο-
κόττῳ, συνθέμενος ἐπιγαμίαν καὶ ἀντιλαβὼν ἐλέφαν-
τας πεντακοσίους. 



Strabo Geogr., Geographica 
Book 15, chapter 2, section 10, line 20

                  ἔστι δὲ τὰ μεσημβρινὰ μὲν τοῦ ὄρους 
τοῦ Παροπαμισοῦ Ἰνδικά τε καὶ Ἀριανά· τὰ δὲ προς-
άρκτια τὰ μὲν πρὸς ἑσπέραν Βάκτρια . 



Strabo Geogr., Geographica 
Book 15, chapter 2, section 10, line 23

                   διαχειμάσας δ' αὐτόθι ὑπερδέξιον 
ἔχων τὴν Ἰνδικὴν καὶ πόλιν κτίσας ὑπερήκρισεν εἰς 
τὴν Βακτριανὴν διὰ ψιλῶν ὁδῶν πλὴν τερμίνθου θα-
μνώδους ὀλίγης, ἀπορούμενος καὶ τροφῆς ὥστε ταῖς 
τῶν κτηνῶν σαρξὶ χρῆσθαι, καὶ ταύταις ὠμαῖς διὰ τὴν 
ἀξυλίαν· πρὸς δὲ τὴν ὠμοσιτίαν πεπτικὸν ἦν αὐτοῖς 
τὸ σίλφιον πολὺ πεφυκός. 



Strabo Geogr., Geographica 
Book 15, chapter 2, section 11, line 1

Περὶ ταῦτα δέ που τὰ μέρη τῆς ὁμόρου τῇ Ἰνδικῇ 
καὶ τὴν Χααρηνὴν εἶναι συμβαίνει· ἔστι δὲ τῶν ὑπὸ 
τοῖς Παρθυαίοις αὕτη προσεχεστάτη τῇ Ἰνδικῇ· διέχει 
δὲ τῆς * Ἀριανῆς δι' Ἀραχωτῶν καὶ τῆς λεχθείσης ὀρει-
νῆς σταδίους μυρίους * ἐνακισχιλίους. 



Strabo Geogr., Geographica 
Book 15, chapter 2, section 13, line 1

Λέγουσι μὲν οὖν καὶ οἱ νῦν πλέοντες εἰς Ἰνδοὺς 
μεγέθη θηρίων καὶ ἐπιφανείας, ἀλλ' οὔτε ἀθρόων οὔτ' 
ἐπιφερομένων πολλάκις, ἀλλ' ἀποσοβηθέντα τῇ κραυγῇ 
καὶ τῇ σάλπιγγι ἀπαλλάττεσθαι. 



Strabo Geogr., Geographica 
Book 15, chapter 2, section 14, line 2

Ἡ δὲ Καρμανία τελευταία μέν ἐστι τῆς ἀπὸ τοῦ 
Ἰνδοῦ παραλίας, ἀρκτικωτέρα δ' ἐστὶ πολὺ τῆς τοῦ Ἰν-
δοῦ ἐκβολῆς· τὸ μέντοι πρῶτον αὐτῆς ἄκρον ἔκκειται 
πρὸς νότον εἰς τὴν μεγάλην θάλατταν, ποιήσασα δὲ τὸ 
στόμα τοῦ Περσικοῦ κόλπου πρὸς τὴν ἀπὸ τῆς εὐδαί-
μονος Ἀραβίας ἄκραν ἐν ἀπόψει οὖσαν, κάμπτεται 
πρὸς τὸν Περσικὸν κόλπον ἕως ἂν συνάψῃ τῇ Περ-
σίδι· πολλὴ δὲ κἀν τῇ μεσογαίᾳ ἐστὶν ἐκτεινομένη 
μεταξὺ τῆς Γεδρωσίας καὶ τῆς Περσίδος, παραλλάτ-
τουσα πλέον τῆς Γεδρωσίας πρὸς τὴν ἄρκτον. 



Strabo Geogr., Geographica 
Book 15, chapter 3, section 7, line 19

                 ἀλλ' ὁ ἐκτοπισμὸς τῆς Ἀλεξάνδρου στρα-
τιᾶς εἰς Βάκτρα καὶ Ἰνδοὺς πολλά τε ἄλλα νεωτερισθῆ-
ναι παρεσκεύασε, καὶ δὴ καὶ τοῦθ' ἓν τῶν νεωτερι-
σθέντων ὑπῆρξεν. 



Strabo Geogr., Geographica 
Book 15, chapter 3, section 11, line 13

                              ἡ μὲν δὴ μεσόγαια τοιαύ-
τη· ἡ δὲ παραλία τεναγώδης ἐστὶ καὶ ἀλίμενος· διὰ 
τοῦτο γοῦν καὶ φησὶν ὁ Νέαρχος μηδὲ καθοδηγῶν ἐπι-
χωρίων τυγχάνειν ἡνίκα τῷ στόλῳ παρέπλει πρὸς τὴν 
Βαβυλωνίαν ἐκ τῆς Ἰνδικῆς, ὅτι προσόρμους οὐκ εἶχεν, 
οὐδ' ἀνθρώπων εὐπορεῖν οἷός τ' ἦν τῶν ἡγησομένων 
κατ' ἐμπειρίαν. 



Strabo Geogr., Geographica 
Book 16, chapter 1, section 9, line 3

Διαρρεῖται δ' ὑπὸ πλειόνων μὲν ποταμῶν ἡ χώρα, 
μεγίστων δὲ τοῦ τε Εὐφράτου καὶ τοῦ Τίγριος· μετὰ 
γὰρ τοὺς Ἰνδικοὺς οὗτοι λέγονται δευτερεύειν κατὰ 
τὰ νότια μέρη τῆς Ἀσίας οἱ ποταμοί· ἔχουσι δ' ἀνά-
πλους ὁ μὲν ἐπὶ τὴν Ὦπιν καὶ τὴν νῦν Σελεύκειαν (ἡ 
δὲ Ὦπις κώμη ἐμπόριον τῶν κύκλῳ τόπων) ὁ δ' ἐπὶ 
Βαβυλῶνα πλειόνων ἢ τρισχιλίων σταδίων. 



Strabo Geogr., Geographica 
Book 16, chapter 2, section 39, line 11

θείας ἔχει, παρά γε τοῖς ἀνθρώποις ἐπεπίστευτο καὶ 
ἐνενόμιστο, καὶ διὰ τοῦτο καὶ οἱ μάντεις ἐτιμῶντο 
ὥστε καὶ βασιλείας ἀξιοῦσθαι, ὡς τὰ παρὰ τῶν θεῶν 
ἡμῖν ἐκφέροντες παραγγέλματα καὶ ἐπανορθώματα καὶ 
ζῶντες καὶ ἀποθανόντες· τοιοῦτος δὲ ὁ Ἀμφιάρεως 
καὶ ὁ Τροφώνιος καὶ [ὁ] Ὀρφεὺς καὶ ὁ Μουσαῖος καὶ ὁ 
παρὰ τοῖς Γέταις θεός, τὸ μὲν παλαιὸν Ζάμολξις Πυ-
θαγόρειός τις, καθ' ἡμᾶς δὲ ὁ τῷ Βυρεβίστᾳ θεσπίζων 
Δεκαίνεος· παρὰ δὲ τοῖς Βοσπορηνοῖς Ἀχαΐκαρος, 
παρὰ δὲ τοῖς Ἰνδοῖς οἱ γυμνοσοφισταί, παρὰ δὲ τοῖς 
Πέρσαις οἱ μάγοι καὶ νεκυομάντεις καὶ ἔτι οἱ λεγόμε-
νοι λεκανομάντεις καὶ ὑδρομάντεις, παρὰ δὲ τοῖς Ἀσσυ-
ρίοις οἱ Χαλδαῖοι, παρὰ δὲ τοῖς Ῥωμαίοις οἱ Τυρρηνι-
κοὶ οἰωνοσκόποι. 



Strabo Geogr., Geographica 
Book 16, chapter 4, section 2, line 21

                                                τὰ δ' ἔσχατα 
πρὸς νότον καὶ ἀνταίροντα τῇ Αἰθιοπίᾳ βρέχεταί τε 
θερινοῖς ὄμβροις καὶ δισπορεῖται παραπλησίως τῇ Ἰν-
δικῇ, ποταμοὺς δ' ἔχει καταναλισκομένους εἰς πεδία 
καὶ λίμνας, εὐκαρπία δ' ἐστὶν ἥ τε ἄλλη καὶ μελιτουρ-
γεῖα δαψιλῆ, βοσκημάτων τε ἀφθονία πλὴν ἵππων καὶ 
ἡμιόνων καὶ ὑῶν, ὄρνεά τε παντοῖα πλὴν χηνῶν καὶ 
ἀλεκτορίδων. 



Strabo Geogr., Geographica 
Book 16, chapter 4, section 9, line 11

                                            μετὰ δὲ τὴν Ἐλαίαν 
αἱ Δημητρίου σκοπιαὶ καὶ βωμοὶ Κόνωνος· ἐν δὲ τῇ 
μεσογαίᾳ καλάμων Ἰνδικῶν φύεται πλῆθος· καλεῖται 
δὲ ἡ χώρα Κορακίου· ἦν δέ τις ἐν βάθει Ἐνδέρα γυ-
μνητῶν ἀνθρώπων κατοικία, τόξοις χρωμένων καλα-
μίνοις καὶ πεπυρακτωμένοις οἰστοῖς· ἀπὸ δένδρων δὲ 
τοξεύουσι τὰ θηρία τὸ πλέον, ἔστι δ' ὅτε καὶ ἀπὸ γῆς· 
πολὺ δ' ἐστὶ παρ' αὐτοῖς πλῆθος τῶν ἀγρίων βοῶν· 
ἀπὸ δὲ τῆς τούτων καὶ τῶν ἄλλων θηρίων κρεοφαγίας 
ζῶσιν, ἐπὰν δὲ μηδὲν θηρεύσωσι, τὰ ξηρὰ δέρματα ἐπ' 
ἀνθρακιᾶς ὀπτῶντες ἀρκοῦνται τῇ τοιαύτῃ τροφῇ. 



Strabo Geogr., Geographica 
Book 16, chapter 4, section 10, line 5

Ἔτι δ' ὑπὲρ τούτων ὡς πρὸς μεσημβρίαν οἱ κυνα-
μολγοί, ὑπὸ δὲ τῶν ἐντοπίων ἄγριοι καλούμενοι, κα-
τάκομοι, καταπώγωνες, κύνας ἐκτρέφοντες εὐμεγέθεις, 
οἷς θηρεύουσι τοὺς ἐπερχομένους ἐκ τῆς πλησιοχώρου 
βόας Ἰνδικούς, εἴθ' ὑπὸ θηρίων ἐξελαυνομένους εἴτε 
σπάνει νομῆς· ἡ δ' ἔφοδος αὐτῶν ἀπὸ θερινῶν τρο-  
πῶν μέχρι μέσου χειμῶνος. 



Strabo Geogr., Geographica 
Book 16, chapter 4, section 16, line 24

                                       καὶ δρακόντων δ' εἴ-
ρηκε μεγέθη τριάκοντα πηχῶν ὁ Ἀρτεμίδωρος ἐλέφαν-
τας καὶ ταύρους χειρουμένων, μετριάσας ταύτῃ γε· οἱ 
γὰρ Ἰνδικοὶ μυθωδέστεροι καὶ οἱ Λιβυκοί, οἷς γε καὶ 
πόα ἐπιπεφυκέναι λέγεται. 



Strabo Geogr., Geographica 
Book 16, chapter 4, section 24, line 21

               ἐκ μὲν οὖν τῆς Λευκῆς κώμης εἰς Πέ-
τραν, ἐντεῦθεν δ' εἰς Ῥινοκόλουρα τῆς πρὸς Αἰγύπτῳ 
Φοινίκης τὰ φορτία κομίζεται κἀντεῦθεν εἰς τοὺς ἄλ-
λους, νυνὶ δὲ τὸ πλέον εἰς τὴν Ἀλεξάνδρειαν τῷ Νεί-
λῳ· κατάγεται δ' ἐκ τῆς Ἀραβίας καὶ τῆς Ἰνδικῆς εἰς 
Μυὸς ὅρμον· εἶθ' ὑπέρθεσις εἰς Κοπτὸν τῆς Θηβαΐ-
δος καμήλοις ἐν διώρυγι τοῦ Νείλου κειμένην· [εἶτ'] 
εἰς Ἀλεξάνδρειαν. 



Strabo Geogr., Geographica 
Book 16, chapter 4, section 25, line 4

Τὴν μὲν οὖν ἀρωματοφόρον διαιροῦσιν εἰς τέττα-
ρας μερίδας, ὥσπερ εἰρήκαμεν· τῶν ἀρωμάτων δὲ λί-
βανον μὲν καὶ σμύρναν ἐκ δένδρων γίνεσθαί φασι, κα-
σίαν δὲ καὶ ἐκ θάμνων· τινὲς δὲ τὴν πλείω ἐξ Ἰνδῶν 
εἶναι, τοῦ δὲ λιβάνου βέλτιστον τὸν πρὸς τῇ Περσίδι. 



Strabo Geogr., Geographica 
Book 16, chapter 4, section 27, line 37

                                            τῆς δὲ τῶν 
Ἀράβων εὐδαιμονίας καὶ Ἀλέξανδρον ἄν τις ποιή-
σαιτο μάρτυρα τὸν διανοηθέντα, ὥς φασι, καὶ βασί-
λειον αὐτὴν ποιήσασθαι μετὰ τὴν ἐξ Ἰνδῶν ἐπάνοδον. 



Strabo Geogr., Geographica 
Book 17, chapter 1, section 13, line 19

                                           ὅπου οὖν 
ὁ κάκιστα καὶ ῥᾳθυμότατα τὴν βασιλείαν διοικῶν το-
σαῦτα προσωδεύετο, τί χρὴ νομίσαι τὰ νῦν διὰ τοσαύ-
της ἐπιμελείας οἰκονομούμενα καὶ τῶν Ἰνδικῶν ἐμπο-
ριῶν καὶ τῶν Τρωγλοδυτικῶν ἐπηυξημένων ἐπὶ τοσοῦ-
τον; 



Strabo Geogr., Geographica 
Book 17, chapter 1, section 13, line 24

     πρότερον μέν γε οὐδ' εἴκοσι πλοῖα ἐθάρρει τὸν 
Ἀράβιον κόλπον διαπερᾶν ὥστε ἔξω τῶν στενῶν ὑπερ-
κύπτειν, νῦν δὲ καὶ στόλοι μεγάλοι στέλλονται μέχρι 
τῆς Ἰνδικῆς καὶ τῶν ἄκρων τῶν Αἰθιοπικῶν, ἐξ ὧν ὁ 
πολυτιμότατος κομίζεται φόρτος εἰς τὴν Αἴγυπτον, 
κἀντεῦθεν πάλιν εἰς τοὺς ἄλλους ἐκπέμπεται τόπους, 
ὥστε τὰ τέλη διπλάσια συνάγεται τὰ μὲν εἰσαγωγικὰ 
τὰ δὲ ἐξαγωγικά· τῶν δὲ βαρυτίμων βαρέα καὶ τὰ 
τέλη. 



Strabo Geogr., Geographica 
Book 17, chapter 1, section 45, line 8

                                            ἐφάνη δὴ τῇ 
πείρᾳ πολὺ τὸ χρήσιμον, καὶ νῦν ὁ Ἰνδικὸς φόρτος 
ἅπας καὶ ὁ Ἀράβιος καὶ τοῦ Αἰθιοπικοῦ ὁ τῷ Ἀραβίῳ 
κόλπῳ κατακομιζόμενος εἰς Κοπτὸν φέρεται, καὶ τοῦτ' 
ἔστιν ἐμπόριον τῶν τοιούτων φορτίων. 



Strabo Geogr., Geographica 
Book 17, chapter 1, section 46, line 32

        ὑπὲρ δὲ τοῦ Μεμνονίου θῆκαι βασιλέων ἐν σπη-
λαίοις λατομηταὶ περὶ τετταράκοντα, θαυμαστῶς κα-
τεσκευασμέναι καὶ θέας ἄξιαι· ἐν δὲ ταῖς Θήβαις ἐπί 
τινων ὀβελίσκων ἀναγραφαὶ δηλοῦσαι τὸν πλοῦτον 
τῶν τότε βασιλέων καὶ τὴν ἐπικράτειαν, ὡς μέχρι Σκυ-
θῶν καὶ Βακτρίων καὶ Ἰνδῶν καὶ τῆς νῦν Ἰωνίας δια-
τείνασαν, καὶ φόρων πλῆθος καὶ στρατιᾶς περὶ ἑκατὸν 
μυριάδας. 



Strabo Geogr., Geographica 
Book 17, chapter 2, section 4, line 4

Τοῖς δ' Αἰγυπτιακοῖς καὶ ταῦτα προσθετέον ὅσα 
ἰδιάζοντα, οἷον ὁ Αἰγύπτιος λεγόμενος κύαμος ἐξ οὗ 
τὸ κιβώριον, καὶ ἡ βύβλος· ἐνταῦθα γὰρ καὶ παρ' Ἰν-
δοῖς μόνον· ἡ δὲ περσέα ἐνταῦθα μόνον καὶ παρ' Αἰ-
θίοψι, δένδρον μέγα, καρπὸν ἔχον γλυκὺν καὶ μέγαν, 
καὶ ἡ συκάμινος ἡ ἐκφέρουσα τὸν λεγόμενον καρπὸν 
συκόμορον· σύκῳ γὰρ ἔοικεν· ἄτιμον δ' ἐστὶ κατὰ 
τὴν γεῦσιν· γίνεται δὲ καὶ τὸ κόρσιον καὶ ὅμοιόν τι 
πεπέρει τράγημα, μικρῷ αὐτοῦ μεῖζον. 



Strabo Geogr., Geographica 
Book 17, chapter 3, section 5, line 15

                                       Βόγον δὲ τὸν βα-
σιλέα τῶν Μαυρουσίων ἀναβάντα ἐπὶ τοὺς ἑσπερίους 
Αἰθίοπας καταπέμψαι τῇ γυναικὶ δῶρα καλάμους τοῖς 
Ἰνδικοῖς ὁμοίους, ὧν ἕκαστον γόνυ χοίνικας χωροῦν 
ὀκτώ· καὶ ἀσπαράγων δ' ἐμφερῆ μεγέθη. 



Strabo Geogr., Geographica 
Book 17, chapter 3, section 7, line 33

τοὺς δὲ Φαρουσίους ἔνιοί φασιν Ἰνδοὺς εἶναι τοὺς συγ-
κατελθόντας Ἡρακλεῖ δεῦρο. 



Strabo Geogr., Geographica 
Book 17, chapter 3, section 10, line 23

                                                     ὡς δὲ 
λέγεται πρὸς τὴν οἰκουμένην ὅλην καὶ τὰς ἐσχατιὰς 
τὰς τοιαύτας, οἵα καὶ ἡ Ἰνδικὴ καὶ ἡ Ἰβηρία, λέγοι ἂν 
εἰ ἄρα τὴν τοιαύτην ἀπόφασιν. 



Strabo Geogr., Geographica 
Book 17, chapter 3, section 24, line 21

                   τῆς δὲ μεσογαίας καὶ τῆς ἐν βάθει 
τὴν μὲν ἔχουσιν αὐτοί, τὴν δὲ Παρθυαῖοι καὶ οἱ ὑπὲρ 
τούτων βάρβαροι, πρός τε ταῖς ἀνατολαῖς καὶ ταῖς ἄρ-
κτοις Ἰνδοὶ καὶ Βάκτριοι καὶ Σκύθαι, εἶτ' Ἄραβες καὶ 
Αἰθίοπες· προστίθεται δὲ ἀεί τι παρ' ἐκείνων αὐτοῖς. 


Strabo Geogr., Fragmenta (0099: 003)
“FGrH \#91”.
Volume-Jacobyʹ-F 2a,91,F, fragment 3, line 2

STRABON II 1, 9: ἅπαντες μὲν τοίνυν οἱ περὶ τῆς 
Ἰνδικῆς γράψαντες ὡς ἐπὶ τὸ πολὺ ψευδολόγοι γεγόνασι, καθ' ὑπερβολὴν 
δὲ Δηίμαχος (III)· τὰ δὲ δεύτερα λέγει Μεγασθένης (III)· Ὀνησίκριτος 
(134 T 11) δὲ καὶ Νέαρχος (133 T 14) καὶ ἄλλοι τοιοῦτοι παραψελλίζοντες 
ἤδη. 

\end{greek}



\section{Manetho}
\blockquote[From Wikipedia\footnote{\url{http://en.wikipedia.org/wiki/Manetho}}]{

Manetho (play /ˈmænɨθoʊ/; Ancient Greek: Μανέθων, Manethōn, or Μανέθως, Manethōs) was an Egyptian historian and priest from Sebennytos (ancient Egyptian: Tjebnutjer) who lived during the Ptolemaic era, approximately during the 3rd century BC. While some historians maintain that Manetho was from Rome and composed his work c. 200 C.E. [1]

Manetho wrote the Aegyptiaca (History of Egypt). His work is of great interest to Egyptologists, and is often used as evidence for the chronology of the reigns of pharaohs. The earliest and only surviving reference to Manetho's Aegyptiaca is that of the Jewish historian Josephus in his work "Against Apion".}
\begin{greek}

Manetho Astrol., Apotelesmatica (2583: 001)
“Poetae bucolici et didactici”, Ed. Koechly, A.
Paris: Didot, 1862.

Manetho Astrol., Apotelesmatica 
Book 4, line 149

ἢν δὲ Σεληναίης ἑλικοδρόμος ἄστατος ἀστὴρ 
Ἑρμείαν σύμφωνον ἔχῃ κατὰ κόσμου ἀταρπόν, 
καὶ μούνη Κυθέρεια συνῇ καλῷ Φαέθοντι, 
ῥεκτῆρας χρυσοῖο καὶ Ἰνδογενοῦς ἐλέφαντος 
ἐργοπόνους δείκνυσι, καὶ ἐν πραπίδεσσιν ἀρίστους 
ἔσσεσθαι, θριγκῶν τε καὶ εὐτοίχων κανονισμῶν   
κοσμήτας, μάλα τοι πεπονημένα τεχνάζοντας. 



Manetho Astrol., Apotelesmatica 
Book 1, line 297

εἰ δὲ Σεληναίης ἑλικώπιδος ἄστατος ἀστὴρ 
Ἑρμείαν σύμφωνον ἔχοι κατὰ κόσμου ἀταρπόν, 
καὶ μούνη Κυθέρεια συνῇ καλῷ Φαέθοντι, 
ῥεκτῆρας χρυσοῖο καὶ Ἰνδογενοῦς ἐλέφαντος 
ἐργοπόνους ῥέζει καὶ ζωγραφίης μεδέοντας, 
εὐφυέας θριγκῶν τε καὶ εὐτυχέας κανονισμῶν   
κοσμήτας, μάλα τοι πεπονημένα τεχνάζοντας. 

\end{greek}



\section{Eusebius}
\blockquote[From Wikipedia\footnote{\url{http://en.wikipedia.org/wiki/Eusebius}}]{Eusebius (c. AD 263 – 339) (also called Eusebius of Caesarea and Eusebius Pamphili) was a Roman historian, exegete and Christian polemicist. He became the Bishop of Caesarea in Palestine about the year 314. Together with Pamphilus, he was a scholar of the Biblical canon. He wrote Demonstrations of the Gospel, Preparations for the Gospel, and On Discrepancies between the Gospels, studies of the Biblical text. As "Father of Church History" he produced the Ecclesiastical History, On the Life of Pamphilus, the Chronicle and On the Martyrs.}
\begin{greek}
Eusebius Scr. Eccl., Theol., Praeparatio evangelica (2018: 001)
“Eusebius Werke, Band 8: Die Praeparatio evangelica”, Ed. Mras, K.
Berlin: Akademie–Verlag, 43.1:1954; 43.2:1956; Die griechischen christlichen Schriftsteller 43.1 \& 43.2.
Book 2, chapter 1, section 14, line 1

                                                      κτίσαι δὲ καὶ πόλεις οὐκ 
ὀλίγας ἐν Ἰνδοῖς. 



Eusebius Scr. Eccl., Theol., Praeparatio evangelica 
Book 2, chapter 2, section 5, line 1

         στρατεῦσαι δὲ εἰς τὴν Ἰνδικὴν τριετεῖ χρόνῳ. 



Eusebius Scr. Eccl., Theol., Praeparatio evangelica 
Book 6, chapter 10, section 14, line 1

                                                                    παρὰ Ἰνδοῖς 
καὶ Βάκτροις εἰσὶ χιλιάδες πολλαὶ τῶν λεγομένων Βραχμάνων, οἵτινες κατὰ 
παράδοσιν τῶν προγόνων καὶ νόμων οὔτε φονεύουσιν οὔτε ξόανα σέβονται, 
οὐκ ἐμψύχου γεύονται, οὐ μεθύσκονταί ποτε, οἴνου καὶ σίκερος μὴ γευόμενοι, 
οὐ κακίᾳ τινὶ κοινωνοῦσι προσέχοντες τῷ θεῷ, τῶν ἄλλων Ἰνδῶν φονευόν-
των καὶ ἑταιρευόντων καὶ μεθυσκομένων καὶ σεβομένων ξόανα καὶ πάντα 
σχεδὸν καθ' εἱμαρμένην φερομένων. 



Eusebius Scr. Eccl., Theol., Praeparatio evangelica 
Book 6, chapter 10, section 15, line 2

                                      ἔστι δὲ ἐν τῷ αὐτῷ κλίματι τῆς 
Ἰνδίας φυλή τις Ἰνδῶν, οἵτινες τοὺς ἐμπίπτοντας ξένους ἀγρεύοντες καὶ 
θύοντες ἐσθίουσι· καὶ οὔτε οἱ ἀγαθοποιοὶ τῶν ἀστέρων κεκωλύκασι τούτους 
μὴ μιαιφονεῖν καὶ μὴ ἀθεμιτογαμεῖν οὔτε οἱ κακοποιοὶ ἠνάγκασαν τοὺς Βραχ-
μᾶνας κακουργεῖν. 



Eusebius Scr. Eccl., Theol., Praeparatio evangelica 
Book 6, chapter 10, section 33, line 1

                           οἱ Μῆδοι πάντες τοῖς μετὰ σπουδῆς τρεφομένοις 
κυσὶ τοὺς νεκροὺς ἔτι ἐμπνέοντας παραβάλλουσι, καὶ οὐ πάντες σὺν τῇ μήνῃ 
τὸν Ἄρεα ἐφ' ἡμερινῆς γενέσεως ἐν Καρκίνῳ ὑπὸ γῆν ἔχουσιν. Ἰνδοὶ 
τοὺς νεκροὺς καίουσι, μεθ' ὧν συγκαίουσιν ἑκούσας τὰς γυναῖκας, καὶ οὐ 
δήπου πᾶσαι αἱ καιόμεναι ζῶσαι Ἰνδῶν γυναῖκες ἔχουσιν ὑπὸ γῆν ἐπὶ νυκτε-
ρινῆς γενέσεως σὺν Ἄρει τὸν ἥλιον ἐν Λέοντι ὁρίοις Ἄρεος. 



Eusebius Scr. Eccl., Theol., Praeparatio evangelica 
Book 6, chapter 10, section 33, line 3

                                                                            Ἰνδοὶ 
τοὺς νεκροὺς καίουσι, μεθ' ὧν συγκαίουσιν ἑκούσας τὰς γυναῖκας, καὶ οὐ 
δήπου πᾶσαι αἱ καιόμεναι ζῶσαι Ἰνδῶν γυναῖκες ἔχουσιν ὑπὸ γῆν ἐπὶ νυκτε-
ρινῆς γενέσεως σὺν Ἄρει τὸν ἥλιον ἐν Λέοντι ὁρίοις Ἄρεος. 



Eusebius Scr. Eccl., Theol., Praeparatio evangelica 
Book 6, chapter 10, section 35, line 5

        παντὶ ἔθνει καὶ πάσῃ ἡμέρᾳ καὶ παντὶ τ<ρ>όπῳ τῆς γενέσεως γεννῶν-
ται ἄνθρωποι· κρατεῖ δὲ ἐν ἑκάστῃ μοίρᾳ τῶν ἀνθρώπων νόμος καὶ ἔθος διὰ 
τὸ αὐτεξούσιον τοῦ ἀνθρώπου· καὶ οὐκ ἀναγκάζει ἡ γένεσις τοὺς Σῆρας μὴ   
θέλοντας φονεύειν ἢ τοὺς Βραχμᾶνας κρεοφαγεῖν ἢ τοὺς Πέρσας ἀθεμίτως 
μὴ γαμεῖν ἢ τοὺς Ἰνδοὺς μὴ καίεσθαι ἢ τοὺς Μήδους μὴ ἐσθίεσθαι ὑπὸ κυ-
νῶν ἢ τοὺς Πάρθους μὴ πολυγαμεῖν ἢ τὰς ἐν τῇ Μεσοποταμίᾳ γυναῖκας μὴ 
σωφρονεῖν ἢ τοὺς Ἕλληνας μὴ γυμνάζεσθαι γυμνοῖς τοῖς σώμασιν ἢ τοὺς Ῥω-
μαίους μὴ κρατεῖν ἢ τοὺς Γάλλους μὴ γαμεῖσθαι ἢ τὰ ἄλλα βάρβαρα ἔθνη 
ταῖς ὑπὸ τῶν Ἑλλήνων λεγομέναις Μούσαις κοινωνεῖν· ἀλλ', ὡς προεῖπον, 
ἕκαστον ἔθνος καὶ ἕκαστος τῶν ἀνθρώπων χρῆται τῇ ἑαυτοῦ ἐλευθερίᾳ ὡς 
βούλεται καὶ ὅτε βούλεται, καὶ δουλεύει τῇ γενέσει καὶ τῇ φύσει δι' ἣν περί-
κειται σάρκα, πῆ μὲν ὡς βούλεται, πῆ δὲ ὡς μὴ βούλεται. 



Eusebius Scr. Eccl., Theol., Praeparatio evangelica 
Book 6, chapter 10, section 38, line 2

         μνημονεύειν τε ὀφείλετε ὧν προεῖπον, ὅτι καὶ ἐν ἑνὶ κλίματι καὶ 
ἐν μιᾷ χώρᾳ τῶν Ἰνδῶν εἰσιν ἀνθρωποφάγοι Ἰνδοὶ καί εἰσιν οἱ ἐμψύχων 
ἀπεχόμενοι· καὶ ὅτι οἱ Μαγουσαῖοι οὐκ ἐν Περσίδι μόνῃ τὰς θυγατέρας γα-
μοῦσιν, ἀλλὰ καὶ ἐν παντὶ ἔθνει, ὅπου ἂν οἰκήσωσι, τοὺς τῶν προγόνων φυ-
λάσσοντες νόμους καὶ τῶν μυστηρίων αὐτῶν τὰς τελετάς. 



Eusebius Scr. Eccl., Theol., Praeparatio evangelica 
Book 9, chapter 5, section 5,6, line 3

                                                            <Ἐ>κεῖνος τοίνυν   
τὸ μὲν γένος ἦν Ἰουδαῖος, ἐκ τῆς Κοίλης Συρίας, οὗτοι δ' εἰσὶν ἀπόγονοι τῶν 
ἐν Ἰνδοῖς φιλοσόφων· καλοῦνται δέ, ὥς φασιν, οἱ φιλόσοφοι παρὰ μὲν Ἰνδοῖς 
Καλανοί, παρὰ δὲ Σύροις Ἰουδαῖοι, τοὔνομα λαβόντες ἀπὸ τοῦ τόπου. 



Eusebius Scr. Eccl., Theol., Praeparatio evangelica 
Book 9, chapter 6, section 5, line 2

                                            
 Ἔτι πρὸς τούτοις ἑξῆς ὑποβὰς τάδε φησί· 
        “Φανερώτατα δὲ Μεγασθένης ὁ συγγραφεὺς ὁ Σελεύκῳ τῷ Νικάνορι συμ-
βεβιωκὼς ἐν τῇ τρίτῃ τῶν Ἰνδικῶν ὧδε γράφει· ‘Ἅπαντα μέντοι τὰ περὶ φύ-
σεως εἰρημένα παρὰ τοῖς ἀρχαίοις λέγεται καὶ παρὰ τοῖς ἔξω τῆς Ἑλλάδος 
φιλοσοφοῦσι, τὰ μὲν παρ' Ἰνδοῖς ὑπὸ τῶν Βραχμάνων, τὰ δὲ ἐν Συρίᾳ ὑπὸ τῶν 
καλουμένων Ἰουδαίων. 



Eusebius Scr. Eccl., Theol., Praeparatio evangelica 
Book 10, chapter 4, section 15, line 4

                                                              ἀλλὰ γὰρ 
ὁ δηλούμενος τὰ παρ' ἑκάστοις σοφὰ πολυπραγμονῶν ἐπῆλθε Βαβυλῶνα 
καὶ Αἴγυπτον καὶ πᾶσαν τὴν Περσῶν, τοῖς τε μάγοις καὶ τοῖς ἱερεῦσι μαθη-
τευόμενος, ἀκηκοέναι τε πρὸς τούτοις Βραχμάνων ἱστόρηται (Ἰνδῶν δέ 
εἰσιν οὗτοι φιλόσοφοι) καὶ παρ' ὧν μὲν ἀστρολογίαν, παρ' ὧν δὲ γεωμετρίαν 
ἀριθμητικήν τε παρ' ἑτέρων καὶ μουσικὴν καὶ ἄλλα παρ' ἄλλων συλλεξά-
μενος, μόνον παρὰ τῶν σοφῶν Ἑλλήνων ἔσχεν οὐδέν, πενίᾳ σοφίας καὶ 
ἀπορίᾳ συνοικούντων· ἔμπαλιν δ' οὖν τῶν ἔξωθεν αὐτῷ πεπορισμένων 
αἴτιος αὐτὸς τῆς μαθήσεως κατέστη τοῖς Ἕλλησιν. 



Eusebius Scr. Eccl., Theol., Praeparatio evangelica 
Book 10, chapter 9, section 10, line 6

                      τοσαῦτα δὲ ἀπὸ τοῦ δηλωθέντος ἔτους τῆς Κέκροπος βασι-
λείας τὸν ἀνωτέρω χρόνον ἀπαριθμούμενος ἐπὶ Νίνον ἥξεις τὸν Ἀσσύριον, ὃν 
πρῶτόν φασιν ἁπάσης τῆς Ἀσίας πλὴν Ἰνδῶν κεκρατηκέναι· οὗ Νίνος ἐπώνυ-
μος πόλις, ἣ Νινευὴ παρ' Ἑβραίοις ὠνόμασται, καθ' ὃν Ζωροάστρης ὁ μάγος 
Βακτρίων ἐβασίλευσε. 



Eusebius Scr. Eccl., Theol., Praeparatio evangelica 
Book 11, chapter 3, section 8, line 2

                                                  φησὶ δ' Ἀριστόξενος ὁ μουσι-
κὸς Ἰνδῶν εἶναι τὸν λόγον τοῦτον. 



Eusebius Scr. Eccl., Theol., Praeparatio evangelica 
Book 11, chapter 3, section 8, line 4

                                           Ἀθήνησι γὰρ ἐντυχεῖν Σωκράτει τῶν 
ἀνδρῶν ἐκείνων ἕνα τινὰ κἄπειτα αὐτοῦ πυνθάνεσθαι τί ποιῶν φιλοσοφοίη· 
τοῦ δὲ εἰπόντος ὅτι ζητῶν περὶ τοῦ ἀνθρωπείου βίου, καταγελάσαι τὸν Ἰνδόν, 
λέγοντα μὴ δύνασθαί τινα τὰ ἀνθρώπεια κατιδεῖν ἀγνοοῦντά γε τὰ θεῖα. 



Eusebius Scr. Eccl., Theol., Praeparatio evangelica 
Book 13, chapter 3, section 26, line 4

8καὶ ἄλλα τοιαῦτα πολλὰ μὴ ἡμῖν ψευδέσθωσαν· μηδ' αὖ ὑπὸ τούτων 
ἀναπειθόμεναι αἱ μητέρες τὰ παιδία ἐκδειματούντων, λέγουσαι τοὺς μύθους 
κακῶς, ὡς ἄρα θεοί τινες περιέρχονται νύκτωρ πολλοῖς ζῴοις καὶ παντοδαποῖς 
ἰνδαλλόμενοι, ἵνα μὴ ἅμα μὲν εἰς θεοὺς βλασφημῶσιν, ἅμα δὲ τοὺς παῖδας 
ἀπεργάζωνται δειλοτέρους. 



Eusebius Scr. Eccl., Theol., Praeparatio evangelica 
Book 13, chapter 3, section 43, line 5

                                                                                      εἰ 
δέ πη τὸν τοῦ θεοῦ λόγον εἰσάγουσιν ἐν εἴδει καὶ σχήματι ἀνθρωπείῳ παρα-
φαινόμενον, λεκτέον ὡς οὐ κατὰ τοὺς Ἑλλήνων μύθους ὁμοίως Πρωτεῖ 
καὶ Θέτιδι καὶ Ἥρᾳ οὐδ' ὡς οἱ θεοὶ οἱ 8“περιερχόμενοι νύκτωρ πολλοῖς ζῴοις   
καὶ παντοδαποῖς ἰνδαλλόμενοι” καὶ τὸν τοῦ θεοῦ λόγον ἀνθρώποις πεφηνό-
τα εἰσάγουσιν οἱ Ἑβραίων λόγοι, ἀλλ' ὡς αὐτὸς ὁ Πλάτων δεῖν ποτέ φησιν 
ἐπὶ φίλων εὐεργεσίᾳ, 8“ὅταν διὰ μανίαν ἤ τινα ἄνοιαν κακόν τι ἐπιχειρῶσι 
πράττειν, τότε ἀποτροπῆς ἕνεκα ὡς φάρμακον χρήσιμον γενέσθαι” τὴν τοῦ 
θεοῦ εἰς ἀνθρώπους πάροδον. 



Eusebius Scr. Eccl., Theol., Historia ecclesiastica (2018: 002)
“Eusèbe de Césarée. Histoire ecclésiastique, 3 vols.”, Ed. Bardy, G.
Paris: Cerf, 1:1952; 2:1955; 3:1958, Repr. 3:1967; Sources chrétiennes 31, 41, 55.
Book 5, chapter 10, section 2, line 4

   τοσαύτην δ' οὖν φασιν αὐτὸν ἐκθυμοτάτῃ 
διαθέσει προθυμίαν περὶ τὸν θεῖον λόγον ἐνδείξασθαι, ὡς 
καὶ κήρυκα τοῦ κατὰ Χριστὸν εὐαγγελίου τοῖς ἐπ' 
ἀνατολῆς ἔθνεσιν ἀναδειχθῆναι, μέχρι καὶ τῆς Ἰνδῶν 
στειλάμενον γῆς. 



Eusebius Scr. Eccl., Theol., Historia ecclesiastica 
Book 5, chapter 10, section 3, line 2

                   ἦσαν γάρ, ἦσαν εἰς ἔτι τότε πλείους 
εὐαγγελισταὶ τοῦ λόγου, ἔνθεον ζῆλον ἀποστολικοῦ μιμήματος 
συνεισφέρειν ἐπ' αὐξήσει καὶ οἰκοδομῇ τοῦ θείου λόγου   
προμηθούμενοι· 
         ὧν εἷς γενόμενος καὶ ὁ Πάνταινος, 
καὶ εἰς Ἰνδοὺς ἐλθεῖν λέγεται, ἔνθα λόγος εὑρεῖν αὐτὸν 
προφθάσαν τὴν αὐτοῦ παρουσίαν τὸ κατὰ Ματθαῖον 
εὐαγγέλιον παρά τισιν αὐτόθι τὸν Χριστὸν ἐπεγνωκόσιν, 
οἷς Βαρθολομαῖον τῶν ἀποστόλων ἕνα κηρῦξαι αὐτοῖς τε 
Ἑβραίων γράμμασι τὴν τοῦ Ματθαίου καταλεῖψαι γραφήν, 
ἣν καὶ σῴζεσθαι εἰς τὸν δηλούμενον χρόνον. 



Eusebius Scr. Eccl., Theol., De martyribus Palaestinae (recensio brevior) (2018: 003)
“Eusèbe de Césarée. Histoire ecclésiastique, vol. 3”, Ed. Bardy, G.
Paris: Cerf, 1958, Repr. 1967; Sources chrétiennes 55.
Chapter 6, section 2, line 4

ἔθους τὸ πρὶν ὄντος ἐπὶ βασιλέων, εἰ καὶ ἄλλοτε, τὰς 
φιλοτίμους θέας πλείους τοῖς θεαταῖς ἐμπαρέχειν θυμηδίας 
καινῶν καὶ ξένων τά τε συνήθη παραλλαττόντων θεαμάτων, 
ζῴων ἔσθ' ὅπῃ τῶν ἐξ Ἰνδίας ἢ Αἰθιοπίας ἢ καὶ ἄλλοθεν   
εἰσκομιζομένων ἢ καὶ ἀνδρῶν ἐντέχνοις τισὶ σωμασκίαις 
παραδόξους ψυχαγωγίας τοῖς ὁρῶσιν ἐνδεικνυμένων, πάντως 
που καὶ τότε, οἷα βασιλέως τὰς θέας παρέχοντος, πλεῖόν τι 
καὶ παράδοξον χρῆν ὑπάρξαι ταῖς φιλοτιμίαις. 



Eusebius Scr. Eccl., Theol., Demonstratio evangelica (2018: 005)
“Eusebius Werke, Band 6: Die Demonstratio evangelica”, Ed. Heikel, I.A.
Leipzig: Hinrichs, 1913; Die griechischen christlichen Schriftsteller 23.
Book 1, chapter 2, section 13, line 8

                                                                      τῆς 
τε γὰρ κατὰ σάρκα συγγενείας τίς ἦν πρὸς τὸν Ἀβραὰμ συγγένεια 
Σκύθαις, φέρε εἰπεῖν, ἢ Αἰγυπτίοις ἢ Αἰθίοψιν ἢ Ἰνδοῖς ἢ Βρεττα-
νοῖς ἢ Ἱσπανοῖς; 



Eusebius Scr. Eccl., Theol., Demonstratio evangelica 
Book 3, chapter 4, section 45, line 8

   ἔστω γὰρ ἐπὶ τῆς οἰκείας γῆς καλινδουμένους ἀγροί-
κους ἄνδρας πλανᾶν καὶ πλανᾶσθαι, καὶ μὴ ἐφ' ἡσυχίας βάλλεσθαι 
τὸ πρᾶγμα· κηρύττειν δ' εἰς πάντας τὸ τοῦ Ἰησοῦ ὄνομα, καὶ τὰς 
παραδόξους πράξεις αὐτοῦ κατά τε ἀγροὺς καὶ κατὰ πόλιν διδάσκειν, 
καὶ τοὺς μὲν αὐτῶν τὴν Ῥωμαίων ἀρχὴν καὶ αὐτήν τε τὴν βασιλι-
κωτάτην πόλιν νείμασθαι, τοὺς δὲ τὴν Περσῶν, τοὺς δὲ τὴν Ἀρμε-
νίων, ἑτέρους δὲ τὸ Πάρθων ἔθνος, καὶ αὖ πάλιν τὸ Σκυθῶν, τινὰς 
δὲ ἤδη καὶ ἐπ' αὐτὰ τῆς οἰκουμένης ἐλθεῖν τὰ ἄκρα, ἐπί τε τὴν Ἰνδῶν 
φθάσαι χώραν, καὶ ἑτέρους ὑπὲρ τὸν Ὠκεανὸν παρελθεῖν ἐπὶ τὰς 
καλουμένας Βρεττανικὰς νήσους, ταῦτα οὐκ ἔτ' ἔγωγε ἡγοῦμαι κατὰ 
ἄνθρωπον εἶναι, μή τί γε κατὰ εὐτελεῖς καὶ ἰδιώτας, πολλοῦ δεῖ 
κατὰ πλάνους καὶ γόητας. 



Eusebius Scr. Eccl., Theol., Demonstratio evangelica 
Book 3, chapter 7, section 11, line 2

   Πέρσας δὲ καὶ Ἀρμενίους, καὶ 
Χαλδαίους, καὶ Σκύθας, καὶ Ἰνδούς, καὶ εἴ τινα βαρβάρων γένοιτο 
ἔθνη, πῶς πείσομεν τῶν μὲν πατρίων θεῶν ἀφίστασθαι, ἕνα δὲ τὸν 
πάντων δημιουργὸν σέβειν; 



Eusebius Scr. Eccl., Theol., Onomasticon (2018: 011)
“Eusebius Werke, Band 3.1: Das Onomastikon”, Ed. Klostermann, E.
Leipzig: Hinrichs, 1904; Die griechischen christlichen Schriftsteller 11.1.
Page 6, line 19

Αἰλάμ (Gen. 14, 1). ἐν ἐσχάτοις ἐστὶ <Παλαιστίνης> παρακειμένη 
τῇ πρὸς μεσημβρίαν ἐρήμῳ καὶ τῇ πρὸς αὐτὴν ἐρυθρᾷ θαλάσσῃ, 
πλωτῇ οὔσῃ τοῖς τε ἀπ' Αἰγύπτου περῶσι καὶ τοῖς ἀπὸ τῆς Ἰνδικῆς. 



Eusebius Scr. Eccl., Theol., Onomasticon 
Page 80, line 24

                          χώρα πρὸς ἀνατολάς, <ἣν> προϊὼν ἐκ παρα-
δείσου Φισὼν κυκλοῖ, ὁ παρ' Ἕλλησι Γάγγης, «ἐπὶ τὴν Ἰνδικὴν φερό-  
μενος». 



Eusebius Scr. Eccl., Theol., Onomasticon 
Page 82, line 2

         καὶ ἑνὸς δὲ τῶν ἀπογόνων Νῶε ἦν ὄνομα Εὐειλάτ, ὃν σὺν 
τοῖς ἀδελφοῖς Ἰώσιππος «ἀπὸ Κωφῆνος ποταμοῦ τῆς Ἰνδικῆς καὶ 
τῆς πρὸς <αὐτῇ> Σηρίας» κατοικῆσαι ἱστορεῖ. 



Eusebius Scr. Eccl., Theol., Onomasticon 
Page 102, line 1

                                                ἄλλοι δὲ τὴν Ἰνδίαν ὑπε-
τύπωσαν. 



Eusebius Scr. Eccl., Theol., Onomasticon 
Page 124, line 3

Μανασσῆ (Gen 10, 30). χώρα τῆς Ἰνδικῆς, ἣν κατῴκησαν οἱ υἱοὶ 
Ἰεκτὰν υἱοῦ Ἐβέρ. 



Eusebius Scr. Eccl., Theol., Onomasticon 
Page 150, line 14

Σωφειρά (Gen 10, 30). «ὄρος ἀνατολῶν» πρὸς τῇ Ἰνδικῇ, παρ' 
ᾧ κατῴκησαν υἱοὶ Ἰεκτὰν υἱοῦ Ἐβέρ, οὕς φησιν Ἰώσιππος «ἀπὸ 
Κωφῆνος ποταμοῦ τῆς τε Ἰνδικῆς καὶ τῆς πρὸς αὐτῇ Σηρίας» κα-
τασχεῖν. 



Eusebius Scr. Eccl., Theol., Onomasticon 
Page 160, line 20

                                            ὄρος ἀνατολῶν ἐν τῇ Ἰνδικῇ. 



Eusebius Scr. Eccl., Theol., Onomasticon 
Page 166, line 9

                                                  ἔστιν δὲ ποταμὸς ὃν 
Ἕλληνες Γάγγην ὀνομάζουσιν· ἐκ μὲν τοῦ παραδείσου προϊών, ἐπὶ 
δὲ «τὴν Ἰνδικὴν φερόμενος ἐκδίδωσιν εἰς τὸ πέλαγος». 



Eusebius Scr. Eccl., Theol., Onomasticon 
Page 176, line 15

                  ἦν δὲ καὶ ἑνὸς τῶν ἀπογόνων Ἐβὲρ ὄνομα Οὐφείρ, 
οὗ <τοὺς> υἱοὺς «ἀπὸ Κωφῆνος ποταμοῦ τῆς Ἰνδικῆς καὶ τῆς πρὸς 
αὐτῇ Σηρίας» κατοικῆσαι Ἰώσιππος ἱστορεῖ, ἀφ' οὗ καὶ τὴν χώραν 
εἰκότως τῆς προσηγορίας τυχεῖν. 



Eusebius Scr. Eccl., Theol., Contra Hieroclem (2018: 017)
“Flavii Philostrati opera, vol. 1”, Ed. Kayser, C.L.
Leipzig: Teubner, 1870, Repr. 1964.
Page 382, line 9

                                           τὴν ἀπὸ Περ-
σίδος ἐπ' Ἰνδοὺς πορείαν ἄγει παραλαβὼν αὐτὸν ὁ 
λόγος. 



Eusebius Scr. Eccl., Theol., Contra Hieroclem 
Page 382, line 27

                                                       ἐπὶ 
τούτοις ὁ Φιλόστρατος ὁ τἀληθὲς τιμᾶν πρὸς τοῦ 
Φιλαλήθους μεμαρτυρημένος, ὅρα τῆς ἀληθείας ὁποῖα 
δείγματα παρίστησι· γενόμενον παρ' Ἰνδοῖς τὸν 
Ἀπολλώνιον παραστήσασθαί φησιν ἑρμηνέα καὶ δι' 
αὐτοῦ προσδιαλέγεσθαι Φραώτῃ, τοῦτο δ' εἶναι τῷ 
βασιλεῖ τῶν Ἰνδῶν ὄνομα, καὶ ὁ μικρῷ πρόσθεν κατ' 
αὐτὸν πασῶν γλωσσῶν συνεὶς νῦν αὖ κατὰ τὸν αὐ-
τὸν ἑρμηνέως δεῖται. 



Eusebius Scr. Eccl., Theol., Contra Hieroclem 
Page 383, line 6

                         καὶ μεταξὺ ὁ μὲν τῶν Ἰνδῶν 
βασιλεὺς καὶ ταῦτα βάρβαρος ὢν τὴν φύσιν τὸν ἑρ-
μηνέα ἐκποδὼν μεταστησάμενος Ἑλλάδι χρῆται πρὸς 
αὐτὸν τῇ ὁμιλίᾳ παιδείαν καὶ πολυμάθειαν ἐνδει-
κνύμενος, ὁ δὲ οὐδ' ὥς, ὅτι δὴ καὶ αὐτὸς τῆς παρ' 
αὐτοῖς οὐκ ἀμαθῶς ἔχοι φωνῆς δέον ἐπεφιλοτιμή-
σατο, ἀλλὰ καὶ λαλοῦντος Ἑλλάδι γλώσσῃ τοῦ Ἰν-
δοῦ ἐκπλήττεται, ᾗ φησιν ὁ Φιλόστρατος ἀκόλουθα, 
ὡς ἔοικεν, ἑαυτῷ γράφων. 



Eusebius Scr. Eccl., Theol., Contra Hieroclem 
Page 383, line 22

οὐδὲ γὰρ ἐς διδασκάλους γε, οἶμαι, ἀναφέρεις, ἐπεὶ 
μηδὲ εἶναι Ἰνδοῖς (εἰκὸς) διδασκάλους τούτου. 



Eusebius Scr. Eccl., Theol., Contra Hieroclem 
Page 383, line 27

                                     εἶτα καί τισι τοῦ 
Ἰνδοῦ δικάζοντος περὶ θησαυροῦ φωραθέντος ἐν 
ἀγρῷ, πότερα τῷ πριαμένῳ ἢ τῷ τὸ χωρίον ἀποδο-
μένῳ δέοι νεῖμαι τοῦτον, ὁ πάντα φιλόσοφος καὶ 
θεοῖς κεχαρισμένος ἐρωτηθεὶς ἐπικρίνει τῷ πρια-
μένῳ, λογισμὸν δὴ αὐτοῖς ῥήμασιν ἐπειπὼν “ὡς οὐκ 
ἂν οἱ θεοὶ τὸν μὲν ἀφείλοντο καὶ τὴν γῆν, εἰ μὴ   
φαῦλος ἦν, τῷ δ' αὖ καὶ τὰ ὑπὸ γῆν δοῦναι, εἰ μὴ 
βελτίων ἦν τοῦ ἀποδομένου. 



Eusebius Scr. Eccl., Theol., Contra Hieroclem 
Page 387, line 7

                                                   ἐπὶ 
τοιούτῳ δὴ τῷ συμποσίῳ κατὰ τὸν αὐτὸν Φιλόστρα-
τον βασιλεὺς ἐγχωριάζων Ἰνδοῖς εἰσάγεται συμπίνων 
τοῖς φιλοσόφοις, τοῦτον δὲ ἐνυβρίζειν καὶ ἐμπαρ-
οινεῖν φιλοσοφίᾳ μεθύσκεσθαί τε παρ' αὐτοῖς καὶ 
ἀντιπαρεξάγειν Ἡλίῳ καὶ ἀλαζονεύεσθαι ἱστορεῖ, 
καὶ πάλιν τὸν Ἀπολλώνιον δι' ἑρμηνέως τὰ παρὰ 
τούτου μανθάνειν καὶ αὖ πάλιν πρὸς αὐτὸν διαλέ-
γεσθαι ὑφερμηνεύοντος τοῦ Ἰάρχα· καὶ πῶς οὐ θαυ-
μάζειν ἄξιον, ὅπως τὸν οὕτως ὑβριστὴν καὶ ἀτοπώ-
τατον παροινεῖν καὶ μεθύσκεσθαι παρὰ τηλικούτοις 
εἰκὸς ἦν, ὃν οὐδὲ παρεῖναι ἄξιον ἐν φιλοσόφων μή 




Eusebius Scr. Eccl., Theol., Contra Hieroclem 
Page 389, line 12

ταῦτα δὲ νῦν εἰπὼν ὁ τἀληθὲς τιμᾶν παρὰ τῷ Φι-
λαλήθει νενομισμένος μεθ' ἕτερα τῆς γραφῆς, ὡς 
ἂν δὴ γοητείαν τῶν Βραχμάνων καταγνοὺς καὶ ταύ-
της ἐλεύθερον καταστῆσαι τὸν Ἀπολλώνιον φροντί-
σας ἐπιφέρει φάσκων κατὰ λέξιν· “ἰδὼν δὲ παρὰ 
τοῖς Ἰνδοῖς τοὺς τρίποδας καὶ τοὺς οἰνοχόους καὶ 
ὅσα αὐτόματα ἐσφοιτᾶν εἶπον, οὔθ', ὅπως σοφίζοιντο 
αὐτά, ἤρετο, οὔτε ἐδεήθη μαθεῖν, ἀλλ' ἐπῄνει μέν, 
ζηλοῦν δὲ οὐκ ἠξίου. 



Eusebius Scr. Eccl., Theol., Contra Hieroclem 
Page 390, line 2

                                    ἐπανελθόντα φησὶν 
ἀπὸ τῆς Ἰνδῶν χώρας ἐπὶ τὴν Ἑλλάδα κοινωνὸν τῶν 
θεῶν πρὸς αὐτῶν τῶν θεῶν ἀνακεκηρῦχθαι, οἳ καὶ 
τοὺς κάμνοντας ὡς αὐτὸν ἐφ' ὑγείᾳ παρέπεμπον, 
καὶ δῆτα ὡς ἐξ Ἀράβων καὶ μάγων καὶ Ἰνδῶν παρά-
δοξόν τινα καὶ θεῖον ἡμῖν αὐτὸν ἀγαγὼν παραδόξων 
ἐντεῦθεν ἀφηγημάτων κατάρχεται. 



Eusebius Scr. Eccl., Theol., Contra Hieroclem 
Page 390, line 12

                                     καίτοι ἄν τις 
εἴποι εὐλόγως, ὅτι δὴ εἰ θειοτέρας ἢ κατ' ἄνθρωπον 
φύσεως ἦν, πάλαι, ἀλλ' οὐ νῦν ἔδει, πρὸ τῆς δὲ 
ἑτέρων μεταλήψεως τῶν θαυμασίων κατάρχεσθαι, 
περιττὴ δ' ἂν καὶ ἡ ἐξ Ἀράβων αὐτῷ μάγων τε καὶ 
Ἰνδῶν διὰ σπουδῆς ἐπεχειρεῖτο πολυμάθεια, εἰ δή 
τις κατὰ τὴν δοθεῖσαν ὑπῆρξεν ὑπόθεσιν· ἀλλ' οὗ-
τός γε κατὰ τὸν φιλαλήθη συγγραφέα νῦν δὴ πάρεστι 
μετὰ τοσούτους διδασκάλους τὴν σοφίαν ἐνεπιδει-
κνύμενος, καὶ πρῶτα μὲν οἷα ἐξ Ἀράβων καὶ τῆς 
παρ' αὐτοῖς οἰωνιστικῆς ὁρμώμενος τὸν στρουθόν, 
ὅ τι καὶ βούλοιτο τοὺς ἑτέρους ἐπὶ τροφὴν παρακα-
λῶν ἐφερμηνεύει τοῖς παροῦσιν, εἶτα δὲ λοιμοῦ (ἐν 
Ἐφέσῳ) προαισθόμενος προμαντεύεται τοῖς πολίταις. 



Eusebius Scr. Eccl., Theol., Contra Hieroclem 
Page 392, line 24

εἰ μὴ ἄρα, ἐπειδὴ νεκροῖς ὁμιλῶν εἰσῆκται, ἐπὶ τὸ 
ψυχρότερον μεταποιεῖ τὰς πεύσεις ὁ συγγραφεύς, 
ὡς ἂν ὑπεκλύσειε τὴν ὑπόνοιαν τοῦ πέρα τῶν προς-
ηκόντων αὐτὸν περιειργάσθαι δοκεῖν, καὶ γὰρ δὴ 
καὶ ἀπολογούμενον αὐτὸν ὑπογράφει, ὅτι μὴ κατὰ 
νεκρομαντείαν ὁ τρόπος αὐτῷ τῆς φανείσης ὄψεως 
γένοιτο, “οὔτε γὰρ βόθρον” εἶπεν “Ὀδυσσέως ὀρυ-
ξάμενος, οὐδ' ἀρνῶν αἵμασι ψυχαγωγήσας, ἐς διά-
λεξιν τοῦ Ἀχιλλέως ἦλθον, ἀλλ' εὐξάμενος ὁπόσα 
τοῖς ἥρωσιν Ἰνδοί φασι δεῖν εὔξασθαι. 



Eusebius Scr. Eccl., Theol., Contra Hieroclem 
Page 392, line 26

                                                 καὶ ταῦτα 
νῦν πρὸς τὸν ἑταῖρον ἀποσεμνύνεται ὁ μηδὲν μα-
θεῖν παρ' Ἰνδῶν, μηδὲ ζηλῶσαι τὴν παρ' αὐτοῖς σο-
φίαν πρὸς τοῦ συγγραφέως μεμαρτυρημένος. 



Eusebius Scr. Eccl., Theol., Contra Hieroclem 
Page 395, line 6

δαιμονίᾳ κινήσει προέλεγε καὶ ὅτι τοῖς γόητα ἡγου-
μένοις τὸν ἄνδρα οὐχ ὑγιαίνει ὁ λόγος, δηλοῖ μὲν 
καὶ τὰ εἰρημένα, σκεψώμεθα δὲ κἀκεῖνα· οἱ γόητες, 
ἡγοῦμαι δὲ αὐτοὺς ἐγὼ κακοδαιμονεστάτους ἀνθρώ-  
πων, οἱ μὲν ἐς βασάνους εἰδώλων χωροῦντες, οἱ δ' 
ἐς θυσίας βαρβάρους, οἱ δὲ ἐς τὸ ἐπᾷσαί τι ἢ ἀλεῖψαι 
μεταποιεῖν φασι τὰ εἱμαρμένα, ὁ δὲ εἵπετο μὲν τοῖς 
ἐκ Μοιρῶν καὶ προέλεγεν, ὡς ἀνάγκη ἔσεσθαι αὐτά, 
προέλεγε δὲ οὐ γοητεύων, ἀλλ' ἐξ ὧν οἱ θεοὶ ἔφαι-
νον, ἰδὼν δὲ παρὰ τοῖς Ἰνδοῖς τοὺς τρίποδας καὶ 
τοὺς οἰνοχόους καὶ ὅσα αὐτόματα ἐσφοιτᾶν εἶπον, 
οὔθ' ὅπως σοφίζοιντο αὐτὰ ἤρετο, οὔτε ἐδεήθη μα-
θεῖν, ἀλλ' ἐπῄνει μέν, ζηλοῦν δὲ οὐκ ἠξίου. 



Eusebius Scr. Eccl., Theol., Contra Hieroclem 
Page 395, line 10

                                                         ταῦτα 
δὲ λέγων δῆλός ἐστι τοὺς περιβοήτους Ἰνδῶν φιλο-
σόφους γόητας ἀποφαίνων. 



Eusebius Scr. Eccl., Theol., Contra Hieroclem 
Page 395, line 22

                                                    εἰς-
ῆκται δὴ οὖν παρ' οἷς φησι γυμνοῖς Αἰγυπτίων, ῥή-
μασιν αὐτοῖς ταῦτα φάσκων· “οὐκ ἀπεικός τε πα-
θεῖν μοι δοκῶ φιλοσοφίας ἡττηθεὶς εὖ κεκοσμημένης, 
ἣν ἐς τὸ πρόσφορον Ἰνδοὶ στείλαντες ἐφ' ὑψηλῆς τε 
καὶ θείας μηχανῆς ἐκκυκλοῦσιν. 



Eusebius Scr. Eccl., Theol., Contra Hieroclem 
Page 395, line 29

                 καὶ Δομετιανῷ δὲ εἰσῆκται λέγων 
“καὶ τίς πρὸς Ἰάρχαν σοι πόλεμος ἢ πρὸς Φραώτην 
τοὺς Ἰνδούς; 



Eusebius Scr. Eccl., Theol., Contra Hieroclem 
Page 396, line 16

                                                    τοῦτο 
γὰρ αὐτὸς ἑαυτὸν ὁ Ἀπολλώνιος γεγονέναι τὴν ψυ-
χὴν ἐν ταῖς πρὸς τὸν Ἰνδὸν ὁμιλίαις μικρῷ πρόσθεν 
ἡμῖν δεδήλωκε. 



Eusebius Scr. Eccl., Theol., Contra Hieroclem 
Page 407, line 17

δὴ γόητα αὐτὸν ὑπειλήφασιν αὐτὰ δὴ ταῦτα θαυ-
μάζει λέγων Ἐμπεδοκλέα μὲν καὶ Πυθαγόραν καὶ 
Δημόκριτον τοῖς αὐτοῖς μάγοις ὡμιληκότας οὔπω 
ὑπῆχθαι τέχνῃ, Πλάτωνά τε παρὰ τῶν ἐν Αἰγύπτῳ 
ἱερέων τε καὶ προφητῶν πολλὰ παρειληφότα καὶ 
ταῦτα τοῖς ἰδίοις ἀναμίξαντα λόγοις οὐδαμῶς δόξαι 
τισὶ μαγεύειν, τουτονὶ δὲ οὔπω γιγνώσκεσθαι παρ' 
ἀνθρώποις, ὅτι δὴ ἀπὸ τῆς ἀληθινῆς ὁρμῷτο σο-
φίας, μάγον δὲ αὐτὸν πάλαι τε καὶ εἰσέτι νῦν νενο-
μίσθαι τῷ μάγοις Βαβυλωνίων Ἰνδῶν τε Βραχμᾶσι 
καὶ τοῖς Αἰγυπτίων Γυμνοῖς ὡμιληκέναι. 



Eusebius Scr. Eccl., Theol., Contra Hieroclem 
Page 409, line 18

                καὶ ἔμπαλιν τίνι λόγῳ Πυθαγόραν 
σεμνολογῶν θαυμαστὸν ἐπιγράφῃ διδάσκαλον, καὶ 
Μοιρῶν παίγνιον, ἀλλ' οὐκ ἐραστὴν ὄντα φιλοσο-
φίας οὐκ ἀπολείπεις ἐπαινῶν, Φραώτης δὲ καὶ Ἰάρ-
χας (οἱ]1 Ἰνδῶν φιλόσοφοι τί μᾶλλον παρὰ σοὶ θεῶν 
ἀπηνέγκαντο δόξαν μηδέν τι παιδείας ἴδιον, μηδ' 
ἀρετῆς ἀπενεγκάμενοι κλέος; 



Eusebius Scr. Eccl., Theol., Contra Hieroclem 
Page 410, line 2

               τί δὲ καί, εἰ αὐτῷ σοι πέπρωτο θείῳ 
ὄντι τὴν φύσιν ὑπερᾶραι βασιλέων δόξης, εἰς διδα-  
σκάλων ἐφοίτας καὶ φιλοσόφων Ἀραβίους τε καὶ Βα-
βυλωνίων μάγους καὶ σοφοὺς Ἰνδῶν ἐπολυπραγμό-
νεις; 



Eusebius Scr. Eccl., Theol., Contra Hieroclem 
Page 410, line 31

οὔκουν θαυμάσιος οὔτε τῆς πρώτης γενέσεως καὶ 
τροφῆς, οὔτε τῆς ἐγκυκλίου παιδείας, οὔτε τῆς ἐν 
ἀκμῇ σώφρονος ἀγωγῆς, οὔτ' ἀσκήσεως τῆς ἐν φι-
λοσοφίᾳ, ἦν δ' ἄρα τις Μοιρῶν ἀνάγκη καὶ εἰς Βα-
βυλωνίους ἐλαύνουσα, ὠθούμενος δ' ὥσπερ καὶ τοῖς 
Ἰνδῶν ὡμίλεις σοφοῖς, καὶ ἐπὶ τοὺς Αἰγυπτίων δὲ 
Γυμνοὺς οὐχ ἡ προαίρεσις, οὐδ' ὁ φιλοσοφίας πόθος,   
Μοῖρα δὲ ἦγεν ἄγχουσα καὶ ἐπὶ τὰ Γάδειρα καὶ τὰς 
Ἡρακλείους στήλας ἑῷόν τε καὶ ἑσπέριον Ὠκεανὸν 
ἀλᾶσθαι καὶ αὐταῖς ἀτράκτοις εἰς μάτην ἐξεβιάζετο 
περιστρέφεσθαι. 



Eusebius Scr. Eccl., Theol., Commentarius in Isaiam (2018: 019)
“Eusebius Werke, Band 9: Der Jesajakommentar”, Ed. Ziegler, J.
Berlin: Akademie–Verlag, 1975; Die griechischen christlichen Schriftsteller.
Book 1, section 63, line 112

                                          ⌈ἱστοροῦνται γοῦν τινες αὐτῶν μέχρι καὶ 
τῆς Περσῶν καὶ Ἰνδῶν διεληλυθέναι χώρας.⌉ 
 ⌈ἐπεὶ δὲ ἐδόκουν κατὰ τοὺς χρόνους τοῦ προφήτου <Ἰδουμαῖοι> καὶ Ἀμμα-
νῖται καὶ <Μωαβῖται> τὰ περὶ τὴν Ἰουδαίαν Ἀραβικὰ ἔθνη, πολέμια καὶ ἐχθρὰ 
τυγχάνειν τοῦ παρὰ Ἰουδαίοις τιμωμένου θεοῦ διὰ τὴν ἔκτοπον αὐτῶν εἰδωλο-
λατρίαν εἰκότως τούτων αὐτῶν ὀνομαστὶ μνήμην ὁ λόγος ἐποιήσατο, ὡς καὶ 
αὐτῶν παραδεξομένων τὴν ὑπὸ τῶν ἀποστόλων κηρυχθησομένην θεοσέβειαν.⌉ 
διό φησι· <καὶ ἐπὶ τὴν Ἰδουμαίαν καὶ ἐπὶ Μωὰβ τὰς χεῖρας ἐπιβα-
λοῦσι, καὶ τοὺς υἱοὺς δὲ Ἀμμὼν ὑπακούσεσθαι> τῷ κηρύγματι προ-
φητεύει. 



Eusebius Scr. Eccl., Theol., Vita Constantini (2018: 020)
“Eusebius Werke, Band 1.1: Über das Leben des Kaisers Konstantin”, Ed. Winkelmann, F.
Berlin: Akademie–Verlag, 1975; Die griechischen christlichen Schriftsteller.
Book Pin, chapter 4, section 50, line 1

Γάμοι Κωνσταντίου υἱοῦ αὐτοῦ καίσαρος. 
Ἰνδῶν πρεσβεία καὶ δῶρα. 




Eusebius Scr. Eccl., Theol., Vita Constantini 
Book 1, chapter 8, section 4, line 2

                                  ἡμέροις γέ τοι καὶ σώφροσι θεοσεβείας παραγ-
γέλμασι τὸν αὐτοῦ φραξάμενος στρατόν, ἐπῆλθε μὲν τὴν Βρεττανῶν καὶ 
τοὺς ἐν αὐτῷ οἰκοῦντας ὠκεανῷ τῷ κατὰ δύοντα ἥλιον {περιοριζομένῳ}, τό 
τε Σκυθικὸν ἐπηγάγετο πᾶν, ὑπ' αὐτῇ ἄρκτῳ μυρίοις βαρβάρων ἐξαλλάτ-
τουσι γένεσι τεμνόμενον, ἤδη δὲ καὶ μεσημβρίας ἐπ' ἔσχατα τὴν ἀρχὴν 
ἐκτείνας εἰς αὐτοὺς Βλέμμυάς τε καὶ Αἰθίοπας, οὐδὲ τῶν πρὸς ἀνίσχοντα 
ἥλιον ἀλλοτρίαν ἐποιεῖτο τὴν κτῆσιν, ἐπ' αὐτὰ δὲ τὰ τῆς ὅλης οἰκουμένης 
τέρματα, Ἰνδῶν μέχρι τῶν ἐξωτάτω τῶν τε ἐν κύκλῳ περιοίκων τοῦ παντὸς 
τῆς γῆς {τῷ βίῳ} στοιχείου, φωτὸς εὐσεβείας ἀκτῖσιν ἐκλάμπων, ἅπαντας 
εἶχεν ὑπηκόους, τοπάρχας ἐθνάρχας σατράπας βασιλέας παντοίων βαρβάρων 
ἐθνῶν ἐθελοντὶ ἀσπαζομένους καὶ χαίροντας τοῖς τε παρ' αὐτῶν ξενίοις τε   
καὶ δώροις διαπρεσβευομένους καὶ τὴν πρὸς αὐτὸν γνῶσίν τε καὶ φιλίαν 
περὶ πλείστου ποιουμένους, ὥστε καὶ γραφαῖς εἰκόνων αὐτὸν παρ' αὐτοῖς 
τιμᾶν ἀνδριάντων τε ἀναθήμασι, μόνον τε αὐτοκρατόρων παρὰ τοῖς πᾶσι 
Κωνσταντῖνον γνωρίζεσθαί τε καὶ βοᾶσθαι. 



Eusebius Scr. Eccl., Theol., Vita Constantini 
Book 4, chapter 7, section 1, line 9

Συνεχεῖς γοῦν ἁπανταχόθεν οἱ διαπρεσβευόμενοι δῶρα τὰ παρ' αὐτοῖς πο-
λυτελῆ διεκόμιζον, ὡς καὶ αὐτούς ποτε παρατυχόντας ἡμᾶς πρὸ τῆς αὐλείου 
τῶν βασιλείων πυλῶν στοιχηδὸν ἐν τάξει περίβλεπτα σχήματα βαρβάρων 
ἑστῶτα θεάσασθαι, οἷς ἔξαλλος μὲν ἡ στολή, διαλλάττων δ' ὁ τῶν σχημάτων 
τρόπος, κόμη τε κεφαλῆς καὶ γενείου πάμπολυ διεστῶσα, βλοσυρῶν τε ἦν 
προσώπων βάρβαρος καὶ καταπληκτική τις ὄψις, σωμάτων θ' ἡλικίας ὑπερ-
βάλλοντα μεγέθη· καὶ οἷς μὲν ἐρυθραίνετο τὰ πρόσωπα, οἷς δὲ λευκότερα 
χιόνος ἦν, οἷς δ' ἐβένου καὶ πίττης μελάντερα, οἱ δὲ μέσης μετεῖχον κράσεως, 
ἐπεὶ καὶ Βλεμμύων γένη Ἰνδῶν τε καὶ Αἰθιόπων, οἳ διχθὰ δεδαίαται ἔσχατοι 
ἀνδρῶν, τῇ τῶν εἰρημένων ἐθεωρεῖτο ἱστορίᾳ. 



Eusebius Scr. Eccl., Theol., Vita Constantini 
Book 4, chapter 50, section 1, line 1

Ἐν τούτῳ δὲ καὶ Ἰνδῶν τῶν πρὸς ἀνίσχοντα ἥλιον πρέσβεις ἀπήντων 
δῶρα κομίζοντες, γένη δ' ἦν παντοῖα ἐξαστραπτόντων πολυτελῶν λίθων ζῷά 
τε τῶν παρ' ἡμῖν ἐγνωσμένων ἐναλλάττοντα τὴν φύσιν, ἃ δὴ προσῆγον τῷ 
βασιλεῖ, τὴν εἰς αὐτὸν Ὠκεανὸν δηλοῦντες αὐτοῦ κράτησιν, καὶ ὡς οἱ τῆς   
Ἰνδῶν χώρας καθηγεμόνες εἰκόνων γραφαῖς ἀνδριάντων τ' αὐτὸν ἀναθήμασι 
τιμῶντες αὐτοκράτορα καὶ βασιλέα γνωρίζειν ὡμολόγουν. 



Eusebius Scr. Eccl., Theol., Vita Constantini 
Book 4, chapter 50, section 1, line 8

                                                              ἀρχομένῳ μὲν οὖν 
τῆς βασιλείας αὐτῷ οἱ πρὸς ἥλιον δύοντα <ἐν> Ὠκεανῷ Βρεττανοὶ πρῶτοι 
καθυπετάττοντο, νῦν δ' Ἰνδῶν οἱ τὴν πρὸς ἀνίσχοντα ἥλιον λαχόντες. 



Eusebius Scr. Eccl., Theol., Constantini imperatoris oratio ad coetum sanctorum (2018: 021)
“Eusebius Werke, Band 1: Über das Leben Constantins. Constantins Rede an die heilige Versammlung. Tricennatsrede an Constantin”, Ed. Heikel, I.A.
Leipzig: Hinrichs, 1902; Die griechischen christlichen Schriftsteller 7.
Chapter 16, section 1, line 12

είρητο δὲ καὶ ὁ καιρὸς τῆς ἐνσωματώσεως, φανερὰ δ' ἦν καὶ ἡ αἰτία 
τῆς σαρκώσεως αὐτοῦ, ὅπως τὰ ἐκ τῆς ἀδικίας τε καὶ ἀκολασίας 
ἐκφύοντα γεννήματα, λυμαινόμενα τοῖς δικαίοις ἔργοις καὶ τρόποις, 
ἀναιρεθείη, πᾶσα δὲ ἡ οἰκουμένη φρονήσεώς τε καὶ σωφροσύνης μετ-
άσχοι, ἐπικρατήσαντος σχεδὸν ἐν ταῖς πάντων ψυχαῖς τοῦ θεσπις-
θέντος ὑπὸ τοῦ σωτῆρος νόμου, καὶ θεοσεβείας μὲν ῥωσθείσης, δεισι-
δαιμονίας δὲ ἐξαλειφθείσης, δι' ἣν οὐ μόνον ἀλόγων ζώων σφαγαί, 
ἀλλὰ καὶ ἀνθρωπίνων ἱερευμάτων θυσίαι καὶ ἐναγῆ μιάσματα βωμῶν 
ἐπενοήθη, κατά τε Αἰγυπτίους καὶ Ἀσσυρίους νόμους χαλκηλάτοις 
ἢ καὶ πλαστοῖς ἰνδάλμασιν σφαγιαζόντων ψυχὰς δικαίας. 



Eusebius Scr. Eccl., Theol., De laudibus Constantini (2018: 022)
“Eusebius Werke, vol. 1”, Ed. Heikel, I.A.
Leipzig: Hinrichs, 1902; Die griechischen christlichen Schriftsteller 7.
Chapter 6, section 21, line 8

   θνητῶν δ' ὀφθαλμὸς οὐκ εἶδεν, οὐδὲ ἀκοή τις διέγνω, 
ἀλλ' οὐδὲ νοῦς σάρκα ἠμφιεσμένος οἷός τε ἂν εἴη διαθρῆσαι ἃ τοῖς   
εὐσεβείᾳ διακοσμησαμένοις προητοίμασται, ὥσπερ οὖν καὶ σοί, βασι-
λεῦ θεοσεβέστατε, ᾧ μόνῳ τῶν ἐξ αἰῶνος ἐντεῦθεν ἤδη τὸν ἀνθρώ-
πειον ἀποκαθᾶραι βίον αὐτὸς ὁ τῶν ὅλων παμβασιλεὺς θεὸς ἐδωρή-
σατο, ᾧ καὶ τὸ αὐτοῦ σωτήριον ἀνέδειξε σημεῖον, δι' οὗ τὸν θάνα-
τον καταγωνισάμενος τὸν κατὰ τῶν ἐχθρῶν ἤγειρε θρίαμβον· ὃ δὴ 
νικητικὸν τρόπαιον, δαιμόνων ἀποτρόπαιον, τοῖς τῆς πλάνης ἰνδάλ-
μασιν ἀντιπαρατάξας τὰς κατὰ πάντων ἀθέων πολεμίων τε καὶ 
βαρβάρων ἤδη δὲ καὶ αὐτῶν δαιμόνων, ἄλλων τουτωνὶ βαρβάρων, 
ἤρατο νίκας. 



Eusebius Scr. Eccl., Theol., Generalis elementaria introductio (= Eclogae propheticae) (2018: 023)
“Eusebii Pamphili episcopi Caesariensis eclogae propheticae”, Ed. Gaisford, T.
Oxford: Oxford University Press, 1842.
Page 111, line 12

Διότι ἐγώ εἰμι ὡς πάνθηρ τῷ Ἐφραὶμ, καὶ ὡς λέων 
τῷ οἴκῳ Ἰούδα. Ἐπεὶ καὶ ἐν ἑτέρῳ τόπῳ τοῦ αὐτοῦ 
προφήτου ὁ Κύριος περὶ ἑαυτοῦ φησιν τὸ, καὶ ἔσομαι 
ὡς πάνθηρ, ἀναγκαῖον ἰδεῖν τὰ ἱστορούμενα [ἃ] περὶ 
τοῦ ζῴου· καὶ δὴ ἀπὸ τοῦ πρώτου τῶν Διδύμου 
Φυσικῶν ταῦτα παραθετέον· 
 Πάνθηρ τὸ ζῷον οὐ μόνον ἐστὶ κατὰ τὸ σῶμα εὔμορ-
φον, καθάπερ ἀστερωπὸς, ἀλλ' ἐπεὶ πέφυκεν εὔπνους 
ὑπερβάλλει καὶ τῶν παρ' Ἰνδοῖς ἀρωμάτων ἐν εὐωδίᾳ· 
οὗτος ἕως οὗ οὐ πεπείνηκεν ἐν τῇ καταδύσει μένει, θέλων 
οἰκουρός τις εἶναι· ἐπ' ἂν δὲ τροφῆς ἐπιθυμήσῃ μεταλαβεῖν, 
προελθὼν βαδίζει μόνον· τὰ δ' ἄλλα θηρία ἁλισκόμενα 
ὑπὸ τῆς εὐωδίας αὐτοῦ τῆς περὶ τὸ σῶμα ἀκολουθεῖ κη-
λούμενα· ὁ δὲ πλάγιος τοῖς ὀφθαλμοῖς τὸ ἐπιτήδειον 
αὐτῷ θηρίον αἱρεθῆναι περιβλέπει καὶ ἐπιπηδήσας ἔχει. 
 Τὰ μὲν δὴ περὶ τῆς φύσεως τοῦ ζῴου τοιαῦτα· ὅπως 
δ' ἐπὶ τὴν τοῦ θείου λόγου δύναμιν καὶ τὴν ἐξ αὐτῆς 




Eusebius Scr. Eccl., Theol., Commentaria in Psalmos (2018: 034); MPG 23–24.
Volume 23, page 1101, line 11

             Πῶς δὲ ἀληθεύσει περὶ αὐτοῦ φάσκων ὁ 
Θεὸς, Θήσομαι αὐτὸν ὑψηλὸν παρὰ τοῖς βασι-
λεῦσι τῆς γῆς; Πότ' οὖν ὁ Δαυῒδ παρὰ Πέρσαις 
φέρε, ἢ Σκύθαις, ἢ Ἰνδοῖς, ἢ Αἰθίοψιν, ἢ Μαύροις, ἢ Σπάνοις, ἢ Βρετανοῖς, ἢ παρὰ τοῖς λοιποῖς τῶν 
ἐθνῶν βασιλεῦσιν ὑψώθη; 

\end{greek}


\section{Pseudo-Galenus}
\blockquote[From Wikipedia\footnote{\url{}}]{}
\begin{greek}

Pseudo-Galenus Med., An animal sit quod est in utero (0530: 002)
“Galeni qui fertur libellus Εἰ ζῷον τὸ κατὰ γαστρός”, Ed. Wagner, H., 1914; Diss. Marburg.

Pseudo-Galenus Med., Λέξεις βοτανῶν (0530: 003)
“Anecdota Atheniensia et alia, vol. 2”, Ed. Delatte, A.
Paris: Droz, 1939.
Page 390, line 1

                     καρυόφυλλον τὸ ἐκ τῆς Ἰνδίας κομιζό-
μενον. 



Pseudo-Galenus Med., Λέξεις βοτανῶν 
Page 390, line 19

λιμνεία σφραγὶς πηλὸς λίμνης ἰνδικῆς. 



Pseudo-Galenus Med., Λέξεις βοτανῶν 
Page 390, line 24

                          μαλάβαθρον ἤτοι φύλλον ἰνδικόν. 



Pseudo-Galenus Med., Λέξεις βοτανῶν 
Page 390, line 28

                                                νάρδου ἰνδι-
κοῦ· ἤτοι μαλάβαθρον. 



Pseudo-Galenus Med., Λέξεις βοτανῶν 
Page 392, line 21

                                   φύλλου ἰνδικοῦ ἤτοι μα-
λαβάθρου φύλλων. 



Pseudo-Galenus Med., Introductio seu medicus (0530: 012)
“Claudii Galeni opera omnia, vol. 14”, Ed. Kühn, C.G.
Leipzig: Knobloch, 1827, Repr. 1965.
Volume 14, page 760, line 6

                       προτρεπτικὰ δὲ ὅσα κινεῖ καὶ προ-
τρέπει τὰ φάρμακα, ὡς μὲν ἀγαρικὸν, ἔρις, ῥέον δὲ Ἰν-
δικὸν, στάχυς· ἐντεριώνην δὲ κολοκυνθίδος, τραγάκανθα· 
σκαμμωνίαν δὲ, ἀλόη, ἢ χυλὸς κράμβης, ἢ ῥόδων, ἢ σπέρ-
ματα. 



Pseudo-Galenus Med., De remediis parabilibus libri iii (0530: 029)
“Claudii Galeni opera omnia, vol. 14”, Ed. Kühn, C.G.
Leipzig: Knobloch, 1827, Repr. 1965.
Volume 14, page 429, line 17

                                                           ἀλκυό-
νιον Ἰνδικὸν ἐν οἴνῳ τριπτὸν διακλυζέσθω καὶ ἵστησι τὴν 
κίνησιν. 



Pseudo-Galenus Med., De succedaneis liber (0530: 031)
“Claudii Galeni opera omnia, vol. 19”, Ed. Kühn, C.G.
Leipzig: Knobloch, 1830, Repr. 1965.
Volume 19, page 724, line 11

ἀντὶ ἀλόης Ἰνδικῆς, ἀλόης χλωρᾶς φύλλα, γλαύκιον, λύκιον 
 ἢ κενταύριον. 



Pseudo-Galenus Med., De succedaneis liber 
Volume 19, page 725, line 9

ἀντὶ Ἀρμενίου, μέλαν Ἰνδικόν. 



Pseudo-Galenus Med., De succedaneis liber 
Volume 19, page 733, line 10

ἀντὶ κροκομάγματος, ἀλόη Ἰνδικὴ ἢ ἀγάλλοχον Ἰνδικόν. 



Pseudo-Galenus Med., De succedaneis liber 
Volume 19, page 735, line 16

ἀντὶ μαλαβάθρου, κασσία σφαιρίτης ἢ ναρδοστάχυς ἢ τρά-
 κτυλος ἢ νάρδος Ἰνδική. 



Pseudo-Galenus Med., De succedaneis liber 
Volume 19, page 737, line 9

ἀντὶ νάρδου ἀγρίας, νάρδος Ἰνδική. 



Pseudo-Galenus Med., De succedaneis liber 
Volume 19, page 737, line 10

ἀντὶ νάρδου Ἰνδικῆς, νάρδος Κελτική. 



Pseudo-Galenus Med., De theriaca ad Pamphilianum (0530: 032)
“Claudii Galeni opera omnia, vol. 14”, Ed. Kühn, C.G.
Leipzig: Knobloch, 1827, Repr. 1965.
Volume 14, page 306, line 14

νάρδου Ἰνδικῆς, μαλαβάθρου φύλλων, ἀνὰ 𐅻 * ιστʹ, σμύρ-
νης, κρόκου, ἀνὰ 𐅻 ιβʹ. 



Pseudo-Galenus Med., De theriaca ad Pamphilianum 
Volume 14, page 308, line 12

δικτάμνου Κρητικοῦ, σχοίνου ἄνθους, λιβάνου ἄῤῥενος, τερ-
μινθίνης Χίας, κασσίας σύριγγος μελαίνης, νάρδου Ἰνδικῆς, 
ἀνὰ 𐅻 στʹ. 



Pseudo-Galenus Med., De historia philosophica (0530: 042)
“Doxographi Graeci”, Ed. Diels, H.
Berlin: Reimer, 1879, Repr. 1965.
Section 3, line 44

                                   ὅσπερ <Θεόφραστον> προεστήσατο τῆς κατ'   
αὐτὸν αἱρέσεως καὶ τὸν <Στράτωνα> προήγαγεν εἰς ἴδιόν τινα χαρακτῆρα 
φυσιολογίας *** 
 [εἰσὶ δὲ τῶν γεννικωτέρων φιλοσόφων ἴνδικες δύο, τινὲς μὲν Ἰταλιῶται, 
ὧν <Πυθαγόρας> εὑρετὴς γεγένηται, καὶ ὁ κατὰ τὴν Ἐλαίαν ἀκμάσας] 
*** ταύτης δὲ λέγεται κατάρξαι <Ξενοφάνης> ὁ Κολοφώνιος ἀπορητικῆς 
μᾶλλον ἢ δογματικῆς τοῖς πολλοῖς εἶναι δοκούσης. 



Pseudo-Galenus Med., De optima secta ad Thrasybulum liber (0530: 043)
“Claudii Galeni opera omnia, vol. 1”, Ed. Kühn, C.G.
Leipzig: Knobloch, 1821, Repr. 1964.
Volume 1, page 178, line 10

                                                ἡμῖν μὲν γὰρ παρὰ 
φύσιν, τοῖς Ἰνδοῖς δὲ κατὰ φύσιν. 

\end{greek}


\section{Theophilus of Antioch}
\blockquote[From Wikipedia\footnote{\url{http://en.wikipedia.org/wiki/Theophilus_of_Antioch}}]{Theophilus, Patriarch of Antioch,[1] succeeded Eros c. 169, and was succeeded by Maximus I c. 183, according to Henry Fynes Clinton,[2] but these dates are only approximations. His death probably occurred between 183 and 185.[3]

We gather from his writings (the only remaining being his apology to Autolycus) that he was born a pagan, not far from the Tigris and Euphrates, and was led to embrace Christianity by studying the Holy Scriptures, especially the prophetical books.[4] He makes no reference to his office in his existing writings, nor is any other fact in his life recorded. Eusebius, however, speaks of the zeal which he and the other chief shepherds displayed in driving away the heretics who were attacking Christ's flock, with special mention of his work against Marcion.[5] He made contributions to the departments of Christian literature, polemics, exegetics, and apologetics. William Sanday[6] describes him as "one of the precursors of that group of writers who, from Irenaeus to Cyprian, not only break the obscurity which rests on the earliest history of the Church, but alike in the East and in the West carry it to the front in literary eminence, and distance all their heathen contemporaries".

he one undoubted extant work of Theophilus, the 7th Bishop of Antioch (c. 169–c. 183), is his Apology to Autolycus (Apologia ad Autolycum), a series of books defending Christianity written to a pagan friend.}
\begin{greek}


Theophilus Apol., Ad Autolycum (1725: 001)
“Theophilus of Antioch. Ad Autolycum”, Ed. Grant, R.M.
Oxford: Clarendon Press, 1970.
Book 3, section 5, line 10

               ἔτι δὲ καὶ παρὰ Ἰνδοῖς μυθεύει κατεσθίεσθαι τοὺς 
πατέρας ὑπὸ τῶν ἰδίων τέκνων. 

\end{greek}


\section{Basil of Caesarea}%???
\blockquote[From Wikipedia\footnote{\url{http://en.wikipedia.org/wiki/Basil_of_Caesarea}}]{Basil of Caesarea, also called Saint Basil the Great, (329 or 330[5] – January 1, 379) (Greek: Ἅγιος Βασίλειος ὁ Μέγας) was the Greek bishop of Caesarea Mazaca in Cappadocia, Asia Minor (modern-day Turkey). He was an influential theologian who supported the Nicene Creed and opposed the heresies of the early Christian church, fighting against both Arianism and the followers of Apollinaris of Laodicea. His ability to balance his theological convictions with his political connections made Basil a powerful advocate for the Nicene position.

In addition to his work as a theologian, Basil was known for his care of the poor and underprivileged. Basil established guidelines for monastic life which focus on community life, liturgical prayer, and manual labour. Together with Pachomius he is remembered as a father of communal monasticism in Eastern Christianity. He is considered a saint by the traditions of both Eastern and Western Christianity.}
\begin{greek}
Basilius Scr. Eccl., Homilia in pentecosten (2800: 012); MPG 52.
Volume 52, page 811, line 54

       Ὢ τῶν παραδόξων θαυμάτων! ἀπόστολος ἐλά-
λει, καὶ Ἰνδὸς ἐδιδάσκετο· Ἑβραῖος ἐφθέγγετο, καὶ 
βάρβαρος ἐπαιδεύετο· ἡ χάρις ἐξηχεῖτο, καὶ ἀκοὴ 
τὸν λόγον ἐδέχετο· Γότθοι τὴν φωνὴν ἐπεγίνωσκον,    
καὶ Αἰθίοπες τὴν γλῶτταν ἐγνώριζον· Πέρσαι τοῦ 
λαλοῦντος ἐθαύμαζον, καὶ ἔθνη βάρβαρα ὑπὸ μιᾶς 
ἠρδεύετο γλώττης. 



Basilius Scr. Eccl., De vita et miraculis sanctae Theclae libri ii [Sp.] 
Book 1, section 22, line 22

                                                                Καὶ 
εἴ τις ἔγνω τὸν ἐκ Τυανέων τῶν Καππαδοκῶν Ἀπολλώνιον 
παρὰ τοῖς τὸν ἐκείνου βίον ἀναγεγραφόσιν – ἵνα ἐκ τῶν πάνυ 
πολλῶν τὸ περιφανέστερον εἴπωμεν – , ἔγνω που πάντως καὶ 
τῆς κατὰ τὴν γοητείαν τοῦ ἀνδρὸς τέχνης τὰ μιαρὰ καὶ δυσαγῆ 
ἀποτελέσματα, θεαγωγίας τέ τινας καὶ ψυχαγωγίας καὶ δαιμόνων 
ἐπικλήσεις καὶ λανθανούσας ἀνοσιουργίας· ὡς καὶ παρὰ τῶν ἐν 
Αἰθίοψι καὶ Ἰνδοῖς Γυμνοσοφιστῶν μήτε εἰσδεχθῆναι σπουδαίως, 
ἀλλὰ γὰρ καὶ θᾶττον ἀποπεμφθῆναι, ὡς οὐκ εὐαγὴς οὐδὲ ὅσιος 
ἄνθρωπος, οὐδὲ φιλόσοφος ἀληθῶς, πολὺ δὲ τοῦ κατὰ τὴν γοη-
τείαν μιάσματος ἔχων. 
\end{greek}


\section{<Perictione>}
\blockquote[From Wikipedia\footnote{\url{http://en.wikipedia.org/wiki/Perictione}}]{Perictione or Periktione (Greek: Περικτιόνη; 5th century BC) was the mother of the Greek philosopher Plato.



Two spurious works attributed to Perictione have survived in fragments. These are On the Harmony of Women and On Wisdom. The works do not date from the same time, and are usually assigned to a Perictione I and a Perictione II.[4] Both works belong to the pseudonymous Pythagorean literature. On the Harmony of Women, concerns the duties of a woman to her husband, her marriage, and to her parents; it is written in Ionic Greek, and probably dates to the late 4th or 3rd century BC.[5] On Wisdom offers a philosophical definition of wisdom; it is written in Doric Greek, and probably dates to the 3rd or 2nd century BC.[5]}
\begin{greek}

<Perictione> Phil., Fragmenta (1572: 001)
“The Pythagorean texts of the Hellenistic period”, Ed. Thesleff, H.
Åbo: Åbo Akademi, 1965.
Page 143, line 22

         ὥστ' οὔτε χρυσὸν ἀμφιθήσεται ἢ λίθον Ἰνδικὸν ἢ χώρης ἐόντα 
ἄλλης, οὐδὲ πλέξεται πολυτεχνίῃσι τρίχας, οὐδ' ἀλείψεται Ἀραβίης 
ὀδμῆς ἐμπνέοντα, οὐδὲ χρίσεται πρόσωπον λευκαίνουσα ἢ ἐρυθραί-
νουσα τοῦτο ἢ μελαίνουσα ὀφρύας τε καὶ ὀφθαλμοὺς καὶ τὴν πολιὴν 
τρίχα βαφαῖσι τεχνεωμένη, οὐδὲ λούσεται θαμινά. 

\end{greek}



\section{Polybius}
\blockquote[From Wikipedia\footnote{\url{}}]{}
\begin{greek}

Polybius Hist., Historiae (0543: 001)
“Polybii historiae, vols. 1–4”, Ed. Büttner–Wobst, T.
Leipzig: Teubner, 1:1905; 2:1889; 3:1893; 4:1904, Repr. 1:1962; 2–3:1965; 4:1967.
Book 1, chapter 40, section 15, line 2

                                  θηρία δὲ σὺν αὐ-
τοῖς μὲν Ἰνδοῖς ἔλαβε δέκα, τῶν δὲ λοιπῶν τοὺς 
Ἰνδοὺς ἀπερριφότων μετὰ τὴν μάχην περιελασάμε-
νος ἐκυρίευσε πάντων. 



Polybius Hist., Historiae 
Book 3, chapter 46, section 7, line 1

              τῶν δὲ θηρίων εἰθισμένων τοῖς Ἰνδοῖς 
μέχρι μὲν πρὸς τὸ ὑγρὸν ἀεὶ πειθαρχεῖν, εἰς δὲ τὸ 
ὕδωρ ἐμβαίνειν οὐδαμῶς ἔτι τολμώντων, ἦγον διὰ 
τοῦ χώματος δύο προθέμενοι θηλείας, πειθαρχούν-
των αὐταῖς τῶν θηρίων. 



Polybius Hist., Historiae 
Book 3, chapter 46, section 11, line 3

καὶ τοιούτῳ δὴ τρόπῳ προσαρμοζομένων ἀεὶ σχε-
διῶν δυεῖν, τὰ πλεῖστα τῶν θηρίων ἐπὶ τούτων 
διεκομίσθη, τινὰ δὲ κατὰ μέσον τὸν πόρον ἀπέρ-
ριψεν εἰς τὸν ποταμὸν αὑτὰ διὰ τὸν φόβον· ὧν 
τοὺς μὲν Ἰνδοὺς ἀπολέσθαι συνέβη πάντας, τοὺς 
δ' ἐλέφαντας διασωθῆναι. 



Polybius Hist., Historiae 
Book 5, chapter 84, section 6, line 4

τὰ δὲ πλεῖστα τῶν τοῦ Πτολεμαίου θηρίων ἀπεδει-
λία τὴν μάχην, ὅπερ ἔθος ἐστὶ ποιεῖν τοῖς Λιβυκοῖς 
ἐλέφασι· τὴν γὰρ ὀσμὴν καὶ φωνὴν οὐ μένουσιν, 
ἀλλὰ καὶ καταπεπληγμένοι τὸ μέγεθος καὶ τὴν δύ-
ναμιν, ὥς γ' ἐμοὶ δοκεῖ, φεύγουσιν εὐθέως ἐξ ἀπο-
στήματος τοὺς Ἰνδικοὺς ἐλέφαντας· ὃ καὶ τότε συνέβη 
γενέσθαι. 



Polybius Hist., Historiae 
Book 11, chapter 1, section 12, line 3

              τῶν δὲ θηρίων τὰ μὲν ἓξ ἅμα τοῖς   
ἀνδράσιν ἔπεσε, τὰ δὲ τέτταρα διωσάμενα τὰς τάξεις 
ὕστερον ἑάλω μεμονωμένα καὶ ψιλὰ τῶν Ἰνδῶν. 



Polybius Hist., Historiae 
Book 11, chapter 34, section 11, line 2

                               ὑπερβαλὼν δὲ τὸν Καύ-
κασον καὶ κατάρας εἰς τὴν Ἰνδικήν, τήν τε φιλίαν 
ἀνενεώσατο τὴν πρὸς τὸν Σοφαγασῆνον τὸν βασιλέα 
τῶν Ἰνδῶν, καὶ λαβὼν ἐλέφαντας, ὥστε γενέσθαι 
τοὺς ἅπαντας εἰς ἑκατὸν καὶ πεντήκοντ', ἔτι δὲ 
σιτομετρήσας πάλιν ἐνταῦθα τὴν δύναμιν, αὐτὸς 
μὲν ἀνέζευξε μετὰ τῆς στρατιᾶς, Ἀνδροσθένην δὲ 
τὸν Κυζικηνὸν ἐπὶ τῆς ἀνακομιδῆς ἀπέλιπε τῆς 
γάζης τῆς ὁμολογηθείσης αὐτῷ παρὰ τοῦ βασιλέως. 



Polybius Hist., Historiae 
Book 34, chapter 13, section 1, line 1

       δε ασια.


 Τὰ δ' ἐπ' εὐθείας τούτοις μέχρι τῆς Ἰνδικῆς τὰ αὐτὰ 
κεῖται καὶ παρὰ τῷ Ἀρτεμιδώρῳ, ἅπερ καὶ παρὰ τῷ 
Ἐρατοσθένει. 

\end{greek}



\section{Diogenes Laertius}
\blockquote[From Wikipedia\footnote{\url{http://en.wikipedia.org/wiki/Diogenes_Laertius}}]{Diogenes Laertius (play /daɪˈɒdʒɨˌniːz leɪˈɜrʃəs/; Greek: Διογένης Λαέρτιος, Diogenēs Laertios; fl. c. 3rd century AD) was a biographer of the Greek philosophers. Nothing is known about his life, but his surviving Lives and Opinions of Eminent Philosophers is one of the principal surviving sources for the history of Greek philosophy.}
\begin{greek}

Diogenes Laertius Biogr., Vitae philosophorum (0004: 001)
“Diogenis Laertii vitae philosophorum, 2 vols.”, Ed. Long, H.S.
Oxford: Clarendon Press, 1964, Repr. 1966.
Book 1, section 1, line 3

γεγενῆσθαι γὰρ παρὰ μὲν Πέρσαις Μάγους, παρὰ δὲ Βαβυλωνίοις 
ἢ Ἀσσυρίοις Χαλδαίους, καὶ γυμνοσοφιστὰς παρ' Ἰνδοῖς, παρά 
τε Κελτοῖς καὶ Γαλάταις τοὺς καλουμένους Δρυΐδας καὶ Σεμνο-
θέους, καθά φησιν Ἀριστοτέλης ἐν τῷ Μαγικῷ (Rose 35) καὶ 
Σωτίων ἐν τῷ εἰκοστῷ τρίτῳ τῆς Διαδοχῆς. 



Diogenes Laertius Biogr., Vitae philosophorum 
Book 9, section 35, line 8

                                                                   τοῖς 
τε γυμνοσοφισταῖς φασί τινες συμμῖξαι αὐτὸν ἐν Ἰνδίᾳ καὶ εἰς 
Αἰθιοπίαν ἐλθεῖν. 



Diogenes Laertius Biogr., Vitae philosophorum 
Book 9, section 61, line 5

          
<ΠΥΡΡΩΝ>


 Πύρρων Ἠλεῖος Πλειστάρχου μὲν ἦν υἱός, καθὰ καὶ Διοκλῆς 
ἱστορεῖ· ὥς φησι δ' Ἀπολλόδωρος ἐν Χρονικοῖς (FGrH 244 F 39), 
πρότερον ἦν ζωγράφος, καὶ ἤκουσε Βρύσωνος τοῦ Στίλπωνος, ὡς 
Ἀλέξανδρος ἐν Διαδοχαῖς (FGrH 273 F 92), εἶτ' Ἀναξάρχου, 
ξυνακολουθῶν πανταχοῦ, ὡς καὶ τοῖς γυμνοσοφισταῖς ἐν Ἰνδίᾳ 
συμμῖξαι καὶ τοῖς Μάγοις. 



Diogenes Laertius Biogr., Vitae philosophorum 
Book 9, section 63, line 3

                                             τοῦτο δὲ ποιεῖν ἀκού-
σαντα Ἰνδοῦ τινος ὀνειδίζοντος Ἀναξάρχῳ ὡς οὐκ ἂν ἕτερόν τινα 
διδάξαι οὗτος ἀγαθόν, αὐτὸς αὐλὰς βασιλικὰς θεραπεύων. 



Diogenes Laertius Biogr., Vitae philosophorum 
Book 9, section 65, line 6

καὶ πάλιν ἐν τοῖς Ἰνδαλμοῖς (PPF 9 B 67. 1 – 2, 5)· 
   τοῦτό μοι, ὦ Πύρρων, ἱμείρεται ἦτορ ἀκοῦσαι, 
    πῶς ποτ' ἀνὴρ ὅτ' ἄγεις ῥᾷστα μεθ' ἡσυχίης 
   μοῦνος ἐν ἀνθρώποισι θεοῦ τρόπον ἡγεμονεύων. 



Diogenes Laertius Biogr., Vitae philosophorum 
Book 9, section 105, line 7

                        καὶ ἐν τοῖς Ἰνδαλμοῖς οὕτω λέγει (PPF 
9 B 69), 
   ἀλλὰ τὸ φαινόμενον πάντῃ σθένει οὗπερ ἂν ἔλθῃ. 

 \end{greek}
 

\section{Juba II Rex Mauretaniae}%???
Who is this?
%\blockquote[From Wikipedia\footnote{\url{}}]{}
\begin{greek}

Juba II Rex Mauretaniae <Hist.>, Fragmenta (1452: 003)
“FHG 3”, Ed. Müller, K.
Paris: Didot, 1841–1870.
Fragment 30, line 8

             Σελεύκου τε τοῦ Νικάνορος κτῆμα ᾄδει 
Ἰνδὸν ἐλέφαντα, καὶ μέντοι καὶ διαβιῶναι τοῦτον μέχρι 
τῆς τῶν Ἀντιόχων ἐπικρατείας φησίν. 



Juba II Rex Mauretaniae <Hist.>, Fragmenta 
Fragment 66, line 1

                   .. ιuba in αethiopia gigni tradit 
in litoribus amnis, quem Nilum vocamus, et inde 
nomen trahere. 
 Aelian. N. A. XV, 8: Ἄριστος ἄρα ὁ Ἰνδικὸς (μάρ-
γαρος) γίνεται, καὶ ὁ τῆς θαλάττης τῆς Ἐρυθρᾶς. 



Juba II Rex Mauretaniae <Hist.>, Fragmenta 
Fragment 66, line 8

     Γίνεσθαι δέ φησιν Ἰόβας καὶ ἐν τῷ κατὰ Βόσπορον 
πορθμῷ, καὶ τοῦ Βρεττανικοῦ ἡττᾶσθαι αὐτὸν, τῷ δὲ 
Ἰνδῷ καὶ τῷ Ἐρυθραίῳ μηδὲ τὴν ἀρχὴν ἀντικρίνεσθαι. 



Juba II Rex Mauretaniae <Hist.>, Fragmenta 
Fragment 66, line 9

Ὁ δὲ ἐν Ἰνδίᾳ χερσαῖος οὐ λέγεται φύσιν ἔχειν ἰδίαν, 
ἀλλὰ ἀπογέννημα εἶναι κρυστάλλου, οὐ τοῦ ἐκ τῶν 
παγετῶν συνισταμένου, ἀλλὰ τοῦ ὀρυκτοῦ. 



Juba II Rex Mauretaniae <Hist.>, Fragmenta 
Fragment 87, line 3

                          . οἱ δὲ Δεύνυσον, 
ἐπειδὴ βασιλεὺς ἐγένετο Νύσσης· δεῦνον δὲ τὸν βασι-
λέα λέγουσιν οἱ Ἰνδοὶ, ὡς Ἰόβας. 

\end{greek}


\section{Diophantus}
\blockquote[From Wikipedia\footnote{\url{http://en.wikipedia.org/wiki/Diophantus}}]{Diophantus of Alexandria (Ancient Greek: Διόφαντος ὁ Ἀλεξανδρεύς. b. between A.D. 200 and 214, d. between 284 and 298 at age 84), sometimes called "the father of algebra" ,though this is greatly disputed [1], was an Alexandrian Greek mathematician[2][3][4][5] and the author of a series of books called Arithmetica. These texts deal with solving algebraic equations, many of which are now lost. In studying Arithmetica, Pierre de Fermat concluded that a certain equation considered by Diophantus had no solutions, and noted without elaboration that he had found "a truly marvelous proof of this proposition," now referred to as Fermat's Last Theorem. This led to tremendous advances in number theory, and the study of Diophantine equations ("Diophantine geometry") and of Diophantine approximations remain important areas of mathematical research. Diophantus was the first Greek mathematician who recognized fractions as numbers; thus he allowed positive rational numbers for the coefficients and solutions. In modern use, Diophantine equations are usually algebraic equations with integer coefficients, for which integer solutions are sought. Diophantus also made advances in mathematical notation.}
\begin{greek}

Diophantus Math., Fragmentum [Sp.] (e cod. Paris. suppl. gr. 387, fol. 181r) (2039: 003)
“Diophanti Alexandrini opera omnia, vol. 2”, Ed. Tannery, P.
Leipzig: Teubner, 1895, Repr. 1974.
Volume 2, page 3, line 7

                                    καὶ ἡ μὲν μία ἔχει 
οὕτως· ἀπόγραψαι τοιοῦτον ἀριθμὸν κατὰ τὴν τάξιν 
τῆς <Ἰνδικῆς> μεθόδου· εἶτα ἄρξαι ἀπὸ δεξιῶν ἐπὶ 
ἀριστερά, καθ' ἕκαστον δὲ στοιχεῖον λέγε· γίνεται· οὐ 
γίνεται· γίνεται· οὐ γίνεται· ἕως ἂν τελειωθῶσι τὰ 
στοιχεῖα, καὶ εἰ μὲν τύχῃ τὸ τελευταῖον ὑπὸ τὸ γίνε-
ται, ἄρξαι τοῦ μερισμοῦ ἐκεῖθεν· εἰ δὲ ὑπὸ τὸ οὐ 
γίνεται, καταλιπὼν τὸ τελευταῖον στοιχεῖον ἄρξαι τοῦ 
μερισμοῦ ἀπὸ τοῦ μετ' αὐτὸ στοιχείου τοῦ πρὸς τὰ 
δεξιά, ἐν ᾧ δηλονότι φθάνει τὸ γίνεται. 

\end{greek}


\section{Diogenianus}
\blockquote[From Wikipedia\footnote{\url{http://en.wikipedia.org/wiki/Diogenianus}}]{gation, search

Diogenianus was a Greek grammarian from Heraclea in Pontus (or in Caria) who flourished during the reign of Hadrian.[1] He was the author of an alphabetical lexicon, chiefly of poetical words, abridged from the great lexicon (Περὶ γλωσσῶν) of Pamphilus of Alexandria (AD 50) and other similar works. It was also known by the title Περιεργοπένητες (for the use of "industrious poor students"). It formed the basis of the lexicon, or rather glossary, of Hesychius of Alexandria, which is described in the preface as a new edition of the work of Diogenianus. We still possess a collection of proverbs under his name, probably an abridgment of the collection made by himself from his lexicon (ed. by Ernst von Leutsch and Friedrich Wilhelm Schneidewin in Paroemiographi Graeci, 1. 1839). Diogenianus was also the author of an Anthologion of epigrams about rivers, lakes, cliffs, mountains and mountaintops (Επιγραμμάτων ανθολόγιον περί ποταμών λιμνών κρηνών ορών ακρωρειών) (Anthology of epigramse etc.); and of a list (with map) of all the towns in the world.

Erasmus attributed the origins of this Latin parable to Diogenianus — piscem natare docem (teach fish how to swim).[2]}
\begin{greek}
Diogenianus Gramm., Paroemiae (epitome operis sub nomine Diogeniani) (e cod. Vindob. 133) (0097: 002)
“Corpus paroemiographorum Graecorum, vol. 2”, Ed. von Leutsch, E.L.
Göttingen: Vandenhoeck \& Ruprecht, 1851, Repr. 1958.
Centuria 2, section 20, line 6

          Γύγης γὰρ βουκόλος ὢν γῆς ὑπὸ σεισμοῦ ῥαγεί-  
σης νεκρὸν εὑρὼν φοροῦντα δακτύλιον καὶ τοῦτον περιελό-
μενος φύσιν ἔχοντα ὥστε κατὰ τὰς στροφὰς τῆς σφενδόνης 
ὁρᾶσθαι καὶ μὴ ὅταν βούληται, εἶχε παρ' ἑαυτῷ· μεθ' οὗ 
καὶ κτείνας τὸν πρὸ αὐτοῦ ἐβασίλευεν Ἰνδῶν. 



Diogenianus Gramm., Paroemiae (epitome operis sub nomine Diogeniani) (e cod. Vindob. 133) 
Centuria 3, section 95, line 1

<Ψιττακὸς † Ἰνδέστιος:> ἐπὶ τῶν μιμουμένων τινάς· 
τοιοῦτον γὰρ τὸ ὄρνεον ὥστε ἀνθρωπίνην προΐεσθαι φωνήν. 


\end{greek}


\section{Polyaenus}
\blockquote[From Wikipedia\footnote{\url{http://en.wikipedia.org/wiki/Polyaenus}}]{Polyaenus or Polyenus (play /ˌpɒliˈiːnəs/;[1] see ae (æ) vs. e; Greek: Πoλύαινoς, Poluainos, "many proverbs") was a 2nd century Macedonian author, known best for his Stratagems in War (in Greek, Στρατηγήματα), which has been preserved. The Suda[2] calls him a rhetorician, and Polyaenus himself writes that he was accustomed to plead causes before the emperor.[3] He dedicated Stratagems in War to Marcus Aurelius (161–180) and Verus (161–169), while they were engaged in the Parthian war (162–165), about 163 CE, at which time he was too old to accompany them in their campaigns.[4]}
\begin{greek}
Polyaenus Rhet., Strategemata (0616: 001)
“Polyaeni strategematon libri viii”, Ed. Woelfflin, E., Melber, J.
Leipzig: Teubner, 1887, Repr. 1970.
Book 1, chapter 1, section 1, line 1

Διόνυσος ἐπ' Ἰνδοὺς ἐλαύνων, ἵνα δέχοιντο αἱ 
πόλεις αὐτὸν, ὅπλοις μὲν φανεροῖς τὴν στρατιὰν οὐχ' 
ὥπλισεν, ἐσθῆσι δὲ λεπταῖς καὶ νεβρίσι· δόρατα ἦν 
κισσῷ πεπυκασμένα· ὁ θύρσος εἶχεν αἰχμήν· κυμβάλοις 
καὶ τυμπάνοις ἐσήμαινεν ἀντὶ σάλπιγγος καὶ οἴνου 
τοὺς πολεμίους γεύων εἰς ὄρχησιν ἔτρεπεν καὶ ὅσα ἄλλα 
Βακχικὰ ὄργια. 



Polyaenus Rhet., Strategemata 
Book 1, chapter 1, section 1, line 8

                   πάντα δὲ ἦν Διονύσου στρατηγήματα, 
οἷς Ἰνδοὺς καὶ τὴν ἄλλην Ἀσίαν ἐχειρώσατο. 



Polyaenus Rhet., Strategemata 
Book 1, chapter 1, section 2, line 1

Διόνυσος ἐν Ἰνδικῇ τῆς στρατιᾶς οὐ φερούσης τὸ 
φλογῶδες τοῦ ἀέρος κατελάβετο τρικόρυφον ὄρος τῆς 
Ἰνδικῆς. 



Polyaenus Rhet., Strategemata 
Book 1, chapter 1, section 3, line 1

Διόνυσος Ἰνδοὺς ἑλὼν αὐτούς τε Ἰνδοὺς καὶ 
Ἀμαζόνας ἄγων συμμάχους εἰς τὴν Βακτρίων ἐνέ-
βαλεν· ὁρίζει δὲ τὴν Βακτρίαν ποταμὸς Σαράγγης. 



Polyaenus Rhet., Strategemata 
Book 1, chapter 3, section 4, line 1

Ἡρακλῆς ἐν Ἰνδικῇ θυγατέρα ἐποιήσατο, ἣν ἐκά-
λεσε Πανδαίην. 



Polyaenus Rhet., Strategemata 
Book 1, chapter 3, section 4, line 2

                 ταύτῃ νείμας μοῖραν τῆς Ἰνδικῆς 
πρὸς μεσημβρίαν καθήκουσαν εἰς θάλασσαν διένειμε 
τοὺς ἀρχομένους εἰς κώμας τξεʹ προστάξας καθ' ἑκά-
στην ἡμέραν μίαν κώμην ἀποφέρειν τὸν βασίλειον 
φόρον, ἵνα τοὺς ἤδη δόντας ἔχοι συμμάχους ἡ βασι-
λεύουσα κατανοοῦσα ἀεὶ τοὺς δοῦναι ὀφείλοντας. 



Polyaenus Rhet., Strategemata 
Book 4, chapter 3, section 9, line 1

Ἀλέξανδρος ἦν ἐν Ἰνδοῖς καὶ τὸν Ὑδάσπην ποτα-
μὸν διαβαίνειν ἔμελλεν· Πῶρος Ἰνδῶν βασιλεὺς ἐπέ-
κεινα τοῦ ποταμοῦ παρετάσσετο, καὶ διαβαίνειν ἀδύ-
νατον ἦν. 



Polyaenus Rhet., Strategemata 
Book 4, chapter 3, section 9, line 8

                                                     τοῦτο 
ἐγίγνετο πολλάκις καὶ ἐπὶ πολλὰς ἡμέρας, ὥστε οἱ 
μὲν Ἰνδοὶ κατεγέλασαν τῆς δειλίας τῶν πολεμίων καὶ 
συμπαραθέοντες ἐπαύσαντο, ὡς οὐκ ἄν ποτε διαβῆναι 
τολμησάντων, οἳ τοσάκις οὐκ ἐθάρρησαν. 



Polyaenus Rhet., Strategemata 
Book 4, chapter 3, section 9, line 13

                                              Ἀλέξανδρος   
δὲ ὀξυτάτῳ δρόμῳ παρὰ τὰς ὄχθας ἀναδραμὼν, ἐπι-
βὰς πλοίων καὶ ζευγμάτων καὶ διφθερῶν χόρτου πε-
πληρωμένων διέβη τὸν ποταμὸν ἐξαπατήσας Ἰνδοὺς 
τῷ τῆς διαβάσεως ἀνελπίστῳ. 



Polyaenus Rhet., Strategemata 
Book 4, chapter 3, section 10, line 1

Ἀλέξανδρος κατέστρεφε τὴν Ἰνδῶν. 



Polyaenus Rhet., Strategemata 
Book 4, chapter 3, section 10, line 3

                                          τῶν δὲ στρα-
τιωτῶν ἐφελκομένων λάφυρα Περσικὰ καὶ πλοῦτον 
ὑπέρογκον ἐπὶ τῶν ἁμαξῶν καὶ τὴν πρὸς Ἰνδοὺς μά-
χην οὐκ ἀναγκαίαν ἡγουμένων τοσαῦτα δὴ κεκτημέ-
νων, πρώτας ὑπέπρησε τὰς βασιλικὰς ἁμάξας, εἶτα 
τὰς τῶν ἄλλων. 



Polyaenus Rhet., Strategemata 
Book 4, chapter 3, section 20, line 1

Ἀλέξανδρος χωρίῳ τῆς Ἰνδικῆς ἐχυρῷ προσεκά-
θητο· φοβηθέντες Ἰνδοὶ συνέθεντο μεθ' ὅπλων ἐξελ-
θεῖν. 



Polyaenus Rhet., Strategemata 
Book 4, chapter 3, section 20, line 5

                           Ἀλέξανδρος ἐπῆγε τὴν δύ-
ναμιν τῶν Ἰνδῶν ἐπιβοωμένων τὰς συνθήκας. 



Polyaenus Rhet., Strategemata 
Book 4, chapter 3, section 21, line 16

      ὁπότ' οὖν οἱ τοῦ λαιοῦ μέρους τὴν δεξιὰν οὐρα-
γίαν ἰδόντες ἀλαλάξαντες προσῆγον, ὁμοίως δὲ καὶ οἱ 
ἀπὸ τοῦ δεξιοῦ ἐφ' ἡνίαν στρέφοντες ἐπῆγον τοῖς 
Ἰνδοῖς· οἱ δὲ συγκλεισθῆναι φοβηθέντες ὥρμησαν ὁμοῦ 
πάντες ἐπὶ τὴν στενὴν ἔξοδον, ὥστε οἱ μὲν ὑπὸ τῶν   
Μακεδόνων ἀνῃροῦντο, οἱ δὲ πλείους ὑπ' ἀλλήλων 
καταπατούμενοι διεφθείροντο. 



Polyaenus Rhet., Strategemata 
Book 4, chapter 3, section 22, line 21

               ἐν τούτῳ φθάνουσιν οἱ περὶ τὸν Ἀλέ-
ξανδρον ἱππεῖς ἐκπεριελθόντες καὶ κατὰ νώτου συνε-
λάσαντες τοὺς Ἰνδοὺς τελεωτάτην νίκην ἀνείλοντο 
(μεθ' ἣν βασιλεὺς Ἰνδικῆς Ἀλέξανδρος ἦν). 



Polyaenus Rhet., Strategemata 
Book 4, chapter 3, section 24, line 6

                       ἐν γοῦν Βάκτροις καὶ Ὑρκανίοις 
καὶ Ἰνδοῖς δικάζων εἶχε τὴν σκηνὴν ὧδε πεποιημέ-
νην. 



Polyaenus Rhet., Strategemata 
Book 4, chapter 3, section 30, line 1

Ἀλέξανδρος Καθαίους, μοῖραν Ἰνδῶν ἐξ ἀπονοίας 
ἀντιστᾶσαν ἡβηδὸν ἔκτεινε καὶ πόλιν αὐτῶν Σάγγαλα 
τὴν καρτερωτάτην κατέσκαψεν. 



Polyaenus Rhet., Strategemata 
Book 4, chapter 3, section 30, line 3

                                διῆκε τοὺς Ἰνδοὺς 
φήμη πονηρὰ ὡς Ἀλεξάνδρου φονικῶς καὶ βαρ-
βαρικῶς πολεμοῦντος. 



Polyaenus Rhet., Strategemata 
Book 4, chapter 3, section 30, line 6

                       ὁ δὲ μεταβάλλειν τὴν δόξαν 
βουλόμενος ἄλλην πόλιν (διὰ) τῆς Ἰνδικῆς ἑλὼν, ὁμή-
ρους λαβὼν, σπεισάμενος ἐπὶ τρίτην πόλιν ἦλθεν εὐ-
μεγέθη καὶ πολυάνθρωπον, τάξας πρὸ τῆς φάλαγγος 
τοὺς ὁμήρους, γέροντας, παῖδας, γυναῖκας. 



Polyaenus Rhet., Strategemata 
Book 4, chapter 3, section 30, line 13

                                      αὐτίκα ἡ φήμη δια-
δραμοῦσα ἔπεισεν Ἰνδοὺς ἑκόντας Ἀλέξανδρον δέχεσθαι. 



Polyaenus Rhet., Strategemata 
Book 4, chapter 6, section 3, line 7

                    Ἀντίγονος τοῦ λοιποῦ προσέταξε τοῖς 
Ἰνδοῖς τρέφειν ὗς μετὰ τῶν ἐλεφάντων, ἵνα τὴν ὄψιν 
αὐτῶν καὶ τὴν κραυγὴν τὰ θηρία φέρειν ἐθίζοιτο. 



Polyaenus Rhet., Strategemata 
Book 4, chapter 21, section 1, line 2

Περσεὺς Ῥωμαίων ἐλέφαντας ἀγόντων τοὺς μὲν 
ἐκ Λιβύης, τοὺς δὲ Ἰνδοὺς παρὰ Ἀντιόχου Συρίας 
βασιλέως, ἵνα μὴ καινὸν καὶ φοβερὸν τοῖς ἵπποις τὸ 
θηρίον φανείη, προσέταξε τοῖς χειροτέχναις εἴδωλα 
ξύλινα κατασκευάζειν ἐλεφάντων ἰδέαν καὶ χρόαν ἔχοντα. 



Polyaenus Rhet., Strategemata 
Book 6, chapter 50, section t, line 1

                                           Ἀλεξάνδρου 
ἐν Βαβυλῶνι τελευτήσαντος εἰς Ἔφεσον ὑπὸ Περδίκ-
κου κατεπέμφθη κριθησόμενος κατὰ τοὺς νόμους· 
Ἀναξαγόρας δὲ καὶ Κόδρος διάραντες εἰς Ἀθήνας 
ἐσώθησαν, τὴν δὲ Ἀλεξάνδρου τελευτὴν ἀκούσαντες 
εἰς Ἔφεσον ἐπανελθόντες καὶ τὸν ἀδελφὸν Διόδωρον 
ἀνέσωσαν.   
ΠΙΝΔ*αΡΟΣ. 




Polyaenus Rhet., Strategemata 
Book 8, chapter 50, section 1, line 27

                          καὶ ἐπὶ τοσοῦτον ἔπεισαν 
τοὺς ὑπηκόους, ἐφ' ὅσον μεταπεμφθεὶς ὑπ' αὐτῶν 
Πτολεμαῖος ἧκεν ὁ πατὴρ τῆς ἀνῃρημένης καὶ διαπέμ-
πων ἀπὸ τῆς προσηγορίας τοῦ πεφονευμένου παιδὸς 
καὶ τῆς ἀνῃρημένης Βερενίκης ὡς ἔτι ζώντων ἐπιστολὰς 
ἀπὸ τοῦ Ταύρου μέχρι τῆς Ἰνδικῆς χωρὶς πολέμου 
καὶ μάχης ἐκράτησε τῷ στρατηγήματι τῆς Παναρίστης 
χρησάμενος. 

\end{greek}



\section{Dionysius Periegetes}
\blockquote[From Wikipedia\footnote{\url{http://en.wikipedia.org/wiki/Dionysius_Periegetes}}]{Dionysius Periegetes (Διονύσιος ὁ Περιηγητής, literally Dionysius the Voyager or Traveller, often Latinized to Dionysius Periegeta) was the author of a description of the habitable world in Greek hexameter verse written in a terse and elegant style. His lifedates, and indeed his origins, are not known, but he is believed to have been from Alexandria and to have flourished around the time of Hadrian (r. 117–138 CE), though some put him as late as the end of the 3rd century.

The work enjoyed popularity in ancient times as a schoolbook. It was translated into Latin by Rufus Festus Avienus, and by the grammarian Priscian. There is a commentary by Eustathius of Thessalonica.}
\begin{greek}
Dionysius Perieg., Orbis descriptio (0084: 001)
“Dionysios von Alexandria. Das Lied von der Welt”, Ed. Brodersen, K.
Hildesheim: Olms, 1994.
Line 37

εἷς μὲν ἐών, πολλῇσι δ' ἐπωνυμίῃσιν ἀρηρώς· 
ἤτοι μὲν Λοκροῖο παρ' ἐσχατιὴν ζεφύροιο 
Ἄτλας Ἑσπέριος κικλήσκεται, αὐτὰρ ὕπερθεν 
πρὸς βορέην, ἵνα παῖδες ἀρειμανέων Ἀριμασπῶν, 
πόντον μιν καλέουσι πεπηγότα τε Κρόνιόν τε· 
ἄλλοι δ' αὖ καὶ νεκρὸν ἐφήμισαν εἵνεκ' ἀφαυροῦ 
ἠελίου· βράδιον γὰρ ὑπεὶρ ἅλα τήνδε φαείνει, 
αἰεὶ δὲ σκιερῇσι παχύνεται ἐν νεφέλῃσιν· 
αὐτὰρ ὅθι πρώτιστα φαείνεται ἀνθρώποισιν, 
ἠῷον καλέουσι καὶ Ἰνδικὸν οἶδμα θαλάσσης· 
ἄγχι δ' Ἐρυθραῖόν τε καὶ Αἰθόπιον καλέουσιν 
πρὸς νότον ἔνθα τε πολλὸς ἀοικήτου χθονὸς ἀγκὼν 
ἐκτέταται, μαλεροῖσι κεκαυμένος ἠελίοισιν. 



Dionysius Perieg., Orbis descriptio 
Line 578

οὐχ οὕτω Θρήϊκος ἐπ' ᾐόσιν Ἀψύνθοιο 
Βιστονίδες καλέουσιν ἐρίβρομον Εἰραφιώτην, 
οὐδ' οὕτω σὺν παισὶ μελανδίνην ἀνὰ Γάγγην 
Ἰνδοὶ κῶμον ἄγουσιν ἐριβρεμέτῃ Διονύσῳ, 
ὡς κεῖνον κατὰ χῶρον ἀνευάζουσι γυναῖκες. 



Dionysius Perieg., Orbis descriptio 
Line 625

σχῆμα δέ τοι Ἀσίης ῥυσμὸς πέλει ἀμφοτεράων 
ἠπείρων, ἑτέρωθεν ἀλίγκιον εἴδεϊ κώνου, 
ἑλκόμενον κατὰ βαιὸν ἐπ' ἀντολίης μυχὰ πάσης, 
ἔνθα τε καὶ στῆλαι Θηβαιγενέος Διονύσου 
ἑστᾶσιν, πυμάτοιο παραὶ ῥόον Ὠκεανοῖο, 
Ἰνδῶν ὑστατίοισιν ἐν οὔρεσιν, ἔνθα τε Γάγγης 
λευκὸν ὕδωρ Νυσαῖον ἐπὶ πλαταμῶνα κυλίνδει. 



Dionysius Perieg., Orbis descriptio 
Line 639

μέσσα γε μὴν πάσης Ἀσίης ὄρος ἀμφιβέβηκεν, 
ἀρξάμενον γαίης Παμφυλίδος ἄχρι καὶ Ἰνδῶν, 
ἄλλοτε μὲν λοξόν τε καὶ ἀγκύλον, ἄλλοτε δ' αὖτε 
ἴχνεσιν ὀρθότατον· Ταῦρον δέ ἑ κικλήσκουσιν, 
οὕνεκα ταυροφανές τε καὶ ὀρθόκραιρον ὁδεύει, 
οὔρεσιν ἐκταδίοισι πολυσχιδὲς ἔνθα καὶ ἔνθα. 



Dionysius Perieg., Orbis descriptio 
Line 701

τῷ δ' ἐνὶ ναιετάουσιν ἑωθινὸν ἔθνος Ἰβήρων, 
οἵ ποτε Πυρήνηθεν ἐπ' ἀντολίην ἀφίκοντο, 
ἀνδράσιν Ὑρκανίοισιν ἀπεχθέα δῆριν ἔχοντες, 
καὶ Καμαριτάων φῦλον μέγα, τοί ποτε Βάκχον 
Ἰνδῶν ἐκ πολέμοιο δεδεγμένοι ἐξείνισσαν 
καὶ μετὰ Ληνάων ἱερὸν χορὸν ἐστήσαντο, 
ζώματα καὶ νεβρῖδας ἐπὶ στήθεσσι βαλόντες, 
εὐοῖ Βάκχε λέγοντες· ὁ δὲ φρεσὶ φίλατο δαίμων 
κείνων ἀνθρώπων γενεὴν καὶ ἤθεα γαίης. 



Dionysius Perieg., Orbis descriptio 
Line 890

οἶσθα γάρ, ἐν πρώτοισιν ἐμεῦ εἰπόντος ἀκούσας, 
πᾶσαν ἕως Ἰνδῶν Ἀσίην ὄρος ἄνδιχα τέμνειν. 



Dionysius Perieg., Orbis descriptio 
Line 893

κεῖνό τοι ἐν πλευροῖσι βορειότερον τελέοιτο, 
Νεῖλος δ' ἑσπέριον πλευρὸν πέλοι· αὐτὰρ ἑῷον 
Ἰνδικὸς Ὠκεανός· νότιον δ' ἁλὸς οἴδματ' Ἐρυθρῆς. 



Dionysius Perieg., Orbis descriptio 
Line 1074

χωρὶς μὲν Κόρος ἐστὶ μέγας, χωρὶς δὲ Χοάσπης, 
ἕλκων Ἰνδὸν ὕδωρ, παρά τε ῥείων χθόνα Σούσων. 



Dionysius Perieg., Orbis descriptio 
Line 1088

τῶν δὲ πρὸς ἀντολίην Γεδρωσῶν ἕλκεται γαῖα, 
γείτων Ὠκεανοῦ μεγακήτεος, οἷσι πρὸς αὐγὰς 
Ἰνδὸν πὰρ ποταμὸν νότιοι Σκύθαι ἐνναίουσιν, 
ὅς ῥά τ' Ἐρυθραίης κατεναντίον εἶσι θαλάσσης, 
λαβρότατος ῥόον ὠκὺν ἐπὶ νότον ὀρθὸν ἐλαύνων, 
ἀρξάμενος τὰ πρῶτ' ἀπὸ Καυκάσου ἠνεμόεντος. 



Dionysius Perieg., Orbis descriptio 
Line 1107

πρὸς δ' αὐγὰς Ἰνδῶν ἐρατὴ παραπέπταται αἶα, 
πασάων πυμάτη, παρὰ χείλεσιν Ὠκεανοῖο, 
ἥν ῥά τ' ἀνερχόμενος μακάρων ἐπὶ ἔργα καὶ ἀνδρῶν 
ἠέλιος πρώτῃσιν ἐπιφλέγει ἀκτίνεσσιν. 



Dionysius Perieg., Orbis descriptio 
Line 1132

ἤτοι μὲν πισύρεσσιν ἐπὶ πλευρῇσιν ἄρηρε 
πάσῃσιν λοξῇσιν, ἀλιγκίη εἴδεϊ ῥόμβου· 
ἀλλά τοι ἑσπερίοις μὲν ὁμούριος ὕδασιν Ἰνδὸς 
γαῖαν ἀποτμήγει, νότιον δ' ἁλὸς οἴδματ' Ἐρυθρῆς, 
Γάγγης δ' εἰς αὐγάς, ὁ δὲ Καύκασος ἐς δύσιν ἄρκτων. 



Dionysius Perieg., Orbis descriptio 
Line 1137

καὶ τὴν μὲν πολλοί τε καὶ ὄλβιοι ἄνδρες ἔχουσιν, 
οὐχ ἅμα ναιετάοντες ὁμώνυμοι, ἀλλὰ διαμφὶς 
κεκριμένοι, ποταμοῦ μὲν ἀπειρεσίου πέλας Ἰνδοῦ 
Δαρδανέες, τόθι λοξὸν ἀπὸ σκοπέλων Ἀκεσίνην 
συρόμενον δέχεται πλωτὸς νήεσσιν Ὑδάσπης. 



Dionysius Perieg., Orbis descriptio 
Line 1161

αὐτὸς δ' ὁππότε φῦλα κελαινῶν ὤλεσεν Ἰνδῶν, 
Ἠμωδῶν ὀρέων ἐπεβήσατο, τῶν ὑπὸ πέζαν 
ἕλκεται ἠῴοιο μέγας ῥόος Ὠκεανοῖο. 



Dionysius Perieg., Ixeuticon sive De aucupio (paraphrasis) (olim sub auctore Eutecnio) (0084: 003)
“Dionysii ixeuticon seu de aucupio libri tres in epitomen metro solutam redacti”, Ed. Garzya, A.
Leipzig: Teubner, 1963.
Chapter 1, section 32, line 1

Ἀκήκοα δέ, ὡς παρὰ τοῖς Ἰνδοῖς ὄρνις εἴη γονέων 
ἄτερ καὶ μίξεως χωρὶς ὑφιστάμενος, φοῖνιξ ὄνομα, καὶ   
βιοῦν φασιν ἐπὶ πλεῖστον καὶ μετὰ πάσης ἀφοβίας αὐτόν, 
ὡς οὔτε τόξοις, οὔτε λίθοις, οὔτε καλάμοις ἢ πάγαις τῶν 
ἀνδρῶν τι κατ' αὐτοῦ ποιεῖν πειρωμένων. 

\end{greek}


\section{<Damigeron Magus>}%???
%\blockquote[From Wikipedia\footnote{\url{}}]{}
Who is this?
%http://searchworks.stanford.edu/view/7242932
\begin{greek}

<Damigeron Magus>, De lapidibus (e codd. Vat. gr. 578 + Ambros. 95 sup.) (2655: 002)
“”Ein unedierter Tractat περὶ λίθων ””, Ed. Mesk, J., 1897; Wiener Studien.
Page 319, line 3

                                                        οὗτος ὁ λίθος γεν-
νᾶται ἐν Ἰνδίᾳ, ὅπου ὁ Φισὼν ποταμὸς ἐκ τοῦ παραδείσου ἔρχεται· 
οὗτος ὅρασιν ἔχει ὁμοίαν τῇ χλόῃ τῆς γῆς καὶ ὁ μὲν πρασώδης οὗτος 
καλεῖται νερωνιανός· ὁ δὲ παρὰ τοῦτον ὑποχλωριάζων λέγεται σμάρα-
γδος ὑακτορίζων· ἐὰν δὲ ᾖ ὑπόχλωρος, ἀσπροειδὴς ἔλαττον τούτου, 
λέγεται τακτώριος. 



<Damigeron Magus>, De lapidibus (e codd. Vat. gr. 578 + Ambros. 95 sup.) 
Page 319, line 26

                                           γίνεται δὲ ἐν τῇ Ἰνδικῇ, ὅπου 
καὶ ὁ προγεγραμμένος. 



<Damigeron Magus>, De lapidibus (e codd. Vat. gr. 578 + Ambros. 95 sup.) 
Page 320, line 18

Λίθος ὀνυχίτης· οὗτος ἐν τῇ Ἰνδικῇ γίνεται λευκὰς ζώνας πλείστας 
ἔχων ἐν ἑαυτῷ ἀεριζούσας. 


\end{greek}


\section{Favorinus}
\blockquote[From Wikipedia\footnote{\url{http://en.wikipedia.org/wiki/Favorinus}}]{
Jump to: navigation, search

Favorinus of Arelate (ca. 80–160 AD) was a Roman sophist and philosopher who flourished during the reign of Hadrian.

He was of Gaulish ancestry, born in Arelate (Arles). He is described as a hermaphrodite (ἀνδρόθηλυς) by birth. He received an exquisite education, first in Gallia Narbonensis and then in Rome, and at an early age began his lifelong travels through Greece, Italy and the East. His extensive knowledge, combined with great oratorical powers, raised him to eminence both in Athens and in Rome. With Plutarch, with Herodes Atticus, to whom he bequeathed his library at Rome, with Demetrius the Cynic, Cornelius Fronto, Aulus Gellius, and with Hadrian himself, he lived on intimate terms; his great rival, whom he violently attacked in his later years, was Polemon of Smyrna.}

\begin{greek}

Favorinus Phil., Rhet., Fragmenta (1377: 003)
“Favorino di Arelate. Opere”, Ed. Barigazzi, A.
Florence: Monnier, 1966.
Fragment 85, line 1

Steph. Byz. Ἀραχωτοί· πόλις Ἰνδικῆς, ἀπὸ Ἀραχώτου πο-
ταμοῦ ῥέοντος ἀπὸ τοῦ Καυκάσου, ὡς Φαβωρῖνος καὶ Στράβων 
ἑνδεκάτῃ. 

\end{greek}


\section{Chrysermus of Alexandria}
\blockquote[From Brill's New Pauly\footnote{\url{http://referenceworks.brillonline.com/entries/brill-s-new-pauly/chrysermus-of-alexandria-e233990?s.num=12}}]{(IDélos 1525). C. lived in about 150-120 BC; administrative official, ‘relative of king Ptolemy’, exegete (i.e. head of the civil service in Alexandria), director of the museum and ἐπὶ τῶν ἰατρῶν, a title that is often understood to mean the person responsible for all Egyptian doctors, which in turn led to the conclusion that there was a state organization of doctors. Kudlien is of the opinion that the title refers to the person responsible for the person in char…}
\begin{greek}

[Chrysermus] Hist., Fragmenta (2195: 002)
“FGH 4”, Ed. Müller, K.
Paris: Didot, 1841–1870.
Fragment t4, line 1

                                           Φωραθέντος δὲ 
τούτου, Ἀγησίλαος ὁ πατὴρ μέχρι τοῦ ναοῦ τῆς Χαλ-
κιοίκου συνεδίωξεν Ἀθηνᾶς, καὶ τὰς θύρας τοῦ τεμέ-
νους πλίνθῳ φράξας, λιμῷ ἀπέκτεινεν· ἡ δὲ μήτηρ καὶ 
ἄταφον ἔρριψεν· ὡς Χρύσερμος ἐν δευτέρῳ Ἱστορικῶν. 
ΙΝΔ*ιΚΑ. 


[Chrysermus] Hist., Fragmenta 
Fragment 4, line 9

                                               – Κατορύσσουσι 
δὲ κατ' ἐνιαυτὸν γραῦν κατάκριτον, παρὰ τὸν ὀνομα-
ζόμενον λόφον Θηρόγονον· ἅμα γὰρ τὴν πρεσβῦτιν 
ἑρπετῶν πλῆθος ἐκ τῆς ἀκρωρείας ἐξέρχεται, καὶ τὰ 
περιϊπτάμενα τῶν ἀλόγων ζῴων κατεσθίει· καθὼς   
Χρύσερμος ἐν πʹ (ηʹ?) Ἰνδικῶν. 

\end{greek}


\section{Athenaeus Mechanicus}
\blockquote[From Wikipedia\footnote{\url{http://en.wikipedia.org/wiki/Athenaeus_Mechanicus}}]{Athenaeus Mechanicus is the author of a book on siegecraft, On Machines (Ancient Greek: Περὶ μηχανημάτων). He is identified by modern scholars with Athenaeus of Seleucia, a member of the Peripatetic school active in the mid-to-late 1st century BC, at Rome and elsewhere.[1][2]

The treatise is addressed to Marcus Claudius Marcellus, and thus will have been composed before Marcellus' death in 23 BC (and possibly at a time when its addressee was preparing to go out on campaign).[1] It describes a number of siege engines. Among the earlier mechanicians cited as sources by Athenaeus are Agesistratus, Diades of Pella, and Philo of Byzantium. Whitehead and Blyth analyze the treatise into a preface, a section on "good practice," a section on "bad practice," a section on Athenaeus' own innovations, and an epilogue "emphasizing preparation for war as a deterrent, and defending Athenaeus' own record against unnamed critics."[2] The work is technical but not without signs of Athenaeus' philosophical culture: "He comes across as a philosopher, and he expounds about time and opportunity, but also claims to be enough of a technical expert to devise new machines, and to describe old ones accurately."[1] Much of Athenaeus' work (9.4-27.6) is closely parallel to Vitruvius, De architectura 10.13-16, a fact probably to be explained by the two authors' shared reliance on a common source.[6]}

\begin{greek}
Athenaeus Mech., De machinis (1204: 001)
“Griechische Poliorketiker, vol. 1”, Ed. Schneider, R.
Berlin: Weidmann, 1912; Abhandlungen der königlichen Gesellschaft der Wissenschaften zu Göttingen, Philol.–hist. Kl., N.F. 12, no. 5.
Section 5, line 8

                       Ὅθεν οὐ κακῶς δόξειεν ἂν πρὸς αὐτοὺς 
εἰρηκέναι Κάλανος ὁ Ἰνδός· ‘Ἑλλήνων δὲ φιλοσόφοις οὐκ ἐξ-
ομοιούμεθα, παρ' οἷς ὑπὲρ μικρῶν πραγμάτων πολλοὶ λόγοι 
ἀναλίσκονται· ἡμεῖς δέ, φησίν, ὑπὲρ τῶν μεγίστων ἐλάχιστα εἰώ-
θαμεν παραγγέλλειν, ὅπως εὐμνημόνευτα πᾶσιν ᾖ. 

\end{greek}


\section{Oppian of Apamea}

\blockquote[From Wikipedia\footnote{\url{http://en.wikipedia.org/wiki/Oppian}}]{Oppian or Oppianus (Ancient Greek: Ὀππιανός) was the name of the authors of two (or three) didactic poems in Greek hexameters, formerly identified, but now generally regarded[citation needed] as two different persons: Oppian of Corycus (or Anazarbus) in Cilicia; and Oppian of Apamea (or Pella) in Syria.



Oppian of Apamea (or Pella) in Syria. His extant poem on hunting (Cynegetica) is dedicated to the emperor Caracalla, so that it must have been written after 211. It consists of about 2150 lines, and is divided into four books, the last of which, seems incomplete. The author evidently knew the Halieutica, and perhaps intended his poem as a supplement. Like his namesake, he shows considerable knowledge of his subject and close observation of nature; but in style and poetical merit he is inferior to him. His versification also is less correct. The improbability of there having been two poets of the same name, writing on subjects so closely akin and such near contemporaries, may perhaps be explained by assuming that the real name of the author of the Cynegetica was not Oppian, but that he has been confused with his predecessor. In any case, it seems clear that the two were not identical.

A third poem on bird-catching (Ixeutika), also formerly attributed to an Oppian, is lost; a paraphrase in Greek prose by a certain Eutecnius is extant. The author is probably one Dionysius, who is mentioned by the Suda as the author of a treatise on stones (Lithiaca).}

\begin{greek}

Oppianus Epic., Cynegetica (0024: 001)
“Oppian, Colluthus, Tryphiodorus”, Ed. Mair, A.W.
Cambridge, Mass.: Harvard University Press, 1928, Repr. 1963.
Book 3, line 259

Ἔστι δ' ἐϋκρήμνοις ἐπὶ τέρμασιν Αἰθιοπήων 
ἱππάγρων πολὺ φῦλον, ἀκαχμένον ἰοφόροισι 
δοιοῖς χαυλιόδουσι· ποδῶν γε μὲν οὐ μίαν ὁπλήν, 
χηλὴν δ' αὖ φορέουσι διπλῆν, ἰκέλην ἐλάφοισι· 
χαίτη δ' αὐχενίη μεσάτην ῥάχιν ἀμφιβεβῶσα 
οὐρὴν ἐς νεάτην μετανίσσεται· οὐδὲ βροτείην 
δουλοσύνην ἔτλη ποθ' ὑπερφίαλον γένος αἰνόν· 
ἀλλ' εἰ καί ποθ' ἕλοιεν ἐϋστρέπτοισι βρόχοισιν 
ἵππαγρον δολίοισι λόχοις μελανόχροες Ἰνδοί, 
οὔτε βορὴν ἐθέλει μετὰ χείλεσιν αἶψα πάσασθαι 
οὔτε πιεῖν, ὀλοὸς δὲ φέρειν ζυγὸν ἔπλετο δοῦλον. 


Oppianus Epic., Cynegetica 
Book 4, line 165

οὐ τοῖον Γάγγαο ῥόος πρόσθ' ἠελίοιο 
Ἰνδὸν ὑπὲρ δάπεδον Μαρυανδέα λαὸν ἀμείβων 
μυκᾶται βρύχημα πελώριον, ὁππότε κρημνῶν 
ἐκπροθορὼν ἐκάλυψε μέλαν δέμας αἰγιαλοῖο· 
ὅστε καὶ εὐρύτατός περ ἐὼν καί τ' εἴκοσιν ἄλλοις 
κυρτοῦται ποταμοῖσι κορυσσόμενος λάβρον ὕδωρ· 
οἷον ἐπισμαραγεῖ δρίος ἄσπετον ἠδὲ χαράδραι   
βρυχηθμοῖς ὀλοοῖσιν, ἐπιβρέμεται δ' ὅλος αἰθήρ. 

\end{greek}


\section{\emph{Physiologus}}

Allegorical stories about animals.

\blockquote[From Wikipedia\footnote{\url{http://en.wikipedia.org/wiki/Physiologus}}]{The Physiologus is a didactic text written or compiled in Greek by an unknown author, in Alexandria; its composition has been traditionally dated to the 2nd century AD by readers who saw parallels with writings of Clement of Alexandria, who is asserted to have known the text, though Alan Scott[1] has made a case for a date at the end of the third or in the 4th century. The Physiologus consists of descriptions of animals, birds, and fantastic creatures, sometimes stones and plants, provided with moral content. Each animal is described, and an anecdote follows, from which the moral and symbolic qualities of the animal are derived. Manuscripts are often, but not always, given illustrations, often lavish.

The story is told of the lion whose cubs are born dead and receive life when the old lion breathes upon them, and of the phœnix which burns itself to death and rises on the third day from the ashes; both are taken as types of Christ. The unicorn also which only permits itself to be captured in the lap of a pure virgin is a type of the Incarnation; the pelican that sheds its own blood in order to sprinkle its dead young, so that they may live again, is a type of the salvation of mankind by the death of Christ on the Cross.

Some allegories set forth the deceptive enticements of the Devil and his defeat by Christ; others present qualities as examples to be imitated or avoided.}

\begin{greek}

Physiologus, Physiologus (redactio prima) (2654: 001)
“Physiologus”, Ed. Sbordone, F.
Rome: Dante Alighieri–Albrighi, Segati, 1936, Repr. 1976.
Section 7, line 4

Ἔστι πετεινὸν ἐν τῇ Ἰνδίᾳ, φοῖνιξ λεγόμενον· κατὰ πεντακόσια ἔτη 
εἰσέρχεται εἰς τὰ ξύλα τοῦ Λιβάνου, καὶ γεμίζει τὰς πτέρυγας αὑτοῦ 
ἀρωμάτων, καὶ σημαίνει τῷ ἱερεῖ τῆς Ἡλιουπόλεως τῷ μηνὶ τῷ νέῳ,   
τῷ Νησὰν ἢ τῷ Ἀδάρ, τουτέστι τῷ Φαμενὼθ ἢ τῷ Φαρμουθί. 



Physiologus, Physiologus (redactio prima) 
Section 19, line 6

Ἐὰν οὖν ἔγκυος γένηται, πορεύεται ἐν τῇ Ἰνδίᾳ καὶ λαμβάνει τὸν 
εὐτόκιον λίθον. 



Physiologus, Physiologus (redactio prima) 
Section 34, line 1

Ἔστι δένδρον ἐν τῇ Ἰνδικῇ περιδέξιον καλούμενον, ὁ δὲ καρπὸς αὐτοῦ 
γλυκύτατός ἐστι καὶ χρηστὸς σφόδρα. 



Physiologus, Physiologus (redactio prima) 
Section 44b, line 1

Ἐν δὲ τῇ Ἰνδίᾳ ἦν ἡ παροῦσα πῖνα φυτευτὴ ἐν τῷ βυθῷ τῆς θαλάς-
σης ὑστερουμέν<ῳ> γλυκαίων ὑδάτων· τὸ<ν> Δαιμάϊον μῆνα ὑδατοφορᾷ ἐν 
Ἰνδίᾳ, καὶ δεομένη ἡ πῖνα γλυκαίου ὕδατος, ἐξέρχεται ἄνω τῆς θαλάσσης, 
καὶ βροντᾷ καὶ <ἀ>στράπτει καὶ βρέχει, καὶ ἡ πῖνα δέχεται τὴν βοὴν 
τῆς βροντῆς καὶ τὸ πῦρ <τὸ> φλογ<ίζον> τῆς ἀστραπῆς καὶ τὴν στάξιν 
τοῦ ὕδατος, εὐθὺς δὲ πάλιν πορεύεται εἰς τὰ ἴδια. 



Physiologus, Physiologus (redactio prima) 
Section 46, line t

Περὶ λίθου ἰνδικοῦ. 




Physiologus, Physiologus (redactio prima) 
Section 46, line 1

Ἔστι λίθος ἰνδικός, [ὀνόματι βατράχιος], τοιαύτην φύσιν ἔχων· ἐὰν 
ἄνθρωπος ὑδρωπικὸς τυγχάνῃ, οἱ τεχνῖται ἰατροὶ ζητοῦσι τὸν λίθον ἐκεῖ-
νον, καὶ δεσμεύουσιν αὐτὸν τῷ ὑδρωπικῷ ὥρας τρεῖς, καὶ ὅλα τὰ ὕδατα   
συμπίνει τοῦ ὑδρωπικοῦ ὁ λίθος. 



Physiologus, Physiologus (redactio prima) 
Section 46, line 14

Καλῶς οὖν ὁ Φυσιολόγος ἔλεξε περὶ τοῦ ἰνδικοῦ λίθου. 

\end{greek}




\section{Cornelius Alexander Polyhistor}
\blockquote[From Wikipedia\footnote{\url{http://en.wikipedia.org/wiki/Cornelius_Alexander}}]{Lucius Cornelius Alexander Polyhistor (Ancient Greek: Ἀλέξανδρος ὁ Πολυΐστωρ; flourished in the first half of the 1st century B.C.; also called Alexander of Miletus) was a Greek scholar who was enslaved by the Romans during the Mithridatic War and taken to Rome as a tutor. After his release, he continued to live in Italy as a Roman citizen. He was so productive a writer that he earned the surname polyhistor. The majority of his writings are now lost, but the fragments that remain shed valuable light on antiquarian and eastern Mediterranean subjects.[1] Among his works were historical and geographical accounts of nearly all the countries of the ancient world, and the book Upon the Jews (Ancient Greek: Περὶ Ἰουδαίων) which excerpted many works which might otherwise be unknown.}

\begin{greek}

Cornelius Alexander Polyhist., Fragmenta (0697: 003)
“FHG 3”, Ed. Müller, K.
Paris: Didot, 1841–1870.
Fragment 2, line 3

Agathias II, 25: Πρῶτοι μὲν 
γὰρ, ὧν ἀκοῇ ἴσμεν, Ἀσσύριοι λέγονται ἅπασαν τὴν 
Ἀσίαν χειρώσασθαι, πλὴν Ἰνδῶν τῶν ὑπὲρ Γάγγην 
ποταμὸν ἱδρυμένων. 

Cornelius Alexander Polyhist., Fragmenta 
Fragment t95-97, line 1

                                            ... Ἀλέξανδρος ἐν 
τῷ Περὶ Κύπρου· «Τὴν δὲ Γορδίαν ἀποδοῦναι Χυ-
τρίοις»· καὶ πάλιν· «Εὐρυνόην τῶν Χυτρίων βασιλεὺς 
ἔγημεν.» 
ΙΝΔ*ιΚΑ. 

Cornelius Alexander Polyhist., Fragmenta 
Fragment 95, line 6

Clemens Alex. Strom. III, 7: 
Βραχμᾶναι γοῦν οὔτε ἔμψυχον ἐσθίου-
σιν οὔτε οἶνον πίνουσιν, ἀλλ' οἱ μὲν αὐτῶν καθ' ἑκά-
στην ἡμέραν ὡς ἡμεῖς τὴν τροφὴν προσίενται· ἔνιοι 
δ' αὐτῶν διὰ τριῶν ἡμερῶν, ὥς φησιν Ἀλέξανδρος 
ὁ Πολυΐστωρ ἐν τοῖς Ἰνδικοῖς. 

Cornelius Alexander Polyhist., Fragmenta 
Fragment 96-97, line 1

Stephan. Byz.: <Τοπάζιος>, νῆσος Ἰνδική. 

Cornelius Alexander Polyhist., Fragmenta 
Fragment 146, line 4

Idem IX, 61, de Pyrrhone: Ἤκουσε Βρύσωνος 
τοῦ Στίλπωνος, ὡς Ἀλέξανδρος ἐν Διαδοχαῖς, εἶτ' Ἀνα-
ξάρχῳ ξυνακολουθῶν πανταχοῦ, ὡς καὶ τοῖς Γυμνο-
σοφισταῖς ἐν Ἰνδίᾳ συμμῖξαι καὶ τοῖς Μάγοις. 


\end{greek}

\section{Julius Pollux}
\blockquote[From Wikipedia\footnote{\url{http://en.wikipedia.org/wiki/Julius_Pollux}}]{Julius Pollux (Ἰούλιος Πολυδεύκης, Ioulios Poludeukes) (2nd century) was a Greek[1] or Egyptian[2] grammarian and sophist from Alexandria who taught at Athens, where he was appointed professor of rhetoric at the Academy by the emperor Commodus—on account of his melodious voice, according to Philostratus' Lives of the Sophists. Nothing of his rhetorical works has survived except some of their titles (in the Suda). Pollux was the author of the Onomasticon, a Greek thesaurus or dictionary of Attic synonyms and phrases, arranged not alphabetically but according to subject-matter, in ten books. It supplies in passing much rare and valuable information on many points of classical antiquity—objects in daily life, the theater, politics—and quotes numerous fragments of lost works. Pollux was probably the person satirized by Lucian as a worthless and ignorant person who gains a reputation as an orator by sheer effrontery, and pilloried in his Lexiphanes, a satire upon the affectation of obscure and obsolete words. A first Latin translation, published at Venice in 1502, made Julius Pollux more available to Renaissance antiquaries and scholars, and anatomists, who adopted obscure Greek words for parts of the body. Julius Pollux was invaluable for William Smith's Dictionary of Greek and Roman Antiquities, 1842, etc.}
\begin{greek}

Julius Pollux Gramm., Onomasticon (0542: 001)
“Pollucis onomasticon, 2 vols.”, Ed. Bethe, E.
Leipzig: Teubner, 9.1:1900; 9.2:1931, Repr. 1967; Lexicographi Graeci 9.1–9.2.
Book 1, section 213, line 4

παραφυλακτέον δὲ ὅτι ὁ Ξενοφῶν (R Eq I 15) οἴεται 
τὸν ἵππον ἀστραγάλους ἔχειν, Ἀριστοτέλους (Hist An I p 499 20) 
τοῦ περὶ ταῦτα δεινοῦ φάσκοντος μηδὲν τῶν μωνύχων ἔχειν 
ἀστραγάλους, μηδὲ τὸν ὄνον, μόνον δὲ τὸν Ἰνδικόν, ᾧ καὶ κέρας 
ἐκ τοῦ μετώπου ἐκπεφυκέναι λέγει. 



Julius Pollux Gramm., Onomasticon 
Book 4, section 142, line 4

                                                             τὰ δ' ἔκ-
σκευα πρόσωπα Ἀκταίων ἐστὶ κερασφόρος, ἢ Φινεὺς τυφλός, ἢ 
Θάμυρις τὸν μὲν ἔχων γλαυκὸν ὀφθαλμὸν τὸν δὲ μέλανα, ἢ Ἄργος 
πολυόφθαλμος, ἢ Εὐίππη ἡ Χείρωνος ὑπαλλαττομένη εἰς ἵππον παρ' 
Εὐριπίδῃ, ἢ Τυρὼ πελιδνὴ τὰς παρειὰς παρὰ Σοφοκλεῖ – τοῦτο 
δ' ὑπὸ τῆς μητρυιᾶς Σιδηροῦς πληγαῖς πέπονθεν – ἢ Ἀχιλλεὺς 
ἐπὶ Πατρόκλῳ ἄκομος, ἢ Ἀμυμώνη, ἢ ποταμὸς ἢ ὄρος, ἢ Γοργώ, ἢ 
Δίκη ἢ Θάνατος ἢ Ἐρινὺς ἢ Λύσσα ἢ Οἶστρος ἢ Ὕβρις, ἢ Κένταυρος 
ἢ Τιτὰν ἢ Γίγας ἢ Ἰνδὸς ἢ Τρίτων, τάχα δὲ καὶ Πόλις καὶ †Πρίαμος 
καὶ Πειθὼ καὶ Μοῦσαι καὶ Ὧραι καὶ Μιθάκου Νύμφαι καὶ Πλειάδες 
καὶ Ἀπάτη καὶ Μέθη καὶ Ὄκνος καὶ Φθόνος. 



Julius Pollux Gramm., Onomasticon 
Book 5, section 37, line 3

γενναῖαι κύνες Λάκαιναι, Ἀρκάδες, Ἀργολίδες, Λοκρίδες, 
Κελτικαί, Ἰβηρικαί, Καρῖναι, Κρῆσσαι, Μολοττικαί, Ἐρετρικαί, Ὑρκα-
ναί, Ἰνδικαί. 



Julius Pollux Gramm., Onomasticon 
Book 5, section 38, line 4

               Ἀριστοτέλης (Hist Anim VIII p 607 3) δὲ τὰς Ἰνδι-
κὰς κυνὸς καὶ τίγριδος λέγει τρίτην γενεάν· τὰς γὰρ προτέρας 
δύο ζῷα γίνεσθαι θηριώδη. 



Julius Pollux Gramm., Onomasticon 
Book 5, section 38, line 7

                               Νίκανδρος δ' ὁ Κολοφώνιος (frg 97 Schn)   
τοὺς Ἰνδικοὺς κύνας ἀπογόνους εἶναί φησι τῶν Ἀκταίονος κυνῶν, 
αἳ μετὰ τὴν λύτταν σωφρονήσασαι, διαβᾶσαι τὸν Εὐφράτην ἐπλανή-
θησαν εἰς Ἰνδούς· ὥσπερ καὶ τὰς Χαονίδας καὶ Μολοττίδας ἀπο-
γόνους εἶναί φησι κυνός, ὃν Ἥφαιστος ἐκ χαλκοῦ Δημονησίου 
χαλκευσάμενος, ψυχὴν ἐνθείς, δῶρον ἔδωκε Διὶ κἀκεῖνος Εὐρώπῃ, 
αὕτη δὲ Μίνῳ καὶ Μίνως Πρόκριδι καὶ Πρόκρις Κεφάλῳ. 



Julius Pollux Gramm., Onomasticon 
Book 5, section 41, line 7

                             οἱ δὲ κυναμολγοὶ κύνες εἰσὶ περὶ τὰ ἕλη 
τὰ μεσημβρινά, γάλα δὲ βοῶν ποιοῦνται τὴν τροφήν, καὶ τοὺς ἐπιόντας 
τοῦ θέρους τῷ ἔθνει βοῦς Ἰνδικοὺς καταγωνίζονται, ὡς ἱστορεῖ Κτη-
σίας (frg 62 C. Müller). 



Julius Pollux Gramm., Onomasticon 
Book 5, section 42, line 8

ἔνδοξος δὲ καὶ ὁ Ἠπειρωτικὸς Κέρβερος, καὶ ὁ Ἀλεξάνδρου Περί-
τας, τὸ θρέμμα τὸ Ἰνδικόν· ἐκράτει δ' οὗτος λέοντος, ἑκατὸν μνῶν 
ἐωνημένος, καὶ ἀποθανόντι αὐτῷ πόλιν φησὶ Θεόπομπος (FHG I 
334) Ἀλέξανδρον ἐποικίσαι. 



Julius Pollux Gramm., Onomasticon 
Book 5, section 43, line 2

                                    λέγουσι δὲ τοὺς γενναιοτέρους τῶν 
Ἰνδικῶν ἄλλο μὲν θηρίον ἀπαξιοῦν μεταθεῖν, λέοντι δ' ὡς ἀξιομάχῳ 
προσαγωνίζεσθαι μόνῳ, ἔχεσθαί τ' ὀδὰξ ἐμφύντας, ὥστε κἂν ἁλῷ τὸ 
θηρίον, πολλὰ πράγματα τοὺς κυνηγοὺς ἔχειν ὡς ἀποσπάσαι τοῦ 
θηρίου τοὺς κύνας. 



Julius Pollux Gramm., Onomasticon 
Book 5, section 43, line 6

                      τὸν δ' Ἀλέξανδρον ἐπὶ πείρᾳ λαβόντα παρὰ 
Σωπείθους τοιούτους κύνας ἐν Ἰνδοῖς, πολλὰ θηρίων εἴδη παρα-
βαλεῖν τινὶ τῶν κυνῶν· τὸν δ' ἐκταθέντα κατὰ γῆς ἀτρεμεῖν ὡς 
οὐδὲν πρὸς αὐτὸν οὖσαν τὴν θήραν τὴν ἄτιμον. 



Julius Pollux Gramm., Onomasticon 
Book 7, section 75, line 1

καὶ μὴν καὶ τὰ βύσσινα, καὶ ἡ βύσσος λίνου τι εἶδος παρ' Ἰνδοῖς. 

\end{greek}



\section{Thessalus of Tralles}
\blockquote[From Wikipedia\footnote{\url{http://en.wikipedia.org/wiki/Thessalus_of_Tralles}}]{
Jump to: navigation, search

Thessalus of Tralles (fl. circa 70-95 AD) was a famous Roman physician and early adherent to the Methodic school of medicine.[1] He lived in Rome,[2] where he was the court physician of Emperor Nero. It was here that he died and was buried, and his tomb was to be seen on the Via Appia.[3]

He was from Tralles in Lydia. He was the son of a weaver, and followed the same employment himself in his youth.[3] This, however, he soon gave up, and, though he had a poor general education, he embraced the medical profession, by which he acquired for a time a great reputation, and amassed a large fortune. He adopted the principles of the Methodic school, but modified and developed them. He appears to have exalted himself at the expense to his predecessors;[2] asserting that none of them had contributed to the advance of medical science,[3] and boasting that he himself could teach the art of healing in six months. Galen frequently mentions him, but always in terms of contempt,[4] and is often abusive towards him.

He supported a method of treatment that he named metasyncrisis.[5] His object was, in obstinate chronic cases, where other remedies failed, to attempt a thorough change in the fundamental constitution of the organism (syncrisis). He began by the application, for three days, of strong vegetable remedies, both internally and externally, together with which, a strict regimen and emetics were applied. This was the preparation to a system of fasting, which concluded with a course of restoratives.[6]

Interestingly, Thessalus regarded the chicory plant to be an herb of the sun.[7] He wrote several medical works, of which only the titles and a few sentences remain.[6]}

\begin{greek}

Thessalus Astrol., Med., De virtutibus herbarum (e cod. Paris. gr. 2502 + Vindob. med. gr. 23) (1004: 001)
“Thessalos von Tralles”, Ed. Friedrich, H.–V.
Meisenheim am Glan: Hain, 1968; Beiträge zur klassischen Philologie 28.
Book 1, chapter 12, section 3, line 7

                                                  δʹ, λυκίου 
Ἰνδικοῦ δρ. 



Thessalus Astrol., Med., De virtutibus herbarum (e cod. Paris. gr. 2502 + Vindob. med. gr. 23) 
Book 1, chapter 12, section 4, line 5

                    ιβʹ, λυκίου <Ἰνδικοῦ> δρ. 



Thessalus Astrol., Med., De virtutibus herbarum (e cod. Paris. gr. 2502 + Vindob. med. gr. 23) 
Book 2, chapter 6, section 8, line 3

                              λυκίου Ἰνδικοῦ δρ. 



Thessalus Astrol., Med., De virtutibus herbarum (e cod. Monac. 542) (1004: 003)
“Thessalos von Tralles”, Ed. Friedrich, H.–V.
Meisenheim am Glan: Hain, 1968; Beiträge zur klassischen Philologie 28.
Book 1, chapter 4, section 5, line 4

                                                      ϛʹ, νάρ-
δου Ἰνδικῆς δρ. 



Thessalus Astrol., Med., De virtutibus herbarum (e cod. Monac. 542) 
Book 1, chapter 9, section 3, line 3

ἐὰν δὲ μετὰ ῥοδίνου μίξῃς τὸν χυλὸν καὶ ἀλείψῃς ⌊σου⌋ 
τὰς ὄψεις, ἡδέως ὑπὸ πάντων θεαθήσῃ· ἐὰν δὲ μετὰ μέλιτος 
καὶ λυκίου Ἰνδικοῦ καὶ ὄξους δριμυτάτου {καὶ} τοῦ χυλοῦ   
δῷς γυναικὶ ἐπιχρίσασθαι, λαμπρυνεῖ τὰς ὄψεις καὶ τετανω-
τέρας ποιήσει. 



Thessalus Astrol., Med., De virtutibus herbarum (e cod. Monac. 542) 
Book 1, chapter 9, section 6, line 6

                                               ιβʹ, λυκίου 
Ἰνδικοῦ δρ. 



Thessalus Astrol., Med., De virtutibus herbarum (e cod. Monac. 542) 
Book 1, chapter 9, section 7, line 2

   ἐὰν 
δὲ {μετὰ} μέλιτος καὶ λυκίου Ἰνδικοῦ καὶ τοῦ χυλοῦ ἴσον 
ἴσῳ μίξῃς καὶ τὸν καυλὸν τοῦ μορίου περιχρίσῃς, ἑτοιμότε-
ρος ἔσῃ πρὸς συνουσίαν καὶ τῇ πλησιαζομένῃ ἡδονὴν ἀπεργά-
σῃ. 



Thessalus Astrol., Med., De virtutibus herbarum (e cod. Monac. 542) 
Book 1, chapter 11, section 4, line 5

                        βʹ⌋, λυκίου Ἰνδικοῦ δρ. 

\end{greek}


\section{Xenophon of Ephesus}
\blockquote[From Wikipedia\footnote{\url{http://en.wikipedia.org/wiki/Xenophon_of_Ephesus}}]{arch

Xenophon of Ephesus (fl. 2nd century–3rd century CE?) was a Greek writer. His surviving work is the Ephesian Tale of Anthia and Habrocomes, one of the earliest novels as well as one of the sources for Shakespeare's Romeo and Juliet.

He is not to be confused with the earlier and more famous Athenian soldier and historian, Xenophon.
}
\begin{greek}

Xenophon Scr. Erot., Ephesiaca (0641: 001)
“Xénophon d'Éphèse. Les Éphésiaques ou le roman d'Habrocomès et d'Anthia”, Ed. Dalmeyda, G.
Paris: Les Belles Lettres, 1926, Repr. 1962.
Book 3, chapter 11, section 2, line 2

   Ἔρχεται δή τις εἰς 
Ἀλεξάνδρειαν ἐκ τῆς Ἰνδικῆς τῶν ἐκεῖ βασιλέων κατὰ 
θέαν τῆς πόλεως καὶ κατὰ χρείαν ἐμπορίας, Ψάμμις τὸ 
ὄνομα. 



Xenophon Scr. Erot., Ephesiaca 
Book 4, chapter 1, section 5, line 2

Ἐνταῦθα ἔγνωσαν λῃστεύειν· πολὺ γὰρ πλῆθος ἐμπόρων 
τὸ διοδεῦον ἦν τῶν τε ἐπ' Αἰθιοπίαν καὶ τῶν ἐπὶ Ἰνδικὴν 
φοιτώντων· ἦν δὲ αὐτοῖς καὶ τὸ λῃστήριον ἀνθρώπων 
πεντακοσίων. 



Xenophon Scr. Erot., Ephesiaca 
Book 4, chapter 3, section 3, line 5

   Ἡ δὲ ὡς Ἀλεξάν-
δρειαν παρελθοῦσα ἐγένετο ἐν Μέμφει, ηὔχετο τῇ Ἴσιδι 
στᾶσα πρὸ τοῦ ἱεροῦ «ὦ μεγίστη θεῶν, μέχρι μὲν νῦν 
ἁγνὴ μένω νομιζομένη σή, καὶ γάμον ἄχραντον Ἁβροκόμῃ 
τηρῶ· τοὐντεῦθεν δὲ ἐπὶ Ἰνδοὺς ἔρχομαι, μακρὰν μὲν τῆς 
Ἐφεσίων γῆς, μακρὰν δὲ τῶν Ἁβροκόμου λειψάνων. 

\end{greek}

\section{Pausanias}
\blockquote[From Wikipedia\footnote{\url{http://en.wikipedia.org/wiki/Pausanias}}]{Pausanias (play /pɔːˈseɪniəs/; Ancient Greek: Παυσανίας Pausanías) was a Greek traveler and geographer of the 2nd century AD, who lived in the times of Hadrian, Antoninus Pius and Marcus Aurelius. He is famous for his Description of Greece (Ἑλλάδος περιήγησις), a lengthy work that describes ancient Greece from firsthand observations, and is a crucial link between classical literature and modern archaeology. This is how Andrew Stewart assesses him:[1]}
\begin{greek}

Pausanias Attic., Ἀττικῶν ὀνομάτων συναγωγή (1569: 001)
“Untersuchungen zu den attizistischen Lexika”, Ed. Erbse, H.
Berlin: Akademie–Verlag, 1950; Abhandlungen der deutschen Akademie der Wissenschaften zu Berlin, Philosoph.–hist. Kl..
Alphabetic letter kappa, entry 25*, line 1

<Κερκῖται>· ἔθνος Ἰνδικόν, ὃ χρῆται μικρῷ πηδαλίῳ τῷ καλουμένῳ κερκέτῃ· 
<οὗτος δὲ> μηχάνημα σιδηροῦν, ὃ ἐξαρτᾶται τῆς νεώς, ὅταν ᾖ ἄνεμος, πρὸς τὸ ἀντέχειν <ὡς ὁ 
δελφίς>. 

\end{greek}

\section{Philo Mech.}%???
Right Philo?
\blockquote[From Wikipedia\footnote{\url{}}]{}
\begin{greek}

Philo Mech., Parasceuastica et poliorcetica (1599: 002)
“Exzerpte aus Philons Mechanik B. VII und VIII”, Ed. Diels, H., Schramm, E.
Berlin: Reimer, 1920; Abhandlungen der preussischen Akademie der Wissenschaften, Philosoph.–hist. Kl., no. 12.
Thevenot page 89, line 43

     (48) συμφέρει δὲ καὶ κηπία ἐν 
ταῖς ἰδίαις οἰκίαις καὶ ἐν ταῖς ἀκροπόλεσιν καὶ 
ἔν τε τοῖς <ἄλσεσι καὶ> τεμένεσι τῶν θεῶν κα-
         τασκευάζειν ὑγείας 
ἕνεκεν καὶ ἐάν τις συμβαίνῃ πολιορκία· φυτευ-  
θεισῶν γὰρ συκεῶν καὶ φοινίκων, ἐὰν ἡ πόλις 
φέρῃ, καὶ σπαρείσης τῆς Ἰνδικῆς καὶ Ἑλληνικῆς κο-
λοκύνθης καὶ ἄρων καὶ κράμβης καὶ θρίδακος 
καὶ τῶν ἄλλων λαχάνων οὐ μικρὰν παρέχεται 
ἐπικουρίαν. 
\end{greek}


\section{Philostratus Major}
\blockquote[From Wikipedia\footnote{\url{}}]{}
\begin{greek}

Philostratus Major Soph., Imagines (1600: 001)
“Philostrati maioris imagines”, Ed. Benndorf, O., Schenkl, K.
Leipzig: Teubner, 1893.
Book 1, chapter 28, section 5, line 7

γράφει δὴ Λοκρίδας Λακαίνας Ἰνδικὰς Κρητικάς, τὰς 
μὲν ἀγερώχους καὶ ὑλακτούσας, ** τὰς δὲ ἐννοούσας, 
αἱ δὲ μεθέπουσι καὶ σεσήρασι κατὰ τοῦ ἴχνους. 



Philostratus Major Soph., Imagines 
Book 1, chapter 29, section 1, line 2

ΠΕΡΣΕΥΣ


 Ἀλλ' οὐκ Ἐρυθρά γε αὕτη θάλασσα οὐδ' 
Ἰνδοὶ ταῦτα, Αἰθίοπες δὲ καὶ ἀνὴρ Ἕλλην ἐν Αἰθιο-
πίᾳ. 



Philostratus Major Soph., Imagines 
Book 2, chapter 12, section t, line 1

   ταῦτα αἱ Νύμφαι πανσυδί, σὺ δὲ κατὰ δή-
μους αὐτὰς ὅρα· τὰ μὲν γὰρ τῶν Ναΐδων εἴδη – ῥανί-
δας ἀπορραίνουσιν αὗται τῆς κόμης – ὁ δὲ περὶ ταῖς 
Βουκόλοις αὐχμὸς οὐδὲν φαυλότερος τῆς δρόσου, αἱ 
δὲ Ἀνθοῦσαι τὰς χαίτας ἐκπεφύκασιν ὑακινθίνοις 
ὁμοίως ἄνθεσιν. 
ΠΙΝΔ*αΡΟΣ


 Οἶμαι θαῦμά σοι εἶναι τὰς μελίττας οὕτω 
γλίσχρως γεγραμμένας, ὧν γε καὶ προνομαία δήλη 
καὶ πόδες καὶ πτερὰ καὶ τὸ χρῶμα τῆς στολῆς οὐκ 
ἀτακτοῦσιν, ἴσα τῇ φύσει διαποικιλλούσης αὐτὰ τῆς 
γραφῆς. 

\end{greek}



\section{Pseudo-Dioscorides}%???
\blockquote[From Wikipedia\footnote{\url{}}]{}
\begin{greek}

Date for pseudographia?

Pseudo-Dioscorides Med., De lapidibus (1118: 003)
“Les lapidaires de l'antiquité et du Moyen Age, vol. 2.1”, Ed. Ruelle, C.É.
Paris: Leroux, 1898.
Section 1, line 3

       
ΠΕΡΙ ΛΙΘΩΝ


 <Λιθάργυρος>· Ἀγαρηνοὶ <μάρτικ>· ἡ μὲν ἐκ μολιβδίτιδος ἄμμου γεννᾶται 
χωνευομένη ἄχρι τελείας ἐκπυρώσεως, ἡ δὲ ἐξ ἀργύρου, εἴτε ἐκ μολίβδου· γίνεται 
δὲ ἐν Ἀττικῇ καὶ Ἰνδίᾳ καὶ Σικελίᾳ καὶ Ἱσπανίᾳ. 



Pseudo-Dioscorides Med., De lapidibus 
Section 14, line 1

<Λίθος ἱερακίτης> καὶ ἰνδικὸς περιαπτόμενος μηρῷ δεξιῷ τὰς αἱμορροΐδας 
ἀναξηραίνουσιν, ὡς καὶ ἡμεῖς ἐπειράθημεν· ὁ δὲ Διογένης ἐν τῷ περὶ λίθων οὕτω 
φησίν· Ἱερακίτης λίθος ὑπόχλωρος μέν ἐστι καὶ πρὸς τὸ μέλαν ἐπικλίνει· δύναμιν 
δὲ ἔχει ἀναξηραντικὴν αἱμορροΐδων. 



Pseudo-Dioscorides Med., De lapidibus 
Section 15, line 1

<Λίθος ἰνδικὸς> τὴν μὲν χρόαν ἐστὶν ὑπόπυρρος· τριβόμενος δὲ πορφυροῦν   
ἀνίησι χυλόν· καὶ μετ' ἀκράτου οἴνου πινόμενος αἱμοπτυϊκοὺς ὠφελεῖ· καὶ αἱμορ-
ροΐδας ἀναξηραίνει. 

\end{greek}



\section{Evagrius Scholasticus}
\blockquote[From Wikipedia\footnote{\url{http://en.wikipedia.org/wiki/Evagrius_Scholasticus}}]{Evagrius Scholasticus (Greek: Εὐάγριος Σχολαστικός) was a Syrian scholar and intellectual living in the 6th century AD, and an aide to the patriarch Gregory of Antioch.[1] His surviving work, Ecclesiastical History, comprises a six-volume collection concerning the Church's history from the First Council of Ephesus (431) to Maurice’s reign during his life.

Evagrius’s only surviving work, Ecclesiastical History, addresses the history of the Eastern Roman Empire from the official beginning of the Nestorian controversy at the First Council of Ephesus in 431 to the time in which he was writing, 593. The book’s contents focus mainly on religious matters, describing the events surrounding notable bishops and holy men.

The editio princeps was published in 1544 under the name of Robertus Stephanus (better known as Robert Estienne). John Christopherson, bishop of Chichester, made a Latin translation of the Ecclesiastical History, which was published after his death in 1570. Translations into English appeared much later: the first was by Edward Walford, which was published at London in 1846; Michael Whitby's translation was published in 2001 by Liverpool University Press as part of their "Texts in Translation Series."

Some historians, particularly Pauline Allen, allege that Evagrius’s Chalcedonian theological stance directly influenced his selection of information, in order to defend Chalcedonian-aligned political agents against negative reputation.[10] Whitby, however, emphasizes the legal scholar’s acceptance and inclusion of information written by other historians who adopted opposing stances, when he discerned that their accounts were reliable.[11] For example, Evagrius Scholasticus relies heavily on Zachariah’s textual study of history even though he was a monophysite, occasionally omitting minor facets of his work that explicitly promote his theology, but largely considering him to be dependable. Allen also reasons that Evagrius built on Zachariah’s work because his was the only comprehensive historical account of events taking place from Theodoret of Cyrus’s time till his own era. Unfortunately, however, Zachariah’s original manuscripts have been lost.[12]

Evagrius is much less critical of the Emperor Justinian and his wife Theodora, in comparison with Procopius, who described the two as physically manifest demons. Because of regional affiliations Evagrius depicts the emperor in a more sympathetic light, praising his moderate approach to justice and his restraint towards excessive persecution, yet still decrying his heresy and displays of wealth. Evagrius’s ambivalence to Justinian is especially evident when he describes him as a virtuous man yet blind to impending defeat in his self-initiated war with Persia.[13] Chesnut also comments on how the Roman historian and scholar endues his “Ecclesiastical History” with a dramatic style, using themes from classical Greek tragedies to characterize Justinian’s life, particularly Fortune’s grand fluctuations.[14]

Evagrius builds upon the documents written by Zachariah, Symeon Stylites the Elder, Eustathius of Epiphania, John Malalas, Zosimus, and Procopius of Caesarea.[15]

“The Ecclesiastical History” is considered an important and relatively authoritative account of the timeline it traces, since Evagrius draws on other scholars’ material, explicitly acknowledging his sources. He meticulously organizes information taken from other written historical works in order to validate his account more effectively than other theological scholars of his time, thus diminishing confusion for future historian’s interested in studying his work.[16] However, historians acknowledge that there are serious logical errors inherent in Evagrius’s surviving work, which is common for its epoch, namely the problematic chronological sequencing and skimming over of undeniably notable events such as major wars and other secular events. When the scholar mentions important occasions in his own life, lack of chronological labeling is especially evident - which can provide complications to those analyzing his book.[17]}
\begin{greek}
Evagrius Scholasticus Scr. Eccl., Historia ecclesiastica (2733: 001)
“The ecclesiastical history of Evagrius with the scholia”, Ed. Bidez, J., Parmentier, L.
London: Methuen, 1898, Repr. 1979.
Page 135, line 10

                                              Καὶ ὁ ἕτερος 
δὲ Λογγῖνος τὸ πολὺ τῆς τυραννίδος συνέχων, ὁ ἐπίκλην 
Σελινούντιος, καὶ Ἴνδης σὺν αὐτῷ, πρὸς Ἰωάννου τοῦ 
ἐπίκλην Κυρτοῦ στέλλονται τῷ Ἀναστασίῳ ζωγρίαι· ὃ 
μάλιστα τόν τε βασιλέα τούς τε Βυζαντίους τεθεράπευκε, 
θριάμβου δίκην ἀνὰ τὰς λεωφόρους τῆς πόλεως ἀνά τε 
τὴν ἱπποδρομίαν Λογγίνου τε καὶ Ἴνδου περιενεχθέντων, 
καὶ τῶν ἐκ σιδήρου πεποιημένων ἁλύσεων ἀνὰ τοὺς αὐ-
χένας καὶ τὰς χεῖρας περιβεβλημένων. 



Evagrius Scholasticus Scr. Eccl., Historia ecclesiastica 
Page 222, line 15

                                                    Ἡ δέ 
γε ἀμπεχόνην χρυσόπαστον, ἁλουργίδι καὶ λίθοις Ἰνδῶν 
κεκοσμημένην, στεφάνους τε χρυσῷ πολλῷ καὶ ταῖς ἐκ 
λίθων ποικιλίαις τε καὶ διαυγείαις τιμαλφεστάτους, ἅ-
παντάς τε τοὺς ἐν ἀξιώσεσι περὶ τὴν αὐλὴν καὶ στρατείαις 
ἐναριθμίους, κηρούς τε γαμηλίους ἐξάπτοντας μεγαλο-
πρεπῶς τε ἐσταλμένους καὶ ἐξ ὧν γνωρίζοιντο, καὶ τὴν 
νυμφαγωγὸν πανήγυριν ἀνυμνοῦντας· ὥστε τῆς πομπῆς 
ἐκείνης μηδὲν τῶν ἐν ἀνθρώποις σεμνοπρεπέστερον ἢ 
εὐδαιμονέστερον γενέσθαι πώποτε. 

\end{greek}


\section{Porphyrius}
\blockquote[From Wikipedia\footnote{\url{http://en.wikipedia.org/wiki/Porphyry_(philosopher)}}]{Porphyry of Tyre (Greek: Πορφύριος, Porphyrios, AD 234–c. 305) was a Neoplatonic philosopher who was born in Tyre.[1] He edited and published the Enneads, the only collection of the work of his teacher Plotinus. He also wrote many works himself on a wide variety of topics.[2] His Isagoge, or Introduction, is an introduction to logic and philosophy,[3] and in Latin translation it was the standard textbook on logic throughout the Middle Ages.[4] In addition, through several of his works, most notably Philosophy from Oracles and Against the Christians, he was involved in a controversy with a number of early Christians,[5] and his commentary on Euclid's Elements was used as a source by Pappus of Alexandria.[6]}
\begin{greek}


Porphyrius Phil., Vita Plotini (2034: 001)
“Plotini opera, vol. 1”, Ed. Henry, P., Schwyzer, H.–R.
Leiden: Brill, 1951.


Porphyrius Phil., Vita Plotini 
Section 3, line 17

                                                Καὶ ἀπ' ἐκεί-
νης τῆς ἡμέρας συνεχῶς τῷ Ἀμμωνίῳ παραμένοντα τοσαύ-
την ἕξιν ἐν φιλοσοφίᾳ κτήσασθαι, ὡς καὶ τῆς παρὰ τοῖς 
Πέρσαις ἐπιτηδευομένης πεῖραν λαβεῖν σπεῦσαι καὶ τῆς 
παρ' Ἰνδοῖς κατορθουμένης. 



Porphyrius Phil., De abstinentia (2034: 003)
“Porphyrii philosophi Platonici opuscula selecta, 2nd edn.”, Ed. Nauck, A.
Leipzig: Teubner, 1886, Repr. 1963.
Book 3, section 3, line 18

οὐδὲ γὰρ τῆς Ἰνδῶν οἱ Ἕλληνες οὐδὲ τῆς Σκυθῶν ἢ 
Θρᾳκῶν ἢ Σύρων οἱ ἐν τῇ Ἀττικῇ τραφέντες· ἀλλ' 
ἴσα κλαγγῇ γεράνων ὁ τῶν ἑτέρων τοῖς ἑτέροις ἦχος 
προσπίπτει. 



Porphyrius Phil., De abstinentia 
Book 3, section 4, line 28

         ἡ δ' Ἰνδικὴ ὕαινα, ἣν κοροκότταν οἱ ἐπιχώ-
ριοι καλοῦσι, καὶ ἄνευ διδασκάλου οὕτω φθέγγεται 
ἀνθρωπικῶς, ὡς καὶ ἐπιφοιτᾶν ταῖς οἰκίαις καὶ καλεῖν 
ὃν <ἂν> ἴδῃ εὐχείρωτον αὑτῇ, καὶ μιμεῖταί γε τοῦ φιλ-
τάτου καὶ ᾧ ἂν πάντως ὑπακούσειεν ὁ κληθεὶς τὸ 
φθέγμα· ὡς καίπερ εἰδότας τοὺς Ἰνδοὺς διὰ τῆς ὁμοιό-
τητος ἐξαπατᾶσθαι καὶ ἀναλίσκεσθαι ἐξιόντας τε καὶ 
πρὸς τὸ φθέγμα ὑπακούοντας. 



Porphyrius Phil., De abstinentia 
Book 4, section 17, line 1

      ἀλλ' οὗτοι μὲν δίκας καὶ παρὰ θεοῖς καὶ παρ' 
ἀνθρώποις ὧν ἁμαρτάνουσιν ἐκτίνοντες αὐτῇ πρῶτον 
τῇ τοιαύτῃ διαθέσει ἱκανὴν τιμωρίαν διδόασιν· ἡμεῖς 
δ' ἔτι τῶν ἀλλοφύλων ἐθνῶν ἑνὸς μνημονεύσαντες 
ἐνδόξου τε καὶ δικαίου περί τε τὰ θεῖα πεπιστευμένου   
εὐσεβοῦς, ἐπ' ἄλλα μεταβησόμεθα. Ἰνδῶν γὰρ τῆς 
πολιτείας εἰς πολλὰ νενεμημένης, ἔστι τι γένος παρ' 
αὐτοῖς τὸ τῶν θεοσόφων, οὓς γυμνοσοφιστὰς καλεῖν 
εἰώθασιν Ἕλληνες. 



Porphyrius Phil., De abstinentia 
Book 4, section 17, line 12

       ἔχει δὲ τὰ κατ' αὐτοὺς τοῦτον τὸν τρόπον, 
ὡς Βαρδησάνης ἀνὴρ Βαβυλώνιος ἐπὶ τῶν πατέρων 
ἡμῶν γεγονὼς καὶ ἐντυχὼν τοῖς περὶ Δάνδαμιν πε-
πεμμένοις Ἰνδοῖς πρὸς τὸν Καίσαρα ἀνέγραψεν. 



Porphyrius Phil., De abstinentia 
Book 4, section 17, line 16

                                                       πάν-
τες γὰρ Βραχμᾶνες ἑνός εἰσι γένους· ἐξ ἑνὸς γὰρ 
πατρὸς καὶ μιᾶς μητρὸς πάντες κατάγουσιν· Σαμαναῖοι 
δὲ οὐκ εἰσὶ τοῦ αὐτοῦ γένους, ἀλλ' ἐκ παντὸς τοῦ 
τῶν Ἰνδῶν ἔθνους, ὡς ἔφαμεν, συνειλεγμένοι· οὔτε 
δὲ βασιλεύεται Βραχμὰν οὔτε συντελεῖ τι τοῖς ἄλλοις. 



Porphyrius Phil., In Platonis Timaeum commentaria (fragmenta) (2034: 009)
“Porphyrii in Platonis Timaeum commentariorum fragmenta”, Ed. Sodano, A.R.
Naples: n.p., 1964.
Book 2, fragment 28, line 22

δαίοις ἡ εὐχὴ μάλιστα προσήκει, διότι συναφὴ πρὸς τὸ θεῖόν ἐστι, τῷ δὲ 
ὁμοίῳ τὸ ὅμοιον συνάπτεσθαι φιλεῖ, τοῖς δὲ θεοῖς ὁ σπουδαῖος ὁμοιότατος, 
καὶ διότι ἐν <φρουρᾷ> ὄντες οἱ τῆς ἀρετῆς ἀντεχόμενοι καὶ ὑπὸ τοῦ   
σώματος ὡς δεσμωτηρίου συνειλημμένοι δεῖσθαι τῶν θεῶν ὀφείλουσι 
περὶ τῆς ἐντεῦθεν μεταστάσεως, καὶ ὅτι ὡς παῖδας πατέρων ἀποσπα-
σθέντας εὔχεσθαι προσήκει περὶ τῆς πρὸς τοὺς ἀληθινοὺς ἡμῶν πατέ-
ρας, τοὺς θεούς, ἐπανόδου, καὶ ὅτι ἀπάτορές τινες ἄρα καὶ ἀμήτορες 
ἐοίκασιν εἶναι οἱ μὴ ἀξιοῦντες εὔχεσθαι μηδὲ ἐπιστρέφειν εἰς τοὺς 
κρείττονας, καὶ ὅτι καὶ ἐν πᾶσι τοῖς ἔθνεσιν οἱ σοφίᾳ διενεγκόντες 
περὶ εὐχὰς ἐσπούδασαν, Ἰνδῶν μὲν Βραχμᾶνες, Μάγοι δὲ Περσῶν, 
Ἑλλήνων δὲ οἱ θεολογικώτατοι, οἳ καὶ τελετὰς κατεστήσαντο καὶ μυ-
στήρια· Χαλδαῖοι δὲ καὶ τὸ ἄλλο θεῖον ἐθεράπευσαν καὶ αὐτὴν τὴν 
ἀρετὴν τῶν θεῶν θεὸν εἰπόντες ἐσέφθησαν, πολλοῦ δέοντες διὰ τὴν 
ἀρετὴν ὑπερφρονεῖν τῆς ἱερᾶς θρησκείας· καὶ ἐπὶ πᾶσι τούτοις, ὅτι 
μέρος ὄντας τοῦ παντὸς δεῖσθαι προσήκει τοῦ παντός· παντὶ γὰρ ἡ 
πρὸς τὸ ὅλον ἐπιστροφὴ παρέχεται τὴν σωτηρίαν· εἴτε οὖν ἀρετὴν 
ἔχεις, παρακλητέον σοι τὸ τὴν ὅλην ἀρετὴν προειληφός· τὸ γὰρ πᾶν 
ἀγαθὸν αἴτιόν ἐστι καὶ σοὶ τοῦ ἀγαθοῦ τοῦ σοὶ προσήκοντος· εἴτε σω-
ματικόν τι ζητεῖς ἀγαθόν, ἔστιν ἡ παντὸς σώματος ἐν τῷ κόσμῳ

συ-



Porphyrius Phil., Quaestionum Homericarum ad Iliadem pertinentium reliquiae (2034: 014)
“Porphyrii quaestionum Homericarum ad Iliadem pertinentium reliquiae, fasc. 1 \& 2”, Ed. Schrader, H.
Leipzig: Teubner, 1:1880; 2:1882.
Iliad book 1, section 340, line 11

                                                                  αὐτὸς δὲ ὁ βα-
σιλεὺς οὐ μεταπέμπεται Νέστορα σκοπούμενον περὶ τῶν συμφερόντων 
ἀλλ' αὐτὸς ἄπεισιν· <ἥδε δέ οἱ κατὰ θυμὸν ἀρίστη φαίνετο βουλή,   
Νέστορ' ἔπι πρῶτον Νηλήιον ἐλθέμεν ἀνδρῶν> (Κ 17). παρ' 
Ἰνδοῖς τε τοὺς Βραχμᾶνας, οἵπερ εἰσὶ παρ' αὐτοῖς οἱ φιλόσοφοι, λόγος 
τοὺς βασιλέας ἀπαντῶντας προσκυνεῖν. 

\end{greek}

\section{Harpocration}
\blockquote[From Wikipedia\footnote{\url{http://en.wikipedia.org/wiki/Harpocration}}]{Valerius Harpocration (Ancient Greek: Οὐαλέριος or Βαλέριος Ἀρποκρατίων) was a Greek grammarian of Alexandria, probably working in the 2nd century CE. He is possibly the Harpocration mentioned by Julius Capitolinus (Life of Verus, 2) as the Greek tutor of Lucius Verus (2nd century AD); some authorities place him much later, on the ground that he borrowed from Athenaeus. His Lexicon of the Ten Orators, which has come down to us in an incomplete form, contains, in more or less alphabetical order, notes on well-known events and persons mentioned by the orators, and explanations of legal and commercial expressions. As nearly all the lexicons to the Greek orators have been lost, Harpocration's work is especially valuable. Amongst his authorities were the writers of Atthides (histories of Attica), the grammarian Didymus Chalcenterus, Dionysius of Halicarnassus, and the lexicographer Dionysius, son of Tryphon. The book also contains contributions to the history of Attic oratory and Greek literature generally. His Collection of Florid Expressions, a sort of anthology or chrestomathy attributed to him by the Suda, is lost, but elements of it survive in later lexica. A series of articles in the margin of a Cambridge manuscript of the Lexicon forms the basis of the Lexicon rhetoricum Cantabrigiense by Peter Paul Dobree.}

\begin{greek}
Harpocration Gramm., Lexicon in decem oratores Atticos (1389: 001)
“Harpocrationis lexicon in decem oratores Atticos, vol. 1”, Ed. Dindorf, W.
Oxford: Oxford University Press, 1853, Repr. 1969.
Page 103, line 9

                      εἴη δ' ἂν σκεῦός τι πρὸς τὸ κρατῆρας ἢ λέβητας ἤ 
τι τούτων οὐκ ἀλλότριον ἐπικεῖσθαι ἐπιτήδειον, ὡς Καλλίξενός τε ἐν 
δʹ περὶ Ἀλεξανδρείας ὑποσημαίνει καὶ Δαΐμαχος ὁ Πλαταιεὺς ἐν βʹ 
περὶ Ἰνδικῆς. 

\end{greek}


\section{Alexander of Aphrodisias}%???

Check this is the right Alexander

\blockquote[From Wikipedia\footnote{\url{http://en.wikipedia.org/wiki/Alexander_of_Aphrodisias}}]{Alexander of Aphrodisias (Ancient Greek: Ἀλέξανδρος ὁ Ἀφροδισιεύς; fl. 200 AD) was a Peripatetic philosopher and the most celebrated of the Ancient Greek commentators on the writings of Aristotle. He was a native of Aphrodisias in Caria, and lived and taught in Athens at the beginning of the 3rd century, where he held a position as head of the Peripatetic school. He wrote many commentaries on the works of Aristotle, and still extant are those on the Prior Analytics, Topics, Meteorology, Sense and Sensibilia, and Metaphysics. Several original treatises also survive, and include a work On Fate, in which he argues against the Stoic doctrine of necessity; and one On the Soul. His commentaries on Aristotle were considered so useful that he was styled, by way of pre-eminence, "the commentator" (ὁ ἐξηγητής).}
\begin{greek}

Alexander Phil., Problemata (lib. 1–2) [Sp.] (0732: 002)
“Physici et medici Graeci minores, vol. 1”, Ed. Ideler, J.L.
Berlin: Reimer, 1841, Repr. 1963.
Book 2, section 60, line 40

                                                            ὁμοίως 
δὲ καὶ ἐπὶ τῶν πέντε αἰσθήσεων εὑρίσκομεν ἐπί τινων μίαν 
μᾶλλον ἐπιτεταμένην· οἷον ἐν κυνὶ μὲν τὴν ὀσφρητικήν· ἐν 
δὲ ἀετῷ τὴν ὀπτικήν, ἐν δὲ τῷ λεγομένῳ πρωτογεύστῃ Ἰν-
δικῷ ζῴῳ ὄντι τὴν γευστικήν, ἐν δὲ ὄφεσι καί τισιν ὀρνέοις 
τὴν ἁπτικήν· κατὰ δὲ τοὺς χῆνας τὴν ἀκουστικήν. 



Alexander Phil., In Aristotelis metaphysica commentaria (0732: 004)
“Alexandri Aphrodisiensis in Aristotelis metaphysica commentaria”, Ed. Hayduck, M.
Berlin: Reimer, 1891; Commentaria in Aristotelem Graeca 1.
Page 379, line 31

                  πλείω γὰρ οὗτοι ταὐτὰ πάθη ἔχουσιν ὡς ἐπὶ τὸ πλεῖστον 
τῶν καθ' ἃ διαφέρουσιν ἀλλήλων, οἷον Γερμανὸς Γερμανῷ καὶ Ἰνδὸς Ἰνδῷ. 



Alexander Phil., In Aristotelis metaphysica commentaria 
Page 697, line 9

καὶ διὰ τοῦτο, φησί, καὶ τὴν αἴσθησιν καὶ τὴν ἐγρήγορσιν καὶ τὴν νόησιν 
ἥδιστά φαμεν, ὅτι ἐνέργειαί τινες οὖσαι ἰνδάλματά τινα καὶ σκιαί εἰσιν τῆς 
ἣν ἐνεργοῦμεν ἐνεργείας καὶ ἣν ζῶμεν ζωήν, ὅταν ὁ ἡμέτερος νοῦς γένηταί 
πως τὰ νοητά. 



Alexander Phil., In Aristotelis meteorologicorum libros commentaria (0732: 008)
“Alexandri Aphrodisiensis in Aristotelis meteorologicorum libros commentaria”, Ed. Hayduck, M.
Berlin: Reimer, 1899; Commentaria in Aristotelem Graeca 3.2.
Page 57, line 15

             ῥεῖν δέ φησι καὶ τὸν Ἰνδὸν ἐξ αὐτοῦ, μέγιστον ὄντα ποτα-
μὸν ἁπάντων. 



Alexander Phil., In Aristotelis meteorologicorum libros commentaria 
Page 105, line 20

                          τὸ γὰρ ἀπὸ Ἡρακλείων στηλῶν μέχρι τῆς 
Ἰνδικῆς, ὅ ἐστι κατὰ μῆκος καὶ γνωρίμως πᾶν οἰκούμενον (εἰσὶ δὲ αἱ 
μὲν Ἡράκλειαι στῆλαι πρὸς δυσμαῖς, ἡ δ' Ἰνδικὴ πρὸς ταῖς ἀνατολαῖς) 
τοῦ ἐξ Αἰθιοπίας, ἥ ἐστιν ἐσχάτη οἰκουμένη πρὸς τῷ θερινῷ τροπικῷ 
καὶ τῇ μεσημβρίᾳ, πρὸς τὴν Μαιῶτιν καὶ τοὺς ἐσχάτους τῆς Σκυθίας τό-
πους, οἵ εἰσι πρὸς τῇ ἄρκτῳ, ἐγγὺς διπλάσιόν ἐστιν· οὕτως γὰρ ἔχειν 
φησὶν αὐτὰ πρὸς ἄλληλα μεγέθους, ὡς πέντε πρὸς τρία, τοῦ τε πλοῦ καὶ 
τῶν ὁδῶν συναριθμουμένων τε καὶ συντιθεμένων. 



Alexander Phil., In Aristotelis meteorologicorum libros commentaria 
Page 105, line 29

                                                    καίτοι τὸ μὲν ἐπὶ πλά-
τος πᾶν τὸ οἰκούμενον εἴληπται μέχρι τῶν ἀοικήτων ἐφ' ἑκάτερα προδή-
λως, τὰ δ' ἐφ' ἑκάτερα τῶν κατὰ τὸ μῆκος οἰκουμένων ἐπί τε Ἡρακλείας 
στήλας καὶ ἐπὶ τὴν Ἰνδικὴν τῷ ὑπὸ θαλάσσης διαλαμβάνεσθαι, οὐ τῷ ἀοίκητα 
εἶναι οὐ φαίνεται συνείροντα, ὡς πᾶσαν ἐν κύκλῳ φαίνεσθαι περιοικουμένην. 



Alexander Phil., Problemata (lib. 3–4) [Sp.] (0732: 017)
“Alexandri Aphrodisiensis quae feruntur problematorum liber iii et iiii”, Ed. Usener, H., 1859; Programm Gymnasium Joachimsthal.
Book 3, section 2, line 5

                                                                Τὸ δὲ σάκχαρον παρὰ τοῖς Ἰνδοῖς 
οὕτω λεγόμενον μέλιτός ἐστι πῆξις, τοῦ ἡλίου τὴν ἐν τῷ ἀέρι δρόσον πηγνύοντος ἐπὶ τὸ γλυκύ, 
ὥσπερ καὶ ἐν τῷ ὄρει τῷ Λιβάνῳ καλουμένῳ γίγνεται τοιοῦτον· ἔστι δὲ ὅμοιον χόνδρῳ ἅλατος, 
λευκὸν εὔθρυπτον γλυκύ. 

\end{greek}


\section{Pseudo--Sosthenes}%???
\blockquote[From Wikipedia\footnote{\url{http://en.wikipedia.org/wiki/Sosthenes}}]{Sosthenes (Greek, "safe in strength") was the chief ruler of the synagogue at Corinth, who, according to the New Testament, was seized and beaten by the mob in the presence of Gallio, the Roman governor, when he refused to proceed against Paul at the instigation of the Jews (Acts 18:12-17). The motives of this assault against Sosthenes (an assault made by the Greeks according to Acts 18:17) are not recorded.

Some identify him with one whom Paul calls "Sosthenes our brother," a convert to the faith and co-author of the First Epistle to the Corinthians (1 Corinthians 1:1-2). It is not clear whether this identification is tenable. It has also been alleged that Sosthenes is a later name of Crispus, who is mentioned in Acts 18:8 and 1 Corinthians 1:14. [1]

He is listed among the Seventy Disciples.}
\begin{greek}

[Sosthenes] Hist., Fragmenta (2568: 002)
“FHG 4”, Ed. Müller, K.
Paris: Didot, 1851.
Fragment 1, line 9

           Ζεὺς δι' ἐρωτικὴν ἐπιθυμίαν ἐκ Λύκτου, πό-
λεως Κρητικῆς, Ἄργην νύμφην ἁρπάσας, ἀπήνεγκεν εἰς 
ὄρος τῆς Αἰγύπτου, Ἄργιλλον καλούμενον· καὶ ἐγέννη-
σεν ἐξ αὐτῆς υἱὸν, καλούμενον Διόνυσον· ὃς ἀκμάσας εἰς 
τιμὴν τῆς μητρὸς τὸν λόφον Ἄργιλλον μετωνόμασε· 
στρατολογήσας δὲ Πᾶνας καὶ Σατύρους, ἰδίοις σκή-
πτροις Ἰνδοὺς ὑπέταξε· νικήσας δὲ καὶ Ἰβηρίαν, Πᾶνα 
κατέλιπεν ἐπιμελητὴν τῶν τόπων· ὃς τὴν χώραν ἀπ' 
αὐτοῦ Πανίαν μετωνόμασεν· ἣν οἱ μεταγενέστεροι πα-
ραγώγως Σπανίαν προσηγόρευσαν· καθὼς ἱστορεῖ Σω-
σθένης ἐν ιγʹ Ἰβηρικῶν. 
\end{greek}


\section{Dionysius Halicarnassensis}

\blockquote[From Wikipedia\footnote{\url{http://en.wikipedia.org/wiki/Dionysius_Halicarnassensis}}]{Dionysius of Halicarnassus (Ancient Greek: Διονύσιος Ἀλεξάνδρου Ἁλικαρνᾱσσεύς, Dionysios son of Aléxandros, of Halikarnassós, c. 60 BC–after 7 BC) was a Greek historian and teacher of rhetoric, who flourished during the reign of Caesar Augustus. His literary style was Atticistic — imitating Classical Attic Greek in its prime.

His great work, entitled Ῥωμαϊκὴ Ἀρχαιολογία (Rhōmaikē archaiologia, Roman Antiquities), embraced the history of Rome from the mythical period to the beginning of the First Punic War. It was divided into twenty books, of which the first nine remain entire, the tenth and eleventh are nearly complete, and the remaining books exist in fragments in the excerpts of Constantine Porphyrogenitus and an epitome discovered by Angelo Mai in a Milan manuscript. The first three books of Appian, and Plutarch's Life of Camillus also embody much of Dionysius.

His chief object was to reconcile the Greeks to the rule of Rome, by dilating upon the good qualities of their conquerors and also by arguing, using more ancient sources, that the Romans were genuine descendants (book 1,11) of the older Greeks.[2] According to him, history is philosophy teaching by examples, and this idea he has carried out from the point of view of the Greek rhetorician. But he has carefully consulted the best authorities, and his work and that of Livy are the only connected and detailed extant accounts of early Roman history.}

\begin{greek}
Dionysius Halicarnassensis Hist., Rhet., Antiquitates Romanae (0081: 001)
“Dionysii Halicarnasei antiquitatum Romanarum quae supersunt, 4 vols.”, Ed. Jacoby, K.
Leipzig: Teubner, 1:1885; 2:1888; 3:1891; 4:1905, Repr. 1967.
Book 7, chapter 70, section 4, line 6

                                             μάλιστα 
δὲ τοῦτο πεπόνθασιν οἱ βάρβαροι διὰ πολλὰς αἰτίας,   
ἃς οὐ καιρὸς ἐν τῷ παρόντι λέγειν, καὶ χρόνος οὐθεὶς 
μέχρι τοῦ παρόντος ἀπομαθεῖν ἢ παρανομῆσαί τι περὶ 
τοὺς ὀργιασμοὺς τῶν θεῶν ἔπεισεν οὔτ' Αἰγυπτίους 
οὔτε Λίβυας οὔτε Κελτοὺς οὔτε Σκύθας οὔτ' Ἰνδοὺς 
οὔτ' ἄλλο βάρβαρον ἔθνος οὐδὲν ἁπλῶς· εἰ μή τινες 
ὑφ' ἑτέρων ἐξουσίᾳ ποτὲ γενόμενοι τὰ τῶν κρατησάν-
των ἠναγκάσθησαν ἐπιτηδεύματα μεταλαβεῖν. 



Dionysius Halicarnassensis Hist., Rhet., Antiquitates Romanae 
Book 20, chapter 12, section 3, line 6

                     Vales. Ambr. 
 Ἀναβάντων δὲ τῶν σὺν τῷ Πύρρῳ μετὰ τῶν 
ἐλεφάντων αἴσθησιν οἱ Ῥωμαῖοι λαβόντες σκυμνίον 
ἐλέφαντος τιτρώσκουσιν, ὃ πολλὴν ἀκοσμίαν τοῖς Ἕλ-
λησιν ἐνεποίησε καὶ φυγήν· οἱ δὲ Ῥωμαῖοι δύο μὲν 
ἐλεφάντας ἀποκτείνουσιν, ὀκτὼ δὲ κατακλείσαντες εἰς 
χωρίον ἀνέξοδον παραδόντων τῶν ἐπ' αὐτοῖς Ἰνδῶν 
ζῶντας παραλαμβάνουσι, τῶν δὲ στρατιωτῶν πολὺν 
φόνον ἐργάζονται. 
\end{greek}




\section{Apollonius Phil.}

See Philostratus' \emph{Life of Apollonius}.\footnote{\url{http://en.wikipedia.org/wiki/Life_of_Apollonius_of_Tyana}}

\blockquote[From Wikipedia\footnote{\url{http://en.wikipedia.org/wiki/Apollonius_of_Tyana}}]{Apollonius of Tyana (Ancient Greek: Ἀπολλώνιος ὁ Τυανεύς; ca. 15?–ca. 100? CE[2]) was a Greek Neopythagorean philosopher from the town of Tyana in the Roman province of Cappadocia in Asia Minor. Little is known about him with certainty. Being a 1st-century orator and philosopher around the time of Christ, he was compared with Jesus of Nazareth by Christians in the 4th century[3] and by various popular writers in modern times.



Apollonius was born into a respected and wealthy Greek family,[4][5] his dates however are uncertain. His primary biographer, Philostratus the Elder (c.170–247 CE) places him c. 3 BCE to 97 CE.[6] Others agree that he was roughly a contemporary of Jesus of Nazareth. Charles P. Eells[7] states that his date of birth was three years before Jesus, whose date of birth is also uncertain. However, Philostratus, in his Life of Apollonius of Tyana, places him staying in the court of King Vardanes I of Parthia for a while, who ruled between c.40–47 CE. Apollonius began a five year silence at about the age of 20, and after the completion of this silence travelled to Mesopotamia and Iran. Philostratus also mentions emperors Nero, Vespasian, Titus, Domitian, and Nerva at various points throughout Apollonius’ life. Given this information, a timeline of roughly the years 15–98 CE can be established for his adult life.



Philostratus devoted two and a half of the eight books of his Life of Apollonius (1.19–3.58) to the description of a journey of his hero to India. According to Philostratus' Life, en route to the Far East, Apollonius reached Hierapolis Bambyce (Manbij) in Syria (not Nineveh, as some scholars believed), where he met Damis, a native of that city who became his lifelong companion. Pythagoras, whom the Neo-Pythagoreans regarded as an exemplary sage, was believed to have travelled to India. Hence such a feat made Apollonius look like a good Pythagorean who spared no pains in his efforts to discover the sources of oriental piety and wisdom. As some details in Philostratus’ account of the Indian adventure seem incompatible with known facts, modern scholars are inclined to dismiss the whole story as a fanciful fabrication, but not all of them rule out the possibility that the Tyanean actually did visit India.[23]

What seemed to be independent evidence showing that Apollonius was known in India has now been proved to be forged. In two Sanskrit texts quoted by Sanskritist Vidhushekhara Bhattacharya in 1943[24] he appears as "Apalūnya", in one of them together with Damis (called "Damīśa"), it is claimed that Apollonius and Damis were Western yogis, who later on were converted to the correct Advaita philosophy.[25] Some have believed that these Indian sources derived their information from a Sanskrit translation of Philostratus’ work (which would have been a most uncommon and amazing occurrence), or even considered the possibility that it was really an independent confirmation of the historicity of the journey to India.[26] Only in 1995 were the passages in the Sanskrit texts proven to be interpolations by a modern (late 19th century) forger.[27]}

\begin{greek}

Apollonius Phil., Apollonii epistulae [Dub.] (0619: 003)
“Flavii Philostrati opera, vol. 1”, Ed. Kayser, C.L.
Leipzig: Teubner, 1870, Repr. 1964.
Epistle 59, line 2

      Βασιλεὺς Βαβυλωνίων Γάρμος Νεο-
γύνδῃ Ἰνδῶν βασιλεῖ. 



Apollonius Phil., Apollonii epistulae [Dub.] 
Epistle 59, line 4

                           Εἰ μὴ περίεργος ἦς, οὐκ 
ἂν ἦς ἐν τοῖς ἀλλοτρίοις πράγμασι δίκαιος, οὐδὲ 
ἂν ἄρχων ἐν Ἰνδοῖς ἐδίκαζες Βαβυλωνίοις. 

\end{greek}

\section{Ammonius Saccas}%???
 
\blockquote[From Wikipedia\footnote{\url{http://en.wikipedia.org/wiki/Ammonius_Saccas}}]{Ammonius Saccas (3rd century AD) (Ancient Greek: Ἀμμώνιος Σακκᾶς) was a Greek philosopher from Alexandria who was often referred to as one of the founders of Neoplatonism. He is mainly known as the teacher of Plotinus, whom he taught for eleven years from 232 to 243. He was undoubtably the biggest influence on Plotinus in his development of Neoplatonism, although little is known about his own philosophical views. Later Christian writers stated that Ammonius was a Christian, but it is now generally assumed that there was a different Ammonius of Alexandria who wrote biblical texts.}
\begin{greek}
Ammonius Phil., In Aristotelis librum de interpretatione commentarius (4016: 003)
“Ammonius in Aristotelis de interpretatione commentarius”, Ed. Busse, A.
Berlin: Reimer, 1897; Commentaria in Aristotelem Graeca 4.5.
Page 30, line 33

          τῶν οὖν τοιούτων φωνῶν χωρίζει τὸ ὄνομα προστεθὲν τὸ <κατὰ 
συνθήκην,> ταὐτὸν σημαῖνον τῷ θέσει· συνέθεντο γὰρ πρὸς ἀλλήλους 
Ἕλληνες μὲν τοῖσδε τοῖς ὀνόμασι τὰ πράγματα καλεῖν, Ἰνδοὶ δὲ ἑτέροις 
καὶ Αἰγύπτιοι ἄλλοις, καὶ οἱ αὐτοὶ τὰ αὐτὰ ποτὲ μὲν ἄλλοις ποτὲ δὲ 
ἑτέροις. 
\end{greek}


\section{Agathemerus}

\blockquote[From Wikipedia\footnote{\url{http://en.wikipedia.org/wiki/Agathemerus}}]{Agathemerus (Greek: Ἀγαθήμερος) was a Greek geographer who during the Roman Greece period published a small two-part geographical work titled A Sketch of Geography in Epitome (τῆς γεωγραφίας ὑποτυπώσεις ἐν ἐπιτομῇ), addressed to his pupil Philon. The son of Orthon, Agathemerus is speculated to have lived in the 3rd century. Although much is not known about Agathemerus historically, he lived after Ptolemy, whom he often quotes, and before the formation of Constantinople on the site of Byzantium by Constantine the Great in 328 AD as he mentions only the old city Byzantium. From his speaking of Albion ἐν ᾗ στρατόπεδα ἵδρυται, it has been thought that he wrote not very long after the erection of the wall of Severus. This is probably true, but the language is scarcely definite enough to establish the point.[1]

Agathemerus's work consists chiefly of extracts from Ptolemy, Artemidorus Ephesius and other earlier writers. In his work, he gives a short account of the various forms assigned to the Earth by previous geographers. He calculated the distances between land masses and seas, and then laid down important distances on the inhabited part of the Earth using the stadiametric method.}

\begin{greek}
Agathemerus Geogr., Geographiae informatio (0090: 001)
“Geographi Graeci minores, vol. 2”, Ed. Müller, K.
Paris: Didot, 1861, Repr. 1965.
Section 2, line 12

                                                  Πρῶτος δὲ 
Δημόκριτος, πολύπειρος ἀνὴρ, συνεῖδεν, ὅτι προμήκης 
ἐστὶν ἡ γῆ, ἡμιόλιον τὸ μῆκος τοῦ πλάτους ἔχουσα· 
συνῄνεσε τούτῳ καὶ Δικαίαρχος ὁ Περιπατητικός· Εὔ-
δοξος δὲ τὸ μῆκος διπλοῦν τοῦ πλάτους, ὁ δὲ Ἐρατο-
σθένης πλεῖον τοῦ διπλοῦ· Κράτης δὲ ὡς ἡμικύκλιον, 
Ἵππαρχος δὲ τραπεζοειδῆ, ἄλλοι οὐροειδῆ, Ποσειδώνιος 
δὲ ὁ Στωϊκὸς σφενδονοειδῆ καὶ μεσόπλατον ἀπὸ νότου 
εἰς βορρᾶν, στενὴν [δὲ] πρὸς ἕω καὶ δύσιν, τὰ πρὸς εὖ-
ρον δ' ὅμως πλατύτερα [τὰ] πρὸς τὴν Ἰνδικήν. 



Agathemerus Geogr., Geographiae informatio 
Section 7, line 8

                                             Ἔθνη δὲ 
οἰκεῖν τὰ πέρατα κατ' ἀπηλιώτην Βακτριανοὺς, κατ' 
εὖρον Ἰνδοὺς, κατὰ Φοίνικα Ἐρυθρὰν θάλασσαν καὶ 
Αἰθιοπίαν, κατὰ νότον τὴν ὑπὲρ Αἴγυπτον Αἰθιοπίαν, 
κατὰ λευκόνοτον τοὺς ὑπὲρ Σύρτεις Γαράμαντας, κατὰ 
Λίβα Αἰθίοπας δυσμικοὺς [τοὺς] ὑπὲρ Μαύρους, κατὰ 
ζέφυρον Στήλας καὶ ἀρχὰς Λιβύης καὶ Εὐρώπης, κατ' 
ἀργέστην Ἰβηρίαν τὴν νῦν Ἱσπανίαν, κατὰ δὲ 
Θρασκίαν [Κελτοὺς καὶ τὰ ὅμορα, κατὰ δ' ἀπαρκτίαν] 
τοὺς ὑπὲρ Θρᾴκην Σκύθας, κατὰ δὲ βορρᾶν Πόντον, 
Μαιῶτιν, Σαρμάτας· κατὰ καικίαν Κασπίαν θάλασσαν 
καὶ Σάκας. 



Agathemerus Geogr., Geographiae informatio 
Section 15, line 5

Μῆκος δὲ τῆς οἰκουμένης ἀπὸ Γάγγου εἰς τὰ 
Γάδειρα σταδίων μυριάδων ϛʹ καὶ ͵ηφμεʹ, οὔτως· ἀπὸ 
μὲν Γάγγου ποταμοῦ ἕως Μυριάνδρου τῆς ἐν Ἰσσικῷ 
κόλπῳ σταδίων μυριάδων δʹ καὶ ͵αψκεʹ οὕτως· ἀπὸ 
Γάγγου ἐπὶ ἐκβολὰς Ἰνδοῦ ποταμοῦ σταδίων μυρίων 
͵ϛ· ἀπὸ Ἰνδοῦ ἕως Κασπίων πυλῶν μυρίων ͵ετʹ· ἐπὶ 
τὸν Εὐφράτην σταδίων μυρίων νʹ· εἰς Μυρίανδρον στα-
δίων τοεʹ. 

\end{greek}


\section{Celsus}
\blockquote[From Wikipedia\footnote{\url{http://en.wikipedia.org/wiki/Celsus}}]{Celsus (Greek: Κέλσος) was a 2nd century Greek philosopher and opponent of Early Christianity. He is known for his literary work, The True Word (Account, Doctrine or Discourse) (Λόγος Ἀληθής), written about by Origen. This work, c. 177[1] is the earliest known comprehensive attack on Christianity.

According to Origen, Celsus was the author of an anti-Christian work titled The True Word. This work was lost, but we have Origen's account of it in his writings.[2] It was during the reign of Philip the Arab that Origen received this work for rebuttal.[3] Origen's refutation of The True Word contained its text, interwoven with Origen's replies. Origen's work has survived and thereby preserved Celsus' work with it.[4]

Celsus seems to have been interested in Ancient Egyptian religion,[5] and he seemed to know of Jewish logos-theology, both of which suggest The True Word was composed in Alexandria.[6] Celsus wrote at a time when Christianity was being actively persecuted[7] and when there seems to have been more than one emperor.[8][9][10][11][12]

As an anti-Christian Greek philosopher, Celsus mounted an attack on Christianity. Celsus wrote that Jesus's father was a Roman soldier named Panthera. The views of Celsus drew responses from Origen who considered it a fabricated story.[13][14] Raymond E. Brown states that the story of Pantera is a fanciful explanation of the birth of Jesus which includes very little historical evidence - Brown's analysis does not presuppose the doctrine of the "virgin birth," but cites the lack of historical evidence for Celsus' assertion.[15] In addition, Celsus addressed the miracles of Jesus, holding that "Jesus performed His miracles by sorcery (γοητεία)":[16][17][18]}
\begin{greek}

Celsus Phil., Ἀληθὴς λόγος (1248: 001)
“Der Ἀληθὴς λόγος des Kelsos”, Ed. Bader, R.
Stuttgart: Kohlhammer, 1940; Tübinger Beiträge zur Altertumswissenschaft 33.
Chapter 1, section 14c, line 6

ὅρα οὖν εὐθέως τὸ φίλαυτον τοῦ τοῖσδε μέν τισι πιστεύ-
οντος ὡς σοφοῖς ἔθνεσι, τῶνδε δὲ καταγινώσκοντος ὡς πάντῃ ἀνοή-
των. ἄκουε γὰρ λέγοντος τοῦ Κέλσου ὅτι ἔστιν ἀρχαῖος ἄνωθεν 
λόγος, περὶ ὃν δὴ ἀεὶ καὶ τὰ ἔθνη τὰ σοφώτατα καὶ πόλεις καὶ   
ἄνδρες σοφοὶ κατεγένοντο. καὶ οὐκ ἐβουλήθη ἔθνος σοφώτατον εἰπεῖν 
κἂν παραπλησίως Αἰγυπτίοις καὶ Ἀσσυρίοις καὶ Ἰνδοῖς καὶ Πέρσαις 
καὶ Ὀδρύσαις καὶ Σαμόθρᾳξι καὶ Ἐλευσινίοις τοὺς Ἰουδαίους. 



Celsus Phil., Ἀληθὴς λόγος 
Chapter 1, section 24, line 6

Μετὰ ταῦτά φησιν, ὅτι οἱ αἰπόλοι καὶ ποιμένες ἕνα ἐνόμισαν 
θεὸν εἴτε Ὕψιστον εἴτ' Ἀδωναῖον εἴτ' Οὐράνιον εἴτε Σαβαὼθ 
εἴτε καὶ ὅπῃ καὶ ὅπως χαίρουσιν ὀνομάζοντες τόνδε τὸν κόσμον· καὶ 
πλεῖον οὐδὲν ἔγνωσαν. καὶ ἐν τοῖς ἑξῆς δέ φησι μηδὲν διαφέρειν 
τῷ παρ' Ἕλλησι φερομένῳ ὀνόματι τὸν ἐπὶ πᾶσι θεὸν καλεῖν Δία ἢ 
τῷ δεῖνα, φέρ' εἰπεῖν, παρ' Ἰνδοῖς ἢ τῷ δεῖνα παρ' Αἰγυπτίοις.   
 Ἴδωμεν δὲ τίνα τρόπον συκοφαντεῖ Ἰουδαίους ὁ πάντ' 
ἐπαγγελλόμενος εἰδέναι Κέλσος λέγων αὐτοὺς σέβειν ἀγγέλους 
καὶ γοητείᾳ προσκεῖσθαι, ἧς ὁ Μωϋσῆς αὐτοῖς γέγονεν ἐξηγητής. 
 ἐπαγγέλλεται δὲ διδάξειν ἑξῆς, πῶς καὶ Ἰουδαῖοι ὑπὸ ἀμαθίας 
ἐσφάλησαν ἐξαπατώμενοι. 
 ἐπαγγειλάμενος δ' ὁ Κέλσος ὕστερον διδάξειν τὰ περὶ Ἰουδαίων 
πρῶτον ποιεῖται τὸν λόγον περὶ τοῦ σωτῆρος ἡμῶν ὡς γενομένου 
ἡγεμόνος τῇ καθὸ Χριστιανοί ἐσμεν γενέσει ἡμῶν καί φησιν 
αὐτὸν πρὸ πάνυ ὀλίγων ἐτῶν τῆς διδασκαλίας ταύτης καθηγήσασθαι 




Celsus Phil., Ἀληθὴς λόγος 
Chapter 5, section 34, line 23

                                      Σκύθαις γε μὴν καὶ ἀνθρώπους 
δαίνυσθαι καλόν· Ἰνδῶν δέ εἰσιν οἳ καὶ τοὺς πατέρας ἐσθίοντες 
ὅσια δρᾶν νομίζουσι. 



Celsus Phil., Ἀληθὴς λόγος 
Chapter 5, section 34, line 35

                               Δαρεῖος δὲ μετὰ ταῦτα καλέσας Ἰνδῶν τοὺς 
καλεομένους Καλατίας, οἳ τοὺς γονέας κατεσθίουσιν, εἴρετο παρεόντων 
τῶν Ἑλλήνων καὶ δι' ἑρμηνέος μανθανόντων τὰ λεγόμενα, ἐπὶ τίνι 
χρήματι δεξαίατ' ἂν τελευτῶντας τοὺς πατέρας κατακαίειν πυρί· 
οἱ δὲ ἀμβώσαντες μέγα εὐφημέειν μιν ἐκέλευον. 



Celsus Phil., Ἀληθὴς λόγος 
Chapter 6, section 80, line 9

                      .. ἔνθεον ἔθνος εἶναι τῷ Κέλσῳ δοκοῦσιν, ἀλλὰ 
καὶ Ἰνδοί, ὧν τινας ἐν τοῖς προειρημένοις ἔλεγε καὶ ἀνθρωπείων 
γεγεῦσθαι σαρκῶν. 

\end{greek}


\section{Ephraem Syrus}
\blockquote[From Wikipedia\footnote{\url{http://en.wikipedia.org/wiki/Ephraem_Syrus}}]{Ephrem the Syrian (Syriac: ܡܪܝ ܐܦܪܝܡ ܣܘܪܝܝܐ, Mār Efrêm Sûryāyâ;[1] Greek: Ἐφραίμ ὁ Σῦρος; Latin: Ephraem Syrus; ca. 306 – 373) was a Syriac deacon and a prolific Syriac-language hymnographer and theologian of the 4th century.[2][3][4][5] His works are hailed by Christians throughout the world and many denominations venerate him as a saint. He has been declared a Doctor of the Church in Roman Catholicism. He is especially beloved in the Syriac Orthodox Church.}

\begin{greek}
Ephraem Syrus Theol., Sermo asceticus (4138: 005)
“Ὁσίου Ἐφραίμ τοῦ Σύρου ἔργα, vol. 1”, Ed. Phrantzoles, Konstantinos G.
Thessalonica: Το περιβόλι της Παναγίας, 1988, Repr. 1995.
Page 130, line 2

                                                                            Θαυμαστὸν   
γὰρ ἦν τὸ πρᾶγμα, ἀδελφοί, θεωροῦντες τὰ ἄγρια ζῷα συναγόμενα εἰς ἕν· ἐλέ-
φαντας μὲν ἀπὸ Ἰνδικῆς καὶ Περσίδος ἐρχομένους· λέοντας καὶ παρδάλεις μετὰ 
προβάτων καὶ αἰγῶν μιγάδας, καὶ μηδὲν ἀδικοῦντας· ἑρπετά τε καὶ πετεινὰ ἄνευ 
τινὸς διώκοντος ἐρχόμενα καὶ κύκλῳ τῆς κιβωτοῦ αὐλιζόμενα· καὶ ταῦτα ἐπὶ 
ἡμέρας ἱκανάς· αὐτόν τε τὸν Νῶε μετὰ σπουδῆς κατασκευάζειν τὴν κιβωτόν, καὶ 
ἐμβοῶντα αὐτοῖς, μετανοεῖτε, καὶ οὐκ ἠνείχοντο. 



Ephraem Syrus Theol., De recordatione mortis et de uirtute ac de diuitiis (4138: 062)
“Ὁσίου Ἐφραίμ τοῦ Σύρου ἔργα, vol. 4”, Ed. Phrantzoles, Konstantinos G.
Thessalonica: Το περιβόλι της Παναγίας, 1992.
Page 251, line 14

                          Τίς ἀνακλιθήσεται ἐπὶ ταῖς ἀργυρενδέτοις κλίναις καὶ 
τῶν ἐξ Ἰνδικῆς ἐχόντων τὸν ἀπαρτισμόν; 



Ephraem Syrus Theol., Sermo in Ionam prophetam et de paenitentia Niniuitarum (4138: 153)
“Ὁσίου Ἐφραίμ τοῦ Σύρου ἔργα, vol. 7”, Ed. Phrantzoles, Konstantinos G.
Thessalonica: Το περιβόλι της Παναγίας, 1998.
Page 319, line 8

                                                                                       Ὡς Ἰνδοὶ 
ἐφαίνοντο ἐκ τῶν μελλόντων κακῶν. 

\end{greek}


\section{Aelius Herodianus}

\blockquote[From Wikipedia\footnote{\url{http://en.wikipedia.org/wiki/Aelius_Herodianus}}]{Aelius Herodianus (Latin; Greek Αἴλιος Ἡρωδιανός) or Herodian (fl. 2nd c. CE) was one of the most celebrated grammarians of Greco-Roman antiquity. He is usually known as Herodian except when there is a danger of confusion with the historian also named Herodian.

He was the son of Apollonius Dyscolus and was born in Alexandria. From there he seems to have moved to Rome, where he gained the favour of the emperor Marcus Aurelius, to whom he dedicated a work on prosody.}

\begin{greek}

Aelius Herodianus et Pseudo-Herodianus Gramm., Rhet., De prosodia catholica (0087: 001)
“Grammatici Graeci, vol. 3.1”, Ed. Lentz, A.
Leipzig: Teubner, 1867, Repr. 1965.
Part+volume 3,1, page 13, line 12

                        <Ἀτιντάν> υἱὸς Μακεδόνος καὶ ἔθνος Μακεδονίας. 
<Βραχμάν>20 Ἰνδικὸν ἔθνος σοφώτατον, οὓς καὶ Βράχμας καλοῦσιν. 



Aelius Herodianus et Pseudo-Herodianus Gramm., Rhet., De prosodia catholica 
Part+volume 3,1, page 19, line 1

Τὰ εἰς <ων> παραληγόμενα <α> μακρῷ βαρύνεται, <Ἄων> ἔθνος Βοιω-  
τίας, <Δάων> ἔθνος τῆς Ἰνδικῆς ἀπὸ Δάονος, <Χάων> ἔθνος Ἠπείρου. 



Aelius Herodianus et Pseudo-Herodianus Gramm., Rhet., De prosodia catholica 
Part+volume 3,1, page 43, line 9

Τὰ εἰς <εξ> ὀνόματα βαρύνεται, <Λέλεξ, Βέρεξ> ἔθνος μεταξὺ Ἰν-
δίας καὶ Αἰθιοπίας ὡς Τιμοκράτης ὁ Ἀδραμυττηνός, <ἐπίτεξ, ἀγχίτεξ, 
πινυτάλεξ> καὶ εἴ τι ὅμοιον. 



Aelius Herodianus et Pseudo-Herodianus Gramm., Rhet., De prosodia catholica 
Part+volume 3,1, page 52, line 3

                                                                 Βαρύνεται   
δὲ τὰ εἰς <ας> καθαρόν, <Νικίας, Λυσίας, Ἀρχίας, Λοξίας, Γορ-
γίας, Ἐρυξίας, δρακοντίας, ὀνοματίας, κοππατίας, ταρα-
ξίας, κοχλίας, τραυματίας, Παπίας, Ὠπίας> ἔθνος Ἰνδικόν. 



Aelius Herodianus et Pseudo-Herodianus Gramm., Rhet., De prosodia catholica 
Part+volume 3,1, page 52, line 4

Ἑκαταῖος Ἀσίᾳ «ἐν δὲ αὐτοῖσι οἰκέουσι ἄνθρωποι παρὰ τὸν Ἰνδὸν πο-
ταμὸν Ὠπίαι, ἐν δὲ τεῖχος βασιλήϊον. 



Aelius Herodianus et Pseudo-Herodianus Gramm., Rhet., De prosodia catholica 
Part+volume 3,1, page 52, line 6

                                              μέχρι τούτου Ὠπίαι, ἀπὸ δὲ 
τούτων ἐρημίη μέχρις Ἰνδῶν». 



Aelius Herodianus et Pseudo-Herodianus Gramm., Rhet., De prosodia catholica 
Part+volume 3,1, page 52, line 12

        <Καλατίας> γένος Ἰνδικόν. 



Aelius Herodianus et Pseudo-Herodianus Gramm., Rhet., De prosodia catholica 
Part+volume 3,1, page 53, line 23

                                                                      <Σάλγας> 
ποταμὸς τῆς Μαυριτανίας, <Βησσύγας> ποταμὸς τῆς Ἰνδικῆς. 



Aelius Herodianus et Pseudo-Herodianus Gramm., Rhet., De prosodia catholica 
Part+volume 3,1, page 54, line 22

* Τὰ εἰς <κας> βαρύνονται, <Πελέκας, Περδίκκας, Κώκας, 
Κρίκας> ποταμός, <Σάκας, Λάκας, Πολύκκας> ποταμὸς Μακεδονίας, 
<Ματάκας> ὄνομα εὐνούχου, <Ἰνδύκας, Κοτύκας> βασιλεὺς Παφλαγο-
νίας. 



Aelius Herodianus et Pseudo-Herodianus Gramm., Rhet., De prosodia catholica 
Part+volume 3,1, page 60, line 6

Τὰ εἰς <βης> δισύλλαβα παραληγόμενα φωνήεντι βαρύνεται <Λάβης, 
λέβης, Κέβης, Κάβης, Χάβης, Σίβης>20, Ἰνδικὸν ἔθνος, <Βύβης> πό-
λις κατὰ Πευκετίους καὶ τὸ ἐθνικὸν οἱ Βύβαι ὁμοφώνως ὡς Λοκροί καὶ 
Δελφοί. 



Aelius Herodianus et Pseudo-Herodianus Gramm., Rhet., De prosodia catholica 
Part+volume 3,1, page 62, line 5

                                Στράβων ιζʹ (p. 802), «καὶ Λύκων πόλις 
καὶ Μένδης, ὅπου τὸν Πᾶνα τιμῶσι καὶ τὸν τράγον» <Πάνδης> καὶ 
<Σίνδης> ἔθνη Ἰνδικά, <Ἔσδης> ἔθνος Ἰβηρικόν. 



Aelius Herodianus et Pseudo-Herodianus Gramm., Rhet., De prosodia catholica 
Part+volume 3,1, page 62, line 24

                     <Κέλτης>· οὕτως γὰρ Στράβων φησὶ τοὺς Κελτούς. 
<Κώφης>20 Ἰνδικὸς ποταμός, ὡς Στράβων «Χοάσπης εἰς τὸν Κώφην ἐμ-
βάλλει» (p. 697) καὶ πάλιν «μετὰ τὸν Κώφην ὁ Ἰνδός, εἶτα ὁ Ὑδά-
σπης, εἶτα ὁ Ἀκεσίνης καὶ ὕστατος ὁ Ὕπανις» (ibid.). 



Aelius Herodianus et Pseudo-Herodianus Gramm., Rhet., De prosodia catholica 
Part+volume 3,1, page 62, line 25

<Κώφης>20 Ἰνδικὸς ποταμός, ὡς Στράβων «Χοάσπης εἰς τὸν Κώφην ἐμ-
βάλλει» (p. 697) καὶ πάλιν «μετὰ τὸν Κώφην ὁ Ἰνδός, εἶτα ὁ Ὑδά-
σπης, εἶτα ὁ Ἀκεσίνης καὶ ὕστατος ὁ Ὕπανις» (ibid.). 



Aelius Herodianus et Pseudo-Herodianus Gramm., Rhet., De prosodia catholica 
Part+volume 3,1, page 66, line 18

                                         τὸ μὲν <α> οἷον <Ἰλιάδης, Μενοιτιάδης, 
Πυλάδης, Ἀλκιβιάδης· Μιλτιάδης, Δημάδης> ἐκ τοῦ Δημεά-
δης· <Ἀργεάδης> ὁ Ἀργεῖος· <Δειράδης>, ἀφ' οὗ δῆμος Δειράδες, 
<Δολογκιάδης> οἱ Δόλογκοι ἔθνος Θρᾴκης· <Ἐνδυμιωνιάδης> οἱ 
Ἐπειοί, <Ἠλιάδης> ὁ Ἠλεῖος, <Βερενικιάδης> ὁ Βερενικεύς, <Παρο-
πανισσάδης> οἱ Παροπανίσσῳ ὄρει Ἰνδικῆς παροικοῦντες. 



Aelius Herodianus et Pseudo-Herodianus Gramm., Rhet., De prosodia catholica 
Part+volume 3,1, page 68, line 33

                    <Ὑδάρκης> ἔθνος Ἰνδικόν. 



Aelius Herodianus et Pseudo-Herodianus Gramm., Rhet., De prosodia catholica 
Part+volume 3,1, page 69, line 2

                                                                       <Ὀξυδράκης> 
ἔθνος Ἰνδικόν, ἀφ' ὧν σώσας Ἀλέξανδρον Πτολεμαῖος σωτὴρ ἐκλήθη· 
οἱ δὲ ψεῦδος τὸ περὶ τῶν Ὀξυδρακῶν. 



Aelius Herodianus et Pseudo-Herodianus Gramm., Rhet., De prosodia catholica 
Part+volume 3,1, page 71, line 19

Ἔτι τὰ εἰς <αρης>, εἰ μὴ ἐπίθετα εἴη, <Κυαξάρης, Παντάρης, 
Σωχάρης, Τυνδάρης, Ἀφάρης, Ἀμφιάρης, Καβάρης> ὄνομα 
ποταμοῦ, <Γανδάρης>20 Ἰνδῶν ἔθνος. 



Aelius Herodianus et Pseudo-Herodianus Gramm., Rhet., De prosodia catholica 
Part+volume 3,1, page 72, line 17

                                                                        <Σαρ-
μάτης> ἔθνος Σκυθικόν, <Σαυρομάτης> ἔθνος Ἰνδικόν. 



Aelius Herodianus et Pseudo-Herodianus Gramm., Rhet., De prosodia catholica 
Part+volume 3,1, page 75, line 3

                                                                     <Ἀρα-
χώτης> ποταμὸς Ἰνδικός, ὃς καὶ Ἀραχωτός. 



Aelius Herodianus et Pseudo-Herodianus Gramm., Rhet., De prosodia catholica 
Part+volume 3,1, page 76, line 10

<ἑδρίτης, Ὠρίτης> ἔθνος Ἰνδῶν αὐτόνομον. 



Aelius Herodianus et Pseudo-Herodianus Gramm., Rhet., De prosodia catholica 
Part+volume 3,1, page 76, line 11

                                                         Στράβων πεντεκαιδεκάτῃ 
(p. 720) «τῷ ὁρίζοντι αὐτοὺς ἀπὸ τῶν ἑξῆς Ὠριτῶν· Ἰνδῶν δέ ἐστι καὶ 
αὕτη μερίς, ἔθνος αὐτόνομον». 



Aelius Herodianus et Pseudo-Herodianus Gramm., Rhet., De prosodia catholica 
Part+volume 3,1, page 76, line 13

                                      καὶ Ἀπολλόδωρος δευτέρῳ «ἔπειτα 
Ὠρίτας τε καὶ Γεδρωσίους, ὧν τοὺς μὲν Ἰνδοὺς ὡς ἐνοικοῦντας πέ-
τραν . 



Aelius Herodianus et Pseudo-Herodianus Gramm., Rhet., De prosodia catholica 
Part+volume 3,1, page 76, line 26

                            <Ἁρματίτης> ἐθνικὸν Ἁρμάτων πόλεως πλη-
θυντικῶς Ἰνδικῆς ὡς τοῦ ἕρμα <ἑρματίτης>· ἔστι καὶ πόλις Ἁρματίτης. 



Aelius Herodianus et Pseudo-Herodianus Gramm., Rhet., De prosodia catholica 
Part+volume 3,1, page 76, line 32

             <Ὀρβίτης> ἔθνος Ἰνδικόν, ὡς Ἀπολλόδωρος δευτέρῳ, περὶ 
Ἀλεξάνδρειαν. 



Aelius Herodianus et Pseudo-Herodianus Gramm., Rhet., De prosodia catholica 
Part+volume 3,1, page 86, line 23

* Τὰ εἰς <βις> δισύλλαβα ὀξύνεται ἢ βαρύνεται· καὶ ὀξύνεται μὲν τὰ 
παρώνυμα οἷον <λαβίς> παρὰ τὸ λαβή καὶ ἐπιθετικὰ οἷον <Λεσβίς> καὶ 
ἐκτείνοντα τὸ <ι> οἷον <βαλβίς>· τὰ δὲ μὴ οὕτως ἔχοντα βαρύνεται, <ἶβις, 
Ἄρβις> ποταμὸς Ἰνδικῆς καὶ ἔθνος· λέγεται δὲ καὶ Ἄραβις. 



Aelius Herodianus et Pseudo-Herodianus Gramm., Rhet., De prosodia catholica 
Part+volume 3,1, page 98, line 7

                                                                       τὸ δὲ 
<Σάραπις> νῆσος ἐν Ἰνδικῷ κόλπῳ βαρύνεται. 



Aelius Herodianus et Pseudo-Herodianus Gramm., Rhet., De prosodia catholica 
Part+volume 3,1, page 121, line 3

              <Ἰαλύσιος, ἐτώσιος, Ῥώσιος, Ἀραχώσιος, Περκώ-
σιος, Κριθώσιος, Γεδρώσιος> ἔθνος Ἰνδικόν. 



Aelius Herodianus et Pseudo-Herodianus Gramm., Rhet., De prosodia catholica 
Part+volume 3,1, page 128, line 5

Τὰ εἰς <ος> καθαρὸν ὑπὲρ δύο συλλαβὰς τῷ <ω> μετὰ <ι> προσγεγραμ-
μένου παραληγόμενα προπερισπᾶται, <πατρῷος, ἡρῷος> καὶ <Ἡρῷος, 
ἠῷος, Ἀχελῷος> ποταμὸς Ἀκαρνανίας ἀπὸ Ἀχελῴου ἐλθόντος ἐκ Θετ-
ταλίας μετὰ Ἀλκμαίωνος καὶ τὸ ἐθνικὸν ὁμοφώνως, <αἰδῷος, Μι-
νῷος, Ἐλβῷος, Σαρδῷος, Πυθῷος, Ληθῷος, Ἰνδῷος, Γελῷος> 
ὁ οἰκῶν Γέλαν πόλιν Σικελίας ἴσως ἀπὸ τοῦ γέλως. 



Aelius Herodianus et Pseudo-Herodianus Gramm., Rhet., De prosodia catholica 
Part+volume 3,1, page 130, line 14

                    <Δυρβαῖος> ἔθνος καθῆκον εἰς Βάκτρους καὶ τὴν Ἰνδι-
κήν. 



Aelius Herodianus et Pseudo-Herodianus Gramm., Rhet., De prosodia catholica 
Part+volume 3,1, page 130, line 16

      Κτησίας ἐν Περσικῶν ιʹ «χώρη δὲ πρὸς αὐτὸν πρόσκειται Δυρβαῖοι, 
πρὸς τὴν Βακτρίην καὶ Ἰνδικὴν κατατείνοντες. 



Aelius Herodianus et Pseudo-Herodianus Gramm., Rhet., De prosodia catholica 
Part+volume 3,1, page 141, line 15

                                                                               τὸ δὲ 
<Σύναγγος> πόλις Φοινίκης καὶ <Σάλαγγος> ἔθνος Ἰταλίας – ἔστι δὲ 
καὶ ἕτερον ἔθνος Ἰνδικόν – διπλασιαζόμενον ἔχει τὸ <γ>. 



Aelius Herodianus et Pseudo-Herodianus Gramm., Rhet., De prosodia catholica 
Part+volume 3,1, page 142, line 27

                                                                                  σεση-
μείωται τὸ <Ἰνδός> ποταμὸς καὶ ἐθνικὸν καὶ <Ὀρδός> ἔθνος Μακεδονίας. 



Aelius Herodianus et Pseudo-Herodianus Gramm., Rhet., De prosodia catholica 
Part+volume 3,1, page 143, line 26

Τὰ εἰς <ζος> πάντα βαρύνεται, <ὄζος, ῥοῖζος, Γάζος> πόλις Ἰν-
δική, <Τόπαζος> νῆσος Ἰνδικὴ καὶ λίθος ὁμώνυμος τῇ νήσῳ. 



Aelius Herodianus et Pseudo-Herodianus Gramm., Rhet., De prosodia catholica 
Part+volume 3,1, page 167, line 18

Τὰ εἰς <μος> προσηγορικὰ ἔχοντα τὴν πρὸ τέλους συλλαβὴν εἰς <λ> 
καταλήγουσαν ὀξύνεται, <ὀφθαλμός, τιλμός, ψαλμός, παλμός, 
ἰνδαλμός>. 



Aelius Herodianus et Pseudo-Herodianus Gramm., Rhet., De prosodia catholica 
Part+volume 3,1, page 181, line 8

                          ὀξύνονται δὲ ταῦτα· <Τυρσηνός, Ἀβυδηνός, 
Ἀσσακηνός> ἔθνος Ἰνδικόν. 



Aelius Herodianus et Pseudo-Herodianus Gramm., Rhet., De prosodia catholica 
Part+volume 3,1, page 192, line 5

                         ἔστι καὶ Τύρος τῆς Λακωνικῆς καὶ νῆσος πρὸς 
τῇ Ἐρυθρᾷ θαλάσσῃ, ἣν Ἀρτεμίδωρος Τύλον διὰ τοῦ <λ> καλεῖ, ἔστι καὶ 
πόλις Ἰνδίας καὶ Λυδίας καὶ Πισιδίας. 



Aelius Herodianus et Pseudo-Herodianus Gramm., Rhet., De prosodia catholica 
Part+volume 3,1, page 198, line 18

                     Κάσπειρος> πόλις Πάρθων προσεχὴς τῇ Ἰνδικῇ 
καὶ τὸ ἐθνικὸν ὁμοφώνως. 



Aelius Herodianus et Pseudo-Herodianus Gramm., Rhet., De prosodia catholica 
Part+volume 3,1, page 212, line 30

<Παροπάνισσος> πόλις καὶ ὄρος Ἰνδικῆς, ἀφ' οὗ Παροπανισσάδαι οἱ 
οἰκοῦντες. 



Aelius Herodianus et Pseudo-Herodianus Gramm., Rhet., De prosodia catholica 
Part+volume 3,1, page 221, line 29

<Ἀραχωτός> ποταμὸς Ἰνδικῆς ῥέων ἀπὸ Καυκάσου, ὡς Φαβωρῖνος καὶ 
Στράβων ἑνδεκάτῃ (p. 513). 



Aelius Herodianus et Pseudo-Herodianus Gramm., Rhet., De prosodia catholica 
Part+volume 3,1, page 221, line 30

                                    καὶ ἀπ' αὐτοῦ Ἀραχωτοί πόλις Ἰνδικῆς. 



Aelius Herodianus et Pseudo-Herodianus Gramm., Rhet., De prosodia catholica 
Part+volume 3,1, page 228, line 7

ἔστι καὶ ἄλλη τῆς Ἰνδικῆς. 



Aelius Herodianus et Pseudo-Herodianus Gramm., Rhet., De prosodia catholica 
Part+volume 3,1, page 241, line 19

                                     <Μωριεῖς> ἔθνος Ἰνδικὸν ἐν ξυλίνοις 
οἰκοῦντες οἴκοις, ὡς Εὐφορίων. 



Aelius Herodianus et Pseudo-Herodianus Gramm., Rhet., De prosodia catholica 
Part+volume 3,1, page 248, line 8

Τὰ εἰς <α> μακρὸν καὶ εἰς <η> τῶν εἰς <ους> περισπωμένων περισπᾶται 
ἀργυροῦς <ἀργυρᾶ>, ἔστι δὲ καὶ μητρόπολις τῆς ἐν Ἰνδικῇ Ταπροβάνης 
νήσου. 



Aelius Herodianus et Pseudo-Herodianus Gramm., Rhet., De prosodia catholica 
Part+volume 3,1, page 252, line 28

* Τὰ εἰς <δα> βαρύνεται, <Σίνδα> πόλις πρὸς τῷ μεγάλῳ κόλπῳ τῆς 
Ἰνδικῆς, <Πέδα> πόλις Αὐσονική, <Σίβδα> πόλις Καρίας, <Γάδδα> χωρίον 
Ἀραβίας. 



Aelius Herodianus et Pseudo-Herodianus Gramm., Rhet., De prosodia catholica 
Part+volume 3,1, page 255, line 9

                                                                                    καὶ 
<Βουκεφάλα> πόλις Ἰνδικῆς, ἣν ἔκτισεν Ἀλέξανδρος «ἐπ' ἀμφοτέραις 
ταῖς ὄχθαις τοῦ Ὑδάσπου ποταμοῦ πόλεις ᾤκισε, Νίκαιαν – Βουκεφά-
λαν δὲ ἔνθα διαβάντος καὶ μαχομένου ἀπέθανεν αὐτοῦ ὁ ἵππος Βουκε-
φάλας λεγόμενος». 



Aelius Herodianus et Pseudo-Herodianus Gramm., Rhet., De prosodia catholica 
Part+volume 3,1, page 256, line 30

                 <Σώλιμνα> πόλις Ἰνδίας. 



Aelius Herodianus et Pseudo-Herodianus Gramm., Rhet., De prosodia catholica 
Part+volume 3,1, page 257, line 27

                                                   <Κάρμανα> νῆσος τῆς 
Ἰνδικῆς. 



Aelius Herodianus et Pseudo-Herodianus Gramm., Rhet., De prosodia catholica 
Part+volume 3,1, page 257, line 29

                                                         <Μάργανα> πόλις τῆς Ἰν-
δικῆς. 



Aelius Herodianus et Pseudo-Herodianus Gramm., Rhet., De prosodia catholica 
Part+volume 3,1, page 258, line 8

                                 <Κάρμινα> νῆσος Ἰνδική. 



Aelius Herodianus et Pseudo-Herodianus Gramm., Rhet., De prosodia catholica 
Part+volume 3,1, page 259, line 20

* Τὰ εἰς <αρα> ὑπὲρ δύο συλλαβὰς παροξύνεται, <Κυπάρα> κρήνη 
Σικελίας, ἣ καὶ Ἀρέθουσα ἐλέγετο, <Βατεράρα> πόλις Λιγύων. <Ἰνδάρα> 
Σικανῶν πόλις. 



Aelius Herodianus et Pseudo-Herodianus Gramm., Rhet., De prosodia catholica 
Part+volume 3,1, page 264, line 2

                                              <Παναίουρα> πόλις Ἰνδικὴ περὶ 
τὸν Ἰνδὸν ποταμόν. 



Aelius Herodianus et Pseudo-Herodianus Gramm., Rhet., De prosodia catholica 
Part+volume 3,1, page 264, line 9

                                                    <Βαλβέρουρα>· οὕτως 
τινὲς Ἰνδικὴν πόλιν Ἰβηρίας φασίν, μεθ' ὧν καὶ <δίφουρα> ἡ γέφυρα. 



Aelius Herodianus et Pseudo-Herodianus Gramm., Rhet., De prosodia catholica 
Part+volume 3,1, page 266, line 25

                                <Νῦσα> πόλεις πολλαί, ἐν Ἑλικῶνι, ἐν Θρᾴκῃ, ἐν 
Καρίᾳ, ἐν Ἀραβίᾳ, ἐν Αἰγύπτῳ, ἐν Νάξῳ, ἐν Ἰνδοῖς, ἐπὶ τοῦ Καυκά-
σου ὄρους, ἐν Εὐβοίᾳ. 



Aelius Herodianus et Pseudo-Herodianus Gramm., Rhet., De prosodia catholica 
Part+volume 3,1, page 268, line 19

                     ἔστι καὶ νῆσος μία τῶν Κυκλάδων καὶ τρίτη Ἰνδι-
κῆς, ἣν ἀναγράφει Φίλων καὶ Δημοδάμας ὁ Μιλήσιος – καὶ ἐκ τῆς 
ἐπί <ἔπισσα> παρ' Ἑκαταίῳ. 



Aelius Herodianus et Pseudo-Herodianus Gramm., Rhet., De prosodia catholica 
Part+volume 3,1, page 271, line 1

* Τὰ εἰς <τα> δισύλλαβα σπάνια ὄντα βαρύνεται, <Γέντα> πόλις Ἰν-  
δικὴ τῆς ἐκτὸς Γάγγου. 



Aelius Herodianus et Pseudo-Herodianus Gramm., Rhet., De prosodia catholica 
Part+volume 3,1, page 272, line 5

                                  τετάρτη ἐν Ἰνδοῖς. 



Aelius Herodianus et Pseudo-Herodianus Gramm., Rhet., De prosodia catholica 
Part+volume 3,1, page 272, line 32

           <Κάθαια> πόλις Ἰνδική. 



Aelius Herodianus et Pseudo-Herodianus Gramm., Rhet., De prosodia catholica 
Part+volume 3,1, page 274, line 14

                                                         τετάρτη πόλις Ὠριτῶν, 
ἔθνους Ἰχθυοφάγων, κατὰ τὸν περίπλουν τῆς Ἰνδικῆς. 



Aelius Herodianus et Pseudo-Herodianus Gramm., Rhet., De prosodia catholica 
Part+volume 3,1, page 274, line 15

                                                               πέμπτη ἐν τῇ 
Ὠπιανῇ, κατὰ τὴν Ἰνδικήν. 



Aelius Herodianus et Pseudo-Herodianus Gramm., Rhet., De prosodia catholica 
Part+volume 3,1, page 274, line 15

                                   ἕκτη πάλιν Ἰνδικῆς. 



Aelius Herodianus et Pseudo-Herodianus Gramm., Rhet., De prosodia catholica 
Part+volume 3,1, page 274, line 16

                                                             ἑβδόμη ἐν Ἀρίοις, ἔθνει 
Παρθυαίων κατὰ τὴν Ἰνδικήν. 



Aelius Herodianus et Pseudo-Herodianus Gramm., Rhet., De prosodia catholica 
Part+volume 3,1, page 274, line 19

                                                                            τες-
σαρεσκαιδεκάτη παρὰ Σωριανοῖς, Ἰνδικῷ ἔθνει. 



Aelius Herodianus et Pseudo-Herodianus Gramm., Rhet., De prosodia catholica 
Part+volume 3,1, page 274, line 20

                                                       πεντεκαιδεκάτη παρὰ τοῖς 
Ἀραχώτοις, ὁμοροῦσα τῇ Ἰνδικῇ. 



Aelius Herodianus et Pseudo-Herodianus Gramm., Rhet., De prosodia catholica 
Part+volume 3,1, page 277, line 33

                                        <Βουκεφάλεια> πόλις ἐπὶ τῷ Βουκε-
φάλῳ ἵππῳ, ἣν ἔκτισεν Ἀλέξανδρος ἐν Ἰνδίᾳ παρὰ τὸν Ὑδάσπην 
ποταμόν. 



Aelius Herodianus et Pseudo-Herodianus Gramm., Rhet., De prosodia catholica 
Part+volume 3,1, page 277, line 34

                                                     <Γήρεια> πόλις Ἰνδική. 



Aelius Herodianus et Pseudo-Herodianus Gramm., Rhet., De prosodia catholica 
Part+volume 3,1, page 278, line 20

                                            ιαʹ μεταξὺ Σκυθίας καὶ Ἰνδικῆς. 



Aelius Herodianus et Pseudo-Herodianus Gramm., Rhet., De prosodia catholica 
Part+volume 3,1, page 279, line 35

                                                                              <Σάνεια> πόλις 
Ἰνδική. 



Aelius Herodianus et Pseudo-Herodianus Gramm., Rhet., De prosodia catholica 
Part+volume 3,1, page 287, line 5

                                                                         <Ἀετία>· 
οὕτως ἐκλήθη ἡ Αἴγυπτος ἀπό τινος Ἰνδοῦ Ἀετοῦ. 



Aelius Herodianus et Pseudo-Herodianus Gramm., Rhet., De prosodia catholica 
Part+volume 3,1, page 288, line 6

<Γεδρωσία> χώρα Ἰνδική. 



Aelius Herodianus et Pseudo-Herodianus Gramm., Rhet., De prosodia catholica 
Part+volume 3,1, page 297, line 10

                                                <Δαρσανία> πόλις Ἰνδική. 



Aelius Herodianus et Pseudo-Herodianus Gramm., Rhet., De prosodia catholica 
Part+volume 3,1, page 297, line 14

                      <Καρμανία> χώρα τῆς Ἰνδικῆς. 



Aelius Herodianus et Pseudo-Herodianus Gramm., Rhet., De prosodia catholica 
Part+volume 3,1, page 306, line 20

                                 <Ῥοδόη> πόλις Ἰνδική. 



Aelius Herodianus et Pseudo-Herodianus Gramm., Rhet., De prosodia catholica 
Part+volume 3,1, page 316, line 14

                                                   Πολύβιος γʹ. <Ἰνδική> πόλις 
Ἰβηρίας πλησίον Πυρήνης. 



Aelius Herodianus et Pseudo-Herodianus Gramm., Rhet., De prosodia catholica 
Part+volume 3,1, page 321, line 9

<Πατάλη> πόλις Ἰνδική, ἣ καὶ Πάταλα λέγεται, <Πετάλη, Μυρτάλη. 



Aelius Herodianus et Pseudo-Herodianus Gramm., Rhet., De prosodia catholica 
Part+volume 3,1, page 328, line 24

        <Ῥωγάνη> πόλις ἐν τῇ Ἰνδικῇ. 



Aelius Herodianus et Pseudo-Herodianus Gramm., Rhet., De prosodia catholica 
Part+volume 3,1, page 328, line 25


<Ταπροβάνη> νῆσος μεγίστη ἐν τῇ Ἰνδικῇ θαλάσσῃ. 



Aelius Herodianus et Pseudo-Herodianus Gramm., Rhet., De prosodia catholica 
Part+volume 3,1, page 333, line 5

                      τὸ δὲ <Παταληνή> νῆσος Ἰνδική ὀξύνεται, τινὲς δὲ 
καὶ βαρυτόνως Παταλήνην ὡς Πριήνην ἀναγινώσκουσιν. 



Aelius Herodianus et Pseudo-Herodianus Gramm., Rhet., De prosodia catholica 
Part+volume 3,1, page 341, line 28

                     ἔστι καὶ ἄλλη χερρόνησος τῆς Ἰνδικῆς <Χρυσῆ> καλου-
μένη. 



Aelius Herodianus et Pseudo-Herodianus Gramm., Rhet., De prosodia catholica 
Part+volume 3,1, page 345, line 9

                                                  <Ἀργάντη> πόλις Ἰνδίας, 
ὡς Ἑκαταῖος. 



Aelius Herodianus et Pseudo-Herodianus Gramm., Rhet., De prosodia catholica 
Part+volume 3,1, page 346, line 27

                          τὸ δὲ <Μαράχη> πόλις Ἰνδικὴ κύριόν ἐστιν. 



Aelius Herodianus et Pseudo-Herodianus Gramm., Rhet., De prosodia catholica 
Part+volume 3,1, page 352, line 29

                  ἔστι καὶ Ἅρματα πόλις πληθυντικῶς Ἰνδικῆς – <αἷμα, 
βῆμα, ῥῆμα, χρῆμα, σῆμα>, ὅθεν <Κυνόσσημα> τόπος Λιβύης. 



Aelius Herodianus et Pseudo-Herodianus Gramm., Rhet., De prosodia catholica 
Part+volume 3,1, page 361, line 30

                                                          <Σεσίνδιον> πόλις 
Ἰνδική. 



Aelius Herodianus et Pseudo-Herodianus Gramm., Rhet., De prosodia catholica 
Part+volume 3,1, page 368, line 4

                                                                       ἔστι καὶ 
Βυζάντιον ἕτερον ἐν τῇ Ἰνδικῇ. 



Aelius Herodianus et Pseudo-Herodianus Gramm., Rhet., De prosodia catholica 
Part+volume 3,1, page 379, line 14

* Τὰ εἰς <γον> καὶ <δον> οὐδέτερα κύρια ὑπὲρ δύο συλλαβὰς προπαρ-
οξύνεται, <Βήσσυγα> ἐμπόριον τῆς Ἰνδικῆς. 



Aelius Herodianus et Pseudo-Herodianus Gramm., Rhet., De prosodia catholica 
Part+volume 3,1, page 380, line 11

                                                τὸ δὲ <Ἠμωδόν> ὄρος Ἰνδι-
κὸν ὀξύνεται, τινὲς δὲ προπαροξυτόνως Ἤμωδον λέγουσιν. 



Aelius Herodianus et Pseudo-Herodianus Gramm., Rhet., De prosodia catholica 
Part+volume 3,1, page 380, line 22

* Τὰ εἰς <κον> κύρια βαρύνεται, <Μάζακα> πόλις Καππαδοκίας ἡ νῦν 
Καισάρεια, <Μάσσακα> πόλις Ἰνδῶν, Ἀρριανὸς ἐν Ἰνδικοῖς. 



Aelius Herodianus et Pseudo-Herodianus Gramm., Rhet., De prosodia catholica 
Part+volume 3,1, page 381, line 13

                                                                                  ἔστι 
καὶ Ἰνδικῆς. 



Aelius Herodianus et Pseudo-Herodianus Gramm., Rhet., De prosodia catholica 
Part+volume 3,1, page 381, line 14

                                                                              <Πάταλα> 
πόλις Ἰνδική. 



Aelius Herodianus et Pseudo-Herodianus Gramm., Rhet., De prosodia catholica 
Part+volume 3,1, page 381, line 33

                                        <Τάξιλα> πόλις Ἰνδική. 



Aelius Herodianus et Pseudo-Herodianus Gramm., Rhet., De prosodia catholica 
Part+volume 3,1, page 388, line 6

                         <Παλίμβοθρα> πόλις Ἰνδική. 



Aelius Herodianus et Pseudo-Herodianus Gramm., Rhet., De prosodia catholica 
Part+volume 3,1, page 398, line 1

<Σήρ> ἔθνος Ἰνδικόν, ὅθεν σηρικὰ τὰ πολυτελῆ ἱμάτια. 



Aelius Herodianus et Pseudo-Herodianus Gramm., Rhet., Περὶ παθῶν (0087: 009)
“Grammatici Graeci, vol. 3.2”, Ed. Lentz, A.
Leipzig: Teubner, 1870, Repr. 1965.
Part+volume 3,2, page 331, line 1

                     τοῦ <ι> τραπέντος εἰς <ε> γίνεται Δεόνυσος (οὕτω 
γὰρ Σάμιοι προφέρουσι) καὶ συναιρέσει Δεύνυσος ὡς Θεόδοτος Θεύδο-  
τος· ἔνιοι δέ φασιν, ὅτι ἐπειδὴ ἐβασίλευσε Νύσης· κατὰ δὲ τὴν Ἰνδῶν 
φωνὴν δεῦνος ὁ βασιλεύς. 



Aelius Herodianus et Pseudo-Herodianus Gramm., Rhet., Περὶ παθῶν 
Part+volume 3,2, page 354, line 4

                                                        σημαίνει δὲ τὸ 
ἄγαλμα ἢ ὁμοίωμα οἷον «δείκηλα προΐαλλεν» (Apoll. Rhod. IV 1672)   
καὶ «δείκελον Ἰφιγόνης» παρὰ Παρθενίῳ· ὥσπερ παρὰ τὸ πέμπω πέμ-
πελος (σημαίνει δὲ τὸν πολλῶν ἐνιαυτῶν ὄντα), οὕτω καὶ ἀπὸ τοῦ 
δείκω τὸ δεικνύω γίνεται δείκελος καὶ ἐκτάσει τοῦ <ε> εἰς <η> δείκηλον τὸ 
ἴνδαλμα. 



Aelius Herodianus et Pseudo-Herodianus Gramm., Rhet., Περὶ ὀρθογραφίας (0087: 011)
“Grammatici Graeci, vol. 3.2”, Ed. Lentz, A.
Leipzig: Teubner, 1870, Repr. 1965.
Part+volume 3,2, page 444, line 18

Τὰ διὰ τοῦ <ινδος> εἴτε δισύλλαβα εἴτε ὑπὲρ δύο συλλαβὰς διὰ τοῦ 
<ι> γράφεται οἷον Ἰνδός, Ἄλινδος, Ἴσινδος πόλις Μακεδονίας, Ἄριν-
δος ὄνομα ποταμοῦ. 



Aelius Herodianus et Pseudo-Herodianus Gramm., Rhet., Περὶ ὀρθογραφίας 
Part+volume 3,2, page 492, line 28

                                      ἔνιοι φασίν, ὅτι, ἐπειδὴ ἐβασίλευσε 
Νύσης, δεῦνον δὲ τὸν βασιλέα λέγουσιν οἱ Ἰνδοί, ὡς Ἰόβας. 


Aelius Herodianus et Pseudo-Herodianus Gramm., Rhet., Περὶ ὀρθογραφίας 
Part+volume 3,2, page 592, line 20

<Τόπαζος> νῆσος Ἰνδική. 



Aelius Herodianus et Pseudo-Herodianus Gramm., Rhet., Περὶ κλίσεως ὀνομάτων (0087: 013)
“Grammatici Graeci, vol. 3.2”, Ed. Lentz, A.
Leipzig: Teubner, 1870, Repr. 1965.
Part+volume 3,2, page 652, line 13

Τὰ εἰς <κας> λήγοντα ἅπαντα ἰσοσυλλάβως κλίνεται, οἷον Πελέκας 
Πελέκα, Κώκας Κώκα, Κρίκας Κρίκα (ἔστι δὲ ὄνομα ποταμοῦ), Σάκας 
Σάκα, Πολύκκας Πολύκκα (ἔστι δὲ ποταμὸς Μακεδονίας), Ματάκας 
Ματάκα (ἔστι δὲ ὄνομα εὐνούχου]1, Ἰνδύκας Ἰνδύκα, Κοτύκας Κοτύκα 
(ἔστι δὲ ὄνομα βασιλέως Παφλαγονίας). 



Aelius Herodianus et Pseudo-Herodianus Gramm., Rhet., Περὶ κλίσεως ὀνομάτων 
Part+volume 3,2, page 653, line 15

                               σεσημείωται τὸ Βάμβλας Βάμβλα ἰσοσυλ-
λάβως κλινόμενον, ἔστι δὲ ὄνομα βασιλέως Ἰνδῶν. 



Aelius Herodianus et Pseudo-Herodianus Gramm., Rhet., Περὶ παρωνύμων (0087: 026)
“Grammatici Graeci, vol. 3.2”, Ed. Lentz, A.
Leipzig: Teubner, 1870, Repr. 1965.
Part+volume 3,2, page 872, line 23

Ἀργάντη πόλις Ἰνδίας. 



Aelius Herodianus et Pseudo-Herodianus Gramm., Rhet., Περὶ παρωνύμων 
Part+volume 3,2, page 872, line 24

                              τὸ ἐθνικὸν ἔδει <Ἀργανταῖος>, ἀλλὰ ὁ 
τύπος τῶν Ἰνδῶν ἢ <Ἀργαντηνός ἢ Ἀργαντίτης>. 



Aelius Herodianus et Pseudo-Herodianus Gramm., Rhet., Περὶ παρωνύμων 
Part+volume 3,2, page 881, line 26

Τόπαζος νῆσος Ἰνδική. 



Aelius Herodianus et Pseudo-Herodianus Gramm., Rhet., Περὶ παρωνύμων 
Part+volume 3,2, page 892, line 2

                                                             ἔστι καὶ Ἅρματα 
πόλις πληθυντικῶς Ἰνδικῆς. 




Aelius Herodianus et Pseudo-Herodianus Gramm., Rhet., Partitiones (= Ἐπιμερισμοί) [Sp.?] (e codd. Paris. 2543 + 2570) 
Page 170, line 16

Τὰ εἰς ιξ βαρύτονα, μὴ Δωρικῶς τρεπόμενα εἰς α, διὰ 
τοῦ ἰῶτα γράφονται· οἷον· πέρδιξ· φοῖνιξ· κώδιξ· ἴνδιξ· 
καὶ τὰ ὅμοια. 



Aelius Herodianus et Pseudo-Herodianus Gramm., Rhet., Partitiones (= Ἐπιμερισμοί) [Sp.?] (e codd. Paris. 2543 + 2570) 
Page 229, line 4

Τὰ εἰς ων λήγοντα ὀξύτονα θηλυκὰ, διὰ τοῦ ο μικροῦ 
κλίνονται· οἷον· τρυγὼν, τρυγόνος· ἀηδόνος· Γοργόνος· 
σταγόνος· λαγόνος· σιαγόνος· χιόνος· θηλαμόνος· ἀμαζόνος· 
Χαλκηδόνος· Χαρκηδόνος· καὶ Ὀλοσσόνος· ἰνδικτιῶνος δὲ 
μέγα. 

\end{greek}

\section{Archigenes}

\blockquote[From Wikipedia\footnote{\url{http://en.wikipedia.org/wiki/Archigenes}}]{Archigenes ('Αρχιγένης), an eminent ancient Greek physician, who lived in the 1st and 2nd centuries.

He was the most celebrated of the sect of the Eclectici, and was a native of Apamea in Syria; he practised at Rome in the time of Trajan, 98-117, where he enjoyed a very high reputation for his professional skill. He is, however, reprobated as having been fond of introducing new and obscure terms into the science, and having attempted to give to medical writings a dialectic form, which produced rather the appearance than the reality of accuracy. Archigenes published a treatise on the pulse, on which Galen wrote a Commentary; it appears to have contained a number of minute and subtle distinctions, many of which have no real existence, and were for the most part the result rather of a preconceived hypothesis than of actual observation; and the same remark may be applied to an arrangement which he proposed of fevers.

He, however, not only enjoyed a considerable degree of the public confidence during his lifetime, but left behind him a number of disciples, who for many years maintained a respectable rank in their profession. The name of the father of Archigenes was Philippus; he was a pupil of Agathinus, whose life he once saved; and he died at the age either of sixty-three or eighty-three.[1]

The titles of several of his works are preserved, of which, however, nothing but a few fragments remain; some of these have been preserved by other ancient authors, and some are still in manuscript in the King's Library at Paris.[2] By some writers he is considered to have belonged to the sect of the Pneumatici.[3]

He is mentioned several times by Juvenal, in his Satires.[4]}

\begin{greek}

Archigenes Med., Fragmenta (0661: 001)
“Frammenti medicinali di Archigene”, Ed. Brescia, C.
Naples: Libreria Scientifica Editrice, 1955.
Page 17, line 18

                                                            ἔστι δὲ ἡ σύνθεσις 
τοῦ κύφεως αὕτη· λίτου γαγάτου 𐆄 <γ> ἑλενίου 𐆄 <ς> ἀσπαλάθου φλοιοῦ 𐆄 <ζ> 
ῥίζης ἀσφοδέλου 𐆄 <δ> βράθυος 𐆄 <ς> ἀρκευθίδος φλέρια <ρ> βδέλλης πετραικῆς 
𐆄 <ζ> ἰσχάδων λιπαρῶν ἀτέγκτων 𐆄 <β> ἀμμωνιακοῦ 𐆄 <γ> ὀνύχου Ἰνδικοῦ 
σπέρματος πηγάνου ἀγρίου κόστου ἀνὰ 𐆄 <ς> καρύων κο. 



Archigenes Med., Fragmenta 
Page 19, line 28

                       βαλαυστίων ταρʹ <α> μάκερος ταρʹ <α>𐅵 ὀμφακίου ταρʹ <α> 
ὑοσκυάμου σπέρματος ταρʹ <α> κέρατος ἐλαφείου ταρʹ 𐅵 ὀπίου ταρʹ <β> βάτου 
ἀώρου καρποῦ ταρʹ <α> μύρτου μέλανος ταρʹ <α> σμύρνης τρωγλ<οδ>ύτιδος 
ταρʹ 𐅵 λυκίου Ἰνδικοῦ ταρʹ 𐅵 κοραλίου ταρʹ <α> σιδίων ταρʹ 𐅵 κρόκου ταρʹ <α> σχοι-
νάνθης ταρʹ 𐅵 ῥόδων ξηρῶν ταρʹ 𐅵 Λημνίας σφραγίδος ταρʹ <α>𐅵 ἀρνογλώσσου 
χυλοῦ ταρʹ <α> ἀκακίας ταρʹ <α> λαδάνου ταρʹ <α> μαράθρου σπόρου ἀνίσου ἀνὰ 
ταρʹ 𐅵 στυπτηρίας λιβάνου ἀνὰ ταρʹ <α> ῥοῦ μαγειρικοῦ ταρʹ 𐅵 γῆς Σαμίας 
ταρʹ 𐅵 κοιλίας περιστερᾶς ταρʹ 𐅵 κοιλίας ἀλεκτρυόνος ἀσταφίδων ἄνευ τῶν 
γιγάρτων ῥόδων ἀνὰ ταρʹ τ𐅵 κόμμεος ταρʹ <α> χαλκίτεως κισσήρεως ὀπτῆς 
ἀνὰ ταρʹ 𐅵 χυλοῦ ὑποκιστίδος ταρʹ <α> λαγωοῦ πιτύας ταρʹ . 



Archigenes Med., Fragmenta inedita (0661: 002)
“”Frammenti inediti di Archigene””, Ed. Calabrò, G.L., 1961; Bollettino del comitato per la preparazione della edizione nazionale dei classici greci e latini 9.
Page 70, line 13

                   ἡ κυφοειδὴς καὶ πρὸς τὰ ἐν θώρακι πάντα· σταφίδων σαρκὸς 𐅻 <κε>, 
κρόκου 𐅻 <α>, καλάμου Ἰνδικοῦ 𐅻 <β>, βδελλίου 𐅻 <β>, κινναμώμου 𐅻 <α>, κασίας 𐅻 <γ>, 
σχοίνου ἄνθους 𐅻 <β>, σμύρνης 𐅻 <δ>, τερεβινθίνης 𐅻 <δ>, ἀσπαλάθου ῥινήματος 𐅻 <β>, 
ναρδοστάχυος 𐅻 <γ>, μέλιτος 𐅻 <ις>, οἴνου τὸ ἀρκοῦν, γλυκέος τὸ αὔταρκες. 



Archigenes Med., Fragmenta inedita 
Page 70, line 25

Ἄλλη κυφοειδὴς Ἀνδρομάχου ποιεῖ καὶ πρὸς βῆχας καὶ ἀναγωγὰς ὑγρῶν 
κρόκου, κινναμώμου, σμύρνης ἀνὰ 𐅻 <α>, βδελλίου 𐅻 <δ>, ἀσπαλάθου τετρώβολον, 
σχοίνου ἄνθους 𐅻 <γ>, καλάμου Ἰνδικοῦ 𐅻 <β>, κασίας 𐅻 <α>, ναρδοστάχυος 𐅻 <α>, τερε-
βινθίνης 𐅻 <ις>, μέλιτος κατὰ κοτύλης C, σταφίδων λιπαρῶν 𐅻 <ρξ> τὰς σάρκας κεκα-
θαρμένας, οἴνου παλαιοῦ στύφοντος τὸ ἀρκοῦν εἰς τὸ βρέξαι τὴν σταφίδα καὶ τὸ 
βδέλλιον καὶ τὴν σμύρνην. 

\end{greek}

\section{Herodian}

\blockquote[From Wikipedia\footnote{\url{http://en.wikipedia.org/wiki/Herodian}.}]{Herodian or Herodianus of Syria (ca. 170–240) was a minor Roman civil servant who wrote a colourful history in Greek titled History of the Empire from the Death of Marcus in eight books covering the years 180 to 238. His work is not entirely reliable although his relatively unbiased account of Elagabalus is more useful than that of Cassius Dio. He was a Greek (perhaps from Antioch) who appears to have lived for a considerable period of time in Rome, but possibly without holding any public office. From his extant work, we gather that he was still living at an advanced age during the reign of Gordianus III, who ascended the throne in 238. Beyond this, nothing is known of his life.}

Herodianus Hist., Ab excessu divi Marci (0015: 001)
“Herodiani ab excessu divi Marci libri octo”, Ed. Stavenhagen, K.
Leipzig: Teubner, 1922, Repr. 1967.
Book 1, chapter 15, section 5, line 1

                               τότε γοῦν εἴδομεν ὅσα ἐν γρα-
φαῖς ἐθαυμάζομεν· ἀπό τε γὰρ Ἰνδῶν καὶ Αἰθιόπων, εἴ 
τι πρότερον ἄγνωστον ἦν, μεσημβρίας τε καὶ τῆς ἀρκ-
τῴας γῆς ζῷα πάντα φονεύων Ῥωμαίοις ἔδειξε. 

On marksmanship and dissolution of an emperor:``From India and Ethiopia, from lands to the north and to the south, any animals hitherto unknown he displayed to the Romans and then dispatched them. On one occasion he shot arrows with crescent-shaped heads at Moroccan ostriches, birds that move with great speed, both because of their swiftness afoot and the sail-like nature of their wings.''\footnote{\url{http://www.tertullian.org/fathers/herodian_01_book1.htm}.}

\section{Arrian of Nicomedia}

\subsection{About Arrian}

\blockquote[From Wikipedia]{Arrian of Nicomedia ( /ˈæriən/; Latin: Lucius Flavius Arrianus Xenophon; Greek: Ἀρριανός c. AD 86 – 160) was a Roman (ethnic Greek)[3] historian, public servant, military commander and philosopher of the 2nd-century Roman period. As with other authors of the Second Sophistic, Arrian wrote primarily in Attic (Indica is in Herodotus' Ionic dialect, his philosophical works in Koine Greek).}

On \emph{Anabasis Alexandri}:
\blockquote[From Wikipedia]{Anabasis Alexandri (Greek: Ἀλεξάνδρου ἀνάβασις Alexándrou anábasis), the Campaigns of Alexander by Arrian, is the most important source on Alexander the Great.
The Greek term anabasis referred to an expedition from a coastline into the interior of a country. The term katabasis referred to a trip from the interior to the coast. So a more literal translation would be The Expedition of Alexander.
This work on Alexander is one of the few surviving complete accounts of the Macedonian conqueror's expedition. Arrian was able to use sources which are now lost, such as the contemporary works by Callisthenes (the nephew of Alexander's tutor Aristotle), Onesicritus, Nearchus, and Aristobulus, and the slightly later work of Cleitarchus. Most important of all, Arrian had the biography of Alexander by Ptolemy, one of Alexander's leading generals and possibly his half-brother.
It is primarily a military history; it has little to say about Alexander's personal life, his role in Greek politics or the reasons why the campaign against Persia was launched in the first place.}

\subsection{\emph{Historia Indica}}
Text: Flavius Arrianus Hist., Phil., Historia Indica (0074: 002)
“Flavii Arriani quae exstant omnia, vol. 2”, Ed. Roos, A.G., Wirth, G.
Leipzig: Teubner, 1968 (1st edn. corr.).

\begin{greek}

ΙΝΔΙΚΗ

1.1.1
 Τὰ ἔξω Ἰνδοῦ ποταμοῦ τὰ πρὸς ἑσπέρην ἔστε ἐπὶ πο-
ταμὸν Κωφῆνα Ἀστακηνοὶ καὶ Ἀσσακηνοί, ἔθνεα Ἰνδικά,
1.2.1
ἐποικέουσιν, ἀλλ' οὔτε μεγάλοι τὰ σώματα, καθάπερ οἱ
ἐντὸς τοῦ Ἰνδοῦ ᾠκισμένοι, οὔτε ἀγαθοὶ ὡσαύτως τὸν
θυμὸν οὐδὲ μέλανες ὡσαύτως τοῖς πολλοῖς Ἰνδοῖσιν.
1.3.1
οὗτοι πάλαι μὲν Ἀσσυρίοις ὑπήκοοι ἦσαν, ἔπει<τα Μή-
δοισιν, ἐπὶ> δὲ Μήδοισι Περσέων ἤκουον, καὶ φόρους
ἀπέφερον Κύρῳ τῷ Καμβύσου ἐκ τῆς γῆς σφῶν, οὓς
1.4.1
ἔταξε Κῦρος. Νυσαῖοι δὲ οὐκ Ἰνδικὸν γένος ἐστίν, ἀλλὰ
τῶν ἅμα Διονύσῳ ἐλθόντων ἐς τὴν γῆν τὴν Ἰνδῶν,
τυχὸν μὲν [καὶ] Ἑλλήνων, ὅσοι ἀπόμαχοι αὐτῶν ἐγέ-
νοντο ἐν τοῖς πολέμοις οὕστινας πρὸς Ἰνδοὺς Διόνυσος
1.5.1
ἐπολέμησε, τυχὸν δὲ καὶ τῶν ἐπιχωρίων τοὺς ἐθέλοντας  
τοῖς Ἕλλησι συνῴκισε, τήν τε χώρην Νυσαίην ὠνόμασεν
ἀπὸ τῆς τροφοῦ τῆς Νύσης Διόνυσος καὶ τὴν πόλιν
1.6.1
αὐτὴν Νῦσαν. καὶ τὸ ὄρος τὸ πρὸς τῇ πόλει, ὅτου ἐν
τῇσιν ὑπωρείῃσιν ᾤκισται ἡ Νῦσα, Μηρὸς κληίζεται ἐπὶ
1.7.1
τῇ συμφορῇ ᾗτινι ἐχρήσατο εὐθὺς γενόμενος. ταῦτα μὲν
οἱ ποιηταὶ ἐπὶ Διονύσῳ ἐποίησαν, καὶ ἐξηγείσθων αὐτὰ
1.8.1
ὅσοι λόγιοι Ἑλλήνων ἢ βαρβάρων· ἐν Ἀσσακηνοῖσι δὲ
Μάσσακα, πόλις μεγάλη, ἵναπερ καὶ τὸ κράτος τῆς γῆς
ἐστι τῆς Ἀσσακίης· καὶ ἄλλη πόλις Πευκελαῗτις, μεγάλη
καὶ αὐτή, οὐ μακρὰν τοῦ Ἰνδοῦ. ταῦτα μὲν ἔξω τοῦ
1.8.5
Ἰνδοῦ ποταμοῦ ᾤκισται πρὸς ἑσπέρην ἔστε ἐπὶ τὸν
2.1.1
Κωφῆνα· τὰ δὲ ἀπὸ τοῦ Ἰνδοῦ πρὸς ἕω, τοῦτό μοι ἔστω
ἡ Ἰνδῶν γῆ καὶ Ἰνδοὶ οὗτοι ἔστωσαν.
 ὅροι δὲ τῆς Ἰνδῶν γῆς πρὸς μὲν βορέου ἀνέμου ὁ
2.2.1
Ταῦρος τὸ ὄρος. καλέεται δὲ οὐ Ταῦρος ἔτι ἐν τῇ γῆ
ταύτῃ, ἀλλὰ ἄρχεται μὲν ἀπὸ θαλάσσης ὁ Ταῦρος τῆς
κατὰ Παμφύλους τε καὶ Λυκίην καὶ Κίλικας παρατείνει  
τε ἔστε τὴν πρὸς ἕω θάλασσαν, τέμνων τὴν Ἀσίην πᾶσαν,
2.3.1
ἄλλο δὲ ἄλλῃ καλέεται τὸ ὄρος, τῇ μὲν Παραπάμισος,
τῇ δὲ Ἠμωδός, ἄλλῃ δὲ Ἴμαον κληίζεται, καὶ τυχὸν ἄλλα
2.4.1
καὶ ἄλλα ἔχει οὐνόματα. Μακεδόνες δὲ οἱ ξὺν Ἀλεξάνδρῳ
στρατεύσαντες Καύκασον αὐτὸ ἐκάλεον, ἄλλον τοῦτον
Καύκασον, οὐ τὸν Σκυθικόν, ὡς καὶ [τὸν] ἐπέκεινα τοῦ
2.5.1
Καυκάσου λόγον κατέχειν ὅτι ἦλθεν Ἀλέξανδρος. τὰ
πρὸς ἑσπέρην δὲ τῆς Ἰνδῶν γῆς ὁ ποταμὸς ὁ Ἰνδὸς
ἀπείργει ἔστε ἐπὶ τὴν μεγάλην θάλασσαν, ἵναπερ αὐτὸς
κατὰ δύο στόματα ἐκδιδοῖ, οὐ συνεχέα ἀλλήλοισι τὰ
2.5.5
στόματα, κατάπερ τὰ πέντε τοῦ Ἴστρου ἐστὶ συνεχέα,
2.6.1
ἀλλ' ὡς τὰ τοῦ Νείλου, ὑπ' ὅτων τὸ Δέλτα ποιέεται τὸ
Αἰγύπτιον, ὧδέ τι καὶ τῆς Ἰνδῶν γῆς Δέλτα ποιέει ὁ
Ἰνδὸς ποταμός, οὐ μεῖον τοῦ Αἰγυπτίου, καὶ τοῦτο
2.7.1
Πάταλα τῇ Ἰνδῶν γλώσσῃ καλέεται. τὸ δὲ πρὸς νότου
τε ἀνέμου καὶ μεσαμβρίης αὐτὴ ἡ μεγάλη θάλασσα
ἀπείργει τὴν Ἰνδῶν γῆν, καὶ τὰ πρὸς ἕω ἡ αὐτὴ θά-
2.8.1
λασσα ἀπείργει. τὰ μὲν πρὸς μεσημβρίης κατὰ Πάταλά
τε καὶ τοῦ Ἰνδοῦ τὰς ἐκβολὰς ὤφθη πρός τε Ἀλεξάνδρου
καὶ Μακεδόνων καὶ πολλῶν Ἑλλήνων· τὰ δὲ πρὸς ἕω  
Ἀλέξανδρος μὲν οὐκ ἐπῆλθε τὰ [δὲ] πρόσω ποταμοῦ
2.9.1
Ὑφάσιος, ὀλίγοι δὲ ἀνέγραψαν τὰ μέχρι ποταμοῦ Γάγ-
γεω καὶ ἵνα τοῦ Γάγγεω αἱ ἐκβολαὶ καὶ πόλις Παλίμ-
3.1.1
βοθρα μεγίστη Ἰνδῶν πρὸς τῶν Γάγγῃ. ἐμοὶ δὲ <Ἐρα-
τοσθένης> ὁ Κυρηναῖος πιστότερος ἄλλου ἔστω, ὅτι γῆς
3.2.1
περιόδου πέρι ἔμελεν Ἐρατοσθένει. οὗτος ἀπὸ τοῦ ὄρεος
τοῦ Ταύρου, ἵνα τοῦ Ἰνδοῦ αἱ πηγαί, παρ' αὐτὸν <τὸν>
Ἰνδὸν ποταμὸν ἰόντι ἔστε ἐπὶ τὴν μεγάλην θάλασσαν
καὶ τοῦ Ἰνδοῦ τὰς ἐκβολὰς μυρίους σταδίους καὶ τρισχι-
3.3.1
λίους τὴν πλευρὴν λέγει ἐπέχειν τῆς γῆς τῆς Ἰνδῶν. ταυ-
τησὶ δὲ ἀντίπορον πλευρὴν ποιέει τὴν ἀπὸ τοῦ αὐτοῦ
ὄρεος παρὰ τὴν ἑῴην θάλασσαν, οὐκέτι ταύτῃ τῇ πλευρῇ
ἴσην, ἀλλὰ ἄκρην γὰρ ἀνέχειν ἐπὶ μέγα εἴσω εἰς τὸ πέ-
3.3.5
λαγος, ἐς τρισχιλίους σταδίους μάλιστα ἀνατείνουσαν τὴν
ἄκρην· εἴη ἂν ὦν αὐτῷ ἡ πλευρὴ τῆς Ἰνδῶν γῆς <ἡ>
πρὸς ἕω μυρίους καὶ ἑξακισχιλίους σταδίους ἐπέχουσα.
3.4.1
τοῦτο μὲν αὐτῷ πλάτος τῆς Ἰνδῶν γῆς συμβαίνει, μῆκος
δὲ τὸ ἀπ' ἑσπέρης ἐπὶ ἕω ἔστε μὲν ἐπὶ πόλιν Παλίμ-
βοθρα μεμετρημένον σχοίνοισι λέγει ἀναγράφειν καὶ
– εἶναι γὰρ ὁδὸν βασιληίην – τοῦτο ἐπέχειν ἐς μυ-  
3.4.5
ρίους σταδίους· τὰ δὲ ἐπέκεινα οὐκέτι ὡσαύτως ἀτρεκέα·
3.5.1
φήμας δὲ ὅσοι ἀνέγραψαν, ξὺν τῇ ἄκρῃ τῇ ἀνεχούσῃ ἐς
τὸ πέλαγος ἐς μυρίους σταδίους μάλιστα ἐπέχειν λέγου-
σιν· εἶναι δὲ ἂν ὦν τὸ μῆκος τῆς Ἰνδῶν γῆς σταδίων
3.6.1
μάλιστα δισμυρίων. <Κτησίης> δὲ ὁ Κνίδιος τὴν Ἰνδῶν
γῆν ἴσην τῇ ἄλλῃ Ἀσίῃ λέγει, οὐδὲν λέγων, οὐδὲ <Ὀνη-
σίκριτος>, τρίτην μοῖραν τῆς πάσης γῆς. <Νέαρχος> δὲ
μηνῶν τεσσάρων ὁδὸν τὴν δι' αὐτοῦ τοῦ πεδίου τῆς
3.7.1
Ἰνδῶν γῆς. <Μεγασθένει> δὲ τὸ ἀπὸ ἀνατολῶν ἐς ἑσπέ-
ρην πλάτος ἐστὶ τῆς Ἰνδῶν γῆς ὅ τι περ οἱ ἄλλοι μῆκος
ποιέουσι· καὶ λέγει <Μεγασθένης> μυρίων καὶ ἑξακισχι-
3.8.1
λίων σταδίων εἶναι ἵναπερ τὸ βραχύτατον αὐτοῦ. τὸ δὲ
ἀπὸ ἄρκτου πρὸς μεσημβρίην, τοῦτο δὲ αὐτῷ μῆκος γί-
νεται, καὶ ἐπέχει <σταδίους> τριηκοσίους καὶ δισχιλίους
καὶ δισμυρίους ἵναπερ τὸ στενότατον αὐτοῦ.
3.9.1
 ποταμοὶ δὲ τοσοίδε εἰσὶν ἐν τῇ Ἰνδῶν γῇ ὅσοι οὐδὲ
ἐν τῇ πάσῃ Ἀσίῃ. μέγιστοι μὲν ὁ Γάγγης τε καὶ ὁ Ἰν-
δός, ὅτου καὶ ἡ γῆ ἐπώνυμος, ἄμφω τοῦ τε Νείλου τοῦ
Αἰγυπτίου καὶ τοῦ Ἴστρου τοῦ Σκυθικοῦ, καὶ εἰ ἐς
3.10.1
ταὐτὸ συνέλθοι αὐτοῖσι τὸ ὕδωρ, μέζονες. δοκέειν δὲ 

ἔμοιγε, καὶ ὁ Ἀκεσίνης μέζων ἐστὶ τοῦ τε Ἴστρου καὶ τοῦ
Νείλου, ἵναπερ παραλαβὼν ἅμα τόν τε Ὑδάσπεα καὶ τὸν
Ὑδραώτεα καὶ τὸν Ὕφασιν ἐμβάλλει ἐς τὸν Ἰνδόν, ὡς
3.10.5
καὶ τριάκοντα αὐτῷ στάδια τὸ πλάτος ταύτῃ εἶναι· καὶ  
τυχὸν καὶ ἄλλοι πολλοὶ μέζονες ποταμοὶ ἐν τῇ Ἰνδῶν γῇ
ῥέουσιν.
4.1.1
 ἀλλὰ οὔ μοι ἀτρεκὲς ὑπὲρ τῶν ἐπέκεινα Ὑφάσιος
ποταμοῦ ἰσχυρίσασθαι, ὅτι οὐ πρόσω τοῦ Ὑφάσιος ἦλθεν
4.2.1
Ἀλέξανδρος. αὐτοῖν δὲ τοῖν μεγίστοιν ποταμοῖν τοῦ τε
Γάγγεω καὶ τοῦ Ἰνδοῦ τὸν Γάγγεα μεγέθει πολύ τι
ὑπερφέρειν <Μεγασθένης> ἀνέγραψε, καὶ ὅσοι ἄλλοι
4.3.1
μνήμην τοῦ Γάγγεω ἔχουσιν· αὐτόν τε γὰρ μέγαν ἀνίς-
χειν ἐκ τῶν πηγέων, δέχεσθαί τε ἐς ἑωυτὸν τόν τε
Καϊνὰν ποταμὸν καὶ τὸν Ἐραννοβόαν καὶ τὸν Κοσσό-
ανον, πάντας πλωτούς, ἔτι δὲ Σῶνόν τε ποταμὸν καὶ
4.4.1
Σιττόκατιν καὶ Σολόματιν, καὶ τούτους πλωτούς, ἐπὶ
δὲ Κονδοχάτην τε καὶ Σάμβον καὶ Μάγωνα καὶ Ἀγό-
ρανιν καὶ Ὤμαλιν. ἐμβάλλουσι δὲ ἐς αὐτὸν Κομμινά-
σης τε μέγας ποταμὸς καὶ Κάκουθις καὶ Ἀνδώματις ἐξ
4.5.1
ἔθνεος Ἰνδικοῦ τοῦ Μαδυανδινῶν ῥέων, καὶ ἐπὶ τού-
τοισιν Ἄμυστις παρὰ πόλιν Καταδούπην, καὶ Ὀξύμαγις
ἐπὶ <τοῖσι> Παζάλαις καλουμένοισι· καὶ Ἐρέννεσις ἐν
4.6.1
Μάθαις, ἔθνει Ἰνδικῷ, συμβάλλει τῷ Γάγγῃ. τού-
των λέγει <Μεγασθένης> οὐδένα εἶναι τοῦ Μαιάνδρου
4.7.1
ἀποδέοντα, ἵναπερ ναυσίπορος ὁ Μαίανδρος. εἶναι ὦν  
τὸ εὖρος τῷ Γάγγῃ, ἔνθαπερ αὐτὸς ἑωυτοῦ στεινότατος,
ἐς ἐκατὸν σταδίους· πολλαχῆ δὲ καὶ λιμνάζειν, ὡς μὴ
ἄποπτον εἶναι τὴν πέρην χώρην, ἵναπερ χθαμαλή τέ
4.8.1
ἐστι καὶ οὐδαμῆ γηλόφοισιν ἀνεστηκυῖα. τῷ δὲ Ἰνδῷ ἐς
ταὐτὸν ἔρχεται. Ὑδραώτης μὲν ἐν Καμβισθόλοις, παρει-
ληφὼς τόν τε Ὕφασιν ἐν Ἀστρύβαις καὶ τὸν Σαράγγην
ἐκ Κηκαίων καὶ τὸν Σύδρον ἐξ Ἀττακηνῶν <ῥέοντα>, ἐς
4.9.1
Ἀκεσίνην ἐμβάλλει. Ὑδάσπης δὲ ἐν Συδράκαις ἄγων
ἅμα οἷ τὸν Σίναρον ἐν Ἀρίσπῃσιν ἐς τὸν Ἀκεσίνην ἐκ-
4.10.1
διδοῖ καὶ οὗτος. ὁ δὲ Ἀκεσίνης ἐν Μαλλοῖς ξυμβάλλει
τῷ Ἰνδῷ· καὶ Τούταπος δὲ μέγας ποταμὸς ἐς τὸν Ἀκε-
σίνην ἐκδιδοῖ. τούτων ὁ Ἀκεσίνης ἐμπλησθεὶς καὶ τῇ
ἐπικλήσει ἐκνικήσας αὐτὸς τῷ ἑωυτοῦ ἤδη ὀνόματι ἐς-
4.11.1
βάλλει ἐς τὸν Ἰνδόν. Κωφὴν δὲ ἐν Πευκελαΐτιδι, ἅμα
οἷ ἄγων Μαλάμαντόν τε καὶ Σόαστον καὶ Γαροίαν, ἐκ-
4.12.1
διδοῖ ἐς τὸν Ἰνδόν. καθύπερθε δὲ τουτέων Πάρεννος
καὶ Σάπαρνος, οὐ πολὺ διέχοντες, ἐμβάλλουσιν ἐς τὸν
Ἰνδόν. Σόανος δὲ ἐκ τῆς ὀρεινῆς τῆς Ἀβισσαρέων ἔρη-
μος ἄλλου ποταμοῦ ἐκδιδοῖ ἐς αὐτόν. καὶ τουτέων τοὺς  
4.13.1
πολλοὺς <Μεγασθένης> λέγει ὅτι πλωτοί εἰσιν. οὔκουν
ἀπιστίαν χρὴ ἔχειν ὑπέρ τε τοῦ Ἰνδοῦ καὶ τοῦ Γάγγεω μηδὲ
συμβλητοὺς εἶναι αὐτοῖσι τόν τε Ἴστρον καὶ τοῦ Νείλου
4.14.1
τὸ ὕδωρ. ἐς μέν γε τὸν Νεῖλον οὐδένα ποταμὸν ἐκδι-
δόντα ἴσμεν, ἀλλ' ἀπ' αὐτοῦ διώρυχας τετμημένας κατὰ
4.15.1
τὴν χώρην τὴν Αἰγυπτίην· ὁ δὲ Ἴστρος ὀλίγος μὲν
ἀνίσχει ἀπὸ τῶν πηγέων, δέχεται δὲ πολλοὺς ποταμούς,
ἀλλὰ οὔτε πλήθει ἴσους τοῖς Ἰνδῶν ποταμοῖσιν, οἳ
ἐς τὸν Ἰνδὸν καὶ τὸν Γάγγην ἐκδιδοῦσιν, πλωτοὺς δὲ
4.15.5
δὴ καὶ κάρτα ὀλίγους, ὧν τοὺς μὲν αὐτὸς ἰδὼν οἶδα, τὸν
4.16.1
Ἔνον τε καὶ τὸν Σάον. Ἔνος μὲν ἐν μεθορίῳ τῆς Νω-
ρικῶν καὶ Ῥαιτῶν γῆς μίγνυται τῷ Ἴστρῳ, ὁ δὲ Σάος
κατὰ Παίονας. ὁ δὲ χῶρος, ἵναπερ συμβάλλουσιν οἱ πο-
ταμοί, Ταυροῦνος καλέεται. ὅστις δὲ καὶ ἄλλον οἶδε
4.16.5
ναυσίπορον τῶν ἐς τὸν Ἴστρον ἐκδιδόντων, ἀλλὰ οὐ
πολλούς που οἶδε.
5.1.1
 τὸ δὲ αἴτιον ὅστις ἐθέλει φράζειν τοῦ πλήθεός τε
καὶ μεγέθεος τῶν Ἰνδῶν ποταμῶν, φραζέτω· ἐμοὶ δὲ καὶ
5.2.1
ταῦτα ὡς ἀκοὴ ἀναγεγράφθω. ἐπεὶ καὶ ἄλλων πολλῶν
ποταμῶν οὐνόματα <Μεγασθένης> ἀνέγραψεν, οἳ ἔξω
τοῦ Γάγγεώ τε καὶ τοῦ Ἰνδοῦ ἐκδιδοῦσιν ἐς τὸν ἑῷόν
τε καὶ μεσημβρινὸν τὸν ἔξω πόντον, ὥστε τοὺς πάντας
5.2.5
ὀκτὼ καὶ πεντήκοντα λέγει ὅτι εἰσὶν Ἰνδοὶ ποταμοί,  
5.3.1
ναυσίποροι πάντες. ἀλλ' οὐδὲ <Μεγασθένης> πολλὴν
δοκέει μοι ἐπελθεῖν τῆς Ἰνδῶν χώρης, πλήν γε <δὴ> ὅτι
πλεῦνα ἢ οἱ ξὺν Ἀλεξάνδρῳ τῷ Φιλίππου ἐπελθόντες·
συγγενέσθαι γὰρ Σανδροκόττῳ λέγει, τῷ μεγίστῳ βασιλεῖ
5.4.1
Ἰνδῶν, καὶ Πώρου ἔτι τούτῳ μείζονι. οὗτος ὦν ὁ <Μεγα-
σθένης> λέγει, οὔτε Ἰνδοὺς ἐπιστρατεῦσαι οὐδαμοῖσιν ἀν-
5.5.1
θρώποισιν, οὔτε Ἰνδοῖσιν ἄλλους ἀνθρώπους, ἀλλὰ
Σέσωστριν μὲν τὸν Αἰγύπτιον, τῆς Ἀσίας καταστρεψά-
μενον τὴν πολλήν, ἔστε ἐπὶ τὴν Εὐρώπην σὺν στρατιῇ
5.6.1
ἐλάσαντα ὀπίσω ἀπονοστῆσαι, Ἰδάνθυρσον δὲ τὸν Σκύ-
θεα ἐκ Σκυθίης ὁρμηθέντα πολλὰ μὲν τῆς Ἀσίης ἔθνεα
καταστρέψασθαι, ἐπελθεῖν δὲ καὶ τὴν Αἰγυπτίων γῆν
5.7.1
κρατέοντα. Σεμίραμιν δὲ τὴν Ἀσσυρίην ἐπιχειρέειν μὲν
στέλλεσθαι εἰς Ἰνδούς, ἀποθανεῖν δὲ πρὶν τέλος ἐπιθεῖναι
τοῖς βουλεύμασιν. ἀλλὰ Ἀλέξανδρον γὰρ στρατεῦσαι ἐπ'
5.8.1
Ἰνδοὺς μοῦνον. καὶ πρὸ Ἀλεξάνδρου Διονύσου μὲν πέρι
πολλὸς λόγος κατέχει ὡς καὶ τούτου στρατεύσαντος ἐς  
Ἰνδοὺς καὶ καταστρεψαμένου Ἰνδούς, Ἡρακλέος δὲ πέρι
5.9.1
οὐ πολλός. Διονύσου μέν γε καὶ Νῦσα πόλις μνῆμα οὐ
φαῦλον τῆς στρατηλασίης, καὶ ὁ Μηρὸς τὸ ὄρος, καὶ ὁ
κισσὸς ὅτι ἐν τῷ ὄρει τούτῳ φύεται, καὶ αὐτοὶ οἱ Ἰνδοὶ
ὑπὸ τυμπάνων τε καὶ κυμβάλων στελλόμενοι ἐς τὰς
5.9.5
μάχας, καὶ ἐσθὴς αὐτοῖσι κατάστικτος ἐοῦσα, κατάπερ 


τοῦ Διονύσου τοῖσι βάκχοισιν· Ἡρακλέος δὲ οὐ πολλὰ
ὑπομνήματα. ἀλλὰ τὴν Ἄορνον γὰρ πέτρην, ἥντινα
Ἀλέξανδρος βίῃ ἐχειρώσατο, ὅτι Ἡρακλέης οὐ δυνατὸς
ἐγένετο ἐξελεῖν, Μακεδονικὸν δοκέει μοί τι κόμπασμα,
5.10.5
κατάπερ ὦν καὶ τὸν Παραπάμισον Καύκασον ἐκάλεον
Μακεδόνες, οὐδέν τι προσήκοντα τοῦτον τῷ Καυκάσῳ.
5.11.1
καί τι καὶ ἄντρον ἐπιφρασθέντες ἐν Παραπαμισάδαισι,
τοῦτο ἔφασαν ἐκεῖνο εἶναι τοῦ Προμηθέως τοῦ Τιτῆνος
τὸ ἄντρον, ἐν ὅτῳ ἐκρέματο ἐπὶ τῇ κλοπῇ τοῦ πυρός.
5.12.1
καὶ δὴ καὶ ἐν Σίβαισιν, Ἰνδικῷ γένει, ὅτι δορὰς ἀμπε-
χομένους εἶδον τοὺς Σίβας, ἀπὸ τῆς Ἡρακλέους στρα-
τηλασίης ἔφασκον τοὺς ὑπολειφθέντας εἶναι τοὺς Σίβας·
καὶ γὰρ καὶ σκυτάλην φορέουσί τε οἱ Σίβαι καὶ τῇσι
5.12.5
βουσὶν αὐτῶν ῥόπαλον ἐπικέκαυται, καὶ τοῦτο ἐς μνήμην
5.13.1
ἀνέφερον τοῦ ῥοπάλου τοῦ Ἡρακλέους. εἰ δέ τῳ πιστὰ
ταῦτα, ἄλλος ἂν οὗτος Ἡρακλέης εἴη, οὐχ ὁ Θηβαῖος ἢ  
ὁ Τύριος [οὗτος] ἢ ὁ Αἰγύπτιος, ἤ τις καὶ κατὰ τὴν
ἄνω χώρην οὐ πόρρω τῆς Ἰνδῶν γῆς ᾠκισμένος μέγας
5.13.5
βασιλεύς.
6.1.1
 ταῦτα μέν μοι ἐκβολὴ ἔστω τοῦ λόγου ἐς τὸ μὴ πι-
στὰ φαίνεσθαι ὅσα ὑπὲρ τῶν ἐπέκεινα τοῦ Ὑφάσιος πο-
ταμοῦ Ἰνδῶν μετεξέτεροι ἀνέγραψαν· (ἔστε γὰρ ἐπὶ τὸν
Ὕφασιν οἱ τῆς Ἀλεξάνδρου στρατηλασίης μετασχόντες
6.2.1
οὐ πάντη ἄπιστοί εἰσιν)· ἐπεὶ καὶ τόδε λέγει <Μεγα-
σθένης> ὑπὲρ ποταμοῦ Ἰνδικοῦ, Σίλαν μὲν εἶναί οἱ
ὄνομα, ῥέειν δὲ ἀπὸ κρήνης ἐπωνύμου τοῦ ποταμοῦ διὰ
τῆς χώρης τῆς Σιλαίων, καὶ τούτων ἐπωνύμων τοῦ πο-
6.3.1
ταμοῦ τε καὶ τῆς κρήνης, τὸ δὲ ὕδωρ παρέχεσθαι τοι-
όνδε. οὐδὲν εἶναι ὅτῳ ἀντέχει τὸ ὕδωρ, <καὶ> οὔτε τι
νήχεσθαι ἐπ' αὐτοῦ οὔτε τι ἐπιπλεῖν, ἀλλὰ πάντα γὰρ
ἐς βυσσὸν δύνειν· οὕτω τι ἀμενηνότερον πάντων εἶναι
6.3.5
τὸ ὕδωρ ἐκεῖνο καὶ ἠεροειδέστερον.
6.4.1
 ὕεται δὲ ἡ Ἰνδῶν γῆ τοῦ θέρεος, μάλιστα μὲν τὰ ὄρεα,
Παραπάμισός τε καὶ ὁ Ἠμωδὸς καὶ τὸ Ἰμαϊκὸν ὄρος,
καὶ ἀπὸ τουτέων μεγάλοι καὶ θολεροὶ οἱ ποταμοὶ ῥέουσιν.
6.5.1
ὕεται δὲ τοῦ θέρους καὶ τὰ πεδία τῶν Ἰνδῶν, ὥστε
λιμνάζει τὰ πολλὰ αὐτέων. καὶ ἔφυγεν ἡ Ἀλεξάνδρου
στρατιὴ ἀπὸ τοῦ Ἀκεσίνου ποταμοῦ μέσου θέρεος, ὑπερ-
6.6.1
βαλόντος τοῦ ὕδατος ἐς τὰ πεδία. ὥστε ἀπὸ τῶνδε ἔξεστι  
τεκμηριοῦσθαι καὶ τοῦ Νείλου τὸ πάθημα τοῦτο, ὅτι
εἰκὸς [εἶναι] ὕεσθαι τὰ Αἰθιόπων ὄρεα τοῦ θέρεος, καὶ
ἀπ' ἐκείνων ἐμπιπλάμενον τὸν Νεῖλον ὑπερβάλλειν ὑπὲρ
6.7.1
τὰς ὄχθας ἐς τὴν γῆν τὴν Αἰγυπτίην. θολερὸς ὦν καὶ
οὗτος ῥέει ἐν τῆδε τῇ ὥρῃ, ὡς οὔτε ἂν ἀπὸ χιόνος τηκο-
μένης ἔρρεεν, οὔτε εἰ πρὸς τῶν ὥρῃ θέρεος πνεόντων
ἐτησίων ἀνέμων ἀνεκόπτετό οἱ τὸ ὕδωρ· ἄλλως τε οὐδὲ
6.8.1
χιονόβλητα εἴη ἂν τὰ Αἰθιόπων ὄρεα ὑπὸ καύματος. ὕεσθαι
δὲ κατάπερ τὰ Ἰνδῶν οὐκ ἔξω ἐστὶ τοῦ εἰκότος, ἐπεὶ
καὶ τἄλλα <ἡ> Ἰνδῶν γῆ οὐκ ἀπέοικε τῆς Αἰθιοπίης καὶ
οἱ ποταμοὶ οἱ Ἰνδοὶ ὁμοίως τῷ Νείλῳ τῷ Αἰθιοπηίῳ τε
6.8.5
καὶ Αἰγυπτίῳ κροκοδείλους τε φέρουσιν, ἔστιν δὲ οἳ
αὐτῶν καὶ ἰχθύας καὶ ἄλλα κήτεα ὅσα ὁ Νεῖλος πλὴν
ἵππου τοῦ ποταμίου,  – <Ὀνησίκριτος> δὲ καὶ τοὺς
6.9.1
ἵππους τοὺς ποταμίους λέγει ὅτι φέρουσι – τῶν τε ἀν-
θρώπων αἱ ἰδέαι οὐ πάντη ἀπᾴδουσιν αἱ Ἰνδῶν τε καὶ
Αἰθιόπων. οἱ μὲν πρὸς νότου ἀνέμου Ἰνδοὶ τοῖς Αἰθίοψι
μᾶλλόν τι ἐοίκασι μέλανές τε ἰδέσθαι εἰσὶ καὶ ἡ κόμη
6.9.5
αὐτοῖς μέλαινα, πλήν γε δὴ ὅτι σιμοὶ οὐχ ὡσαύτως οὐδὲ
οὐλόκρανοι ὡς Αἰθίοπες. οἱ δὲ βορειότεροι τούτων κατ'
Αἰγυπτίους μάλιστα ἂν εἶεν τὰ σώματα.
7.1.1
 ἔθνεα δὲ Ἰνδικὰ εἴκοσι καὶ ἑκατὸν τὰ ἅπαντα λέγει  
<Μεγασθένης>, δυοῖν δέοντα. καὶ πολλὰ μὲν εἶναι ἔθνεα
Ἰνδικὰ καὶ αὐτὸς συμφέρομαι <Μεγασθένει>, τὸ δὲ
ἀτρεκὲς οὐκ ἔχω εἰκάσαι ὅπως ἐκμαθὼν ἀνέγραψεν, οὐδὲ
7.1.5
πολλοστὸν μέρος τῆς Ἰνδῶν γῆς ἐπελθών, οὐδὲ ἐπιμι-
7.2.1
ξίης πᾶσι τοῖς γένεσιν ἐούσης ἐς ἀλλήλους. πάλαι μὲν
δὴ νομάδας εἶναι Ἰνδούς, καθάπερ Σκυθέων τοὺς οὐκ
ἀροτῆρας, οἳ ἐπὶ τῇσιν ἁμάξῃσι πλανώμενοι ἄλλοτε ἄλ-
λην τῆς Σκυθίης ἀμείβουσιν, οὔτε πόληας οἰκέοντες οὔτε
7.3.1
ἱερὰ θεῶν σέβοντες. οὕτω μηδὲ Ἰνδοῖσι πόληας εἶναι
μηδὲ ἱερὰ θεῶν δεδομημένα, ἀλλ' ἀμπίσχεσθαι μὲν δο-
ρὰς θηρίων ὅσων κατακάνοιεν, σιτέεσθαι δὲ τῶν δεν-
δρέων τὸν φλοιόν. καλέεσθαι δὲ τὰ δένδρεα ταῦτα τῇ
7.3.5
Ἰνδῶν φωνῆ τάλα, καὶ φύεσθαι ἐπ' αὐτῶν, κατάπερ τῶν
7.4.1
φοινίκων ἐπὶ τῇσι κορυφῇσιν, οἷά περ τολύπας. σιτέεσθαι
δὲ καὶ τῶν θηρίων ὅσα ἕλοιεν ὠμοφαγέοντας, πρίν γε
7.5.1
δὴ Διόνυσον ἐλθεῖν ἐς τὴν χώρην τῶν Ἰνδῶν. Διόνυσον
δὲ ἐλθόντα, ὡς καρτερὸς ἐγένετο Ἰνδῶν, πόληάς τε
οἰκίσαι καὶ νόμους θέσθαι τῇσι πόλεσιν, οἴνου τε δο-
τῆρα Ἰνδοῖς γενέσθαι κατάπερ Ἕλλησι, καὶ σπείρειν δι-
7.6.1
δάξαι τὴν γῆν διδόντα αὐτὸν σπέρματα, ἢ οὐκ ἐλάσαντος
ταύτῃ Τριπτολέμου, ὅτε περ ἐκ Δήμητρος ἐστάλη σπεί-
ρειν τὴν γῆν πᾶσαν, ἢ πρὸ Τριπτολέμου τις οὗτος Διό-
νυσος ἐπελθὼν τὴν Ἰνδῶν γῆν σπέρματά σφισιν
7.7.1
ἔδωκε καρποῦ τοῦ ἡμέρου. βόας τε ὑπ' ἄροτρον ζεῦξαι  
Διόνυσον πρῶτον καὶ ἀροτῆρας ἀντὶ νομάδων ποιῆσαι
Ἰνδῶν τοὺς πολλοὺς καὶ ὁπλίσαι ὅπλοισι τοῖσιν ἀρηίοισι.
7.8.1
καὶ θεοὺς σέβειν ὅτι ἐδίδαξε Διόνυσος ἄλλους τε καὶ
μάλιστα δὴ ἑωυτὸν κυμβαλίζοντας καὶ τυμπανίζοντας·
καὶ ὄρχησιν δὲ ἐκδιδάξαι τὴν σατυρικήν, τὸν κόρδακα 


παρ' Ἕλλησι καλούμενον, καὶ κομᾶν [Ἰνδοὺς] τῷ θεῷ
μιτρηφορέειν τε ἀναδεῖξαι καὶ μύρων ἀλοιφὰς ἐκδιδάξαι,
ὥστε καὶ εἰς Ἀλέξανδρον ἔτι ὑπὸ κυμβάλων τε καὶ
τυμπάνων ἐς τὰς μάχας Ἰνδοὶ καθίσταντο.
8.1.1
 ἀπιόντα δὲ ἐκ τῆς Ἰνδῶν γῆς, ὥς οἱ ταῦτα κεκοσμέατο,
καταστῆσαι βασιλέα τῆς χώρης Σπατέμβαν, τῶν ἑταίρων
ἕνα τὸν βακχωδέστατον· τελευτήσαντος δὲ Σπατέμβα
τὴν βασιληίην ἐκδέξασθαι Βουδύαν τὸν τούτου παῖδα.
8.2.1
καὶ τὸν μὲν πεντήκοντα καὶ δύο ἔτεα βασιλεῦσαι Ἰνδῶν,
τὸν πατέρα, τὸν δὲ παῖδα εἴκοσιν ἔτεα. καὶ τούτου
8.3.1
παῖδα ἐκδέξασθαι τὴν βασιληίην Κραδεύαν, καὶ τὸ ἀπὸ
τοῦδε τὸ πολὺ μὲν κατὰ γένος ἀμείβειν τὴν βασιληίην,
παῖδα παρὰ πατρὸς ἐκδεχόμενον· εἰ δὲ ἐκλείποι τὸ γένος,
οὕτω δὴ ἀριστίνδην καθίστασθαι Ἰνδοῖσι βασιλέας.
8.4.1
Ἡρακλέα δέ, ὅντινα ἐς Ἰνδοὺς ἀφικέσθαι λόγος κατέχει,
8.5.1
παρ' αὐτοῖσιν Ἰνδοῖσι γηγενέα λέγεσθαι. τοῦτον τὸν
Ἡρακλέα μάλιστα πρὸς Σουρασηνῶν γεραίρεσθαι, Ἰν-
δικοῦ ἔθνεος, ἵνα δύο πόληες μεγάλαι, Μέθορά τε καὶ
Κλεισόβορα· καὶ ποταμὸς Ἰωμάνης πλωτὸς διαρρεῖ τὴν  
8.6.1
χώρην αὐτῶν· τὴν σκευὴν δὲ οὗτος ὁ Ἡρακλέης ἥντινα
ἐφόρεε <Μεγασθένης> λέγει ὅτι ὁμοίην τῷ Θηβαίῳ
Ἡρακλεῖ, ὡς αὐτοὶ Ἰνδοὶ ἀπηγέονται. καὶ τούτῳ ἄρσε-
νας μὲν παῖδας πολλοὺς κάρτα γενέσθαι ἐν τῇ Ἰνδῶν
8.6.5
γῇ – πολλῇσι γὰρ δὴ γυναιξὶν ἐς γάμον ἐλθεῖν καὶ
τοῦτον τὸν Ἡρακλέα – , θυγατέρα δὲ μουνογενέην.
8.7.1
οὔνομα δὲ εἶναι τῇ παιδὶ Πανδαίην, καὶ τὴν χώρην,
ἵνα τε ἐγένετο καὶ ἧστινος ἐπέτρεψεν αὐτῇ ἄρχειν
Ἡρακλέης, Πανδαίην <καλεῖσθαι> τῆς παιδὸς ἐπώνυμον.
καὶ ταύτῃ ἐλέφαντας μὲν γενέσθαι ἐκ τοῦ πατρὸς ἐς
8.7.5
πεντακοσίους, ἵππον δὲ ἐς τετρακισχιλίην, πεζῶν δὲ ἐς
8.8.1
τὰς τρεῖς καὶ δέκα μυριάδας. καὶ τάδε μετεξέτεροι Ἰνδῶν
περὶ Ἡρακλέους λέγουσιν, ἐπελθόντα αὐτὸν πᾶσαν γῆν
καὶ θάλασσαν καὶ καθήραντα ὅ τι περ κακόν, καινὸν
8.9.1
εἶδος ἐξευρεῖν ἐν τῇ θαλάσσῃ κόσμου γυναικηίου, ὅντινα
καὶ εἰς τοῦτο ἔτι οἵ τε ἐξ Ἰνδῶν τῆς χώρης τὰ ἀγώγιμα
παρ' ἡμέας ἀγινέοντες σπουδῇ ὠνεόμενοι ἐκκομίζουσι,
καὶ Ἑλλήνων δὲ πάλαι καὶ Ῥωμαίων νῦν ὅσοι πολυ-
8.9.5
κτέανοι καὶ εὐδαίμονες μέζονι ἔτι σπουδῆ ὠνέονται,
τὸν μαργαρίτην δὴ τὸν θαλάσσιον οὕτω τῇ Ἰνδῶν
8.10.1
γλώσσῃ καλεόμενον. τὸν γὰρ Ἡρακλέα, ὡς καλόν οἱ
ἐφάνη τὸ φόρημα, ἐκ πάσης τῆς θαλάσσης ἐς τὴν Ἰνδῶν
γῆν συναγινέειν τὸν μαργαρίτην δὴ τοῦτον, τῇ θυγατρὶ  
8.11.1
τῇ ἑωυτοῦ εἶναι κόσμον. καὶ λέγει <Μεγασθένης>, θη-
ρεύεσθαι τὴν κόγχην αὐτοῦ δικτύοισι, νέμεσθαι δ' ἐν
τῇ θαλάσσῃ κατὰ ταὐτὸ πολλὰς κόγχας, κατάπερ τὰς
μελίσσας. καὶ εἶναι γὰρ καὶ τοῖσι μαργαρίτῃσι βασιλέα
8.12.1
ἢ βασίλισσαν, ὡς τῇσι μελίσσῃσι. καὶ ὅστις μὲν ἐκεῖνον
κατ' ἐπιτυχίην συλλάβοι, τοῦτον δὲ εὐπετέως περιβάλ-
λειν καὶ τὸ ἄλλο σμῆνος τῶν μαργαριτῶν· εἰ δὲ διαφύγοι
σφᾶς ὁ βασιλεύς, τούτῳ δὲ οὐκέτι θηρατοὺς εἶναι τοὺς
8.12.5
ἄλλους. τοὺς ἑλόντας δὲ περιορᾶν κατασαπῆναί σφισι
8.13.1
τὴν σάρκα, τῷ δὲ ὀστέῳ ἐς κόσμον χρῆσθαι. καὶ εἶναι
γὰρ καὶ παρ' Ἰνδοῖσι τὸν μαργαρίτην τριστάσιον κατὰ
τιμὴν πρὸς χρυσίον τὸ ἄπεφθον, καὶ τοῦτο ἐν τῇ Ἰνδῶν
γῇ ὀρυσσόμενον.
9.1.1
 ἐν δὲ τῇ χώρῃ ταύτῃ, ἵνα ἐβασίλευσεν ἡ θυγάτηρ
τοῦ Ἡρακλέος, τὰς μὲν γυναῖκας ἑπταέτεις ἐούσας ἐς
ὥρην γάμου ἰέναι, τοὺς δὲ ἄνδρας τεσσαράκοντα ἔτεα
9.2.1
τὰ πλεῖστα βιώσκεσθαι. καὶ ὑπὲρ τούτου λεγόμενον
λόγον εἶναι παρὰ Ἰνδοῖσιν. Ἡρακλέα, ὀψιγόνου οἱ γε-
νομένης τῆς παιδός, ἐπεί τε δὴ ἐγγὺς ἔμαθεν ἑαυτῷ
ἐοῦσαν τὴν τελευτήν, οὐκ ἔχοντα ὅτῳ ἀνδρὶ ἐκδῷ τὴν
9.2.5
παῖδα ἑωυτοῦ ἐπαξίῳ, αὐτὸν μιγῆναι τῇ παιδὶ ἑπταέτεϊ
ἐούσῃ, ὡς γένος ἐξ οὗ τε κἀκείνης ὑπολείπεσθαι Ἰνδῶν
9.3.1
βασιλέας. ποιῆσαι ὦν αὐτὴν Ἡρακλέα ὡραίην γάμου·
καὶ ἐκ τοῦδε ἅπαν τὸ γένος τοῦτο ὅτου ἡ Πανδαίη
9.4.1
ἐπῆρξε, ταὐτὸν τοῦτο γέρας ἔχειν παρὰ Ἡρακλέος. ἐμοὶ  
δὲ δοκεῖ, εἴπερ ὦν τὰ ἐς τοσόνδε ἄτοπα Ἡρακλέης οἷός
τε ἦν ἐξεργάζεσθαι, κἂν αὑτὸν ἀποφῆναι μακροβιώτερον,
9.5.1
ὡς ὡραίῃ μιγῆναι τῇ παιδί. ἀλλὰ γὰρ εἰ ταῦτα ὑπὲρ τῆς
ὥρης τῶν ταύτῃ παίδων ἀτρεκέα ἐστίν, ἐς ταὐτὸν φέρειν
δοκεῖ ἔμοιγε ἐς ὅ τι περ καὶ <τὰ> ὑπὲρ τῶν ἀνδρῶν τῆς
ἡλικίης ὅτι τεσσαρακοντούτεες ἀποθνήσκουσιν οἱ πρε-
9.6.1
σβύτατοι αὐτῶν. οἷς γὰρ τό τε γῆρας τοσῷδε ταχύτερον
ἐπέρχεται καὶ ὁ θάνατος ὁμοῦ τῷ γήρᾳ, πάντως που καὶ
9.7.1
ἡ ἀκμὴ πρὸς λόγον τοῦ τέλεος ταχυτέρη ἐπανθέει. ὥστε
τριακοντούτεες μὲν ὠμογέροντες ἄν που εἶεν αὐτοῖσιν
οἱ ἄνδρες, εἴκοσι δὲ ἔτεα γεγονότες οἱ ἔξω ἥβης νεηνί-
σκοι, ἡ δὲ ἀκροτάτη ἥβη ἀμφὶ τὰ πεντεκαίδεκα ἔτεα· καὶ
9.7.5
τῇσι γυναιξὶν ὥρη τοῦ γάμου κατὰ λόγον ἂν οὕτω ἐς
9.8.1
τὰ ἑπτὰ ἔτεα συμβαίνοι. καὶ γὰρ τοὺς καρποὺς ἐν ταύτῃ
τῇ χώρῃ πεπαίνεσθαί τε ταχύτερον [μὲν] τῆς ἄλλης
αὐτὸς οὗτος <Μεγασθένης> ἀνέγραψεν καὶ φθίνειν τα-
χύτερον.
9.9.1
 ἀπὸ μὲν δὴ Διονύσου βασιλέας ἠρίθμεον Ἰνδοὶ ἐς
Σανδρόκοττον τρεῖς καὶ πεντήκοντα καὶ ἑκατόν, ἔτεα δὲ
δύο καὶ τεσσαράκοντα καὶ ἑξακισχίλια· ἐν δὲ τούτοισι
τρὶς τὸ πᾶν εἰς ἐλευθερίην ***, τὴν δὲ καὶ ἐς τριακό-  
9.10.1
σια, τὴν δὲ εἴκοσίν τε ἐτέων καὶ ἑκατόν. πρεσβύτερόν
τε Διόνυσον Ἡρακλέος δέκα καὶ πέντε γενεῇσιν Ἰνδοὶ
λέγουσιν· ἄλλον δὲ οὐδένα ἐμβαλεῖν ἐς γῆν τὴν Ἰνδῶν 


ἐπὶ πολέμῳ, οὐδὲ Κῦρον τὸν Καμβύσεω, καίτοι ἐπὶ
9.10.5
Σκύθας ἐλάσαντα καὶ τἄλλα πολυπραγμονέστατον δὴ
τῶν κατὰ τὴν Ἀσίαν βασιλέων γενόμενον τὸν Κῦρον.
9.11.1
ἀλλὰ Ἀλέξανδρον γὰρ ἐλθεῖν τε καὶ κρατῆσαι [πάντων]
τοῖς ὅπλοις ὅσους γε δὴ ἐπῆλθε· καὶ ἂν καὶ πάντων κρα-
9.12.1
τῆσαι, εἰ ἡ στρατιὴ ἤθελεν. οὐ μὲν δὴ οὐδὲ Ἰνδῶν τινὰ
ἔξω τῆς οἰκείης σταλῆναι ἐπὶ πολέμῳ διὰ δικαιότητα.
10.1.1
 λέγεται δὲ καὶ τάδε, μνημεῖα ὅτι Ἰνδοὶ τοῖς τελευ-
τήσασιν οὐ ποιέουσιν, ἀλλὰ τὰς ἀρετὰς γὰρ τῶν ἀνδρῶν
ἱκανὰς ἐς μνήμην τίθενται τοῖσιν ἀποθανοῦσι καὶ τὰς
10.2.1
ᾠδὰς αἳ αὐτοῖσιν ἐπᾴδονται. πόλεων δὲ καὶ ἀριθμὸν οὐκ
εἶναι ἂν ἀτρεκὲς ἀναγράψαι τῶν Ἰνδικῶν ὑπὸ πλήθεος·
ἀλλὰ γὰρ ὅσαι παραποτάμιαι αὐτέων ἢ παραθαλάσσιαι,
10.3.1
ταύτας μὲν ξυλίνας ποιέεσθαι· οὐ γὰρ ἂν ἐκ πλίνθου
ποιεομένας διαρκέσαι ἐπὶ χρόνον τοῦ τε ὕδατος ἕνεκα
τοῦ ἐξ οὐρανοῦ καὶ ὅτι οἱ ποταμοὶ αὐτοῖσιν ὑπερβάλ-
λοντες ὑπὲρ τὰς ὄχθας ἐμπιμπλᾶσι τοῦ ὕδατος τὰ πεδία.
10.4.1
ὅσαι δὲ ἐν ὑπερδεξίοις τε καὶ μετεώροις τόποισι καὶ
τούτοισι ψιλοῖσιν ᾠκισμέναι εἰσί, ταύτας δὲ ἐκ πλίνθου
10.5.1
τε καὶ πηλοῦ ποιέεσθαι. μεγίστην δὲ πόλιν Ἰνδοῖσιν
εἶναι <τὴν> Παλίμβοθρα καλεομένην, ἐν τῇ Πρασίων
γῇ, ἵνα αἱ συμβολαί εἰσι τοῦ τε Ἐραννοβόα ποταμοῦ καὶ  
τοῦ Γάγγεω· τοῦ μὲν Γάγγεω, τοῦ μεγίστου ποταμῶν·
10.5.5
ὁ δὲ Ἐραννοβόας τρίτος μὲν ἂν εἴη τῶν Ἰνδῶν ποτα-
μῶν, μέζων δὲ τῶν ἄλλῃ καὶ οὗτος, ἀλλὰ ξυγχωρέει
αὐτὸς τῷ Γάγγῃ, ἐπειδὰν ἐμβάλῃ ἐς αὐτὸν τὸ ὕδωρ.
10.6.1
καὶ λέγει <Μεγασθένης> μῆκος μὲν ἐπέχειν τὴν πόλιν
καθ' ἑκατέρην τὴν πλευρήν, ἵναπερ μακροτάτη αὐτὴ
ἑωυτῆς ᾤκισται, ἐς ὀγδοήκοντα σταδίους, τὸ δὲ πλάτος
10.7.1
ἐς πεντεκαίδεκα. τάφρον δὲ περιβεβλῆσθαι τῇ πόλει τὸ
εὖρος ἑξάπλεθρον, τὸ δὲ βάθος τριήκοντα πήχεων· πύρ-
γους δὲ ἑβδομήκοντα καὶ πεντακοσίους ἔχειν τὸ τεῖχος
10.8.1
καὶ πύλας τέσσαρας καὶ ἑξήκοντα. εἶναι δὲ καὶ τόδε
μέγα ἐν τῇ Ἰνδῶν γῇ, πάντας Ἰνδοὺς εἶναι ἐλευθέρους,
10.9.1
οὐδέ τινα δοῦλον εἶναι Ἰνδόν. τοῦτο μὲν Λακεδαιμονίοι-
σιν ἐς ταὐτὸ συμβαίνει καὶ Ἰνδοῖσι. Λακεδαιμονίοις μέν
γε οἱ εἵλωτες δοῦλοί εἰσιν καὶ τὰ δούλων ἐργάζονται,
Ἰνδοῖσι δὲ οὐδὲ ἄλλος δοῦλός ἐστι, μήτι γε Ἰνδῶν τις.
11.1.1
 νενέμηνται δὲ οἱ πάντες Ἰνδοὶ ἐς ἑπτὰ μάλιστα γένεα.
ἓν μὲν αὐτοῖσιν οἱ σοφισταί εἰσι, πλήθει μὲν μείους τῶν
11.2.1
ἄλλων, δόξῃ δὲ καὶ τιμῇ γεραρώτατοι· οὔτε γάρ τι τῷ
σώματι ἐργάζεσθαι ἀναγκαίη σφιν προσκέαται οὔτε τι  
ἀποφέρειν ἀφ' ὅτων πονέουσιν ἐς τὸ κοινόν. οὐδέ τι
ἄλλο ἀνάγκης ἁπλῶς ἐπεῖναι τοῖς σοφιστῇσιν, ὅτι μὴ
11.2.5
θύειν τὰς θυσίας τοῖσι θεοῖσιν ὑπὲρ τοῦ κοινοῦ <τῶν>
11.3.1
Ἰνδῶν· καὶ ὅστις δὲ ἰδίᾳ θύει, ἐξηγητὴς αὐτῷ τῆς θυ-
σίης τῶν τις σοφιστῶν τούτων γίνεται, ὡς οὐκ ἂν ἄλλως
11.4.1
κεχαρισμένα τοῖς θεοῖς θύσαντας. εἰσὶ δὲ καὶ μαντικῆς
οὗτοι μοῦνοι Ἰνδῶν δαήμονες, οὐδὲ ἐφεῖται ἄλλῳ μαν-
11.5.1
τεύεσθαι ὅτι μὴ σοφιστῇ ἀνδρί. μαντεύονται δὲ ὅσα
ὑπὲρ τῶν ὡρέων τοῦ ἔτεος καὶ εἴ τις ἐς τὸ κοινὸν συμ-
φορὴ καταλαμβάνει· τὰ ἴδια <δὲ> ἑκάστοισιν οὔ σφιν
μέλει μαντεύεσθαι, ὡς οὐκ ἐξικνεομένης τῆς μαντικῆς
11.5.5
ἐς τὰ μικρότερα ἢ ὡς οὐκ ἄξιον <ὂν> ἐπὶ τούτοισι πο-
11.6.1
νέεσθαι. ὅστις δὲ ἁμάρτοι ἐς τρὶς μαντευσάμενος, τούτῳ
δὲ ἄλλο μὲν κακὸν γίνεσθαι οὐδέν, σιωπᾶν δὲ εἶναι ἐπά-
ναγκες τοῦ λοιποῦ· καὶ οὐκ ἔστιν ὅστις ἐξαναγκάσει τὸν
ἄνδρα τοῦτον φωνῆσαι, ὅτου ἡ σιωπὴ κατακέκριται.
11.7.1
οὗτοι γυμνοὶ διαιτῶνται οἱ σοφισταί, τοῦ μὲν χειμῶνος
ὑπαίθριοι ἐν τῷ ἡλίῳ, τοῦ δὲ θέρεος, ἐπὴν ὁ ἥλιος κατ-
έχῃ, ἐν τοῖς λειμῶσι καὶ τοῖσιν ἕλεσιν ὑπὸ δένδρεσι
μεγάλοισιν, ὧν τὴν σκιὴν <Νέαρχος> λέγει ἐς πέντε πλέ-  
11.7.5
θρα ἐν κύκλῳ ἐξικνέεσθαι, καὶ ἂν καὶ μυρίους ἀνθρώ-
πους ὑπὸ ἑνὶ δένδρεϊ σκιάζεσθαι· τηλικαῦτα εἶναι ταῦτα
11.8.1
τὰ δένδρεα. σιτέονται δὲ <τὰ> ὡραῖα καὶ τὸν φλοιὸν
τῶν δένδρων, γλυκύν τε ὄντα τὸν φλοιὸν καὶ τρόφιμον
οὐ μεῖον ἤπερ αἱ βάλανοι τῶν φοινίκων.
11.9.1
 δεύτεροι δ' ἐπὶ τούτοισιν οἱ γεωργοί εἰσιν, οὗτοι πλή-
θει πλεῖστοι Ἰνδῶν ἐόντες. καὶ τούτοισιν οὔτε ὅπλα
ἐστὶν ἀρήια οὔτε μέλει τὰ πολεμήια ἔργα, ἀλλὰ τὴν
χώρην οὗτοι ἐργάζονται, καὶ τοὺς φόρους τοῖς τε βασι-
11.9.5
λεῦσι καὶ τῇσι πόλεσιν, ὅσαι αὐτόνομοι, οὗτοι ἀποφέ-
11.10.1
ρουσι. καὶ εἰ πόλεμος ἐς ἀλλήλους τοῖσιν Ἰνδοῖσι τύχοι,
τῶν ἐργαζομένων τὴν γῆν οὐ θέμις σφιν ἅπτεσθαι οὐδὲ
αὐτὴν τὴν γῆν τέμνειν, ἀλλὰ οἳ μὲν πολεμοῦσι καὶ κα-
τακαίνουσιν ἀλλήλους ὅπως τύχοιεν, οἳ δὲ πλησίον
11.10.5
αὐτῶν κατ' ἡσυχίαν ἀροῦσιν ἢ τρυγῶσιν ἢ κλαδῶσιν ἢ
θερίζουσιν.
11.11.1
 τρίτοι δέ εἰσιν Ἰνδοῖσιν οἱ νομέες, οἱ ποιμένες τε
καὶ βουκόλοι. καὶ οὗτοι οὔτε κατὰ πόληας οὔτε ἐν τῇσι
κώμῃσιν οἰκέουσι νομάδες τέ εἰσι καὶ ἀνὰ τὰ ὄρεα βιο-
τεύουσι. φόρον δὲ καὶ οὗτοι ἀπὸ τῶν κτηνέων ἀποφέρουσι,
11.11.5
καὶ θηρεύουσιν οὗτοι ἀνὰ τὴν χώρην ὄρνιθάς τε καὶ
ἄγρια θηρία.
12.1.1
 τέταρτον δέ ἐστι τὸ δημιουργικόν τε καὶ καπηλικὸν  
γένος. καὶ οὗτοι λειτουργοί εἰσι καὶ φόρον ἀποφέρου-
σιν ἀπὸ τῶν ἔργων τῶν σφετέρων, πλήν γε δὴ ὅσοι τὰ
ἀρήια ὅπλα ποιέουσιν· οὗτοι δὲ καὶ μισθὸν ἐκ τοῦ κοι-
12.1.5
νοῦ προσλαμβάνουσιν. ἐν δὲ τούτῳ τῷ γένει οἵ τε ναυ-
πηγοὶ καὶ οἱ ναῦταί εἰσιν, ὅσοι κατὰ τοὺς ποταμοὺς 

πλώουσι.
12.2.1
 πέμπτον δὲ γένος ἐστὶν Ἰνδοῖσιν οἱ πολεμισταί,
πλήθει μὲν δεύτερον μετὰ τοὺς γεωργούς, πλείστῃ δὲ
ἐλευθερίῃ τε καὶ εὐθυμίῃ ἐπιχρεόμενον. καὶ οὗτοι ἀς-
12.3.1
κηταὶ μόνων τῶν πολεμικῶν ἔργων εἰσίν· τὰ δὲ ὅπλα
ἄλλοι αὐτοῖς ποιέουσι καὶ ἵππους ἄλλοι παρέχουσι καὶ
διακονοῦσιν ἐπὶ στρατοπέδου ἄλλοι, οἳ τούς τε ἵππους
αὐτοῖς θεραπεύουσι καὶ τὰ ὅπλα ἐκκαθαίρουσι καὶ τοὺς
12.3.5
ἐλέφαντας ἄγουσι καὶ τὰ ἅρματα κοσμέουσί τε καὶ ἡνιο-
12.4.1
χεύουσιν. αὐτοὶ δέ, ἔστ' ἂν μὲν πολεμεῖν δέῃ, πολεμοῦ-
σιν, εἰρήνης δὲ γενομένης εὐθυμέονται· καί σφιν μισθὸς
ἐκ τοῦ κοινοῦ τοσόσδε ἔρχεται ὡς καὶ ἄλλους τρέφειν
ἀπ' αὐτοῦ εὐμαρέως.
12.5.1
 ἕκτοι δέ εἰσιν Ἰνδοῖσιν οἱ ἐπίσκοποι καλεόμενοι. οὗτοι
ἐφορῶσι τὰ γινόμενα κατά τε τὴν χώρην καὶ κατὰ τὰς
πόληας, καὶ ταῦτα ἀναγγέλλουσι τῷ βασιλεῖ, ἵναπερ βα-
σιλεύονται Ἰνδοί, ἢ τοῖς τέλεσιν, ἵναπερ αὐτόνομοί εἰσι.
12.5.5
καὶ τούτοις οὐ θέμις ψεῦδος ἀγγεῖλαι οὐδέν, οὐδέ τις
Ἰνδῶν αἰτίην ἔσχε ψεύσασθαι.
12.6.1
 ἕβδομοι δέ εἰσιν οἱ ὑπὲρ τῶν κοινῶν βουλευόμενοι
ὁμοῦ τῷ βασιλεῖ ἢ κατὰ πόληας ὅσαι αὐτόνομοι σὺν
12.7.1
τῇσιν ἀρχῇσι. πλήθει μὲν ὀλίγον τὸ γένος τοῦτό ἐστι,  
σοφίῃ δὲ καὶ δικαιότητι ἐκ πάντων προκεκριμένον.
ἔνθεν οἵ τε ἄρχοντες αὐτοῖσιν ἐπιλέγονται καὶ ὅσοι νο-
μάρχαι καὶ ὕπαρχοι καὶ θησαυροφύλακές τε καὶ στρα-
12.7.5
τοφύλακες, ναύαρχοί τε καὶ ταμίαι καὶ τῶν κατὰ γεωρ-
γίην ἔργων ἐπιστάται.
12.8.1
 γαμέειν δὲ ἐξ ἑτέρου γένεος οὐ θέμις, οἷον τοῖσι
γεωργοῖσιν ἐκ τοῦ δημιουργικοῦ ἢ ἔμπαλιν. οὐδὲ δύο
τέχνας ἐπιτηδεύειν τὸν αὐτὸν οὐδὲ τοῦτο θέμις, οὐδὲ
ἀμείβειν ἐξ ἑτέρου γένεος εἰς ἕτερον, οἷον γεωργικὸν ἐκ
12.9.1
νομέως γενέσθαι ἢ νομέα ἐκ δημιουργικοῦ. μοῦνόν
σφισιν ἀνεῖται σοφιστὴν ἐκ παντὸς γένεος γενέσθαι, ὅτι
οὐ μαλθακὰ τοῖσι σοφιστῇσίν εἰσι τὰ πρήγματα ἀλλὰ
πάντων ταλαιπωρότατα.
13.1.1
 θηρῶσι δὲ Ἰνδοὶ τὰ μὲν ἄλλα ἄγρια θηρία κατάπερ
καὶ Ἕλληνες, ἡ δὲ τῶν ἐλεφάντων σφιν θήρα οὐδέν τι
ἄλλῃ ἔοικεν, ὅτι καὶ ταῦτα τὰ θηρία οὐδαμοῖσιν ἄλλοισι
13.2.1
θηρίοις ἐπέοικεν. ἀλλὰ τόπον γὰρ ἐπιλεξάμενοι ἄπεδον
καὶ καυματώδεα ἐν κύκλῳ τάφρον ὀρύσσουσιν, ὅσον
μεγάλῳ στρατοπέδῳ ἐπαυλίσασθαι. τῆς δὲ τάφρου τὸ
εὖρος ἐς πέντε ὀργυιὰς ποιέονται, βάθος τε ἐς τέσσαρας.
13.3.1
τὸν δὲ χοῦν ὅντινα ἐκβάλλουσιν ἐκ τοῦ ὀρύγματος, ἐπὶ
τὰ χείλεα ἑκάτερα τῆς τάφρου ἐπιφορήσαντες ἀντὶ τεί-
13.4.1
χεος διαχρέονται, αὐτοὶ δὲ ὑπὸ τῷ χώματι τῷ ἐπὶ τοῦ
χείλεος τοῦ ἔξω τῆς τάφρου σκηνάς σφιν ὀρυκτὰς ποιέον-  
ται, καὶ διὰ τουτέων ὀπὰς ὑπολείπονται, δι' ὧν φῶς τε
αὐτοῖσιν εἰσέρχεται καὶ τὰ θηρία προσάγοντα καὶ ἐσε-
13.5.1
λαύνοντα ἐς τὸ ἕρκος σκέπτονται. ἐνταῦθα ἐντὸς τοῦ
ἕρκεος καταστήσαντες τῶν τινας θηλέων τρεῖς ἢ τές-
σαρας, ὅσαι μάλιστα τὸν θυμὸν χειροήθεες, μίαν εἴσο-
δον ἀπολιμπάνουσι κατὰ τὴν τάφρον, γεφυρώσαντες τὴν
13.5.5
τάφρον· καὶ ταύτῃ χοῦν τε καὶ πόαν πολλὴν ἐπιφέρουσι
τοῦ μὴ ἀρίδηλον εἶναι τοῖσι θηρίοισι τὴν γέφυραν, μή
13.6.1
τινα δόλον ὀισθῶσιν. αὐτοὶ μὲν οὖν ἐκποδὼν σφᾶς
<ποι>έουσι κατὰ τῶν σκηνέων τῶν ὑπὸ τῇ τάφρῳ δεδυ-
κότες, οἱ δὲ ἄγριοι ἐλέφαντες ἡμέρης μὲν οὐ πελάζουσι
τοῖσιν οἰκουμένοισι, νύκτωρ δὲ πλανῶνταί τε πάντη καὶ
13.6.5
ἀγεληδὸν νέμονται τῷ μεγίστῳ καὶ γενναιοτάτῳ σφῶν
13.7.1
ἑπόμενοι, κατάπερ αἱ βόες τοῖσι ταύροισιν. ἐπεὰν ὦν τῷ
ἕρκει πελάσωσι, τήν τε φωνὴν ἀκούοντες τῶν θηλέων
καὶ τῇ ὀδμῇ αἰσθόμενοι, δρόμῳ ἵενται ὡς ἐπὶ τὸν χῶρον
τὸν πεφραγμένον· ἐκπεριελθόντες δὲ τῆς τάφρου τὰ
13.7.5
χείλεα εὖτ' ἂν τῇ γεφύρῃ ἐπιτύχωσιν, κατὰ ταύτην ἐς
13.8.1
τὸ ἕρκος ὠθέονται. οἱ δὲ ἄνθρωποι αἰσθόμενοι τὴν ἔσο-
δον τῶν ἐλεφάντων τῶν ἀγρίων, οἳ μὲν αὐτῶν τὴν γέ-
φυραν ὀξέως ἀφεῖλον, οἳ δὲ ἐπὶ τὰς πέλας κώμας ἀπο-
δραμόντες ἀγγέλλουσι τοὺς ἐλέφαντας ὅτι ἐν τῷ ἕρκει
13.9.1
ἔχονται· οἳ δὲ ἀκούσαντες ἐπιβαίνουσι τῶν κρατίστων
τε τὸν θυμὸν καὶ [τῶν] χειροηθεστάτων ἐλεφάντων, ἐπι-
βάντες δὲ ἐλαύνουσιν ὡς ἐπὶ τὸ ἕρκος, ἐλάσαντες δὲ οὐκ
αὐτίκα μάχης ἅπτονται, ἀλλ' ἐῶσι γὰρ λιμῷ τε ταλαι-
13.9.5
πωρηθῆναι τοὺς ἀγρίους ἐλέφαντας καὶ ὑπὸ τῷ δίψει  
13.10.1
δουλωθῆναι. εὖτ' ἂν δέ σφισι κακῶς ἔχειν δοκέωσι,
τηνικαῦτα ἐπιστήσαντες αὖθις τὴν γέφυραν ἐλαύνουσί
τε ὡς ἐς τὸ ἕρκος, καὶ τὰ μὲν πρῶτα μάχη ἵσταται κρα-
τερὴ τοῖσιν ἡμέροισι τῶν ἐλεφάντων πρὸς τοὺς ἑαλω-
13.10.5
κότας· ἔπειτα κρατέονται μὲν κατὰ τὸ εἰκὸς οἱ ἄγριοι
13.11.1
ὑπό τε τῇ ἀθυμίῃ καὶ τῷ λιμῷ ταλαιπωρούμενοι. οἳ δὲ
ἀπὸ τῶν ἐλεφάντων καταβάντες παρειμένοισιν ἤδη τοῖ-
σιν ἀγρίοισι τοὺς πόδας ἄκρους συνδέουσιν, ἔπειτα
ἐγκελεύονται τοῖσιν ἡμέροισι πληγαῖς σφᾶς κολάζειν
13.11.5
πολλαῖς, ἔστ' ἂν ἐκεῖνοι ταλαιπωρεύμενοι ἐς γῆν πέ-
σωσι. παραστάντες δὲ βρόχους περιβάλλουσιν αὐτοῖσι
κατὰ τοὺς αὐχένας, καὶ αὐτοὶ ἐπιβαίνουσι κειμένοισι.
13.12.1
τοῦ δὲ μὴ ἀποσείεσθαι τοὺς ἀμβάτας μηδέ τι ἄλλο ἀτά-
σθαλον ἐργάζεσθαι, τὸν τράχηλον αὐτοῖσιν ἐν κύκλῳ
μαχαιρίῳ ὀξεῖ ἐπιτέμνουσι, καὶ τὸν βρόχον κατὰ τὴν
τομὴν περιδέουσιν, ὡς ἀτρέμα ἔχειν τὴν κεφαλήν τε καὶ
13.13.1
τὸν τράχηλον ὑπὸ τοῦ ἕλκεος. εἰ γὰρ περιστρέφοιντο 

ὑπὸ ἀτασθαλίης, τρίβεται αὐτοῖσι τὸ ἕλκος ὑπὸ τῷ κάλῳ.
οὕτω μὲν ὦν ἀτρέμα ἴσχουσι καὶ αὐτοὶ γνωσιμαχέοντες ἤδη
14.1.1
ἄγονται κατὰ τὸν δεσμὸν πρὸς τῶν ἡμέρων. ὅσοι δὲ
νήπιοι αὐτῶν ἢ διὰ κακότητα οὐκ ἄξιοι ἐκτῆσθαι, τού-
14.2.1
τους ἐῶσιν ἀπαλλάττεσθαι ἐς τὰ σφέτερα ἤθεα. ἀγ<αγ>όν-
τες δὲ εἰς τὰς κώμας τοὺς ἁλόντας τοῦ τε χλωροῦ καλά-
14.3.1
μου καὶ τῆς πόας τὰ πρῶτα ἐμφαγεῖν ἔδοσαν, οἳ δὲ ὑπὸ
ἀθυμίης οὐκ ἐθέλουσιν οὐδὲν σιτέεσθαι, τοὺς δὲ περιι-
στάμενοι οἱ Ἰνδοὶ ᾠδαῖσι τε καὶ τυμπάνοισι καὶ κυμβά-
λοισιν ἐν κύκλῳ κρούοντές τε καὶ ἐπᾴδοντες κατευνά-
14.4.1
ζουσι. θυμόσοφον γὰρ εἴπερ τι ἄλλο θηρίον ὁ ἐλέφας,  
καί τινες ἤδη αὐτῶν τοὺς ἀμβάτας σφῶν ἐν πολέμῳ
ἀποθανόντας ἄραντες αὐτοὶ ἐξήνεγκαν ἐς ταφήν, οἳ δὲ
καὶ ὑπερήσπισαν κειμένους, οἳ δὲ καὶ πεσόντων προε-
14.4.5
κινδύνευσαν, ὃ δέ τις πρὸς ὀργὴν ἀποκτείνας τὸν ἀμ-
14.5.1
βάτην ὑπὸ μετανοίης τε καὶ ἀθυμίης ἀπέθανεν. εἶδον
δὲ ἔγωγε καὶ κυμβαλίζοντα ἤδη ἐλέφαντα καὶ ἄλλους
ὀρχεομένους, κυμβάλοιν τῷ κυμβαλίζοντι πρὸς τοῖν
σκελοῖν τοῖν ἔμπροσθεν προσηρτημένοιν, καὶ πρὸς τῇ
14.6.1
προβοσκίδι καλεομένῃ ἄλλου κυμβάλου· ὃ δὲ ἐν μέρει
τῇ προβοσκίδι ἔκρουε τὸ κύμβαλον ἐν ῥυθμῷ πρὸς ἑκα-
τέροιν τοῖν σκελοῖν, οἳ δὲ ὀρχεόμενοι ἐν κύκλῳ τε ἐχό-
ρευον, καὶ ἐπαίροντές τε καὶ ἐπικάμπτοντες τὰ ἔμπρο-
14.6.5
σθεν σκέλεα ἐν τῷ μέρει ἐν ῥυθμῷ καὶ οὗτοι ἔβαινον,
14.7.1
καθότι ὁ κυμβαλίζων σφίσιν ὑφηγέετο. βαίνεται δὲ
ἐλέφας ἦρος ὥρῃ, κατάπερ βοῦς ἢ ἵππος, ἐπεὰν τῇσι
θηλέῃσιν αἱ παρὰ τοῖσι κροτάφοισιν ἀναπνοαὶ ἀνοιχθεῖ-
σαι ἐκπνέωσιν. κύει δὲ τοὺς ἐλαχίστους μὲν ἑκκαίδεκα
14.7.5
μῆνας, τοὺς πλείστους δὲ ὀκτωκαίδεκα. τίκτει δὲ ἕν,
κατάπερ ἵππος, καὶ τοῦτο ἐκτρέφει τῷ γάλακτι ἐς ἔτος
14.8.1
ὄγδοον. ζῶσι δὲ ἐλεφάντων οἱ πλεῖστα ἔτεα ζῶντες ἐς
διηκόσια, πολλοὶ δὲ νόσῳ προτελευτῶσιν αὐτῶν· γήρᾳ
14.9.1
δὲ ἐς τόσον ἔρχονται. καὶ ἔστιν αὐτοῖσι τῶν μὲν ὀφθαλ-
μῶν ἴαμα τὸ βόειον γάλα ἐγχεόμενον, πρὸς δὲ τὰς ἄλλας
νόσους ὁ μέλας οἶνος πινόμενος, ἐπὶ δὲ τοῖσιν ἕλκεσι  
τὰ ὕεια κρέα ὀπτώμενα καὶ καταπλασσόμενα· ταῦτα παρ'
14.9.5
Ἰνδοῖσίν ἐστιν αὐτοῖσι ἰάματα.
15.1.1
 τοῦ δὲ ἐλέφαντος τὴν τίγριν πολλόν τι ἀλκιμωτέρην
Ἰνδοὶ ἄγουσι. τίγριος δὲ δορὴν μὲν ἰδεῖν λέγει <Νέαρ-
χος>, αὐτὴν δὲ τίγριν οὐκ ἰδεῖν· ἀλλὰ τοὺς Ἰνδοὺς γὰρ
ἀπηγέεσθαι, τίγριν εἶναι μέγεθος μὲν ἡλίκον τὸν μέγι-
15.1.5
στον ἵππον, τὴν δὲ ὠκύτητα καὶ ἀλκὴν οἵην οὐδενὶ ἄλλῳ
15.2.1
εἰκάσαι· τίγριν γὰρ ἐπεὰν ὁμοῦ ἔλθῃ ἐλέφαντι, ἐπιπη-
δᾶν τε ἐπὶ τὴν κεφαλὴν τοῦ ἐλέφαντος καὶ ἄγχειν
15.3.1
εὐπετέως. ταύτας δέ, ἅστινας καὶ ἡμεῖς ὁρέομεν καὶ
τίγριας καλέομεν, θῶας εἶναι αἰόλους καὶ μέζονας ἤπερ
15.4.1
τοὺς ἄλλους θῶας. ἐπεὶ καὶ ὑπὲρ τῶν μυρμήκων λέγει
<Νέαρχος> μύρμηκα μὲν αὐτὸς οὐκ ἰδέειν, ὁποῖον δή τινα
μετεξέτεροι διέγραψαν γίνεσθαι ἐν τῇ Ἰνδῶν γῇ, δορὰς
δὲ καὶ τούτων ἰδεῖν πολλὰς ἐς τὸ στρατόπεδον κατακο-
15.5.1
μισθείσας τὸ Μακεδονικόν. <Μεγασθένης> δὲ καὶ ἀτρε-
κέα εἶναι ὑπὲρ τῶν μυρμήκων τὸν λόγον ἱστορέει τού-
τους εἶναι τοὺς τὸν χρυσὸν ὀρύσσοντας, οὐκ αὐτοῦ τοῦ
χρυσοῦ ἕνεκα, ἀλλὰ φύσι γὰρ κατὰ τῆς γῆς ὀρύσσουσιν,
15.5.5
ἵνα φωλεύσαιεν, κατάπερ οἱ ἡμέτεροι οἱ σμικροὶ μύρμηκες  
15.6.1
ὀλίγον τῆς γῆς ὀρύσσουσιν. ἐκείνους δέ – εἶναι γὰρ
ἀλωπεκέων μέζονας – πρὸς λόγον τοῦ μεγέθεος σφῶν
καὶ τὴν γῆν ὀρύσσειν· τὴν δὲ γῆν χρυσῖτιν εἶναι, καὶ
15.7.1
ἀπὸ ταύτης γίνεσθαι Ἰνδοῖσι τὸν χρυσόν. ἀλλὰ <Μεγα-
σθένης> τε ἀκοὴν ἀπηγέεται, καὶ ἐγὼ ὅτι οὐδὲν τούτου
ἀτρεκέστερον ἀναγράψαι ἔχω, ἀπίημι ἑκὼν τὸν ὑπὲρ
15.8.1
τῶν μυρμήκων λόγον. σιττακοὺς δὲ <Νέαρχος> μὲν ὡς
δή τι θαῦμα ἀπηγέεται ὅτι γίνονται ἐν τῇ Ἰνδῶν γῇ,
καὶ ὁποῖος ὄρνις ἐστὶν ὁ σιττακός, καὶ ὅπως φωνὴν ἵει
15.9.1
ἀνθρωπίνην. ἐγὼ δὲ ὅτι αὐτός τε πολλοὺς ὀπώπεα καὶ
ἄλλους ἐπισταμένους ᾔδεα τὸν ὄρνιθα, οὐδὲν ὡς <ὑπὲρ>
ἀτόπου δῆθεν ἀπηγήσομαι· οὐδὲ ὑπὲρ τῶν πιθήκων
τοῦ μεγέθεος, ἢ ὅτι καλοὶ παρ' Ἰνδοῖς πίθηκοί εἰσιν,
15.9.5
οὐδὲ ὅπως θηρῶνται ἐρέω. καὶ γὰρ ταῦτα γνώριμα ἐρῶ,
15.10.1
πλήν γε δὴ ὅτι καλοί που πίθηκοί εἰσιν. καὶ ὄφιας δὲ
λέγει <Νέαρχος> θηρευθῆναι αἰόλους μὲν καὶ ταχέας,
μέγαθος δέ, ὃν μὲν λέγει ἑλεῖν Πείθωνα τὸν Ἀντιγέ-
νεος, πήχεων ὡς ἑκκαίδεκα. αὐτοὺς δὲ τοὺς Ἰνδοὺς πολὺ
15.10.5
μείζονας τούτων λέγειν εἶναι τοὺς μεγίστους ὄφεας.
15.11.1
ὅσοι δὲ ἰητροὶ Ἕλληνες, τούτοισιν οὐδὲν ἄκος ἐξεύρητο  
ὅστις ὑπὸ ὄφεως δηχθείη Ἰνδικοῦ· ἀλλ' αὐτοὶ γὰρ οἱ
Ἰνδοὶ ἰῶντο τοὺς πληγέντας. καὶ ἐπὶ τῷδε <Νέαρχος> λέγει
<ὅτι> συλλελεγμένους ἀμφ' αὑτὸν εἶχεν Ἀλέξανδρος
15.11.5
Ἰνδῶν ὅσοι ἰητρικὴν σοφώτατοι, καὶ κεκήρυκτο ἀνὰ τὸ
στρατόπεδον, ὅστις δηχθείη, ἐπὶ τὴν σκηνὴν φοιτᾶν τὴν
15.12.1
βασιλέως. οἱ δὲ αὐτοὶ οὗτοι καὶ τῶν ἄλλων νούσων τε
καὶ παθέων ἰητροὶ ἦσαν. οὐ πολλὰ δὲ ἐν Ἰνδοῖσι πάθεα
γίνεται, ὅτι αἱ ὧραι σύμμετροί εἰσιν αὐτόθι· εἰ δέ τι
μεῖζον καταλαμβάνοι, τοῖσι σοφιστῇσιν ἀνεκοινοῦντο·
15.12.5
καὶ ἐκεῖνοι οὐκ ἄνευ θεοῦ ἐδόκεον ἰῆσθαι ὅ τι περ
ἰήσιμον.

16.1.1
 ἐσθῆτι δὲ Ἰνδοὶ λινέῃ χρέονται, κατάπερ λέγει <Νέ-
αρχος>, λίνου τοῦ ἀπὸ τῶν δενδρέων, ὑπὲρ ὅτων μοι
ἤδη λέλεκται. τὸ δὲ λίνον τοῦτο ἢ λαμπρότερον τὴν
χροιήν ἐστιν ἄλλου λίνου παντός, ἢ μέλανες αὐτοὶ ἐόν-
16.2.1
τες λαμπρότερον τὸ λίνον φαίνεσθαι ποιέουσιν. ἔστι δὲ
κιθὼν λίνεος αὐτοῖς ἔστε ἐπὶ μέσην τὴν κνήμην, εἷμα
δὲ τὸ μὲν περὶ τοῖσιν ὤμοισι περιβεβλημένον, τὸ δὲ περὶ
16.3.1
τῇσι κεφαλῇσιν εἰλιγμένον. καὶ ἐνώτια Ἰνδοὶ φορέου-
σιν ἐλέφαντος ὅσοι κάρτα εὐδαίμονες· οὐ γὰρ πάντες
16.4.1
Ἰνδοὶ φορέουσι. τοὺς δὲ πώγωνας λέγει <Νέαρχος> ὅτι
βάπτονται Ἰνδοί, χροιὴν δὲ ἄλλην καὶ ἄλλην <βάπτον-  
ται>, οἳ μὲν ὡς λευκοὺς φαίνεσθαι οἵους λευκοτάτους,
οἳ δὲ κυανέους, τοῖς δὲ φοινικέους εἶναι, τοῖς δὲ καὶ
16.5.1
πορφυρέους, ἄλλοις πρασοειδέας· καὶ σκιάδια ὅτι προ-
βάλλονται τοῦ θέρεος ὅσοι οὐκ ἠμελημένοι Ἰνδῶν. ὑπο-
δήματα δὲ λευκοῦ δέρματος φορέουσι, περιττῶς καὶ
ταῦτα ἠσκημένα· καὶ τὰ ἴχνη τῶν ὑποδημάτων αὐτοῖσι
16.5.5
ποικίλα καὶ ὑψηλά, τοῦ μέζονας φαίνεσθαι.
16.6.1
 ὁπλίσιος δὲ τῆς Ἰνδῶν οὐκ ὡυτὸς εἷς τρόπος ἀλλ'
οἱ μὲν πεζοὶ αὐτοῖσι τόξον τε ἔχουσι, ἰσόμηκες τῷ φο-
ρέοντι τὸ τόξον, καὶ τοῦτο κάτω ἐπὶ τὴν γῆν θέντες καὶ
τῷ ποδὶ τῷ ἀριστερῷ ἀντιβάντες, οὕτως ἐκτοξεύουσι,
16.7.1
τὴν νευρὴν ἐπὶ μέγα ὀπίσω ἀπαγαγόντες· ὁ γὰρ ὀιστὸς
αὐτοῖσιν ὀλίγον ἀποδέων τριπήχεος, οὐδέ τι ἀντέχει
τοξευθὲν πρὸς Ἰνδοῦ ἀνδρὸς τοξικοῦ, οὔτε ἀσπὶς οὔτε
16.8.1
θώρηξ οὔτε <εἴ> τι <τὸ κάρτα> καρτερὸν ἐγένετο. ἐν δὲ
τῇσιν ἀριστερῇσι πέλται εἰσὶν αὐτοῖσιν ὠμοβόιναι, στει-
νότεραι μὲν ἢ κατὰ τοὺς φορέοντας, μήκει δὲ οὐ πολλὸν
16.9.1
ἀποδέουσαι. τοῖσι δὲ ἄκοντες ἀντὶ τόξων εἰσί. μάχαιραν
δὲ πάντες φορέουσι, πλατείην δὲ καὶ τὸ μῆκος οὐ μείω
τριπήχεος· καὶ ταύτην, ἐπεὰν συστάδην καταστῇ αὐτοῖ-
σιν ἡ μάχη – τὸ δὲ οὐκ εὐμαρέως Ἰνδοῖσιν ἐς ἀλλή-
16.9.5
λους γίνεται – ἀμφοῖν τοῖν χεροῖν καταφέρουσιν ἐς
16.10.1
τὴν πληγήν, τοῦ καρτερὴν τὴν πληγὴν γενέσθαι. οἱ δὲ
ἱππέες ἀκόντια δύο αὐτοῖσιν ἔχουσιν, οἷα τὰ σαύνια
ἀκόντια, καὶ πέλτην [τὴν] μικροτέρην τῶν πεζῶν. οἱ  
δὲ ἵπποι αὐτοῖσιν οὐ σεσαγμένοι εἰσίν, οὐδὲ χαλινοῦνται
16.10.5
τοῖσιν Ἑλληνικοῖσι χαλινοῖσιν ἢ τοῖσι Κελτικοῖσιν ἐμ-
16.11.1
φερέως, ἀλλὰ περὶ ἄκρῳ τῷ στόματι τοῦ ἵππου ἐν κύκλῳ
ἔχουσι δέρμα ὠμοβόινον ῥαπτὸν περιηρτημένον, καὶ ἐν
τούτῳ χάλκεα κέντρα ἢ σιδήρεα, οὐ κάρτα ὀξέα, ἔσω
ἐστραμμένα· τοῖσι δὲ πλουσίοισιν ἐλεφάντινα κέντρα
16.11.5
ἐστίν. ἐν δὲ τῷ στόματι σίδηρον αὐτοῖσιν οἱ ἵπποι ἔχου-
σιν, οἷόν περ ὀβελόν, ἔνθεν ἐξηρτημένοι εἰσὶν αὐτοῖσιν
16.12.1
οἱ ῥυτῆρες· ἐπεὰν ὦν ἐπαγάγωσι τὸν ῥυτῆρα, ὅ τε ὀβελὸς 


κρατέει τὸν ἵππον, καὶ τὰ κέντρα, οἷα δὴ ἐξ αὐτοῦ ἠρτη-
μένα, κεντέοντα οὐκ ἐᾷ ἄλλο τι ἢ πείθεσθαι τῷ ῥυτῆρι.
17.1.1
 τὰ δὲ σώματα ἰσχνοί τέ εἰσιν Ἰνδοὶ καὶ εὐμήκεες, καὶ
κοῦφοι πολλόν τι ὑπὲρ τοὺς ἄλλους ἀνθρώπους. ὀχή-
ματα δὲ τοῖς μὲν πολλοῖς Ἰνδῶν κάμηλοί εἰσιν καὶ ἵπποι
17.2.1
καὶ ὄνοι, τοῖς δὲ εὐδαίμοσιν ἐλέφαντες. βασιλικὸν γὰρ
ὄχημα ἐλέφας παρ' Ἰνδοῖς ἐστι, δεύτερον δὲ τιμῇ ἐπὶ
τούτῳ τὰ τέθριππα, τρίτον δὲ αἱ κάμηλοι. τὸ δὲ ἐφ'
17.3.1
ἑνὸς ἵππου ὀχέεσθαι ἄτιμον. αἱ γυναῖκες δὲ αὐτοῖσιν,
ὅσαι κάρτα σώφρονες, ἐπὶ μὲν ἄλλῳ μισθῷ οὐκ ἄν τι
διαμάρτοιεν, ἐλέφαντα δὲ λαβοῦσα γυνὴ μίσγεται τῷ
δόντι· οὐδὲ αἰσχρὸν Ἰνδοὶ ἄγουσι τὸ ἐπὶ ἐλέφαντι μι-
17.3.5
γῆναι, ἀλλὰ καὶ σεμνὸν δοκέει τῇσι γυναιξὶν ἀξίην τὸ
17.4.1
κάλλος φανῆναι ἐλέφαντος. γαμέουσι δὲ οὔτε τι διδόντες
οὔτε λαμβάνοντες, ἀλλὰ ὅσαι ἤδη ὡραῖαι γάμου, ταύτας
οἱ πατέρες προάγοντες ἐς τὸ ἐμφανὲς καθιστᾶσιν ἐκλέ-
ξασθαι τῷ νικήσαντι πάλην ἢ πὺξ ἢ δρόμον ἢ κατ'
17.5.1
ἄλλην τινὰ ἀνδρείαν προκριθέντι. σιτοφάγοι δὲ καὶ  
ἀροτῆρες Ἰνδοί εἰσιν, ὅσοι γε μὴ ὄρειοι αὐτῶν· οὗτοι
δὲ τὰ θήρεια κρέα σιτέονται.
17.6.1
 ταῦτά μοι ἀπόχρη δεδηλῶσθαι ὑπὲρ Ἰνδῶν, ὅσα γνω-
ριμώτατα <Νέαρχός> τε καὶ <Μεγασθένης>, δοκίμω ἄνδρε,
17.7.1
ἀνεγραψάτην, ἐπεὶ οὐδὲ ἡ ὑπόθεσίς μοι τῆσδε τῆς συγ-
γραφῆς τὰ Ἰνδῶν νόμιμα ἀναγράψαι ἦν, ἀλλ' ὅπως γὰρ
παρεκομίσθη Ἀλεξάνδρῳ ἐς Πέρσας ἐξ Ἰνδῶν ὁ στόλος·
ταῦτα δὲ ἐκβολή μοι ἔστω τοῦ λόγου.
18.1.1
 Ἀλέξανδρος γὰρ, ἐπειδὴ οἱ παρεσκεύαστο τὸ ναυτι-
κὸν ἐπὶ τοῦ Ὑδάσπεω τῇσιν ὄχθῃσιν, ἐπιλεγόμενος ὅσοι
τε Φοινίκων καὶ ὅσοι Κύπριοι ἢ Αἰγύπτιοι εἵποντο ἐν
τῇ ἄνω στρατηλασίῃ, ἐκ τούτων ἐπλήρου τὰς νέας, ὑπη-
18.1.5
ρεσίας τε αὐτῇσι καὶ ἐρέτας ἐπιλεγόμενος ὅσοι τῶν θα-
18.2.1
λασσίων ἔργων δαήμονες. ἦσαν δὲ καὶ νησιῶται ἄνδρες
οὐκ ὀλίγοι ἐν τῇ στρατιῇ οἷς ταῦτα ἔμελε, καὶ Ἴωνες
18.3.1
καὶ Ἑλλησπόντιοι. τριήραρχοι δὲ αὐτῷ ἐπεστάθησαν ἐκ
Μακεδόνων μὲν Ἡφαιστίων τε Ἀμύντορος καὶ Λεόννατος
ὁ Εὔνου καὶ Λυσίμαχος ὁ Ἀγαθοκλέους καὶ Ἀσκληπιό-
δωρος ὁ Τιμάνδρου καὶ Ἄρχων ὁ Κλεινίου καὶ Δημό-
18.3.5
νικος ὁ Ἀθηναίου καὶ Ἀρχίας ὁ Ἀναξιδότου καὶ Ὀφέλλας
Σειληνοῦ καὶ Τιμάνθης Παντιάδου. οὗτοι μὲν Πελ-
18.4.1
λαῖοι· ἐκ δὲ Ἀμφιπόλεως ἦγον οἵδε· [ἐκ Κρήτης] Νέαρ-
χος Ἀνδροτίμου, ὃς τὰ ἀμφὶ τῷ παράπλῳ ἀνέγραψε, καὶ
Λαομέδων Λαρίχου, καὶ Ἀνδροσθένης Καλλιστράτου·
18.5.1
ἐκ δὲ Ὀρεστίδος Κράτερός τε ὁ Ἀλεξάνδρου καὶ Περ-  
δίκκας ὁ Ὀρόντεω· Ἐορδαῖοι δὲ Πτολεμαῖός τε ὁ Λάγου
καὶ Ἀριστόνους ὁ Πεισαίου. ἐκ Πύδνης δὲ Μήτρων τε ὁ
18.6.1
Ἐπιχάρμου καὶ Νικαρχίδης ὁ Σίμου. ἐπὶ δὲ Ἄτταλός τε
ὁ Ἀνδρομένεος Τυμφαῖος καὶ Πευκέστας Ἀλεξάνδρου
Μιεζεὺς καὶ Πείθων Κρατεύα Ἀλκομενεὺς καὶ Λεόν-
νατος Ἀντιπάτρου Αἰγαῖος καὶ Πάνταυχος Νικολάου
18.6.5
Ἀλωρίτης καὶ Μυλλέας Ζωΐλου Βεροιαῖος. οὗτοι μὲν οἱ
18.7.1
σύμπαντες Μακεδόνες· Ἑλλήνων δὲ Μήδιος μὲν Ὀξυ-
θέμιδος Λαρισαῖος, Εὐμένης δὲ Ἱερωνύμου ἐκ Καρδίης,
Κριτόβουλος δὲ Πλάτωνος Κῷος, καὶ Θόας Μηνοδώρου
18.8.1
καὶ Μαίανδρος Μανδρογένεος Μάγνητες, Ἄνδρων δὲ
Καβήλεω Τήιος. Κυπρίων δὲ Νικοκλέης Πασικράτεος
Σόλιος καὶ Νιθάφων Πνυταγόρεω Σαλαμίνιος. ἦν δὲ
δὴ καὶ Πέρσης αὐτῷ τριήραρχος, Βαγώας ὁ Φαρνούχεος.
18.9.1
τῆς δὲ αὐτοῦ Ἀλεξάνδρου νεὼς κυβερνήτης ἦν Ὀνησί-  
κριτος Ἀστυπαλαιεύς, γραμματεὺς δὲ τοῦ στόλου παντὸς
18.10.1
Εὐαγόρας Εὐκλέωνος Κορίνθιος. ναύαρχος δὲ αὐτοῖσιν
ἐπεστάθη Νέαρχος Ἀνδροτίμου, τὸ γένος μὲν Κρὴς ὁ
18.11.1
Νέαρχος, ᾤκει δὲ ἐν Ἀμφιπόλει τῇ ἐπὶ Στρυμόνι. ὡς
δὲ ταῦτα ἐκεκόσμητο Ἀλεξάνδρῳ, ἔθυε τοῖς θεοῖσιν ὅσοι
τε πάτριοι ἢ μαντευτοὶ αὐτῷ καὶ Ποσειδῶνι καὶ Ἀμφι-
τρίτῃ καὶ Νηρηίσι καὶ αὐτῷ τῷ Ὠκεανῷ, καὶ τῷ Ὑδάσπῃ
18.11.5
ποταμῷ, ἀπ' ὅτου ὡρμᾶτο, καὶ τῷ Ἀκεσίνῃ, ἐς ὅντινα
ἐκδιδοῖ ὁ Ὑδάσπης, καὶ τῷ Ἰνδῷ, ἐς ὅντινα ἄμφω ἐκδι-
18.12.1
δοῦσιν· ἀγῶνές τε αὐτῷ μουσικοὶ καὶ γυμνικοὶ ἐποιεῦντο,
καὶ ἱερεῖα τῇ στρατιῇ πάσῃ κατὰ τέλεα ἐδίδοτο.
19.1.1
 ὡς δὲ πάντα ἐξήρτυτο αὐτῷ ἐς ἀναγωγήν, Κράτερον
μὲν τὰ ἐπὶ θάτερα τοῦ Ὑδάσπεω ἰέναι σὺν στρατιῇ [πεζῇ]
ἐκέλευσε πεζικῇ τε καὶ ἱππικῇ· ἐς τὸ ἐπὶ θάτερα <δὲ>
Ἡφαιστίων αὐτῷ παρεπορεύετο σὺν ἄλλῃ στρατιῇ πλεί-
19.1.5
ονι ἔτι τῆς Κρατέρῳ συντεταγμένης. καὶ τοὺς ἐλέφαν-
τας Ἡφαιστίων αὐτῷ ἦγεν, ὄντας ἐς διακοσίους. αὐτὸς
19.2.1
δὲ τούς τε ὑπασπιστὰς καλεομένους ἅμα οἷ ἦγε καὶ τοὺς
τοξότας πάντας καὶ τῶν ἱππέων τοὺς ἑταίρους καλεομέ-
19.3.1
νους, τοὺς πάντας ἐς ὀκτακισχιλίους. τοῖσι μὲν δὴ ἀμφὶ
Κράτερον καὶ Ἡφαιστίωνα ἐτέτακτο ἵνα προπορευθέντες
19.4.1
ὑπομένοιεν τὸν στόλον. Φίλιππον δέ, ὃς αὐτῷ σατράπης
τῆς χώρης ταύτης ἦν, ἐπὶ τοῦ Ἀκεσίνου ποταμοῦ τὰς
19.5.1
ὄχθας πέμπει, ἅμα στρατιῇ πολλῇ καὶ τοῦτον· ἤδη γὰρ
καὶ δώδεκα μυριάδες αὐτῷ μάχιμοι εἵποντο σὺν οἷς ἀπὸ
θαλάσσης τε αὐτὸς ἀνήγαγε καὶ αὖθις οἱ ἐπὶ συλλογὴν
αὐτῷ στρατιᾶς πεμφθέντες ἧκον ἔχοντες, παντοῖα ἔθνεα  
19.5.5
βαρβαρικὰ ἅμα οἷ ἄγοντι καὶ πᾶσαν ἰδέην ὡπλισμένα.
19.6.1
αὐτὸς δὲ ἄρας ταῖς ναυσὶ κατέπλει κατὰ τὸν Ὑδάσπεα
ἔστε ἐπὶ τοῦ Ἀκεσίνου τε καὶ τοῦ Ὑδάσπεω τὰς συμβο-
19.7.1
λάς. νῆες δὲ αἱ σύμπασαι αὐτῷ ὀκτακόσιαι ἦσαν, αἵ τε
μακραὶ καὶ ὅσα στρογγύλα πλοῖα καὶ ἄλλαι ἱππαγωγοὶ 

καὶ σιτία ἅμα τῇ στρατιῇ ἄγουσαι. ὅπως μὲν δὴ κατὰ
τοὺς ποταμοὺς κατέπλευσεν αὐτῷ ὁ στόλος, καὶ ὅσα ἐν
τῷ παράπλῳ ἔθνεα κατεστρέψατο, καὶ ὅπως διὰ κινδύνου
αὐτὸς ἐν Μαλλοῖς ἧκε, καὶ τὸ τρῶμα ὃ ἐτρώθη ἐν Μαλ-
19.8.5
λοῖς, καὶ Πευκέστας τε καὶ Λεόννατος ὅπως ὑπερήσπισαν
αὐτὸν πεσόντα, πάντα ταῦτα λέλεκταί μοι ἤδη ἐν τῇ
19.9.1
ἄλλῃ τῇ Ἀττικῇ ξυγγραφῇ. ὁ δὲ λόγος ὅδε τοῦ παρά-
πλου μοι ἀφήγησίς ἐστιν, ὃν Νέαρχος σὺν τῷ στόλῳ
παρέπλευσεν ἀπὸ τοῦ Ἰνδοῦ τῶν ἐκβολέων ὁρμηθεὶς
κατὰ τὴν θάλασσαν τὴν μεγάλην ἔστε ἐπὶ τὸν κόλπον
19.9.5
τὸν Περσικόν, ἣν δὴ Ἐρυθρὴν θάλασσαν μετεξέτεροι
καλέουσι.
20.1.1
 <Νεάρχῳ> δὲ λέλεκται ὑπὲρ τούτων ὅδε ὁ λόγος. πό-
θον μὲν εἶναι Ἀλεξάνδρῳ ἐκπεριπλῶσαι τὴν θάλασσαν
20.2.1
τὴν ἀπὸ Ἰνδῶν ἔστε ἐπὶ τὴν Περσικήν, ὀκνέειν δὲ αὐτὸν
τοῦ τε πλόου τὸ μῆκος καὶ μή τινι ἄρα χώρῃ ἐρήμῳ
ἐγκύρσαντες ἢ ὅρμων ἀπόρῳ ἢ οὐ ξυμμέτρως ἐχούσῃ
τῶν ὡραίων, οὕτω δὴ διαφθαρῇ αὐτῷ ὁ στόλος, καὶ οὐ
20.2.5
φαύλη κηλὶς αὕτη τοῖς ἔργοισιν αὐτοῦ τοῖσι μεγάλοισιν  
ἐπιγενομένη τὴν πᾶσαν εὐτυχίην αὐτῷ ἀφανίσῃ· ἀλλὰ
ἐκνικῆσαι γὰρ αὐτῷ τὴν ἐπιθυμίην τοῦ καινόν τι αἰεὶ
20.3.1
καὶ ἄτοπον ἐργάζεσθαι. ἀπόρως δὲ ἔχειν ὅντινα οὐκ
ἀδύνατόν τε ἐς τὰ ἐπινοούμενα ἐπιλέξαιτο καὶ ἅμα τῶν
ἐν νηὶ ἀνδρῶν, ὡς καὶ [τῶν] τοιοῦτον στόλον στελλο-
μένων, ἀφελεῖν τὸ δεῖμα τοῦ δὴ ἠμελημένως αὐτοὺς ἐς
20.4.1
προῦπτον κίνδυνον ἐκπέμπεσθαι. λέγει δὴ ὁ <Νέαρχος>
ἑωυτῷ ξυνοῦσθαι τὸν Ἀλέξανδρον ὅντινα προχειρίσηται
ἐξηγέεσθαι τοῦ στόλου. ὡς δὲ ἄλλου καὶ ἄλλου ἐς μνή-
μην ἰόντα τοὺς μὲν ὡς οὐκ ἐθέλοντας κινδυνεύειν ὑπὲρ
20.4.5
οὗ ἀπολέγειν, τοὺς δὲ ὡς μαλακοὺς τὸν θυμόν, τοὺς δὲ
ὡς πόθῳ τῆς οἰκηίης κατεχομένους, τοῖς δὲ ἄλλο καὶ
20.5.1
ἄλλο ἐπικαλέοντα, τότε δὴ αὐτὸν ὑποστάντα εἰπεῖν ὅτι
“ὦ βασιλεῦ, ἐγώ τοι ὑποδέκομαι ἐξηγήσεσθαι τοῦ στό-
λου, καὶ εἰ τὰ ἀπὸ τοῦ θείου ξυνεπιλαμβάνοι, περιάξω
τοι σῴας τὰς νέας καὶ τοὺς ἀνθρώπους ἔστε ἐπὶ τὴν
20.5.5
Περσίδα γῆν, εἰ δὴ πλωτός τέ ἐστιν ὁ ταύτῃ πόντος καὶ
20.6.1
τὸ ἔργον οὐκ ἄπορον γνώμῃ ἀνθρωπηίῃ.” Ἀλέξανδρον
δὲ λόγῳ μὲν οὐ φάναι ἐθέλειν ἐς τοσήνδε ταλαιπωρίην
καὶ τοσόνδε κίνδυνον τῶν τινα ἑαυτοῦ φίλων ἐμβάλλειν,
αὐτὸν δὲ ταύτῃ δὴ καὶ μᾶλλον οὐκ ἀνιέναι ἀλλὰ λιπαρεῖν.
20.7.1
οὕτω δὴ ἀγαπῆσαί τε Ἀλέξανδρον τοῦ Νεάρχου τὴν
προθυμίην, καὶ ἐπιστῆσαι αὐτὸν ἄρχειν τοῦ στόλου
20.8.1
παντός. καὶ τότε δὴ ἔτι μᾶλλον τῆς στρατιῆς ὅ τι περ
ἐπὶ τῷ παράπλῳ τῷδε ἐτάσσετο καὶ τὰς ὑπηρεσίας ἵλεω  
ἔχειν τὴν γνώμην, ὅτι δὴ Νέαρχόν γε οὔποτε ἂν Ἀλέ-
ξανδρος προήκατο ἐς κίνδυνον καταφανέα, εἰ μή σφι
20.9.1
σωθήσεσθαι ἔμελλεν. λαμπρότης τε πολλὴ τῇ παρασκευῇ
ἐποῦσα καὶ κόσμος τῶν νεῶν καὶ σπουδαὶ τῶν τριηράρ-
χων ἀμφὶ τὰς ὑπηρεσίας τε καὶ τὰ πληρώματα ἐκπρε-
πέες καὶ τοὺς πάντ' ἤδη πάλαι κατοκνέοντας ἐς ῥώμην
20.9.5
ἅμα καὶ ἐλπίδας χρηστοτέρας ὑπὲρ τοῦ παντὸς ἔργου
20.10.1
ἐπηρκότα ἦν. πολὺ δὲ δὴ συνεπιλαβέσθαι ἐς εὐθυμίην
τῇ στρατιῇ τὸ δὴ αὐτὸν Ἀλέξανδρον ὁρμηθέντα κατὰ
τοῦ Ἰνδοῦ τὰ στόματα ἀμφότερα ἐκπλῶσαι ἐς τὸν πόν-
τον σφάγιά τε τῷ Ποσειδῶνι ἐντεμεῖν καὶ ὅσοι ἄλλοι
20.10.5
θεοὶ θαλάσσιοι, καὶ δῶρα μεγαλοπρεπέα τῇ θαλάσσῃ
20.11.1
χαρίσασθαι. τῇ τε ἄλλῃ τῇ Ἀλεξάνδρου παραλόγῳ εὐτυ-
χίῃ πεποιθότας οὐδὲν ὅ τι οὐ τολμητόν τε ἐκείνῳ καὶ
ἐρκτὸν ἡγέεσθαι.
21.1.1
 ὡς δὲ τὰ ἐτήσια πνεύματα ἐκοιμήθη, ἃ δὴ τοῦ θέρεος
τὴν ὥρην πᾶσαν κατέχει ἐκ τοῦ πελάγεος ἐπιπνέοντα
ἐπὶ τὴν γῆν καὶ ταύτῃ ἄπορον τὸν πλοῦν ποιέοντα, τότε
δὴ ὡρμῶντο ἐπὶ ἄρχοντος Ἀθήνησι Κηφισοδώρου, εἰκάδι  
21.1.5
τοῦ Βοηδρομιῶνος μηνός, καθότι Ἀθηναῖοι ἄγουσιν,
ὡς δὲ Μακεδόνες τε καὶ Ἀσιανοὶ ἦγον <μηνὸς Ὑπερ-
βερεταίου, ἔτος> τὸ ἑνδέκατον βασιλεύοντος Ἀλεξάνδρου.
21.2.1
θύει δὲ καὶ Νέαρχος πρὸ τῆς ἀναγωγῆς Διὶ Σωτῆρι καὶ
ἀγῶνα ποιέει καὶ οὗτος γυμνικόν. ἄραντες δὲ ἀπὸ τοῦ
ναυστάθμου τῇ πρώτῃ ἡμέρῃ κατὰ τὸν Ἰνδὸν ποταμὸν
ὁρμίζονται πρὸς διώρυχι μεγάλῃ, καὶ μένουσιν αὐτοῦ
21.2.5
δύο ἡμέρας· Στοῦρα δὲ ὄνομα ἦν τῷ χώρῳ· στάδιοι ἀπὸ
21.3.1
τοῦ ναυστάθμου ἐς ἑκατόν. τῇ τρίτῃ δὲ ἄραντες ἔπλεον
ἔστε ἐπὶ διώρυχα ἄλλην σταδίους τριάκοντα, ἁλμυρὴν
ἤδη ταύτην τὴν διώρυχα· ἀνῄει γὰρ ἡ θάλασσα ἐς
αὐτήν, μάλιστα μὲν ἐν τῇσι πλημμυρίῃσιν, ὑπέμενε δὲ
21.3.5
καὶ ἐν τῇ<σιν> ἀμπώτεσι τὸ ὕδωρ μεμιγμένον τῷ πο-
21.4.1
ταμῷ· Καύμανα δὲ οὔνομα ἦν τῷ χώρῳ. ἐνθένδε εἴκοσι
σταδίους καταπλώσαντες ἐς Κορέεστιν ὁρμίζονται ἔτι
21.5.1
κατὰ τὸν ποταμόν. ἐνθένδε ὁρμηθέντες ἔπλεον οὐκ ἐπὶ
πολλόν· ἕρμα γὰρ ἐφάνη αὐτοῖσι κατὰ τὴν ἐκβολὴν τὴν
ταύτῃ τοῦ Ἰνδοῦ καὶ τὰ κύματα ἐρρόχθει πρὸς τῇ ἠιόνι,
21.6.1
καὶ ἡ ἠιὼν αὕτη τραχεῖα ἦν. ἀλλὰ ἵναπερ μαλθακὸν ἦν
τοῦ ἕρματος, ταύτῃ διώρυχα ποιήσαντες ἐπὶ σταδίους
πέντε, διῆγον τὰς νέας, ἐπειδὴ ἡ πλήμμυρα ἐπῆλθεν ἡ
21.7.1
ἐκ τοῦ πόντου. ἐκπεριπλώσαντες δὲ σταδίους πεντήκοντα
καὶ ἑκατὸν ὁρμίζονται ἐς Κρώκαλα νῆσον ἀμμώδεα καὶ
μένουσιν αὐτοῦ τὴν ἄλλην ἡμέραν. προσοικέει δὲ ταύτῃ  
21.8.1
ἔθνος Ἰνδικόν, οἱ Ἀράβιες καλεόμενοι, ὧν καὶ ἐν τῇ
μέζονι ξυγγραφῇ μνήμην ἔσχον, καὶ ὅτι εἰσὶν ἐπώνυμοι
ποταμοῦ Ἀράβιος, ὃς διὰ τῆς γῆς αὐτῶν ῥέων ἐκδιδοῖ
ἐς θάλασσαν, ὁρίζων τούτων τε τὴν χώρην καὶ τὴν 


Ὠρειτέων. ἐκ δὲ Κρωκάλων ἐν δεξιᾷ μὲν ἔχοντες ὄρος
τὸ καλεόμενον αὐτοῖσιν Εἶρον, ἐν ἀριστερᾷ δὲ νῆσον
ἁλιτενέα ἔπλεον· ἡ δὲ νῆσος παρατεταμένη τῇ ἠιόνι
21.10.1
κόλπον στεινὸν ποιέει. διεκπλεύσαντες δὲ ταύτην ὁρμί-
ζονται ἐν λιμένι εὐόρμῳ. ὅτι δὲ μέγας τε καὶ καλὸς ὁ
λιμήν, Νεάρχῳ ἔδοξεν ἐπονομάζειν αὐτὸν Ἀλεξάνδρου
21.11.1
λιμένα. νῆσος δέ ἐστιν ἐπὶ τῷ στόματι τοῦ λιμένος ὅσον
σταδίους δύο ἀπέχουσα· Βίβακτα ὄνομα τῇ νήσῳ, ὁ δὲ
χῶρος ἅπας Σάγγαδα. ἡ δὲ νῆσος καὶ τὸν λιμένα, προ-
21.12.1
κειμένη πρὸ τοῦ πελάγεος, αὐτὴ ἐποίεεν. ἐνταῦθα
πνεύματα μεγάλα ἐκ τοῦ πόντου ἔπνεε καὶ συνεχέα, καὶ
Νέαρχος δείσας τῶν βαρβάρων μή τινες συνταχθέντες
ἐφ' ἁρπαγὴν τοῦ στρατοπέδου τραποίατο, ἐκτειχίζει τὸν
21.13.1
χῶρον λιθίνῳ τείχει. τέσσαρες δὲ καὶ εἴκοσιν ἡμέραι τῇ
μονῇ ἐγένοντο. καὶ λέγει ὅτι μύας τε ἐθήρων τοὺς θα-
λασσίους οἱ στρατιῶται, καὶ ὄστρεια δὲ καὶ τοὺς σωλῆ-
νας καλεομένους, ἄτοπα τὸ μέγεθος, ὡς τοῖσιν ἐν τῇδε  
21.13.5
τῇ ἡμετέρῃ θαλάσσῃ συμβαλέειν· καὶ ὕδωρ ὅτι ἁλμυρὸν
ἐπίνετο.
22.1.1
 ἅμα τε ὁ ἄνεμος ἐπαύσατο καὶ οἳ ἀνήγοντο. καὶ πε-
ραιωθέντες σταδίους ἐς ἑξήκοντα ὁρμίζονται πρὸς αἰ-
γιαλῷ ψαμμώδεϊ· νῆσος δὲ ἐπῆν τῷ αἰγιαλῷ ἐρήμη.
22.2.1
ταύτην δὲ πρόβλημα ποιησάμενοι ὡρμίσθησαν· Δόμαι
οὔνομα τῇ νήσῳ. ὕδωρ δὲ οὐκ ἦν ἐν τῷ αἰγιαλῷ, ἀλλὰ
προελθόντες ἐς τὴν μεσογαίην ὅσον εἴκοσι σταδίους ἐπι-
22.3.1
τυγχάνουσιν ὕδατι καλῷ. τῇ δὲ ὑστεραίῃ ἐς νύκτα αὐ-
τοῖς ὁ πλόος ἐγίνετο ἐς Σάραγγα σταδίους τριακοσίους,
καὶ ὁρμίζονται πρὸς αἰγιαλῷ, καὶ ὕδωρ ἦν ἀπὸ τοῦ
22.4.1
αἰγιαλοῦ ὅσον ὀκτὼ σταδίους. ἐνθένδε πλώσαντες ὁρ-
μίζονται ἐν Σακάλοισι, τόπῳ ἐρήμῳ. καὶ διεκπλώσαντες
σκοπέλους δύο, οὕτω τι ἀλλήλοις πελάζοντας, ὥστε τοὺς
ταρσοὺς τῶν νεῶν ἅπτεσθαι ἔνθεν καὶ ἔνθεν τῶν πε-
22.4.5
τρέων, καθορμίζονται ἐν Μοροντοβάροις, σταδίους διελ-
22.5.1
θόντες ἐς τριακοσίους· ὁ δὲ λιμὴν μέγας καὶ εὔκυκλος
καὶ βαθὺς καὶ ἄκλυστος, ὁ δὲ ἔσπλους ἐς αὐτὸν στεινός.
τοῦτον τῇ γλώσσῃ τῇ ἐπιχωρίῃ Γυναικῶν λιμένα ἐκά-
22.6.1
λεον, ὅτι γυνὴ τοῦ χώρου τούτου πρώτη ἐπῆρξεν. ὡς
δὲ διὰ τῶν σκοπέλων διεξέπλεον, κύμασι τε μεγάλοις
ἐνέκυρσαν καὶ τῇ θαλάσσῃ ῥοώδει. ἀλλὰ ἐκπεριπλῶσαι
22.7.1
γὰρ ὑπὲρ τοὺς σκοπέλους μέγα ἔργον ἐφαίνετο. ἐς δὲ
τὴν ὑστεραίην ἔπλεον νῆσον ἐν ἀριστερᾷ ἔχοντες πρὸ
τοῦ πελάγεος οὕτω τι τῷ αἰγιαλῷ συναφέα ὥστε εἰ-
κάσαι ἂν διώρυχα εἶναι τὸ μέσον τοῦ τε αἰγιαλοῦ καὶ
22.7.5
τῆς νήσου· στάδιοι οἱ πάντες ἑβδομήκοντα τοῦ διέκπλου.
καὶ ἐπί τε τοῦ αἰγιαλοῦ δένδρεα ἦν πολλὰ καὶ δασέα,
22.8.1
καὶ ἡ νῆσος ὕλῃ παντοίῃ σύσκιος. ὑπὸ δὲ τὴν ἕω ἔπλεον  
ἔξω τῆς νήσου κατὰ ῥηχίην στενήν· ἔτι γὰρ ἡ ἀνάπωτις
κατεῖχε. πλώσαντες δὲ ἐς ἑκατὸν καὶ εἴκοσι σταδίους
ὁρμίζονται ἐν τῷ στόματι τοῦ Ἀράβιος ποταμοῦ· καὶ
22.8.5
λιμὴν μέγας καὶ καλὸς πρὸς τῷ στόματι, ὕδωρ δὲ οὐκ
ἦν πότιμον· τοῦ γὰρ Ἀράβιος αἱ ἐκβολαὶ ἀναμεμιγμέναι
22.9.1
τῷ πόντῳ ἦσαν. ἀλλὰ τεσσαράκοντα σταδίους ἐς τὸ ἄνω
προχωρήσαντες λάκκῳ ἐπιτυγχάνουσι, καὶ ἔνθεν ὑδρευ-
22.10.1
σάμενοι ὀπίσω ἀπενόστησαν. νῆσος δὲ ἐπὶ τῷ λιμένι
ὑψηλὴ καὶ ἔρημος, καὶ περὶ ταύτην ὀστρείων τε καὶ
ἰχθύων παντοδαπῶν θήρα. μέχρι μὲν τοῦδε Ἀράβιες,
ἔσχατοι Ἰνδῶν ταύτῃ ᾠκισμένοι, τὰ δὲ ἀπὸ τοῦδε
22.10.5
Ὠρεῖται ἐπεῖχον.
23.1.1
 ὁρμηθέντες δὲ ἐκ τῶν ἐκβολῶν τοῦ Ἀράβιος παρέ-
πλεον τῶν Ὠρειτέων τὴν χώρην. καὶ ὁρμίζονται ἐν Πα-
γάλοισι, πλώσαντες σταδίους ἐς διακοσίους, πρὸς ῥηχίῃ.
ἀλλὰ ἀγκύρῃσι γὰρ ἐπήβολος ἦν ὁ χῶρος. τὰ μὲν οὖν
23.1.5
πληρώματα ἀπεσάλευον ἐν τῇσι νηυσίν, οἱ δὲ ἐφ' ὕδωρ
23.2.1
ἐκβάντες ὑδρεύοντο. τῇ δ' ὑστεραίῃ ἀναχθέντες ἅμα
ἡμέρῃ καὶ πλώσαντες σταδίους ἐς τριάκοντα καὶ τετρα-
κοσίους κατάγονται ἑσπέριοι ἐς Κάβανα, καὶ ὁρμίζονται
πρὸς αἰγιαλῷ ἐρήμῳ. καὶ ἐνταῦθα ῥηχίη τραχείη ἦν, καὶ
23.3.1
ἐπὶ τῷδε μετεώρους τὰς νέας ὡρμίσαντο. κατὰ τοῦτον
τὸν πλόον πνεῦμα ὑπολαμβάνει τὰς νέας μέγα ἐκ πόν-
του, καὶ νέες δύο μακραὶ διαφθείρονται ἐν τῷ πλόῳ,
καὶ κέρκουρος· οἱ δὲ ἄνθρωποι σῴζονται ἀπονηξάμενοι,
23.4.1
ὅτι οὐ πόρρω τῆς γῆς ὁ πλόος ἐγίνετο. ἀμφὶ δὲ μέσας  
νύκτας ἀναχθέντες πλέουσιν ἔστε ἐπὶ Κώκαλα, <ἃ> τοῦ
αἰγιαλοῦ, ἔνθεν ὡρμήθησαν, ἀπεῖχε σταδίους διακοσίους·
καὶ αἱ μὲν νῆες σαλεύουσαι ὥρμεον, τὰ πληρώματα δὲ
23.4.5
ἐκβιβάσας Νέαρχος πρὸς τῇ γῇ ηὐλίσθη, ὅτι ἐπὶ πολλὸν
τεταλαιπωρηκότες ἐν τῇ θαλάσσῃ ἀναπαύσασθαι ἐπό-
θεον· στρατόπεδον δὲ περιεβάλετο τῶν βαρβάρων τῆς
23.5.1
φυλακῆς ἕνεκα. ἐν τούτῳ τῷ χώρῳ Λεόννατος, ὅτῳ τὰ
Ὠρειτῶν ἐξ Ἀλεξάνδρου ἐπετέτραπτο, μάχῃ μεγάλῃ νικᾷ
Ὠρείτας τε καὶ ὅσοι Ὠρείταις συνεπέλαβον τοῦ ἔργου.
καὶ κτείνει αὐτῶν ἑξακισχιλίους, καὶ τοὺς ἡγεμόνας πάν-
23.5.5
τας· τῶν δὲ σὺν Λεοννάτῳ ἱππεῖς μὲν ἀποθνήσκουσι πεν-
τεκαίδεκα, τῶν δὲ πεζῶν ἄλλοι τε οὐ πολλοὶ καὶ Ἀπολ-
23.6.1
λοφάνης ὁ Γαδρωσίων σατράπης. ταῦτα μὲν δὴ ἐν τῇ
ἄλλῃ ξυγγραφῇ ἀναγέγραπται, καὶ ὅπως Λεόννατος ἐπὶ
τῷδε ἐστεφανώθη πρὸς Ἀλεξάνδρου χρυσῷ στεφάνῳ
23.7.1
ἐν Μακεδόσιν. ἐνταῦθα σῖτος ἦν νενημένος κατὰ πρός-
ταγμα Ἀλεξάνδρου ἐς ἐπισιτισμὸν τῷ στρατῷ, καὶ ἐμ-
23.8.1
βάλλονται σιτία ἡμερέων δέκα ἐς τὰς νέας. καὶ τῶν

νεῶν ὅσαι πεπονήκεσαν κατὰ τὸν πλόον μέχρι τοῦδε
ἐπεσκεύασαν, καὶ τῶν ναυτέων ὅσοι ἐν τῷ ἔργῳ βλα-
κεύειν ἐφαίνοντο Νεάρχῳ, τούτους μὲν πεζῇ ἄγειν Λεον-
23.8.5
νάτῳ ἔδωκεν, αὐτὸς δὲ ἀπὸ τῶν σὺν Λεοννάτῳ στρα-
τιωτῶν συμπληροῖ τὸ ναυτικόν.
24.1.1
 ἐνθένδε ὁρμηθέντες ἔπλεον ἀκραεί, καὶ διελθόντες
σταδίους ἐς πεντακοσίους ὡρμίζοντο πρὸς ποταμῷ χει-
24.2.1
μάρρῳ· Τόμηρος ὄνομα ἦν τῷ ποταμῷ. καὶ λίμνη ἦν
ἐπὶ ταῖς ἐκβολαῖς τοῦ ποταμοῦ, τὰ δὲ βράχεα τὰ πρὸς  
τῷ αἰγιαλῷ ἐπῴκεον ἄνθρωποι ἐν καλύβαις πνιγηραῖς.
καὶ οὗτοι ὡς προσπλέοντας εἶδον, ἐθάμβησάν τε καὶ
24.2.5
παρατείναντες σφᾶς παρὰ τὸν αἰγιαλὸν ἐτάχθησαν ὡς
24.3.1
ἀπομαχούμενοι πρὸς τοὺς ἐκβαίνοντας. λόγχας δὲ ἐφό-
ρεον παχέας, μέγεθος ὡς ἑξαπήχεας· ἀκωκὴ δὲ οὐκ ἐπῆν
σιδηρέη, ἀλλὰ τὸ ὀξὺ αὐτῇσι πεπυρακτωμένον ταὐτὸ
24.4.1
ἐποίεε. πλῆθος δὲ ἦσαν ὡς ἑξακόσιοι. καὶ τούτους Νέαρ-
χος ὡς ὑπομένοντάς τε καὶ παρατεταγμένους κατεῖδε,
τὰς μὲν νέας ἀνακωχεύειν κελεύει ἐντὸς βέλους, ὡς τὰ
τοξεύματα ἐς τὴν γῆν ἀπ' αὐτῶν ἐξικνεῖσθαι· αἱ γὰρ
24.4.5
τῶν βαρβάρων λόγχαι παχέαι φαινόμεναι ἀγχέμαχοι μέν,
24.5.1
ἄφοβοι δὲ ἐς τὸ ἐσακοντίζεσθαι ἦσαν. αὐτὸς δὲ τῶν στρα-
τιωτῶν ὅσοι αὐτοί τε κουφότατοι καὶ κουφότατα ὡπλι-
σμένοι τοῦ τε νεῖν δαημονέστατοι, τούτους δὲ ἐκνήξασθαι
24.6.1
κελεύει ἀπὸ ξυνθήματος. πρόσταγμα δέ σφισιν ἦν, ὅπως
τις ἐκνηξάμενος σταίη ἐν τῷ ὕδατι, προσμένειν τὸν πα-
ραστάτην οἱ ἐσόμενον, μηδὲ <ἐμ>βάλλειν πρόσθεν ἐς
τοὺς βαρβάρους, πρὶν ἐπὶ τριῶν ἐς βάθος ταχθῆναι τὴν
24.6.5
φάλαγγα, τότε <δὲ> δρόμῳ ἤδη ἰέναι ἐπαλαλάξαντας.
24.7.1
ἅμα δὲ ἐρρίπτουν ἑωυτοὺς οἱ ἐπὶ τῷδε τεταγμένοι ἐκ
τῶν νεῶν ἐς τὸν πόντον, καὶ ἐνήχοντο ὀξέως, καὶ ἵσταντο
ἐν κόσμῳ, καὶ φάλαγγα ἐκ σφῶν ποιησάμενοι δρόμῳ  
ἐπῄεσαν αὐτοί τε ἀλαλάζοντες τῷ Ἐνυαλίῳ καὶ οἱ ἐπὶ
24.7.5
τῶν νεῶν ξυνεπήχεον, τοξεύματά τε καὶ ἀπὸ μηχανῶν
24.8.1
βέλεα ἐφέροντο ἐς τοὺς βαρβάρους. οἳ δὲ τήν τε λαμ-
πρότητα τῶν ὅπλων ἐκπλαγέντες καὶ τῆς ἐφόδου τὴν
ὀξύτητα καὶ πρὸς τῶν τοξευμάτων τε καὶ τῶν ἄλλων
βελῶν βαλλόμενοι, οἷα δὴ ἡμίγυμνοι ἄνθρωποι, οὐδὲ
24.8.5
ὀλίγον ἐς ἀλκὴν τραπέντες ἐγκλίνουσι. καὶ οἱ μὲν αὐτοῦ
24.9.1
φεύγοντες ἀποθνήσκουσιν, οἳ δὲ καὶ ἁλίσκονται· ἔστι
δὲ οἳ καὶ διέφυγον ἐς τὰ ὄρεα. ἦσαν δὲ οἱ ἁλόντες τά
τε ἄλλα σώματα δασέες καὶ τὰς κεφαλάς, καὶ τοὺς ὄνυ-
χας θηριώδεες· τοῖς γὰρ δὴ ὄνυξιν ὅσα σιδήρῳ διαχρᾶ-
24.9.5
σθαι ἐλέγοντο καὶ τοὺς ἰχθύας τούτοισι παρασχίζοντες
κατεργάζεσθαι καὶ τῶν ξύλων ὅσα μαλακώτερα. τὰ δὲ
ἄλλα τοῖς λίθοισι τοῖσιν ὀξέσιν ἔκοπτον· σίδηρος γὰρ
αὐτοῖσιν οὐκ ἦν. ἐσθῆτα δὲ ἐφόρεον δέρματα θήρεια,
οἳ δὲ καὶ ἰχθύων τῶν μεγάλων [τε] τὰ παχέα.
25.1.1
 ἐνταῦθα νεωλκοῦσι τὰς νέας, καὶ ὅσαι πεπονηκυῖαι
αὐτῶν ἐπισκευάζουσι. τῇ δὲ ἕκτῃ ἡμέρῃ ἐστέλλοντο, καὶ
πλώσαντες σταδίους ἐς τριακοσίους ἀφικνέονται ἐς χῶ-
ρον, ὃς δὴ ἔσχατος ἦν τῆς Ὠρειτῶν γῆς· Μάλανα τῷ
25.2.1
χώρῳ ὄνομα. Ὠρεῖται δὲ ὅσοι ἄνω ἀπὸ θαλάσσης οἰκέου-
σιν, ἐσταλμένοι μὲν κατάπερ Ἰνδοί εἰσι, καὶ τὰ ἐς πό-
λεμον ὡσαύτως παραρτέονται· γλῶσσα δὲ ἄλλη αὐτοῖσι
25.3.1
καὶ ἄλλα νόμαια. μῆκος τοῦ παράπλου παρὰ μὲν χώρην
τὴν Ἀραβίων ἐς χιλίους μάλιστα σταδίους, ἔνθενπερ  
ὡρμήθησαν, παρὰ δὲ τὴν Ὠρειτῶν γῆν ἑξακόσιοι καὶ
25.4.1
χίλιοι. παραπλεόντων δὲ τὴν Ἰνδῶν γῆν (τὸ ἐντεῦθεν
γὰρ οὐκέτι Ἰνδοί εἰσι) λέγει <Νέαρχος> ὅτι αἱ σκιαὶ
25.5.1
αὐτοῖσιν οὐ ταὐτὸ ἐποίεον· ἀλλὰ ὅπου μὲν ἐπὶ πολὺ τοῦ
πόντου ὡς πρὸς μεσημβρίαν προχωρήσειαν, αἳ δὲ καὶ
αὐταὶ [αἱ σκιαὶ] πρὸς μεσημβρίην τετραμμέναι ἐφαίνοντο·
ὁπότε δὲ τὸ μέσον τῆς ἡμέρης ἐπέχοι ὁ ἥλιος, ἤδη δὲ
25.6.1
καὶ ἔρημα σκιῆς πάντα ὤφθη αὐτοῖσι. τῶν τε ἀστέρων
ὅσους πρόσθεν μετεώρους κατεώρων, οἳ μὲν ἀφανέες
πάντη ἦσαν, οἳ δὲ πρὸς αὐτῇ τῇ γῇ ἐφαίνοντο, καταδύ-
νοντές τε καὶ αὐτίκα ἀνατέλλοντες οἱ πάλαι ἀειφανέες.
25.7.1
καὶ ταῦτα οὐκ ἀπεικότα δοκέει μοι ἀναγράψαι Νέαρχος·
ἐπεὶ καὶ ἐν Συήνῃ τῇ Αἰγυπτίῃ, ἐπεὰν τροπὰς ἄγῃ θέ-
ρεος ὥρῃ ὁ ἥλιος, φρέαρ ἀποδεδειγμένον ἐστί, καὶ τοῦτο
ἄσκιον ἐν μεσημβρίῃ φαίνεται· ἐν Μερόῃ δὲ πάντα
25.8.1
ἄσκια τῇ αὐτῇ ὥρῃ. εἰκὸς ὦν καὶ ἐν Ἰνδοῖσιν, ἅτε πρὸς
μεσαμβρίην ᾠκισμένοισι, τὰ αὐτὰ δὴ πάθεα ἐπέχειν, καὶ
μάλιστα δὴ κατὰ τὸν πόντον τὸν Ἰνδικόν, ὅσῳ μᾶλλον
αὐτοῖσιν ἡ θάλασσα πρὸς μεσαμβρίην κέκλιται. ταῦτα
25.8.5
μὲν δὴ ὧδε ἐχέτω.
26.1.1
 ἐπὶ δὲ Ὠρείτῃσι κατὰ μὲν μεσογαίην Γαδρώσιοι ἐπεῖ-
χον, ὧν τὴν χώρην χαλεπῶς διεξῆλθεν ἅμα τῇ στρατιῇ
Ἀλέξανδρος, καὶ κακὰ τοσαῦτα ἔπαθεν, ὅσα οὐδὲ τὰ
σύμπαντα τῆς συμπάσης στρατηλασίης. ταῦτά μοι ἐν τῇ
26.2.1
μέζονι συγγραφῇ ἀναγέγραπται. κάτω δὲ Γαδρωσίων  
παρὰ τὴν θάλασσαν αὐτὴν οἱ Ἰχθυοφάγοι καλεόμενοι
οἰκοῦσι· παρὰ τούτων τὴν γῆν ἔπλεον. τῇ μὲν πρώτῃ
ἡμέρῃ περὶ τὴν δευτέρην φυλακὴν ἀναχθέντες καταί-
26.2.5
ρουσιν εἰς Βαγίσαρα· στάδιοι τοῦ παράπλου ἑξακόσιοι.
26.3.1
λιμήν τε ἔνι αὐτόθι εὔορμος, καὶ κώμη Πάσιρα, ἀπέ-
χουσα ἀπὸ θαλάσσης ἑξήκοντα σταδίους, καὶ οἱ πρόσοι-
26.4.1
κοι αὐτῆς Πασιρέες. ἐς δὲ τὴν ὑστεραίαν πρωΐτερον τῆς
ὥρης ἀναχθέντες περιπλέουσιν ἄκρην ἐπὶ πολύ τε ἀνέ-
χουσαν ἐς τὸν πόντον καὶ αὐτὴν ὑψηλὴν καὶ κρημνώ-

δεα. φρέατα δὲ ὀρύξαντες, ὕδωρ [οὐκ] ὀλίγον καὶ πο-
νηρὸν ἀρυσάμενοι ταύτῃ μὲν τῇ ἡμέρᾳ ἐπὶ ἀγκυρέων
26.6.1
ὥρμεον, ὅτι ῥηχίη κατὰ τὸν αἰγιαλὸν ἀνεῖχεν· ἐς δὲ τὴν
ὑστεραίην καταίρουσιν ἐς Κόλτα, σταδίους ἐλθόντες
διακοσίους. ἐνθένδε ἕωθεν πλεύσαντες σταδίους ἑξακο-
σίους ἐν Καλίμοισιν ὁρμίζονται. κώμη πρὸς τῷ αἰγιαλῷ,
26.6.5
φοίνικες δὲ περὶ αὐτὴν ὀλίγοι πεφύκεσαν, καὶ βάλανοι
ἐπ' αὐτοῖσι χλωραὶ ἐπῆσαν. καὶ νῆσος ὡς ἑκατὸν στα-
δίους ἀπὸ τοῦ αἰγιαλοῦ ἀπέχουσα, Καρνίνη ὄνομα.
26.7.1
ἐνταῦθα ξένια Νεάρχῳ προσφέρουσιν οἱ κωμῆται πρό-
βατα καὶ ἰχθύας· καὶ τῶν προβάτων τὰ κρέα λέγει ὅτι
ἦν ἰχθυώδεα, ἴσα τοῖς τῶν ὀρνίθων τῶν πελαγίων, ὅτι
καὶ αὐτὰ ἰχθύων σιτέεται· πόα γὰρ οὐκ ἔνι ἐν τῇ χώρῃ.  
26.8.1
ἀλλὰ τῇ ὑστεραίῃ πλεύσαντες ἐς σταδίους διακοσίους
ὁρμίζονται πρὸς αἰγιαλῷ καὶ κώμῃ ἀπὸ θαλάσσης ἐς
σταδίους τριάκοντα ἀπεχούσῃ· ἡ μὲν κώμη Κύσα ἐκα-
26.9.1
λέετο, Καρβὶς δὲ τῷ αἰγιαλῷ ὄνομα ἦν. ἐνταῦθα
πλοίοις ἐπιτυγχάνουσι σμικροῖσιν, οἷα ἁλιέων εἶναι πλοῖα
οὐκ εὐδαιμόνων· αὐτοὺς δὲ οὐ καταλαμβάνουσιν, ἀλλ'
ἔφυγον γὰρ καθορμιζομένας κατιδόντες τὰς νέας. σῖτός
26.9.5
τε αὐτόθι οὐκ ἐνῆν, καὶ ἐπιλελοίπει τὴν στρατιὴν ὁ
πολλός· ἀλλὰ αἶγας ἐμβαλόμενοι ἐς τὰς νέας, οὕτω δὴ
26.10.1
ἀπέπλεον. καὶ περιπλώσαντες ἄκρην ὑψηλὴν ὅσον πεν-
τήκοντα καὶ ἑκατὸν σταδίους ἀνέχουσαν ἐς τὸν πόντον,
κατάγονται ἐν λιμένι ἀκλύστῳ. καὶ ὕδωρ αὐτόθι ἦν, καὶ
ἁλιέες ᾤκεον· Μόσαρνα ὄνομα ἦν τῷ λιμένι.
27.1.1
 ἐνθένδε καὶ ἡγεμὼν τοῦ πλόου λέγει <Νέαρχος> ὅτι
συνέπλωσεν αὐτοῖσιν, Ὑδράκης ὄνομα, Γαδρώσιος·
ὑπέστη δὲ Ὑδράκης καταστήσειν αὐτοὺς μέχρι Καρμα-
νίης. τὸ δὲ ἀπὸ τοῦδε οὐκέτι χαλεπὰ ἦν, ἀλλὰ μᾶλλόν
27.2.1
τι† ὀνομαζόμενα, ἔστε ἐπὶ τὸν κόλπον τὸν Περσικόν. ἐκ
δὲ Μοσάρνων νυκτὸς ἐπάραντες πλώουσι σταδίους ἑπτα-
κοσίους καὶ πεντήκοντα ἐς Βάλωμον αἰγιαλόν· ἐνθένδε
ἐς Βάρνα κώμην σταδίους τετρακοσίους, ἵνα φοίνικές τε
27.2.5
πολλοὶ ἐνῆσαν καὶ κῆπος, καὶ ἐν τῷ κήπῳ μύρριναι ἐπε-  
φύκεσαν καὶ ἄλλα ἄνθεα, ἀφ' ὅτων στεφανώματα τοῖσι
κωμήτῃσιν ἐπλέκοντο· ἐνταῦθα πρῶτον δένδρεά τε εἶδον
ἥμερα, καὶ ἀνθρώπους οὐ πάντη θηριώδεας ἐποικέοντας.
27.3.1
ἐνθένδε ἐς διακοσίους σταδίους περιπλώσαντες καταί-
ρουσιν ἐς Δενδρόβοσα, καὶ αἱ νέες ἐπ' ἀγκυρέων ἐσά-
27.4.1
λευσαν. ἐνθένδε ἀμφὶ μέσας νύκτας ἄραντες ἐς Κώφαντα
λιμένα ἀπίκοντο, τετρακοσίους μάλιστα σταδίους διεκ-
27.5.1
πλώσαντες· ἐνταῦθα ἁλιέες τε ᾤκεον, καὶ πλοῖα αὐτοῖσιν
ἦν μικρὰ καὶ πονηρά· καὶ ταῖς κώπαις οὐ κατὰ σκαλμὸν
ἤρεσσον ὡς ὁ Ἑλλήνων νόμος, ἀλλ' ὥσπερ ἐν† ποταμῷ
τὸ ὕδωρ ἐπιβάλλοντες ἔνθεν καὶ ἔνθεν, κατάπερ οἱ σκά-
27.5.5
πτοντες τὴν γῆν. ὕδωρ δὲ πολύ τε ἦν ἐν τῷ λιμένι καὶ
27.6.1
καθαρόν. περὶ δὲ πρώτην φυλακὴν ἄραντες καταίρουσιν
ἐς Κύιζα, ἐς ὀκτακοσίους σταδίους διεκπλώσαντες, ἵνα
αἰγιαλός τε ἔρημος ἦν καὶ ῥαχίη. αὐτόθι ὦν ἐπ' ἀγκυ-
27.7.1
ρέων ὥρμεον, κατὰ ναῦν τε ἐδειπνοποιέοντο. ἐνθένδε
διεκπλώσαντες σταδίους πεντακοσίους ἀπίκοντο ἔς τινα
πόλιν μικρήν, οἰκεομένην ἐπὶ γηλόφου οὐ πόρρω τοῦ
27.8.1
αἰγιαλοῦ. καὶ Νέαρχος ἐπιφρασθεὶς ὅτι σπείρεσθαι τὴν
χώρην εἰκός, λέγει πρὸς Ἀρχίην, ὃς ἦν Ἀναξιδότου μὲν
παῖς, Πελλαῖος, συνέπλει δὲ Νεάρχῳ, τῶν ἐν αἴνῃ ὢν
Μακεδόνων – πρὸς τοῦτον λέγει ὅτι καταληπτέον σφίσιν
27.9.1
εἴη τὸ χωρίον· ἑκόντας τε γὰρ οὐκ ἂν οἴεσθαι δοῦναι τῇ
στρατιῇ σιτία, βίῃ τε οὐχ οἷόν τε εἶναι ἐξαιρέειν, πο-  
λιορκίης δὲ καὶ τριβῆς δεήσειν, σφᾶς δὲ ἐπιλελοιπέναι
τὰ σιτία. ὅτι δὲ ἡ γῆ σιτοφόρος, τῇ καλάμῃ τεκμηριοῦ-
27.9.5
σθαι, ἥντινα οὐ πόρρω τοῦ αἰγιαλοῦ ἀφεώρων βαθέην.
27.10.1
ταῦτα ἐπεί σφισιν ἐδόκεε, τὰς μὲν ἄλλας νέας κελεύει
παραρτέεσθαι ὡς ἐς πλόον, καὶ ὁ Ἀρχίης αὐτῷ ἐξήρτυε
τὰ ἐς τὸν πλόον· αὐτὸς δὲ ὑπολειφθεὶς μετὰ μιῆς νεὼς
ἐπὶ θέαν δῆθεν τῆς πόλιος ᾔει.
28.1.1
 προσάγοντι δὲ αὐτῷ πρὸς τὰ τείχεα φιλίως ξένια ἔφε-
ρον ἐκ τῆς πόλιος θύννους τε ἐν κριβάνοισιν ὀπτούς
– οὗτοι γὰρ ἔσχατοι τῶν Ἰχθυοφάγων οἰκέοντες πρῶτοι
ἐν αὐτοῖσιν ὤφθησαν οὐκ ὠμοφαγέοντες – καὶ πέμ-
28.2.1
ματα ὀλίγα καὶ βαλάνους τῶν φοινίκων. ὃ δὲ ταῦτα μὲν
ἀσμένως δέκεσθαι ἔφη, ἐθέλειν δὲ θεήσασθαι τὴν πόλιν·
28.3.1
οἳ δὲ εἴων παρελθεῖν. ὡς δὲ εἴσω πυλῶν παρῆλθε, δύο
μὲν τῶν τοξοτῶν κατέχειν κελεύει τὴν πυλίδα, αὐτὸς δὲ
μετὰ δύο ἄλλων καὶ τοῦ ἑρμηνέως ἐπὶ τὸ τεῖχος τὸ ταύτῃ
ἀνελθὼν ἐσήμηνε τοῖς ἀμφὶ τὸν Ἀρχίην ὅπως συνέκειτο.
28.3.5
συνέκειτο γὰρ τὸν μὲν σημῆναι, τὸν δὲ συμβαλόντα ποι-
28.4.1
έειν τὸ τεταγμένον. ἰδόντες δὲ τὸ σημήιον οἱ Μακεδόνες
ἐπώκελλόν τε κατὰ τάχος τὰς νέας καὶ ἐξεπήδων σπουδῇ
ἐς τὴν θάλασσαν, οἱ δὲ βάρβαροι ἐκπλαγέντες τοῖς γινο-
28.5.1
μένοις ἐπὶ τὰ ὅπλα ἔθεον. ὁ δὲ ἑρμηνεὺς ὁ σὺν Νεάρχῳ
ἐκήρυσσε σῖτον διδόναι τῇ στρατιῇ, εἰ σώαν ἐθέλουσιν
ἔχειν τὴν πόλιν· οἳ δὲ ἠρνοῦντο εἶναι σφίσι, καὶ ἅμα  
προσέβαλλον τῷ τείχει. ἀλλὰ ἀνέστελλον αὐτοὺς οἱ το-
28.5.5
ξόται οἱ ἀμφὶ τὸν Νέαρχον, ἐξ ὑπερδεξίου τοξεύοντες.
28.6.1
ὡς δὲ ἔμαθον ἐχομένην τε ἤδη καὶ ὅσον οὔπω ἀνδρα-
ποδισθησομένην σφίσι τὴν πόλιν, τότε δὲ δὴ ἐδέοντο
τοῦ Νεάρχου τὸν μὲν σῖτον ὅσπερ ἦν αὐτοῖσι λαβόντα
28.7.1
ἀπάγειν, τὴν πόλιν δὲ μὴ διαφθεῖραι. Νέαρχος δὲ τὸν
μὲν Ἀρχίην κελεύει καταλαβεῖν τὰς πύλας καὶ τὸ κατ'
αὐτὰς τεῖχος, αὐτὸς δὲ συμπέμπει τοὺς κατοψομένους τὸν 

σῖτον εἰ ἀδόλως δεικνύουσιν. οἳ δὲ τὸ μὲν ἀπὸ τῶν
ἰχθύων τῶν ὀπτῶν ἀληλεσμένον ἄλευρον πολὺ ἐδεί-
κνυσαν, πυροὺς δὲ καὶ κριθὰς ὀλίγας· καὶ γὰρ καὶ ἐτύγ-
χανον σίτῳ μὲν τῷ ἀπὸ τῶν ἰχθύων, τοῖσι δὲ ἄρτοισιν
28.9.1
ὅσα ὄψῳ διαχρεόμενοι. ὡς δὲ τὰ ὄντα ἐπεδείκνυον, οὕτω
δὴ ἐκ τῶν παρόντων ἐπισιτισάμενοι ἀνήγοντο, καὶ ὁρμί-
ζονται πρὸς ἄκρην, ἥντινα οἱ ἐπιχώριοι ἱρὴν Ἡλίου
ἦγον· οὔνομα τῇ ἄκρῃ Βάγεια.
29.1.1
 ἐνθένδε ἀμφὶ μέσας νύκτας ἄραντες διεκπλώουσι στα-
δίους ἐς χιλίους ἐς Τάλμενα λιμένα εὔορμον. ἐνθένδε
ἐς Κανασίδα πόλιν ἐρήμην σταδίους ἐς τετρακοσίους,
ἵνα τινὶ φρέατι ὀρυκτῷ ἐπιτυγχάνουσι, καὶ φοίνικες
29.1.5
ἄγριοι ἐπεφύκεσαν. τούτων τοὺς ἐγκεφάλους κόπτοντες
29.2.1
ἐσιτέοντο· σῖτος γὰρ ἐπελελοίπει τὴν στρατιήν. καὶ
κακῶς ἤδη ὑπὸ λιμοῦ ἔχοντες ἔπλεον τήν τε ἡμέρην καὶ
29.3.1
τὴν νύκτα, καὶ ὁρμίζονται πρὸς αἰγιαλῷ ἐρήμῳ. Νέ-
αρχος δὲ καταδείσας μὴ ἄρα ἐς τὴν γῆν ἐκβάντες ἀπο-
λίποιεν τὰς νέας ὑπὸ ἀθυμίης, ἐπὶ τῷδε μετεώρους ἔσχε
29.4.1
τὰς νέας ἐπ' ἀγκυρέων. ἐνθένδε ἀναχθέντες ἐς Κανάτην
ὁρμίζονται, σταδίους ὡς ἑπτακοσίους καὶ πεντήκοντα  
διεκπλώσαντες. ἔστι δὲ καὶ αἰγιαλὸς ἐνταῦθα καὶ διώ-
29.5.1
ρυχες βραχεῖαι. ἐνθένδε σταδίους ὀκτακοσίους πλώσαν-
τες ἐν Ταοῖσιν ὁρμίζονται, κῶμαι δὲ μικραὶ καὶ πονηραὶ
ἐπῆσαν. καὶ οἱ μὲν ἄνθρωποι ἐκλείπουσι τὰ οἰκία, αὐτοὶ
δὲ σίτῳ τινὶ ὀλίγῳ ἐπιτυγχάνουσι, καὶ βαλάνοις ἐκ φοι-
29.5.5
νίκων. καὶ καμήλους ἑπτὰ ὅσαι ἐγκατελήφθησαν κατα-
29.6.1
κόψαντες, ἀπὸ τούτων τὰ κρέα ἐσιτέοντο. ὑπὸ δὲ τὴν
ἕω ἀναχθέντες σταδίους τριακοσίους πλώουσι, καὶ καθ-
ορμίζονται ἐς Δαγάσειρα· ἔνθα νομάδες τινὲς ἄνθρωποι
29.7.1
ᾤκεον. ἐνθένδε ἄραντες τήν τε νύκτα καὶ τὴν ἡμέρην οὐδέν
τι ἐλινύοντες ἔπλεον, ἀλλὰ διελθόντες γὰρ σταδίους
χιλίους τε καὶ ἑκατὸν ἐξέπλωσαν τὸ ἔθνος τῶν Ἰχθυο-
φάγων, πολλὰ κακὰ ταύτῃ παθόντες ἀπορίῃ τῶν ἀναγ-
29.8.1
καίων. ὁρμίζονται δὲ οὐ πρὸς τῇ γῇ – ῥηχίη γὰρ ἦν
ἐπὶ πολλὸν ἀνέχουσα – , ἀλλὰ μετέωροι ἐπ' ἀγκυρέων·
μῆκος τοῦ παράπλου τῶν Ἰχθυοφάγων τῆς χώρης ὀλίγῳ
πλεῦνες στάδιοι μύριοι.
29.9.1
 οὗτοι οἱ Ἰχθυοφάγοι σιτέονται, κατ' ὅ τι περ καὶ
κληίζονται, ἰχθύας, ὀλίγοι μὲν αὐτῶν ἁλιεύοντες τοὺς
ἰχθύας – ὀλίγοισι γὰρ καὶ πλοῖα ἐπὶ τῷδε πεποίηται
καὶ τέχνη ἐξεύρηται ἐπὶ τῇ θήρῃ τῶν ἰχθύων – , τὸ
29.10.1
πολὺ δὲ ἡ ἀνάπωτις αὐτοῖσι παρέχει. οἳ δὲ δίκτυα ἐπὶ
τῷδε πεποίηνται, μέγαθος καὶ ἐς δύο σταδίους τὰ πολλὰ
αὐτῶν. πλέκουσι δὲ αὐτὰ ἐκ τοῦ φλοιοῦ τῶν φοινίκων,
29.11.1
στρέφοντες τὸν φλοιὸν ὥσπερ λίνον. ἐπεὰν δὲ ἡ θά-
λασσα ὑπονοστήσῃ καὶ γῆ ὑπολειφθῇ, ἵνα μὲν ξηρὴ ἡ
γῆ ὑπολείπεται, ἐρήμη τὸ πολύ ἐστιν ἰχθύων· ἔνθα δὲ  
βαθέα ἐστίν, ὑπολείπεταί τι τοῦ ὕδατος καὶ ἐν τῷδε
29.11.5
κάρτα πολλοὶ ἰχθύες, οἱ μὲν πολλοὶ σμικροὶ αὐτῶν, οἳ
δὲ καὶ μέζονες· τούτοις περιβάλλοντες τὰ δίκτυα αἱ-
29.12.1
ρέουσι. σιτέονται δὲ ὠμοὺς μέν, ὅπως ἀνειρύουσιν ἐκ
τοῦ ὕδατος, τοὺς ἁπαλωτάτους αὐτῶν· τοὺς δὲ μέζονάς
τε καὶ σκληροτέρους ὑπὸ ἡλίῳ αὐαίνοντες, εὖτ' ἂν
ἀφαυανθῶσι, καταλοῦντες ἄλευρα ἀπ' αὐτῶν ποιέονται
29.12.5
καὶ ἄρτους, οἳ δὲ μάζας ἐκ τούτων τῶν ἀλεύρων πές-
29.13.1
σουσι. καὶ τὰ βοσκήματα αὐτοῖσι τοὺς ἰχθύας ξηροὺς
σιτέονται· ἡ γὰρ χώρη ἔρημος λειμώνων οὐδὲ ποίην
29.14.1
φέρει. θηρεύουσι δὲ καὶ καράβους πολλαχῆ καὶ ὄστρεια
καὶ τὰ κογχύλια· ἅλες δὲ αὐτόματοι γίνονται ἐν τῇ
29.15.1
χώρῃ· ἀπὸ τούτων ἔλαιον ποιέουσιν. οἳ μὲν δὴ αὐτῶν
ἐρήμους τόπους οἰκέουσιν ἄδενδρόν τε τὴν χώρην καὶ
ἄφορον καρπῶν ἡμέρων, τούτοισιν ἀπὸ τῶν ἰχθύων ἡ
πᾶσα δίαιτα πεποίηται· ὀλίγοι δὲ αὐτῶν σπείρουσιν ὅσον
29.15.5
τῆς χώρης, καὶ τούτῳ κατάπερ ὄψῳ χρῶνται πρὸς τοὺς
29.16.1
ἰχθύας· ὁ γὰρ σῖτος αὐτοῖσίν εἰσιν οἱ ἰχθύες. οἰκία δὲ
πεποίηνται οἱ μὲν εὐδαιμονέστατοι αὐτῶν ὅσα κήτεα ἐκ-
βάλλει ἡ θάλασσα τούτων τὰ ὀστᾶ ἐπιλεγόμενοι <καὶ>
τούτοισιν ἀντὶ ξύλων χρεόμενοι, καὶ θύρας τὰ ὀστέα
29.16.5
ὅσα πλατέα αὐτῶν ἁλίσκεται ἀπὸ τούτων ποιέονται·
τοῖσι δὲ πολλοῖς καὶ πενεστέροισιν ἀπὸ τῶν ἀκανθῶν τῶν
ἰχθύων τὰ οἰκία ποιέεται.  
30.1.1
 Κήτεα δὲ μεγάλα ἐν τῇ ἔξω θαλάσσῃ βόσκεται, καὶ
30.2.1
ἰχθύες πολὺ μέζονες ἢ ἐν τῇδε τῇ εἴσω. καὶ λέγει <Νέ-
αρχος>, ὁπότε ἀπὸ Κυΐζων παρέπλεον, ὑπὸ τὴν ἕω ὀφ-
θῆναι ὕδωρ ἄνω ἀναφυσώμενον τῆς θαλάσσης οἷά περ
30.3.1
ἐκ πρηστήρων βίᾳ ἀναφερόμενον, ἐκπλαγέντας δὲ σφᾶς
πυνθάνεσθαι τῶν κατηγεομένων τοῦ πλόου ὅ τι εἴη καὶ
ἀπ' ὅτου τὸ πάθημα· τοὺς δὲ ὑποκρίνασθαι ὅτι κήτεα
ταῦτα φερόμενα κατὰ τὸν πόντον ἀναφυσᾷ ἐς τὸ ἄνω
30.3.5
τὸ ὕδωρ. καὶ τοῖσι ναύτῃσιν ἐκπλαγεῖσιν ἐκ τῶν χειρῶν
30.4.1
τὰ ἐρετμὰ ἐκπεσεῖν, αὐτὸς δὲ ἐπιὼν παρακαλεῖν τε καὶ
θαρσύνειν, καὶ κατ' οὕστινας παραπλέων ἐγένετο, ἐς
μέτωπόν τε κελεῦσαι καταστῆσαι ὡς ἐπὶ ναυμαχίῃ τὰς
νέας, καὶ ἐπαλαλάζοντας ὁμοῦ τῷ ῥοθίῳ πυκνήν τε καὶ
30.5.1
ξὺν κτύπῳ πολλῷ τὴν εἰρεσίην ποιέεσθαι. οὕτως ἀνα-
θαρσήσαντας ὁμοῦ δὴ πλέειν ἀπὸ ξυνθήματος. ὡς δὲ
ἐπέλαζον ἤδη τοῖσι θηρίοισιν, ἐνταῦθα αὐτοὺς μὲν ὅσον
αἱ κεφαλαὶ αὐτοῖσιν ἐχώρεον ἐπαλαλάξαι, τὰς δὲ σάλ-
30.5.5
πιγγας σημῆναι, καὶ τὸν κτύπον ἀπὸ τῆς εἰρεσίης ὡς
30.6.1
ἐπὶ μήκιστον κατασχεῖν. οὕτω δὴ ὁρώμενα ἤδη κατὰ τὰς 

πρῴρας τῶν νεῶν τὰ κήτεα ἐς βυθὸν δῦναι ἐκπλαγέντα,
καὶ οὐ πολλῷ ὕστερον κατὰ τὰς πρύμνας ἀναδύντα ἀνα-
σχεῖν καὶ τῆς θαλάσσης αὖθις ἀναφυσῆσαι ἐπὶ μέγα.
30.7.1
ἔνθεν κρότον τε ἐπὶ τῇ παραλόγῳ σωτηρίᾳ γενέσθαι
τῶν ναυτέων, καὶ αἶνον ἐς τὸν Νέαρχον τῆς τε τόλμης  
30.8.1
καὶ τῆς σοφίης. τούτων μετεξέτερα τῶν κητέων ἐποκέλ-
λειν πολλαχοῦ τῆς χώρης, ἐπειδὰν ἀνάπωτις κατάσχῃ, ἐν
τοῖσι βράχεσιν ἐχόμενα, τὰ δὲ καὶ ὑπὸ χειμώνων σκλη-
ρῶν ἐς τὴν χέρσον ἐξωθέεσθαι, καὶ οὕτω δὴ καὐτὰ ση-
30.8.5
πόμενα ἀπόλλυσθαί τε καὶ τὰς σάρκας αὐτοῖσι περι-
ρρεούσας ὑπολείπειν τὰ ὀστέα χρῆσθαι τοῖσιν ἀνθρώποι-
30.9.1
σιν ἐς τὰ οἰκία. εἶναι ὦν τὰ μὲν ἐν τῇσι πλευρῇσιν αὐτῶν
ὀστέα δοκοὺς τοῖσιν οἰκήμασιν ὅσα μεγάλα, τὰ δὲ μι-
κρότερα στρωτῆρας· τὰ δὲ ἐν τῇσι σιαγόσι, ταῦτα δὲ
εἶναι τὰ θύρετρα, οἷα δὴ πολλῶν καὶ εἰς εἴκοσι καὶ
30.9.5
πέντε ὀργυιὰς ἀνηκόντων τὸ μέγεθος.
31.1.1
 εὖτε δὲ παρέπλεον τὴν χώρην τῶν Ἰχθυοφάγων, λόγον
ἀκούουσι περὶ νήσου τινός, ἣ κεῖται μὲν ἀπέχουσα τῆς
ταύτῃ ἠπείρου σταδίους ἐς ἑκατόν, ἐρήμη δέ ἐστιν οἰκη-
31.2.1
τόρων. ταύτην ἱρὴν Ἡλίου ἔλεγον εἶναι οἱ ἐπιχώριοι
καὶ Νόσαλα καλέεσθαι, οὐδέ τινα ἀνθρώπων καταίρειν
ἐθέλειν ἐς αὐτήν· ὅστις δ' ἂν ἀπειρίῃ προσχῇ, γίνεσθαι
31.3.1
ἀφανέα. ἀλλὰ λέγει <Νέαρχος> κέρκουρόν σφι ἕνα πλή-
ρωμα ἔχοντα Αἰγυπτίων οὐ πόρρω τῆς νήσου ταύτης
γενέσθαι ἀφανέα, καὶ ὑπὲρ τούτου τοὺς ἡγεμόνας τοῦ
πλόου ἰσχυρίζεσθαι ὅτι ἄρα κατάραντες ὑπ' ἀγνοίης εἰς
31.4.1
τὴν νῆσον γένοιντο ἀφανέες. Νέαρχος δὲ πέμπει κύκλῳ
περὶ τὴν νῆσον τριηκόντορον, κελεύσας μὴ κατασχεῖν μὲν
ἐς τὴν νῆσον, ἐμβοᾶν δὲ τοῖς ἀνθρώποις ὡς μάλιστα ἐν
χρῷ παραπλέοντας, καὶ τὸν κυβερνήτην ὀνομάζοντας καὶ
31.5.1
ὅτου ἄλλου οὐκ ἀφανὲς τὸ οὔνομα. ὡς δὲ οὐδένα ὑπα-  
κούειν, τότε δὲ αὐτὸς λέγει πλεῦσαι ἐς τὴν νῆσον καὶ
κατασχεῖν δὴ προσαναγκάσαι τοὺς ναύτας οὐκ ἐθέλον-
τας, καὶ ἐκβῆναι αὐτὸς καὶ ἐλέγξαι κενὸν μῦθον ἐόντα
31.6.1
τὸν περὶ τῆς νήσου λόγον. ἀκοῦσαι δὲ καὶ ἄλλον λόγον
ὑπὲρ τῆς νήσου ταύτης λεγόμενον, οἰκῆσαι τὴν νῆσον
ταύτην μίαν τῶν Νηρηίδων· τὸ δὲ οὔνομα οὐ λέγεσθαι
τῆς Νηρηίδος. ταύτῃ δὲ ὅστις πελάσειε τῇ νήσῳ, τούτῳ
31.6.5
συγγίνεσθαι μέν, ἰχθὺν δὲ αὐτὸν ἐξ ἀνθρώπου ποιέου-
31.7.1
σαν ἐμβάλλειν ἐς τὸν πόντον. Ἥλιον δὲ ἀχθεσθέντα τῇ
Νηρηίδι κελεύειν μετοικίζεσθαι αὐτὴν ἐκ τῆς νήσου·
τὴν δὲ ὁμολογεῖν μὲν ὅτι ἐξοικισθήσεται, δεῖσθαι δέ οἱ
τὸ πάθημα <παυθῆναι>. καὶ τὸν Ἥλιον ὑποδέξασθαι,
31.8.1
τοὺς δὲ δὴ ἀνθρώπους οὕστινας [ἂν] ἰχθύας ἐξ ἀνθρώ-
πων πεποιήκει κατελεήσαντα ἀνθρώπους αὖθις ἐξ ἰχθύων
ποιῆσαι, καὶ ἀπὸ τούτων τῶν Ἰχθυοφάγων τὸ γένος καὶ
31.9.1
εἰς Ἀλέξανδρον κατελθεῖν. καὶ ταῦτα ὅτι ψεύδεα ἐξελέγ-
χει Νέαρχος, οὐκ ἐπαινῶ αὐτὸν ἔγωγε τῆς σχολῆς τε καὶ
σοφίης, οὔτε κάρτα χαλεπὰ ἐξελεγχθῆναι ἐόντα, ταλαί-
πωρόν τε ὂν γιγνώσκων τοὺς παλαιοὺς λόγους ἐπιλεγό-
31.9.5
μενον ἐξελέγχειν ὄντας ψευδέας.
32.1.1
 ὑπὲρ τοὺς Ἰχθυοφάγους Γαδρώσιοι ἐς τὸ ἄνω οἰκέουσι
γῆν πονηρὴν καὶ ψαμμώδεα, ἔνθεν καὶ τὰ πολλὰ κακὰ
ἡ στρατιή τε Ἀλεξάνδρῳ ἔπαθεν καὶ αὐτὸς Ἀλέξανδρος,
32.2.1
ὥς μοι ἤδη ἐν τῷ ἄλλῳ λόγῳ ἀπήγηται. ὡς δὲ ἐς τὴν
Καρμανίην ἀπὸ τῶν Ἰχθυοφάγων κατῆρεν ὁ στρατός,
ἐνταῦθα ἵνα πρῶτον τῆς Καρμανίης ὡρμίσαντο, ἐπ' ἀγ-
κυρέων ἐσάλευσαν, ὅτι ῥηχίη παρετέτατο ἐς τὸ πέλαγος  
32.3.1
τρηχείη. ἐνθένδε οὐκέτι ὡσαύτως πρὸς ἡλίου δυομένου
ἔπλωον, ἀλλὰ τὸ μεταξὺ δύσιός τε ἡλίου καὶ τῆς ἄρκτου
32.4.1
οὕτω μᾶλλόν τι αἱ πρῷραι αὐτοῖσιν ἐπεῖχον, καὶ οὕτω
ἡ Καρμανίη τῶν Ἰχθυοφάγων τῆς γῆς καὶ τῶν Ὠρειτῶν
εὐδενδροτέρη τε καὶ εὐκαρποτέρη ἐστὶ καὶ ποιώδης μᾶλ-
32.5.1
λόν τι καὶ ἔνυδρος. ὁρμίζονται δὲ ἐν Βάδει χώρῳ τῆς
Καρμανίης οἰκουμένῳ, δένδρεά τε πολλὰ ἥμερα πεφυ-
κότα ἔχοντι πλὴν ἐλαίης, καὶ ἀμπέλους ἀγαθάς, καὶ σι-
32.6.1
τοφόρῳ. ἐνθένδε ὁρμηθέντες καὶ διεκπλώσαντες στα-
δίους ὀκτακοσίους πρὸς αἰγιαλῷ ὁρμίζονται ἐρήμῳ, καὶ
καθορῶσιν ἄκρην μακρὴν ἀνέχουσαν ἐπὶ πολλὸν ἐς τὸ
πέλαγος· ἀπέχειν δὲ ἐφαίνετο ἡ ἄκρη πλόον ὡς ἡμέρης.
32.7.1
καὶ οἱ τῶν χώρων ἐκείνων δαήμονες τῆς Ἀραβίης ἔλε-
γον τὴν ἀνίσχουσαν ταύτην ἄκρην, καλέεσθαι <δὲ> Μά-
κετα· ἔνθεν τὰ κιννάμωμά τε καὶ ἄλλα τοιουτότροπα ἐς
32.8.1
Ἀσσυρίους ἀγινέεσθαι. καὶ ἀπὸ τοῦ αἰγιαλοῦ τούτου,
ἵναπερ ὁ στόλος ἐσάλευε, καὶ τῆς ἄκρης, ἥντινα καταν-
τικρὺ ἀφεώρων ἀνέχουσαν ἐς τὸ πέλαγος, ὁ κόλπος
– ἐμοί τε δοκεῖ καὶ Νεάρχῳ ὡσαύτως ἐδόκεεν – ἐς τὸ
32.9.1
εἴσω ἀναχεῖται, ὅπερ εἰκὸς ἡ Ἐρυθρὴ θάλασσα. ταύτην
τὴν ἄκρην ὡς κατεῖδον, Ὀνησίκριτος μὲν ἐπέχοντας ἐπ'
αὐτὴν πλέειν ἐκέλευεν, ὡς μὴ κατὰ τὸν κόλπον ἐλα-
32.10.1
στρέοντας ταλαιπωρέεσθαι. Νέαρχος δὲ ὑποκρίνεται
νήπιον εἶναι Ὀνησίκριτον, εἰ ἀγνοέει ἐπ' ὅτῳ ἐστάλη  
32.11.1
πρὸς Ἀλεξάνδρου ὁ στόλος. οὐ γὰρ ὅτι ἀπορίη ἦν πεζῇ
διασωθῆναι πάντα αὐτῷ τὸν στρατόν, ἐπὶ τῷδε ἄρα ἐκ-
πέμψαι τὰς νέας, ἀλλὰ ἐθέλοντα αἰγιαλούς τε τοὺς κατὰ
τὸν παράπλουν κατασκέψασθαι καὶ ὅρμους καὶ νησῖδας,
32.11.5
καὶ ὅστις κόλπος ἐσέχοι ἐκπεριπλῶσαι τοῦτον, καὶ πό-
λιας ὅσαι ἐπιθαλάσσιαι, καὶ εἴ τις ἔγκαρπος γῆ καὶ εἴ
32.12.1
τις ἐρήμη. σφᾶς ὦν οὐ χρῆναι ἀφανίσαι τὸ ἔργον, πρὸς
τέρματι ἤδη ἐόντας τῶν πόνων, ἄλλως τε οὐδὲ ἀπόρως 

ἔτι τῶν ἀναγκαίων ἐν τῷ παράπλῳ ἔχοντας. δεδιέναι
τε, ὅτι ἡ ἄκρη ἐς μεσημβρίην ἀνέχει, μὴ ἐρήμῳ τε τῇ
32.13.1
ταύτῃ γῇ καὶ ἀνύδρῳ καὶ φλογώδει ἐγκύρσειαν. ταῦτα
ἐνίκα, καί μοι δοκέει περιφανέως σῶσαι τὴν στρατιὴν
τῇδε τῇ βουλῇ Νέαρχος· τὴν γὰρ δὴ ἄκρην ἐκείνην καὶ
τὴν πρὸς αὐτῇ χώρην πᾶσαν ἐρήμην τε εἶναι λόγος κατ-
32.13.5
έχει καὶ ὕδατος ἀπορίῃ ἔχεσθαι.
33.1.1
 ἀλλὰ ἔπλωον γὰρ ἀπὸ τοῦ αἰγιαλοῦ ἄραντες τῇ γῇ
προσεχέες, καὶ πλώσαντες σταδίους ὡς ἑπτακοσίους ἐν
ἄλλῳ αἰγιαλῷ ὡρμίσαντο· Νεόπτανα ὄνομα τῷ αἰγιαλῷ.
33.2.1
καὶ αὖθις ὑπὸ τὴν ἕω ἀνήγοντο, καὶ πλεύσαντες στα-
δίους ἑκατὸν ὁρμίζονται κατὰ ποταμὸν Ἄναμιν· ὁ δὲ
χῶρος Ἁρμόζεια ἐκαλέετο. δαψιλέα δὲ ἤδη καὶ πάμφορα
33.3.1
<τὰ> ταύτῃ ἦν, πλὴν ἐλαῖαι οὐ πεφύκεσαν. ἐνταῦθα ἐκ-
βαίνουσί τε ἐκ τῶν νεῶν καὶ ἀπὸ τῶν πολλῶν πόνων
ἄσμενοι ἀνεπαύοντο, μεμνημένοι ὅσα κακὰ κατὰ τὴν
θάλασσαν πεπονθότες ἦσαν καὶ πρὸς τῇ γῇ τῶν Ἰχθυο-
33.3.5
φάγων, τήν τε ἐρημίην τῆς χώρης καὶ τοὺς ἀνθρώπους
ὅπως θηριώδεες καὶ τὰς σφῶν ἀπορίας ἐπιλεγόμενοι.
33.4.1
καί τινες αὐτῶν ἀπὸ θαλάσσης ἐς τὸ πρόσω ἀνῆλθον,
ἀποσκεδασθέντες τῆς στρατιῆς κατὰ ζήτησιν ἄλλος ἄλλου.  
33.5.1
ἐνταῦθα ἄνθρωπός σφισιν ὤφθη χλαμύδα τε φορῶν
Ἑλληνικὴν καὶ τὰ ἄλλα ὡς Ἕλλην ἐσκευασμένος, καὶ
φωνὴν Ἑλλάδα ἐφώνεε. τοῦτον οἱ πρῶτοι ἰδόντες δα-
κρῦσαι ἔλεγον· οὕτω τι παράλογόν σφισι φανῆναι ἐκ
33.5.5
τῶν τοσῶνδε κακῶν Ἕλληνα μὲν ἄνθρωπον ἰδεῖν, Ἑλ-
33.6.1
λάδος δὲ φωνῆς ἀκοῦσαι. ἐπηρώτων τε ὁπόθεν ἥκοι καὶ
ὅστις ὤν· ὃ δὲ ἀπὸ τοῦ στρατοπέδου τοῦ Ἀλεξάνδρου
ἀποσκεδασθῆναι ἔλεγε, καὶ εἶναι οὐ πόρρω τὸ στρατό-
33.7.1
πεδον καὶ αὐτὸν Ἀλέξανδρον. τοῦτον τὸν ἄνθρωπον
βοῶντές τε καὶ κροτέοντες ἀνάγουσι παρὰ τὸν Νέαρχον·
καὶ Νεάρχῳ πάντα ἔφρασε, καὶ ὅτι πέντε ἡμερέων ὁδὸν
ἀπέχει τὸ στρατόπεδον καὶ ὁ βασιλεὺς ἀπὸ τῆς θαλάς-
33.8.1
σης. τόν τε ὕπαρχον τῆς χώρης ταύτης δείξειν ἔφη Νε-
άρχῳ, καὶ ἔδειξε· καὶ μετὰ τούτου Νέαρχος γνώμην
33.9.1
ποιέεται, ὅπως ἀναβήσεται πρὸς βασιλέα. τότε μὲν δὴ
ἐπὶ τὰς νέας ἀπῆλθον· ὑπὸ δὲ τὴν ἕω τὰς νέας ἐνεώλ-
κεεν, ἐπισκευῆς τε εἵνεκα, ὅσαι αὐτῶν κατὰ τὸν πλοῦν
πεπονήκεσαν, καὶ ἅμα ὅτι ἐν τῷ χώρῳ τούτῳ ὑπολείπε-
33.10.1
σθαί οἱ ἐδόκεε τὸν πολλὸν στρατόν. χάρακά τε ὦν περι-
βάλλεται διπλοῦν περὶ τῷ ναυστάθμῳ, καὶ τεῖχος γήϊνον
καὶ τάφρον βαθείην, ἀπὸ τοῦ ποταμοῦ τῆς ὄχθης ἀρξά-
μενος ἔστε ἐπὶ τὸν αἰγιαλόν, ἵνα αἱ νέες αὐτῷ ἀνειρυ-
33.10.5
σμέναι ἦσαν.
34.1.1
 ἐν ᾧ δὲ ὁ Νέαρχος ταῦτα ἐκόσμεε, τῆς χώρης ὁ ὕπαρ-
χος πεπυσμένος ὅπως ἐν μεγάλῃ φροντίδι ἔχοι Ἀλέξαν-
δρος τὰ ἀμφὶ τὸν στόλον τοῦτον, μέγα δή τι ἀγαθὸν ἐξ
Ἀλεξάνδρου ἂν ἔγνω πείσεσθαι, εἰ πρῶτός οἱ ἀπαγγεί-
34.1.5
λειε τοῦ στρατοῦ τὴν σωτηρίην καὶ τὸν Νέαρχον ὅτι οὐ
34.2.1
πολλῷ ὕστερον ἀφίξεται ἐς ὄψιν τὴν βασιλέος. οὕτω
δὴ τὴν βραχυτάτην ἐλάσας ἀπαγγέλλει Ἀλεξάνδρῳ ὅτι
Νέαρχος οὗτος προσάγει ἀπὸ τῶν νεῶν. τότε μὲν δὴ  
καίπερ ἀπιστέων τῷ λόγῳ Ἀλέξανδρος ἀλλὰ ἐχάρη γε
34.3.1
κατὰ τὸ εἰκὸς τῇ ἀγγελίῃ· ὡς δὲ ἡμέρη τε ἄλλη ἐξ ἄλλης
ἐγίνετο, καὶ ξυντιθέντι αὐτῷ τῆς ἀγγελίης τὸν χρόνον
34.4.1
οὐκέτι πιστὰ τὰ ἐξηγγελμένα ἐφαίνετο, πεμπόμενοί τε
ἄλλοι ἐπ' ἄλλοισιν ὡς ἐπὶ κομιδῇ τοῦ Νεάρχου οἳ μέν
τινες ὀλίγον τῆς ὁδοῦ προελθόντες καὶ οὐδενὶ ἐγκύρ-
σαντες κενοὶ ἐπανῄεσαν, οἳ δὲ καὶ πορρωτέρω ἐλθόντες
34.4.5
καὶ διαμαρτόντες τῶν ἀμφὶ τὸν Νέαρχον οὐδὲ αὐτοὶ
34.5.1
ἐπανῄεσαν, ἐνταῦθα δὴ τὸν μὲν ἄνθρωπον ἐκεῖνον, ὡς
κενά τε ἀγγείλαντα καὶ λυπηρότερά οἱ τὰ πρήγματα
ποιήσαντα τῇ ματαίῃ εὐφροσύνῃ, συλλαβεῖν κελεύει
Ἀλέξανδρος, αὐτὸς δὲ τῇ τε ὄψει καὶ τῇ γνώμῃ δῆλος
34.6.1
ἦν μεγάλῳ ἄχει βεβλημένος. ἐν τούτῳ δὲ τῶν τινες κατὰ
ζήτησιν τοῦ Νεάρχου ἐσταλμένων ἵππους τε ἐπὶ κομιδῇ
αὐτῶν καὶ ἀπήνας δὲ ἄγοντες ἐντυγχάνουσι κατὰ τὴν
ὁδὸν αὐτῷ τε Νεάρχῳ καὶ τῷ Ἀρχίῃ καὶ πέντε ἢ ἓξ ἅμα
34.7.1
αὐτοῖσιν· μετὰ τοσούτων γὰρ ἀνῄει. καὶ ἐντυχόντες οὔτε
αὐτὸν ἐγνώρισαν οὔτε τὸν Ἀρχίην – οὕτω τοι κάρτα
ἀλλοῖοι ἐφάνησαν, κομόωντές τε καὶ ῥυπόωντες καὶ με-
στοὶ ἅλμης καὶ ῥικνοὶ τὰ σώματα καὶ ὠχροὶ ὑπὸ ἀγρυ-
34.8.1
πνίης τε καὶ τῆς ἄλλης ταλαιπωρίης – ἀλλὰ ἐρομένοις
γὰρ αὐτοῖς ἵναπερ εἴη Ἀλέξανδρος, ὑποκρινάμενοι τὸν
34.9.1
χῶρον οἳ δὲ παρήλαυνον. Ἀρχίης δὲ ἐπιφρασθεὶς λέγει
πρὸς Νέαρχον “ὦ Νέαρχε, τούτους τοὺς ἀνθρώπους δι'
ἐρημίας ἐλαύνειν τὴν αὐτὴν ἡμῖν ὁδὸν οὐκ ἐπ' ἄλλῳ τινὶ
συντίθημι [ἢ] ὅτι μὴ κατὰ ζήτησιν τὴν ἡμετέρην ἀπε-
34.10.1
σταλμένους. ὅτι δὲ οὐ γιγνώσκουσιν ἡμέας, οὐκ ἐν
θώματι ποιέομαι· οὕτω γάρ τι ἔχομεν κακῶς ὡς ἄγνω-
στοι εἶναι. φράσωμεν αὐτοῖσιν οἵτινές εἰμεν, καὶ αὐτοὺς  
34.11.1
ἐρώμεθα καθότι ταύτῃ ἐλαύνουσιν.” ἔδοξε τῷ Νεάρχῳ
ἐναίσιμα λέγειν· καὶ ἤροντο ὅποι ἐλαύνουσιν· οἳ δὲ
ὑποκρίνονται ὅτι κατὰ ζήτησιν Νεάρχου τε καὶ τοῦ στρα-
34.12.1
τοῦ τοῦ ναυτικοῦ. ὃ δέ “οὗτος” ἔφη “ἐγώ εἰμι Νέαρ-
χος, καὶ Ἀρχίας οὗτος. ἀλλ' ἄγετε ἡμέας· ἡμεῖς δὲ τὰ
ὑπὲρ τῆς στρατιῆς Ἀλεξάνδρῳ ἀπηγησόμεθα.”
35.1.1
 ἀναλαβόντες <ὦν> αὐτοὺς ἐπὶ τὰς ἀπήνας ὀπίσω
ἤλαυνον. καί τινες αὐτῶν τούτων ὑποφθάσαι ἐθελή-
σαντες τὴν ἀγγελίην, προδραμόντες λέγουσιν Ἀλεξάνδρῳ
ὅτι “οὗτός τοι Νέαρχος, καὶ σὺν αὐτῷ Ἀρχίης καὶ πέντε 


35.1.5
ἄλλοι κομίζονται παρὰ σέ,” ὑπὲρ δὲ τοῦ στρατοῦ παντὸς
35.2.1
οὐδὲν εἶχον ὑποκρίνασθαι. τοῦτο ἐκεῖνο συνθεὶς Ἀλέ-
ξανδρος, τοὺς μὲν παραλόγως ἀποσωθῆναι, τὴν στρα-
τιὴν δὲ πᾶσαν διεφθάρθαι αὐτῷ, οὐ τοσόνδε τοῦ Νε-
άρχου τε καὶ τοῦ Ἀρχίου τῇ σωτηρίῃ ἔχαιρεν, ὅσον
35.3.1
ἐλύπει αὐτὸν ἀπολομένη ἡ στρατιὴ πᾶσα. οὔπω πάντα
ταῦτα εἴρητο, καὶ ὁ Νέαρχός τε καὶ ὁ Ἀρχίης προσῆ-
γον. τοὺς δὲ μόγις καὶ χαλεπῶς ἐπέγνω Ἀλέξανδρος,
ὅτι τε κομόωντας καὶ κακῶς ἐσταλμένους καθεώρα,
35.3.5
ταύτῃ μᾶλλόν τι βεβαιότερον αὐτῷ τὸ ἄχος ὑπὲρ τῆς
35.4.1
στρατιῆς τῆς ναυτικῆς ἐγίνετο. ὃ δὲ τὴν δεξιὰν τῷ
Νεάρχῳ ἐμβαλὼν καὶ ἀπαγαγὼν μόνον αὐτὸν ἀπὸ τῶν
ἑταίρων τε καὶ τῶν ὑπασπιστῶν, πολλὸν ἐπὶ χρόνον
35.5.1
ἐδάκρυεν· ὀψὲ δὲ ἀνενεγκὼν “ἀλλὰ ὅτι σύγε ἡμῖν ἐπαν-
ήκεις σῷος” ἔφη “καὶ Ἀρχίης οὗτος, ἔχοι ἂν ἔμοιγε ὡς
ἐπὶ συμφορῇ τῇ ἁπάσῃ μετρίως· αἱ δέ τοι νέες καὶ ἡ
35.6.1
στρατιὴ κοίῳ τινὶ τρόπῳ διεφθάρησαν;” ὃ δὲ ὑπολαβών
“ὦ βασιλεῦ,” ἔφη “καὶ αἱ νέες τοι σῷαί εἰσι καὶ ὁ
στρατός· ἡμεῖς δὲ οὗτοι ἄγγελοι τῆς σωτηρίας αὐτῶν  
35.7.1
ἥκομεν.” ἔτι μᾶλλον ἐδάκρυεν Ἀλέξανδρος, καθότι ἀνέλ-
πιστός οἱ ἡ σωτηρίη τοῦ στρατοῦ ἐφαίνετο, καὶ ὅπου
ὁρμέουσιν αἱ νέες ἀνηρώτα. ὃ δὲ “αὗται” ἔφη “ἐν τῷ
στόματι τοῦ Ἀνάμιδος ποταμοῦ ἀνειρυσμέναι ἐπισκευά-
35.8.1
ζονται.” Ἀλέξανδρος δὲ τόν τε Δία τὸν Ἑλλήνων καὶ
τὸν Ἄμμωνα τὸν Λιβύων ἐπόμνυσιν, ἦ μὴν μειζόνως
ἐπὶ τῇδε τῇ ἀγγελίῃ χαίρειν ἢ ὅτι τὴν Ἀσίην πᾶσαν
ἐκτημένος ἔρχεται. καὶ γὰρ καὶ τὸ ἄχος οἱ ἐπὶ τῇ ἀπω-
35.8.5
λείῃ τῆς στρατιῆς ἀντίρροπον γενέσθαι τῇ ἄλλῃ πάσῃ
εὐτυχίῃ.
36.1.1
 ὁ δὲ ὕπαρχος τῆς χώρης, ὅντινα συνειλήφει Ἀλέξαν-
δρος ἐπὶ τῆς ἀγγελίης τῇ ματαιότητι, παρόντα κατιδὼν
36.2.1
τὸν Νέαρχον, πίπτει τε αὐτῷ πρὸς τὰ γόνατα, καὶ
“οὗτός τοι” ἔφη “ἐγώ εἰμι, ὃς ἀπήγγειλα Ἀλεξάνδρῳ
ὅτι σῷοι ἥκετε· ὁρᾷς ὅπως διάκειμαι.” οὕτω δὴ δεῖται
Ἀλεξάνδρου Νέαρχος ἀφεῖναι τὸν ἄνδρα, καὶ ἀφίεται.
36.3.1
Ἀλέξανδρος δὲ σωτήρια τοῦ στρατοῦ ἔθυε Διὶ Σωτῆρι
καὶ Ἡρακλεῖ καὶ Ἀπόλλωνι Ἀλεξικάκῳ καὶ Ποσειδῶνί
τε καὶ ὅσοι ἄλλοι θαλάσσιοι θεοί, καὶ ἀγῶνα ἐποίεε
γυμνικόν τε καὶ μουσικόν, καὶ πομπὴν ἔπεμπε· καὶ
36.3.5
Νέαρχος ἐν πρώτοισιν ἐπόμπευε ταινίῃσί τε καὶ ἄνθεσι
36.4.1
πρὸς τῆς στρατιῆς βαλλόμενος. ὡς δὲ ταῦτά οἱ τέλος
εἶχε, λέγει πρὸς Νέαρχον “ἐγώ σε, ὦ Νέαρχε, οὐκέτι
θέλω τὸ πρόσω οὔτ' οὖν κινδυνεύειν οὔτε ταλαιπωρέε-
σθαι, ἀλλὰ ἄλλος γὰρ τοῦ ναυτικοῦ ἐξηγήσεται τὸ ἀπὸ
36.5.1
τοῦδε ἔστε καταστῆσαι αὐτὸ ἐς Σοῦσα.” Νέαρχος δὲ
ὑπολαβὼν λέγει “ὦ βασιλεῦ, ἐγὼ μέν τοι πάντα πείθε-
σθαι ἐθέλω τε καὶ ἀναγκαίη μοί ἐστιν. ἀλλὰ εἰ δή τι  
καὶ σὺ ἐμοὶ χαρίζεσθαι ἐθέλοις, μὴ ποιήσῃς ὧδε, ἀλλά
36.5.5
με ἔασον ἐξηγήσασθαι ἐς ἅπαν τοῦ στρατοῦ, ἔστε σοι
36.6.1
σῴας καταστήσω ἐς Σοῦσα τὰς νέας, μηδὲ τὰ μὲν χα-
λεπὰ αὐτοῦ τε καὶ ἄπορα ἐμοὶ ἐπιτετραμμένα ἐκ σοῦ
ἔστω, τὰ δὲ εὐπετέα τε καὶ κλέους ἤδη ἑτοίμου ἐχόμενα,
36.7.1
ταῦτα δὲ ἀφαιρεθέντα ἄλλῳ ἐς χεῖρας διδόσθω.” ἔτι
λέγοντα παύει αὐτὸν Ἀλέξανδρος, καὶ χάριν προσωμο-
λόγει εἰδέναι. οὕτω δὴ καταπέμπει αὐτόν, στρατιὴν δοὺς
36.8.1
ἐς παραπομπὴν ὡς διὰ φιλίας ἰόντι ὀλίγην. τῷ δὲ οὐδὲ
τὰ τῆς ὁδοῦ τῆς ἐπὶ θάλασσαν ἔξω πόνου ἐγένετο, ἀλλὰ
συλλελεγμένοι γὰρ οἱ κύκλῳ βάρβαροι τὰ ἐρυμνὰ τῆς
χώρης τῆς Καρμανίης κατεῖχον, ὅτι καὶ ὁ σατράπης
36.8.5
αὐτοῖσι τετελευτήκει κατὰ πρόσταξιν Ἀλεξάνδρου, ὁ δὲ
νεωστὶ καθεστηκὼς Τληπόλεμος οὔπω βέβαιον τὸ κράτος
36.9.1
εἶχε. καὶ δὶς ὦν καὶ τρὶς τῇ αὐτῇ ἡμέρῃ ἄλλοισι καὶ
ἄλλοισι τῶν βαρβάρων ἐπιφαινομένοισιν ἐς χεῖρας ᾔεσαν,
καὶ οὕτως οὐδέν τι ἐλινύσαντες μόλις καὶ χαλεπῶς ἐπὶ
θάλασσαν ἀπεσώθησαν. ἐνταῦθα θύει Νέαρχος Διὶ
36.9.5
Σωτῆρι καὶ ἀγῶνα ποιεῖ γυμνικόν.
37.1.1
 ὡς δὲ αὐτῷ τὰ θεῖα ἐν κόσμῳ πεποίητο, οὕτω δὴ ἀνή-
γοντο. παραπλώσαντες δὲ νῆσον ἐρήμην τε καὶ τραχείην
ἐν ἄλλῃ νήσῳ ὁρμίζονται, μεγάλῃ ταύτῃ καὶ οἰκουμένῃ,
πλώσαντες σταδίους τριηκοσίους ἔνθενπερ ὡρμήθησαν.
37.2.1
καὶ ἡ μὲν ἐρήμη νῆσος Ὀργάνα ἐκαλέετο, ἐς ἣν δὲ ὡρ-
μίσθησαν Ὀάρακτα, ἄμπελοί τε ἐν αὐτῇ ἐπεφύκεσαν καὶ
φοίνικες, καὶ σιτοφόρος <ἦν>· τὸ δὲ μῆκος [ἦν] τῆς νή-  
σου στάδιοι ὀκτακόσιοι. καὶ ὁ ὕπαρχος τῆς νήσου Μα-
37.2.5
ζήνης συνέπλει αὐτοῖσι μέχρι Σούσων ἐθελοντὴς ἡγε-
37.3.1
μὼν τοῦ πλόου. ἐν ταύτῃ τῇ νήσῳ ἔλεγον καὶ τοῦ πρώτου
δυναστεύσαντος τῆς χώρης ταύτης δείκνυσθαι τὸν τάφον·
ὄνομα δὲ αὐτῷ Ἐρύθρην εἶναι, ἀπ' ὅτου καὶ τὴν ἐπω-
νυμίην τῇ θαλάσσῃ ταύτῃ εἶναι Ἐρυθρὴν καλέεσθαι.
37.4.1
ἐνθένδε ἐκ τῆς νήσου ἄραντες ἔπλεον· καὶ τῆς νήσου
αὐτῆς παραπλώσαντες ὅσον διακοσίους σταδίους ὁρμί-
ζονται ἐν αὐτῇ αὖθις, καὶ καθορῶσιν ἄλλην νῆσον, ἀπέ-
χουσαν τῆς μεγάλης ταύτης τεσσαράκοντα μάλιστα στα-
37.5.1
δίους. Ποσειδῶνος ἱρὴ ἐλέγετο εἶναι καὶ ἄβατος. ὑπὸ
δὲ τὴν ἕω ἀνήγοντο, καὶ καταλαμβάνει αὐτοὺς ἀνάπωτις
οὕτω τι καρτερή, ὥστε τρεῖς τῶν νεῶν ἐποκείλασαι ἐν
τῷ ξηρῷ ἐσχέθησαν, αἱ δὲ ἄλλαι χαλεπῶς διεκπλώουσαι
37.6.1
τὰς ῥηχίας ἐς τὰ βάθεα ἀπεσώθησαν. αἱ δὲ ἐποκείλασαι
τῆς πλημμυρίδος ἐπιγενομένης αὖθις ἐξέπλωσάν τε καὶ
37.7.1
δευτεραῖαι κατήγοντο ἵναπερ ὁ πᾶς στόλος. ὁρμίζονται 


δὲ ἐς νῆσον ἄλλην, διέχουσαν τῆς ἠπείρου ὅσον τριακο-
37.8.1
σίους σταδίους, πλώσαντες τετρακοσίους. ἐντεῦθεν ὑπὸ
τὴν ἕω ἔπλεον, νῆσον ἐρήμην ἐν ἀριστερᾷ παραμείβον-
τες· ὄνομα δὲ τῇ νήσῳ Πύλωρα. καὶ ὁρμίζον<ται> πρὸς
†ἰδωδώνῃ, πολιχνίῳ σμικρῷ καὶ πάντων ἀπόρῳ ὅτι μὴ
37.8.5
ὕδατος καὶ ἰχθύων· ἰχθυοφάγοι γὰρ καὶ οὗτοι ὑπ' ἀναγ-
37.9.1
καίης ἦσαν, ὅτι πονηρὰν γῆν νέμονται. ἐνθένδε ὑδρευ-
σάμενοι καταίρουσιν ἐς Ταρσίην ἄκρην ἀνατείνουσαν
37.10.1
ἐς τὸ πέλαγος, πλώσαντες σταδίους τριακοσίους. ἔνθεν
ἐς Καταίην, νῆσον ἐρήμην, ἁλιτενέα· αὕτη ἱερὴ Ἑρμέω  
καὶ Ἀφροδίτης ἐλέγετο· στάδιοι τοῦ πλόου τριηκόσιοι.
37.11.1
ἐς ταύτην ὅσα ἔτη ἀφίεται ἐκ τῶν περιοίκων πρόβατα
καὶ αἶγες ἱρὰ τῷ Ἑρμῇ καὶ τῇ Ἀφροδίτῃ, καὶ ταῦτα
ἀπηγριωμένα ἦν ὁρᾶν ὑπὸ χρόνου τε καὶ ἐρημίης.
38.1.1
 μέχρι τοῦδε Καρμανίη· τὰ δὲ ἀπὸ τοῦδε Πέρσαι
ἔχουσι. μῆκος τοῦ πλόου παρὰ τὴν Καρμανίην χώρην
στάδιοι τρισχίλιοι καὶ ἑπτακόσιοι. ζώουσι δὲ κατάπερ
Πέρσαι, ὅτι καὶ ὅμοροι εἰσι Πέρσῃσι, καὶ τὰ ἐς τὸν πό-
38.2.1
λεμον ὡσαύτως κοσμέονται. ἐνθένδε ἄραντες ἐκ τῆς
νήσου τῆς ἱρῆς παρὰ τὴν Περσίδα ἤδη ἔπλεον, καὶ κατά-
γονται ἐς Ἴλαν χῶρον, ἵνα λιμὴν πρὸς νήσου σμικρῆς
καὶ ἐρήμης γίνεται· οὔνομα τῇ νήσῳ Καΐκανδρος, ὁ δὲ
38.3.1
πλόος στάδιοι τετρακόσιοι. ὑπὸ δὲ τὴν ἕω ἐς ἄλλην
νῆσον πλεύσαντες ὁρμίζονται οἰκουμένην, ἵνα καὶ μαρ-
γαρίτην θηρᾶσθαι λέγει <Νέαρχος> κατάπερ ἐν τῇ Ἰν-
δῶν θαλάσσῃ. ταύτης τῆς νήσου τὴν ἄκρην παραπλώ-
38.3.5
σαντες σταδίους ὡς τεσσαράκοντα, ἐνταῦθα ὡρμίσθησαν.
38.4.1
ἐνθένδε πρὸς ὄρει ὁρμίζονται ὑψηλῷ – Ὦχος ὄνομα τῷ
38.5.1
ὄρει – ἐν λιμένι εὐόρμῳ, καὶ ἁλιέες αὐτοῦ ᾤκεον. καὶ
ἔνθεν πλώσαντες σταδίους τετρακοσίους τε καὶ πεντή-
κοντα ὁρμίζονται ἐν Ἀποστάνοισι· καὶ πλοῖα πολλὰ αὐ-
τόθι ὥρμεε, κώμη τε ἐπῆν ἀπέχουσα ἀπὸ θαλάσσης στα-
38.6.1
δίους ἑξήκοντα. νυκτὸς δὲ ἐπάραντες ἔνθεν ἐσπλώουσιν
ἐς κόλπον συνοικεόμενον πολλῇσι κώμῃσι. στάδιοι τοῦ
πλόου τετρακόσιοι· ὁρμίζονται δὲ πρὸς ὑπωρείην. ταύτῃ  
φοίνικές τε πολλοὶ ἐπεφύκεσαν καὶ ὅσα ἄλλα ἀκρόδρυα
38.7.1
ἐν τῇ Ἑλλάδι γῇ φύεται. ἔνθεν ἄραντες ἐς Γώγανα
παραπλέουσι σταδίους μάλιστα ἐς ἑξακοσίους ἐς χώρην
οἰκουμένην· ὁρμίζονται δὲ τοῦ ποταμοῦ τοῦ χειμάρρου
– ὄνομα δὲ Ἀρεών – ἐν τῇσιν ἐκβολῇσιν. ἐνταῦθα
38.7.5
χαλεπῶς ὁρμίζονται· στεινὸς γὰρ ἦν ὁ ἔσπλους κατὰ τὸ
στόμα, ὅτι βράχεα τὰ κύκλῳ αὐτοῦ ἡ ἀνάπωτις ἐποίεε.
38.8.1
καὶ ἔνθεν αὖ ἐν στόματι ἄλλου ποταμοῦ ὁρμίζονται,
διεκπλώσαντες σταδίους ἐς ὀκτακοσίους· Σιτακὸς ὄνομα
τῷ ποταμῷ ἦν· οὐδὲ ἐν τούτῳ εὐμαρέως ὁρμίζονται,
καὶ ὁ πλόος ἅπας οὗτος ὁ παρὰ τὴν Περσίδα βράχεα τε
38.9.1
ἦσαν καὶ ῥηχίαι καὶ τενάγεα. ἐνταῦθα σῖτον καταλαμ-
βάνουσι πολὺν ξυγκεκομισμένον κατὰ πρόσταξιν βασι-
λέως, ὡς σφίσιν εἶναι ἐπισιτίσασθαι· ἐνταῦθα ἔμειναν
ἡμέρας τὰς πάσας μίαν καὶ εἴκοσι, καὶ τὰς ναῦς ἀνειρυ-
38.9.5
σάμενοι, ὅσαι μὲν πεπονήκεσαν ἐπεσκεύαζον, τὰς δὲ
ἄλλας ἐθεράπευον.
39.1.1
 ἐνθένδε ὁρμηθέντες εἰς Ἱέρατιν πόλιν ἀφίκοντο, ἐς
χῶρον οἰκούμενον. ἑπτακόσιοι καὶ πεντήκοντα στάδιοι
ὁ πλόος· ὡρμίσθησαν δὲ ἐν διώρυχι ἀπὸ τοῦ ποταμοῦ
39.2.1
ἐμβεβλημένῃ ἐς θάλασσαν, ᾗ ὄνομα ἦν Ἡράτεμις. ἅμα
δὲ ἡλίῳ ἀνίσχοντι παραπλέουσιν ἐς ποταμὸν χειμάρρουν,
ὄνομα Πάδαργον, ὁ δὲ χῶρος χερρόνησος ἅπας. καὶ ἐν
αὐτῷ κῆποί τε πολλοὶ καὶ ἀκρόδρυα παντοῖα ἐφύετο·
39.3.1
ὄνομα τῷ χώρῳ Μεσαμβρίη. ἐκ Μεσαμβρίης δὲ ὁρμη-
θέντες καὶ διεκπλώσαντες σταδίους μάλιστα ἐς διακο-
σίους ἐς Ταόκην ὁρμίζονται ἐπὶ ποταμῷ Γράνιδι. καὶ
ἀπὸ τούτου ἐς τὸ ἄνω <τὰ> Περσῶν βασίλεια ἦν, ἀπέ-  
39.3.5
χοντα τοῦ ποταμοῦ τῶν ἐκβολέων σταδίους ἐς διακο-
39.4.1
σίους. κατὰ τοῦτον τὸν παράπλουν λέγει <Νέαρχος>
ὀφθῆναι κῆτος ἐκβεβλημένον ἐς τὴν ἠιόνα, καὶ τοῦτο
προσπλώσαντάς τινας τῶν ναυτῶν ἐκμετρῆσαι καὶ φάναι
39.5.1
εἶναι πήχεων πεντήκοντα· δέρμα δὲ αὐτῷ εἶναι φολι-
δωτόν, οὕτω τι ἐς βάθος ἧκον ὡς καὶ ἐπὶ πῆχυν ἐπέχειν,
ὄστρειά τε καὶ λοπάδας καὶ φυκία πολλὰ ἔχειν ἐπιπεφυ-
κότα. καὶ δελφῖνας λέγει ὅτι καθορᾶν ἦν πολλοὺς ἀμφὶ
39.5.5
τῷ κήτει, καὶ τῶν ἐν τῇ ἔσω θαλάσσῃ μείζονας τοὺς δελ-
39.6.1
φῖνας. ἐνθένδε ὁρμηθέντες κατάγονται ἐς Ῥώγονιν πο-
ταμὸν χειμάρρουν ἐν λιμένι εὐόρμῳ· μῆκος τοῦ παρά-
39.7.1
πλου στάδιοι διακόσιοι. ἐνθένδε τετρακοσίους σταδίους
διεκπλώσαντες αὐλίζονται ἐν ποταμῷ χειμάρρῳ· Βρίζανα
τῷ ποταμῷ ὄνομα. ἐνταῦθα χαλεπῶς ὡρμίσαντο, ὅτι
ῥηχίη ἦν καὶ βράχεα, καὶ χοιράδες ἐκ τοῦ πόντου ἀνεῖ-
39.8.1
χον. ἀλλ' ὅτε ἡ πλήμμυρα ἐπῄει, τότε ὡρμίσαντο· ὑπο-
νοστήσαντος δὲ τοῦ ὕδατος, ἐπὶ ξηρῷ ὑπελείφθησαν αἱ
νῆες. ἐπεὶ δὲ ἡ πλημμυρὶς ἐν τάξει ἀμείβουσα ἐπῆλθε,
39.9.1
τότε δὴ ἐκπλώσαντες ὁρμίζονται ἐπὶ ποταμῷ· ὄνομα δὲ
τῷ ποταμῷ Ἄροσις, μέγιστος τῶν ποταμῶν, ὡς λέγει
<Νέαρχος>, ὅσοι ἐν τῷ παράπλῳ τῷδε ἐμβάλλουσιν ἐς τὸν
ἔξω πόντον.  
40.1.1
 μέχρι τοῦδε Πέρσαι οἰκέουσι, τὰ δὲ ἀπὸ τούτων Σού-
σιοι. Σουσίων δὲ ἔθνος αὐτόνομον κατύπερθε προσοι-
κέει· Οὔξιοι καλοῦνται, ὑπὲρ ὅτων λέλεκταί μοι ἐν τῇ
ἄλλῃ συγγραφῇ ὅτι λῃσταί εἰσι. μῆκος τοῦ παράπλου
40.1.5
τῆς Περσίδος χώρης στάδιοι τετρακόσιοι καὶ τετρακις-


χίλιοι. τὴν δὲ Περσίδα γῆν τρίχα νενεμῆσθαι τῶν
ὡρέων λόγος κατέχει. τὸ μὲν αὐτῆς πρὸς τῇ Ἐρυθρῇ
θαλάσσῃ οἰκεόμενον ἀμμῶδές τε εἶναι καὶ ἄκαρπον ὑπὸ
40.3.1
καύματος, τὸ δὲ ἐπὶ τῷδε ὡς πρὸς ἄρκτον τε καὶ βο-
ρέην ἄνεμον ἰόντων καλῶς κεκρᾶσθαι τῶν ὡρέων, καὶ
τὴν χώρην ποιώδεά τε εἶναι καὶ λειμῶνας ὑδρηλούς, καὶ
ἄμπελον πολλὴν φέρειν καὶ ὅσοι ἄλλοι καρποὶ πλὴν
40.4.1
ἐλαίης, παραδείσοις τε παντοίοισι τεθηλέναι καὶ ποτα-
μοῖσι καθαροῖσι διαρρέεσθαι καὶ λίμνῃσι, καὶ ὄρνισιν
ὁκόσοισιν ἀμφὶ ποταμούς τε καὶ λίμνας ἐστὶ τὰ ἤθεα
ἵπποισί τε ἀγαθὴν εἶναι καὶ τοῖσιν ἄλλοισιν ὑποζυγίοισι
40.5.1
νέμεσθαι, καὶ ὑλώδεά τε πολλαχῆ καὶ πολύθηρον. τὴν
δὲ πρόσω ἔτι ἐπ' ἄρκτον ἰόντων χειμερίην τε καὶ νιφε-
τώδεα ***, ὥστε πρέσβεις τινὰς ἐκ τοῦ Εὐξείνου πόν-
του λέγει <Νέαρχος> κάρτα ὀλίγην ὁδὸν διελθόντας ἐν-
40.5.5
τυχεῖν κατ' ὁδὸν ἰόντι τῆς Περσίδος καὶ θῶμα γενέσθαι
Ἀλεξάνδρῳ καὶ εἰπεῖν Ἀλεξάνδρῳ τῆς ὁδοῦ τὴν βραχύ-  
40.6.1
τητα. Σουσίοις δὲ πρόσοικοι ὅτι εἰσὶν Οὔξιοι λέλεκταί
μοι, κατάπερ Μάρδοι μὲν Πέρσαισι προσεχέες οἰκέουσι,
40.7.1
λῃσταὶ καὶ οὗτοι, Κοσσαῖοι δὲ Μήδοισι. καὶ ταῦτα
πάντα τὰ ἔθνεα ἡμέρωσεν Ἀλέξανδρος, χειμῶνος ὥρῃ
ἐπιπεσὼν αὐτοῖσιν, ὅτε ἄβατον σφῶν τὴν χώρην ἦγον.
40.8.1
καὶ πόληας ἐπέκτισε τοῦ μὴ νομάδας ἔτι εἶναι ἀλλὰ
ἀροτῆρας καὶ γῆς ἐργάτας, καὶ ἔχειν ὑπὲρ ὅτων δειμαί-
νοντες μὴ κακὰ ἀλλήλους ἐργάσονται. ἐνθένδε τὴν Σου-
40.9.1
σίων γῆν παρήμειβεν ὁ στρατός. καὶ ταῦτα οὐκέτι ὡσαύ-
τως ἀτρεκέως λέγει <Νέαρχος> ὅτι ἔστιν οἱ ἐκφράσαι,
πλήν γε δὴ τοὺς ὅρμους τε καὶ τὸ μῆκος τοῦ πλόου·
40.10.1
τὴν χώρην γὰρ τεναγώδεά τε εἶναι τὴν πολλὴν καὶ ῥη-
χίῃσιν ἐπὶ μέγα ἐς τὸν πόντον ἐσέχουσαν καὶ ταύτῃ
σφαλερὴν ἐγκαθορμίζεσθαι· πελαγίοισιν ὦν σφίσι τὴν
40.11.1
κομιδὴν τὸ πολὺ γίνεσθαι. ὁρμηθῆναι μὲν δὴ ἐκ τοῦ
ποταμοῦ τῶν ἐκβολέων, ἵναπερ ηὐλίσθησαν ἐπὶ τοῖσιν
οὔροισι τῆς Περσίδος, ὕδωρ δὲ ἐμβαλέσθαι καὶ πέντε
ἡμερέων· οὐκ ἔφασκον γὰρ εἶναι ὕδωρ οἱ καθηγεμόνες
40.11.5
τοῦ πλόου.
41.1.1
 σταδίους δὲ πεντακοσίους κομισθέντες ὁρμίζονται
ἐπὶ στόματι λίμνης ἰχθυώδεος, ᾗ οὔνομα Κατάδερβις·
καὶ νησὶς ἐπῆν τῷ στόματι· Μαργάστανα τῇ νησῖδι οὔ-  
41.2.1
νομα. ἐνθένδε ὑπὸ τὴν ἕω ἐκπλώσαντες κατὰ βράχεα
ἐκομίζοντο ἐπὶ μιᾶς νεώς· πασσάλοις δὲ ἔνθεν καὶ ἔν-
θεν πεπηγόσιν ἀπεδηλοῦτο τὰ βράχεα, κατάπερ ἐν τῷ
μεσσηγὺς Λευκάδος τε νήσου ἰσθμῷ καὶ Ἀκαρνανίης
41.2.5
ἀποδέδεικται σημεῖα τοῖσι ναυτιλλομένοισι τοῦ μὴ ἐπο-
41.3.1
κέλλειν ἐν τοῖσι βράχεσι τὰς νέας. ἀλλὰ τὰ μὲν κατὰ
Λευκάδα ψαμμώδεα ὄντα καὶ τοῖσιν ἐποκείλασι ταχεῖαν
τὴν ὑπονόστησιν ἐνδιδοῖ· κεῖθι δὲ πηλός ἐστιν ἐφ' ἑκά-
τερα τοῦ πλεομένου βαθὺς καὶ ἰλυώδης, ὥστε οὐδεμιᾷ
41.4.1
μηχανῇ ἐποκείλασιν ἦν ἀποσωθῆναι. οἵ τε γὰρ κοντοὶ
κατὰ τοῦ πηλοῦ δύνοντες αὐτοὶ οὐδέν τι ἐπωφέλουν,
ἀνθρώπῳ τε ἐκβῆναι τοῦ ἀπῶσαι τὰς νέας ἐς τὰ πλεό-
μενα ἄπορον ἐγίνετο· ἔδυνον γὰρ κατὰ τοῦ πηλοῦ ἔστε
41.5.1
ἐπὶ τὰ στήθεα. οὕτω δὴ χαλεπῶς διεκπλώσαντες στα-
δίους ἑξακοσίους κατὰ ναῦν ἕκαστοι ὁρμισθέντες ἐνταῦ-
41.6.1
θα δείπνου ἐμνήσθησαν. τὴν νύκτα δὲ ἤδη κατὰ βάθεα
ἔπλεον καὶ τὴν ἐφεξῆς ἡμέρην ἔστε ἐπὶ βουλυτόν· καὶ
ἦλθον σταδίους ἐνακοσίους, καὶ καθωρμίσθησαν ἐπὶ
τοῦ στόματος τοῦ Εὐφράτου πρὸς κώμῃ τινὶ τῆς Βαβυ-
41.7.1
λωνίης χώρης – ὄνομα δὲ αὐτῇ Διρίδωτις – , ἵνα λι-
βανωτόν τε ἀπὸ τῆς Γερραίης γῆς οἱ ἔμποροι ἀγινέουσι
41.8.1
καὶ τὰ ἄλλα ὅσα θυμιήματα ἡ Ἀράβων γῆ φέρει. ἀπὸ
δὲ τοῦ στόματος τοῦ Εὐφράτου ἔστε Βαβυλῶνα πλοῦν
λέγει <Νέαρχος> σταδίους εἶναι ἐς τρισχιλίους καὶ τρια-
κοσίους.
42.1.1
 ἐνταῦθα ἀγγέλλεται Ἀλέξανδρον ἐπὶ Σούσων στέλ-  
λεσθαι. ἔνθεν καὶ αὐτοὶ τὸ ὀπίσω ἔπλεον, ὡς κατὰ τὸν
Πασιτίγριν ποταμὸν ἀναπλώσαντες συμμῖξαι Ἀλεξάν-
42.2.1
δρῳ. ἔπλεον δὴ τὸ ἔμπαλιν ἐν ἀριστερᾷ τὴν γῆν τὴν
Σουσίδα ἔχοντες, καὶ παραπλέουσι λίμνην, ἐς ἣν ὁ Τί-
42.3.1
γρης ἐσβάλλει ποταμός, ὃς ῥέων ἐξ Ἀρμενίης παρὰ πό-
λιν Νῖνον, πάλαι ποτὲ μεγάλην καὶ εὐδαίμονα, τὴν μέ-
σην ἑωυτοῦ τε καὶ τοῦ Εὐφράτου ποταμοῦ γῆν Με-
42.4.1
σοποταμίην ἐπὶ τῷδε κληίζεσθαι ποιέει. ἀπὸ δὲ τῆς
λίμνης ἐς αὐτὸν τὸν ποταμὸν ἀνάπλους στάδιοι ἑξακόσιοι,
ἵνα καὶ κώμη τῆς Σουσίδος, ἣν καλέουσιν Ἄγινιν· αὕτη
δὲ ἀπέχει Σούσων σταδίους ἐς πεντακοσίους. μῆκος τοῦ
42.4.5
παράπλου τῆς Σουσίων γῆς ἔστε ἐπὶ <τὸ> στόμα τοῦ
42.5.1
Πασιτίγριδος ποταμοῦ στάδιοι δισχίλιοι. ἐνθένδε κατὰ
τὸν Πασιτίγριν ἄνω ἀνέπλεον διὰ χώρης οἰκουμένης
καὶ εὐδαίμονος. ἀναπλώσαντες δὲ σταδίους ὡς πεντή-
κοντα καὶ ἑκατὸν αὐτοῦ ὁρμίζονται, προσμένοντες οὕς-
42.5.5
τινας ἐστάλκει Νέαρχος σκεψομένους ἵνα ὁ βασιλεὺς
42.6.1
εἴη. αὐτὸς δὲ ἔθυε θεοῖς τοῖς σωτῆρσι, καὶ ἀγῶνα ἐποίεε,
42.7.1
καὶ ἡ στρατιὴ ἡ ναυτικὴ πᾶσα ἐν εὐθυμίῃσιν ἦν. ὡς δὲ
προσάγων ἤδη Ἀλέξανδρος ἠγγέλλετο, ἔπλεον ἤδη αὖθις
ἐς τὸ ἄνω κατὰ τὸν ποταμόν· καὶ πρὸς τῇ σχεδίῃ ὁρμί-
ζονται, ἐφ' ᾗ τὸ στράτευμα διαβιβάσειν ἔμελλεν Ἀλέ-  
42.8.1
ξανδρος ἐς Σοῦσα. ἐνταῦθα ἀνεμίχθη ὁ στρατός, καὶ
θυσίαι πρὸς Ἀλεξάνδρου ἐθύοντο ἐπὶ τῶν νεῶν τε καὶ
τῶν ἀνθρώπων τῇ σωτηρίῃ, καὶ ἀγῶνες ἐποιέοντο· 

καὶ Νέαρχος ὅποι παραφανείη τῆς στρατιῆς, ἄνθεσί τε
42.9.1
καὶ ταινίῃσιν ἐβάλλετο. ἔνθα καὶ χρυσῷ στεφάνῳ στεφα-
νοῦνται ἐξ Ἀλεξάνδρου Νέαρχός τε καὶ Λεόννατος, Νέ-
αρχος μὲν ἐπὶ τοῦ ναυτικοῦ τῇ σωτηρίῃ, Λεόννατος δὲ
ἐπὶ τῇ νίκῃ, ἣν Ὠρείτας τε ἐνίκησε καὶ τοὺς Ὠρείταις
42.10.1
προσοικέοντας βαρβάρους. οὕτω μὲν ἀπεσώθη Ἀλεξάν-
δρῳ ἐκ τοῦ Ἰνδοῦ τῶν ἐκβολέων ὁρμηθεὶς ὁ στρατός.
43.1.1
 τὰ δὲ ἐν δεξιᾷ τῆς Ἐρυθρῆς θαλάσσης ὑπὲρ τὴν Βα-
βυλωνίην Ἀραβίη ἡ πολλή ἐστι, καὶ ταύτης τὰ μὲν κατ-
ήκει ἔστε ἐπὶ τὴν θάλασσαν τὴν κατὰ Φοινίκην τε καὶ
τὴν Παλαιστίνην Συρίην, πρὸς δυομένου δὲ ἡλίου ὡς
43.1.5
ἐπὶ τὴν εἴσω θάλασσαν Αἰγύπτιοι τῇ Ἀραβίῃ ὁμουρέ-
43.2.1
ουσι. κατὰ δὲ Αἴγυπτον εἰσέχων ἐκ τῆς μεγάλης θαλάς-
σης κόλπος δῆλον ποιέει ὅτι ἕνεκά γε τοῦ σύρρουν εἶναι
τὴν ἔξω θάλασσαν περίπλους ἂν ἦν ἐκ Βαβυλῶνος ἐς
τὸν κόλπον τοῦτον <τὸν> ἐπέχοντα ὡς ἐπ' Αἴγυπτον.
43.3.1
ἀλλὰ γὰρ οὔ τις παρέπλωσε ταύτῃ οὐδαμῶν ἀνθρώπων
ὑπὸ καύματος καὶ ἐρημίης, εἰ μή τινές γε πελάγιοι κο-
43.4.1
μιζόμενοι. ἀλλὰ οἱ ἀπ' Αἰγύπτου γὰρ ἐς Σοῦσα ἀποσω-
θέντες τῆς στρατιῆς τῆς Καμβύσεω καὶ οἱ παρὰ Πτολε-  
μαίου τοῦ Λάγου παρὰ Σέλευκον τὸν Νικάτορα στα-
43.5.1
λέντες ἐς Βαβυλῶνα διὰ τῆς Ἀραβίης χώρης ἰσθμόν τινα
διαπορευθέντες ἐν ἡμέρῃσιν ὀκτὼ ταῖς πάσαις ἄνυδρον
καὶ ἐρήμην χώρην ἐπῆλθον ἐπὶ καμήλων σπουδῇ ἐλαύ-
νοντες ὕδωρ τε σφιν ἐπὶ τῶν καμήλων φέροντες καὶ νυ-
43.5.5
κτοπορέοντες· τὰς γὰρ ἡμέρας ὑπαίθριοι ἀνέχεσθαι διὰ
43.6.1
καῦμα ἀδύνατοι ἦσαν. τοσούτου δεῖ τά γε ἐπέκεινα
ταύτης τῆς χώρης, ἥντινα ἰσθμὸν ἀπεφαίνομεν ἐκ τοῦ
κόλπου τοῦ Ἀραβίου κατήκοντα ἐς τὴν Ἐρυθρὰν θά-
λασσαν, οἰκεόμενα εἶναι, ὁπότε τὰ πρὸς ἄρκτον μᾶλλον
43.7.1
αὐτῶν ἀνέχοντα ἔρημά τέ ἐστι καὶ ψαμμώδεα. ἀλλὰ γὰρ
ἀπὸ τοῦ Ἀραβίου κόλπου τοῦ κατ' Αἴγυπτον ὁρμηθέντες
ἄνθρωποι ἐκπεριπλώσαντες τὴν πολλὴν Ἀραβίην ἐλθεῖν
ἐς τὴν κατὰ Σοῦσά τε καὶ Πέρσας θάλασσαν, ἐς τοσόνδε
43.7.5
ἄρα παραπλώσαντες τῆς Ἀραβίης ἐς ὅσον σφίσι τὸ
ὕδωρ ἐπήρκεσε τὸ ἐμβληθὲν ἐς τὰς νέας, ἔπειτα ὀπίσω
43.8.1
ἀπενόστησαν. ἐκ βαβυλῶνός τε οὕστινας ἔστειλεν Ἀλέ-
ξανδρος ὡς ἐπὶ μήκιστον πλέοντας ἐν δεξιᾷ τῆς Ἐρυ-
θρῆς θαλάσσης γνῶναι τοὺς ταύτῃ χώρους, οὗτοι νή-
σους μέν τινας κατεσκέψαντο ἐν τῷ παράπλῳ κειμένας,
43.9.1
καί που καὶ τῆς ἠπείρου τῆς Ἀραβίης προσέσχον, τὴν δὲ
ἄκρην, ἥντινα καταντικρὺ τῆς Καρμανίης ἀνέχουσαν λέ-
γει φανῆναι σφίσι <Νέαρχος>, οὐκ ἔστιν ὅστις ὑπερβα-
43.10.1
λὼν ἐπικάμψαι ἐς τὸ ἐπὶ θάτερα δυνατὸς ἐγένετο. δοκέω
δὲ ὡς εἴπερ πλωτά τε ἦν καὶ βαδιστὰ <τὰ> ταύτῃ, ὑπ'  
Ἀλεξάνδρου ἂν τῆς πολυπραγμοσύνης ἐξελήλεγκτο πλω-
43.11.1
τά τε καὶ βαδιστὰ ἐόντα. καὶ Ἄννων δὲ ὁ Λίβυς ἐκ Καρ-
χηδόνος ὁρμηθεὶς ὑπὲρ μὲν Ἡρακλείας στήλας ἐξέπλω-
σεν ἐς τὸν ἔξω πόντον, ἐν ἀριστερᾷ τὴν Λιβύην γῆν
ἔχων, καὶ ἔστε μὲν πρὸς ἀνίσχοντα ἥλιον ὁ πλόος αὐτῷ
43.12.1
ἐγένετο τὰς πάσας πέντε καὶ τριάκοντα ἡμέρας· ὡς δὲ
δὴ ἐς μεσημβρίην ἐξετράπετο, πολλῇσιν ἀμηχανίῃσιν
ἐνετύγχανεν ὕδατός τε ἀπορίῃ καὶ καύματι ἐπιφλέγοντι
43.13.1
καὶ ῥύαξι πυρὸς ἐς τὸν πόντον ἐμβάλλουσιν. ἀλλ' ἡ
Κυρήνη γὰρ τῆς Λιβύης ἐν τοῖς ἐρημοτέροις πεπολισμένη
ποιώδης τέ ἐστι καὶ μαλθακὴ καὶ εὔυδρος καὶ ἄλσεα καὶ
λειμῶνες, καὶ καρπῶν παντοίων καὶ κτηνέων πάμφορός
43.13.5
<ἐστι> ἔστε ἐπὶ τοῦ σιλφίου τὰς ἐκφύσεις· ὑπὲρ δὲ τὸ
σίλφιον τὰ ἄνω αὐτῆς ἔρημα καὶ ψαμμώδεα.
43.14.1
 οὗτός μοι ὁ λόγος ἀναγεγράφθω, φέρων καὶ αὐτὸς
ἐς Ἀλέξανδρον τὸν Φιλίππου, τὸν Μακεδόνα.   
\end{greek}

\subsubsection{English translation}
Text: Arrian: Anabasis Alexandri: Book VIII (Indica)
Tr. E. Iliff Robson (1933)\footnote{From http://www.fordham.edu/halsall/ancient/arrian-bookVIII-India.asp which got it from http://www.und.ac.za/und/classics/india/arrian.htm}

I. ALL the territory that lies west of the river Indus up to the river Cophen is inhabited by Astacenians and Assacenians, Indian tribes. But they are not, like the Indians dwelling within the river Indus, tall of stature, nor similarly brave in spirit, nor as black as the greater part of the Indians. These long ago were subject to the Assyrians; then to the Medes, and so they became subject to the Persians; and they paid tribute to Cyrus son of Cambyses from their territory, as Cyrus commanded. The Nysaeans are not an Indian race; but part of those who came with Dionysus to India; possibly even of those Greeks who became past service in the wars which Dionysus waged with Indians; possibly also volunteers of the neighbouring tribes whom Dionysus settled there together with the Greeks, calling the country Nysaea from the mountain Nysa, and the city itself Nysa. And the mountain near the city, on whose foothills Nysa is built, is called Merus because of the incident at Dionysus' birth. All this the poets sang about Dionysus; and I leave it to the narrators of Greek or Eastern history to recount them. Among the Assacenians is Massaca, a great city, where resides the chief authority of the Assacian land; and another city Peucela, this also a great city, not far from the Indus. These places then are inhabited on this side of the Indus towards the west, as far as the river Cophen.

II. But the parts from the Indus eastward, these I shall call India, and its inhabitants Indians. The boundary of the land of India towards the north is Mount Taurus. It is not still called Taurus in this land; but Taurus begins from the sea over against Pamphylia and Lycia and Cilicia; and reaches as far as the Eastern Ocean, running right across Asia. But the mountain has different names in different places; in one, Parapamisus, in another Hemodus; elsewhere it is called Imaon, and perhaps has all sorts of other names; but the Macedonians who fought with Alexander called it Caucasus; another Caucasus, that is, not the Scythian; so that the story ran that Alexander came even to the far side of the Caucasus. The western part of India is bounded by the river Indus right down to the ocean, where the river runs out by two mouths, not joined together as are the five mouths of the Ister; but like those of the Nile, by which the Egyptian delta is formed; thus also the Indian delta is formed by the river Indus, not less than the Egyptian; and this in the Indian tongue is called Pattala. Towards the south this ocean bounds the land of India, and eastward the sea itself is the boundary. The southern part near Pattala and the mouths of the Indus were surveyed by Alexander and Macedonians, and many Greeks; as for the eastern part, Alexander did not traverse this beyond the river Hyphasis. A few historians have described the parts which are this side of the Ganges and where are the mouths of the Ganges and the city of Palimbothra, the greatest Indian city on the Ganges.

III. I hope I may be allowed to regard Eratosthenes of Cyrene as worthy of special credit, since he was a student of Geography. He states that beginning with Mount Taurus, where are the springs of the river Indus, along the Indus to the Ocean, and to the mouths of the Indus, the side of India is thirteen thousand stades in length. The opposite side to this one, that from the same mountain to the Eastern Ocean, he does not reckon as merely equal to the former side, since it has a promontory running well into the sea; the promontory stretching to about three thousand stades. So then he would make this side of India, to the eastward, a total length of sixteen thousand stades. This he gives, then, as the breadth of India. Its length, however, from west to east, up to the city of Palimbothra, he states that he gives as measured by reed-measurements; for there is a royal road; and this extends to ten thousand stades; beyond that, the information is not so certain. Those, however, who have followed common talk say that including the promontory, which runs into the sea, India extends over about ten thousand stades; but farther north its length is about twenty thousand stades. But Ctesias of Cnidus affirms that the land of India is equal in size to the rest of Asia, which is absurd; and Onesicritus is absurd, who says that India is a third of the entire world; Nearchus, for his part, states that the journey through the actual plain of India is a four months' journey. Megasthenes would have the breadth of India that from east to west which others call its length; and he says that it is of sixteen thousand stades, at its shortest stretch. From north to south, then, becomes for him its length, and it extends twenty-two thousand three hundred stades, to its narrowest point. The Indian rivers are greater than any others in Asia; greatest are the Ganges and the Indus, whence the land gets its name; each of these is greater than the Nile of Egypt and the Scythian Ister, even were these put together; my own idea is that even the Acesines is greater than the Ister and the Nile, where the Acesines having taken in the Hydaspes, Hydraotes, and Hyphasis, runs into the Indus, so that its breadth there becomes thirty stades. Possibly also other greater rivers run through the land of India.

IV. As for the yonder side of the Hyphasis, I cannot speak with confidence, since Alexander did not proceed beyond the Hyphasis. But of these two greatest rivers, the Ganges and the Indus, Megasthenes wrote that the Ganges is much greater than the Indus, and so do all others who mention the Ganges; for (they say) the Ganges is already large as it comes from its springs, and receives as tributaries the river Cainas and the Erannoboas and the Cossoanus, all navigable; also the river Sonus and the Sittocatis and the Solomatis, these likewise navigable. Then besides there are the Condochates and the Sambus and Magon and Agoranis and Omalis; and also there run into it the Commenases, a great river, and the Cacuthis and Andomatis, flowing from the Indian tribe of the Mandiadinae; after them the Amystis by the city Catadupas, and the Oxymagis at the place called Pazalae, and the Errenysis among the Mathae, an Indian tribe, also meet the Ganges. Megasthenes says that of these none is inferior to the Maeander, where the Maeander is navigable. The breath therefore of the Ganges, where it is at its narrowest, runs to a hundred stades; often it spreads into lakes, so that the opposite side cannot be seen, where it is low and has no projections of hills. It is the same with the Indus; the Hydraotes, in the territory of the Cambistholians, receives the Hyphasis in that of the Astrybae, and the Saranges from the Cecians, and the Neydrus from the Attacenians, and flows, with these, into the Acesines. The Hydaspes also among the Oxydracae receives the Sinarus among the Arispae and it too flows out into the Acesines. The Acesines among the Mallians joins the Indus; and the Tutapus, a large river, flows into the Acesines. All these rivers swell the Acesines, and proudly retaining its own name it flows into the Indus. The Cophen, in the Peucelaetis, taking with it the Malantus, the Soastus, and the Garroeas, joins the Indus. Above these the Parenus and Saparnus, not far from one another, flow into the Indus. The Soanus, from the mountains of the Abissareans, without any tributary, flows into it. Most of these Megasthenes reports to be navigable. It should not then be incredible that neither Nile nor Ister can be even compared with Indus or Ganges in volume of water. For we know of no tributary to the Nile; rather from it canals have been cut through the land of Egypt. As for the Ister, it emerges from its springs a meagre stream, but receives many tributaries; yet not equal in number to the Indian tributaries which flow into Indus or Ganges; and very few of these are navigable; I myself have only noticed the Enus and the Saus. The Enus on the line between Norica and Rhaetia joins the Ister, the Saus in Paeonia. The country where the rivers join is called Taurunus. If anybody is aware of other navigable rivers which form tributaries to the Ister, he certainly does not know many.

V. I hope that anyone who desires to explain the cause of the number and size of the Indian rivers will do so; and that my remarks may be regarded as set down on hearsay only. For Megasthenes has recorded names of many other rivers, which beyond the Ganges and the Indus run into the eastern and southern outer ocean; so that he states the number of Indian rivers in all to be fifty-eight, and these all navigable. But not even Megasthenes, so far as I can see, travelled over any large part of India; yet a good deal more than the followers of Alexander son of Philip did. For he states that he met Sandracottus, the greatest of the Indian kings, and Porus, even greater than he was. This Megasthenes says, moreover, that the Indians waged war on no men, nor other men on the Indians, but on the other hand that Sesostris the Egyptian, after subduing the most part of Asia, and after invading Europe with an army, yet returned back; and Indathyrsis the Scythian who started from Scythia subdued many tribes of Asia, and invaded Egypt victoriously; but Semiramis the Assyrian queen tried to invade India, but died before she could carry out her purposes; it was in fact Alexander only who actually invaded India. Before Alexander, too, there is a considerable tradition about Dionysus as having also invaded India, and having subdued the Indians; about Heracles there is not much tradition. As for Dionysus, the city of Nysa is no mean memorial of his expedition, and also Mount Merus, and the growth of ivy on this mountain then the habit of the Indians themselves setting out to battle with the sound of drums and cymbals; and their dappled costume, like that worn by the bacchanals, of Dionysus. But of Heracles the memorials are slight. Yet the story of the rock Aornos, which Alexander forced, namely, that Heracles could not capture it, I am inclined to think a Macedonian boast; just as the Macedonians called Parapamisus by the name of Caucasus, though it has nothing to do with Caucasus. And besides, learning that there was a cave among the Parapamisadae, they said that this was the cave of Prometheus the Titan, in which he was crucified for his theft of the fire. Among the Sibae, too, an Indian tribe, having noticed them clad with skins they used to assert that they were relics of Heracles' expedition. What is more, as the Sibae carried a club, and they brand their cattle with a club, they referred this too to some memory of Heracles' club. If anyone believes this, at least it must be some other Heracles, not he of Thebes, but either of Tyre or of Egypt, or some great king of the higher inhabited country near India.

VI. This then must be regarded as a digression, so that too much credence may not be given to the stories which certain persons have related about the Indians beyond the Hyphasis; for those who served under Alexander are reasonably trustworthy up to the Hyphasis. For Megasthenes tells us this also about an Indian river; its name is Silas, it flows from a spring of the same name as the river through the territory of the Sileans, the people also named both from river and spring; its water has the following peculiarity; nothing is supported by it, nothing can swim in it or float upon it, but everything goes straight to the bottom; so far is this water thinner and more aery than any other. In the summer there is rain through India; especially on the mountains, Parapamisus and Hemodus and the Imaus, and from them the rivers run great and turbulent. The plains of India also receive rain in summer, and much part of them becomes swamp; in fact Alexander's army retired from the river Acesines in midsummer, when the river had overflowed on to the plains; from these, therefore, one can gauge the flooding of the Nile, since probably the mountains of Ethiopia receive rain in summer, and from them the Nile is swollen and overflows its banks on to the land of Egypt the Nile therefore also runs turbid this time of the year, as it probably would not be from melting snow; nor yet if its stream was dammed up by the seasonal winds which blow during the summer; and besides, the mountains of Ethiopia are probably not snowcovered, on account of the heat. But that they receive rain as India does is not outside the bounds of probability; since in other respects India is not unlike Ethiopia, and the Indian rivers have crocodiles like the Ethiopian and Egyptian Nile; and some of the Indian rivers have fish and other large water animals like those of the Nile, save the river-horse: though Onesicritus states that they do have the river-horse also. The appearance of the inhabitants, too, is not so far different in India and Ethiopia; the southern Indians resemble the Ethiopians a good deal, and, are black of countenance, and their hair black also, only they are not as snub-nosed or so woolly-haired as the Ethiopians; but the northern Indians are most like the Egyptians in appearance.

VII. Megasthenes states that there are one hundred and eighteen Indian tribes. That there are many, I agree with Megasthenes; but I cannot conjecture how he learnt and recorded the exact number, when he never visited any great part of India, and since these different races have not much intercourse one with another. The Indians, he says, were originally nomads, as are the non-agricultural Scythians, who wandering in their waggons inhabit now one and now another part of Scythia; not dwelling in cities and not reverencing any temples of the gods; just so the Indians also had no cities and built no temples; but were clothed with the skins of animals slain in the chase, and for food ate the bark of trees; these trees were called in the Indian tongue Tala, and there grew upon them, just as on the tops of palm trees, what look like clews of wool. They also used as food what game they had captured, eating it raw, before, at least, Dionysus came into India. But when Dionysus had come, and become master of India, he founded cities, and gave laws for these cities, and became to the Indians the bestower of wine, as to the Greeks, and taught them to sow their land, giving them seed. It may be that Triptolemus, when he was sent out by Demeter to sow the entire earth, did not come this way; or perhaps before Triptolemus this Dionysus whoever he was came to India and gave the Indians seeds of domesticated plants; then Dionysus first yoked oxen to the plough and made most of the Indians agriculturists instead of wanderers, and armed them also with the arms of warfare. Further, Dionysus taught them to reverence other gods, but especially, of course, himself, with clashings of cymbals and beating of drums and dancing in the Satyric fashion, the dance called among Greeks the 'cordax'; and taught them to wear long hair in honour of the god, and instructed them in the wearing of the conical cap and the anointings with perfumes; so that the Indians came out even against Alexander to battle with the sound of cymbals and drums.

VIII. When departing from India, after making all these arrangements, he made Spatembas king of the land, one of his Companions, being most expert in Bacchic rites; when Spatembas died, Budyas his son reigned in his stead; the father was King of India fifty-two years, and the son twenty years; and his son, again, came to the throne, one Cradeuas; and his descendants for the most part received the kingdom in succession, son succeeding father; if the succession failed, then the kings were appointed for some pre-eminence. But Heracles, whom tradition states to have arrived as far as India, was called by the Indians themselves 'Indigenous.' This Heracles was chiefly honoured by the Surasenians, an Indian tribe, among whom are two great cities, Methora and Cleisobora, and the navigable river Iobares flows through their territory. Megasthenes also says that the garb which this Heracles wore was like that of the Theban Heracles, as also the Indians themselves record; he also had many sons in his country, for this Heracles too wedded many wives; he had only one daughter, called Pandaea; as also the country in which she was born, and to rule which Heracles educated her, was called Pandaea after the girl; here she possessed five hundred elephants given by her father, four thousand horsemen, and as many as a hundred and thirty thousand foot-soldiers. This also some writers relate about Heracles; he traversed all the earth and sea, and when he had rid the earth of evil monsters he found in the sea a jewel much affected by women. And thus, even to our day, those who bring exports from India to our country purchase these jewels at great price and export them, and all Greeks in old time, and Romans now who are rich and prosperous, are more eager to buy the sea pearl, as it is called in the Indian tongue for that Heracles, the jewel appearing to him charming, collected from all the sea to India this kind of pearl, to adorn his daughter. And Megasthenes says that this oyster is taken with nets; that it is a native of the sea, many oysters being together, like bees; and that the pearl oysters have a king or queen, as bees do. Should anyone by chance capture the king, he can easily surround the rest of the oysters; but should the king slip through, then the others cannot be taken; and of those that are taken, the Indians let their flesh rot, but use the skeleton as an ornament. For among the Indians this pearl sometimes is worth three times its weight in solid gold, which is itself dug up in India.

IX. In this country where Heracles' daughter was queen, the girls are marriageable at seven years, and the men do not live longer than forty years. About this there is a story among the Indians, that Heracles, to whom when in mature years this daughter was born, realizing that his own end was near, and knowing of no worthy husband to whom he might bestow his daughter, himself became her husband when she was seven, so that Indian kings, their children, were left behind. Heracles made her then marriageable, and hence all the royal race of Pandaea arose, with the same privilege from Heracles. But I think, even if Heracles was able to accomplish anything so absurd, he could have lengthened his own life, so as to mate with the girl when of maturer years. But really if this about the age of the girls in this district is true, it seems to me to tend the same way as the men's age, since the oldest of them die at forty years. For when old age comes on so much sooner and death with age, maturity will reasonably be earlier, in proportion to the end; so that at thirty the men might be on the threshold of old age, and at twenty, men in their prime, and manhood at about fifteen, so that the women might reasonably be marriageable at seven. For that the fruits ripen earlier in this country than elsewhere, and perish earlier, this Megasthenes himself tells us. From Dionysus to Sandracottus the Indians counted a hundred and fifty-three kings, over six thousand and forty-two years, and during this time thrice [Movements were made] for liberty . . . this for three hundred years; the other for a hundred and twenty years; the Indians say that Dionysus was fifteen generations earlier than Heracles; but no one else ever invaded India, not even Cyrus son of Cambyses, though he made an expedition against the Scythians, and in all other ways was the most energetic of the kings in Asia; but Alexander came and conquered by force of arms all the countries he entered; and would have conquered the whole world had his army been willing. But no Indian ever went outside his own country on a warlike expedition, so righteous were they.

X. This also is related; that Indians do not put up memorials to the dead; but they regard their virtues as sufficient memorials for the departed, and the songs which they sing at their funerals. As for the cities of India, one could not record their number accurately by reason of their multitude; but those of them which are near rivers or near the sea, they build of wood; for if they were built of brick, they could not last long because of the rain, and also because their rivers overflow their banks and fill the plains with water. But such cities as are built on high and lofty places, they make of brick and clay. The greatest of the Indian cities is called Palimbothra, in the district of the Prasians, at the confluence of the Erannoboas and the Ganges; the Ganges, greatest of all rivers; the Erannoboas may be the third of the Indian rivers, itself greater than the rivers of other countries; but it yields precedence to the Ganges, when it pours into it its tributary stream. And Megasthenes says that the length of the city along either side, where it is longest, reaches to eighty stades its breadth to fifteen; and a ditch has been dug round the city, six plethra in breadth, thirty cubits high; and on the wall are five hundred and seventy towers, and sixty-four gates. This also is remarkable in India, that all Indians are free, and no Indian at all is a slave. In this the Indians agree with the Lacedaemonians. Yet the Lacedaemonians have Helots for slaves, who perform the duties of slaves; but the Indians have no slaves at all, much less is any Indian a slave.

XI. The Indians generally are divided into seven castes. Those called the wise men are less in number than the rest, but chiefest in honour and regard. For they are under no necessity to do any bodily labour; nor to contribute from the results of their work to the common store; in fact, no sort of constraint whatever rests upon these wise men, save to offer the sacrifices to the gods on behalf of the people of India. Then whenever anyone sacrifices privately, one of these wise men acts as instructor of the sacrifice, since otherwise the sacrifice would not have proved acceptable to the gods. These Indians also are alone expert in prophecy, and none, save one of the wise men, is allowed to prophesy. And they prophesy about the seasons of the year, or of any impending public calamity: but they do not trouble to prophesy on private matters to individuals, either because their prophecy does not condescend to smaller things, or because it is undignified for them to trouble about such things. And when one has thrice made an error in his prophecy, he does not suffer any harm, except that he must for ever hold his peace; and no one will ever persuade such a one to prophesy on whom this silence has been enjoined. These wise men spend their time naked, during the winter in the open air and sunshine, but in summer, when the sun is strong, in the meadows and the marsh lands under great trees; their shade Nearchus computes to reach five plethra all round, and ten thousand men could take shade under one tree; so great are these trees. They eat fruits in their season, and the bark of the trees; this is sweet and nutritious as much as are the dates of the palm. Then next to these come the farmers, these being the most numerous class of Indians; they have no use for warlike arms or warlike deeds, but they till the land; and they pay the taxes to the kings and to the cities, such as are self-governing; and if there is internal war among the Indians, they may not touch these workers, and not even devastate the land itself; but some are making war and slaying all comers, and others close by are peacefully ploughing or gathering the fruits or shaking down apples or harvesting. The third class of Indians are the herdsmen, pasturers of sheep and cattle, and these dwell neither by cities nor in the villages. They are nomads and get their living on the hillsides, and they pay taxes from their animals; they hunt also birds and wild game in the country.

XII The fourth class is of artisans and shopkeepers; these are workers, and pay tribute from their works, save such as make weapons of war; these are paid by the community. In this class are the shipwrights and sailors, who navigate the rivers. The fifth class of Indians is the soldiers' class, next after the farmers in number; these have the greatest freedom and the most spirit. They practise military pursuits only. Their weapons others forge for them, and again others provide horses; others too serve in the camps, those who groom their horses and polish their weapons, guide the elephants, and keep in order and drive the chariots. They themselves, when there is need of war, go to war, but in time of peace they make merry; and they receive so much pay from the community that they can easily from their pay support others. The sixth class of Indians are those called overlookers. They oversee everything that goes on in the country or in the cities; and this they report to the King, where the Indians are governed by kings, or to the authorities, where they are independent. To these it is illegal to make any false report; nor was any Indian ever accused of such falsification. The seventh class is those who deliberate abbut the community together with the King, or, in such cities as are self-governing, with the authorities. In number this class is small, but in wisdom and uprightness it bears the palm from all others; from this class are selected their governors, district governors, and deputies, custodians of the treasures, officers of army and navy, financial officers, and overseers of agricultural works. To marry out of any class is unlawful -- as, for instance, into the farmer class from the artisans, or the other way; nor must the same man practise two pursuits; nor change from one class into another, as to turn farmer from shepherd, or shepherd from artisan. It is only permitted to join the wise men out of any class; for their business is not an easy one, but of all most laborious.

XIII. Most wild animals which the Greeks hunt the Indians hunt also, but these have a way of hunting elephants unlike all other kinds of hunting, just as these animals are unlike other animals. It is this they choose a place that is level and open to the sun's heat, and dig a ditch in a circle, wide enough for a great army to camp within it. They dig the ditch five fathoms broad, and four deep. The earth which they throw out of the ditch they heap on either side of the ditch, and so use it as a wall; then they make shelters for themselves, dug out of the wall on the outside of the ditch, and leave small windows in them; through these the light comes in, and also they watch the animals coming in and charging into the enclosure. Then within the enclosure they leave some three or four of the females, those that are tamest, and leave only one entrance by the ditch, making a bridge over it; and here they heap much earth and grass so that the animals cannot distinguish the bridge, and so suspect any guile. The hunters then keep themselves out of the way, hiding under the shelters dug in the ditch. Now the wild elephants do not approach inhabited places by daylight, but at night they wander all about and feed in herds, following the largest and finest of their number, as cows do the bulls. And when they approach the ditch and hear the trumpeting of the females and perceive them by their scent, they rush to the walled enclosure; and when, working round the outside edge of the ditch, they find the bridge, they push across it into the enclosure. Then the hunters, perceiving the entry of the wild elephants, some smartly remove the bridge, others hurrying to the neighbouring villages report that the elephants are caught in the enclosure; and the inhabitants on hearing the news mount the most spirited, and at the same time most disciplined elephants, and then drive them towards the enclosure, and when they have driven them thither they do not at once join battle, but allow the wild elephants to grow distressed by hunger and to be tamed by thirst. But when they think they are sufficiently distressed, then they erect the bridge again, and enter the enclosure; and at first there is a fierce battle between the tamed elephants and the captives, and then, as one would expect, the wild elephants are tamed, distressed as they are by a sinking of their spirits and by hunger. Then the riders dismounting from the tamed elephants tie together the feet of the now languid wild ones; then they order the tamed elephants to punish the rest by repeated blows, till in their distress they fall to earth; then they come near them and throw nooses round their necks; and climb on them as they lie there. And that they may not toss their drivers nor do them any injury, they make an incision in their necks with a sharp knife, all round, and bind their noose round the wound, so that by reason of the sore they keep their heads and necks still. For were they to turn round to do mischief, the wound beneath the rope chafes them. And so they keep quiet, and perceiving that they are conquered, they are led of by the tamed elephants by the rope.

XIV. Such elephants as are not yet full grown or from some defect are not worth the acquiring, they allow to depart to their own laim, Then they lead of their captives to the villages and first of all give them green shoots and grass to eat; but they, from want of heart, are not willing to eat anything; so the Indians range themselves about them and with songs and drums and cymbals, beating and singing, lull them to sleep. For if there is an intelligent animal, it is the elephant. Some of them have been known, when their drivers have perished in battle, to have caught them up and carried them to burial; others have stood over them and protected them. Others, when they have fallen, have actively fought for them; one, indeed, who in a passion slew his driver, died from remorse and grief. I myself have seen an elephant clanging the cymbals, and others dancing; two cymbals were fastened to the player's forelegs, and one on his trunk, and he rhythmically beat with his trunk the cymbal on either leg in turn; the dancers danced in circle, and raising and bending their forelegs in turn moved also rhythmically, as the player with the cymbals marked the time for them. The elephants mate in spring, as do oxen and horses, when certain pores about the temples of the females open and exhale; the female bears its offispring sixteen months at the least, eighteen at most; it has one foal, as does a mare; and this it suckles till its eighth year. The longest-lived elephants survive to two hundred years; but many die before that by disease; but as far as mere age goes, they reach this age. If their eyes are affected, cow's milk injected cures them; for their other sicknesses a draught of dark wine, and for their wounds swine's flesh roast, and laid on the spot, are good. These are the Indian remedies for them.

XV. The Indians regard the tiger as much stronger than the elephant. Nearchus writes that he had seen a tiger's skin, but no tiger; the Indians record that the tiger is in size as great as the largest horse, and its swiftness and strength without parallel, for a tiger, when it meets an elephant, leaps on to the head and easily throttles it. Those, however, which we see and call tigers are dappled jackals, but larger than ordinary jackals. Nay, about ants also Nearchus says that he himself saw no ant, of the sort which some writers have described as native of India; he saw, however, several of their skins brought into the Macedonian camp.Megasthenes, however confirms the accounts given about these ants; that ants do dig up gold, not indeed for the gold, but as they naturally burrow, that they may make holes, just as our small ants excavate a small amount of earth; but these, which are bigger than foxes, dig up earth also proportionate to their size; the earth is auriferous, and thus the Indians get their gold. Megasthenes, however, merely quotes hearsay, and as I have no certainty to write on the subject, I readily dismiss this subject of ants. But Nearchus describes, as something miraculous, parrots, as being found in India, and describes the parrot, and how it utters a human voice. But I having seen several, and knowing others acquainted with this bird, shall not dilate on them as anything remarkable; nor yet upon the size of the apes, nor the beauty of some Indian apes, and the method of capture. For I should only say what everyone knows, except perhaps that apes are anywhere beautiful. And further Nearchus says that snakes are hunted there, dappled and swift; and that which he states Peithon son of Antigenes to have caught, was upwards of sixteen cubits; but the Indians (he proceeds) state that the largest snakes are much larger than this. No Greek physicians have discovered a remedy against Indian snake-bite; but the Indians themselves used to cure those who were struck. And Nearchus adds that Alexander had gathered about him Indians very skilled in physic, and orders were sent round the camp that anyone bitten by a snake was to report at the royal pavilion. But there are not many illnesses in India, since the seasons are more temperate than with us. If anyone is seriously ill, they would inform their wise men, and they were thought to use the divine help to cure what could be cured.

XVI. The Indians wear linen garments, as Nearchus says, the linen coming from the trees of which I have already made mention. This linen is either brighter than the whiteness of other linen, or the people's own blackness makes it appear unusually bright. They have a linen tunic to the middle of the calf, and for outer garments, one thrown round about their shoulders, and one wound round their heads. They wear ivory ear-rings, that is, the rich Indians; the common people do not use them. Nearchus writes that they dye their beards various colours; some therefore have these as white-looking as possible, others dark, others crimson, others purple, others grass-green. The more dignified Indians use sunshades against the summer heat. They have slippers of white skin, and these too made neatly; and the soles of their sandals are of different colours, and also high, so that the wearers seem taller. Indian war equipment differs; the infantry have a bow, of the height of the owner; this they poise on the ground, and set their left foot against it, and shoot thus; drawing the bowstring a very long way back; for their arrows are little short of three cubits, and nothing can stand against an arrow shot by an Indian archer, neither shield nor breastplate nor any strong armour. In their left hands they carry small shields of untanned hide, narrower than their bearers, but not much shorter. Some have javelins in place of bows. All carry a broad scimitar, its length not under three cubits; and this, when they have a hand-to-hand fight -- and Indians do not readily fight so among themselves -- they bring down with both hands in smiting, so that the stroke may be an effective one. Their horsemen have two javelins, like lances, and a small shield smaller than the infantry's. The horses have no saddles, nor do they use Greek bits nor any like the Celtic bits, but round the end of the horses' mouths they have an untanned stitched rein fitted; in this they have fitted, on the inner side, bronze or iron spikes, but rather blunted; the rich people have ivory spikes; within the mouth of the horses is a bit, like a spit, to either end of which the reins are attached. Then when they tighten the reins this bit masters the horse, and the spikes, being attached thereto, prick the horse and compel it to obey the rein.

XVII. The Indians in shape are thin and tall and much lighter in movement than the rest of mankind. They usually ride on camels, horses, and asses; the richer men on elephants. For the elephant in India is a royal mount; then next in dignity is a four-horse chariot, and camels come third; to ride on a single horse is low. Their women, such as are of great modesty, can be seduced by no other gift, but yield themselves to anyone who gives an elephant; and the Indians think it no disgrace to yield thus on the gift of an elephant, but rather it seems honourable for a woman that her beauty should be valued at an elephant. They marry neither giving anything nor receiving anything; such girls as are marriageable their fathers bring out and allow anyone who proves victorious in wrestling or boxing or running or shows pre-eminence in any other manly pursuit to choose among them. The Indians eat meal and till the ground, except the mountaineers; but these eat the flesh of game. This must be enough for a description of the Indians, being the most notable things which Nearchus and Megasthenes, men of credit, have recorded about them. But as the main subject of this my history was not to write an account of the Indian customs but the way in which Alexander's navy reached Persia from India, this must all be accounted a digression.

XVIII. For Alexander, when his fleet was made ready on the banks of the Hydaspes, collected together all the Phoenicians and all the Cyprians and Egyptians who had followed the northern expedition. From these he manned his ships, picking out as crews and rowers for them any who were skilled in seafaring. There were also a good many islanders in the army, who understood these things, and Ionians and Hellespontines. As commanders of triremes were appointed, from the Macedonians, Hephaestion son of Amyntor, and Leonnatus son of Eunous, Lysimachus son of Agathocles, and Asclepiodorus son of Timander, and Archon son of Cleinias, and Demonicus son of Athenaeus, Archias son of Anaxidotus, Ophellas son of Seilenus, Timanthes son of Pantiades; all these were of Pella. From Amphipolis these were appointed officers: Nearchus son of Androtimus, who wrote the account of the voyage; and Laomedon son of Larichus, and Androsthenes son of Callistratus; and from Orestis. Craterus son of Alexander, and Perdiccas son of Orontes. Of Eordaea, Ptolemaeus son of Lagos and Aristonous son of Peisaeus; from Pydna, Metron son of Epicharmus and Nicarchides son of Simus. Then besides, Attalus son of Andromenes, of Stympha Peucestas son of Alexander, from Mieza; Peithon son of Crateuas, of Alcomenae; Leonnatus son of Antipater, of Aegae; Pantauchus son of Nicolaus, of Aloris; Mylleas son of Zoilus, of Beroea; all these being Macedonians. Of Greeks, Medius son of Oxynthemis, of Larisa; Eumenes son of Hieronymus, from Cardia; Critobulus, son of Plato, of Cos; Thoas son of Menodorus, and Maeander, son of Mandrogenes, of Magnesia; Andron son of Cabeleus, of Teos; of Cyprians, Nicocles son of Pasicrates, of Soh; and Nithaphon son of Pnytagoras, of Salamis. Alexander appointed also a Persian trierarch, Bagoas son of Pharnuces; but of Alexander's own ship the helmsman was Onesicritus of Astypalaea; and the accountant of the whole fleet was Euagoras son of Eucleon, of Corinth. As admiral was appointed Nearchus, son of Androtimus, Cretan by race, and he lived. in Amphipolis on the Strymon. And when Alexander had made all these dispositions, he sacrificed to the gods, both the gods of his race and all of whom the prophets had warned him, and to Poseidon and Amphitrite and the Nereids and to Ocean himself and to the river Hydaspes, whence he started, and to the Acesines, into which the Hydaspes runs, and to the Indus, into which both run; and he instituted contests of art and of athletics, and victims for sacrifice were given to all the army, according to their detachments.

XIX. Then when he had made all ready for starting the voyage, Alexander ordered Craterus to march by the one side of the Hydaspes with his army, cavalry and infantry alike; Hephaestion had already started along the other, with another army even bigger than that under Craterus. Hephaestion took with him the elephants, up to the number of two hundred. Alexander himself took with him all the peltasts, as they are called, and all the archers, and of the cavalry, those called 'Companions'; in all, eight thousand. But Craterus and Hephaestion, with their forces, were ordered to march ahead and await the fleet. But he sent Philip, whom he had made satrap of this country, to the banks of the river Acesines, Philip also with a considerable force; for by this time a hundred and twenty thousand men of fighting age were following him, together with those whom he himself had brought from the sea-coast; and with those also whom his officers, sent to recruit forces, had brought back; so that he now led all sorts of Oriental tribes, and armed in every sort of fashion. Then he himself loosing his ships sailed down the Hydaspes to the meeting-place of Acesines and Hydaspes. His whole fleet of ships was eighteen hundred, both ships of war and merchantmen, and horse transports besides and others bringing provisions together with the troops. And how his fleet descended the rivers, and the tribes he conquered on the descent, and how he endangered himself among the Mallians, and the wound he there received, then the way in which Peucestas and Leonnatus defended him as he lay there -- all this I have related already in my other history, written in the Attic dialect. This my present work, however, is a story of the voyage, which Nearchus successfully undertook with his fleet starting from the mouths of the Indus by the Ocean to the Persian Gulf, which some call the Red Sea.

XX. On this Nearchus writes thus: Alexander had a vehement desire to sail the sea which stretches from India to Persia; but he disliked the length of the voyage and feared lest, meeting with some country desert or without roadsteads, or not properly provided with the fruits of the earth, his whole fleet might be destroyed; and this, being no small blot on his great achievements, might wreck all his happiness; but yet his desire to do something unusual and strange won the day; still, he was in doubt whom he should choose, as equal to his designs; and also as the right man to encourage the personnel of the fleet, -- sent as they were on an expedition of this kind, so that they should not feel that they were being sent blindly to manifest dangers. And Nearchus says that Alexander discussed with him whom he should select to be admiral of this fleet; but as mention was made of one and another, and as Alexander rejected some, as not willing to risk themselves for his sake, others as chicken-hearted, others as consumed by desire for home, and finding some objection to each; then Nearchus himself spoke and pledged himself thus : '0 King, I undertake to lead your fleet! And may God help the emprise! I will bring your ships and men safe to Persia, if this sea is so much as navigable and the undertaking not above human powers.' Alexander, however, replied that he would not allow one of his friends to run such risks and endure such distress; yet Nearchus, did not slacken in his request, but besought Alexander earnestly; till at length Alexander accepted Nearchus' willing spirit, and appointed him admiral of the entire fieet, on which the part of the army which was detailed to sail on this voyage and the crews felt easier in mind, being sure that Alexander would never have exposed Nearchus to obvious danger unless they also were to come through safe. Then the splendour of the whole preparations and the smart equipment of the ships, and the outstanding enthusiasm of the commanders of the triremes about the different services and the crews had uplifted even those who a short while ago were hesitating, both to bravery and to higher hopes about the whole affair; and besides it contributed not a little to the general good spirits of the force that Alexander himself had started down the Indus and had explored both outlets, even into the Ocean, and had offered victims to Poseidon, and all the other sea gods, and gave splendid gifts to the sea. Then trusting as they did in Alexander's generally remarkable good fortune, they felt that there was nothing that he might not dare, and nothing that he could not carry through.

XXI. Now when the trade winds had sunk to rest, which continue blowing from the Ocean to the land all the summer season, and hence render the voyage impossible, they put to sea, in the archonship at Athens of Cephisodorus, on the twentieth day of the month Boedromion, as the Athenians reckon it; but as the Macedonians and Asians counted it, it was ... the eleventh year of Alexander's reign. Nearchus also sacrificed, before weighing anchor, to Zeus the Saviour, and he too held an athletic contest. Then moving out from their roadstead, they anchored on the first day in the Indus river near a great canal, and remained there two days; the district was called Stura; it was about a hundred stades from the roadstead. Then on the third day they started forthand sailed to another canal, thirty stades' distance, and this canal was already-salt; for the sea came up into it, especially at full tides, and then at the ebb the water remained there, mingled with the river water. This place was called Caumara. Thence they sailed twenty stades and anchored at Coreestis, still on the river. Thence they started again and sailed not so very far, for they saw a reef at this outlet of the river Indus, and the waves were breaking violently on the shore, and the shore itself was very rough. But where there was a softer part of the reef, they dug a channel, five stades long, and brought the ships down it, when the flood tide came up from the sea. Then sailing round, to a distance of a hundred and fifty stades, they anchored at a sandy island called Crocala, and stayed there through the next day; and there lives here an Indian race called Arabeans, of whom I made mention in my larger history; and that they have their name from the river Arabis, which runs through their country and finds its outlet in the sea, forming the boundary between this country and that of the Oreitans. From Crocala, keeping on the right hand the hill they call Irus, they sailed on, with a low-lying island on their left; and the island running parallel with the shore makes a narrow bay. Then when they had sailed through this, they anchored in a harbour with good anchorage; and as Ne'archus considered the harbour a large and fine one, he called it Alexander's Haven. At the heads of the harbour there lies an island, about two stades away, called Bibacta; the neighbouring region, however, is called Sangada. This island, forming a barrier to the sea, of itself makes a harbour. There constant strong winds were blowing off the ocean. Nearchus therefore, fearing lest some of the natives might collect to plunder the camp, surrounded the place with a stone wall. He stayed there thirty-three days; and through that time, he says, the soldiers hunted for mussels, oysters, and razor-fish, as they are called; they were all of unusual size. much larger than those of our seas. They also drank briny water.

XXII. On the wind falling, they weighed anchor; and after sailing sixty stades they moored off a sandy shore; there was a desert island near the shore. They used this, therefore, as a breakwater and moored there: the island was called Domai. On the shore there was no water, but after advancing some twenty stades inland they found good water. Next day they sailed up to nightfall to Saranga, some three hundred stades, and moored off the beach, and water was found about eight stades from the beach. Thence they sailed and moored at Sacala, a desert spot. Then making their way through two rocks, so close together that the oar-blades of the ships touched the rocks to port and starboard, they moored at Morontobara, after sailing some three hundred stades. The harbour is spacious, circular, deep, and calm, but its entrance is narrow. They called it, in the natives' language, 'The Ladies' Pool,' since a lady was the first sovereign of this district. When they had got safe through the rocks, they met great waves, and the sea running strong; and moreover it seemed very hazardous to sail seaward of the cliffs. For the next day, however, they sailed with an island on their port beam, so as to break the sea, so close indeed to the beach that one would have conjectured that it was a channel cut between the island and the coast. The entire passage was of some seventy stades. On the beach were many thick trees, and the island was wholly covered with shady forest. About dawn, they sailed outside the island, by a narrow and turbulent passage; for the tide was still falling. And when they had sailed some hundred and twenty stades they anchored in the mouth of the river Arabis. There was a fine large harbour by its mouth; but there was no drinking water; for the mouths of the Arabis were mixed with sea-water. However, after penetrating forty stades inland they found a water-hole, and after drawing water thence they returned back again. By the harbour was a high island, desert, and round it one could get oysters and all kinds of fish. Up to this the country of the Arabeans extends; they are the last Indians settled in this direction; from here on the territory, of the Oreitans begins.

XXIII. Leaving the outlets of the Arabis they coasted along the territory of the Oreitans, and anchored at Pagala, after a voyage of two hundred stades, near a breaking sea; but they were able all the same to cast anchor. The crews rode out the seas in their vessels, though a few went in seach of water, and procured it. Next day they sailed at dawn, and after making four hundred and thirty stades they put in towards evening at Cabana, and moored on a desert shore. There too was a heavy surf, and so they anchored their vessels well out to sea. It was on this part of the voyage that a heavy squall from seaward caught the fleet, and two warships were lost on the passage, and one galley; the men swam off and got to safety, as they were sailing quite near the land. But about midnight they weighed anchor and sailed as far as Cocala, which was about two hundred stades from the beach off which they had anchored. The ships kept the open sea and anchored, but Nearchus disembarked the crews and bivouacked on shore; after all these toils and dangers in the sea, they desired to rest awhile. The camp was entrenched, to keep off the natives. Here Leonnatus, who had been in charge of operations against the Oreitans, beat in a great battle the Oreitans, along with others who had joined their enterprise. He slew some six thousand of them, including all the higher officers; of the cavalry with Leonnatus, fifteen fell, and of his infantry, among a few others, Apollophanes satrap of Gadrosia. This I have related in my other history, and also how Leonnatus was crowned by Alexander for this exploit with a golden coronet before the Macedonians. There provision of corn had been gathered ready, by Alexander's orders, to victual the host; and they took on board ten days' rations. The ships which had suffered in the passage so far they repaired; and whatever troops Nearchus thought were inclined to malinger he handed over to Leonnatus, but he himself recruited his fleet from Leonnatus' soldiery.

XXIV. Thence they set sail and progressed with a favouring wind; and after a passage of five hundred stades the anchored by a torrent, which ,was called Tomerus. There was a lagoon at the mouths of the river, and the depressions near the bank were inhabited by natives in stifling cabins. These seeing the convoy sailing up were astounded, and lining along the shore stood ready to repel any who should attempt a landing. They carried thick spears, about six cubits long; these had no iron tip, but the same result was obtained by hardening the point with fire. They were in number about six hundred. Nearchus observed these evidently standing firm and drawn up in order, and ordered the ships to hold back within range, so that their missiles might reach the shore; for the natives' spears, which looked stalwart, were good for close fighting, but had no terrors against a volley. Then Nearchus took the lightest and lightest-armed troops, such as were also the best swimmers, and bade them swim off as soon as the word was given. Their orders were that, as soon as any swimmer found bottom, he should await his mate, and not attack the natives till they had their formation three deep; but then they were to raise their battle cry and charge at the double. On the word, those detailed for this service dived from the ships into the sea, and swam smartly, and took up their formation in orderly manner, and having made a phalanx, charged, raising, for their part, their battle cry to the God of War, and those on shipboard raised the cry along with them; and arrows and missiles from the engines were hurled against the natives. They, astounded at the flash of the armour, and the swiftness of the charge, and attacked by showers of arrows and missiles, half naked as they were, never stopped to resist but gave way. Some were killed in flight; others were captured; but some escaped into the hills. Those captured were hairy, not only their heads but the rest of their bodies; their nails were rather like beasts' claws; they used their nails (according to report) as if they were iron tools; with these they tore asunder their fishes, and even the less solid kinds of wood; everything else they cleft with sharp stones; for iron they did not possess. For clothing they wore skins of animals, some even the thick skins of the larger fishes.

XXV. Here the crews beached their ships and repaired such as had suffered. On the sixth day from this they set sail, and after voyaging about three hundred stades they came to a country which was the last point in the territory of the Oreitans: the district was called Malana. Such Oreitans as live inland, away from the sea, dress as the Indians do, and equip themselves similarly for warfare; but their dialect and customs differ. The length of the coasting voyage along the territory of the Arabeis was about a thousand, stades from the point of departure; the length of the Oreitan coast sixteen hundred. As they sailed along the land of India for thence onward the natives are no longer Indians --Nearchus states that their shadows were not cast in the same way; but where they were making for the high seas and steering a southerly course, their shadows appeared to fall southerly too; but whenever the sun was at midday, then everything seemed shadowless. Then such of the stars as they had seen hitherto in the sky, some were completely hidden, others showed themselves low down towards the earth; those they had seen continually before were now observed both setting, and then at once rising again. I think this tale of Nearchus' is likely; since in Syene of Egypt, when the sun is at the summer solstice, people show a well where at midday one sees no shade; and in Meroe, at the same season, no shadows are cast. So it seems reasonable that in India too, since they are far southward, the same natural phenomena may occur, and especially in the Indian Ocean, just because it particularly runs southward. But here I must leave this subject.

XXVI. Next to the Oreitans, more inland, dwelt the Gadrosians, whose country Alexander and his army had much pains in traversing; indeed they suffered more than during all the rest of his expedition: all this I have related in my larger history. Below the Gadrosians, as you follow the actual coast, dwell the people called the Fish-eaters. The fleet sailed past their country. On the first day they unmoored about the second watch, and put in at Bagisara; a distance along the coast of about six hundred stades. There is a safe harbour there, and a village called Pasira, some sixty stades from the sea; the natives about it are called Pasireans. The next day they weighed anchor earlier than usual and sailed round a promontory which ran far seaward, and was high, and precipitous. Then they dug wells; and obtained only a little water, and that poor and for that day they rode at anchor, because there was heavy surf on the beach. Next day they put in at Colta after a voyage of two hundred stades. Thence they departed at dawn, and after voyaging six hundred stades anchored at Calyba. A village is on the shore, a few date-palms grew near it, and there were dates, still green, upon them. About a hundred stades from the beach is an island called Carnine. There the villagers brought gifts to Nearchus, sheep and fishes; the mutton, he says, had a fishy taste, like the flesh of the sea-birds, since even the sheep feed on fish; for there is no grass in the place. However, on the next day they sailed two hundred stades and moored off a beach, and a village about thirty stades from the sea; it was called Cissa, an Carbis was the name of the strip of coast. There they found a few boats, the sort which poor fishermen might use; but the fishermen themselves they did not find, for they had run away as soon as they saw the ships anchoring. There was no corn there, and the army had spent most of its store; but they caught and embarked there some goats, and so sailed away. Rounding a tall cape running some hundred and fifty stades into the sea, they put in at a calm harbour; there was water there, and fishermen dwelt near; the harbour was called Mosarna.

XXVII. Nearchus tells us that from this point a pilot sailed with them, a Gadrosian called Hydraces. He had promised to take them as far as Carmania; from thence on the navigation was not difficult, but the districts were better known, up to the Persian Gulf. From Mosarna they sailed at night, seven hundred and fifty stades, to the beach of Balomus. Thence again to Barna, a village, four hundred stades, where there were many date-palms and a garden; and in the garden grew myrtles and abundant flowers, of which wreaths were woven by the natives. There for the first time they saw garden-trees, and men dwelling there not entirely like animals. Thence they coasted a further two hundred stades and reached Dendrobosa and the ships kept the roadstead at anchor. Thence about midnight they sailed and came to a harbour Cophas, after a voyage of about four hundred stades; here dwelt fishermen, with small and feeble boats; and they did not row with their oars on a rowlock, as the Greeks do, but as you do in a river, propelling the water on this side or that like labourers digging I the soil. At the harbour was abundant pure water. About the first watch they weighed anchor and arrived at Cyiza, after a passage of eight hundred stades, where there was a desert beach and a heavy surf. Here, therefore, they anchored, and each ship took its own meal. Thence they voyaged five hundred stades and arrived at a small town built near the shore on a hill. Nearchus, who imagined that the district must be tilled, told Archias of Pella, son of Anaxidotus, who was sailing with Nearchus, and was a notable Macedonian, that they must surprise the town, since he had no hope that the natives would give the army provisions of their good-will; while he could not capture the town by force, but this would require a siege and much delay; while they in the meanwhile were short of provisions. But that the land did produce corn he could gather from the straw which they saw lying deep near the beach. When they had come to this resolve, Nearchus bade the fleet in general to get ready as if to go to sea; and Archias, in his place, made all ready for the voyage; but Nearchus himself was left behind with a single ship and went off as if to have a look at the town.

XXVIII. As Nearchus approached the walls, the natives brought him, in a friendly way, gifts from the city; tunny-fish baked in earthen pans; for there dwell the westernmost of the Fish-eating tribes, and were the first whom the Greeks had seen cooking their food; and they brought also a few cakes and dates from the palms. Nearchus said that he accepted these gratefully; and desired to visit the town, and they permitted him to enter. But as soon as he passed inside the gates, he bade two of the archers to occupy the postern, while he and two others, and the interpreter, mounted the wall on this side and signalled to Archias and his men as had been arranged: that Nearchus should signal, and Archias understand and do what had been ordered. On seeing the signal the Macedonians beached their ships with all speed; they leapt in haste into the sea, while the natives, astounded at this manoeuvre, ran to their arms. The interpreter with Nearchus cried out that they should give corn to the army, if they wanted to save their city; and the natives replied that they had none, and at the same time attacked the wall. But the archers with Nearchus shooting from above easily held them up. When, however, the natives saw that their town was already occupied and almost on the way to be enslaved, they begged Nearchus to take what corn they had and retire, but not to destroy the town. Nearchus, however, bade Archias to seize the gates and the neighbouring wall; but he sent with the natives some soldiers to see whether they would without any trick reveal their corn. They showed freely their flour, ground down from the dried fish; but only a small quantity of corn and barley. In fact they used as flour what they got from the fish; and loaves of corn flour they used as a delicacy. When, however, they had shown all they had, the Greeks provisioned themselves from what was there, and put to sea, anchoring by a headland which the inhabitants regarded as sacred to the Sun: the headland was called Bageia.

XXIX. Thence, weighing anchor about midnight, they voyaged another thousand stades to Talmena, a harbour giving good anchorage. Thence they went to Canasis, a deserted town, four hundred stades farther; here they found a well sunk; and near by were growing wild date-palms. They cut out the hearts of these and ate them; for the army had run short of food. In fact they were now really distressed by hunger, and sailed on therefore by day and night, and anchored off a desolate shore. But Nearchus, afraid that they would disembark and leave their ships from faint-heartedness, purposely kept the ships in the open roadstead. They sailed thence and anchored at Canate, after a voyage of seven hundred and fifty stades. Here there are a beach and shallow channels. Thence they sailed eight hundred stades, anchoring at Troea; there were small and poverty-stricken villages on the coast. The inhabitants deserted their huts and the Greeks found there a small quantity of corn, and dates from the palms. They slaughtered seven camels which had been left there, and ate the flesh of them. About daybreak they weighed anchor and sailed three hundred stades, and anchored at Dagaseira; there some wandering tribe dwelt. Sailing thence they sailed without stop all night andday, and after a voyage of eleven hundred stades they got past the country of the Fish-eaters, where they had been much distressed by want of food. They did not moor near shore, for there was a long line of surf, but at anchor, in the open. The length of the voyage along the coast of the Fish-eaters is a little above ten thousand stades. These Fish-eaters live on fish; and hence their name; only a few of them fish, for only a few have proper boats and have any skill in the art of catching fish; but for the most part it is the receding tide which provides their catch. Some have made nets also for this kind of fishing; most of them about two stades in length. They make the nets from the bark of the date-palm, twisting the bark like twine. And when the sea recedes and the earth is left, where the earth remains dry it has no fish, as a rule; but where there are hollows, some of the water remains, and in this a large number of fish, mostly small, but some large ones too. They throw their nets over these and so catch them. They eat them raw, just as they take them from the water, that is, the more tender kinds; the larger ones, which are tougher, they dry in the sun till they are quite sere and then pound them and make a flour and bread of them; others even make cakes of this flour. Even their flocks are fed on the fish, dried; for the country has no meadows and produces no grass. They collect also in many places crabs and oysters and shell-fish. There are natural salts in the country; from these they make oil. Those of them who inhabit the desert parts of their country, treeless as it is and with no cultivated parts, find all their sustenance in the fishing but a few of them sow part of their district, using the corn as a relish to the fish, for the fish form their bread. The richest among them have built huts; they collect the bones of any large fish which the sea casts up, and use them in place of beams. Doors they make from any flat bones which they can pick up. But the greater part of them, and the poorer sort, have huts made from the fishes' backbones.

XXX. Large whales live in the outer ocean, and fishes much larger than those in our inland sea. Nearchus states that when they left Cyiza, about daybreak they saw water being blown upwards from the sea as it might be shot upwards by the force of a waterspout. They were astonished, and asked the pilots of the convoy what it might be and how it was caused; they replied that these whales as they rove about the ocean spout up the water to a great height; the sailors, however, were so startled that the oars fell from their hands. Nearchus went and encouraged and cheered them, and whenever he sailed past any vessel, he signalled them to turn the ship's bow on towards the whales as if to give them battle; and raising their battle cry with the sound of the surge to row with rapid strokes and with a great deal of noise. So they all took heart of grace and sailed together according to signal. But when they actually were nearing the monsters, then they shouted with all the power of their throats, and the bugles blared, and the rowers made the utmost splashings with their oars. So the whales, now visible at the bows of the ships, were scared, and dived into the depths; then not long afterwards they came up astern and spouted the sea-water on high. Thereupon joyful applause welcomed this unexpected salvation, and much praise was showered on Nearchus for his courage and prudence. Some of these whales go ashore at different parts of the coast; and when the ebb comes, they are caught in the shallows; and some even were cast ashore high and dry; thus they would perish and decay, and their flesh rotting off them would leave the bones convenient to be used by the natives for their huts. Moreover, the bones in their ribs served for the larger beams for their dwellings; and the smaller for rafters; the jawbones were the doorposts, since many of these whales reached a length of five-and-twenty fathoms.

XXXI. While they were coasting along the territory of the Fish-eaters, they heard a rumour about an island,' which lies some little distance from the mainland in this direction, about a hundred stades, but is uninhabited. The natives said that it was sacred to the Sun and was called Nosala, and that no human being ever of his own will put in there; but that anyone who ignorantly touched there at once disappeared. Nearchus, however, says that one of his galleys with an Egyptian crew was lost with all hands not far from this island, and that the pilots stoutly averred about it that they had touched ignorantly on the island and so had disappeared. But Nearchus sent a thirty-oar to sail round the island, with orders not to put in, but that the crew should shout loudly, while coasting round as near as they dared; and should call on the lost helmsman by name, or any of the crew whose name they knew. As no one answered, he tells us that he himself sailed up to the island, and compelled his unwilling crew to put in; then he went ashore and exploded this island fairy-tale. They heard also another current story about this island, that one of the Nereids dwelt there; but the name of this Nereid was not told. She showed much friendliness to any sailor who approached the island; but then turned him into a fish and threw him into the sea. The Sun then became irritated with the Nereid, and bade her leave the island; and she agreed to remove thence, but begged that the spell on her be removed; the Sun consented; and such human beings as she had turned into fishes he pitied, and turned them again from fishes into human beings, and hence arose the people called Fish-eaters, and so they descended to Alexander's day. Nearchus shows that all this is mere legend; but I have no commendation for his pains and his scholarship; the stories are easy enough to demolish; and I regard it as tedious to relate these old tales and then prove them all false.

XXXII. Beyond these Fish-eaters the Gadrosians inhabit the interior, a poor and sandy territory; this was where Alexander's army and Alexander himself suffered so seriously, as I have already related in my other book. But when the fleet, leaving the Fish-eaters, put in at Carmania, they anchored in the open, at the point where they first touched Carmania; since there was a long and rough line of surf parallel with the coast. From there they sailed no further due west, but took a new course and steered with their bows pointing between north and west. Carmania is better wooded than the country of the Fisheaters, and bears more fruits; it has more grass, and is well watered. They moored at an inhabited place called Badis, in Carmania; with many cultivated trees growing, except the olive tree, and good vines; it also produced corn. Thence they set out and voyaged eight hundred stades, and moored off a desert shore; and they sighted a long cape jutting out far into the ocean; it seemed as if the headland itself was a day's sail away. Those who had knowledge of the district said that this promontory belonged to Arabia, and was called Maceta; and that thence the Assyrians imported cinnamon and other spices. From this beach of which the fleet anchored in the open roadstead, and the promontory, which they sighted opposite them, running out into the sea, the bay (this is my opinion, and Nearchus held the same) runs back into the interior, and would seem to be the Red Sea. When they sighted this cape, Onesicritus bade them take their course from it and sail direct to it, in order not to have the trouble of coasting round the bay. Nearchus, however, replied that Onesicritus was a fool, if he was ignorant of Alexander's purpose in despatching the expedition. It was not because he was unequal to the bringing all his force safely through on foot that he had despatched the fleet; but he desired to reconnoitre the coasts that lay on the line of the voyage, the roadsteads, the islets; to explore thoroughly any bay which appeared, and to learn of any cities which lay on the sea-coast; and to find out what land was fruitful, and what was desert. They must therefore not spoil Alexander's undertaking, especially when they were almost at the close of their toils, and were, moreover, no longer in any difficulty about provisions on their coasting cruise. His own fear was, since the cape ran a long way southward, that they would find the land there waterless and sun-scorched. This view prevailed; and I think that Nearchus evidently saved the expeditionary force by this decision; for it is generally held that this cape and the country about it are entirely desert and quite denuded of water.

XXXIII. They sailed then, leaving this part of the shore, hugging the land; and after voyaging some seven hundred stades they anchored off another beach, called Neoptana. Then at dawn they moved off seaward, and after traversing a hundred stades, they moored by the river Anamis; the district was called Harmozeia. All here was friendly, and produced fruit of all sorts, except that olives did hot grow there. There they disembarked, and had a welcome rest from their long toils, remembering the miseries they had endured by sea and on the coast of the Fish-eaters; recounting one to another the desolate character of the country, the almost bestial nature of the inhabitants, and their own distresses. Some of them advanced some distance inland, breaking away from the main force, some in pursuit of this, and some of that. There a man appeared to them, wearing a Greek cloak, and dressed otherwise in the Greek fashion, and speaking Greek also. Those who first sighted him said that they burst into tears, so strange did it seem after all these miseries to see a Greek, and to hear Greek spoken. They asked whence he came, who he was; and he said that he had become separated from Alexander's camp, and that the camp, and Alexander himself, were not very far distant. Shouting aloud and clapping their hands they brought this man to Nearchus; and he told Nearchus everything, and that the camp and the King himself were distant five days' journey from the coast. He also promised to show Nearchus, the governor of this district and did so; and Nearchus took counsel with him how to march inland to meet the King. For the moment indeed he returned to the ship; but at dawn he had the ships drawn up on shore, to repair any which had been damaged on the voyage; and also because he had determined to leave the greater part of his force behind here. So he had a double stockade built round the ships' station, and a mud wall with a deep trench, beginning from the bank of the river and going on to the beach, where his ships had been dragged ashore.

XXXIV. While Nearchus was busied with these arrangements, the governor of the country, who had been told that Alexander felt the deepest concern about this expedition, took for granted that he would receive some great reward from Alexander if he should be the first to tell him of the safety of the expeditionary force, and that Nearchus would presently appear before the King. So then he hastened by the shortest route and told Alexander: 'See, here is Nearchus coming from the ships.' On this Alexander, though not believing what was told him, yet, as he naturally would be, was pleased by the news itself. But when day succeeded day, and Alexander, reckoning the time when he received the good news, could not any longer believe it, when, moreover, relay sent after relay, to escort Nearchus, either went a part of the route, and meeting no one, came back unsuccessful, or went on further, and missing Nearchus' party, did not themselves return at all, then Alexander bade the man be arrested for spreading a false tale and making things all the worse by this false happiness; and Alexander showed both by his looks and his mind that he was wounded with a very poignant grief. Meanwhile, however, some of those sent to search for Nearchus, who had horses to convey him, and chariots, did meet on the way Nearchus and Archias, and five or six others; that was the number of the party which came inland with him. On this meeting they recognized neither Nearchus nor Archias -- so altered did they appear; with their hair long, unwashed, covered with brine, wizened, pale from sleeplessness and all their other distresses; when, however, they asked where Alexander might be, the search party gave reply as to the locality and passed on. Archias, however, had a happy thought, and said to Nearchus: 'I suspect, Nearchus, that these persons who are traversing the same road as ours through this desert country have been sent for the express purpose of finding us; as for their failure to recognize us, I do not wonder at that; we are in such a sorry plight as to be unrecognizable. Let us tell them who we are and ask them why they come hither.' Nearchus approved; they did ask whither the party was going; and they replied: 'To look for Nearchus and his naval force.' Whereupon, 'Here am I, Nearchus,' said he, 'and here is Archias. Do you lead on; we will make a full report to Alexander about the expeditionary force.'

XXXV. The soldiers took them up in their cars and drove back again. Some of them , anxious to be beforehand with the good news, ran forward and told Alexander: 'Here is Nearchus; and with him Archias and five besides, coming to your presence.' They could not, however, answer any questions about the fleet. Alexander thereupon became possessed of the idea that these few had been miraculously saved, but that his whole army had perished; and did not so much rejoice at the safe arrival of Nearchus and Archias, as he was bitterly pained by the loss of all his force. Hardly had the soldiers told this much, when Nearchus and Archias approached; Alexander could only with great difficulty recognize them; and seeing them as he did long-haired and ill-clad, his grief for the whole fleet and its personnel received even greater surety. Giving his right hand to Nearchus and leading him aside from the Companions and the bodyguard, for a long time he wept; but at length recovering himself he said: 'That you come back safe to us, and Archias here, the entire disaster is tempered to me; but how perished the fleet and the force?' 'Sir,' he replied, 'your ships and men are safe; we are come to tell with our own lips of their safety.' On this Alexander wept the more, since the safety of the force had seemed too good to be true; and then he enquired where the ships were anchored. Nearchus replied: 'They are all drawn up at the mouth of the river Anamis, and are undergoing a refit.' Alexander then called to witness Zeus of the Greeks and the Libyan, Ammon that in good truth he rejoiced more at this news than because he had conquered all Asia since the grief he had felt at the supposed loss of the fleet cancelled all his other good fortune.

XXXVI. The governor of the province, however, whom Alexander had arrested for his false tidings, seeing Nearchus there on the spot, fell at his feet:

'Here,' he said, 'am I, who reported your safe arrival to Alexander; you see in what plight I now am.' So Nearchus begged Alexander to let him go, and he was let off. Alexander then sacrificed thank-offerings for the safety of his host, to Zeus the Saviour, Heracles, Apollo the Averter of Evil, Poseidon and all the gods of the sea; and he held a contest of art and of athletics, and also a procession; Nearchus was in the front row in the procession, and the troops showered on him ribbons and flowers. At the end of the procession Alexander said to Nearchus: 'I will not let you, Nearchus, run risks or suffer distresses again like those of the past; some other admiral shall henceforth command the navy till he brings it into Susa.' Nearchus, however, broke in and said: 'King, I will obey you in all things, as is my bounden duty; but should you desire to do me a gracious favour, do not this thing, but let me be the admiral of your fleet right up to the end, till I bring your ships safe to Susa. Let it not be said that you entrusted me with the difficult and desperate work, but the easy task which leads to ready fame was taken away and put into another's hands.' Alexander checked his speaking further and thanked him warmly to boot; and so he sent him back a signal giving him a force as escort, but a small one, as he was going through friendly territory. Yet his journey to the sea was not untroubled; the natives of the country round about were in possession of the strong places of Carmania, since their satrap had been put to death by Alexander's orders, and his successor appointed, Tlepolemus, had not established his authority. Twice then or even thrice on the one day the party came into conflict with different bodies of natives who kept coming up, and thus without losing any time they only just managed to get safe to the sea-coast. Then Nearchus sacrificed to Zeus the Saviour and held an athletic meeting.

XXXVII. When therefore Nearchus had thus duly performed all his religious duties, they weighed anchor. Coasting along a rough and desert island, they anchored off another island, a large one, and inhabited; this was after a voyage of three hundred stades, from their point of departure. The desert island was called Organa, and that off which they moored Oaracta. Vines grew on it and date-palms; and it produced corn; the length of the island was eight hundred stades. The governor of the island, Mazenes, sailed with them as far as Susa as a volunteer pilot. They said that in this island the tomb of the first chief of this territory was shown; his name was Erythres, and hence came the name of the sea. Thence they weighed anchor and sailed onward, and when they had coasted about two hundred stades along this same island they anchored off it once more and sighted another island, about forty stades from this large one. It was said to be sacred to Poseidon, and not to be trod by foot of man. About dawn they put out to sea, and were met by so violent an ebb that three of the ships ran ashore and were held hard and fast on dry land, and the rest only just sailed through the surf and got safe into deep water. The ships, however, which ran aground were floated off when next flood came, and arrived next day where the main fleet was. They moored at another island, about three hundred stades from the mainland, after a voyage of four hundred stades. Thence they sailed about dawn, and passed on their port side a desert island; its name was Pylora. Then they anchored at Sisidona, a desolate little township, with nothing but water and fish; for the natives here were fish-eaters whether they would or not, because they dwelt in so desolate a territory. Thence they got water, and reached Cape Tarsias, which runs right out into the sea, after a voyage of three hundred stades. Thence they made for Cataea, a desert island, and low-lying; this was said to be sacred to Hermes and Aphrodite; the voyage was of three hundred stades. Every year the natives round about send sheep and goats as sacred to Hermes and Aphrodite, and one could see them, now quite wild from lapse of time and want of handling.

XXXVIII. So far extends Carmania; beyond this is Persia. The length of the voyage along the Carmanian coast is three thousand seven hundred stades. The natives' way of life is like that of the Persians, to whom they are also neighbours; and they wear the same military equipment. The Greeks moved on thence, from the sacred island, and were already coasting along Persian territory; they put in at a place called Eas, where a harbour is formed by a small desert island, which is called Cecandrus; the voyage thither is four hundred stades. At daybreak they sailed to another island, an inhabited one, and anchored there; here, according to Nearchus, there is pearl fishing, as in the Indian Ocean. They sailed along the point of this island, a distance of forty stades, and there moored. Next they anchored off a tall hill, called Ochus, in a safe harbour; fishermen dwelt on its banks. Thence they sailed four hundred and fifty stades, and anchored off Apostana; many boats were anchored there, and there was a village near, about sixty stades from the sea. They weighed anchor at night and sailed thence to a gulf, with a good many villages settled round about. This was a voyage of four hundred stades; and they anchored below a mountain, on which grew many date-pahns and other fruit trees such as flourish in Greece. Thence they um-noored and sailed along to Gogana, about six hundred stades, to an inhabited district; and they anchored off the torrent, called Areon, just at its outlet. The anchorage there was uncomfortable; the entrance was narrow, just at the mouth, since the ebb tide caused shallows in all the neighbourhood of the outlet. After this they anchored again, at another river-mouth, after a voyage of about eight hundred stades. This river was called Sitacus. Even here, however, they did not find a pleasant anchorage; in fact this whole voyage along Persia was shallows, surf, and lagoons. There they found a great supply of corn; brought together there by the King's orders, for their provisioning; there they abode twenty-one days in all; they drew up the ships, and repaired those that had suffered, and the others too they put in order.

XXXIX. Thence they started and reached the city of Hieratis, a populous place. The voyage was of seven hundred and fifty stades; and they anchored in a channel running from the river to the sea and called Heratemis. At sunrise they sailed along the coast to a torrent called Padagrus; the entire district forms. a peninsula. There were many gardens, and all sorts of fruit trees were growing there; the name of the place was Mesambria. From Mesambria they sailed and after a voyage of about two hundred stades anchored at Taoce on the river Granis. Inland from here was a Persian royal residence, about two hundred stades from the mouth of the river. On this voyage, Nearchus says, a great whale was seen, stranded on the shore, and some of the sailors sailed past it and measured it, and said it was of ninety cubits' length. Its hide was scaly, and so thick that it was a cubit in depth; and it had many oysters, limpets, and seaweeds growing on it. Nearchus also says that they could see many dolphins round the whale, and these larger than the Mediterranean dolphins. Going on hence, they put in at the torrent Rogonis, in a good harbour; the length of this voyage was two hundred stades. Thence again they sailed four hundred stades and bivouacked on the side of a torrent; its name was Brizana. Then they found difficult anchorage; there were surf, and shallows, and reefs showing above the sea. But when the flood tide came in, they were able to anchor; when, however,, the tide retired again, the ships were left high and dry. Then when the flood duly returned, they sailed out, and anchored in a river called Oroatis, greatest, according to Nearchus, of all the rivers which on this coast run into the Ocean.

XL. The Persians dwell up to this point and the Susians next to them. Above the Susians lives another independent tribe; these are called Uxians, and in my earlier history I have described them as brigands. The length of the voyage along the Persian coast was four thousand four hundred stades. The Persian land is divided, they say, into three climatic zones. The part which lies by the Red Sea is sandy and sterile, owing to the heat. Then the next zone, northward, has a temperate climate; the country is grassy and has lush meadows and many

vines and all other fruits except the olive; it is rich with all sorts of gardens, has pure rivers running through, and also lakes, and is good both for all sorts of birds which frequent rivers and lakes, and for horses, and also pastures the other domestic animals, and is well wooded, and has plenty of game. The next zone, still going northward, is wintry and snowy, Nearchus. tells us of some envoys from the Black Sea who after quite a short journey met Alexander traversing Persia and caused him no small astonishment; and they explained to Alexander how short the journey was. I have explained that the Uxians are neighbours to the Susians, as the Mardians they also are brigands live next the Persians, and the Cossaeans come next to the Medes. All these tribes Alexander reduced, coming upon them in winter-time, when they thought their country unapproachable. He also founded cities so that they should no longer be nomads but cultivators, and tillers of the ground, and so having a stake in the country might be deterred from raiding one another. From here the convoy passed along the Susian territory. About this part of the voyage Nearchus says he cannot speak with accurate detail, except about the roadsteads and the length of the voyage. This is because the country is for the most part marshy and ruins out well into the sea, with breakers, and is very hard to get good anchorage in. So their voyage was mostly in the open sea. They sailed out, therefore from the mouths of the river, where they had encamped, just on the Persian border, taking on board water for five days; for the pilots said that they would meet no fresh water.

XLI. Then after traversing five hundred stades they anchored in the mouth of a lake, full of fish, called Cataderbis: at the mouth was a small island called Margastana. Thence about daybreak they sailed out and passed the shallows in columns of single ships; the shallows were marked on either side by poles driven down, just as in the strait between the island Leucas and Acarnania signposts have been set up for navigators so that the ships should not ground on the shallows. However, the shallows round Leucas are sandy and render it easy for those aground to get off; but here it is mud on both sides of the channel, both deep and tenacious; once aground there, they could not possibly get of. For the punt-poles sank into the mud and gave them no help, and it proved impossible for the crews to disembark and push the ships off, for they sank up to their breasts in the ooze. Thus then they sailed out with great difficulty and traversed six hundred stades, each crew abiding by its ship; and then they took thought for supper. During the night, however, they were fortunate in reaching deep sailing water and next day also, up to the evening; they sailed nine hundred stades, and anchored in the mouth of the Euphrates near a village of Babylonia, called Didotis; here the merchants gather together frankincense from the neighbouring country and all other sweet-smelling spices which Arabia produces. From the mouth of the Euphrates to Babylon Nearchus says it is a voyage of three thousand three hundred stades.

XLII. There they heard that Alexander was departing towards Susa. They therefore sailed back, in order to sail up the Pasitigris and meet Alexander. So they sailed back, with the land of Susia on their left, and they went along the lake into which the Tigris runs. It flows from Armenia past the city of Ninus, which once was a great and rich city, and so makes the region between itself and the Euphrates; that is why it is called 'Between the Rivers.' The voyage from the lake up to the river itself is six hundred stades, and there is a village of Susia called Aginis; this village is five hundred stades from Susa. The length of the voyage along Susian territory to the mouth of the Pasitigris is two thousand stades. From there they sailed up the Pasitigris through inhabited and prosperous country. Then they had sailed up about a hundred and fifty stades they moored there, waiting for the scouts whom Nearchus had sent to see where the King was. He himself sacrificed to the Saviour gods, and held an athletic meeting, and the whole naval force made merry. And when news was brought that Alexander was now approaching they sailed again up the river; and they moored near the pontoon bridge on which Alexander intended to take his army over to Susa. There the two forces met; Alexander offered sacrifices for his ships and men, come safe back again, and games were held; and whenever Nearchus appeared in the camp, the troops pelted him with ribbons and flowers. There also Nearchus and Leonnatus were crowned by Alexander with a golden crown; Nearchus for the safe conveying of the ships, Leonnatus for the victory he had achieved among the Oreitans and the natives who dwelt next to them. Thus then Alexander received safe back his navy, which had started from the mouths of the Indus.

XLIII. On the right side of the Red Sea beyond Babylonia is the chief part of Arabia, and of this a part comes down to the sea of Phoenicia and Palestinian Syria, but on the west, up to the Mediterranean, the Egyptians are upon the Arabian borders. Along Egypt a gulf running in from the Great Sea makes it clear that by reason of the gulf's joining with the High Seas one might sail round from Babylon into this gulf which runs into Egypt. Yet, in point of fact, no one has yet sailed round this way by reason of the heat and the desert nature of the coasts, only a few people who sailed over the open sea. But those of the army of Cambyses who came safe from Egypt to Susa and those troops who were sent from Ptolemy Lagus to Seleucus Nicator at Babylon through Arabia crossed an isthmus in a period of eight days and passed through a waterless and desert country, riding fast upon camels, carrying water for themselves on their camels, and travelling by night; for during the day they could not come out of shelter by reason of the heat. So far is the region on the other side of this stretch of land, which we have demonstrated to be an isthmus from the Arabian gulf running into the Red Sea, from being inhabited, that its northern parts are quite desert and sandy. Yet from the Arabian gulf which runs along Egypt people have started, and have circumnavigated the greater part of Arabia hoping to reach the sea nearest to Susa and Persia, and thus have sailed so far round the Arabian coast as the amount of fresh water taken aboard their vessels have permitted, and then have returned home again. And those whom Alexander sent from Babylon, in order that, sailing as far as they could on the right of the Red Sea, they might reconnoitre the country on this side, these explorers sighted certain islands lying on their course, and very possibly put in at the mainland of Arabia. But the cape which Nearchus says his party sighted running out into the sea opposite Carmania no one has ever been able to round, and thus turn inwards towards the far side. I am inclined to think that had this been navigable,ft and had there been any passage, it would have been proved navigable, and a passage found, by the indefatigable energy of Alexander. Moreover, Hanno the Libyan started out from Carthage and passed the pillars of Heracles and sailed into the outer Ocean, with Libya on his port side, and he sailed on towards the east, five-and-thirty days all told. But when at last he turned southward, he fell in with every sort of difficulty, want of water, blazing heat, and fiery streams running into the sea. But Cyrene, lying in the more desert parts of Africa, is grassy and fertile and well-watered; it bears all sorts of fruits and animals, right up to the region where the silphium grows; beyond this silphium belt its upper parts are bare and sandy. Here this my history shall cease, which, as well as my other, deals with Alexander of Macedon son of Philip.

\subsection{Indian selections from \emph{Anabasis Alexandri}}

Text: Flavius Arrianus Hist., Phil., Alexandri anabasis (0074: 001)
“Flavii Arriani quae exstant omnia, vol. 1”, Ed. Roos, A.G., Wirth, G.
Leipzig: Teubner, 1967 (1st edn. corr.).
\begin{greek}
Book 3, chapter 8, section 3, line 1

Βεβοηθήκεσαν γὰρ Δαρείῳ Ἰνδῶν τε ὅσοι Βακτρίοις 
ὅμοροι καὶ αὐτοὶ Βάκτριοι καὶ Σογδιανοί· τούτων μὲν 
πάντων ἡγεῖτο Βῆσσος ὁ τῆς Βακτρίων χώρας σατρά-
πης. 



Flavius Arrianus Hist., Phil., Alexandri anabasis 
Book 3, chapter 8, section 4, line 2

            Βαρσαέ<ν>της δὲ Ἀραχωτῶν σατράπης Ἀρα-
χωτούς τε ἦγε καὶ τοὺς ὀρείους Ἰνδοὺς καλουμένους. 



Flavius Arrianus Hist., Phil., Alexandri anabasis 
Book 3, chapter 8, section 6, line 6

                                                      ἐλέ-
γετο δὲ ἡ πᾶσα στρατιὰ ἡ Δαρείου ἱππεῖς μὲν ἐς τε-
τρακισμυρίους, πεζοὶ δὲ ἐς ἑκατὸν μυριάδας, καὶ ἅρ-
ματα δρεπανηφόρα διακόσια, ἐλέφαντες δὲ οὐ πολλοί, 
ἀλλὰ ἐς πεντεκαίδεκα μάλιστα Ἰνδοῖς τοῖς ἐπὶ τάδε 
τοῦ Ἰνδοῦ ἦσαν. 



Flavius Arrianus Hist., Phil., Alexandri anabasis 
Book 3, chapter 11, section 5, line 3

                  κατὰ τὸ μέσον δέ, ἵνα ἦν βασιλεὺς 
Δαρεῖος, οἵ τε συγγενεῖς οἱ βασιλέως ἐτετάχατο καὶ οἱ 
μηλοφόροι Πέρσαι καὶ Ἰνδοὶ καὶ Κᾶρες οἱ ἀνάσπαστοι 
καλούμενοι καὶ οἱ Μάρδοι τοξόται· Οὔξιοι δὲ καὶ Βα-
βυλώνιοι καὶ οἱ πρὸς τῇ ἐρυθρᾷ θαλάσσῃ καὶ Σιττα-
κηνοὶ εἰς βάθος ἐπιτεταγμένοι ἦσαν. 



Flavius Arrianus Hist., Phil., Alexandri anabasis 
Book 3, chapter 13, section 1, line 3

Ὡς δὲ ὁμοῦ ἤδη τὰ στρατόπεδα ἐγίγνετο, ὤφθη 
Δαρεῖός τε καὶ οἱ ἀμφ' αὐτόν, οἵ τε μηλοφόροι Πέρ-
σαι καὶ Ἰνδοὶ καὶ Ἀλβανοὶ καὶ Κᾶρες οἱ ἀνάσπαστοι 
καὶ οἱ Μάρδοι τοξόται, κατ' αὐτὸν Ἀλέξανδρον τεταγ-
μένοι καὶ τὴν ἴλην τὴν βασιλικήν. 



Flavius Arrianus Hist., Phil., Alexandri anabasis 
Book 3, chapter 14, section 5, line 2

καὶ ταύτῃ παραρραγείσης αὐτοῖς τῆς τάξεως κατὰ τὸ 
διέχον διεκπαίουσι τῶν τε Ἰνδῶν τινες καὶ τῆς Περ-
σικῆς ἵππου ὡς ἐπὶ τὰ σκευοφόρα τῶν Μακεδόνων· 
καὶ τὸ ἔργον ἐκεῖ καρτερὸν ἐγίγνετο. 



Flavius Arrianus Hist., Phil., Alexandri anabasis 
Book 3, chapter 15, section 1, line 8

         καὶ πρῶτα μὲν τοῖς φεύγουσι τῶν πολεμίων 
ἱππεῦσι, τοῖς τε Παρθυαίοις καὶ τῶν Ἰνδῶν ἔστιν οἷς 
καὶ Πέρσαις τοῖς πλείστοις καὶ κρατίστοις ἐμβάλλει. 



Flavius Arrianus Hist., Phil., Alexandri anabasis 
Book 3, chapter 25, section 8, line 6

                                  Βαρσαέντης δέ, ὃς τότε 
κατεῖχε τὴν χώραν, εἷς ὢν τῶν ξυνεπιθεμένων Δαρείῳ 
ἐν τῇ φυγῇ, προσιόντα Ἀλέξανδρον μαθὼν ἐς Ἰνδοὺς 
τοὺς ἐπὶ τάδε τοῦ Ἰνδοῦ ποταμοῦ ἔφυγε. 



Flavius Arrianus Hist., Phil., Alexandri anabasis 
Book 3, chapter 25, section 8, line 8

                                                  ξυλλαβόντες 
δὲ αὐτὸν οἱ Ἰνδοὶ παρ' Ἀλέξανδρον ἀπέστειλαν, καὶ ἀπο-
θνήσκει πρὸς Ἀλεξάνδρου τῆς ἐς Δαρεῖον ἀδικίας ἕνεκα. 



Flavius Arrianus Hist., Phil., Alexandri anabasis 
Book 3, chapter 28, section 1, line 5

                                                   ἐπῆλθε 
δὲ καὶ τῶν Ἰνδῶν τοὺς προσχώρους Ἀραχώταις. 



Flavius Arrianus Hist., Phil., Alexandri anabasis 
Book 3, chapter 29, section 2, line 5

                                                      ὁ δὲ 
Ὄξος ῥέει μὲν ἐκ τοῦ ὄρους τοῦ Καυκάσου, ἔστι δὲ 
ποταμῶν μέγιστος τῶν ἐν τῇ Ἀσίᾳ, ὅσους γε δὴ Ἀλέξ-
ανδρος καὶ οἱ ξὺν Ἀλεξάνδρῳ ἐπῆλθον, πλὴν τῶν Ἰν-
δῶν ποταμῶν· οἱ δὲ Ἰνδοὶ πάντων ποταμῶν μέγιστοί 
εἰσιν. 



Flavius Arrianus Hist., Phil., Alexandri anabasis 
Book 4, chapter 15, section 6, line 2

                                                    αὑτῷ 
δὲ τὰ Ἰνδῶν ἔφη ἐν τῷ τότε μέλειν. 



Flavius Arrianus Hist., Phil., Alexandri anabasis 
Book 4, chapter 22, section 3, line 2

Ἐκ Βάκτρων δὲ ἐξήκοντος ἤδη τοῦ ἦρος ἀναλαβὼν 
τὴν στρατιὰν προὐχώρει ὡς ἐπ' Ἰνδούς, Ἀμύνταν 
ἀπολιπὼν ἐν τῇ χώρᾳ τῶν Βακτρίων καὶ ξὺν αὐτῷ 
ἱππέας μὲν τρισχιλίους καὶ πεντακοσίους, πεζοὺς δὲ 
μυρίους. 



Flavius Arrianus Hist., Phil., Alexandri anabasis 
Book 4, chapter 22, section 6, line 4

                                                       ἀφ-
ικόμενος δὲ ἐς Νίκαιαν πόλιν καὶ τῇ Ἀθηνᾷ θύσας 
προὐχώρει ὡς ἐπὶ τὸν Κωφῆνα, προπέμψας κήρυκα 
ὡς Ταξίλην τε καὶ τοὺς ἐπὶ τάδε τοῦ Ἰνδοῦ ποταμοῦ, 
κελεύσας ἀπαντᾶν ὅπως ἂν ἑκάστοις προχωρῇ. 



Flavius Arrianus Hist., Phil., Alexandri anabasis 
Book 4, chapter 22, section 6, line 7

                                                      καὶ 
Ταξίλης τε καὶ οἱ ἄλλοι ὕπαρχοι ἀπήντων, δῶρα τὰ 
μέγιστα παρ' Ἰνδοῖς νομιζόμενα κομίζοντες, καὶ τοὺς 
ἐλέφαντας δώσειν ἔφασκον τοὺς παρὰ σφίσιν ὄντας, 
ἀριθμὸν ἐς πέντε καὶ εἴκοσιν. 



Flavius Arrianus Hist., Phil., Alexandri anabasis 
Book 4, chapter 22, section 7, line 3

Ἔνθα δὴ διελὼν τὴν στρατιὰν Ἡφαιστίωνα μὲν 
καὶ Περδίκκαν ἐκπέμπει ἐς τὴν Πευκελαῶτιν χώραν 
ὡς ἐπὶ τὸν Ἰνδὸν ποταμόν, ἔχοντας τήν τε Γοργίου 
τάξιν καὶ Κλείτου καὶ Μελεάγρου καὶ τῶν ἑταίρων 
ἱππέων τοὺς ἡμίσεας καὶ τοὺς μισθοφόρους ἱππέας 
ξύμπαντας, προστάξας τά τε κατὰ τὴν ὁδὸν χωρία ἢ 
βίᾳ ἐξαιρεῖν ἢ ὁμολογίᾳ παρίστασθαι καὶ ἐπὶ τὸν 
Ἰνδὸν ποταμὸν ἀφικομένους παρασκευάζειν ὅσα ἐς τὴν 
διάβασιν τοῦ ποταμοῦ ξύμφορα. 



Flavius Arrianus Hist., Phil., Alexandri anabasis 
Book 4, chapter 22, section 8, line 2

                                                  καὶ οὗτοι 
ὡς ἀφίκοντο πρὸς τὸν Ἰνδὸν ποταμόν, ἔπρασσον ὅσα 
ἐξ Ἀλεξάνδρου ἦν τεταγμένα. 



Flavius Arrianus Hist., Phil., Alexandri anabasis 
Book 4, chapter 24, section 3, line 1

Τὸν δὲ ἡγεμόνα αὐτὸν τῶν ταύτῃ Ἰνδῶν Πτολε-
μαῖος ὁ Λάγου πρός τινι ἤδη γηλόφῳ ὄντα κατιδὼν 
καὶ τῶν ὑπασπιστῶν ἔστιν οὓς ἀμφ' αὐτὸν ξὺν πολὺ 
ἐλάττοσιν αὐτὸς ὢν ὅμως ἐδίωκεν ἔτι ἐκ τοῦ ἵππου· 
ὡς δὲ χαλεπὸς ὁ γήλοφος τῷ ἵππῳ ἀναδραμεῖν ἦν, 
τοῦτον μὲν αὐτοῦ καταλείπει παραδούς τινι τῶν 
ὑπασπιστῶν ἄγειν, αὐτὸς δὲ ὡς εἶχε πεζὸς τῷ Ἰνδῷ 
εἵπετο. 



Flavius Arrianus Hist., Phil., Alexandri anabasis 
Book 4, chapter 24, section 4, line 3

              καὶ ὁ μὲν Ἰνδὸς τοῦ Πτολεμαίου διὰ τοῦ 
θώρακος παίει ἐκ χειρὸς ἐς τὸ στῆθος ξυστῷ μακρῷ, 
καὶ ὁ θώραξ ἔσχε τὴν πληγήν· Πτολεμαῖος δὲ τὸν 
μηρὸν διαμπὰξ βαλὼν τοῦ Ἰνδοῦ καταβάλλει τε καὶ 
σκυλεύει αὐτόν. 



Flavius Arrianus Hist., Phil., Alexandri anabasis 
Book 4, chapter 24, section 5, line 8

καὶ οὗτοι ἐπιγενόμενοι μόγις ἐξέωσαν τοὺς Ἰνδοὺς ἐς 
τὰ ὄρη καὶ τοῦ νεκροῦ ἐκράτησαν. 



Flavius Arrianus Hist., Phil., Alexandri anabasis 
Book 4, chapter 25, section 3, line 3

                          καὶ γίγνεται καὶ τούτοις μάχη 
καρτερὰ τοῦ χωρίου τῇ χαλεπότητι καὶ ὅτι οὐ κατὰ τοὺς 
ἄλλους τοὺς ταύτῃ βαρβάρους οἱ Ἰνδοί, ἀλλὰ πολὺ δή 
τι ἀλκιμώτατοι τῶν προσχώρων εἰσίν. 



Flavius Arrianus Hist., Phil., Alexandri anabasis 
Book 4, chapter 26, section 1, line 4

                                ὡς δὲ προσῆγεν ἤδη 
τοῖς τείχεσι, θαρρήσαντες οἱ βάρβαροι τοῖς μισθοφόροις 
τοῖς ἐκ τῶν πρόσω Ἰνδῶν, ἦσαν γὰρ οὗτοι ἐς ἑπτα-
κισχιλίους, ὡς στρατοπεδευομένους εἶδον τοὺς Μακε-
δόνας, δρόμῳ ἐπ' αὐτοὺς ᾔεσαν. 



Flavius Arrianus Hist., Phil., Alexandri anabasis 
Book 4, chapter 26, section 4, line 4

                                                         οἱ   
δὲ Ἰνδοὶ τῷ τε παραλόγῳ ἐκπλαγέντες καὶ ἅμα ἐν 
χερσὶ γεγενημένης τῆς μάχης ἐγκλίναντες ἔφευγον ἐς 
τὴν πόλιν. 



Flavius Arrianus Hist., Phil., Alexandri anabasis 
Book 4, chapter 26, section 5, line 4

              ἐπαγαγὼν δὲ τὰς μηχανὰς τῇ ὑστεραίᾳ 
τῶν μὲν τειχῶν τι εὐμαρῶς κατέσεισε, βιαζομένους δὲ 
ταύτῃ τοὺς Μακεδόνας ᾗ παρέρρηκτο τοῦ τείχους οὐκ 
ἀτόλμως οἱ Ἰνδοὶ ἠμύνοντο, ὥστε ταύτῃ μὲν τῇ ἡμέρᾳ 
ἀνεκαλέσατο τὴν στρατιάν. 



Flavius Arrianus Hist., Phil., Alexandri anabasis 
Book 4, chapter 26, section 5, line 9

                              τῇ δὲ ὑστεραίᾳ τῶν τε 
Μακεδόνων αὐτῶν ἡ προσβολὴ καρτερωτέρα ἐγίγνετο 
καὶ πύργος ἐπήχθη ξύλινος τοῖς τείχεσιν, ὅθεν ἐκ-
τοξεύοντες οἱ τοξόται καὶ βέλη ἀπὸ μηχανῶν ἀφιέμενα 
ἀνέστελλεν ἐπὶ πολὺ τοὺς Ἰνδούς. 



Flavius Arrianus Hist., Phil., Alexandri anabasis 
Book 4, chapter 27, section 2, line 1

Καὶ οἱ Ἰνδοί, ἕως μὲν αὐτοῖς ὁ ἡγεμὼν τοῦ 
χωρίου περιῆν, ἀπεμάχοντο καρτερῶς· ὡς δὲ βέλει 
ἀπὸ μηχανῆς τυπεὶς ἀποθνήσκει ἐκεῖνος, αὐτῶν τε 
οἱ μέν τινες πεπτωκότες ἐν τῇ ξυνεχεῖ πολιορκίᾳ, οἱ 
πολλοὶ δὲ τραυματίαι τε καὶ ἀπόμαχοι ἦσαν, ἐπεκηρυ-
κεύοντο πρὸς Ἀλέξανδρον. 



Flavius Arrianus Hist., Phil., Alexandri anabasis 
Book 4, chapter 27, section 3, line 3

                              τῷ δὲ ἀσμένῳ γίνεται 
ἄνδρας ἀγαθοὺς διασῶσαι· καὶ ξυμβαίνει ἐπὶ τῷδε 
Ἀλέξανδρος τοῖς μισθοφόροις Ἰνδοῖς ὡς καταταχθέντας 
ἐς τὴν ἄλλην στρατιὰν ξὺν αὑτῷ στρατεύεσθαι. 



Flavius Arrianus Hist., Phil., Alexandri anabasis 
Book 4, chapter 27, section 3, line 9

                                    νυκτὸς δὲ ἐπενόουν 
δρασμῷ διαχρησάμενοι ἐς τὰ σφέτερα ἤθη ἀπαναστῆναι 
οὐκ ἐθέλοντες ἐναντία αἴρεσθαι τοῖς ἄλλοις Ἰνδοῖς 
ὅπλα. 



Flavius Arrianus Hist., Phil., Alexandri anabasis 
Book 4, chapter 27, section 4, line 3

        καὶ ταῦτα ὡς ἐξηγγέλθη Ἀλεξάνδρῳ, περιστήσας 
τῆς νυκτὸς τῷ γηλόφῳ τὴν στρατιὰν πᾶσαν κατα-
κόπτει τοὺς Ἰνδοὺς ἐν μέσῳ ἀπολαβών, τήν τε πόλιν 
αἱρεῖ κατὰ κράτος ἐρημωθεῖσαν τῶν προμαχομένων, 
καὶ τὴν μητέρα τὴν Ἀσσακάνου καὶ τὴν παῖδα ἔλαβεν. 



Flavius Arrianus Hist., Phil., Alexandri anabasis 
Book 4, chapter 28, section 2, line 1

              εἰ μὲν δὴ καὶ ἐς Ἰνδοὺς ἀφίκετο ὁ Ἡρακλῆς 
ὁ Θηβαῖος ἢ ὁ Τύριος ἢ ὁ Αἰγύπτιος ἐς οὐδέτερα ἔχω 
ἰσχυρίσασθαι· μᾶλλον δὲ δοκῶ ὅτι οὐκ ἀφίκετο, ἀλλὰ 
πάντα γὰρ ὅσα χαλεπὰ οἱ ἄνθρωποι ἐς τοσόνδε ἄρα 
αὔξουσιν αὐτῶν τὴν χαλεπότητα, ὡς καὶ τῷ Ἡρακλεῖ 
ἂν ἄπορα γενέσθαι μυθεύειν. 



Flavius Arrianus Hist., Phil., Alexandri anabasis 
Book 4, chapter 28, section 5, line 4

                                           καὶ οἱ ἀμφὶ Ἡφαι-
στίωνά τε καὶ Περδίκκαν αὐτῷ ἄλλην πόλιν ἐκτειχί-
σαντες, Ὀροβάτις ὄνομα τῇ πόλει ἦν, καὶ φρουρὰν 
καταλιπόντες ὡς ἐπὶ τὸν Ἰνδὸν ποταμὸν ᾔεσαν· ὡς 
δὲ ἀφίκοντο, ἔπρασσον ἤδη ὅσα ἐς τὸ ζεῦξαι τὸν Ἰνδὸν 
ὑπὸ Ἀλεξάνδρου ἐτέτακτο. 



Flavius Arrianus Hist., Phil., Alexandri anabasis 
Book 4, chapter 28, section 6, line 2

Ἀλέξανδρος δὲ τῆς μὲν χώρας τῆς ἐπὶ τάδε τοῦ 
Ἰνδοῦ ποταμοῦ σατράπην κατέστησε Νικάνορα τῶν 
ἑταίρων. 



Flavius Arrianus Hist., Phil., Alexandri anabasis 
Book 4, chapter 28, section 6, line 3

           αὐτὸς δὲ τὰ μὲν πρῶτα ὡς ἐπὶ τὸν Ἰνδὸν 
ποταμὸν ἦγε, καὶ πόλιν τε Πευκελαῶτιν οὐ πόρρω 
τοῦ Ἰνδοῦ ᾠκισμένην ὁμολογίᾳ παρεστήσατο καὶ ἐν 
αὐτῇ φρουρὰν καταστήσας τῶν Μακεδόνων καὶ Φίλιππον 
ἐπὶ τῇ φρουρᾷ ἡγεμόνα, ὁ δὲ καὶ ἄλλα προσηγάγετο 
μικρὰ πολίσματα πρὸς τῷ Ἰνδῷ ποταμῷ ᾠκισμένα. 



Flavius Arrianus Hist., Phil., Alexandri anabasis 
Book 4, chapter 29, section 3, line 5

ὡς δὲ Ἀλεξάνδρῳ ἄπορον τὴν προσβολὴν κατέμαθον 
οἱ βάρβαροι, ἀναστρέψαντες τοῖς ἀμφὶ Πτολεμαῖον 
αὐτοὶ προσέβαλλον· καὶ γίγνεται αὐτῶν τε καὶ τῶν 
Μακεδόνων μάχη καρτερά, τῶν μὲν διασπάσαι τὸν 
χάρακα σπουδὴν ποιουμένων, τῶν Ἰνδῶν, Πτολεμαίου 
δὲ διαφυλάξαι τὸ χωρίον· καὶ μεῖον σχόντες οἱ βάρ-
βαροι ἐν τῷ ἀκροβολισμῷ νυκτὸς ἐπιγενομένης ἀπ-
εχώρησαν. 



Flavius Arrianus Hist., Phil., Alexandri anabasis 
Book 4, chapter 29, section 4, line 1

Ἀλέξανδρος δὲ τῶν Ἰνδῶν τινα τῶν αὐτομόλων 
πιστόν τε ἄλλως καὶ τῶν χωρίων δαήμονα ἐπιλεξά-
μενος πέμπει παρὰ Πτολεμαῖον τῆς νυκτὸς, γράμ-
ματα φέροντα τὸν Ἰνδόν, ἵνα ἐνεγέγραπτο, ἐπειδὰν 
αὐτὸς προσβάλλῃ τῇ πέτρᾳ, τὸν δὲ ἐπιέναι τοῖς βαρ-
βάροις κατὰ τὸ ὄρος μηδὲ ἀγαπᾶν ἐν φυλακῇ ἔχοντα 
τὸ χωρίον, ὡς ἀμφοτέρωθεν βαλλομένους τοὺς Ἰνδοὺς 
ἀμφιβόλους γίγνεσθαι. 



Flavius Arrianus Hist., Phil., Alexandri anabasis 
Book 4, chapter 29, section 6, line 2

                      ἔστε μὲν γὰρ ἐπὶ μεσημβρίαν 
ξυνειστήκει καρτερὰ μάχη τοῖς τε Ἰνδοῖς καὶ τοῖς 
Μακεδόσιν, τῶν μὲν ἐκβιαζομένων ἐς τὴν πρόσβασιν, 
τῶν δὲ βαλλόντων ἀνιόντας· ὡς δὲ οὐκ ἀνίεσαν οἱ 
Μακεδόνες, ἄλλοι ἐπ' ἄλλοις ἐπιόντες, οἱ δὲ πρόσθεν 
ἀναπαυόμενοι, μόγις δὴ ἀμφὶ δείλην ἐκράτησαν τῆς   
παρόδου καὶ ξυνέμιξαν τοῖς ξὺν Πτολεμαίῳ. 



Flavius Arrianus Hist., Phil., Alexandri anabasis 
Book 4, chapter 30, section 1, line 3

                     ἐς δὲ τὴν ὑστεραίαν οἵ τε σφεν-
δονῆται σφενδονῶντες ἐς τοὺς Ἰνδοὺς ἐκ τοῦ ἤδη 
κεχωσμένου καὶ ἀπὸ τῶν μηχανῶν βέλη ἀφιέμενα 
ἀνέστελλε τῶν Ἰνδῶν τὰς ἐκδρομὰς τὰς ἐπὶ τοὺς 
χωννύοντας. 



Flavius Arrianus Hist., Phil., Alexandri anabasis 
Book 4, chapter 30, section 2, line 1

Οἱ δὲ Ἰνδοὶ πρός τε τὴν ἀδιήγητον τόλμαν τῶν 
ἐς τὸν γήλοφον βιασαμένων Μακεδόνων ἐκπλαγέντες   
καὶ τὸ χῶμα ξυνάπτον ἤδη ὁρῶντες, τοῦ μὲν ἀπο-
μάχεσθαι ἔτι ἀπείχοντο, πέμψαντες δὲ κήρυκας σφῶν 
παρὰ Ἀλέξανδρον ἐθέλειν ἔφασκον ἐνδοῦναι τὴν πέτραν, 
εἴ σφισι σπένδοιτο. 



Flavius Arrianus Hist., Phil., Alexandri anabasis 
Book 4, chapter 30, section 4, line 8

                             εἴχετό τε Ἀλεξάνδρῳ ἡ 
πέτρα ἡ τῷ Ἡρακλεῖ ἄπορος γενομένη καὶ ἔθυεν ἐπ' 
αὐτῇ Ἀλέξανδρος καὶ κατεσκεύασε φρούριον, παραδοὺς 
Σισικόττῳ ἐπιμελεῖσθαι τῆς φρουρᾶς, ὃς ἐξ Ἰνδῶν 
μὲν πάλαι ηὐτομολήκει ἐς Βάκτρα παρὰ Βῆσσον, Ἀλεξ-
άνδρου δὲ κατασχόντος τὴν χώραν τὴν Βακτρίαν 
ξυνεστράτευέ τε αὐτῷ καὶ πιστὸς ἐς τὰ μάλιστα 
ἐφαίνετο. 



Flavius Arrianus Hist., Phil., Alexandri anabasis 
Book 4, chapter 30, section 7, line 1

Αὐτὸς δὲ ὡς ἐπὶ τὸν Ἰνδὸν ποταμὸν ἤδη ἦγε, καὶ 
ἡ στρατιὰ αὐτῷ ὡδοποίει τὸ πρόσω ἰοῦσα ἄπορα 
ἄλλως ὄντα τὰ ταύτῃ χωρία. 



Flavius Arrianus Hist., Phil., Alexandri anabasis 
Book 4, chapter 30, section 7, line 5

                                   ἐνταῦθα ξυλλαμβάνει 
ὀλίγους τῶν βαρβάρων, καὶ παρὰ τούτων ἔμαθεν, ὅτι 
οἱ μὲν ἐν τῇ χώρᾳ Ἰνδοὶ παρὰ Ἀβισάρῃ ἀποπεφευγότες 
εἶεν, τοὺς δὲ ἐλέφαντας ὅτι αὐτοῦ κατέλιπον νέμεσθαι 
πρὸς τῷ ποταμῷ τῷ Ἰνδῷ· καὶ τούτους ἡγήσασθαί οἱ 
τὴν ὁδὸν ἐκέλευσεν ὡς ἐπὶ τοὺς ἐλέφαντας. 



Flavius Arrianus Hist., Phil., Alexandri anabasis 
Book 4, chapter 30, section 8, line 2

                                                     εἰσὶ δὲ 
Ἰνδῶν πολλοὶ κυνηγέται τῶν ἐλεφάντων, καὶ τούτους 
σπουδῇ ἀμφ' αὑτὸν εἶχεν Ἀλέξανδρος, καὶ τότε ἐθήρα 
ξὺν τούτοις τοὺς ἐλέφαντας· καὶ δύο μὲν αὐτῶν 
ἀπόλλυνται κατὰ κρημνοῦ σφᾶς ῥίψαντες ἐν τῇ διώξει,   
οἱ δὲ ἄλλοι ξυλληφθέντες ἔφερόν τε τοὺς ἀμβάτας 
καὶ τῇ στρατιᾷ ξυνετάσσοντο. 



Flavius Arrianus Hist., Phil., Alexandri anabasis 
Book 4, chapter 30, section 9, line 4

                                                      καὶ 
αὗται κατὰ τὸν Ἰνδὸν ποταμὸν ἤγοντο ὡς ἐπὶ τὴν 
γέφυραν, ἥντινα Ἡφαιστίων καὶ Περδίκκας αὐτῷ ἐξ-
ῳκοδομηκότες πάλαι ἦσαν. 



Flavius Arrianus Hist., Phil., Alexandri anabasis 
Book 5, chapter 1, section 1, line 2

ΑΡΡΙΑΝΟΥ 
ΑΛΕΞΑΝΔΡΟΥ ΑΝΑΒΑΣΕΩΣ 
ΒΙΒΛΙΟΝ ΠΕΜΠΤΟΝ


 Ἐν δὲ τῇ χώρᾳ ταύτῃ, ἥντινα μεταξὺ τοῦ τε 
Κωφῆνος καὶ τοῦ Ἰνδοῦ ποταμοῦ ἐπῆλθεν Ἀλέξαν-
δρος, καὶ Νῦσαν πόλιν ᾠκίσθαι λέγουσι· τὸ δὲ κτίσμα 
εἶναι Διονύσου· Διόνυσον δὲ κτίσαι τὴν Νῦσαν ἐπεί 
τε Ἰνδοὺς ἐχειρώσατο, ὅστις δὴ οὗτος ὁ Διόνυσος καὶ 
ὁπότε ἢ ὅθεν ἐπ' Ἰνδοὺς ἐστράτευσεν· οὐ γὰρ ἔχω 
συμβαλεῖν εἰ ὁ Θηβαῖος Διόνυσος [ὃς] ἐκ Θηβῶν ἢ 
ἐκ Τμώλου τοῦ Λυδίου ὁρμηθεὶς ἐπὶ Ἰνδοὺς ἧκε 
στρατιὰν ἄγων, τοσαῦτα μὲν ἔθνη μάχιμα καὶ ἄγνωστα 
τοῖς τότε Ἕλλησιν ἐπελθών, οὐδὲν δὲ αὐτῶν ἄλλο ὅτι 
μὴ τὸ Ἰνδῶν βίᾳ χειρωσάμενος· πλήν γε δὴ ὅτι οὐκ 




Flavius Arrianus Hist., Phil., Alexandri anabasis 
Book 5, chapter 1, section 2, line 3

ΑΛΕΞΑΝΔΡΟΥ ΑΝΑΒΑΣΕΩΣ 
ΒΙΒΛΙΟΝ ΠΕΜΠΤΟΝ


 Ἐν δὲ τῇ χώρᾳ ταύτῃ, ἥντινα μεταξὺ τοῦ τε 
Κωφῆνος καὶ τοῦ Ἰνδοῦ ποταμοῦ ἐπῆλθεν Ἀλέξαν-
δρος, καὶ Νῦσαν πόλιν ᾠκίσθαι λέγουσι· τὸ δὲ κτίσμα 
εἶναι Διονύσου· Διόνυσον δὲ κτίσαι τὴν Νῦσαν ἐπεί 
τε Ἰνδοὺς ἐχειρώσατο, ὅστις δὴ οὗτος ὁ Διόνυσος καὶ 
ὁπότε ἢ ὅθεν ἐπ' Ἰνδοὺς ἐστράτευσεν· οὐ γὰρ ἔχω 
συμβαλεῖν εἰ ὁ Θηβαῖος Διόνυσος [ὃς] ἐκ Θηβῶν ἢ 
ἐκ Τμώλου τοῦ Λυδίου ὁρμηθεὶς ἐπὶ Ἰνδοὺς ἧκε 
στρατιὰν ἄγων, τοσαῦτα μὲν ἔθνη μάχιμα καὶ ἄγνωστα 
τοῖς τότε Ἕλλησιν ἐπελθών, οὐδὲν δὲ αὐτῶν ἄλλο ὅτι 
μὴ τὸ Ἰνδῶν βίᾳ χειρωσάμενος· πλήν γε δὴ ὅτι οὐκ 
ἀκριβῆ ἐξεταστὴν χρὴ εἶναι τῶν ὑπὲρ τοῦ θείου ἐκ 
παλαιοῦ μεμυθευμένων. 



Flavius Arrianus Hist., Phil., Alexandri anabasis 
Book 5, chapter 1, section 2, line 6

Κωφῆνος καὶ τοῦ Ἰνδοῦ ποταμοῦ ἐπῆλθεν Ἀλέξαν-
δρος, καὶ Νῦσαν πόλιν ᾠκίσθαι λέγουσι· τὸ δὲ κτίσμα 
εἶναι Διονύσου· Διόνυσον δὲ κτίσαι τὴν Νῦσαν ἐπεί 
τε Ἰνδοὺς ἐχειρώσατο, ὅστις δὴ οὗτος ὁ Διόνυσος καὶ 
ὁπότε ἢ ὅθεν ἐπ' Ἰνδοὺς ἐστράτευσεν· οὐ γὰρ ἔχω 
συμβαλεῖν εἰ ὁ Θηβαῖος Διόνυσος [ὃς] ἐκ Θηβῶν ἢ 
ἐκ Τμώλου τοῦ Λυδίου ὁρμηθεὶς ἐπὶ Ἰνδοὺς ἧκε 
στρατιὰν ἄγων, τοσαῦτα μὲν ἔθνη μάχιμα καὶ ἄγνωστα 
τοῖς τότε Ἕλλησιν ἐπελθών, οὐδὲν δὲ αὐτῶν ἄλλο ὅτι 
μὴ τὸ Ἰνδῶν βίᾳ χειρωσάμενος· πλήν γε δὴ ὅτι οὐκ 
ἀκριβῆ ἐξεταστὴν χρὴ εἶναι τῶν ὑπὲρ τοῦ θείου ἐκ 
παλαιοῦ μεμυθευμένων. 



Flavius Arrianus Hist., Phil., Alexandri anabasis 
Book 5, chapter 1, section 5, line 3

                                                      Διό-
νυσος γὰρ ἐπειδὴ χειρωσάμενος τὸ Ἰνδῶν ἔθνος ἐπὶ 
θάλασσαν ὀπίσω κατῄει τὴν Ἑλληνικήν, ἐκ τῶν ἀπο-
μάχων στρατιωτῶν, οἳ δὴ αὐτῷ καὶ βάκχοι ἦσαν, κτίζει 
τὴν πόλιν τήνδε μνημόσυνον τῆς αὑτοῦ πλάνης τε καὶ 
νίκης τοῖς ἔπειτα ἐσόμενον, καθάπερ οὖν καὶ σὺ αὐτὸς 
Ἀλεξάνδρειάν τε ἔκτισας τὴν πρὸς Καυκάσῳ ὄρει καὶ 
ἄλλην Ἀλεξάνδρειαν ἐν τῇ Αἰγυπτίων γῇ, καὶ ἄλλας 
πολλὰς τὰς μὲν ἔκτικας ἤδη, τὰς δὲ καὶ κτίσεις ἀνὰ 
χρόνον, οἷα δὴ πλείονα Διονύσου ἔργα ἀποδειξάμενος. 



Flavius Arrianus Hist., Phil., Alexandri anabasis 
Book 5, chapter 1, section 6, line 9

                                      καὶ ἐκ τούτου ἐλευ-
θέραν τε οἰκοῦμεν τὴν Νῦσαν καὶ αὐτοὶ αὐτόνομοι   
καὶ ἐν κόσμῳ πολιτεύοντες· τῆς δὲ ἐκ Διονύσου 
οἰκίσεως καὶ τόδε σοι γενέσθω τεκμήριον· κιττὸς γὰρ 
οὐκ ἄλλῃ τῆς Ἰνδῶν γῆς φυόμενος παρ' ἡμῖν φύεται. 



Flavius Arrianus Hist., Phil., Alexandri anabasis 
Book 5, chapter 2, section 6, line 3

                             καὶ τοὺς Μακεδόνας ἡδέως 
τὸν κισσὸν ἰδόντας, οἷα δὴ διὰ μακροῦ ὀφθέντα (οὐ 
γὰρ εἶναι ἐν τῇ Ἰνδῶν χώρᾳ κισσόν, οὐδὲ ἵναπερ 
αὐτοῖς ἄμπελοι ἦσαν) στεφάνους σπουδῇ ἀπ' αὐτοῦ 
ποιεῖσθαι, ὡς καὶ στεφανώσασθαι εἶχον, ἐφυμνοῦντας 
τὸν Διόνυσόν τε καὶ τὰς ἐπωνυμίας τοῦ θεοῦ ἀνα-
καλοῦντας. 



Flavius Arrianus Hist., Phil., Alexandri anabasis 
Book 5, chapter 3, section 3, line 3

        τὸν δὲ Καύκασον τὸ ὄρος ἐκ τοῦ Πόντου ἐς 
τὰ πρὸς ἕω μέρη τῆς γῆς καὶ τὴν Παραπαμισαδῶν 
χώραν ὡς ἐπὶ Ἰνδοὺς μετάγειν τῷ λόγῳ τοὺς Μακε-
δόνας, Παραπάμισον ὄντα τὸ ὄρος αὐτοὺς καλοῦντας 
Καύκασον τῆς Ἀλεξάνδρου ἕνεκα δόξης, ὡς ὑπὲρ τὸν 
Καύκασον ἄρα ἐλθόντα Ἀλέξανδρον. 



Flavius Arrianus Hist., Phil., Alexandri anabasis 
Book 5, chapter 3, section 4, line 2

                                          ἔν τε αὐτῇ τῇ 
Ἰνδῶν γῇ βοῦς ἰδόντας ἐγκεκαυμένας ῥόπαλον τεκ-
μηριοῦσθαι ἐπὶ τῷδε, ὅτι Ἡρακλῆς ἐς Ἰνδοὺς ἀφίκετο. 



Flavius Arrianus Hist., Phil., Alexandri anabasis 
Book 5, chapter 3, section 5, line 1

Ἀλέξανδρος δὲ ὡς ἀφίκετο ἐπὶ τὸν Ἰνδὸν ποταμόν, 
καταλαμβάνει γέφυράν τε ἐπ' αὐτῷ πεποιημένην πρὸς 
Ἡφαιστίωνος καὶ πλοῖα πολλὰ μὲν σμικρότερα, δύο 
δὲ τριακοντόρους, καὶ παρὰ Ταξίλου τοῦ Ἰνδοῦ δῶρα 
ἥκοντα ἀργυρίου μὲν τάλαντα ἐς διακόσια, ἱερεῖα δὲ 
βοῦς μὲν τρισχιλίας, πρόβατα δὲ ὑπὲρ μύρια, ἐλέφαν-
τας δὲ ἐς τριάκοντα· καὶ ἱππεῖς δὲ ἑπτακόσιοι αὐτῷ 
Ἰνδῶν ἐς ξυμμαχίαν παρὰ Ταξίλου ἧκον· καὶ τὴν πόλιν 
Τάξιλα, τὴν μεγίστην μεταξὺ Ἰνδοῦ τε ποταμοῦ καὶ 
Ὑδάσπου, ὅτι αὐτῷ Ταξίλης ἐνδίδωσιν. 



Flavius Arrianus Hist., Phil., Alexandri anabasis 
Book 5, chapter 4, section 1, line 1

Ὁ δὲ Ἰνδὸς ποταμὸς ὅτι μέγιστος ποταμῶν ἐστι 
τῶν κατὰ τὴν Ἀσίαν τε καὶ τὴν Εὐρώπην, πλὴν 
Γάγγου, καὶ τούτου Ἰνδοῦ ποταμοῦ, καὶ ὅτι αἱ πηγαί 
εἰσιν αὐτῷ ἐπὶ τάδε τοῦ ὄρους τοῦ Παραπαμίσου ἢ 
Καυκάσου, καὶ ὅτι ἐκδίδωσιν ἐς τὴν μεγάλην θάλασσαν 
τὴν κατὰ Ἰνδοὺς ὡς ἐπὶ νότον ἄνεμον, καὶ ὅτι 
δίστομός ἐστιν ὁ Ἰνδὸς καὶ αἱ ἐκβολαὶ αὐτοῦ ἀμφότεραι 
τεναγώδεις, καθάπερ αἱ πέντε τοῦ Ἴστρου, καὶ ὅτι 
Δέλτα ποιεῖ καὶ αὐτὸς ἐν τῇ Ἰνδῶν γῇ τῷ Αἰγυπτίῳ 
Δέλτα παραπλήσιον καὶ τοῦτο Πάταλα καλεῖται τῇ 




Flavius Arrianus Hist., Phil., Alexandri anabasis 
Book 5, chapter 4, section 1, line 11

τῶν κατὰ τὴν Ἀσίαν τε καὶ τὴν Εὐρώπην, πλὴν 
Γάγγου, καὶ τούτου Ἰνδοῦ ποταμοῦ, καὶ ὅτι αἱ πηγαί 
εἰσιν αὐτῷ ἐπὶ τάδε τοῦ ὄρους τοῦ Παραπαμίσου ἢ 
Καυκάσου, καὶ ὅτι ἐκδίδωσιν ἐς τὴν μεγάλην θάλασσαν 
τὴν κατὰ Ἰνδοὺς ὡς ἐπὶ νότον ἄνεμον, καὶ ὅτι 
δίστομός ἐστιν ὁ Ἰνδὸς καὶ αἱ ἐκβολαὶ αὐτοῦ ἀμφότεραι 
τεναγώδεις, καθάπερ αἱ πέντε τοῦ Ἴστρου, καὶ ὅτι 
Δέλτα ποιεῖ καὶ αὐτὸς ἐν τῇ Ἰνδῶν γῇ τῷ Αἰγυπτίῳ 
Δέλτα παραπλήσιον καὶ τοῦτο Πάταλα καλεῖται τῇ 
Ἰνδῶν φωνῇ, ταῦτα μὲν ὑπὲρ τοῦ Ἰνδοῦ τὰ μάλιστα 
οὐκ ἀμφίλογα καὶ ἐμοὶ ἀναγεγράφθω. 



Flavius Arrianus Hist., Phil., Alexandri anabasis 
Book 5, chapter 4, section 2, line 3

                                           ἐπεὶ καὶ ὁ   
Ὑδάσπης καὶ Ἀκεσίνης καὶ Ὑδραώτης καὶ Ὕφασις, καὶ 
οὗτοι Ἰνδοὶ ποταμοὶ ὄντες, τῶν μὲν ἄλλων τῶν Ἀσιανῶν 
ποταμῶν πολύ τι κατὰ μέγεθος ὑπερφέρουσι, τοῦ 
Ἰνδοῦ δὲ μείονές εἰσιν καὶ πολὺ δὴ μείονες, ὅπου καὶ 
αὐτὸς ὁ Ἰνδὸς τοῦ Γάγγου. 



Flavius Arrianus Hist., Phil., Alexandri anabasis 
Book 5, chapter 4, section 2, line 8

                                   Κτησίας μὲν δή, εἰ δή 
τῳ ἱκανὸς καὶ Κτησίας ἐς τεκμηρίωσιν, ἵνα μὲν 
στενότατος αὐτὸς αὑτοῦ ὁ Ἰνδός ἐστι, τεσσαράκοντα 
σταδίους <λέγει> ὅτι διέχουσιν αὐτῷ αἱ ὄχθαι, ἵνα δὲ 
πλατύτατος, καὶ ἑκατόν· τὸ πολὺ δὲ εἶναι αὐτοῦ τὸ 
μέσον τούτοιν. 



Flavius Arrianus Hist., Phil., Alexandri anabasis 
Book 5, chapter 4, section 3, line 1

Τοῦτον τὸν ποταμὸν τὸν Ἰνδὸν ὑπὸ τὴν ἕω διέβαινε 
ξὺν τῇ στρατιᾷ Ἀλέξανδρος ἐς τῶν Ἰνδῶν τὴν γῆν· 
ὑπὲρ ὧν ἐγὼ οὔτε οἷστισι νόμοις διαχρῶνται ἐν τῇδε 
τῇ συγγραφῇ ἀνέγραψα, οὔτε ζῷα εἰ δή τινα ἄτοπα 
ἡ χώρα αὐτοῖς ἐκφέρει, οὔτε ἰχθύας ἢ κήτη ὅσα ἢ 
οἷα ὁ Ἰνδὸς ἢ ὁ Ὑδάσπης ἢ ὁ Γάγγης ἢ οἱ ἄλλοι 
Ἰνδῶν ποταμοὶ φέρουσιν, οὐδὲ τοὺς μύρμηκας τοὺς 
τὸν χρυσόν σφισιν ἐργαζομένους, οὐδὲ τοὺς γρῦπας 
τοὺς φύλακας, οὐδὲ ὅσα ἄλλα ἐφ' ἡδονῇ μᾶλλόν τι 
πεποίηται ἢ ἐς ἀφήγησιν τῶν ὄντων, ὡς τά γε κατ' 




Flavius Arrianus Hist., Phil., Alexandri anabasis 
Book 5, chapter 4, section 4, line 4

                 ἀλλὰ Ἀλέξανδρος γὰρ καὶ οἱ ξὺν 
τούτῳ στρατεύσαντες τὰ πολλὰ ἐξήλεγξαν, ὅσα γε μὴ   
καὶ αὐτῶν ἔστιν οἳ ἐψεύσαντο· ἀχρύσους τε εἶναι 
Ἰνδοὺς ἐξήλεγξαν, ὅσους γε δὴ Ἀλέξανδρος ξὺν τῇ 
στρατιᾷ ἐπῆλθε, πολλοὺς δὲ ἐπῆλθε, καὶ ἥκιστα 
χλιδῶντας κατὰ τὴν δίαιταν, ἀλλὰ μεγάλους μὲν τὰ 
σώματα, οἵους μεγίστους τῶν κατὰ τὴν Ἀσίαν, πεντα-
πήχεις τοὺς πολλοὺς ἢ ὀλίγον ἀποδέοντας, καὶ μελαν-
τέρους τῶν ἄλλων ἀνθρώπων, πλὴν Αἰθιόπων, καὶ τὰ 
πολέμια πολύ τι γενναιοτάτους τῶν γε δὴ τότε 
ἐποίκων τῆς Ἀσίας. 



Flavius Arrianus Hist., Phil., Alexandri anabasis 
Book 5, chapter 4, section 5, line 5

                         τὸ γὰρ Περσῶν τῶν πάλαι, ξὺν 
οἷς ὁρμηθεὶς Κῦρος ὁ Καμβύσου Μήδους τε τὴν 
ἀρχὴν τῆς Ἀσίας ἀφείλετο καὶ ἄλλα ἔθνη τὰ μὲν 
κατεστρέψατο, τὰ δὲ προσχωρήσαντά οἱ ἑκόντα κατέσχεν, 
οὐκ ἔχω ἀτρεκῶς ὥς γε δὴ πρὸς τὰ Ἰνδῶν ξυμβαλεῖν. 



Flavius Arrianus Hist., Phil., Alexandri anabasis 
Book 5, chapter 5, section 1, line 1

Ἀλλὰ ὑπὲρ Ἰνδῶν ἰδίᾳ μοι γεγράψεται ὅσα πιστό-
τατα ἐς ἀφήγησιν οἵ τε ξὺν Ἀλεξάνδρῳ στρατεύσαντες 
καὶ ὁ ἐκπεριπλεύσας τῆς μεγάλης θαλάσσης τὸ κατ' 
Ἰνδοὺς Νέαρχος, ἐπὶ δὲ ὅσα Μεγασθένης τε καὶ 
Ἐρατοσθένης, δοκίμω ἄνδρε, ξυνεγραψάτην, καὶ νόμιμα 
ἅττα Ἰνδοῖς ἐστι καὶ εἰ δή τινα ἄτοπα ζῷα αὐτόθι 
φύεται καὶ τὸν παράπλουν αὐτὸν τῆς ἔξω θαλάσσης. 



Flavius Arrianus Hist., Phil., Alexandri anabasis 
Book 5, chapter 5, section 4, line 2

ὃ δὴ Καύκασον ἐκάλουν οἱ Ἀλεξάνδρῳ ξυστρατεύσαντες 
Μακεδόνες, ὡς μὲν λέγεται τὰ Ἀλεξάνδρου αὔξοντες, 
ὅτι δὴ καὶ ἐπέκεινα ἄρα τοῦ Καυκάσου κρατῶν τοῖς 
ὅπλοις ἦλθεν Ἀλέξανδρος· τυχὸν δὲ καὶ ξυνεχὲς 
τυγχάνει ὂν τοῦτο τὸ ὄρος τῷ ἄλλῳ τῷ Σκυθικῷ 
Καυκάσῳ, καθάπερ οὖν αὐτῷ τούτῳ ὁ Ταῦρος· καὶ 
ἐμοὶ αὐτῷ πρότερόν ποτε ἐπὶ τῷδε λέλεκται Καύκασος 
τὸ ὄρος τοῦτο καὶ ὕστερον τῷδε τῷ ὀνόματι κλη-
θήσεται· τὸν δὲ Καύκασον τοῦτον καθήκειν ἔστε ἐπὶ 
<τὴν> μεγάλην τὴν πρὸς ἕω τε καὶ Ἰνδοὺς θάλασσαν. 



Flavius Arrianus Hist., Phil., Alexandri anabasis 
Book 5, chapter 5, section 5, line 2

τοὺς οὖν ποταμούς, ὅσοι κατὰ τὴν Ἀσίαν λόγου ἄξιοι, 
ἐκ τοῦ Ταύρου τε καὶ τοῦ Καυκάσου ἀνίσχοντας τοὺς 
μὲν ὡς ἐπ' ἄρκτον τετραμμένον ἔχειν τὸ ὕδωρ, καὶ 
τούτων τοὺς μὲν ἐς τὴν λίμνην ἐκδιδόναι τὴν 
Μαιῶτιν, τοὺς δὲ ἐς τὴν Ὑρκανίαν καλουμένην θάλας-
σαν, καὶ ταύτην κόλπον οὖσαν τῆς μεγάλης θαλάσσης, 
τοὺς δὲ ὡς ἐπὶ νότον ἄνεμον τὸν Εὐφράτην τε εἶναι 
καὶ τὸν Τίγρητα καὶ τὸν Ἰνδόν τε καὶ τὸν Ὑδάσπην 
καὶ Ἀκεσίνην καὶ Ὑδραώτην καὶ Ὕφασιν καὶ ὅσοι 
ἐν μέσῳ τούτων τε καὶ τοῦ Γάγγου ποταμοῦ ἐς   
θάλασσαν καὶ οὗτοι ἐσβάλλουσιν ἢ εἰς τενάγη ἀναχε-
όμενοι ἀφανίζονται, καθάπερ ὁ Εὐφράτης ποταμὸς 
ἀφανίζεται. 



Flavius Arrianus Hist., Phil., Alexandri anabasis 
Book 5, chapter 6, section 2, line 3

                                   τῆς δὲ ὡς ἐπὶ νότον 
Ἀσίας τετραχῇ αὖ τεμνομένης μεγίστην μὲν μοῖραν 
τὴν Ἰνδῶν γῆν ποιεῖ Ἐρατοσθένης τε καὶ Μεγασθένης, 
ὃς ξυνῆν μὲν Σιβυρτίῳ τῷ σατράπῃ τῆς Ἀραχωσίας, 
πολλάκις δὲ λέγει ἀφικέσθαι παρὰ Σανδράκοττον τὸν 
Ἰνδῶν βασιλέα, ἐλαχίστην δὲ ὅσην ὁ Εὐφράτης 
ποταμὸς ἀπείργει ὡς πρὸς τὴν ἐντὸς τὴν ἡμετέραν 
θάλασσαν. 



Flavius Arrianus Hist., Phil., Alexandri anabasis 
Book 5, chapter 6, section 2, line 9

           δύο δὲ αἱ μεταξὺ Εὐφράτου τε ποταμοῦ 
καὶ τοῦ Ἰνδοῦ ἀπειργόμεναι αἱ δύο ξυντεθεῖσαι μόλις 
ἄξιαι τῇ Ἰνδῶν γῇ ξυμβαλεῖν. 



Flavius Arrianus Hist., Phil., Alexandri anabasis 
Book 5, chapter 6, section 3, line 2

                                       ἀπείργεσθαι δὲ τὴν 
Ἰνδῶν χώραν πρὸς μὲν ἕω τε καὶ ἀπηλιώτην ἄνεμον 
ἔστε ἐπὶ μεσημβρίαν τῇ μεγάλῃ θαλάσσῃ· τὸ πρὸς 
βορρᾶν δὲ αὐτῆς ἀπείργειν τὸν Καύκασον τὸ ὄρος 
ἔστε ἐπὶ τοῦ Ταύρου τὴν ξυμβολήν· τὴν δὲ ὡς πρὸς 
ἑσπέραν τε καὶ ἄνεμον Ἰάπυγα ἔστε ἐπὶ τὴν μεγάλην 
θάλασσαν ὁ Ἰνδὸς ποταμὸς ἀποτέμνεται. 



Flavius Arrianus Hist., Phil., Alexandri anabasis 
Book 5, chapter 6, section 6, line 6

                                             εἰ δὴ οὖν 
εἷς τε ποταμὸς παρ' ἑκάστοις καὶ οὐ μεγάλοι οὗτοι   
ποταμοὶ ἱκανοὶ γῆν πολλὴν ποιῆσαι ἐς θάλασσαν 
προχεόμενοι, ὁπότε ἰλὺν καταφέροιεν καὶ πηλὸν ἐκ 
τῶν ἄνω τόπων ἔνθενπερ αὐτοῖς αἱ πηγαί εἰσιν, οὐδὲ 
ὑπὲρ τῆς Ἰνδῶν ἄρα χώρας ἐς ἀπιστίαν ἰέναι ἄξιον, 
ὅπως πεδίον τε ἡ πολλή ἐστι καὶ ἐκ τῶν ποταμῶν τὸ 
πεδίον ἔχει προσκεχωσμένον. 



Flavius Arrianus Hist., Phil., Alexandri anabasis 
Book 5, chapter 6, section 7, line 4

                                Ἕρμον μὲν γὰρ καὶ 
Κάϋστρον καὶ Κάϊκόν τε καὶ Μαίανδρον ἢ ὅσοι πολλοὶ 
ποταμοὶ τῆς Ἀσίας ἐς τήνδε τὴν ἐντὸς θάλασσαν 
ἐκδιδοῦσιν οὐδὲ σύμπαντας ξυντεθέντας ἑνὶ τῶν Ἰνδῶν 
ποταμῶν ἄξιον ξυμβαλεῖν πλήθους ἕνεκα τοῦ ὕδατος, 
μὴ ὅτι τῷ Γάγγῃ τῷ μεγίστῳ, ὅτῳ οὔτε <τὸ> τοῦ 
Νείλου ὕδωρ τοῦ Αἰγυπτίου οὔτε ὁ Ἴστρος ὁ κατὰ τὴν 
Εὐρώπην ῥέων ἄξιοι ξυμβαλεῖν, ἀλλ' οὐδὲ τῷ Ἰνδῷ 
ποταμῷ ἐκεῖνοί γε πάντες ξυμμιχθέντες ἐς ἴσον ἔρχονται, 
ὃς μέγας τε εὐθὺς ἀπὸ τῶν πηγῶν ἀνίσχει καὶ πεντε-
καίδεκα ποταμοὺς πάντας τῶν Ἀσιανῶν μείζονας παρα-
λαβὼν καὶ τῇ ἐπωνυμίᾳ κρατήσας οὕτως ἐκδιδοῖ ἐς 
θάλασσαν. 



Flavius Arrianus Hist., Phil., Alexandri anabasis 
Book 5, chapter 6, section 8, line 6

           ταῦτά μοι ἐν τῷ παρόντι περὶ Ἰνδῶν τῆς 
χώρας λελέχθω· τὰ δὲ ἄλλα ἀποκείσθω ἐς τὴν Ἰνδικὴν 
ξυγγραφήν. 



Flavius Arrianus Hist., Phil., Alexandri anabasis 
Book 5, chapter 7, section 1, line 1

Τὸ δὲ ζεῦγμα τὸ ἐπὶ τοῦ Ἰνδοῦ ποταμοῦ ὅπως 
μὲν ἐποιήθη Ἀλεξάνδρῳ οὔτε Ἀριστόβουλος οὔτε 
Πτολεμαῖος, οἷς μάλιστα ἐγὼ ἕπομαι, λέγουσιν· οὐδὲ 
αὐτὸς ἔχω ἀτρεκῶς εἰκάσαι, πότερα πλοίοις ἐζεύχθη ὁ 
πόρος, καθάπερ οὖν ὁ Ἑλλήσποντός τε πρὸς Ξέρξου 
καὶ ὁ Βόσπορός τε καὶ ὁ Ἴστρος πρὸς Δαρείου, ἢ 
γέφυρα κατὰ τοῦ ποταμοῦ διηνεκὴς ἐποιήθη αὐτῷ·   
δοκεῖ δ' ἔμοιγε πλοίοις μᾶλλον ζευχθῆναι· οὐ γὰρ 
ἂν δέξασθαι γέφυραν τὸ βάθος τοῦ ὕδατος, οὐδ' ἂν 
ἐν τοσῷδε χρόνῳ ἔργον οὕτως ἄτοπον ξυντελεσθῆναι. 



Flavius Arrianus Hist., Phil., Alexandri anabasis 
Book 5, chapter 8, section 1, line 2

Ῥωμαίοις μὲν δὴ οὕτω ταῦτα ἐκ παλαιοῦ ἐπήσκηται· 
Ἀλεξάνδρῳ δὲ ὅπως ἐζεύχθη ὁ Ἰνδὸς ποταμὸς οὐκ 
ἔχω εἰπεῖν, ὅτι μηδὲ οἱ συστρατεύσαντες αὐτῷ εἶπον. 



Flavius Arrianus Hist., Phil., Alexandri anabasis 
Book 5, chapter 8, section 2, line 2

                                              ὡς δὲ διέβη πέραν 
τοῦ Ἰνδοῦ ποταμοῦ, καὶ ἐνταῦθα αὖ θύει κατὰ νόμον 
Ἀλέξανδρος. 



Flavius Arrianus Hist., Phil., Alexandri anabasis 
Book 5, chapter 8, section 2, line 3

               ἄρας δὲ ἀπὸ τοῦ Ἰνδοῦ ἐς Τάξιλα 
ἀφίκετο, πόλιν μεγάλην καὶ εὐδαίμονα, τὴν μεγίστην 
τῶν μεταξὺ Ἰνδοῦ τε ποταμοῦ καὶ Ὑδάσπου. 



Flavius Arrianus Hist., Phil., Alexandri anabasis 
Book 5, chapter 8, section 2, line 7

                                                   καὶ 
ἐδέχετο αὐτὸν Ταξίλης ὁ ὕπαρχος τῆς πόλεως καὶ 
αὐτοὶ οἱ τῇδε Ἰνδοὶ φιλίως. 



Flavius Arrianus Hist., Phil., Alexandri anabasis 
Book 5, chapter 8, section 3, line 3

                                                         ἧκον 
δὲ ἐνταῦθα παρ' αὐτὸν καὶ παρὰ Ἀβισάρου πρέσβεις 
τοῦ τῶν ὀρείων Ἰνδῶν βασιλέως ὅ τε ἀδελφὸς τοῦ 
Ἀβισάρου καὶ ἄλλοι ξὺν αὐτῷ οἱ δοκιμώτατοι, καὶ 
παρὰ Δοξάρεως νομάρχου ἄλλοι, δῶρα φέροντες. 



Flavius Arrianus Hist., Phil., Alexandri anabasis 
Book 5, chapter 8, section 3, line 8

                                                 καὶ ἀπο-
δείξας σατράπην τῶν ταύτῃ Ἰνδῶν Φίλιππον τὸν   
Μαχάτα φρουράν τε ἀπολείπει ἐν Ταξίλοις καὶ τοὺς 
ἀπομάχους τῶν στρατιωτῶν διὰ νόσον· αὐτὸς δὲ ἦγεν 
ὡς ἐπὶ τὸν Ὑδάσπην ποταμόν. 



Flavius Arrianus Hist., Phil., Alexandri anabasis 
Book 5, chapter 8, section 4, line 5

                                           ταῦτα ὡς ἔγνω 
Ἀλέξανδρος, Κοῖνον μὲν τὸν Πολεμοκράτους πέμψας 
ὀπίσω ἐπὶ τὸν Ἰνδὸν ποταμὸν τὰ πλοῖα ὅσα παρ-
εσκεύαστο αὐτῷ ἐπὶ τοῦ πόρου τοῦ Ἰνδοῦ ξυντεμόντα 
κελεύει φέρειν ὡς ἐπὶ τὸν Ὑδάσπην ποταμόν. 



Flavius Arrianus Hist., Phil., Alexandri anabasis 
Book 5, chapter 8, section 5, line 8

                                                         αὐτὸς 
δὲ ἀναλαβὼν ἥν τε δύναμιν ἔχων ἧκεν ἐς Τάξιλα 
καὶ πεντακισχιλίους τῶν Ἰνδῶν, οὓς Ταξίλης τε καὶ 
οἱ ταύτῃ ὕπαρχοι ἦγον, ᾔει ὡς ἐπὶ τὸν Ὑδάσπην 
ποταμόν. 



Flavius Arrianus Hist., Phil., Alexandri anabasis 
Book 5, chapter 9, section 4, line 2

                                     ἄλλως τε ἐν μὲν 
τῷ τότε οἱ ποταμοὶ πάντες οἱ Ἰνδικοὶ πολλοῦ τε 
ὕδατος καὶ θολεροῦ ἔρρεον καὶ ὀξέος τοῦ ῥεύματος· 
ἦν γὰρ ὥρα ἔτους ᾗ μετὰ τροπὰς μάλιστα <τὰς> ἐν 
θέρει τρέπεται ὁ ἥλιος· ταύτῃ δὲ τῇ ὥρᾳ ὕδατά τε 
ἐξ οὐρανοῦ ἀθρόα τε καταφέρεται ἐς τὴν γῆν τὴν   
Ἰνδικὴν καὶ αἱ χιόνες αἱ τοῦ Καυκάσου, ἔνθενπερ 
τῶν πολλῶν ποταμῶν αἱ πηγαί εἰσι, κατατηκόμεναι 
αὔξουσιν αὐτοῖς τὸ ὕδωρ ἐπὶ μέγα· χειμῶνος δὲ 
ἔμπαλιν ἴσχουσιν ὀλίγοι τε γίγνονται καὶ καθαροὶ 
ἰδεῖν καὶ ἔστιν ὅπου περάσιμοι, πλήν γε δὴ τοῦ 




Flavius Arrianus Hist., Phil., Alexandri anabasis 
Book 5, chapter 9, section 4, line 12

τῷ τότε οἱ ποταμοὶ πάντες οἱ Ἰνδικοὶ πολλοῦ τε 
ὕδατος καὶ θολεροῦ ἔρρεον καὶ ὀξέος τοῦ ῥεύματος· 
ἦν γὰρ ὥρα ἔτους ᾗ μετὰ τροπὰς μάλιστα <τὰς> ἐν 
θέρει τρέπεται ὁ ἥλιος· ταύτῃ δὲ τῇ ὥρᾳ ὕδατά τε 
ἐξ οὐρανοῦ ἀθρόα τε καταφέρεται ἐς τὴν γῆν τὴν   
Ἰνδικὴν καὶ αἱ χιόνες αἱ τοῦ Καυκάσου, ἔνθενπερ 
τῶν πολλῶν ποταμῶν αἱ πηγαί εἰσι, κατατηκόμεναι 
αὔξουσιν αὐτοῖς τὸ ὕδωρ ἐπὶ μέγα· χειμῶνος δὲ 
ἔμπαλιν ἴσχουσιν ὀλίγοι τε γίγνονται καὶ καθαροὶ 
ἰδεῖν καὶ ἔστιν ὅπου περάσιμοι, πλήν γε δὴ τοῦ 
Ἰνδοῦ καὶ Γάγγου καὶ τυχὸν καὶ ἄλλου του· ἀλλ' ὅ 
γε Ὑδάσπης περατὸς γίνεται. 



Flavius Arrianus Hist., Phil., Alexandri anabasis 
Book 5, chapter 11, section 3, line 8

                καὶ Κρατερὸς ὑπελέλειπτο ἐπὶ στρατο-  
πέδου τήν τε αὑτοῦ ἔχων ἱππαρχίαν καὶ τοὺς ἐξ 
Ἀραχωτῶν καὶ Παραπαμισαδῶν ἱππέας καὶ τῆς φάλαγ-
γος τῶν Μακεδόνων τήν τε Ἀλκέτου καὶ τὴν Πολυ-
πέρχοντος τάξιν καὶ τοὺς νομάρχας τῶν ἐπὶ τάδε 
Ἰνδῶν καὶ τοὺς ἅμα τούτοις τοὺς πεντακισχιλίους. 



Flavius Arrianus Hist., Phil., Alexandri anabasis 
Book 5, chapter 12, section 1, line 7

                                       ἐν μέσῳ δὲ τῆς 
νήσου τε καὶ τοῦ μεγάλου στρατοπέδου, ἵνα αὐτῷ 
Κρατερὸς ὑπελέλειπτο, Μελέαγρός τε καὶ Ἄτταλος καὶ 
Γοργίας ξὺν τοῖς μισθοφόροις ἱππεῦσί τε καὶ πεζοῖς 
ἐτετάχατο· καὶ τούτοις διαβαίνειν παρηγγέλλετο κατὰ 
μέρος, διελόντας τὸν στρατόν, ὁπότε ξυνεχομένους ἤδη 
ἐν τῇ μάχῃ τοὺς Ἰνδοὺς ἴδοιεν. 



Flavius Arrianus Hist., Phil., Alexandri anabasis 
Book 5, chapter 14, section 2, line 6

γνώμην δὲ ἐπεποίητο, ὡς εἰ μὲν προσμίξειαν αὐτῷ οἱ 
ἀμφὶ τὸν Πῶρον ξὺν τῇ δυνάμει ἁπάσῃ, ἢ κρατήσειν 
αὐτῶν οὐ χαλεπῶς τῇ ἵππῳ προσβαλὼν ἢ ἀπομαχεῖσθαί 
γε ἔστε τοὺς πεζοὺς ἐν τῷ ἔργῳ ἐπιγενέσθαι· εἰ δὲ 
πρὸς τὴν τόλμαν τῆς διαβάσεως ἄτοπον γενομένην οἱ 
Ἰνδοὶ ἐκπλαγέντες φεύγοιεν, οὐ πόρρωθεν ἕξεσθαι 
αὐτῶν κατὰ τὴν φυγήν, ὡς πλείονα ἐν τῇ ἀποχωρήσει τὸν 
φόνον γενόμενον ὀλίγον ἔτι ὑπολείπεσθαι αὐτῷ τὸ ἔργον. 



Flavius Arrianus Hist., Phil., Alexandri anabasis 
Book 5, chapter 14, section 3, line 6

Ἀριστόβουλος δὲ λέγει τὸν Πώρου παῖδα φθάσαι 
ἀφικόμενον σὺν ἅρμασιν ὡς ἑξήκοντα πρὶν τὸ ὕστε-
ρον ἐκ τῆς νήσου τῆς μικρᾶς περᾶσαι Ἀλέξανδρον· 
καὶ τοῦτον δυνηθῆναι ἂν εἶρξαι Ἀλέξανδρον τῆς 
διαβάσεως χαλεπῶς καὶ μηδενὸς εἴργοντος περαιω-
θέντα, εἴπερ οὖν καταπηδήσαντες οἱ Ἰνδοὶ ἐκ τῶν 
ἁρμάτων προσέκειντο τοῖς πρώτοις τῶν ἐκβαινόντων·   
ἀλλὰ παραλλάξαι γὰρ ξὺν τοῖς ἅρμασι καὶ ἀκίνδυνον 
ποιῆσαι Ἀλεξάνδρῳ τὴν διάβασιν· καὶ ἐπὶ τούτους 
ἀφεῖναι Ἀλέξανδρον τοὺς ἱπποτοξότας, καὶ τραπῆναι 
αὐτοὺς οὐ χαλεπῶς, πληγὰς λαμβάνοντας. 



Flavius Arrianus Hist., Phil., Alexandri anabasis 
Book 5, chapter 14, section 4, line 2

                                             οἱ δὲ καὶ 
μάχην λέγουσιν ἐν τῇ ἐκβάσει γενέσθαι τῶν Ἰνδῶν 
τῶν ξὺν τῷ παιδὶ τῷ Πώρου ἀφιγμένων πρὸς Ἀλέξ-
ανδρόν τε καὶ τοὺς ξὺν αὐτῷ ἱππέας. 



Flavius Arrianus Hist., Phil., Alexandri anabasis 
Book 5, chapter 15, section 2, line 1

ὡς δὲ κατέμαθεν ἀτρεκῶς τὸ πλῆθος τὸ τῶν Ἰνδῶν, 
ἐνταῦθα δὴ ὀξέως ἐπιπεσεῖν αὐτοῖς ξὺν τῇ ἀμφ' αὑτὸν 
ἵππῳ· τοὺς δὲ ἐγκλῖναι, ὡς Ἀλέξανδρόν τε αὐτὸν 
κατεῖδον καὶ τὸ στῖφος ἀμφ' αὐτὸν τῶν ἱππέων οὐκ 
ἐπὶ μετώπου, ἀλλὰ κατ' ἴλας ἐμβεβληκός. 



Flavius Arrianus Hist., Phil., Alexandri anabasis 
Book 5, chapter 16, section 1, line 2

                                      Ἀλέξανδρος δὲ ὡς 
ἤδη καθεώρα τοὺς Ἰνδοὺς ἐκτασσομένους, ἐπέστησε 
τοὺς ἱππέας τοῦ πρόσω, ὡς ἀναλαμβάνειν τῶν πεζῶν 
τοὺς ἀεὶ προσάγοντας. 



Flavius Arrianus Hist., Phil., Alexandri anabasis 
Book 5, chapter 16, section 2, line 1

             ὡς δὲ τὴν τάξιν κατεῖδε τῶν Ἰνδῶν, κατὰ 
μέσον μέν, ἵνα οἱ ἐλέφαντες προεβέβληντο καὶ πυκνὴ 
ἡ φάλαγξ κατὰ τὰ διαλείποντα αὐτῶν ἐπετέτακτο, οὐκ 
ἔγνω προάγειν, αὐτὰ ἐκεῖνα ὀκνήσας ἅπερ ὁ Πῶρος 
τῷ λογισμῷ ξυνθεὶς ταύτῃ ἔταξεν· ἀλλὰ αὐτὸς μὲν 
ἅτε ἱπποκρατῶν τὴν πολλὴν τῆς ἵππου ἀναλαβὼν ἐπὶ 
τὸ εὐώνυμον κέρας τῶν πολεμίων παρήλαυνεν, ὡς 
ταύτῃ ἐπιθησόμενος. 



Flavius Arrianus Hist., Phil., Alexandri anabasis 
Book 5, chapter 16, section 4, line 2

Ἤδη τε ἐντὸς βέλους ἐγίγνετο καὶ ἐφῆκεν ἐπὶ τὸ 
κέρας τὸ εὐώνυμον τῶν Ἰνδῶν τοὺς ἱπποτοξότας, 
ὄντας ἐς χιλίους, ὡς ταράξαι τοὺς ταύτῃ ἐφεστηκότας 
τῶν πολεμίων τῇ πυκνότητί τε τῶν τοξευμάτων καὶ 
τῶν ἵππων τῇ ἐπελάσει. 



Flavius Arrianus Hist., Phil., Alexandri anabasis 
Book 5, chapter 17, section 1, line 1

Ἐν τούτῳ δὲ οἵ τε Ἰνδοὶ τοὺς ἱππέας πάντοθεν 
ξυναλίσαντες παρίππευον Ἀλεξάνδρῳ ἀντιπαρεξάγοντες 
τῇ ἐλάσει, καὶ οἱ περὶ Κοῖνον, ὡς παρήγγελτο, κατόπιν   
αὐτοῖς ἐπεφαίνοντο. 



Flavius Arrianus Hist., Phil., Alexandri anabasis 
Book 5, chapter 17, section 1, line 4

                        ταῦτα ξυνιδόντες οἱ Ἰνδοὶ ἀμφί-
στομον ἠναγκάσθησαν ποιῆσαι τὴν τάξιν τῆς ἵππου, 
τὴν μὲν ὡς ἐπ' Ἀλέξανδρον τὴν πολλήν τε καὶ 
κρατίστην, οἱ δὲ ἐπὶ Κοῖνόν τε καὶ τοὺς ἅμα τούτῳ 
ἐπέστρεφον. 



Flavius Arrianus Hist., Phil., Alexandri anabasis 
Book 5, chapter 17, section 2, line 2

              τοῦτό τε οὖν εὐθὺς ἐτάραξε τὰς τάξεις 
τε καὶ τὰς γνώμας τῶν Ἰνδῶν καὶ Ἀλέξανδρος ἰδὼν 
τὸν καιρὸν ἐν αὐτῇ τῇ ἐπὶ θάτερα ἐπιστροφῇ τῆς 
ἵππου ἐπιτίθεται τοῖς καθ' αὑτόν, ὥστε οὐδὲ τὴν 
ἐμβολὴν ἐδέξαντο τῶν ἀμφ' Ἀλέξανδρον ἱππέων οἱ 
Ἰνδοί, ἀλλὰ κατηρ[ρ]άχθησαν ὥσπερ εἰς τεῖχός τι φίλιον 
τοὺς ἐλέφαντας. 



Flavius Arrianus Hist., Phil., Alexandri anabasis 
Book 5, chapter 17, section 3, line 9

                                         καὶ ἦν τὸ 
ἔργον οὐδενὶ τῶν πρόσθεν ἀγώνων ἐοικός· τά τε γὰρ 
θηρία ἐπεκθέοντα ἐς τὰς τάξεις τῶν πεζῶν, ὅπῃ 
ἐπιστρέψειεν, ἐκεράϊζε καίπερ πυκνὴν οὖσαν τὴν τῶν 
Μακεδόνων φάλαγγα, καὶ οἱ ἱππεῖς οἱ τῶν Ἰνδῶν 
τοῖς πεζοῖς ἰδόντες ξυνεστηκὸς τὸ ἔργον ἐπιστρέψαντες 
αὖθις καὶ αὐτοὶ ἐπήλαυνον τῇ ἵππῳ. 



Flavius Arrianus Hist., Phil., Alexandri anabasis 
Book 5, chapter 17, section 4, line 7

                       καὶ ἐν τούτῳ πᾶσα ἡ ἵππος 
Ἀλεξάνδρῳ ἐς μίαν ἴλην ἤδη ξυνηγμένη, οὐκ ἐκ παρ-
αγγέλματος, ἀλλὰ ἐν τῷ ἀγῶνι αὐτῷ ἐς τήνδε τὴν 
τάξιν καταστᾶσα, ὅπῃ προσπέσοι τῶν Ἰνδῶν ταῖς 
τάξεσι, ξὺν πολλῷ φόνῳ ἀπελύοντο. 



Flavius Arrianus Hist., Phil., Alexandri anabasis 
Book 5, chapter 17, section 6, line 7

          ἀλλ' οἱ μὲν Μακεδόνες, ἅτε ἐν εὐρυχωρίᾳ 
τε καὶ κατὰ γνώμην τὴν σφῶν προσφερόμενοι τοῖς 
θηρίοις, ὅπῃ μὲν ἐπιφέροιντο εἶκον, ἀποστραφέντων 
δὲ εἴχοντο ἐσακοντίζοντες· οἱ δὲ Ἰνδοὶ ἐν αὐτοῖς 
ἀναστρεφόμενοι τὰ πλείω ἤδη πρὸς ἐκείνων ἐβλάπτοντο. 



Flavius Arrianus Hist., Phil., Alexandri anabasis 
Book 5, chapter 17, section 7, line 8

                         καὶ οὕτως οἱ μὲν ἱππεῖς τῶν 
Ἰνδῶν πλὴν ὀλίγων κατεκόπησαν ἐν τῷ ἔργῳ· ἐκό-
πτοντο δὲ καὶ οἱ πεζοὶ πανταχόθεν ἤδη προσκειμένων 
σφίσι τῶν Μακεδόνων. 



Flavius Arrianus Hist., Phil., Alexandri anabasis 
Book 5, chapter 18, section 1, line 6

             καὶ οὗτοι οὐ μείονα τὸν φόνον ἐν τῇ 
ἀποχωρήσει τῶν Ἰνδῶν ἐποίησαν, ἀκμῆτες ἀντὶ κεκμη-
κότων τῶν ἀμφ' Ἀλέξανδρον ἐπιγενόμενοι τῇ διώξει. 



Flavius Arrianus Hist., Phil., Alexandri anabasis 
Book 5, chapter 18, section 2, line 1

Ἀπέθανον δὲ τῶν Ἰνδῶν πεζοὶ μὲν ὀλίγον ἀπο-
δέοντες τῶν δισμυρίων, ἱππεῖς δὲ ἐς τρισχιλίους, τὰ 
δὲ ἅρματα ξύμπαντα κατεκόπη· καὶ Πώρου δύο παῖδες 
ἀπέθανον καὶ Σπιτάκης ὁ νομάρχης τῶν ταύτῃ Ἰνδῶν 
καὶ τῶν ἐλεφάντων καὶ ἁρμάτων οἱ ἡγεμόνες καὶ οἱ 
ἱππάρχαι καὶ οἱ στρατηγοὶ τῆς στρατιᾶς τῆς Πώρου 
ξύμπαντες . 



Flavius Arrianus Hist., Phil., Alexandri anabasis 
Book 5, chapter 18, section 5, line 1

Πῶρος δὲ μεγάλα ἔργα ἐν τῇ μάχῃ ἀποδειξάμενος 
μὴ ὅτι στρατηγοῦ, ἀλλὰ καὶ στρατιώτου γενναίου, ὡς 
τῶν τε ἱππέων τὸν φόνον κατεῖδε καὶ τῶν ἐλεφάντων 
τοὺς μὲν αὐτοῦ πεπτωκότας, τοὺς δὲ ἐρήμους τῶν 
ἡγεμόνων λυπηροὺς πλανωμένους, τῶν δὲ πεζῶν αὐτῷ 
οἱ πλείους ἀπολώλεσαν, οὐχ ᾗπερ Δαρεῖος ὁ μέγας 
βασιλεὺς ἐξάρχων τοῖς ἀμφ' αὑτὸν τῆς φυγῆς ἀπεχώρει, 
ἀλλὰ ἔστε γὰρ ὑπέμενέ τι τῶν Ἰνδῶν ἐν τῇ μάχῃ   
ξυνεστηκός, ἐς τοσόνδε ἀγωνισάμενος, τετρωμένος δὲ τὸν 
δεξιὸν ὦμον, ὃν δὴ γυμνὸν μόνον ἔχων ἐν τῇ μάχῃ ἀν-
εστρέφετο (ἀπὸ γὰρ τοῦ ἄλλου σώματος ἤρκει αὐτῷ τὰ 
βέλη ὁ θώραξ περιττὸς ὢν κατά τε τὴν ἰσχὺν καὶ τὴν 
ἁρμονίαν, ὡς ὕστερον καταμαθεῖν θεωμένοις ἦν), τότε 
δὴ καὶ αὐτὸς ἀπεχώρει ἐπιστρέψας τὸν ἐλέφαντα. 



Flavius Arrianus Hist., Phil., Alexandri anabasis 
Book 5, chapter 18, section 6, line 4

                                    πέμπει δὴ παρ' αὐτὸν 
πρῶτα μὲν Ταξίλην τὸν Ἰνδόν· καὶ Ταξίλης προς-
ιππεύσας ἐφ' ὅσον οἱ ἀσφαλὲς ἐφαίνετο τῷ ἐλέφαντι 
ὃς ἔφερε τὸν Πῶρον ἐπιστῆσαί τε ἠξίου τὸ θηρίον, 
οὐ γὰρ εἶναί οἱ ἔτι φεύγειν, καὶ ἀκοῦσαι τῶν παρ' 
Ἀλεξάνδρου λόγων. 



Flavius Arrianus Hist., Phil., Alexandri anabasis 
Book 5, chapter 18, section 7, line 7

Ἀλέξανδρος δὲ οὐδὲ ἐπὶ τῷδε τῷ Πώρῳ χαλεπὸς ἐγέ-
νετο, ἀλλ' ἄλλους τε ἐν μέρει ἔπεμπε καὶ δὴ καὶ 
Μερόην ἄνδρα Ἰνδόν, ὅτι φίλον εἶναι ἐκ παλαιοῦ τῷ 
Πώρῳ τὸν Μερόην ἔμαθεν. 



Flavius Arrianus Hist., Phil., Alexandri anabasis 
Book 5, chapter 19, section 3, line 2

           καὶ Ἀλέξανδρος τούτῳ ἔτι μᾶλλον τῷ λόγῳ 
ἡσθεὶς τήν τε ἀρχὴν τῷ Πώρῳ τῶν τε αὐτῶν Ἰνδῶν 
ἔδωκεν καὶ ἄλλην ἔτι χώραν πρὸς τῇ πάλαι οὔσῃ 
πλείονα τῆς πρόσθεν προσέθηκεν· καὶ οὕτως αὐτός 
τε βασιλικῶς κεχρημένος ἦν ἀνδρὶ ἀγαθῷ καὶ ἐκείνῳ 
ἐκ τούτου ἐς ἅπαντα πιστῷ ἐχρήσατο. 



Flavius Arrianus Hist., Phil., Alexandri anabasis 
Book 5, chapter 19, section 3, line 8

                                             τοῦτο τὸ τέλος 
τῇ μάχῃ τῇ πρὸς Πῶρόν τε καὶ τοὺς ἐπέκεινα τοῦ 
Ὑδάσπου ποταμοῦ Ἰνδοὺς Ἀλεξάνδρῳ ἐγένετο ἐπ' 
ἄρχοντος Ἀθηναίοις Ἡγήμονος μηνὸς Μουνυχιῶνος. 



Flavius Arrianus Hist., Phil., Alexandri anabasis 
Book 5, chapter 19, section 4, line 3

καὶ τὴν μὲν Νίκαιαν τῆς νίκης τῆς κατ' Ἰνδῶν ἐπώ-
νυμον ὠνόμασε, τὴν δὲ Βουκεφάλαν ἐς τοῦ ἵππου τοῦ   
Βουκεφάλα τὴν μνήμην, ὃς ἀπέθανεν αὐτοῦ, οὐ βληθεὶς 
πρὸς οὐδενός, ἀλλὰ ὑπὸ καύματος τε καὶ ἡλικίας (ἦν 
γὰρ ἀμφὶ τὰ τριάκοντα ἔτη) καματηρὸς γενόμενος, 
πολλὰ δὲ πρόσθεν ξυγκαμών τε καὶ συγκινδυνεύσας 
Ἀλεξάνδρῳ, ἀναβαινόμενός τε πρὸς μόνου Ἀλεξάνδρου 
[ὁ Βουκεφάλας οὗτος], ὅτι τοὺς ἄλλους πάντας ἀπηξίου 
ἀμβάτας, καὶ μεγέθει μέγας καὶ τῷ θυμῷ γενναῖος. 



Flavius Arrianus Hist., Phil., Alexandri anabasis 
Book 5, chapter 20, section 2, line 5

                                                 Κρατερὸν 
μὲν δὴ ξὺν μέρει τῆς στρατιᾶς ὑπελείπετο τὰς πόλεις 
ἅστινας ταύτῃ ἔκτιζεν ἀναστήσοντά τε καὶ ἐκτειχιοῦντα· 
αὐτὸς δὲ ἤλαυνεν ὡς ἐπὶ τοὺς προσχώρους τῇ Πώρου 
ἀρχῇ Ἰνδούς. 



Flavius Arrianus Hist., Phil., Alexandri anabasis 
Book 5, chapter 20, section 6, line 2

                                    ἧκον δὲ καὶ παρὰ 
τῶν αὐτονόμων Ἰνδῶν πρέσβεις παρ' Ἀλέξανδρον καὶ 
παρὰ Πώρου ἄλλου του ὑπάρχου Ἰνδῶν. 



Flavius Arrianus Hist., Phil., Alexandri anabasis 
Book 5, chapter 20, section 8, line 3

        τούτου τοῦ Ἀκεσίνου τὸ μέγεθος μόνου τῶν 
Ἰνδῶν ποταμῶν Πτολεμαῖος ὁ Λάγου ἀνέγραψεν· 
εἶναι γὰρ ἵνα ἐπέρασεν αὐτὸν Ἀλέξανδρος ἐπὶ τῶν 
πλοίων τε καὶ τῶν διφθερῶν ξὺν τῇ στρατιᾷ τὸ μὲν 
ῥεῦμα ὀξὺ τοῦ Ἀκεσίνου πέτραις μεγάλαις καὶ ὀξείαις, 
καθ' ὧν φερόμενον βίᾳ τὸ ὕδωρ κυμαίνεσθαί τε καὶ 
καχλάζειν, τὸ δὲ εὖρος σταδίους ἐπέχειν πεντεκαίδεκα. 



Flavius Arrianus Hist., Phil., Alexandri anabasis 
Book 5, chapter 20, section 10, line 3

                           εἴη ἂν οὖν ἐκ τοῦδε τοῦ 
λόγου ξυντιθέντι τεκμηριοῦσθαι, ὅτι οὐ πόρρω τοῦ 
ἀληθοῦς ἀναγέγραπται τοῦ Ἰνδοῦ ποταμοῦ τὸ μέγεθος, 
ὅσοις ἐς τεσσαράκοντα σταδίους δοκεῖ τοῦ Ἰνδοῦ εἶναι   
τὸ εὖρος, ἵνα μέσως ἔχει αὐτὸς αὑτοῦ ὁ Ἰνδός· ἵνα δὲ 
στενότατός τε καὶ διὰ στενότητα βαθύτατος ἐς τοὺς 
πεντεκαίδεκα ξυνάγεσθαι· καὶ ταῦτα πολλαχῇ εἶναι τοῦ 
Ἰνδοῦ. 



Flavius Arrianus Hist., Phil., Alexandri anabasis 
Book 5, chapter 21, section 1, line 5

Περάσας δὲ τὸν ποταμὸν Κοῖνον μὲν ξὺν τῇ αὑτοῦ 
τάξει ἀπολείπει αὐτοῦ ἐπὶ τῇ ὄχθῃ προστάξας ἐπι-
μελεῖσθαι τῆς ὑπολελειμμένης στρατιᾶς τῆς διαβάσεως, 
οἳ τόν τε σῖτον αὐτῷ τὸν ἐκ τῆς ἤδη ὑπηκόου τῶν 
Ἰνδῶν χώρας καὶ τὰ ἄλλα ἐπιτήδεια παρακομίζειν 
ἔμελλον. 



Flavius Arrianus Hist., Phil., Alexandri anabasis 
Book 5, chapter 21, section 2, line 2

           Πῶρον δὲ ἐς τὰ αὑτοῦ ἤθη ἀποπέμπει, 
κελεύσας Ἰνδῶν τε τοὺς μαχιμωτάτους ἐπιλεξάμενον 
καὶ εἴ τινας παρ' αὑτῷ ἔχοι ἐλέφαντας, τούτους δὲ 
ἀναλαβόντα[ς] ἰέναι παρ' αὑτόν. 



Flavius Arrianus Hist., Phil., Alexandri anabasis 
Book 5, chapter 21, section 4, line 2

Ἐπὶ τοῦτον ἐλαύνων Ἀλέξανδρος ἀφικνεῖται ἐπὶ 
τὸν Ὑδραώτην ποταμόν, ἄλλον αὖ τοῦτον Ἰνδὸν   
ποταμόν, τὸ μὲν εὖρος οὐ μείονα τοῦ Ἀκεσίνου, ὀξύτητι 
δὲ τοῦ ῥοῦ μείονα. 



Flavius Arrianus Hist., Phil., Alexandri anabasis 
Book 5, chapter 21, section 5, line 7

                ἐνταῦθα Ἡφαιστίωνα μὲν ἐκπέμπει 
δοὺς αὐτῷ μέρος τῆς στρατιᾶς, πεζῶν μὲν φάλαγγας 
δύο, ἱππέων δὲ τήν τε αὑτοῦ καὶ τὴν Δημητρίου 
ἱππαρχίαν καὶ τῶν τοξοτῶν τοὺς ἡμίσεας, ἐς τὴν 
Πώρου τοῦ ἀφεστηκότος χώραν, κελεύσας παραδιδόναι 
ταύτην Πώρῳ τῷ ἄλλῳ, καὶ εἰ δή τινα πρὸς ταῖς 
ὄχθαις τοῦ Ὑδραώτου ποταμοῦ αὐτόνομα ἔθνη Ἰνδῶν 
νέμεται, καὶ ταῦτα προσαγαγόμενον τῷ Πώρῳ ἄρχειν 
ἐγχειρίσαι. 



Flavius Arrianus Hist., Phil., Alexandri anabasis 
Book 5, chapter 22, section 1, line 2

Ἐν τούτῳ δὲ ἐξαγγέλλεται Ἀλεξάνδρῳ τῶν αὐτο-
νόμων Ἰνδῶν ἄλλους τέ τινας καὶ τοὺς καλουμένους 
Καθαίους αὐτούς τε παρασκευάζεσθαι ὡς πρὸς μάχην, 
εἰ προσάγοι τῇ χώρᾳ αὐτῶν Ἀλέξανδρος, καὶ ὅσα ὅμορά 
σφισιν <ἔθνη> ὡσαύτως αὐτόνομα, καὶ ταῦτα παρα-
καλεῖν ἐς τὸ ἔργον· εἶναι δὲ τήν τε πόλιν ὀχυρὰν 
πρὸς ᾗ ἐπενόουν ἀγωνίσασθαι, Σάγγαλα ἦν τῇ πόλει 
ὄνομα, καὶ αὐτοὶ οἱ Καθαῖοι εὐτολμότατοί τε καὶ τὰ 
πολέμια κράτιστοι ἐνομίζοντο, καὶ τούτοις κατὰ τὰ   
αὐτὰ Ὀξυδράκαι, ἄλλο Ἰνδῶν ἔθνος, καὶ Μαλλοί, ἄλλο 
καὶ τοῦτο· ἐπεὶ καὶ ὀλίγῳ πρόσθεν στρατεύσαντας ἐπ' 




Flavius Arrianus Hist., Phil., Alexandri anabasis 
Book 5, chapter 22, section 2, line 8

εἰ προσάγοι τῇ χώρᾳ αὐτῶν Ἀλέξανδρος, καὶ ὅσα ὅμορά 
σφισιν <ἔθνη> ὡσαύτως αὐτόνομα, καὶ ταῦτα παρα-
καλεῖν ἐς τὸ ἔργον· εἶναι δὲ τήν τε πόλιν ὀχυρὰν 
πρὸς ᾗ ἐπενόουν ἀγωνίσασθαι, Σάγγαλα ἦν τῇ πόλει 
ὄνομα, καὶ αὐτοὶ οἱ Καθαῖοι εὐτολμότατοί τε καὶ τὰ 
πολέμια κράτιστοι ἐνομίζοντο, καὶ τούτοις κατὰ τὰ   
αὐτὰ Ὀξυδράκαι, ἄλλο Ἰνδῶν ἔθνος, καὶ Μαλλοί, ἄλλο 
καὶ τοῦτο· ἐπεὶ καὶ ὀλίγῳ πρόσθεν στρατεύσαντας ἐπ' 
αὐτοὺς Πῶρόν τε καὶ Ἀνισάρην ξύν τε τῇ σφετέρᾳ 
δυνάμει καὶ πολλὰ ἄλλα ἔθνη τῶν αὐτονόμων Ἰνδῶν 
ἀναστήσαντας οὐδὲν πράξαντας τῆς παρασκευῆς ἄξιον 
ξυνέβη ἀπελθεῖν. 



Flavius Arrianus Hist., Phil., Alexandri anabasis 
Book 5, chapter 22, section 3, line 4

                            καὶ δευτεραῖος μὲν ἀπὸ τοῦ 
ποταμοῦ τοῦ Ὑδραώτου πρὸς πόλιν ἧκεν ᾗ ὄνομα 
Πίμπραμα· τὸ δὲ ἔθνος τοῦτο τῶν Ἰνδῶν Ἀδραϊσταὶ 
ἐκαλοῦντο. 



Flavius Arrianus Hist., Phil., Alexandri anabasis 
Book 5, chapter 22, section 5, line 6

          Ἀλέξανδρος δὲ τό τε πλῆθος κατιδὼν τῶν 
βαρβάρων καὶ τοῦ χωρίου τὴν φύσιν, ὡς μάλιστα πρὸς 
τὰ παρόντα ἐν καιρῷ οἱ ἐφαίνετο παρετάσσετο· καὶ 
τοὺς μὲν ἱπποτοξότας εὐθὺς ὡς εἶχεν ἐκπέμπει ἐπ' 
αὐτούς, ἀκροβολίζεσθαι κελεύσας παριππεύοντας, ὡς 
μήτε ἐκδρομήν τινα ποιήσασθαι τοὺς Ἰνδοὺς πρὶν 
ξυνταχθῆναι αὐτῷ τὴν στρατιὰν καὶ ὡς πληγὰς γίγνε-
σθαι αὐτοῖς καὶ πρὸ τῆς μάχης ἐντὸς τοῦ ὀχυρώματος. 



Flavius Arrianus Hist., Phil., Alexandri anabasis 
Book 5, chapter 22, section 7, line 8

                         καὶ τούτων τοὺς μὲν ἱππέας ἐπὶ 
τὰ κέρατα διελὼν παρήγαγεν, ἀπὸ δὲ τῶν πεζῶν τῶν 
προσγενομένων πυκνοτέραν τὴν ξύγκλεισιν τῆς φά-
λαγγος ποιήσας αὐτὸς ἀναλαβὼν τὴν ἵππον τὴν ἐπὶ 
τοῦ δεξιοῦ τεταγμένην παρήγαγεν ἐπὶ τὰς κατὰ τὸ 
εὐώνυμον τῶν Ἰνδῶν ἁμάξας. 



Flavius Arrianus Hist., Phil., Alexandri anabasis 
Book 5, chapter 23, section 1, line 2

Ὡς δὲ ἐπὶ τὴν ἵππον προσαγαγοῦσαν οὐκ ἐξέδραμον 
οἱ Ἰνδοὶ ἔξω τῶν ἁμαξῶν, ἀλλ' ἐπιβεβηκότες αὐτῶν 
ἀφ' ὑψηλοῦ ἠκροβολίζοντο, γνοὺς Ἀλέξανδρος ὅτι οὐκ 
εἴη τῶν ἱππέων τὸ ἔργον καταπηδήσας ἀπὸ τοῦ ἵππου 
πεζὸς ἐπῆγε τῶν πεζῶν τὴν φάλαγγα. 



Flavius Arrianus Hist., Phil., Alexandri anabasis 
Book 5, chapter 23, section 2, line 3

                                          καὶ ἀπὸ μὲν 
τῶν πρώτων ἁμαξῶν οὐ χαλεπῶς ἐβιάσαντο οἱ Μακε-
δόνες τοὺς Ἰνδούς· πρὸ δὲ τῶν δευτέρων οἱ Ἰνδοὶ 
παραταξάμενοι ῥᾷον ἀπεμάχοντο, οἷα δὴ πυκνότεροί 
τε ἐφεστηκότες ἐλάττονι τῷ κύκλῳ καὶ τῶν Μακεδόνων 
οὐ κατ' εὐρυχωρίαν ὡσαύτως προσαγόντων σφίσιν, ἐν 
ᾧ τάς τε πρώτας ἁμάξας ὑπεξῆγον καὶ κατὰ τὰ δια-
λείμματα αὐτῶν ὡς ἑκάστοις προὐχώρει ἀτάκτως   
προσέβαλλον· ἀλλὰ καὶ ἀπὸ τούτων ὅμως ἐξώσθησαν 
οἱ Ἰνδοὶ βιασθέντες πρὸς τῆς φάλαγγος. 



Flavius Arrianus Hist., Phil., Alexandri anabasis 
Book 5, chapter 23, section 3, line 1

τῶν πρώτων ἁμαξῶν οὐ χαλεπῶς ἐβιάσαντο οἱ Μακε-
δόνες τοὺς Ἰνδούς· πρὸ δὲ τῶν δευτέρων οἱ Ἰνδοὶ 
παραταξάμενοι ῥᾷον ἀπεμάχοντο, οἷα δὴ πυκνότεροί 
τε ἐφεστηκότες ἐλάττονι τῷ κύκλῳ καὶ τῶν Μακεδόνων 
οὐ κατ' εὐρυχωρίαν ὡσαύτως προσαγόντων σφίσιν, ἐν 
ᾧ τάς τε πρώτας ἁμάξας ὑπεξῆγον καὶ κατὰ τὰ δια-
λείμματα αὐτῶν ὡς ἑκάστοις προὐχώρει ἀτάκτως   
προσέβαλλον· ἀλλὰ καὶ ἀπὸ τούτων ὅμως ἐξώσθησαν 
οἱ Ἰνδοὶ βιασθέντες πρὸς τῆς φάλαγγος. 



Flavius Arrianus Hist., Phil., Alexandri anabasis 
Book 5, chapter 23, section 4, line 5

                                  καὶ Ἀλέξανδρος ταύτην 
μὲν τὴν ἡμέραν περιεστρατοπέδευσε τοῖς πεζοῖς τὴν 
πόλιν ὅσα γε ἠδυνήθη αὐτῷ περιβαλεῖν ἡ φάλαγξ· 
ἐπὶ πολὺ γὰρ ἐπέχον τὸ τεῖχος τῷ στρατοπέδῳ κυκλώ-
σασθαι οὐ δυνατὸς ἐγένετο· κατὰ δὲ τὰ διαλείποντα 
αὐτοῦ, ἵνα καὶ λίμνη οὐ μακρὰν τοῦ τείχους ἦν, τοὺς 
ἱππέας ἐπέταξεν ἐν κύκλῳ τῆς λίμνης, γνοὺς οὐ βαθεῖαν 
οὖσαν τὴν λίμνην καὶ ἅμα εἰκάσας ὅτι φοβεροὶ γενό-
μενοι οἱ Ἰνδοὶ ἀπὸ τῆς προτέρας ἥττης ἀπολείψουσι 
τῆς νυκτὸς τὴν πόλιν. 



Flavius Arrianus Hist., Phil., Alexandri anabasis 
Book 5, chapter 23, section 6, line 7

αὐτομολήσαντες δὲ αὐτῷ τῶν ἐκ τῆς πόλεώς τινες 
φράζουσιν, ὅτι ἐν νῷ ἔχοιεν αὐτῆς ἐκείνης τῆς νυκτὸς 
ἐκπίπτειν ἐκ τῆς πόλεως οἱ Ἰνδοὶ κατὰ τὴν λίμνην, 
ἵναπερ τὸ ἐκλιπὲς ἦν τοῦ χάρακος. 



Flavius Arrianus Hist., Phil., Alexandri anabasis 
Book 5, chapter 24, section 4, line 2

Ἐν τούτῳ δὲ καὶ Πῶρος ἀφίκετο τούς τε ὑπολοί-
πους ἐλέφαντας ἅμα οἷ ἄγων καὶ τῶν Ἰνδῶν ἐς πεντα-
κισχιλίους, αἵ τε μηχαναὶ Ἀλεξάνδρῳ ξυμπεπηγμέναι 
ἦσαν καὶ προσήγοντο ἤδη τῷ τείχει. 



Flavius Arrianus Hist., Phil., Alexandri anabasis 
Book 5, chapter 24, section 5, line 2

                     καὶ ἀποθνήσκουσι μὲν ἐν τῇ 
καταλήψει τῶν Ἰνδῶν ἐς μυρίους καὶ ἑπτακισχιλίους, 
ἑάλωσαν δὲ ὑπὲρ τὰς ἑπτὰ μυριάδας καὶ ἅρματα τρια-
κόσια καὶ ἵπποι πεντακόσιοι. 



Flavius Arrianus Hist., Phil., Alexandri anabasis 
Book 5, chapter 24, section 6, line 8

Θάψας δὲ ὡς νόμος αὐτῷ τοὺς τελευτήσαντας 
Εὐμενῆ τὸν γραμματέα ἐκπέμπει ἐς τὰς δύο πόλεις 
τὰς ξυναφεστώσας τοῖς Σαγγάλοις δοὺς αὐτῷ τῶν 
ἱππέων ἐς τριακοσίους, φράσοντα[ς] τοῖς ἔχουσι τὰς 
πόλεις τῶν τε Σαγγάλων τὴν ἅλωσιν καὶ ὅτι αὐτοῖς 
οὐδὲν ἔσται χαλεπὸν <ἐξ> Ἀλεξάνδρου ὑπομένουσί τε   
καὶ δεχομένοις φιλίως Ἀλέξανδρον· οὐδὲ γὰρ οὐδὲ 
ἄλλοις τισὶ γενέσθαι τῶν αὐτονόμων Ἰνδῶν ὅσοι 
ἑκόντες σφᾶς ἐνέδοσαν. 



Flavius Arrianus Hist., Phil., Alexandri anabasis 
Book 5, chapter 24, section 8, line 4

                                       ὡς δὲ ἀπέγνω 
διώκειν τοῦ πρόσω τοὺς φεύγοντας, ἐπανελθὼν ἐς τὰ 
Σάγγαλα τὴν πόλιν μὲν κατέσκαψε, τὴν χώραν δὲ 
τῶν Ἰνδῶν τοῖς πάλαι μὲν αὐτονόμοις, τότε δὲ ἑκουσίως 
προσχωρήσασι προσέθηκεν. 



Flavius Arrianus Hist., Phil., Alexandri anabasis 
Book 5, chapter 24, section 8, line 9

                           καὶ Πῶρον μὲν ξὺν τῇ 
δυνάμει τῇ ἀμφ' αὐτὸν ἐκπέμπει ἐπὶ τὰς πόλεις αἳ 
προσκεχωρήκεσαν, φρουρὰς εἰσάξοντα εἰς αὐτάς, αὐτὸς 
δὲ ξὺν τῇ στρατιᾷ ἐπὶ τὸν Ὕφασιν ποταμὸν προὐχώρει, 
ὡς καὶ τοὺς ἐπέκεινα Ἰνδοὺς καταστρέψαιτο. 



Flavius Arrianus Hist., Phil., Alexandri anabasis 
Book 5, chapter 25, section 2, line 1

Τὰ δὲ δὴ πέραν τοῦ Ὑφάσιος εὐδαίμονά τε τὴν 
χώραν εἶναι ἐξηγγέλλετο καὶ ἀνθρώπους ἀγαθοὺς μὲν 
γῆς ἐργάτας, γενναίους δὲ τὰ πολέμια καὶ ἐς τὰ ἴδια 
δὲ σφῶν ἐν κόσμῳ πολιτεύοντας (πρὸς γὰρ τῶν 
ἀρίστων ἄρχεσθαι τοὺς πολλούς, τοὺς δὲ οὐδὲν ἔξω   
τοῦ ἐπιεικοῦς ἐξηγεῖσθαι), πλῆθός τε ἐλεφάντων εἶναι 
τοῖς ταύτῃ ἀνθρώποις πολύ τι ὑπὲρ τοὺς ἄλλους 
Ἰνδοὺς, καὶ μεγέθει μεγίστους καὶ ἀνδρείᾳ. 



Flavius Arrianus Hist., Phil., Alexandri anabasis 
Book 5, chapter 25, section 5, line 7

καὶ Αἴγυπτος ξὺν τῇ Λιβύῃ τῇ Ἑλληνικῇ καὶ Ἀραβίας 
ἔστιν ἃ καὶ Συρία ἥ τε κοίλη καὶ ἡ μέση τῶν ποταμῶν, 
καὶ Βαβυλὼν δὲ ἔχεται καὶ τὸ Σουσίων ἔθνος καὶ 
Πέρσαι καὶ Μῆδοι καὶ ὅσων Πέρσαι καὶ Μῆδοι   
ἐπῆρχον, καὶ ὅσων δὲ οὐκ ἦρχον, τὰ ὑπὲρ τὰς Κασπίας 
πύλας, τὰ ἐπέκεινα τοῦ Καυκάσου, ὁ Τάναϊς, τὰ πρόσω 
ἔτι τοῦ Τανάϊδος, Βακτριανοί, Ὑρκάνιοι, ἡ θάλασσα 
ἡ Ὑρκανία, Σκύθας τε ἀνεστείλαμεν ἔστε ἐπὶ τὴν 
ἔρημον, ἐπὶ τούτοις μέντοι καὶ ὁ Ἰνδὸς ποταμὸς διὰ 
τῆς ἡμετέρας ῥεῖ, ὁ Ὑδάσπης διὰ τῆς ἡμετέρας, ὁ 
Ἀκεσίνης, ὁ Ὑδραώτης, τί ὀκνεῖτε καὶ τὸν Ὕφασιν καὶ 
τὰ ἐπέκεινα τοῦ Ὑφάσιος γένη προσθεῖναι τῇ ἡμετέρᾳ 
Μακεδόνων τε ἀρχῇ; 



Flavius Arrianus Hist., Phil., Alexandri anabasis 
Book 5, chapter 26, section 2, line 3

                                                     καὶ 
ἐγὼ ἐπιδείξω Μακεδόσι τε καὶ τοῖς ξυμμάχοις τὸν μὲν 
Ἰνδικὸν κόλπον ξύρρουν ὄντα τῷ Περσικῷ, τὴν δὲ 
Ὑρκανίαν <θάλασσαν> τῷ Ἰνδικῷ· ἀπὸ δὲ τοῦ Περ-
σικοῦ εἰς Λιβύην περιπλευσθήσεται στόλῳ ἡμετέρῳ 
τὰ μέχρι Ἡρακλέους Στηλῶν· ἀπὸ δὲ Στηλῶν ἡ ἐντὸς 
Λιβύη πᾶσα ἡμετέρα γίγνεται καὶ ἡ Ἀσία δὴ οὕτω 
πᾶσα, καὶ ὅροι τῆς ταύτῃ ἀρχῆς οὕσπερ καὶ τῆς γῆς 
ὅρους ὁ θεὸς ἐποίησε. 



Flavius Arrianus Hist., Phil., Alexandri anabasis 
Book 5, chapter 27, section 7, line 9

                                          σὺ δὲ νῦν μὴ 
ἄγειν ἄκοντας· οὐδὲ γὰρ ὁμοίοις ἔτι χρήσῃ ἐς τοὺς 
κινδύνους, οἷς τὸ ἑκούσιον ἐν τοῖς ἀγῶσιν ἀπέσται· 
ἐπανελθὼν δὲ αὐτός [τε], εἰ δοκεῖ, ἐς τὴν οἰκ<ε>ίαν 
καὶ τὴν μητέρα τὴν σαυτοῦ ἰδὼν καὶ τὰ τῶν Ἑλλήνων 
καταστησάμενος καὶ τὰς νίκας ταύτας τὰς πολλὰς καὶ 
μεγάλας ἐς τὸν πατρῷον οἶκον κομίσας οὕτω δὴ ἐξ 
ἀρχῆς ἄλλον στόλον στέλλεσθαι, εἰ μὲν βούλει, ἐπ' 
αὐτὰ ταῦτα τὰ πρὸς τὴν ἕω ᾠκισμένα Ἰνδῶν γένη, 
εἰ δὲ βούλει, ἐς τὸν Εὔξεινον πόντον, εἰ δέ, ἐπὶ 
Καρχηδόνα καὶ τὰ ἐπέκεινα Καρχηδονίων τῆς Λιβύης. 



Flavius Arrianus Hist., Phil., Alexandri anabasis 
Book 5, chapter 29, section 4, line 4

Ἐν τούτῳ δὲ ἀφίκοντο πρὸς αὐτὸν Ἀρσάκης τε ὁ 
τῆς ὁμόρου Ἀβισάρῃ χώρας ὕπαρχος καὶ ὁ ἀδελφὸς 
Ἀβισάρου καὶ οἱ ἄλλοι οἰκεῖοι, δῶρά τε κομίζοντες ἃ 
μέγιστα παρ' Ἰνδοῖς καὶ τοὺς παρ' Ἀβισάρου ἐλέφαντας,   
ἀριθμὸν ἐς τριάκοντα· Ἀβισάρην γὰρ νόσῳ ἀδύνατον 
γενέσθαι ἐλθεῖν. 



Flavius Arrianus Hist., Phil., Alexandri anabasis 
Book 6, chapter 1, section 2, line 1

     πρότερον μέν γε ἐν τῷ Ἰνδῷ ποταμῷ κροκοδείλους 
ἰδών, μόνῳ τῶν ἄλλων ποταμῶν πλὴν Νείλου, πρὸς 
δὲ ταῖς ὄχθαις τοῦ Ἀκεσίνου κυάμους πεφυκότας 
ὁποίους ἡ γῆ ἐκφέρει ἡ Αἰγυπτία, καὶ [ὁ] ἀκούσας 
ὅτι ὁ Ἀκεσίνης ἐμβάλλει ἐς τὸν Ἰνδόν, ἔδοξεν ἐξευρη-
κέναι τοῦ Νείλου τὰς ἀρχάς, ὡς τὸν Νεῖλον ἐνθένδε 
ποθὲν ἐξ Ἰνδῶν ἀνίσχοντα καὶ δι' ἐρήμου πολλῆς γῆς 
ῥέοντα καὶ ταύτῃ ἀπολλύοντα τὸν Ἰνδὸν τὸ ὄνομα, 
ἔπειτα, ὁπόθεν ἄρχεται διὰ τῆς οἰκουμένης χώρας ῥεῖν, 
Νεῖλον ἤδη πρὸς Αἰθιόπων τε τῶν ταύτῃ καὶ

Αἰγυ-



Flavius Arrianus Hist., Phil., Alexandri anabasis 
Book 6, chapter 1, section 4, line 2

                   καὶ δὴ καὶ πρὸς Ὀλυμπιάδα γράφοντα   
ὑπὲρ τῶν Ἰνδῶν τῆς γῆς ἄλλα τε γράψαι καὶ ὅτι 
δοκοίη αὑτῷ ἐξευρηκέναι τοῦ Νείλου τὰς πηγάς, μικροῖς 
δή τισι καὶ φαύλοις ὑπὲρ τῶν τηλικούτων τεκμαιρό-
μενον. 



Flavius Arrianus Hist., Phil., Alexandri anabasis 
Book 6, chapter 1, section 5, line 2

       ἐπεὶ μέντοι ἀτρεκέστερον ἐξήλεγξε τὰ ἀμφὶ 
τῷ ποταμῷ τῷ Ἰνδῷ, οὕτω δὴ μαθεῖν παρὰ τῶν ἐπι-
χωρίων τὸν μὲν Ὑδάσπην τῷ Ἀκεσίνῃ, τὸν Ἀκεσίνην 
δὲ τῷ Ἰνδῷ τό τε ὕδωρ ξυμβάλλοντας καὶ τῷ ὀνόματι 
ξυγχωροῦντας, τὸν Ἰνδὸν δὲ ἐκδιδόντα ἤδη ἐς τὴν 
μεγάλην θάλασσαν, δίστομον τὸν Ἰνδὸν ὄντα, οὐδέ<ν> τι 
αὐτῷ προσῆκον τῆς γῆς τῆς Αἰγυπτίας· τηνικαῦτα δὲ 
τῆς ἐπιστολῆς τῆς πρὸς τὴν μητέρα τοῦτο <τὸ> ἀμφὶ τῷ 
Νείλῳ γραφὲν ἀφελεῖν. 



Flavius Arrianus Hist., Phil., Alexandri anabasis 
Book 6, chapter 2, section 1, line 4

                                          αὐτὸς δὲ 
ξυναγαγὼν τούς τε ἑταίρους καὶ ὅσοι Ἰνδῶν πρέσβεις 
παρ' αὐτὸν ἀφιγμένοι ἦσαν βασιλέα μὲν τῆς ἑαλωκυίας 
ἤδη Ἰνδῶν γῆς ἀπέδειξε Πῶρον, ἑπτὰ μὲν ἐθνῶν τῶν 
ξυμπάντων, πόλεων δὲ ἐν τοῖς ἔθνεσιν ὑπὲρ τὰς 
δισχιλίας. 



Flavius Arrianus Hist., Phil., Alexandri anabasis 
Book 6, chapter 2, section 3, line 2

                                           Φιλίππῳ δὲ τῷ 
σατράπῃ τῆς ἐπέκεινα τοῦ Ἰνδοῦ ὡς ἐπὶ Βακτρίους 
γῆς διαλιπόντι τρεῖς ἡμέρας παρήγγελτο ἕπεσθαι ξὺν 
τοῖς ἀμφ' αὐτόν. 



Flavius Arrianus Hist., Phil., Alexandri anabasis 
Book 6, chapter 3, section 1, line 9

                          καὶ ἐπιβὰς τῆς νεὼς ἀπὸ τῆς 
πρώρας ἐκ χρυσῆς φιάλης ἔσπενδεν ἐς τὸν ποταμόν, 
τόν τε Ἀκεσίνην ξυνεπικαλούμενος τῷ Ὑδάσπῃ, ὅντινα 
μέγιστον αὖ τῶν ἄλλων ποταμῶν ξυμβάλλειν τῷ 
Ὑδάσπῃ ἐπέπυστο καὶ οὐ πόρρω αὐτῶν εἶναι τὰς ξυμ-
βολάς, καὶ τὸν Ἰνδόν, ἐς ὅντινα ὁ Ἀκεσίνης ξὺν τῷ 
Ὑδάσπῃ ἐμβάλλει. 



Flavius Arrianus Hist., Phil., Alexandri anabasis 
Book 6, chapter 3, section 4, line 3

ἐνδιδόντων τὰς ἀρχάς τε καὶ ἀναπαύλας τῇ εἰρεσίᾳ 
καὶ τῶν ἐρετῶν ὁπότε ἀθρόοι ἐμπίπτοντες τῷ ῥοθίῳ 
ἐπαλαλάξειαν· αἵ τε ὄχθαι, ὑψηλότεραι τῶν νεῶν 
πολλαχῇ οὖσαι, ἐς στενόν τε τὴν βοὴν ξυνάγουσαι 
καὶ τῇ ξυναγωγῇ αὐτῇ ἐπὶ μέγα ηὐξημένην ἐς ἀλλήλας 
ἀντέπεμπον, καί που καὶ νάπαι ἑκατέρωθεν τοῦ 
ποταμοῦ τῇ τε ἐρημίᾳ καὶ τῇ ἀντιπέμψει τοῦ κτύπου 
καὶ αὗται ξυνεπελάμβανον· οἵ τε ἵπποι διαφαινόμενοι 
διὰ τῶν ἱππαγωγῶν πλοίων, οὐ πρόσθεν ἵπποι ἐπὶ 
νεῶν ὀφθέντες ἐν τῇ Ἰνδῶν γῇ (καὶ γὰρ καὶ τὸν 
Διονύσου ἐπ' Ἰνδοὺς στόλον οὐκ ἐμέμνηντο γενέσθαι 
ναυτικόν), ἔκπληξιν παρεῖχον τοῖς θεωμένοις τῶν   
βαρβάρων, ὥστε οἱ μὲν αὐτόθεν τῇ ἀναγωγῇ παρα-
γενόμενοι ἐπὶ πολὺ ἐφωμάρτουν, ἐς ὅσους δὲ τῶν 
ἤδη Ἀλεξάνδρῳ προσκεχωρηκότων Ἰνδῶν ἡ βοὴ τῶν 
ἐρετῶν ἢ ὁ κτύπος τῆς εἰρεσίας ἐξίκετο, καὶ οὗτοι 
ἐπὶ τῇ ὄχθῃ κατέθεον καὶ ξυνείποντο ἐπᾴδοντες βαρβα-
ρικῶς. 



Flavius Arrianus Hist., Phil., Alexandri anabasis 
Book 6, chapter 3, section 4, line 4

καὶ τῶν ἐρετῶν ὁπότε ἀθρόοι ἐμπίπτοντες τῷ ῥοθίῳ 
ἐπαλαλάξειαν· αἵ τε ὄχθαι, ὑψηλότεραι τῶν νεῶν 
πολλαχῇ οὖσαι, ἐς στενόν τε τὴν βοὴν ξυνάγουσαι 
καὶ τῇ ξυναγωγῇ αὐτῇ ἐπὶ μέγα ηὐξημένην ἐς ἀλλήλας 
ἀντέπεμπον, καί που καὶ νάπαι ἑκατέρωθεν τοῦ 
ποταμοῦ τῇ τε ἐρημίᾳ καὶ τῇ ἀντιπέμψει τοῦ κτύπου 
καὶ αὗται ξυνεπελάμβανον· οἵ τε ἵπποι διαφαινόμενοι 
διὰ τῶν ἱππαγωγῶν πλοίων, οὐ πρόσθεν ἵπποι ἐπὶ 
νεῶν ὀφθέντες ἐν τῇ Ἰνδῶν γῇ (καὶ γὰρ καὶ τὸν 
Διονύσου ἐπ' Ἰνδοὺς στόλον οὐκ ἐμέμνηντο γενέσθαι 
ναυτικόν), ἔκπληξιν παρεῖχον τοῖς θεωμένοις τῶν   
βαρβάρων, ὥστε οἱ μὲν αὐτόθεν τῇ ἀναγωγῇ παρα-
γενόμενοι ἐπὶ πολὺ ἐφωμάρτουν, ἐς ὅσους δὲ τῶν 
ἤδη Ἀλεξάνδρῳ προσκεχωρηκότων Ἰνδῶν ἡ βοὴ τῶν 
ἐρετῶν ἢ ὁ κτύπος τῆς εἰρεσίας ἐξίκετο, καὶ οὗτοι 
ἐπὶ τῇ ὄχθῃ κατέθεον καὶ ξυνείποντο ἐπᾴδοντες βαρβα-
ρικῶς. 



Flavius Arrianus Hist., Phil., Alexandri anabasis 
Book 6, chapter 3, section 5, line 2

ἀντέπεμπον, καί που καὶ νάπαι ἑκατέρωθεν τοῦ 
ποταμοῦ τῇ τε ἐρημίᾳ καὶ τῇ ἀντιπέμψει τοῦ κτύπου 
καὶ αὗται ξυνεπελάμβανον· οἵ τε ἵπποι διαφαινόμενοι 
διὰ τῶν ἱππαγωγῶν πλοίων, οὐ πρόσθεν ἵπποι ἐπὶ 
νεῶν ὀφθέντες ἐν τῇ Ἰνδῶν γῇ (καὶ γὰρ καὶ τὸν 
Διονύσου ἐπ' Ἰνδοὺς στόλον οὐκ ἐμέμνηντο γενέσθαι 
ναυτικόν), ἔκπληξιν παρεῖχον τοῖς θεωμένοις τῶν   
βαρβάρων, ὥστε οἱ μὲν αὐτόθεν τῇ ἀναγωγῇ παρα-
γενόμενοι ἐπὶ πολὺ ἐφωμάρτουν, ἐς ὅσους δὲ τῶν 
ἤδη Ἀλεξάνδρῳ προσκεχωρηκότων Ἰνδῶν ἡ βοὴ τῶν 
ἐρετῶν ἢ ὁ κτύπος τῆς εἰρεσίας ἐξίκετο, καὶ οὗτοι 
ἐπὶ τῇ ὄχθῃ κατέθεον καὶ ξυνείποντο ἐπᾴδοντες βαρβα-
ρικῶς. 



Flavius Arrianus Hist., Phil., Alexandri anabasis 
Book 6, chapter 3, section 5, line 5

        φιλῳδοὶ γάρ, εἴπερ τινὲς ἄλλοι, Ἰνδοὶ καὶ 
φιλορχήμονες ἀπὸ Διονύσου ἔτι καὶ τῶν ἅμα Διονύσῳ 
βακχευσάντων κατὰ τὴν Ἰνδῶν γῆν. 



Flavius Arrianus Hist., Phil., Alexandri anabasis 
Book 6, chapter 4, section 2, line 5

                                                  προσορμιζό-
μενος δὲ ὅπῃ τύχοι ταῖς ὄχθαις τοὺς προσοικοῦντας 
τῷ Ὑδάσπῃ Ἰνδοὺς τοὺς μὲν ἐνδιδόντας σφᾶς ὁμο-
λογίαις παρελάμβανεν, ἤδη δέ τινας καὶ ἐς ἀλκὴν 
χωρήσαντας βίᾳ κατεστρέψατο. 



Flavius Arrianus Hist., Phil., Alexandri anabasis 
Book 6, chapter 4, section 3, line 3

                                 αὐτὸς δὲ ὡς ἐπὶ τὴν 
Μαλλῶν τε καὶ Ὀξυδρακῶν γῆν σπουδῇ ἔπλει, πλεί-
στους τε καὶ μαχιμωτάτους τῶν ταύτῃ Ἰνδῶν πυνθανό-
μενος καὶ ὅτι ἐξηγγέλλοντο αὐτῷ παῖδας μὲν καὶ 
γυναῖκας ἀποτεθεῖσθαι εἰς τὰς ὀχυρωτάτας τῶν πόλεων,   
αὐτοὶ δὲ ἐγνωκέναι διὰ μάχης ἰέναι πρὸς αὐτόν. 



Flavius Arrianus Hist., Phil., Alexandri anabasis 
Book 6, chapter 6, section 1, line 5

Αὐτὸς δὲ ἀναλαβὼν τοὺς ὑπασπιστάς τε καὶ τοὺς 
τοξότας καὶ τοὺς Ἀγριᾶνας καὶ τῶν πεζεταίρων καλου-
μένων τὴν Πείθωνος τάξιν καὶ τοὺς ἱπποτοξότας τε 
πάντας καὶ τῶν ἱππέων τῶν ἑταίρων τοὺς ἡμίσεας 
διὰ γῆς ἀνύδρου ὡς ἐπὶ Μαλλοὺς ἦγεν, ἔθνος Ἰνδικὸν 
Ἰνδῶν τῶν αὐτονόμων. 



Flavius Arrianus Hist., Phil., Alexandri anabasis 
Book 6, chapter 6, section 4, line 5

                         ὡς δὲ τάχιστα οἱ πεζοὶ ἀφίκοντο, 
Περδίκκαν μὲν τήν τε αὑτοῦ ἱππαρχίαν ἔχοντα καὶ 
τὴν Κλείτου καὶ τοὺς Ἀγριᾶνας πρὸς ἄλλην πόλιν 
ἐκπέμπει τῶν Μαλλῶν, οἷ ξυμπεφευγότες ἦσαν πολλοὶ 
τῶν ταύτῃ Ἰνδῶν, φυλάσσειν τοὺς ἐν τῇ πόλει κελεύ-
σας, ἔργου δὲ μὴ ἔχεσθαι ἔστ' ἂν ἀφίκηται αὐτός, ὡς 
μηδὲ ἀπὸ ταύτης τῆς πόλεως διαφυγόντας τινὰς 
αὐτῶν ἀγγέλους γενέσθαι τοῖς ἄλλοις βαρβάροις ὅτι 
προσάγει ἤδη Ἀλέξανδρος· αὐτὸς δὲ προσέβαλλεν τῷ 
τείχει. 



Flavius Arrianus Hist., Phil., Alexandri anabasis 
Book 6, chapter 7, section 6, line 3

                                           εἴχετό τε ἤδη ἡ ἄκρα, 
καὶ τῶν Ἰνδῶν οἱ μὲν τὰς οἰκίας ἐνεπίμπρασαν καὶ ἐν 
αὐταῖς ἐγκαταλαμβανόμενοι ἀπέθνησκον, οἱ πολλοὶ δὲ 
μαχόμενοι αὐτῶν. 



Flavius Arrianus Hist., Phil., Alexandri anabasis 
Book 6, chapter 8, section 4, line 4

                                   ἀλλὰ καὶ ταύτην ἐξέλι-
πον οἱ Ἰνδοὶ ὡς προσάγοντα Ἀλέξανδρον ἔμαθον. 



Flavius Arrianus Hist., Phil., Alexandri anabasis 
Book 6, chapter 8, section 6, line 5

                       ὡς δὲ κατεῖδον ἱππέας μόνους, ἐπι-
στρέψαντες οἱ Ἰνδοὶ καρτερῶς ἐμάχοντο πλῆθος ὄντες 
ἐς πέντε μυριάδας. 



Flavius Arrianus Hist., Phil., Alexandri anabasis 
Book 6, chapter 8, section 7, line 1

                      καὶ Ἀλέξανδρος ὡς τήν τε φάλαγγα   
αὐτῶν πυκνὴν κατεῖδε καὶ αὐτῷ οἱ πεζοὶ ἀπῆσαν, 
προσβολὰς μὲν ἐποιεῖτο ἐς κύκλους παριππεύων, ἐς 
χεῖρας δὲ οὐκ ᾔει τοῖς Ἰνδοῖς. 



Flavius Arrianus Hist., Phil., Alexandri anabasis 
Book 6, chapter 8, section 7, line 5

                                                         καὶ 
οἱ Ἰνδοὶ ὁμοῦ σφισι πάντων τῶν δεινῶν προσκειμένων 
ἀποστρέψαντες ἤδη προτροπάδην ἔφευγον ἐς πόλιν 
ὀχυρωτάτην τῶν πλησίον. 



Flavius Arrianus Hist., Phil., Alexandri anabasis 
Book 6, chapter 9, section 1, line 4

                               καὶ ἐν τούτῳ οὐ δεξά-
μενοι οἱ Ἰνδοὶ τῶν Μακεδόνων τὴν ὁρμὴν τὰ μὲν 
τείχη τῆς πόλεως λείπουσιν, αὐτοὶ δὲ ἐς τὴν ἄκραν 
ξυνέφευγον. 



Flavius Arrianus Hist., Phil., Alexandri anabasis 
Book 6, chapter 9, section 4, line 3

                    ἤδη τε πρὸς τῇ ἐπάλξει τοῦ 
τείχους ὁ βασιλεὺς ἦν καὶ ἐρείσας ἐπ' αὐτῇ τὴν ἀσπίδα 
τοὺς μὲν ὤθει εἴσω τοῦ τείχους τῶν Ἰνδῶν, τοὺς δὲ 
καὶ αὐτοῦ τῷ ξίφει ἀποκτείνας γεγυμνώκει τὸ ταύτῃ 
τεῖχος· καὶ οἱ ὑπασπισταὶ ὑπέρφοβοι γενόμενοι ὑπὲρ 
τοῦ βασιλέως σπουδῇ ὠθούμενοι κατὰ τὴν αὐτὴν κλί-
μακα συντρίβουσιν αὐτήν, ὥστε οἱ μὲν ἤδη ἀνιόντες 
αὐτῶν κάτω ἔπεσον, τοῖς δὲ ἄλλοις ἄπορον ἐποίησαν 
τὴν ἄνοδον. 



Flavius Arrianus Hist., Phil., Alexandri anabasis 
Book 6, chapter 9, section 5, line 3

Ἀλέξανδρος δὲ ὡς ἐπὶ τοῦ τείχους στὰς κύκλῳ τε 
ἀπὸ τῶν πλησίον πύργων ἐβάλλετο, οὐ γὰρ πελάσαι 
γε ἐτόλμα τις αὐτῷ τῶν Ἰνδῶν, καὶ ὑπὸ τῶν ἐκ τῆς 
πόλεως, οὐδὲ πόρρω τούτων γε ἐσακοντιζόντων (ἔτυχε   
γάρ τι καὶ προσκεχωσμένον ταύτῃ πρὸς τὸ τεῖχος), 
δῆλος μὲν ἦν Ἀλέξανδρος ὢν τῶν τε ὅπλων τῇ λαμ-
πρότητι καὶ τῷ ἀτόπῳ τῆς τόλμης, ἔγνω δὲ ὅτι αὐτοῦ 
μὲν μένων κινδυνεύσει μηδὲν ὅ τι καὶ λόγου ἄξιον 
ἀποδεικνύμενος, καταπηδήσας δὲ εἴσω τοῦ τείχους 
τυχὸν μὲν αὐτῷ τούτῳ ἐκπλήξει τοὺς Ἰνδούς, εἰ δὲ 
μή, καὶ κινδυνεύειν δέοι, μεγάλα ἔργα καὶ τοῖς ἔπειτα 
πυθέσθαι ἄξια ἐργασάμενος οὐκ ἀσπουδεὶ

ἀποθανεῖ-



Flavius Arrianus Hist., Phil., Alexandri anabasis 
Book 6, chapter 9, section 6, line 3

         ἔνθα δὴ ἐρεισθεὶς πρὸς τῷ τείχει τοὺς μέν 
τινας ἐς χεῖρας ἐλθόντας καὶ τόν γε ἡγεμόνα τῶν 
Ἰνδῶν προσφερόμενόν οἱ θρασύτερον παίσας τῷ ξίφει 
ἀποκτείνει· ἄλλον δὲ πελάζοντα λίθῳ βαλὼν ἔσχε καὶ 
ἄλλον λίθῳ, τὸν δὲ ἐγγυτέρω προσάγοντα τῷ ξίφει 
αὖθις. 



Flavius Arrianus Hist., Phil., Alexandri anabasis 
Book 6, chapter 11, section 1, line 1

Ἐν τούτῳ δὲ οἱ μὲν ἔκτεινον τοὺς Ἰνδούς, καὶ 
ἀπέκτεινάν γε πάντας οὐδὲ γυναῖκα ἢ παῖδα ὑπ-
ελείποντο, οἱ δὲ ἐξέφερον τὸν βασιλέα ἐπὶ τῆς ἀσπί-
δος κακῶς ἔχοντα, οὔπω γιγνώσκοντες βιώσιμον ὄντα. 



Flavius Arrianus Hist., Phil., Alexandri anabasis 
Book 6, chapter 11, section 3, line 3

Αὐτίκα ἐν Ὀξυδράκαις τὸ πάθημα τοῦτο γενέσθαι 
Ἀλεξάνδρῳ ὁ πᾶς λόγος κατέχει· τὸ δὲ ἐν Μαλλοῖς 
ἔθνει αὐτονόμῳ Ἰνδικῷ ξυνέβη, καὶ ἥ τε πόλις Μαλλῶν 
ἦν καὶ οἱ βαλόντες Ἀλέξανδρον Μαλλοί, οἳ δὴ ἐγνώ-
κεσαν μὲν ξυμμίξαντες τοῖς Ὀξυδράκαις οὕτω δια-
γωνίζεσθαι, ἔφθη δὲ διὰ τῆς ἀνύδρου ἐπ' αὐτοὺς 
ἐλάσας πρίν τινα ὠφέλειαν αὐτοῖς παρὰ τῶν Ὀξυδρα-  
κῶν γενέσθαι ἢ αὐτοὺς ἐκείνοις τι ἐπωφελῆσαι. 



Flavius Arrianus Hist., Phil., Alexandri anabasis 
Book 6, chapter 13, section 3, line 10

           οἱ δὲ ἐπέλαζον ἄλλος ἄλλοθεν, οἱ μὲν χειρῶν, 
οἱ δὲ γονάτων, οἱ δὲ τῆς ἐσθῆτος αὐτῆς ἁπτόμενοι, 
οἱ δὲ καὶ ἰδεῖν ἐγγύθεν καί τι καὶ ἐπευφημήσαντες 
ἀπελθεῖν· οἱ δὲ ταινίαις ἔβαλλον, οἱ δὲ ἄνθεσιν, ὅσα 
ἐν τῷ τότε ἡ Ἰνδῶν γῆ παρεῖχε. 



Flavius Arrianus Hist., Phil., Alexandri anabasis 
Book 6, chapter 14, section 1, line 6

Ἐν τούτῳ δὲ ἀφίκοντο παρὰ Ἀλέξανδρον τῶν 
Μαλλῶν τῶν ὑπολειπομένων πρέσβεις ἐνδιδόντες τὸ 
ἔθνος, καὶ παρὰ Ὀξυδρακῶν οἵ τε ἡγεμόνες τῶν πόλεων 
καὶ οἱ νομάρχαι αὐτοὶ καὶ ἄλλοι ἅμα τούτοις ἑκατὸν 
καὶ πεντήκοντα οἱ γνωριμώτατοι αὐτοκράτορες περὶ 
σπονδῶν δῶρά τε ὅσα μέγιστα παρ' Ἰνδοῖς κομίζοντες 
καὶ τὸ ἔθνος καὶ οὗτοι ἐνδιδόντες. 



Flavius Arrianus Hist., Phil., Alexandri anabasis 
Book 6, chapter 14, section 2, line 5

                                            συγγνωστὰ δὲ 
ἁμαρτεῖν ἔφασαν οὐ πάλαι παρ' αὐτὸν πρεσβευσά-
μενοι· ἐπιθυμεῖν γάρ, ὥσπερ τινὲς ἄλλοι, ἔτι μᾶλλον 
αὐτοὶ ἐλευθερίας τε καὶ αὐτόνομοι εἶναι, ἥντινα ἐλευ-
θερίαν ἐξ ὅτου Διόνυσος ἐς Ἰνδοὺς ἧκε σώαν σφίσιν 
εἶναι ἐς Ἀλέξανδρον· εἰ δὲ Ἀλεξάνδρῳ δοκοῦν ἐστιν, 
ὅτι καὶ Ἀλέξανδρον ἀπὸ θεοῦ γενέσθαι λόγος κατέχει, 
σατράπην τε ἀναδέξεσθαι, ὅντινα τάττοι Ἀλέξανδρος, 
καὶ φόρους ἀποίσειν τοὺς Ἀλεξάνδρῳ δόξαντας· διδόναι 
δὲ καὶ ὁμήρους ἐθέλειν ὅσους ἂν αἰτῇ Ἀλέξανδρος. 



Flavius Arrianus Hist., Phil., Alexandri anabasis 
Book 6, chapter 14, section 3, line 4

ὁ δὲ χιλίους ᾔτησε τοὺς κρατιστεύοντας τοῦ ἔθνους, 
οὕς, εἰ μὲν βούλοιτο, ἀντὶ ὁμήρων καθέξειν, εἰ δὲ 
μή, ξυστρατεύοντας ἕξειν ἔστ' ἂν διαπολεμηθῇ αὐτῷ 
πρὸς τοὺς ἄλλους Ἰνδούς. 



Flavius Arrianus Hist., Phil., Alexandri anabasis 
Book 6, chapter 14, section 5, line 1

Ὡς δὲ ταῦτα αὐτῷ κεκόσμητο καὶ πλοῖα ἐπὶ τῇ 
διατριβῇ τῇ ἐκ τοῦ τραύματος πολλὰ προσενεναυπή-
γητο, ἀναβιβάσας ἐς τὰς ναῦς τῶν μὲν ἑταίρων ἱππέας 
ἑπτακοσίους καὶ χιλίους, τῶν ψιλῶν δὲ ὅσουσπερ καὶ 
πρότερον, πεζοὺς δὲ ἐς μυρίους, ὀλίγον μέν τι τῷ 
Ὑδραώτῃ ποταμῷ κατέπλευσεν, ὡς δὲ συνέμιξεν ὁ 
Ὑδραώτης τῷ Ἀκεσίνῃ, ὅτι ὁ Ἀκεσίνης κρατεῖ τοῦ 
Ὑδραώτου [ἐν] τῇ ἐπωνυμίᾳ, κατὰ τὸν Ἀκεσίνην αὖ 
ἔπλει, ἔστε ἐπὶ τὴν ξυμβολὴν τοῦ Ἀκεσίνου καὶ τοῦ 
Ἰνδοῦ ἧκεν. 



Flavius Arrianus Hist., Phil., Alexandri anabasis 
Book 6, chapter 14, section 5, line 2

                 τέσσαρες γὰρ οὗτοι μεγάλοι ποταμοὶ καὶ 
ναυσίποροι οἱ τέσσαρες εἰς τὸν Ἰνδὸν ποταμὸν τὸ 
ὕδωρ ξυμβάλλουσιν, οὐ ξὺν τῇ σφετέρᾳ ἕκαστος ἐπω-
νυμίᾳ, ἀλλὰ ὁ Ὑδάσπης μὲν ἐς τὸν Ἀκεσίνην ἐμβάλλει, 
ἐμβαλὼν δὲ τὸ πᾶν ὕδωρ Ἀκεσίνην παρέχεται καλού-
μενον· αὖθις δὲ ὁ Ἀκεσίνης οὗτος ξυμβάλλει τῷ 
Ὑδραώτῃ, καὶ παραλαβὼν τοῦτον ἔτι Ἀκεσίνης ἐστί·   
καὶ τὸν Ὕφασιν ἐπὶ τούτῳ ὁ Ἀκεσίνης παραλαβὼν τῷ 
αὑτοῦ δὴ ὀνόματι ἐς τὸν Ἰνδὸν ἐμβάλλει· ξυμβαλὼν 
δὲ ξυγχωρεῖ δὴ τῷ Ἰνδῷ. 



Flavius Arrianus Hist., Phil., Alexandri anabasis 
Book 6, chapter 14, section 5, line 10

                                 ἔνθεν δὴ ὁ Ἰνδὸς πρὶν ἐς 
τὸ Δέλτα σχισθῆναι οὐκ ἀπιστῶ ὅτι καὶ ἐς ἑκατὸν 
σταδίους ἔρχεται καὶ ὑπὲρ τοὺς ἑκατὸν τυχὸν, ἵναπερ 
λιμνάζει μᾶλλον. 



Flavius Arrianus Hist., Phil., Alexandri anabasis 
Book 6, chapter 15, section 1, line 2

Ἐνταῦθα ἐπὶ ταῖς ξυμβολαῖς τοῦ Ἀκεσίνου καὶ 
Ἰνδοῦ προσέμενεν ἔστε ἀφίκετο αὐτῷ ξὺν τῇ στρατιᾷ 
Περδίκκας καταστρεψάμενος ἐν παρόδῳ τὸ Ἀβαστανῶν 
ἔθνος αὐτόνομον. 



Flavius Arrianus Hist., Phil., Alexandri anabasis 
Book 6, chapter 15, section 1, line 7

                     ἐν τούτῳ δὲ ἄλλαι τε προσγίγνονται 
Ἀλεξάνδρῳ τριακόντοροι καὶ πλοῖα στρογγύλα ἄλλα, ἃ δὴ 
ἐν Ξάθροις ἐναυπηγήθη αὐτῷ, καὶ <Σόγδοι> ἄλλο ἔθνος 
Ἰνδῶν αὐτόνομον προσεχώρησαν. 



Flavius Arrianus Hist., Phil., Alexandri anabasis 
Book 6, chapter 15, section 1, line 8

                                    καὶ παρὰ Ὀσσαδίων, 
καὶ τούτου γένους αὐτονόμου Ἰνδικοῦ, πρέσβεις ἧκον, 
ἐνδιδόντες καὶ οὗτοι τοὺς Ὀσσαδίους. 



Flavius Arrianus Hist., Phil., Alexandri anabasis 
Book 6, chapter 15, section 2, line 3

                                              Φιλίππῳ μὲν δὴ 
τῆς σατραπείας ὅρους ἔταξε τὰς συμβολὰς τοῦ τε Ἀκεσί-
νου καὶ Ἰνδοῦ καὶ ἀπολείπει ξὺν αὐτῷ τούς τε Θρᾷκας 
πάντας καὶ ἐκ τῶν τάξεων ὅσοι ἐς φυλακὴν τῆς χώρας 
ἱκανοὶ ἐφαίνοντο, πόλιν τε ἐνταῦθα κτίσαι ἐκέλευσεν ἐπ' 
αὐτῇ τῇ ξυμβολῇ τοῖν ποταμοῖν, ἐλπίσας μεγάλην τε 
ἔσεσθαι καὶ ἐπιφανῆ ἐς ἀνθρώπους, καὶ νεωσοίκους 
ποιηθῆναι. 



Flavius Arrianus Hist., Phil., Alexandri anabasis 
Book 6, chapter 15, section 4, line 3

Ἔνθα δὴ διαβιβάσας Κρατερόν τε καὶ τῆς στρα-
τιᾶς τὴν πολλὴν καὶ τοὺς ἐλέφαντας ἐπ' ἀριστερὰ τοῦ 
Ἰνδοῦ ποταμοῦ, ὅτι εὐπορώτερά τε ταύτῃ τὰ παρὰ 
τὸν ποταμὸν στρατιᾷ βαρείᾳ ἐφαίνετο καὶ τὰ ἔθνη τὰ 
προσοικοῦντα οὐ πάντῃ φίλια ἦν, αὐτὸς κατέπλει ἐς 
τῶν Σόγδων τὸ βασίλειον. 



Flavius Arrianus Hist., Phil., Alexandri anabasis 
Book 6, chapter 15, section 4, line 9

                                              τῆς δὲ ἀπὸ 
τῶν ξυμβολῶν τοῦ τε Ἰνδοῦ καὶ Ἀκεσίνου χώρας ἔστε 
ἐπὶ θάλασσαν σατράπην ἀπέδειξε[ν Ὀξυάρτην καὶ] 
Πείθωνα ξὺν τῇ παραλίᾳ πάσῃ τῆς Ἰνδῶν γῆς. 



Flavius Arrianus Hist., Phil., Alexandri anabasis 
Book 6, chapter 15, section 5, line 4

Καὶ Κρατερὸν μὲν ἐκπέμπει αὖθις ξὺν τῇ στρατιᾷ 
[διὰ τῆς Ἀραχωτῶν καὶ Δραγγῶν γῆς], αὐτὸς δὲ 
κατέπλει ἐς τὴν Μουσικανοῦ ἐπικράτειαν, ἥντινα 
εὐδαιμονεστάτην τῆς Ἰνδῶν γῆς εἶναι ἐξηγγέλλετο, 
ὅτι οὔπω οὔτε ἀπηντήκει αὐτῷ Μουσικανὸς ἐνδιδοὺς 
αὑτόν τε καὶ τὴν χώραν οὔτε πρέσβεις ἐπὶ φιλίᾳ 
ἐκπέμπει, οὐδέ τι οὔτε αὐτὸς ἐπεπόμφει ἃ δὴ μεγάλῳ 
βασιλεῖ εἰκός, οὔτε τι ᾐτήκει ἐξ Ἀλεξάνδρου. 



Flavius Arrianus Hist., Phil., Alexandri anabasis 
Book 6, chapter 15, section 6, line 7

                                                οὕτω δὴ 
ἐκπλαγεὶς κατὰ τάχος ἀπήντα Ἀλεξάνδρῳ, δῶρά τε τὰ 
πλείστου ἄξια παρ' Ἰνδοῖς κομίζων καὶ τοὺς ἐλέφαντας 
ξύμπαντας ἄγων καὶ τὸ ἔθνος τε καὶ αὑτὸν ἐνδιδοὺς 
καὶ ὁμολογῶν ἀδικεῖν, ὅπερ μέγιστον παρ' Ἀλεξάνδρῳ 
ἦν ἐς τὸ τυχεῖν ὧν τις δέοιτο. 



Flavius Arrianus Hist., Phil., Alexandri anabasis 
Book 6, chapter 16, section 2, line 7

                                 ὁ δὲ τὴν μὲν λείαν τῇ 
στρατιᾷ δίδωσι, τοὺς ἐλέφαντας δὲ ἅμα οἷ ἦγε· καὶ <αἱ> 
ἄλλαι δὲ πόλεις αὐτῷ αἱ ἐν τῇ αὐτῇ χώρᾳ ἐνεδίδοντο 
ἐπιόντι οὐδέ τις ἐτρέπετο ἐς ἀλκήν· οὕτω καὶ Ἰνδοὶ 
πάντες ἐδεδούλωντο ἤδη τῇ γνώμῃ πρὸς Ἀλεξάνδρου 
τε καὶ τῆς Ἀλεξάνδρου τύχης. 



Flavius Arrianus Hist., Phil., Alexandri anabasis 
Book 6, chapter 16, section 3, line 1

Ὁ δὲ ἐπὶ Σάμβον αὖ ἦγε τῶν ὀρείων Ἰνδῶν 
σατράπην ὑπ' αὐτοῦ κατασταθέντα, ὃς πεφευγέναι 
αὐτῷ ἐξηγγέλλετο ὅτι Μουσικανὸν ἀφειμένον πρὸς 
Ἀλεξάνδρου ἐπύθετο καὶ τῆς χώρας τῆς ἑαυτοῦ ἄρχοντα· 
τὰ γὰρ πρὸς Μουσικανὸν αὐτῷ πολέμια ἦν. 



Flavius Arrianus Hist., Phil., Alexandri anabasis 
Book 6, chapter 16, section 5, line 2

ὁ δὲ καὶ ἄλλην πόλιν ἐν τούτῳ ἀποστᾶσαν εἷλεν καὶ 
τῶν Βραχμάνων, οἳ δὴ σοφισταὶ τοῖς Ἰνδοῖς εἰσιν, 
ὅσοι αἴτιοι τῆς ἀποστάσεως ἐγένοντο ἀπέκτεινεν. 



Flavius Arrianus Hist., Phil., Alexandri anabasis 
Book 6, chapter 16, section 5, line 4

                                                           ὑπὲρ 
ὧν ἐγὼ τῆς σοφίας, εἰ δή τίς ἐστιν, ἐν τῇ Ἰνδικῇ 
ξυγγραφῇ δηλώσω. 



Flavius Arrianus Hist., Phil., Alexandri anabasis 
Book 6, chapter 17, section 2, line 7

ἀφίκετο δὲ αὐτῷ καὶ ὁ τῶν Πατάλων τῆς χώρας 
ἄρχων, ὃ δὴ τὸ Δέλτα ἔφην εἶναι τὸ πρὸς τοῦ ποταμοῦ 
τοῦ Ἰνδοῦ ποιούμενον, μεῖζον ἔτι τοῦ Δέλτα τοῦ 
Αἰγυπτίου, καὶ οὗτος τήν τε χώραν αὐτῷ ἐνεδίδου 
πᾶσαν καὶ αὑτόν τε καὶ τὰ αὑτοῦ ἐπέτρεψεν. 



Flavius Arrianus Hist., Phil., Alexandri anabasis 
Book 6, chapter 17, section 4, line 5

            .... Ἡφαιστίων ἐπετάχθη, Πείθωνα δὲ 
τούς τε ἱππακοντιστὰς ἄγοντα καὶ τοὺς Ἀγριᾶνας ἐς 
τὴν ἐπέκεινα ὄχθην τοῦ Ἰνδοῦ διαβιβάσας, οὐχ ᾗπερ 
Ἡφαιστίων τὴν στρατιὰν ἄγειν ἤμελλε, τάς τε ἐκ-
τετειχισμένας ἤδη πόλεις ξυνοικίσαι ἐκέλευσε καὶ εἰ 
δή τινα νεωτερίζοιτο πρὸς τῶν ταύτῃ Ἰνδῶν καὶ 
ταῦτα ἐς κόσμον καταστήσαντα ξυμβάλλειν οἱ ἐς τὰ 
Πάταλα. 



Flavius Arrianus Hist., Phil., Alexandri anabasis 
Book 6, chapter 18, section 2, line 1

Περὶ δὲ τοῖς Πατάλοις σχίζεται τοῦ Ἰνδοῦ τὸ 
ὕδωρ ἐς <δύο> ποταμοὺς μεγάλους, καὶ οὗτοι ἀμφό-
τεροι σώζουσι τοῦ Ἰνδοῦ τὸ ὄνομα ἔστε ἐπὶ τὴν 
θάλασσαν. 



Flavius Arrianus Hist., Phil., Alexandri anabasis 
Book 6, chapter 18, section 4, line 2

                                      οὐκ ἔχοντι δὲ αὐτῷ 
ἡγεμόνα τοῦ πλοῦ, ὅτι πεφεύγεσαν οἱ ταύτῃ Ἰνδοί, 
ἀπορώτερα τὰ τοῦ κατάπλου ἦν· χειμών τε ἐπιγίγνεται 
ἐς τὴν ὑστεραίαν ἀπὸ τῆς ἀναγωγῆς καὶ ὁ ἄνεμος τῷ 
ῥόῳ πνέων ὑπεναντίος κοῖλόν τε ἐποίει τὸν ποταμὸν 
καὶ τὰ σκάφη διέσειεν, ὥστε ἐπόνησαν αὐτῷ αἱ 
πλεῖσται τῶν νεῶν, τῶν δὲ τριακοντόρων ἔστιν αἳ καὶ 
πάντῃ διελύθησαν. 



Flavius Arrianus Hist., Phil., Alexandri anabasis 
Book 6, chapter 18, section 5, line 4

               καὶ τῶν ψιλῶν τοὺς κουφοτάτους 
ἐκπέμψας ἐς τὴν προσωτέρω τῆς ὄχθης χώραν ξυλ-
λαμβάνει τινὰς τῶν Ἰνδῶν, καὶ οὗτοι τὸ ἀπὸ τοῦδε 
ἐξηγοῦντο αὐτῷ τὸν πόρον. 



Flavius Arrianus Hist., Phil., Alexandri anabasis 
Book 6, chapter 19, section 5, line 1

αὐτὸς δὲ ὑπερβαλὼν τοῦ Ἰνδοῦ ποταμοῦ τὰς ἐκβολὰς 
ἐς τὸ πέλαγος ἀνέπλει, ὡς μὲν ἔλεγεν, ἀπιδεῖν εἴ πού 
τις χώρα πλησίον ἀνίσχει ἐν τῷ πόντῳ, ἐμοὶ δὲ δοκεῖ, 
οὐχ ἥκιστα ὡς πεπλευκέναι τὴν μεγάλην τὴν ἔξω 
Ἰνδῶν θάλασσαν. 



Flavius Arrianus Hist., Phil., Alexandri anabasis 
Book 6, chapter 20, section 1, line 9

                              Ἡφαιστίων μὲν δὴ ἐτάχθη 
παρασκευάζειν τὰ πρὸς τὸν ἐκτειχισμόν τε τοῦ ναυ-
στάθμου καὶ τῶν νεωσοίκων τὴν κατασκευήν· καὶ γὰρ 
καὶ ἐνταῦθα ἐπενόει στόλον ὑπολείπεσθαι νεῶν οὐκ 
ὀλίγων πρὸς τῇ πόλει τοῖς Πατάλοις, ἵναπερ ἐσχίζετο 
ὁ ποταμὸς ὁ Ἰνδός. 



Flavius Arrianus Hist., Phil., Alexandri anabasis 
Book 6, chapter 20, section 2, line 1

Αὐτὸς δὲ κατὰ τὸ ἕτερον στόμα τοῦ Ἰνδοῦ κατ-
έπλει αὖθις ἐς τὴν μεγάλην θάλασσαν, ὡς καταμαθεῖν, 
ὅπῃ εὐπορωτέρα ἡ ἐκβολὴ τοῦ Ἰνδοῦ ἐς τὸν πόντον   
γίγνεται· ἀπέχει δὲ ἀλλήλων τὰ στόματα τοῦ ποταμοῦ 
τοῦ Ἰνδοῦ ἐς σταδίους μάλιστα ὀκτακοσίους καὶ 
χιλίους. 



Flavius Arrianus Hist., Phil., Alexandri anabasis 
Book 6, chapter 20, section 4, line 3

                           προσορμισθεὶς οὖν κατὰ τὴν 
λίμνην ἵναπερ οἱ καθηγεμόνες ἐξηγοῦντο, τῶν μὲν 
στρατιωτῶν τοὺς πολλοὺς καταλείπει σὺν Λεοννάτῳ 
αὐτοῦ καὶ τοὺς κερκούρους ξύμπαντας, αὐτὸς δὲ ταῖς 
τριακοντόροις τε καὶ ἡμιολίαις ὑπερβαλὼν τὴν ἐκβολὴν 
τοῦ Ἰνδοῦ καὶ προελθὼν καὶ ταύτῃ ἐς τὴν θάλασσαν 
εὐπορωτέραν τε κατέμαθεν τὴν ἐπὶ τάδε τοῦ Ἰνδοῦ 
ἐκβολὴν καὶ αὐτὸς προσορμισθεὶς τῷ αἰγιαλῷ καὶ τῶν 
ἱππέων τινὰς ἅμα οἷ ἔχων παρὰ θάλασσαν ᾔει στα-
θμοὺς τρεῖς, τήν τε χώραν ὁποία τίς ἐστιν ἡ ἐν 
τῷ παράπλῳ ἐπισκεπτόμενος καὶ φρέατα ὀρύσσεσθαι 
κελεύων, ὅπως ἔχοιεν ὑδρεύεσθαι οἱ πλέοντες. 



Flavius Arrianus Hist., Phil., Alexandri anabasis 
Book 6, chapter 21, section 3, line 11

                    ἐκεῖθεν δὲ ἀναλαβὼν τῶν ὑπασπι-
στῶν τε καὶ τῶν τοξοτῶν τοὺς ἡμίσεας καὶ τῶν πεζε-
ταίρων καλουμένων τὰς τάξεις καὶ τῆς ἵππου τῆς 
ἑταιρικῆς τό τε ἄγημα καὶ ἴλην ἀφ' ἑκάστης ἱππαρχίας 
καὶ τοὺς ἱπποτοξότας ξύμπαντας ὡς ἐπὶ τὴν θάλασσαν 
ἐς ἀριστερὰ ἐτράπετο, ὕδατά τε ὀρύσσειν, ὡς κατὰ τὸν 
παράπλουν ἄφθονα εἴη τῇ στρατιᾷ τῇ παραπλεούσῃ, 
καὶ ἅμα ὡς τοῖς Ὠρείταις τοῖς ταύτῃ Ἰνδοῖς αὐτονόμοις   
ἐκ πολλοῦ οὖσιν ἄφνω ἐπιπεσεῖν, ὅτι μηδὲν φίλιον 
αὐτοῖς ἐς αὐτόν τε καὶ τὴν στρατιὰν ἐπέπρακτο. 



Flavius Arrianus Hist., Phil., Alexandri anabasis 
Book 6, chapter 24, section 2, line 5

                                         οὐ μὴν ἀγνοήσαντα 
Ἀλέξανδρον τῆς ὁδοῦ τὴν χαλεπότητα ταύτῃ ἐλθεῖν, 
τοῦτο μὲν μόνος Νέαρχος λέγει ὧδε, ἀλλὰ ἀκούσαντα 
γὰρ ὅτι οὔπω τις πρόσθεν διελθὼν ταύτῃ ξὺν στρατιᾷ 
ἀπεσώθη, ὅτι μὴ Σεμίραμις ἐξ Ἰνδῶν ἔφυγε. 



Flavius Arrianus Hist., Phil., Alexandri anabasis 
Book 6, chapter 24, section 3, line 2

                             ἐλθεῖν γὰρ δὴ καὶ Κῦρον 
ἐς τοὺς χώρους τούτους ὡς ἐσβαλοῦντα ἐς τὴν Ἰνδῶν 
γῆν, φθάσαι δὲ ὑπὸ τῆς ἐρημίας τε καὶ ἀπορίας τῆς 
ὁδοῦ ταύτης ἀπολέσαντα τὴν πολλὴν τῆς στρατιᾶς. 



Flavius Arrianus Hist., Phil., Alexandri anabasis 
Book 6, chapter 25, section 4, line 4

            ὕεται γὰρ ἡ Γαδρωσίων γῆ ὑπ' ἀνέμων 
τῶν ἐτησίων, καθάπερ οὖν καὶ ἡ Ἰνδῶν γῆ, οὐ τὰ 
πεδία τῶν Γαδρωσίων, ἀλλὰ τὰ ὄρη, ἵναπερ προς-
φέρονταί τε αἱ νεφέλαι ἐκ τοῦ πνεύματος καὶ ἀνα-
χέονται, οὐχ ὑπερβάλλουσαι τῶν ὀρῶν τὰς κορυφάς. 



Flavius Arrianus Hist., Phil., Alexandri anabasis 
Book 6, chapter 27, section 2, line 3

                          ἤδη τε ἐπὶ Καρμανίας προὐ-
χώρει ὁ βασιλεὺς καὶ ἀγγέλλεται αὐτῷ Φίλιππον τὸν 
σατράπην τῆς Ἰνδῶν γῆς ἐπιβουλευθέντα πρὸς τῶν 
μισθοφόρων δόλῳ ἀποθανεῖν, τοὺς δὲ ἀποκτείναντας 
ὅτι οἱ σωματοφύλακες τοῦ Φιλίππου οἱ Μακεδόνες 
τοὺς μὲν ἐν αὐτῷ τῷ ἔργῳ, τοὺς δὲ καὶ ὕστερον 
λαβόντες ἀπέκτειναν. 



Flavius Arrianus Hist., Phil., Alexandri anabasis 
Book 6, chapter 27, section 2, line 8

                        ταῦτα δὲ ὡς ἔγνω, ἐκπέμπει 
γράμματα ἐς Ἰνδοὺς παρὰ Εὔδαμόν τε καὶ Ταξίλην 
ἐπιμελεῖσθαι τῆς χώρας τῆς πρόσθεν ὑπὸ Φιλίππῳ 
τεταγμένης ἔστ' ἂν αὐτὸς σατράπην ἐκπέμψῃ ἐπ' 
αὐτῆς. 



Flavius Arrianus Hist., Phil., Alexandri anabasis 
Book 6, chapter 28, section 2, line 2

Ἤδη δέ τινες καὶ τοιάδε ἀνέγραψαν, οὐ πιστὰ 
ἐμοὶ λέγοντες, ὡς συζεύξας δύο ἁρμαμάξας κατακεί-
μενος ξὺν τοῖς ἑταίροις καταυλούμενος τὴν διὰ Καρ-
μανίας ἦγεν, ἡ στρατιὰ δὲ αὐτῷ ἐστεφανωμένη τε καὶ 
παίζουσα εἵπετο, προὔκειτο δὲ αὐτῇ σῖτά τε καὶ ὅσα 
ἄλλα ἐς τρυφὴν παρὰ τὰς ὁδοὺς συγκεκομισμένα πρὸς 
τῶν Καρμανίων, καὶ ταῦτα πρὸς μίμησιν τῆς Διονύσου 
βακχείας ἀπεικάσθη Ἀλεξάνδρῳ, ὅτι καὶ ὑπὲρ ἐκείνου 
λόγος ἐλέγετο καταστρεψάμενον Ἰνδοὺς Διόνυσον οὕτω 
τὴν πολλὴν τῆς Ἀσίας ἐπελθεῖν, καὶ Θρίαμβόν τε 
αὐτὸν ἐπικληθῆναι τὸν Διόνυσον καὶ τὰς ἐπὶ ταῖς 
νίκαις ταῖς ἐκ πολέμου πομπὰς ἐπὶ τῷ αὐτῷ τούτῳ 
θριάμβους. 



Flavius Arrianus Hist., Phil., Alexandri anabasis 
Book 6, chapter 28, section 3, line 3

             ἀλλὰ ἐκεῖνα ἤδη Ἀριστοβούλῳ ἑπόμενος 
ξυγγράφω, θῦσαι ἐν Καρμανίᾳ Ἀλέξανδρον χαριστήρια 
τῆς κατ' Ἰνδῶν νίκης καὶ ὑπὲρ τῆς στρατιᾶς, ὅτι 
ἀπεσώθη ἐκ Γαδρωσίων, καὶ ἀγῶνα διαθεῖναι μουσικόν 
τε καὶ γυμνικόν· καταλέξαι δὲ καὶ Πευκέσταν ἐς τοὺς 
σωματοφύλακας, ἤδη μὲν ἐγνωκότα σατράπην κατα-
στῆσαι τῆς Περσίδος, ἐθέλοντα δὲ πρὸ τῆς σατρα-
πείας μηδὲ ταύτης τῆς τιμῆς καὶ πίστεως ἀπείρατον 
εἶναι ἐπὶ τῷ ἐν Μαλλοῖς ἔργῳ· εἶναι δὲ αὐτῷ ἑπτὰ 
εἰς τότε σωματοφύλακας, Λεοννάτον Ἀντέου, Ἡφαι-
στίωνα τὸν Ἀμύντορος, Λυσίμαχον Ἀγαθοκλέους, Ἀρι-
στόνουν Πεισαίου, τούτους μὲν Πελλαίους, Περδίκκαν 




Flavius Arrianus Hist., Phil., Alexandri anabasis 
Book 6, chapter 28, section 6, line 4

           τοῦτον μὲν δὴ καταπέμπει αὖθις ἐκπερι-
πλεύσοντα ἔστε ἐπὶ τὴν Σουσιανῶν τε γῆν καὶ τοῦ 
Τίγρητος ποταμοῦ τὰς ἐκβολάς· ὅπως δὲ ἐπλεύσθη 
αὐτῷ τὰ ἀπὸ τοῦ Ἰνδοῦ ποταμοῦ ἐπὶ τὴν θάλασσαν 
τὴν Περσικὴν καὶ τὸ στόμα τοῦ Τίγρητος, ταῦτα ἰδίᾳ 
ἀναγράψω αὐτῷ Νεάρχῳ ἑπόμενος, ὡς καὶ τήνδε εἶναι   
ὑπὲρ Ἀλεξάνδρου Ἑλληνικὴν τὴν συγγραφήν. 



Flavius Arrianus Hist., Phil., Alexandri anabasis 
Book 6, chapter 29, section 2, line 3

ὡς δὲ ἐπὶ τοῖς ὅροις ἦν τῆς Περσίδος, Φρασαόρτην 
μὲν οὐ κατέλαβε σατραπεύοντα ἔτι (νόσῳ γὰρ τε-
τελευτηκὼς ἐτύγχανεν ἐν Ἰνδοῖς ἔτι Ἀλεξάνδρου ὄντος), 
Ὀρξίνης δὲ ἐπεμέλετο τῆς Περσίδος, οὐ πρὸς Ἀλεξ-
άνδρου κατασταθείς, ἀλλ' ὅτι οὐκ ἀπηξίωσεν αὑτὸν 
ἐν κόσμῳ Πέρσας διαφυλάξαι Ἀλεξάνδρῳ οὐκ ὄντος 
ἄλλου ἄρχοντος. 



Flavius Arrianus Hist., Phil., Alexandri anabasis 
Book 7, chapter 1, section 2, line 1

ΑΡΡΙΑΝΟΥ 
ΑΛΕΞΑΝΔΡΟΥ ΑΝΑΒΑΣΕΩΣ 
ΒΙΒΛΙΟΝ ΕΒΔΟΜΟΝ


 Ὡς δὲ ἐς Πασαργάδας τε καὶ ἐς Περσέπολιν 
ἀφίκετο Ἀλέξανδρος, πόθος λαμβάνει αὐτὸν κατα-
πλεῦσαι κατὰ τὸν Εὐφράτην τε καὶ κατὰ τὸν Τίγρητα 
ἐπὶ τὴν θάλασσαν τὴν Περσικὴν καὶ τῶν τε ποταμῶν 
ἰδεῖν τὰς ἐκβολὰς τὰς ἐς τὸν πόντον, καθάπερ τοῦ 
Ἰνδοῦ, καὶ τὴν ταύτῃ θάλασσαν. 



Flavius Arrianus Hist., Phil., Alexandri anabasis 
Book 7, chapter 1, section 5, line 2

                               καὶ ἐπὶ τῷδε ἐπαινῶ τοὺς 
σοφιστὰς τῶν Ἰνδῶν, ὧν λέγουσιν ἔστιν οὓς κατα-
ληφθέντας ὑπ' Ἀλεξάνδρου ὑπαιθρίους ἐν λειμῶνι, 
ἵναπερ αὐτοῖς διατριβαὶ ἦσαν, ἄλλο μὲν οὐδὲν ποιῆσαι 
πρὸς τὴν ὄψιν αὐτοῦ τε καὶ τῆς στρατιᾶς, κρούειν δὲ 
τοῖς ποσὶ τὴν γῆν ἐφ' ἧς βεβηκότες ἦσαν. 



Flavius Arrianus Hist., Phil., Alexandri anabasis 
Book 7, chapter 2, section 2, line 4

                                                          ἐπεὶ 
καὶ ἐς Τάξιλα αὐτῷ ἀφικομένῳ καὶ ἰδόντι τῶν σοφι-
στῶν <τῶν]2 Ἰνδῶν τοὺς γυμνοὺς πόθος ἐγένετο ξυν-
εῖναί τινα οἱ τῶν ἀνδρῶν τούτων, ὅτι τὴν καρτερίαν 
αὐτῶν ἐθαύμασε· καὶ ὁ μὲν πρεσβύτατος τῶν σοφιστῶν, 
ὅτου ὁμιληταὶ οἱ ἄλλοι ἦσαν, Δάνδαμις ὄνομα, οὔτε 
αὐτὸς ἔφη παρ' Ἀλέξανδρον ἥξειν οὔτε τοὺς ἄλλους 
εἴα, ἀλλὰ ὑποκρίνασθαι γὰρ λέγεται ὡς Διὸς υἱὸς καὶ 
αὐτὸς εἴη, εἴπερ οὖν καὶ Ἀλέξανδρος, καὶ ὅτι οὔτε 
δέοιτό του τῶν παρ' Ἀλεξάνδρου, ἔχει<ν> γάρ οἱ εὖ 
τὰ παρόντα, καὶ ἅμα ὁρᾶν τοὺς ξὺν αὐτῷ πλανωμένους 
τοσαύτην γῆν καὶ θάλασσαν ἐπ' ἀγαθῷ οὐδενί, μηδὲ 




Flavius Arrianus Hist., Phil., Alexandri anabasis 
Book 7, chapter 2, section 4, line 1

                                                    οὔτ' 
οὖν ποθεῖν τι αὐτὸς ὅτου κύριος ἦν Ἀλέξανδρος 
δοῦναι, οὔτε αὖ δεδιέναι, ὅτου κρατοίη ἐκεῖνος, ἔστιν   
οὗ εἴργεσθαι· ζῶντι μὲν γάρ οἱ τὴν Ἰνδῶν γῆν ἐξ-
αρκεῖν φέρουσαν τὰ ὡραῖα, ἀποθανόντα δὲ ἀπαλ-
λαγήσεσθαι οὐκ ἐπιεικοῦς ξυνοίκου τοῦ σώματος. 



Flavius Arrianus Hist., Phil., Alexandri anabasis 
Book 7, chapter 3, section 3, line 5

                                                   αὐτῷ δὲ 
παρασκευασθῆναι μὲν ἵππον, ὅτι βαδίσαι ἀδυνάτως   
εἶχεν ὑπὸ τῆς νόσου· οὐ μὴν δυνηθῆναί γε οὐδὲ τοῦ 
ἵππου ἐπιβῆναι, ἀλλὰ ἐπὶ κλίνης γὰρ κομισθῆναι 
φερόμενον, ἐστεφανωμένον τε τῷ Ἰνδῶν νόμῳ καὶ 
ᾄδοντα τῇ Ἰνδῶν γλώσσῃ. 



Flavius Arrianus Hist., Phil., Alexandri anabasis 
Book 7, chapter 3, section 3, line 6

                                  οἱ δὲ Ἰνδοὶ λέγουσιν ὅτι 
ὕμνοι θεῶν ἦσαν καὶ αὐτῶν ἔπαινοι. 



Flavius Arrianus Hist., Phil., Alexandri anabasis 
Book 7, chapter 3, section 6, line 8

                                      ταῦτα καὶ τοιαῦτα 
ὑπὲρ Καλάνου τοῦ Ἰνδοῦ ἱκανοὶ ἀναγεγράφασιν, οὐκ 
ἀχρεῖα πάντα ἐς ἀνθρώπους, ὅτῳ γνῶναι ἐπιμελές, 
[ὅτι] ὡς καρτερόν τέ ἐστι καὶ ἀνίκητον γνώμη ἀνθρω-
πίνη ὅ τι περ ἐθέλοι ἐξεργάσασθαι. 



Flavius Arrianus Hist., Phil., Alexandri anabasis 
Book 7, chapter 4, section 2, line 4

πολλὰ μὲν δὴ ἐπεπλημμέλητο ἐκ τῶν κατεχόντων τὰς 
χώρας ὅσαι δορίκτητοι πρὸς Ἀλεξάνδρου ἐγένοντο ἔς 
τε τὰ ἱερὰ καὶ τάφους καὶ αὐτοὺς τοὺς ὑπηκόους, 
ὅτι χρόνιος ὁ εἰς Ἰνδοὺς στόλος ἐγεγένητο τῷ βασιλεῖ 
καὶ οὐ πιστὸν ἐφαίνετο ἀπονοστήσειν αὐτὸν ἐκ 
τοσῶνδε ἐθνῶν καὶ τοσῶνδε ἐλεφάντων, ὑπὲρ τὸν 
Ἰνδόν τε καὶ Ὑδάσπην καὶ τὸν Ἀκεσίνην καὶ Ὕφασιν 
φθειρόμενον. 



Flavius Arrianus Hist., Phil., Alexandri anabasis 
Book 7, chapter 5, section 5, line 2

                             καὶ ἐστεφάνωσε χρυσοῖς 
στεφάνοις τοὺς ἀνδραγαθίᾳ διαπρέποντας, πρῶτον μὲν 
Πευκέσταν τὸν ὑπερασπίσαντα, ἔπειτα Λεοννάτον, καὶ 
τοῦτον ὑπερασπίσαντα, καὶ διὰ τοὺς ἐν Ἰνδοῖς κινδύ-
νους καὶ τὴν ἐν Ὤροις νίκην γενομένην, ὅτι παραταξά-
μενος σὺν τῇ ὑπολειφθείσῃ δυνάμει πρὸς τοὺς νεωτερί-
ζοντας τῶν τε Ὠρειτῶν καὶ τῶν πλησίον τούτων   
ᾠκισμένων τῇ τε μάχῃ ἐκράτησε καὶ τὰ ἄλλα καλῶς 
ἔδοξε τὰ ἐν Ὤροις κοσμῆσαι. 



Flavius Arrianus Hist., Phil., Alexandri anabasis 
Book 7, chapter 5, section 6, line 2

                                    ἐπὶ τούτοις δὲ Νέαρχον 
ἐπὶ τῷ περίπλῳ τῷ ἐκ τῆς Ἰνδῶν γῆς κατὰ τὴν μεγά-
λην θάλασσαν ἐστεφάνωσε· καὶ γὰρ καὶ οὗτος ἤδη 
ἀφιγμένος ἐς Σοῦσα ἦν· ἐπὶ τούτοις δὲ Ὀνησίκριτον 
τὸν κυβερνήτην τῆς νεὼς τῆς βασιλικῆς· ἔτι δὲ 
Ἡφαιστίωνα καὶ τοὺς ἄλλους τοὺς σωματοφύλακας. 



Flavius Arrianus Hist., Phil., Alexandri anabasis 
Book 7, chapter 9, section 8, line 6

σατράπας τοὺς Δαρείου τήν τε Ἰωνίαν πᾶσαν τῇ 
ὑμετέρᾳ ἀρχῇ προσέθηκα καὶ τὴν Αἰολίδα πᾶσαν καὶ 
Φρύγας ἀμφοτέρους καὶ Λυδούς, καὶ Μίλητον εἷλον 
πολιορκίᾳ· τὰ δὲ ἄλλα πάντα ἑκόντα προσχωρήσαντα 
λαβὼν ὑμῖν καρποῦσθαι ἔδωκα· καὶ τὰ ἐξ Αἰγύπτου 
καὶ Κυρήνης ἀγαθά, ὅσα ἀμαχεὶ ἐκτησάμην, ὑμῖν 
ἔρχεται, ἥ τε κοίλη Συρία καὶ ἡ Παλαιστίνη καὶ ἡ 
μέση τῶν ποταμῶν ὑμέτερον κτῆμά εἰσι, καὶ Βαβυλὼν 
καὶ Βάκτρα καὶ Σοῦσα ὑμέτερα, καὶ ὁ Λυδῶν πλοῦτος   
καὶ οἱ Περσῶν θησαυροὶ καὶ τὰ Ἰνδῶν ἀγαθὰ καὶ ἡ 
ἔξω θάλασσα ὑμέτερα· ὑμεῖς σατράπαι, ὑμεῖς στρατηγοί, 
ὑμεῖς ταξιάρχαι. 



Flavius Arrianus Hist., Phil., Alexandri anabasis 
Book 7, chapter 10, section 6, line 6

βούλεσθε, ἄπιτε πάντες, καὶ ἀπελθόντες οἴκοι ἀπαγ-
γείλατε ὅτι τὸν βασιλέα ὑμῶν Ἀλέξανδρον, νικῶντα 
μὲν Πέρσας καὶ Μήδους καὶ Βακτρίους καὶ Σάκας, 
καταστρεψάμενον δὲ Οὐξίους τε καὶ Ἀραχωτοὺς καὶ 
Δράγγας, κεκτημένον δὲ καὶ Παρθυαίους καὶ Χορας-
μίους καὶ Ὑρκανίους ἔστε ἐπὶ τὴν θάλασσαν τὴν 
Κασπίαν, ὑπερβάντα δὲ τὸν Καύκασον ὑπὲρ τὰς 
Κασπίας πύλας, καὶ περάσαντα Ὄξον τε ποταμὸν καὶ 
Τάναϊν, ἔτι δὲ τὸν Ἰνδὸν ποταμόν, οὐδενὶ ἄλλῳ ὅτι 
μὴ Διονύσῳ περαθέντα, καὶ τὸν Ὑδάσπην καὶ τὸν 
Ἀκεσίνην καὶ τὸν Ὑδραώτην, καὶ τὸν Ὕφασιν δια-
περάσαντα ἄν, εἰ μὴ ὑμεῖς ἀπωκνήσατε, καὶ ἐς τὴν   
μεγάλην θάλασσαν κατ' ἀμφότερα τοῦ Ἰνδοῦ τὰ 
στόματα ἐμβαλόντα, καὶ διὰ τῆς Γαδρωσίας τῆς ἐρήμου 
ἐλθόντα, ᾗ οὐδείς πω πρόσθεν σὺν στρατιᾷ ἦλθε, καὶ 
Καρμανίαν ἐν παρόδῳ προσκτησάμενον καὶ τὴν Ὠρει-
τῶν γῆν, περιπεπλευκότος δὲ ἤδη αὐτῷ τοῦ ναυτικοῦ 
τὴν ἀπ' Ἰνδῶν γῆς εἰς Πέρσας θάλασσαν, ὡς εἰς 




Flavius Arrianus Hist., Phil., Alexandri anabasis 
Book 7, chapter 10, section 7, line 3


Δράγγας, κεκτημένον δὲ καὶ Παρθυαίους καὶ Χορας-
μίους καὶ Ὑρκανίους ἔστε ἐπὶ τὴν θάλασσαν τὴν 
Κασπίαν, ὑπερβάντα δὲ τὸν Καύκασον ὑπὲρ τὰς 
Κασπίας πύλας, καὶ περάσαντα Ὄξον τε ποταμὸν καὶ 
Τάναϊν, ἔτι δὲ τὸν Ἰνδὸν ποταμόν, οὐδενὶ ἄλλῳ ὅτι 
μὴ Διονύσῳ περαθέντα, καὶ τὸν Ὑδάσπην καὶ τὸν 
Ἀκεσίνην καὶ τὸν Ὑδραώτην, καὶ τὸν Ὕφασιν δια-
περάσαντα ἄν, εἰ μὴ ὑμεῖς ἀπωκνήσατε, καὶ ἐς τὴν   
μεγάλην θάλασσαν κατ' ἀμφότερα τοῦ Ἰνδοῦ τὰ 
στόματα ἐμβαλόντα, καὶ διὰ τῆς Γαδρωσίας τῆς ἐρήμου 
ἐλθόντα, ᾗ οὐδείς πω πρόσθεν σὺν στρατιᾷ ἦλθε, καὶ 
Καρμανίαν ἐν παρόδῳ προσκτησάμενον καὶ τὴν Ὠρει-
τῶν γῆν, περιπεπλευκότος δὲ ἤδη αὐτῷ τοῦ ναυτικοῦ 
τὴν ἀπ' Ἰνδῶν γῆς εἰς Πέρσας θάλασσαν, ὡς εἰς 
Σοῦσα ἐπανηγάγετε, ἀπολιπόντες οἴχεσθε, παραδόντες 
φυλάσσειν τοῖς νενικημένοις βαρβάροις. 



Flavius Arrianus Hist., Phil., Alexandri anabasis 
Book 7, chapter 10, section 7, line 8

Τάναϊν, ἔτι δὲ τὸν Ἰνδὸν ποταμόν, οὐδενὶ ἄλλῳ ὅτι 
μὴ Διονύσῳ περαθέντα, καὶ τὸν Ὑδάσπην καὶ τὸν 
Ἀκεσίνην καὶ τὸν Ὑδραώτην, καὶ τὸν Ὕφασιν δια-
περάσαντα ἄν, εἰ μὴ ὑμεῖς ἀπωκνήσατε, καὶ ἐς τὴν   
μεγάλην θάλασσαν κατ' ἀμφότερα τοῦ Ἰνδοῦ τὰ 
στόματα ἐμβαλόντα, καὶ διὰ τῆς Γαδρωσίας τῆς ἐρήμου 
ἐλθόντα, ᾗ οὐδείς πω πρόσθεν σὺν στρατιᾷ ἦλθε, καὶ 
Καρμανίαν ἐν παρόδῳ προσκτησάμενον καὶ τὴν Ὠρει-
τῶν γῆν, περιπεπλευκότος δὲ ἤδη αὐτῷ τοῦ ναυτικοῦ 
τὴν ἀπ' Ἰνδῶν γῆς εἰς Πέρσας θάλασσαν, ὡς εἰς 
Σοῦσα ἐπανηγάγετε, ἀπολιπόντες οἴχεσθε, παραδόντες 
φυλάσσειν τοῖς νενικημένοις βαρβάροις. 



Flavius Arrianus Hist., Phil., Alexandri anabasis 
Book 7, chapter 16, section 2, line 5

                           πόθος γὰρ εἶχεν αὐτὸν καὶ 
ταύτην ἐκμαθεῖν τὴν θάλασσαν τὴν Κασπίαν τε καὶ 
Ὑρκανίαν καλουμένην ποίᾳ τινὶ ξυμβάλλει θαλάσσῃ, 
πότερα τῇ τοῦ πόντου τοῦ Εὐξείνου ἢ ἀπὸ τῆς ἑῴας 
τῆς κατ' Ἰνδοὺς ἐκπεριερχομένη ἡ μεγάλη θάλασσα 
ἀναχεῖται εἰς κόλπον τὸν Ὑρκάνιον, καθάπερ οὖν καὶ 
τὸν Περσικὸν ἐξεῦρε, τὴν Ἐρυθρὰν δὴ καλουμένην 
θάλασσαν, κόλπον οὖσαν τῆς μεγάλης θαλάσσης. 



Flavius Arrianus Hist., Phil., Alexandri anabasis 
Book 7, chapter 16, section 3, line 6

                                                    οὐ 
γάρ πω ἐξεύρηντο αἱ ἀρχαὶ τῆς Κασπίας θαλάσσης, 
καίτοι ἐθνῶν τε αὐτὴν <περι>οικούντων οὐκ ὀλίγων 
καὶ ποταμῶν πλοΐμων ἐμβαλλόντων ἐς αὐτήν· ἐκ 
Βάκτρων μὲν Ὄξος, μέγιστος τῶν Ἀσιανῶν ποταμῶν, 
πλήν γε δὴ τῶν Ἰνδῶν, ἐξίησιν ἐς ταύτην τὴν θάλας-
σαν, διὰ Σκυθῶν δὲ Ἰαξάρτης· καὶ τὸν Ἀράξην δὲ   
τὸν ἐξ Ἀρμενίων ῥέοντα ἐς ταύτην ἐσβάλλειν ὁ πλείων 
λόγος κατέχει. 



Flavius Arrianus Hist., Phil., Alexandri anabasis 
Book 7, chapter 18, section 1, line 5

Ἐπεὶ καὶ τοῖόνδε τινὰ λόγον Ἀριστόβουλος ἀνα-
γέγραφεν, Ἀπολλόδωρον τὸν Ἀμφιπολίτην τῶν ἑταίρων 
τῶν Ἀλεξάνδρου, στρατηγὸν τῆς στρατιᾶς ἣν παρὰ 
Μαζαίῳ τῷ Βαβυλῶνος σατράπῃ ἀπέλιπεν Ἀλέξανδρος, 
ἐπειδὴ συνέμιξεν ἐπανιόντι αὐτῷ ἐξ Ἰνδῶν, ὁρῶντα 
πικρῶς τιμωρούμενον τοὺς σατράπας ὅσοι ἐπ' ἄλλῃ καὶ 
ἄλλῃ χώρᾳ τεταγμένοι ἦσαν, ἐπιστεῖλαι Πειθαγόρᾳ τῷ 
ἀδελφῷ, μάντιν γὰρ εἶναι τὸν Πειθαγόραν τῆς ἀπὸ 
σπλάγχνων μαντείας, μαντεύσασθαι καὶ ὑπὲρ αὐτοῦ 
τῆς σωτηρίας. 



Flavius Arrianus Hist., Phil., Alexandri anabasis 
Book 7, chapter 18, section 6, line 1

καὶ μὲν δὴ καὶ ὑπὲρ Καλάνου τοῦ σοφιστοῦ τοῦ Ἰνδοῦ 
τοῖόσδε τις ἀναγέγραπται λόγος, ὁπότε ἐπὶ τὴν πυρὰν 
ᾔει ἀποθανούμενος, τότε τοὺς μὲν ἄλλους ἑταίρους 
ἀσπάζεσθαι αὐτόν, Ἀλεξάνδρῳ δὲ οὐκ ἐθελῆσαι προς-
ελθεῖν ἀσπασόμενον, ἀλλὰ φάναι γὰρ ὅτι ἐν Βαβυλῶνι 
αὐτῷ ἐντυχὼν ἀσπάσεται. 



Flavius Arrianus Hist., Phil., Alexandri anabasis 
Book 7, chapter 19, section 1, line 5

Παρελθόντι δ' αὐτῷ ἐς Βαβυλῶνα πρεσβεῖαι παρὰ 
τῶν Ἑλλήνων ἐνέτυχον, ὑπὲρ ὅτων μὲν ἕκαστοι πρες-
βευόμενοι οὐκ ἀναγέγραπται, δοκεῖν δ' ἔμοιγε αἱ 
πολλαὶ στεφανούντων τε αὐτὸν ἦσαν καὶ ἐπαινούντων 
ἐπὶ ταῖς νίκαις ταῖς τε ἄλλαις καὶ μάλιστα ταῖς Ἰνδι-
καῖς, καὶ ὅτι σῶος ἐξ Ἰνδῶν ἐπανήκει χαίρειν φα-
σκόντων. 



Flavius Arrianus Hist., Phil., Alexandri anabasis 
Book 7, chapter 20, section 1, line 6

Λόγος δὲ κατέχει ὅτι ἤκουεν Ἄραβας δύο μόνον 
τιμᾶν θεούς, τὸν Οὐρανόν τε καὶ τὸν Διόνυσον, τὸν 
μὲν Οὐρανὸν αὐτόν τε ὁρώμενον καὶ τὰ ἄστρα ἐν οἷ 
ἔχοντα τά τε ἄλλα καὶ τὸν ἥλιον, ἀφ' ὅτου μεγίστη 
καὶ φανοτάτη ὠφέλεια ἐς πάντα ἥκει τὰ ἀνθρώπεια, 
Διόνυσον δὲ κατὰ δόξαν τῆς ἐς Ἰνδοὺς στρατιᾶς. 



Flavius Arrianus Hist., Phil., Alexandri anabasis 
Book 7, chapter 20, section 1, line 10

οὔκουν ἀπαξιοῦν καὶ αὐτὸν τρίτον ἂν νομισθῆναι 
πρὸς Ἀράβων θεόν, οὐ φαυλότερα ἔργα Διονύσου 
ἀποδειξάμενον, εἴπερ οὖν καὶ Ἀράβων κρατήσας ἐπι-
τρέψειεν αὐτοῖς, καθάπερ Ἰνδοῖς, πολιτεύειν κατὰ τὰ 
σφῶν νόμιμα. 



Flavius Arrianus Hist., Phil., Alexandri anabasis 
Book 7, chapter 20, section 2, line 7

               τῆς τε χώρας ἡ εὐδαιμονία ὑπεκίνει 
αὐτόν, ὅτι ἤκουεν ἐκ μὲν τῶν λιμνῶν τὴν κασίαν 
γίγνεσθαι αὐτοῖς, ἀπὸ δὲ τῶν δένδρων τὴν σμύρναν 
τε καὶ τὸν λιβανωτόν, ἐκ δὲ τῶν θάμνων τὸ κιννάμω-
μον τέμνεσθαι, οἱ λειμῶνες δὲ ὅτι νάρδον αὐτόματοι   
ἐκφέρουσι· τό <τε> μέγεθος τῆς χώρας, ὅτι οὐκ ἐλάτ-
των ἡ παράλιος τῆς Ἀραβίας ἤπερ ἡ τῆς Ἰνδικῆς αὐτῷ 
ἐξηγγέλλετο, καὶ νῆσοι αὐτῇ προσκεῖσθαι πολλαί, καὶ 
λιμένες πανταχοῦ τῆς χώρας ἐνεῖναι, οἷοι παρασχεῖν 
μὲν ὅρμους τῷ ναυτικῷ, παρασχεῖν δὲ καὶ πόλεις 
ἐνοικισθῆναι καὶ ταύτας γενέσθαι εὐδαίμονας. 



Flavius Arrianus Hist., Phil., Alexandri anabasis 
Book 7, chapter 20, section 8, line 8

                                          ἦν μὲν γὰρ 
αὐτῷ προστεταγμένον περιπλεῦσαι τὴν χερρόνησον τὴν 
Ἀράβων πᾶσαν ἔστε ἐπὶ τὸν κόλπον τὸν πρὸς Αἰγύπτῳ 
τὸν Ἀράβιον τὸν καθ' Ἡρώων πόλιν· οὐ μὴν ἐτόλμησέ 
γε τὸ πρόσω ἐλθεῖν, καίτοι ἐπὶ τὸ πολὺ παραπλεύσας 
τὴν Ἀράβων γῆν· ἀλλ' ἀναστρέψας γὰρ παρ' Ἀλέξαν-
δρον ἐξήγγειλεν τὸ μέγεθός τε τῆς χερρονήσου θαυ-
μαστόν τι εἶναι καὶ ὅσον οὐ πολὺ ἀποδέον τῆς Ἰνδῶν 
γῆς, ἄκραν τε ἀνέχειν ἐπὶ πολὺ τῆς μεγάλης θαλάσσης· 
ἣν δὴ καὶ τοὺς σὺν Νεάρχῳ ἀπὸ τῆς Ἰνδικῆς πλέοντας, 
πρὶν ἐπικάμψαι ἐς τὸν κόλπον τὸν Περσικόν, οὐ πόρρω 
ἀνατείνουσαν ἰδεῖν τε καὶ παρ' ὀλίγον ἐλθεῖν διαβαλεῖν   
ἐς αὐτήν, καὶ Ὀνησικρίτῳ τῷ κυβερνήτῃ ταύτῃ δοκοῦν· 
ἀλλὰ Νέαρχος λέγει ὅτι αὐτὸς διεκώλυσεν, ὡς ἐκπερι-
πλεύσας τὸν κόλπον τὸν Περσικὸν ἔχοι ἀπαγγεῖλαι 
Ἀλεξάνδρῳ ἐφ' οἷστισι πρὸς αὐτοῦ ἐστάλη· οὐ γὰρ 
ἐπὶ τῷ πλεῦσαι τὴν μεγάλην θάλασσαν ἐστάλθαι, ἀλλ' 

\end{greek}

\section{Nicolaus of Damascus}
\subsection{About}
\blockquote[from Wikipedia]{Nicolaus of Damascus (Greek: Νικόλαος Δαμασκηνός, Nikolāos Damaskēnos) was a Greek[1] historian and philosopher who lived during the Augustan age of the Roman Empire. His name is derived from that of his birthplace, Damascus. He was born around 64 BC.[2]

He was an intimate friend of Herod the Great, whom he survived by a number of years. He was also the tutor of the children of Antony and Cleopatra (born in 40 BC), according to Sophronius.[3] He went to Rome with Herod Archelaus.[4]

His output was vast, but is nearly all lost. His chief work was a universal history in 144 books. He also wrote an autobiography, a life of Augustus, a life of Herod, some philosophical works, and some tragedies and comedies.\footnote{From Wikipedia.}}

\blockquote[From Wikipedia]{One of the most famous passages is his account of an embassy sent by an Indian king "named Pandion (Pandyan kingdom?) or, according to others, Porus" to Augustus around AD 13. He met with the embassy at Antioch. The embassy was bearing a diplomatic letter in Greek, and one of its members was a sramana who burnt himself alive in Athens to demonstrate his faith. The event made a sensation and was quoted by Strabo[13] and Dio Cassius.[14: 54.9] A tomb was made to the sramana, still visible in the time of Plutarch, which bore the mention "ΖΑΡΜΑΝΟΧΗΓΑΣ ΙΝΔΟΣ ΑΠΟ ΒΑΡΓΟΣΗΣ" ("The sramana master from Barygaza in India"): [Quotation of Strabo \emph{Geographica} 15.1.72--73.] \ldots{} This accounts suggests that it may not have been impossible to encounter an Indian religious man in the Levant during the time of Jesus.}

\subsection{Indian embassy to Augustus}
\label{dio_cassius_embassy}
Story preserved in Strabo.
\blockquote[Strabo \emph{Geographica} 15.1.72--73]{\textgreek{(72) προσθείη δ' ἄν τις τούτοις καὶ τὰ παρὰ τοῦ Δαμασκηνοῦ Νικολάου. \\ \indent (73) Φησὶ γὰρ οὗτος ἐν Ἀντιοχείᾳ τῇ ἐπὶ Δάφνῃ παρατυχεῖν τοῖς Ἰνδῶν πρέσβεσιν ἀφιγμένοις παρὰ Καίσαρα τὸν Σεβαστόν· οὓς ἐκ μὲν τῆς ἐπιστολῆς πλείους δηλοῦσθαι, σωθῆναι δὲ τρεῖς μόνους, οὓς ἰδεῖν φησι, τοὺς δ' ἄλλους ὑπὸ μήκους τῶν ὁδῶν διαφθαρῆναι τὸ πλέον· τὴν δ' ἐπιστολὴν ἑλληνίζειν ἐν διφθέρᾳ γεγραμμένην, δηλοῦσαν ὅτι Πῶρος εἴη ὁ γράψας, ἑξακοσίων δὲ ἄρχων βασιλέων ὅμως περὶ πολλοῦ ποιοῖτο φίλος εἶναι Καίσαρι, καὶ ἕτοιμος εἴη δίοδόν τε παρέχειν ὅπῃ βούλεται καὶ συμπράττειν ὅσα καλῶς ἔχει. ταῦτα μὲν ἔφη λέγειν τὴν ἐπιστολήν, τὰ δὲ κομισθέντα δῶρα προσενεγκεῖν ὀκτὼ οἰκέτας γυμνούς, ἐν περιζώμασι καταπεπασμένους ἀρώμασιν· εἶναι δὲ τὰ δῶρα τόν τε ἑρμᾶν, ἀπὸ τῶν ὤμων ἀφῃρημένον ἐκ νηπίου τοὺς βραχίονας, ὃν καὶ ἡμεῖς εἴδομεν, καὶ ἐχίδνας μεγάλας καὶ ὄφιν πηχῶν δέκα καὶ χελώνην ποταμίαν τρίπηχυν πέρδικά τε μείζω γυπός. συνῆν δέ, ὥς φησι, καὶ ὁ Ἀθήνησι κατακαύσας ἑαυτόν· ποιεῖν δὲ τοῦτο τοὺς μὲν ἐπὶ κακοπραγίᾳ ζητοῦντας ἀπαλλαγὴν τῶν παρόντων, τοὺς δ' ἐπ' εὐπραγίᾳ, καθάπερ τοῦτον· ἅπαντα γὰρ κατὰ γνώμην πράξαντα μέχρι νῦν ἀπιέναι δεῖν, μή τι τῶν ἀβουλήτων χρονίζοντι συμπέσοι· καὶ δὴ καὶ γελῶντα ἁλέσθαι γυμνὸν λίπ' ἀληλιμμένον ἐν περιζώματι ἐπὶ τὴν πυράν· ἐπιγεγράφθαι δὲ τῷ τάφῳ “Ζαρμανοχηγὰς Ἰνδὸς ἀπὸ Βαργόσης κατὰ ``τὰ πάτρια Ἰνδῶν ἔθη ἑαυτὸν ἀπαθανατίσας κεῖται.''}}\footnote{Strabo text from \textcite{meineke1877}.}

See also the Dio Cassius version of this (54.9).

\section{Augustus}

\subsection{Res gestae}
As published in the Loeb Classical Library, 1924.\footnote{\url{http://penelope.uchicago.edu/Thayer/E/Roman/Texts/Augustus/Res_Gestae/1*.html}.}
\subsubsection{Latin}
\begin{latin}
(31) Ad me ex India regum legationes saepe missae sunt, nunquam antea visae 51 apud quemquam Romanorum ducem. § Nostram amicitiam petierunt 52 per legatos Bastarnae Scythaeque et Sarmatarum qui sunt citra flumen 53 Tanaim et ultrá reges, Albanorumque réx et Hibérorum et Medorum. 
\end{latin}

\subsubsection{Greek}
As published in the Loeb Classical Library, 1924.\footnote{\url{http://penelope.uchicago.edu/Thayer/E/Roman/Texts/Augustus/Res_Gestae/1*.html}.}
\begin{greek}
(31)  Πρὸς ἐμὲ ἐξ Ἰνδίας βασιλέων πρεσβεῖαι πολλάκις ἀπε στάλησαν, οὐδέποτε πρὸ τούτου χρόνου ὀφθεῖσαι παρὰ 18 Ῥωμαίων ἡγεμόνι. § Τὴν ἡμετέραν φιλίαν ἠξίωσαν διὰ πρέσβεων § Βαστάρναι καὶ Σκύθαι καὶ Σαρμα τῶν οἱ ἐπιτάδε ὄντες τοῦ Τανάιδος ποταμοῦ καὶ οἱ πέραν δὲ βασιλεῖς, καὶ Ἀλβανῶν δὲ καὶ Ἰβήρων καὶ Μήδων βασιλεῖς.
\end{greek}

\subsubsection{English}
(31) Embassies were often sent to me from the kings of India, a thing never seen before in the camp of any general of the Romans. Our friendship was sought, through ambassadors, by the Bastarnae and Scythians, and by the kings of the Sarmatians who live on either side of the river Tanais, and by the king of the Albani and of the Hiberi and of the Medes. 

\section{Aretaeus of Cappadocia}

\blockquote[From Wikipedia.\footnote{\url{http://en.wikipedia.org/wiki/Aretaeus_of_Cappadocia}.}]{Aretaeus (Ἀρεταῖος), is one of the most celebrated of the ancient Greek physicians, of whose life, however, few particulars are known. There is some uncertainty regarding both his age and country, but it seems probable that he practised in the 1st century CE, during the reign of Nero or Vespasian. He is generally styled "the Cappadocian" (Καππάδοξ).}

\begin{greek}
Aretaeus Med., De causis et signis acutorum morborum (lib. 2) (0719: 002)
“Aretaeus, 2nd edn.”, Ed. Hude, K.
Berlin: Akademie–Verlag, 1958; Corpus medicorum Graecorum, vol. 2.

Aretaeus Med., De curatione acutorum morborum libri duo 
Book 2, chapter 10, section 4, line 4

                          τοιγαρῶν καὶ τοῖσι προσθέτοισι εὐώδεσι ἐς τὴν 
χώρην ἐπιβλητέον τῆς ὑστέρης, μύρον ὁκοῖον ἂν ἔῃ προσηνές, ἠδὲ 
ἄδηκτον τὴν ἁφήν, νάρδον ἢ βάκχαρι τὸ Αἰγύπτιον ἢ τὸ διὰ τῶν 
φύλλων τοῦ μαλαβάθρου, τοῦ δένδρεος τοῦ Ἰνδικοῦ, ἢ κινάμωμον 
κοπὲν ξὺν τῶν εὐόσμων τινὶ λίπαϊ· ἐγχρίειν δὲ τάδε τοῖσι γυναικηΐ-
οισι χώροισι. 

\end{greek}

\section{Plutarch}
This is pseudo--Plutarch? Cites Dercullus -- find info on him.

[Dercyllus] Hist., Fragmenta (2196: 002)
“FHG 4”, Ed. Müller, K.
Paris: Didot, 1841–1870.
Fragment 8, line 2

E LIBRO TERTIO.

\begin{greek}
 Plutarch. De fluv. I, 4: Ἀλεξάνδρου τοῦ Μακε-
δόνος μετὰ στρατεύματος εἰς Ἰνδίαν ἐλθόντος, καὶ τῶν   
ἐγχωρίων κρίσιν ἐχόντων ἀντιπολεμεῖν αὐτῷ, Πώρου 
τοῦ βασιλέως τῶν τόπων ἐλέφας αἰφνιδίως οἰστροπλὴξ 
γενόμενος, ἐπὶ τὸν Ἡλίου λόφον ἀνέβη, καὶ ἀνθρωπίνῃ 
φωνῇ χρησάμενος εἶπεν· Δέσποτα βασιλεῦ, τὸ γένος 
ἀπὸ Γηγασίου κατάγων, μηδὲν ἐξ ἐναντίας Ἀλεξάνδρου 
ποιήσῃς· Διὸς γάρ ἐστι Γηγάσιος. 
\end{greek}


\section{Appian}

\blockquote[From Wikipedia.\footnote{\url{}.}]{Appian of Alexandria (play /ˈæpiən/; Ancient Greek: Ἀππιανός Ἀλεξανδρεύς, Appianós Alexandreús; Latin: Appianus Alexandrinus; ca. AD 95 – ca. AD 165) was a Roman historian of Greek ethnicity who flourished during the reigns of Emperors of Rome Trajan, Hadrian, and Antoninus Pius.}

Appianus Hist., Iberica (0551: 007)
“Appiani historia Romana, vol. 1”, Ed. Viereck, P., Roos, A.G., Gabba, E.
Leipzig: Teubner, 1939, Repr. 1962 (1st edn. corr.).
Section 147, line 2

\begin{greek}

ὧδε μὲν τὸ στρατόπεδον καθίστατο τῷ Σκιπίωνι· 
Ἰνδίβιλις δέ, τῶν συνθεμένων τις αὐτῷ δυναστῶν, στα-
σιαζούσης ἔτι τῆς Ῥωμαϊκῆς στρατιᾶς κατέδραμέν τι 
τῆς ὑπὸ τῷ Σκιπίωνι γῆς. 



Appianus Hist., Iberica 
Section 156, line 1

καὶ Σκιπίων μὲν θαυμαζόμενος ἐθριάμβευεν, Ἰνδί-
βιλις δ' οἰχομένου τοῦ Σκιπίωνος αὖθις ἀφίστατο. 



Appianus Hist., Annibaica (0551: 008)
“Appiani historia Romana, vol. 1”, Ed. Viereck, P., Roos, A.G., Gabba, E.
Leipzig: Teubner, 1939, Repr. 1962 (1st edn. corr.).
Section 176, line 3

ὃ δ' ἐπιτηρήσας νύκτα ἀσέληνον καὶ χωρίον, ἐν ᾧ 
Φούλβιος ἑσπέρας τεῖχος μὲν οὐκ ἔφθανεν ἐγεῖραι, τά-
φρον δ' ὀρυξάμενος καὶ διαστήματα ἀντὶ πυλῶν κατα-  
λιπὼν καὶ τὸ χῶμα προβαλὼν ἀντὶ τείχους ἡσύχαζεν, 
ἔς τε λόφον ὑπερκείμενον αὐτοῦ καρτερὸν ἔπεμψε λα-
θὼν ἱππέας, οἷς εἴρητο ἡσυχάζειν, ἕως οἱ Ῥωμαῖοι τὸν 
λόφον ὡς ἔρημον ἀνδρῶν καταλαμβάνωσι, τοῖς δ' ἐλέφασι 
τοὺς Ἰνδοὺς ἐπιβήσας ἐκέλευσεν ἐς τὸ τοῦ Φουλβίου 
στρατόπεδον ἐσβιάζεσθαι διά τε τῶν διαστημάτων καὶ 
διὰ τῶν χωμάτων, ὡς δύναιντο. 



Appianus Hist., Libyca (0551: 009)
“Appiani historia Romana, vol. 1”, Ed. Viereck, P., Roos, A.G., Gabba, E.
Leipzig: Teubner, 1939, Repr. 1962 (1st edn. corr.).
Section 324, line 5

                        αἴτιον δ' ἴσως ὅ τε χειμὼν οὐ 
πολὺ κρύος ἔχων, ὑφ' οὗ φθείρεται πάντα, καὶ τὸ 
θέρος οὐ κατακαῖον ὥσπερ Αἰθίοπάς τε καὶ Ἰνδούς. 



Appianus Hist., Syriaca (0551: 013)
“Appiani historia Romana, vol. 1”, Ed. Viereck, P., Roos, A.G., Gabba, E.
Leipzig: Teubner, 1939, Repr. 1962 (1st edn. corr.).
Section 281, line 7

                                              ἐφεδρεύων δὲ 
ἀεὶ τοῖς ἐγγὺς ἔθνεσι καὶ δυνατὸς ὢν βιάσασθαι καὶ 
πιθανὸς προσαγαγέσθαι ἦρξε Μεσοποταμίας καὶ Ἀρμενίας 
καὶ Καππαδοκίας τῆς Σελευκίδος λεγομένης καὶ Περσῶν 
καὶ Παρθυαίων καὶ Βακτρίων καὶ Ἀράβων καὶ Ταπύρων 
καὶ τῆς Σογδιανῆς καὶ Ἀραχωσίας καὶ Ὑρκανίας καὶ 
ὅσα ἄλλα ὅμορα ἔθνη μέχρις Ἰνδοῦ ποταμοῦ Ἀλεξάνδρῳ 
γεγένητο δορίληπτα, ὡς ὡρίσθαι τῷδε μάλιστα μετὰ 
Ἀλέξανδρον τῆς Ἀσίας τὸ πλέον· ἀπὸ γὰρ Φρυγίας ἐπὶ 
ποταμὸν Ἰνδὸν ἄνω πάντα Σελεύκῳ κατήκουε. 



Appianus Hist., Syriaca 
Section 282, line 2

                                                     καὶ τὸν   
Ἰνδὸν περάσας ἐπολέμησεν Ἀνδροκόττῳ, βασιλεῖ τῶν περὶ 
αὐτὸν Ἰνδῶν, μέχρι φιλίαν αὐτῷ καὶ κῆδος συνέθετο. 



Appianus Hist., Syriaca 
Section 288, line 1

                    Ἀλεξάνδρῳ γὰρ ἐξ Ἰνδῶν ἐς Βαβυ-
λῶνα ἐπανελθόντι καὶ τὰς ἐν αὐτῇ τῇ Βαβυλωνίᾳ λίμνας   
ἐπὶ χρείᾳ τοῦ τὸν Εὐφράτην τὴν Ἀσσυρίδα γῆν ἀρδεύειν 
περιπλέοντι ἄνεμος ἐμπεσὼν ἥρπασε τὸ διάδημα, καὶ 
φερόμενον ἐκρεμάσθη δόνακος ἐν τάφῳ τινὸς ἀρχαίου 
βασιλέως. 



Appianus Hist., Syriaca 
Section 298, line 5

                                                   τὰς δὲ ἄλλας ἐκ 
τῆς Ἑλλάδος ἢ Μακεδονίας ὠνόμαζεν ἢ ἐπὶ ἔργοις ἑαυτοῦ 
τισιν ἢ ἐς τιμὴν Ἀλεξάνδρου τοῦ βασιλέως· ὅθεν ἐστὶν 
ἐν τῇ Συρίᾳ καὶ τοῖς ὑπὲρ αὐτὴν ἄνω βαρβάροις πολλὰ 
μὲν Ἑλληνικῶν, πολλὰ δὲ Μακεδονικῶν πολισμάτων 
ὀνόματα, Βέρροια, Ἔδεσσα, Πέρινθος, Μαρώνεια, Καλλί-
πολις, Ἀχαΐα, Πέλλα, Ὠρωπός, Ἀμφίπολις, Ἀρέθουσα, 
Ἀστακός, Τεγέα, Χαλκίς, Λάρισσα, Ἥραια, Ἀπολλωνία, 
ἐν δὲ τῇ Παρθυηνῇ Σώτειρα, Καλλιόπη, Χάρις, Ἑκατόμ-
πυλος, Ἀχαΐα, ἐν δὲ Ἰνδοῖς Ἀλεξανδρόπολις, ἐν δὲ Σκύθαις 
Ἀλεξανδρέσχατα. 



Appianus Hist., Mithridatica (0551: 014)
“Appiani historia Romana, vol. 1”, Ed. Viereck, P., Roos, A.G., Gabba, E.
Leipzig: Teubner, 1939, Repr. 1962 (1st edn. corr.).
Section 407, line 1

                    ἐνέπιπτε δὲ τοῖς μαχομένοις ἐπὶ τῷ παρα-
λόγῳ τῆς ἀνακλήσεως θόρυβός τε καὶ ἀπορία, μή τι δει-
νὸν ἑτέρωθεν εἴη, μέχρι μαθόντες εὐθὺς ἐν τῷ πεδίῳ 
τὸ σῶμα περιίσταντο καὶ ἐθορύβουν, ἕως Τιμόθεος αὐ-  
τοῖς ὁ ἰατρός, ἐπισχὼν τὸ αἷμα, ἐπέδειξεν αὐτὸν ἐκ με-
τεώρου, οἷόν τι καὶ Μακεδόσιν ἐν Ἰνδοῖς, ὑπὲρ Ἀλεξάν-
δρου δεδιόσιν, ὁ Ἀλέξανδρος αὑτὸν ἐπὶ νεὼς θεραπευό-
μενον ἐπέδειξεν. 



Appianus Hist., Bellum civile (0551: 017)
“Appian's Roman history, vols. 3–4 (ed. H. White)”, Ed. Viereck, P.
Cambridge, Mass.: Harvard University Press, 1913, Repr. 3:1964; 4:1961.
Book 2, chapter 21, section 149, line 24

                     ἀπλώτου τε θαλάσσης ἐν 
Ἰνδοῖς ἀπεπείρασε, καὶ ἐπὶ κλίμακα πρῶτος 
ἀνέβη καὶ ἐς πολεμίων τεῖχος ἐσήλατο μόνος 
καὶ τρισκαίδεκα τραύματα ὑπέστη. 



Appianus Hist., Bellum civile 
Book 2, chapter 21, section 153, line 10

       ἐπανιόντα γὰρ ἐξ Ἰνδῶν ἐς Βαβυλῶνα 
μετὰ τοῦ στρατοῦ καὶ πλησιάζοντα ἤδη παρε-
κάλουν οἱ Χαλδαῖοι τὴν εἴσοδον ἐπισχεῖν ἐν τῷ 
παρόντι. 



Appianus Hist., Bellum civile 
Book 2, chapter 21, section 154, line 3

Ἐγένοντο δὲ καὶ ἐς ἐπιστήμην τῆς ἀρετῆς, 
τῆς τε πατρίου καὶ Ἑλληνικῆς καὶ ξένης, φιλό-
καλοι, τὰ μὲν Ἰνδῶν Ἀλέξανδρος ἐξετάζων τοὺς 
Βραχμᾶνας, οἳ δοκοῦσιν Ἰνδῶν εἶναι μετεωρο-
λόγοι τε καὶ σοφοὶ καθὰ Περσῶν οἱ Μάγοι,   
τὰ δὲ Αἰγυπτίων ὁ Καῖσαρ, ὅτε ἐν Αἰγύπτῳ 
γενόμενος καθίστατο Κλεοπάτραν. 



Appianus Hist., Bellum civile 
Book 5, chapter 1, section 9, line 27

                                                   ἀπο-
πλευσάσης δὲ τῆς Κλεοπάτρας ἐς τὰ οἰκεῖα, ὁ 
Ἀντώνιος ἔπεμπε τοὺς ἱππέας Πάλμυρα πόλιν, 
οὐ μακρὰν οὖσαν ἀπὸ Εὐφράτου, διαρπάσαι, 
μικρὰ μὲν ἐπικαλῶν αὐτοῖς, ὅτι Ῥωμαίων καὶ 
Παρθυαίων ὄντες ἐφόριοι ἐς ἑκατέρους ἐπιδεξίως 
εἶχον (ἔμποροι γὰρ ὄντες κομίζουσι μὲν ἐκ 
Περσῶν τὰ Ἰνδικὰ ἢ Ἀράβια, διατίθενται δ' ἐν 
τῇ Ῥωμαίων), ἔργῳ δ' ἐπινοῶν τοὺς ἱππέας περι-
ουσιάσαι. 

\end{greek}

\section{Dio Cassius}
\subsection{About Dio Cassius}
\blockquote[From Wikipedia]{Lucius Cassius Dio Cocceianus[1][2] (Ancient Greek: Δίων ὁ Κάσσιος, c. AD 150 – 235,[3] known in English as Cassius Dio, Dio Cassius, or Dio (Dione. lib) was a Roman consul and a noted historian writing in Greek. Dio published a history of Rome in 80 volumes, beginning with the legendary arrival of Aeneas in Italy through the subsequent founding of Rome (753 BC), the formation of the Republic (509 BC), and the creation of the Empire (31 BC), up to AD 229; a period of about 1,400 years. Of the 80 books, written over 22 years, many survive into the modern age intact or as fragments, providing modern scholars with a detailed perspective on Roman history.}

\subsection{On Indian embassies to Augustus (54.9.8)}
\label{dio_cassius_embassy}
Records same incident at Niclaus of Damascus (=Strabo \emph{Geographica} 15.1.72--73).
\blockquote[\emph{} \textquote{(8) For a great many embassies came to him, and the people of India, who had already made overtures, now made a treaty of friendship, sending among other gifts tigers, which were then for the first time seen by the Romans, as also, I think by the Greeks. They also gave him a boy who had no shoulders or arms, like our statues of Hermes. (9) And yet, defective as he was, he could use his feet for everything, as if they were hands: with them he would stretch a bow, shoot missiles, and put a trumpet to his lips. How he did this I do not know; I merely state what is recorded. (10) One of the Indians, Zarmarus, for some reason wished to die, — either because, being of the caste of sages, he was on this account moved by ambition, or, in accordance with the traditional custom of the Indians, because of old age, or because he wished to make a display for the benefit of Augustus and the Athenians (for Augustus had reached Athens);— he was therefore initiated into the mysteries of the two goddesses, which were held p307out of season on account, they say, of Augustus, who also was an initiate, and he then threw himself alive into the fire.}\footnote{From Thayer online from the old Loeb.}]{\textgreek{(8) πάμπολλαι γὰρ δὴ πρεσβεῖαι πρὸς αὐτὸν ἀφίκοντο, καὶ οἱ Ἰνδοὶ προκηρυκευσάμενοι πρότερον φιλίαν τότε ἐσπείσαντο, δῶρα πέμψαντες ἄλλα τε καὶ τίγρεις, πρῶτον τότε τοῖς Ῥωμαίοις, νομίζω δ' ὅτι καὶ τοῖς Ἕλλησιν, ὀφθείσας. καί τι καὶ μειράκιόν οἱ ἄνευ ὤμων, οἵους τοὺς Ἑρμᾶς ὁρῶμεν, ἔδωκαν. (9) καὶ μέντοι τοιοῦτον ὂν ἐκεῖνο ἐς πάντα τοῖς ποσὶν ἅτε καὶ χερσὶν ἐχρῆτο, τόξον τε αὐτοῖς ἐπέτεινε καὶ βέλη ἠφίει καὶ ἐσάλπιζεν, οὐκ οἶδ' ὅπως· γράφω γὰρ τὰ λεγόμενα. (10 εἷς δ' οὖν τῶν Ἰνδῶν Ζάρμαρος, εἴτε δὴ τοῦ τῶν σοφιστῶν γένους ὤν, καὶ κατὰ τοῦτο ὑπὸ φιλοτιμίας, εἴτε καὶ ὑπὸ τοῦ γήρως κατὰ τὸν πάτριον νόμον, εἴτε καὶ ἐς ἐπίδειξιν τοῦ τε Αὐγούστου καὶ τῶν Ἀθηναίων (καὶ γὰρ ἐκεῖσε ἦλθεν) ἀποθανεῖν ἐθελήσας ἐμυήθη τε τὰ τοῖν θεοῖν, τῶν μυστηρίων καίπερ οὐκ ἐν τῷ καθήκοντι καιρῷ, ὥς φασι, διὰ τὸν Αὔγουστον καὶ <αὐτὸν> μεμυημένον γενομένων, καὶ πυρὶ ἑαυτὸν ζῶντα ἐξέδωκεν.}\footnote{Text from \textcite{boissevain1895}.}}

\section{Testamentum Salomonis}%???

Note:Questionable entry. See text

\blockquote[From Wikipedia]{The Testament of Solomon is an Old Testament pseudepigraphical work, the authorship of which is ascribed to King Solomon. The text is only found in Christian sources and is not in the Jewish Tanakh or other Jewish sources. It describes how Solomon was enabled to build the Temple by commanding demons by means of a magical ring entrusted to him by the Archangel Michael.

Despite the text's claim to have been a first-hand account of King Solomon's construction of the Temple of Jerusalem, its original publication dates sometime between the 1st and 5th centuries CE,[1] over a thousand years after King Solomon's death and the temple's completion.

The real author or authors of the text remain unknown. The text was originally written in Greek and contains numerous theological and magical themes ranging from Christianity and Judaism to Greek mythology and astrology that possibly hint at a Christian writer with a Greek background.}

\subsubsection{Text}

What is this? The keyword here is \textgreek{ἰνδικτιόνος}.

\begin{greek}
Testamentum Salomonis, Conspectus titulorum (2679: 008)
“The testament of Solomon”, Ed. McCown, C.C.
Leipzig: Hinrichs, 1922.
Page 99, line 12

                                        ἐγράφη παρ' ἐμοῦ Ἰω(ἀννου) 
ἰατροῦ τοῦ αρο(?)· ἐν ἔτει ͵ϛϡμθʹ (ἰνδικτιόνος) δʹ ἐν μηνὶ Δε-
κε(μ)βρίῳ ιδʹ. 

\end{greek}


\section{Himerius}

\blockquote[From Wikipedia\footnote{\url{http://en.wikipedia.org/wiki/Himerius}}]{Himerius (ca. 315-386), Greek sophist and rhetorician. 24 of his orations have reached us complete, and fragments of 12 others.



Himerius was born at Prusa in Bithynia. He completed his education at Athens, whence he was summoned to Antioch in 362 by the emperor Julian to act as his private secretary. After the death of Julian in the following year Himerius returned to Athens, where he established a school of rhetoric, which he compared with that of Isocrates and the Delphic oracle, owing to the number of those who flocked from all parts of the world to hear him. Amongst his pupils were Gregory of Nazianzus and Basil the Great, bishop of Caesarea.

In recognition of his merits, civic rights and the membership of the Areopagus were conferred upon him. The death of his son Rufinus (his lament for whom, called the Μονῳδία, is extant) and that of a favourite daughter greatly affected his health; in his later years he became blind and he died of epilepsy. Although a pagan, who had been initiated into the mysteries of Mithras by Julian, his works show no prejudice against the Christians.

Himerius is a typical representative of the later rhetorical schools. Photius (cod. 165, 243 Bekker) had read 71 speeches by him, of 36 of which he has given an epitome; 24 have come down to us complete and fragments of 12 others. They consist of epideictic or "display" speeches after the style of Aristides, the majority of them having been delivered on special occasions, such as the arrival of a new governor,[1] visits to different cities (Thessalonica, Constantinople), or the death of friends or well-known personages.

The Polemarchicus, like the Menexenus of Plato and the Epitaphios Logos of Hypereides, is a panegyric of those who had given their lives for their country; it is so called because it was originally the duty of the polemarch to arrange the funeral games in honour of those who had fallen in battle. Other declamations, only known from the excerpts in Photius, were imaginary orations put into the mouth of famous persons--Demosthenes advocating the recall of Aeschines from banishment, Hypereides supporting the policy of Demosthenes, Themistocles inveighing against the king of Persia, an orator unnamed attacking Epicurus for atheism before Julian at Constantinople.

Himerius is more of a poet than a rhetorician, and his declamations are valuable as giving prose versions or even the actual words of lost poems by Greek lyric writers. The prose poem on the marriage of his pupil Severus and his greeting to Basil at the beginning of spring are quite in the spirit of the old lyric. Himerius possesses vigour of language and descriptive powers, though his productions are spoilt by too frequent use of imagery, allegorical and metaphorical obscurities, mannerism and ostentatious learning. But they are valuable for the history and social conditions of the time, although lacking the sincerity characteristic of Libanius.}

\begin{greek}

Himerius Soph., Declamationes et orationes (2051: 001)
“Himerii declamationes et orationes cum deperditarum fragmentis”, Ed. Colonna, A.
Rome: Polygraphica, 1951.
Oration 2, line 125

             .. 
 Ὁ δῆμος ὁ τῶν Ἀθηναίων Αἰσχίνῃ γράφει τὴν κάθ-
οδον, ὁ τὰς ἀπ' Ἰνδῶν φήμας ὡς ὅπλα καὶ μάχας φοβού-
μενος . 



Himerius Soph., Declamationes et orationes 
Oration 2, line 131

          .. 
 Κεῖται Βαβυλών, Δαρεῖος οἴχεται, Ἰνδοὺς ἀνῄρηκε, 
Πέρσαι δουλεύουσι, μόναι λείπουσιν Ἀθῆναι τοῖς κατορθώ-
μασιν, ὧν Πέρσαι τοσαυτάκις ᾔσθοντο τῆς ἀρετῆς, ὁσάκις 
ἔδει μάχεσθαι. 



Himerius Soph., Declamationes et orationes 
Oration 12, line 120

                                                                  .. 
 Ἀσία πᾶσα, οὐχ ἣν νῦν οὕτω προσαγορεύομεν, τὴν 
τῆς ὅλης ἐπωνυμίαν ἠπείρου τῷ μέρει μόνῳ τιθέμενοι, ἄρ-
χεται μὲν ἀπ' Ἰνδῶν ἄνω, πρὸς μὲν ἕω καὶ ἄρκτον Ἐρυθρῷ   
κόλπῳ καὶ Φάσιδι, πρὸς δὲ μεσημβρίαν καὶ ἀπιόντα ἥλιον 
Αἰγύπτῳ τε καὶ τῷ Ἰονίῳ πελάγει πρὸς τὰς ἄλλας ἠπεί-
ρους ἀποσχιζομένη καὶ λήγουσα· παρατείνει δὲ αὐτὴν ἐκ 
Προποντίδος εἰς Παμφυλίαν πλευρὰ παρήκουσα, ἣν Αἰγαῖος 
προσκλύζει σύμπασαν, ἐκ μιᾶς καὶ αὐτὸς ἀρχῆς τῇ πλευ-
ρᾷ ταύτῃ τικτόμενος . 



Himerius Soph., Declamationes et orationes 
Oration 18, line 7

ἦλθεν ἐπ' Ἰνδοὺς ὁ Διόνυσος, γένος τὴν Διονύσου χάριν 
ἀρνούμενον· ἦν δὲ ὁ μὲν στρατὸς Βάκχαι καὶ Σάτυροι, τὰ 
δὲ ὅπλα, νεβρίδες καὶ θύρσοι. 



Himerius Soph., Declamationes et orationes 
Oration 18, line 15

                                        ἐπεὶ δὲ ἐν ὅροις Καπ-
παδοκῶν ἦσαν ἀγόμενοι, σκηνοῦσι μὲν ἐπὶ τῷ χείλει τοῦ 
ποταμοῦ, ᾧ καὶ δώσειν τὴν ἐπωνυμίαν ἤμελλον· δεῆσαν 
δὲ τοῖς νάμασι λούσασθαι, ἀμείβεται μὲν ὁ ποταμὸς καὶ 
τὸ ἀργυροῦν ὕδωρ Ἰνδοῖς ὁμιλῆσαν μελαίνεται· οἱ δὲ ὅπερ   
ἦσαν αὐτοί, τοῦτο καὶ εἶναι καὶ καλεῖσθαι τὸν ποταμὸν 
ἀπεργάζονται . 



Himerius Soph., Declamationes et orationes 
Oration 48, line 160

τῆς ἀληθῶς θείας φύσεως τοῖς πειρωμένοις ἐπιδείκνυται· 
ἀρχική τε νόμων ἐθέλει καὶ πόλεων γίνεσθαι, πημαίνει δὲ 
οὐδένα πώποτε· οὐ γὰρ θέμις θείαν ποτὲ φύσιν κακοῦ τινος 
ἀνθρώποις αἰτίαν γίνεσθαι, κρείττων τέ ἐστι φόβων καὶ βα-
σιλεύει τῶν ἡδονῶν, καὶ καθαρὰ πάθους φαίνεται· σῶμα δὲ 
διαπλάττει πρὸς τὴν ἑαυτῆς φύσιν ἁρμόζουσα, ὄμμα μέ<λαν> 
ζητεῖ, πρόσωπον ἐμβριθές, μελῶν συμμετρίαν ἀληθῆ, ὃ δὴ 
κάλλος σοφῶν παῖδες ἐπονομάζουσιν, ἵνα καλόν τε καὶ γεν-
ναῖον ἐξ ἀμφοῖν τὸ σῶμα πήξασα οἷον θεοῦ τινος εἰκόνα 
τοῖς ἀνθρώποις παρέχῃ ἰνδάλλεσθαι. 



Himerius Soph., Declamationes et orationes 
Oration 48, line 291

                                                     τὸν Διόνυσον 
φασὶν οἱ μῦθοι, πρὶν εἰς θεῶν φύσιν ἐλθεῖν, ἱερά τε δρῶντα 
καὶ βουλόμενόν τι πλέον παρ' ἐκεῖνα μαθεῖν, οὕτως εἴς τε 
Αἴγυπτον καὶ τὸν Νεῖλον δραμεῖν, ἔτι δὲ παρ' Ἰνδούς τε 
καὶ Αἰθίοπας καὶ τοὺς ἄλλους ἀνθρώπους ἅπαντας πολυ-
πραγμονοῦντα τὴν φύσιν· ἐλθόντα δὲ ἐκεῖθεν εἰς Ἕλληνας   
ἀσμένως μὲν καὶ πᾶσιν ὀφθῆναι τοῖς Ἕλλησι καὶ τὰς ὑπ' αὐ-
τῶν τιμὰς ἀποδέξασθαι· βουλόμενον δὲ Ἀθηναίοις πρώτοις 
τῶν ἑαυτοῦ δώρων ἀπάρξασθαι καὶ ἀγωγίμων, Ἀθήναζε 
ἐλθεῖν· τοὺς δὲ Ἀθηναίους – τυχεῖν γὰρ τότε πανηγυρί-
ζοντας – δημοσίαν τε ἄγειν Διονύσῳ τὴν πανήγυριν καὶ 
τὸ ἐκ τούτου λοιπὸν ὡς θεῷ πομπεύειν τῷ Διονύσῳ ψη-
φίσασθαι. 



Himerius Soph., Declamationes et orationes 
Oration 61, line 18

                συνεφάπτεται δὲ καὶ παῖς τοῦ πηδαλίου 
τῷ γέροντι, καὶ Ἰνδῷ τοξότῃ τοῦ βέλους ὁ μανθάνων τὴν 
τέχνην ἔφηβος. 


\end{greek}


\section{Oribasius}

\blockquote[From Wikipedia\footnote{\url{http://en.wikipedia.org/wiki/Oribasius}}]{Oribasius or Oreibasius (Greek: Ὀρειβάσιος) (c. 320–400) was a Greek medical writer and the personal physician of the Roman emperor Julian the Apostate. He studied at Alexandria under physician Zeno of Cyprus[2] before joining Julian's retinue. He was involved in Julian's coronation in 361, and remained with the emperor until Julian's death in 363. In the wake of this event, Oribasius was banished to foreign courts for a time, but was later recalled by the emperor Valens.

Oribasius's major works, written at the behest of Julian, are two collections of excerpts from the writings of earlier medical scholars, a collection of excerpts from Galen and the Collectiones, a massive compilation of excerpts from other medical writers of the ancient world. The first of these works is entirely lost, and only 25 of the 70 (or 72) books of the Collectiones survive. The first five surviving books deal with food and drink.[1] This work preserves a number of excerpts from older writers whose writings have otherwise been lost, and has thus been valuable to modern scholars. The earliest known description of a string figure, presented as the surgical sling Plinthios Brokhos by Greek physician Heraklas, is among the preserved material.[2][3]}

\begin{greek}


Oribasius Med., Collectiones medicae (lib. 1–16, 24–25, 43–50) (0722: 001)
“Oribasii collectionum medicarum reliquiae, vols. 1–4”, Ed. Raeder, J.
Leipzig: Teubner, 6.1.1:1928; 6.1.2:1929; 6.2.1:1931; 6.2.2:1933; Corpus medicorum Graecorum, vols. 6.1.1–6.2.2.
Book 2, chapter 58, section 88, line 2

                           λεπάδες βραχεῖαί εἰσιν, ἔν τισι μείζους, ὡς 
ὀστρέων δοκεῖν μὴ ἐναλλάττειν· μέγισται δ' ἐν Ἰνδικῇ, ὡς καὶ τὰ 
ἄλλα πάντα. 



Oribasius Med., Collectiones medicae (lib. 1-16, 24-25, 43-50) 
Book 2, chapter 58, section 147, line 3

                       τάδε μὲν κυρίως καὶ συνήθως κλῄζεται ταρίχη, 
καίτοι συχνῶν καὶ πολυτελῶν ἰχθύων κατὰ τὰς νήσους ἁλιζομένων· 
τρίγλαι γὰρ καὶ φάγροι σκληροὶ ἐκ τῆς Ἰνδικῆς κομιζόμενοί εἰσι μὲν 
κητώδεις, οὐκ ἐνάριθμοι δὲ τοῖς καθαριωτέροις θαλαττίοις. 



Oribasius Med., Collectiones medicae (lib. 1-16, 24-25, 43-50) 
Book 8, chapter 25, section 22, line 1

Ῥόδων ἄνθους, ὀποῦ μήκωνος, ἀκακίας, κόμμεως, βαλαυστίου, 
ὑποκυστίδος χυλοῦ, τούτων ἑκάστου μέρη τρία, κηκῖδος, ἀρνογλώσσου 
σπέρματος, τούτων ἑκατέρου ἀνὰ δύο μοῖραι, λυκίου Ἰνδικοῦ <ἕν>. 



Oribasius Med., Collectiones medicae (lib. 1-16, 24-25, 43-50) 
Book 9, chapter 6, section 1, line 5

Ταῖς χώραις ἔνια μὲν ἀπὸ τῆς, ὡς ἂν εἴποι τις, κοσμικῆς θέσεως 
ὑπάρχει, τινὰ δ' ἀπὸ τῆς ἰδίας, τρίτα δ' ἀπὸ τῶν συμπτωμάτων· ἀπὸ 
μὲν τῆς κοσμικῆς θέσεως ψυχραῖς μὲν εἶναι ταῖς παρὰ τὸν Ἴστρον τε 
καὶ τὴν Μαιῶτιν λίμνην καί, καθόλου φάναι, ταῖς ἀρκτικαῖς, θερμαῖς 
δὲ ταῖς κατὰ τὴν Αἰθιοπίαν καὶ Ἰνδίαν καί, συνελόντι φάναι, ταῖς 
μεσημβριναῖς, εὐκράτοις δὲ ταῖς μέσαις τούτων. 



Oribasius Med., Collectiones medicae (lib. 1-16, 24-25, 43-50) 
Book 11, chapter alpha*, section 2, line 1

<Ἀγάλοχον> ξύλον ἐστὶ φερόμενον ἐκ τῆς Ἰνδίας καὶ Ἀραβίας 
ἐοικὸς θυείᾳ· ἔστι μὲν οὖν εὐῶδες, παραστῦφον ἐν τῇ γεύσει μετὰ 
ποσῆς πικρίας, φλοιὸν ἔχον δερματώδη καὶ ὑποποίκιλον. 



Oribasius Med., Collectiones medicae (lib. 1-16, 24-25, 43-50) 
Book 11, chapter alpha*, section 32, line 6

                                 γεννᾶται δ' ἐν Ἰνδίᾳ πλείστη, ἐξ ἧς καὶ 
τὸ πίεσμα κομίζεται· φύεται δὲ καὶ ἐν τῇ Ἀσίᾳ καί τισι παραθαλας-
σίοις τόποις καὶ νήσοις, ὡς ἐν Ἄνδρῳ. 



Oribasius Med., Collectiones medicae (lib. 1-16, 24-25, 43-50) 
Book 11, chapter beta, section 7, line 5

<Βδέλλιον> δάκρυόν ἐστι δένδρου Ἀραβικοῦ· δόκιμον δ' αὐτοῦ 
τὸ τῇ γεύσει πικρόν, διαυγές, ταυροκολλῶδες, λιπαρὸν διὰ βάθους καὶ 
εὐμάλακτον, ἀμιγὲς ξύλων καὶ ῥυπαρίας, εὐῶδες ἐν τῇ θυμιάσει, ἐοικὸς 
ὄνυχι· ἔστι δέ τι ῥυπαρὸν καὶ μέλαν, ἁδρόβωλον, παλαθῶδες, κομιζό-
μενον ἀπὸ τῆς Ἰνδικῆς· κομίζεται δὲ καὶ ἀπὸ τῆς Πετραίας, ῥητινῶ-
δες, ὑποπέλιον, δευτερεῦον τῇ δυνάμει. 



Oribasius Med., Collectiones medicae (lib. 1-16, 24-25, 43-50) 
Book 11, chapter epsilon, section 1, line 6

          ἔστι δέ τις καὶ Ἰνδική, ἔχουσα διαφύσεις λευκὰς καὶ κιρρὰς 
καὶ σπίλους ὁμοίως πυκνούς· πλὴν βελτίων ἡ πρώτη. 



Oribasius Med., Collectiones medicae (lib. 1-16, 24-25, 43-50) 
Book 11, chapter kappa, section 2, line 1

<Κάλαμος ἀρωματικὸς> φύεται μὲν ἐν Ἰνδίᾳ, ἔστι δ' αὐτοῦ 
κάλλιστος ὁ κιρρός, πυκνογόνατος καὶ εἰς πολλοὺς σκινδαλμοὺς θραυό-
μενος, γέμων ἀραχνίων τὴν σύριγγα ὑπολεύκων, ἔν τε τῇ διαμασήσει 
γλίσχρος, στυπτικός, ὑπόδριμυς. 



Oribasius Med., Collectiones medicae (lib. 1-16, 24-25, 43-50) 
Book 11, chapter kappa, section 6, line 2

<Καρδάμωμον> ἄριστον τὸ ἐκ τῆς Κομμαγηνῆς καὶ Ἀρμενίας καὶ 
Βοσπόρου κομιζόμενον· γεννᾶται δὲ καὶ ἐν Ἰνδίᾳ καὶ Ἀραβίᾳ. 



Oribasius Med., Collectiones medicae (lib. 1-16, 24-25, 43-50) 
Book 11, chapter kappa, section 31, line 2

<Κόστος> διαφέρει ὁ Ἀραβικός, λευκὸς ὢν καὶ κοῦφος, πλείστην 
ἔχων καὶ ἡδεῖαν τὴν ὀσμήν· δευτερεύει δ' ὁ Ἰνδικός, ἁδρὸς ὢν καὶ 
μέλας καὶ κοῦφος ὡς νάρθηξ· τρίτος δ' ἐστὶν ὁ Συριακός, βαρύς, τὴν 
χρόαν πυξώδης, πληκτικὸς τῇ ὀσμῇ· ἄριστος δ' ἐστὶν ὁ πρόσφατος, 
λευκός, πλήρης, δι' ὅλου πυκνός, ξηρός, ἀτερηδόνιστος, ἄβρωμος, τῇ 
γεύσει δηκτικὸς καὶ πυρώδης. 



Oribasius Med., Collectiones medicae (lib. 1-16, 24-25, 43-50) 
Book 11, chapter lambda, section 7, line 4

<Λιβανωτὸς> γεννᾶται μὲν ἐν Ἀραβίᾳ τῇ λιβανωτοφόρῳ καλου-
μένῃ· πρωτεύει δ' ὁ ἄρρην καλούμενος, ἄτμητος λευκός τε καὶ θλα-
σθεὶς ἔνδοθεν λιπαρὸς ἐπιθυμιαθείς τε ταχέως ἐκκαιόμενος· ὁ δ' Ἰν-
δικὸς ὑπόκιρρός ἐστι καὶ πελιδνὸς τὴν χρόαν. 



Oribasius Med., Collectiones medicae (lib. 1-16, 24-25, 43-50) 
Book 11, chapter lambda, section 17, line 15

ἔστι δὲ κάλλιστον τὸ καιόμενον λύκιον καὶ κατὰ τὴν σβέσιν τὸν καπνὸν 
ἐνερευθῆ ἔχον, ἔξωθεν μέλαν, διαιρεθὲν δὲ κιρρόν, ἄβρωμον, στῦφον 
μετὰ πικρίας, χρώματι κροκοειδές, οἷόν ἐστι τὸ Ἰνδικόν, διαφέρον τοῦ 
λοιποῦ καὶ δυναμικώτερον. 



Oribasius Med., Collectiones medicae (lib. 1-16, 24-25, 43-50) 
Book 11, chapter mu, section 6, line 1

<Μάγκορον> εἶδός ἐστι μέλιτος πεπηγότος ἐν Ἰνδίᾳ καὶ τῇ εὐδαί-
μονι Ἀραβίᾳ, εὑρισκόμενον ἐπὶ τῶν καλάμων, ὅμοιον ἁλσὶ τῇ συστάσει 
καὶ θρυβόμενον ὑπὸ τοῖς ὀδοῦσιν ὥσπερ οἱ ἅλες. 



Oribasius Med., Collectiones medicae (lib. 1-16, 24-25, 43-50) 
Book 12, chapter nu*, section 1, line 1

<Νάρδου> ἐστὶ δύο γένη· ἡ μὲν γάρ τις καλεῖται Ἰνδική, ἡ δὲ 
Συριακή, οὐχ ὅτι ἐν Συρίᾳ εὑρίσκεται, ἀλλ' ὅτι τοῦ ὄρους ἐν ᾧ γεννᾶ-
ται τὸ μὲν πρὸς Συρίαν τέτραπται, τὸ δὲ πρὸς Ἰνδούς. 



Oribasius Med., Collectiones medicae (lib. 1-16, 24-25, 43-50) 
Book 12, chapter nu*, section 1, line 7

                                                       τῆς δ' Ἰνδικῆς ἡ μέν 
τις λέγεται Γαγγῖτις ἀπό τινος ποταμοῦ παραρρέοντος, Γάγγου καλου-
μένου, παρ' ᾧ φύεται, ἀσθενεστέρα κατὰ δύναμιν οὖσα διὰ τὸ ἔφυδρον 
τῶν τόπων καὶ ἐπιμηκεστέρα πλείους τε ἔχουσα τοὺς στάχυας ἀπὸ 
τῆς αὐτῆς ῥίζης καὶ πολυκόμους καὶ περιπεπλεγμένους, βρωμώδης 
κατὰ τὴν ὀσμήν. 



Oribasius Med., Collectiones medicae (lib. 1-16, 24-25, 43-50) 
Book 12, chapter nu*, section 5, line 1

<Νάσκαφον> (οἱ δὲ νάκαφθον) ἐκ τῆς Ἰνδικῆς κομίζεται. 



Oribasius Med., Collectiones medicae (lib. 1-16, 24-25, 43-50) 
Book 12, chapter omicron, section 6, line 2

<Ὄνυξ> πῶμά ἐστι κογχυλίου ὅμοιον τῷ τῆς πορφύρας, εὑρισκό-
μενον ἐν Ἰνδίᾳ ἐν ταῖς ναρδοφόροις λίμναις· διὸ καὶ ἀρωματίζει 
νεμομένων τῶν κογχυλίων τὴν νάρδον. 



Oribasius Med., Collectiones medicae (lib. 1-16, 24-25, 43-50) 
Book 12, chapter sigma, section 28, line 1

<Σκίγκος> ὁ μέν τίς ἐστιν Αἰγύπτιος, ὁ δ' Ἰνδικός, ἄλλος δ' ἐν 
τῇ Ἐρυθρᾷ γεννώμενος, ἕτερος δ' ἐν τῇ Ἀπολλωνίᾳ τῆς Μαυρουσιάδος 
εὑρίσκεται. 



Oribasius Med., Collectiones medicae (lib. 1-16, 24-25, 43-50) 
Book 13, chapter iota, section 1, line 1

                                                                 γεννᾶται 
πλεῖστον ἐν Μήλῳ καὶ Λιπάρᾳ. 
 <Ἰνδικὸν> τὸ μὲν αὐτομάτως γίνεται οἱονεὶ ἐκβρασμάτιον τῶν 
Ἰνδικῶν καλάμων, τὸ δὲ βαφικόν ἐστιν ἐπανθισμὸς πορφύρας ἐπαιω-
ρούμενος τοῖς χαλκείοις, ὃν ἀποσύραντες ξηραίνουσιν οἱ τεχνῖται. 



Oribasius Med., Collectiones medicae (lib. 1-16, 24-25, 43-50) 
Book 13, chapter iota, section 1, line 2

<Ἰνδικὸν> τὸ μὲν αὐτομάτως γίνεται οἱονεὶ ἐκβρασμάτιον τῶν 
Ἰνδικῶν καλάμων, τὸ δὲ βαφικόν ἐστιν ἐπανθισμὸς πορφύρας ἐπαιω-
ρούμενος τοῖς χαλκείοις, ὃν ἀποσύραντες ξηραίνουσιν οἱ τεχνῖται. 



Oribasius Med., Collectiones medicae (lib. 1-16, 24-25, 43-50) 
Book 14, chapter 33, section 7, line 5

                                                       μὴ τοίνυν θαύμαζε, 
εἰ κάλαμοι ξηροὶ καὶ τρίχες εὐέκκαυτα μέν ἐστιν, οὐ μὴν ἡμᾶς γε 
θερμαίνει πλησιάζοντα· τὴν ἀρχὴν γὰρ οὐδὲ μεταβάλλεται πρὸς τῆς 
ἐν ἡμῖν θερμασίας, ἵνα ἀντιθερμήνῃ, διὰ τὸ μὴ δύνασθαι καταθραυσθῆ-
ναι χνοωδῶς, ἐπεὶ ὅ γε κάλαμος ὁ ἐξ Ἰνδίας τῷ κόπτεσθαί τε καὶ 
διαττᾶσθαι χνοωδῶς μᾶλλον τοῦ παρ' ἡμῖν ἐναργῶς φαίνεται θερμαί-
νων. 



Oribasius Med., Collectiones medicae (lib. 1-16, 24-25, 43-50) 
Book 15, chapter 1:21, section 13, line 3

                                                – Φοῦ ἡ ῥίζα νάρδῳ παρα-
πλησία τὴν δύναμίν ἐστιν, ἀλλ' εἰς μὲν τὰ πλεῖστα καταδεεστέρα· 
προτρέπει δ' οὖρα τῆς Ἰνδικῆς καὶ Συριακῆς μᾶλλον, ὁμοίως δὲ τῇ 
Κελτικῇ. 



Oribasius Med., Eclogae medicamentorum (0722: 003)
“Oribasii collectionum medicarum reliquae, vol. 4”, Ed. Raeder, J.
Leipzig: Teubner, 1933; Corpus medicorum Graecorum, vol. 6.2.2.
Chapter 9, section 1, line 10

                        καδμείας 𐅻 <ιϛ>, ψιμυθίου 𐅻 <ϛ>, καστορίου 𐅻 <ϛ>, νάρ-
δου Ἰνδικῆς 𐅻 <δ>, στίμμεως 𐅻 <μ>, ἀλόης 𐅻 <ϛ>, κασσίας 𐅻 <δ>, λεπίδος 𐅻 <ε>, 
χαλκοῦ κεκαυμένου 𐅻 <ιϛ>, ῥόδων ἄνθους 𐅻 <η>, λυκίου Ἰνδικοῦ 𐅻 <γ>, λί-
θου σχιστοῦ 𐅻 <δ> 𐅶, κρόκου 𐅻 <ϛ>, μολύβδου κεκαυμένου καὶ πεπλυμένου 
𐅻 <η>, ὀπίου δραχμαὶ <γ>, ἀκακίας 𐅻 <μ>, κόμμεως 𐅻 <μη>, ὕδωρ ὄμβριον· δε-
δοκιμασμένον ἄγαν. 



Oribasius Med., Eclogae medicamentorum 
Chapter 15, section 1, line 3

Χαλκῖτιν λεάνας ἀνάλαβε ἐλλυχνίῳ δεδευμένῳ ὕδατι ἢ πριαπίσκῳ 
καὶ ἐντίθει τοῖς μυκτῆρσιν· ἢ ὠοῦ ὄστρακον καύσας μίσγε αὐτῷ κηκῖ-
δος τὸ ἥμισυ καὶ ὡσαύτως χρῶ· ἢ λυκίῳ Ἰνδικῷ διάψα· ἢ ὀνίδα 
καύσας τὴν [αὐτὴν] σποδὸν ἐμφύσα· ἢ χυλίσας τὴν ὀνίδα ἔνσταζε τὸν 
χυλόν· ἢ μυλίτου λίθου ἐκ πυρᾶς σβεσθέντος ὄξει τὴν ἀτμίδα ὀσφραι-
νέσθω. 



Oribasius Med., Eclogae medicamentorum 
Chapter 54, section 15, line 8

                                                                 > 
Νάρδος πινομένη στεγνοῖ κοιλίαν, σμύρνης καλῆς κυαμιαῖον μέγεθος 
στερεὸν καταπινόμενον, σπέρμα ἀγρίου λαπάθου ὕδατι ἐπιπασθέν, λα-
γωοῦ πυτίας τριώβολον σὺν ὕδατι, κέρατος ἐλαφείου κεκαυμένου καὶ 
πεπλυμένου κοχλιάρια <β> σὺν τραγακάνθῃ, ἢ αὐτὴ ἡ τραγάκανθα, ἢ 
ῥοῦς ἢ κηκὶς ὀμφακίνη ἢ σίδια ἢ βάτου ῥιζῶν ἀφέψημα ἀποτριτωθὲν 
ἢ λάδανον σὺν οἴνῳ αὐστηρῷ ἢ ἀκακίας χυλὸς ἢ ὑποκιστίδος ἢ λύκιον 
Ἰνδικὸν ἢ Σάμιος ἀστὴρ ἢ βαλαύστιον. 



Oribasius Med., Eclogae medicamentorum 
Chapter 76, section 25, line 3

                                                                  αἱ δ' 
ὀχθώδεις ὑπεροχαὶ φλεγμαίνουσαι ἢ εἱλκωμέναι καταχριέσθωσαν λυκίῳ 
Ἰνδικῷ ἢ γλαυκίῳ ἢ ἀλόῃ ἢ τῷ Ἀνδρωνείῳ τροχίσκῳ ἢ τοῖς ὁμοίοις. 



Oribasius Med., Eclogae medicamentorum 
Chapter 87, section 2, line 1

                                                          – <Ἡ Ἰνδή. 



Oribasius Med., Eclogae medicamentorum 
Chapter 87, section 10, line 17

             δεῖ τοίνυν ἀπέχεσθαι τῆς ἀγωγῆς ταύτης ἐπὶ νευροτρώτων 
ἢ νευροθλάστων, θεραπεύειν δὲ τρόπον ὃν ὁ <Γαληνὸς> ἐξεῦρεν οὕ-
τως, ἐπιτιθέντα μὴ σκληρόν, ἀλλ' ὥσπερ ἔμμοτον ἀνιέμενον φάρμακον, 
ὁποῖόν ἐστι τό τε ὑφ' ἡμῶν καλούμενον κίσσινον καὶ τὸ ἐμφερὲς 
αὐτῷ τὸ Γαλήνειον καὶ τὸ μελάγχλωρον ἥ τε Ἰνδὴ καὶ ἡ Ἀθηνᾶ 
ἀνεθεῖσαι ἤ τι τῶν ὁμοίων, ἐπάνω τε ἔρια ἐλαίῳ θερμῷ διάβροχα,   
θεραπεύειν δὲ δὶς τῆς ἡμέρας, ὄρθρου καὶ ἑσπέρας, καταιονοῦντα 
ἐλαίῳ θερμῷ ῥέποντι ἐπὶ τὸ μετρίως θερμότερον· τὸ γὰρ χλιαρὸν 
ἐμπλασσόμενον οὐκ ἐπιτρέπει τοῖς σώμασι διαπνεῖσθαι. 



Oribasius Med., Eclogae medicamentorum 
Chapter 89, section 8, line 2

                                              μετὰ δὲ ταῦτα τοῖς ξη-
ραίνουσι χρηστέον, καθάπερ τῇ Ἰνδῇ καὶ τῇ Ἀθηνᾷ καὶ τῷ μελαγχλώ-
ρῳ τροχίσκῳ· ἐπὶ τέλει δὲ κατουλοῦν ἢ τῷ διὰ καδμείας ἢ τῇ Ἰνδικῇ 
ἢ τῇ Ἀθηνᾷ. 



Oribasius Med., Eclogae medicamentorum 
Chapter 89, section 17, line 1

                                                      καὶ τὰς ἐπὶ τραύματι 
δὲ φλεγμονάς, αἳ γίνονται νικηθέντων τῶν ἀφλεγμάντων φαρμάκων, 
θεραπεύειν καταιονοῦντα μὲν ὕδατι θερμῷ ποτίμῳ ἢ ὑδρελαίῳ, αὐτῷ 
δὲ τῷ ἕλκει τετραφάρμακον ἐπιτιθέντα, ἢ τὸ Μακεδονικὸν ἢ τὴν τοῦ 
Ἀζανίτου, ἀνιεμένας ῥοδίνῳ ἢ ἄλλῳ τινὶ τῶν χαλαστικῶν ἐλαίων, 
ἄνωθεν δὲ καταπλάσσοντα δι' ὑδρελαίου καὶ πυρίνου ἀλεύρου ἢ κρι-
θίνου ἢ ἐξ ἀμφοῖν μικτοῦ, ἐν παρακμῇ δὲ τοῖς ξηραίνουσιν ὡς τῇ 
Ἰνδικῇ καὶ τῇ Ἀθηνᾷ, καὶ τὰ λοιπὰ ἀκολούθως. 



Oribasius Med., Eclogae medicamentorum 
Chapter 97, section 45, line 8

                                                            ἐπὶ δὲ τῶν κόλ-
πους ἐχόντων, μετὰ τὴν ἀνακάθαρσιν, ἣν ἐπιγνωσόμεθα ἐκ τοῦ μηκέτι 
πύον ἐπιφέρεσθαι, κομισάμενοι τοὺς τελαμῶνας, ἐγκλύσομεν πάντα 
τὸν κόλπον οἰνομέλιτι, ἐνιέντες διὰ πασῶν τῶν διαιρέσεων, εἶτα κολ-
λύρια ἐκ μέλιτος ἑφθοῦ πεποιημένα ἐνθήσομεν αὐτοῖς ἐκπληροῦντες 
τὰς ὑποφοράς, ἄνωθέν τε σπλήνιον δυνάμεως παρακολλητικῆς ἐπιβα-
λοῦμεν· δύναται δὲ παρακολλᾶν ἥ τε βάρβαρος καὶ πᾶσαι αἱ δι' 
ἀσφάλτου καὶ ἡ Ἀθηνᾶ καὶ ἡ δι' ἰτεῶν ἥ τε Ἰνδικὴ καὶ ἡ φαιά, καὶ 
μᾶλλον πασῶν ἡ τοῦ ἁλιέως, ᾗ ἡμεῖς χρώμεθα· τοὺς δὲ λεπτὰ ἔχον-
τας τὰ ἐπικείμενα σώματα αἱ δι' ἁλῶν. 



Oribasius Med., Eclogae medicamentorum 
Chapter 98, section 2, line 7

                                                            ἀνακαθαίρουσιν 
αἱ δι' ἁλῶν κηρωταὶ συντακεῖσαι, ἥ τε Ἰνδικὴ καὶ ὁ μελάγχλωρος 
τροχίσκος καὶ ἡ Ἀθηνᾶ καὶ αἱ χλωραὶ ἀνιέμεναι. 



Oribasius Med., Eclogae medicamentorum 
Chapter 147, section 4, line 2

                                                     προποτίζειν δὲ βαλαύ-
στιον μετ' ὀξυκράτου ἢ ὑποκιστίδος χυλὸν ἢ ἀκακίας ἢ λύκιον Ἰνδι-
κὸν ἢ Σάμιον ἀστέρα ἢ ὄμφακα ξηρόν· πρῶτον δ' ἄμεινον σήσαμον 
ὀξυκράτῳ βρεχόμενον, ἄχρις οὗ τρυφερὸν γένηται, καὶ οὕτως ἐσθιό-
μενον· δυνατὸν δὲ καὶ ποτίζειν αὐτό. 



Oribasius Med., Eclogae medicamentorum 
Chapter 147, section 11, line 1

ἔνεργοι δὲ πρὸς τοῦτο καὶ αἱ κολλητικαὶ πᾶσαι ἔμπλαστροι ἥ τε ἁρ-
μονία καὶ ἡ Ἱκεσίου καὶ ἡ Ἀθηνᾶ καὶ ἡ δι' ἰτεῶν, καὶ ἡ μηλίνη καὶ 
ἡ Ἰνδή. 



Oribasius Med., Synopsis ad Eustathium filium (0722: 004)
“Oribasii synopsis ad Eustathium et libri ad Eunapium”, Ed. Raeder, J.
Leipzig: Teubner, 1926, Repr. 1964; Corpus medicorum Graecorum, vol. 6.3.
Book 2, chapter 56, section 22, line 3

                                   – Κόστος καλλίων ἐστὶν ὁ Ἀραβικός, 
λευκὸς ὢν καὶ κοῦφος καὶ πλείστην ἔχων καὶ ἡδεῖαν τὴν ὀσμήν· δευ-
τερεύει δ' ὁ Ἰνδικός, <ἁδρὸς ὢν καὶ μέλας καὶ κοῦφος ὡς νάρθηξ· 
τρίτος δ' ἐστὶν ὁ Συριακὸς βαρύς>, τὴν χρόαν ὢν πυξώδης, πληκτικὸς 
τῇ ὀσμῇ. 



Oribasius Med., Synopsis ad Eustathium filium 
Book 2, chapter 56, section 67, line 1

                – Θεῖον ἄριστον τὸ ἄπυρον καὶ λαμπυρίζον τῇ χρόᾳ, 
διαφανές τε καὶ ἄλιθον· τοῦ δὲ πεπυρωμένου τὸ χλωρὸν καὶ εὐλιπές. 
– Ἰνδικὸν ἄριστόν ἐστι τὸ κυανοειδές τε καὶ ἔγχυλον λεῖον. 



Oribasius Med., Synopsis ad Eustathium filium 
Book 3, chapter 67, section 1, line 4

       Κηροῦ μναῖ <γ>, ἀμμωνιακοῦ μναῖ <β>, ῥητίνης φρυκτῆς μνᾶ <α>, 
μελιλώτου μνᾶς ἥμισυ, προπόλεως, σμύρνης, στύρακος, νάρδου Κελτικῆς, 
κυπέρου Ἰνδικῆς, ἴρεως Ἰλλυρικῆς, καρδαμώμου, πάνακος ἀνὰ 𐅻 <κε>, 
κρόκου 𐅻 <κ>, κασίας, μαστίχης Χίας, ὀποβαλσάμου, ἀμώμου ἀνὰ 𐅻 <ιϛ>, 
οἴνου Ἰταλικοῦ εὐώδους ὅσον ἐξαρκεῖ, νάρδου Ἀσιανῆς ἀρωματικῆς 
𐆃 <α>. 



Oribasius Med., Synopsis ad Eustathium filium 
Book 3, chapter 138, section 1, line 2

Σποδίου 𐅻 <δ> (οἱ δὲ 𐅻 <α>), φλοιοῦ λιβάνου ὀβολόν, σμύρνης ὀβο-
λόν, λεπίδος χαλκοῦ ὀβολόν, ἀκακίας, νάρδου Ἰνδικῆς, μηκωνίου πεφω-
σμένου ἀνὰ ὀβολόν, κόμμεως 𐅻 <δ>, ὕδωρ ὄμβριον. 



Oribasius Med., Synopsis ad Eustathium filium 
Book 3, chapter 139, section 1, line 3

                                                                      Ἔστι 
δὲ τοιόνδε· χαλκοῦ κεκαυμένου καὶ πεπλυμένου ὀβολοὶ <δ>, κρόκου 𐅻 <α>, 
σμύρνης ὀβολοὶ <γ>, νάρδου Ἰνδικῆς ὀβολοὶ <β>, κινναμώμου ὀβολοὶ <β>, 
μήκωνος ὀβολοὶ <β>, πεπέρεως κόκκοι <ιε>, κόμμεως ὀβολοὶ <γ>, οἴνου Χίου 
τὸ ἱκανόν. 



Oribasius Med., Synopsis ad Eustathium filium 
Book 3, chapter 147, section 1, line 2

Στίμμεως 𐅻 <ιϛ>, μολύβδου κεκαυμένου 𐅻 <η>, λεπίδος, κρόκου, ῥόδων 
ἄνθους, σμύρνης, νάρδου Ἰνδικῆς, λιβάνου ἄρρενος, πεπέρεως λευκοῦ 
ἀνὰ 𐅻 <α>, φοινίκων ὀστᾶ <λ>. 



Oribasius Med., Synopsis ad Eustathium filium 
Book 3, chapter 158, section 1, line 4

            Ῥόδων χλωρῶν χωρὶς τῶν λοβῶν 𐅻 <οβ>, καδμείας 𐅻 <κδ>, 
κρόκου 𐅻 <ϛ>, ὀπίου 𐅻 <γ>, στίμμεως 𐅻 <β>, χαλκοῦ 𐅻 <β>, νάρδου Ἰνδικῆς 
𐅻 <α>, σμύρνης 𐆄 <γ>, κόμμεως 𐅻 <κδ>, ὕδωρ ὄμβριον. 



Oribasius Med., Synopsis ad Eustathium filium 
Book 3, chapter 160, section 1, line 3

                                                          Καδμείας, χαλκοῦ, 
κρόκου, λεπίδος χαλκοῦ ἀνὰ 𐅻 <ιβ>, σμύρνης, λίθου αἱματίτου, ῥόδων 
ξηρῶν, νάρδου Ἰνδικῆς, ὀπίου ἀνὰ 𐅻 <δ>, πεπέρεως λευκοῦ κόκκοι <κ>, 
κόμμεως 𐅻 <ιβ>, οἴνου Χίου τὸ αὔταρκες. 



Oribasius Med., Synopsis ad Eustathium filium 
Book 3, chapter 174, section 1, line 5

Ἀνήσσου σπέρματος, <σελίνου σπέρματος, ἄμεως σπέρματος>, σχίνου 
ἄνθους, στυπτηρίας σχιστῆς, ἴρεως, βησασᾶ, ὅ τινες ἁρμαλᾶ καλοῦσι, 
κινναμώμου, σμύρνης τρωγλίτιδος, ἀριστολοχίας μακρᾶς, κασίας, κροκο-
μάγματος, ῥόδων ξηρῶν ἀνὰ 𐆄 <α>, κόστου, χελιδόνων νοσσιᾶς σποδοῦ 
προσφάτου ἀνὰ 𐆄 <γ>, κρόκου 𐆄 <α>, νάρδου Ἰνδικῆς, ἀμώμου ἀνὰ 𐆄 𐅶, 
κηκῖδας <η>. 



Oribasius Med., Synopsis ad Eustathium filium 
Book 8, chapter 25, section 3, line 3

                           εὐώδη δ' αὐτὰ ποιῆσαι βουλόμενος μίξεις 
κυπέρου καὶ μελιλώτου καὶ ῥόδων ξηρῶν καὶ σχίνου ἄνθους ἴρεώς τε 
καὶ στάχυος, νάρδου τε τῆς Ἰνδικῆς καὶ τῆς Κελτικῆς καὶ μαράθρου 
φύλλων καὶ ἀμώμου καὶ κόστου. 



Oribasius Med., Synopsis ad Eustathium filium 
Book 9, chapter 10, section 24, line 1

               – <Ἄλλο·> ῥόδων χλωρῶν τῶν φύλλων 𐅻 <ϛ>, γλυκυρρίζης 
𐅻 <δ>, νάρδου Ἰνδικῆς 𐅻 <δ>. 



Oribasius Med., Libri ad Eunapium (lib. 1–4) (0722: 005)
“Oribasii synopsis ad Eustathium et libri ad Eunapium”, Ed. Raeder, J.
Leipzig: Teubner, 1926, Repr. 1964; Corpus medicorum Graecorum, vol. 6.3.
Book 2, chapter 1,lambda, section 31, line 1

                  τὸ δ' Ἰνδικὸν λύκιον ἰσχυρότερόν ἐστι πρὸς πάντα. 



Oribasius Med., Libri ad Eunapium (lib. 1-4) 
Book 2, chapter 1,nu, section 3, line 1

                                                                γενναιο-
τέρα δ' ἐστὶν ἡ Ἰνδική, μελαντέρα τῆς Συριακῆς οὖσα. 



Oribasius Med., Libri ad Eunapium (lib. 1-4) 
Book 4, chapter 25, section 1, line 2

Καλλιβλέφαρον·


μάλιστα δὲ ποιεῖ νηπίοις καὶ τοῖς ἁπαλοσάρκοις· στίμμεως 𐅻 <ιϛ>, μο-
λύβδου 𐅻 <η>, κρόκου [ἀνὰ] 𐅻 <α>, ῥόδων ἄνθους, σμύρνης, νάρδου Ἰνδι-
κῆς, πεπέρεως λευκοῦ, λιβάνου ἄρρενος ἀνὰ 𐅻 <α>, φοινίκων ὀστᾶ <λ>· 
πάντα βαλὼν εἰς ἄγγος κεραμεοῦν ὄπτα φιλοπόνως, εἶτα τρίψας ἐν 
θυΐᾳ ἐπίβαλε ὀποβαλσάμου κοχλιάρια <β>, καὶ ἀνακόψας <καὶ> ξηράνας χρῶ. 



Oribasius Med., Libri ad Eunapium (lib. 1-4) 
Book 4, chapter 83, section 23, line 2

                       – <Ἄλλο·> ῥόδων χλωρῶν τῶν φύλλων 𐅻 <ϛ>, γλυ-
κυρρίζης 𐅻 <δ>, νάρδου Ἰνδικῆς 𐅻 <δ> . 

\end{greek}


\section{Philumenus}

\blockquote[From Wikipedia\footnote{\url{http://en.wikipedia.org/wiki/Philumenus}}]{Philumenus (Greek: Φιλούμενος), a Greek physician, mentioned by an anonymous writer as one of the most eminent members of his profession. Nothing is known of the events of his life, and with respect to his date, as the earliest author who quotes him is Oribasius,[1] it can only be said that he must have lived in or before the 4th century. It was thought that none of his work survived until 1907 when a manuscript of excerpts of his work De venenatis animalibus eorumque remediis (On poisonous animals and their remedies) was discovered in the Vatican library (codex Vaticanus gr. 284) by the German scholar Wellmann, who published an edition in 1908. [2]

Numerous fragments are preserved by Aëtius Amidenus. He is quoted also by Alexander of Tralles,[3] and Rhazes.[4]}

\begin{greek}
Philumenus Med., De venenatis animalibus eorumque remediis (0671: 001)
“Philumeni de venenatis animalibus eorumque remediis”, Ed. Wellmann, M.
Leipzig: Teubner, 1908; Corpus medicorum Graecorum, vol. 10.1.1.
Chapter 10, section 1, line 6

                                                                         δίδο-
ται τοῦ φαρμάκου καρύου Ποντικοῦ τὸ μέγεθος, σκορπιοπλήκτοις μὲν 
μετ' οἴνου αὐστηροῦ ἅπαξ τῆς ἡμέρας – εἰ δ' ἐπιτείνοι ἡ ὀδύνη, 
δίδου καὶ δίς – , πρὸς δὲ λυσσοδήκτους δίδου με|θ' ὕδατος ἡμέραν 
παρ' ἡμέραν ἐπὶ ἡμέρας <μβ>, προσμίσγων τῷ φαρμάκῳ λύκιον Ἰνδικόν, 
τηρῶν καὶ τὰ ἕλκη ἀκατούλωτα. 


Philumenus Med., De venenatis animalibus eorumque remediis 
Chapter 10, section 3, line 3

                                                   ἔστι δὲ καστορίου, ὀποῦ 
Κυρηναικοῦ, πεπέρεως ἀνὰ 𐅻 <δ>, κόστου, νάρδου Ἰνδικ<ῆς, κρόκ>ου, κεν-
ταυρίου χυλοῦ ἀνὰ 𐅻 <β>, μέλιτος ἀπέφθου κοτύλης ἥμισυ. 

\end{greek}


\section{Cephalion}
\blockquote[From Wikipedia\footnote{\url{http://en.wikipedia.org/wiki/Cephalion}}]{Roman historian of the time of Hadrian. Wrote a history of Assyria from the time of Ninus and Semiramis to that of Alexander the Great. It was written in the Ionic dialect, and was divided into nine books, called by the names of the Muses; and as in this he aped Herodotus, so he is reported to have aimed at resembling Homer by concealing his birthplace. Hadrian banished him to Sicily where this work was composed.}

\begin{greek}
Cephalion Hist., Rhet., Fragmenta (1249: 003)
“FHG 3”, Ed. Müller, K.
Paris: Didot, 1841–1870.
Fragment 1a, line 13

                    Μεθ' ὃν Βαβυλῶνα, φησὶν, ἡ Σεμί-
ραμις ἐτείχισε, τρόπον ὡς πολλοῖσι λέλεκται, Κτησίᾳ, 
Δείνωνι], Ἡροδότῳ καὶ τοῖς μετ' αὐτούς· 
στρατείην τε αὐτῆς κατὰ τῶν Ἰνδῶν καὶ ἧτταν, καὶ 
ὅτι τοὺς ἰδίους ἀνεῖλεν υἱοὺς καὶ ὑπὸ Νινύου τῶν παί-
δων ἑνὸς ἀνῃρέθη, τοῦ διαδεξαμένου τὴν ἀρχήν. 
\end{greek}


\section{Athanasius}
\blockquote[From Wikipedia\footnote{\url{http://en.wikipedia.org/wiki/Athanasius}}]{Athanasius of Alexandria (Greek: Ἀθανάσιος Ἀλεξανδρείας, Athanásios Alexandrías) (b. ca. 296-298 – d. 2 May 373), also referred to as St. Athanasius the Great, St. Athanasius I of Alexandria, St Athanasius the Confessor and (primarily in the Coptic Orthodox Church) St Athanasius the Apostolic, was the 20th bishop of Alexandria. His episcopate lasted 45 years (c. 8 June 328 – 2 May 373), of which over 17 were spent in five exiles ordered by four different Roman emperors. He is considered to be a renowned Christian theologian, a Church Father, the chief defender of Trinitarianism against Arianism, and a noted Egyptian leader of the fourth century.

He is remembered for his role in the conflict with Arius and Arianism. In 325, at the age of 27, Athanasius had a leading role against the Arians in the First Council of Nicaea. At the time, he was a deacon and personal secretary of the 19th Bishop of Alexandria, Alexander. Nicaea was convoked by Constantine I in May–August 325 to address the Arian position that Jesus of Nazareth is of a distinct substance from the Father.[1]

In June 328, at the age of 30, three years after Nicæa and upon the repose of Bishop Alexander, he became archbishop of Alexandria. He continued to lead the conflict against the Arians for the rest of his life and was engaged in theological and political struggles against the Emperors Constantine the Great and Constantius II and powerful and influential Arian churchmen, led by Eusebius of Nicomedia and others. He was known as "Athanasius Contra Mundum". Within a few years of his departure, St. Gregory of Nazianzus called him the "Pillar of the Church". His writings were well regarded by all Church fathers who followed, in both the West and the East. His writings show a rich devotion to the Word-become-man, great pastoral concern, and profound interest in monasticism.

Athanasius is counted as one of the four great Eastern Doctors of the Church in the Roman Catholic Church[2] and in Eastern Orthodoxy, he is labeled the "Father of Orthodoxy". He is also celebrated by many Protestants, who label him "Father of The Canon". Athanasius is venerated as a Christian saint, whose feast day is 2 May in Western Christianity, 15 May in the Coptic Orthodox Church, and 18 January in the other Eastern Orthodox Churches. He is venerated by the Roman Catholic Church, Oriental and Eastern Orthodox churches, the Lutherans, and the Anglican Communion.}
\begin{greek}

\subsection{Contra gentes}
Athanasius Theol., Contra gentes (2035: 001)
“Athanasius. Contra gentes and de incarnatione”, Ed. Thomson, R.W.
Oxford: Clarendon Press, 1971.
Section 9, line 38

                                          οἱ δὲ αὐτῶν, ὥσπερ φιλοτιμού-
μενοι τοῖς χείροσιν, ἐτόλμησαν τοὺς παρ' αὐτῶν ἄρχοντας ἢ καὶ τοὺς 
τούτων παῖδας εἰς θεοὺς ἀναθεῖναι, ἢ διὰ τιμὴν τῶν ἀρξάντων ἢ διὰ 
φόβον τῆς αὐτῶν τυραννίδος· ὡς ὁ ἐν Κρήτῃ παρ' αὐτοῖς περιβόητος 
Ζεύς, καὶ ἐν Ἀρκαδίᾳ Ἑρμῆς καὶ παρὰ μὲν Ἰνδοῖς Διόνυσος, παρὰ 
δὲ Αἰγυπτίοις Ἶσις, καὶ Ὄσιρις, καὶ Ὧρος, καὶ ὁ νῦν Ἀδριανοῦ τοῦ 
Ῥωμαίων βασιλέως παιδικὸς Ἀντίνοος, ὃν καίπερ εἰδότες ἄνθρωπον, 
καὶ ἄνθρωπον οὐ σεμνόν, ἀλλ' ἀσελγείας ἔμπλεων, διὰ φόβον τοῦ 
προστάξαντος σέβουσιν. 



Athanasius Theol., Contra gentes 
Section 23, line 14

        ἀλλὰ καὶ Πελασγοὶ μὲν τοὺς ἐν Θρᾴκῃ θεοὺς διαβάλλουσι· 
Θρᾷκες δὲ τοὺς παρὰ Θηβαίοις οὐ γινώσκουσιν. Ἰνδοὶ δὲ κατὰ 
Ἀράβων, καὶ Ἄραβες κατ' Αἰθιόπων, καὶ Αἰθίοπες κατ' αὐτῶν ἐν 
τοῖς εἰδώλοις διαφέρονται. 



Athanasius Theol., Contra gentes 
Section 24, line 11

                                          Λίβυες πρόβατον, ὃ καλοῦσιν 
Ἄμμωνα, θεὸν ἔχουσι· καὶ τοῦτο πολλοῖς παρ' ἑτέρων εἰς θυσίαν   
σφάζεται. Ἰνδοὶ τὸν Διόνυσον θρησκεύουσι, συμβολικῶς οἶνον 
αὐτὸν ὀνομάζοντες· καὶ τοῦτον τοῖς ἄλλοις σπένδουσιν ἕτεροι. 


\subsection{De incarnatione verbi}
Athanasius Theol., De incarnatione verbi (2035: 002)
“Sur l'incarnation du verbe”, Ed. Kannengiesser, C.
Paris: Cerf, 1973; Sources chrétiennes 199.
Chapter 28, section 3, line 4

   Καὶ ὥσπερ τοῦ πυρὸς ἔχοντος 
κατὰ φύσιν τὸ καίειν, εἰ λέγοι τις εἶναί τι τὸ μὴ δειλιῶν 
αὐτοῦ τὴν καῦσιν, ἀλλὰ καὶ μᾶλλον ἀσθενὲς αὐτὸ δεικνύον, 
οἷον δὴ λέγεται τὸ παρὰ Ἰνδοῖς ἀμίαντον· εἶτα ὁ τῷ 
λεγομένῳ μὴ πιστεύων εἰ πεῖραν θελήσειε λαβεῖν τοῦ 
λεγομένου, πάντως τὸ ἄκαυστον ἐνδυσάμενος καὶ προσβαλὼν 
πυρί, πιστοῦται λοιπὸν τὴν κατὰ τοῦ πυρὸς ἀσθένειαν· 
         ἢ 
ὡς εἴ τις τὸν τύραννον δεδεμένον ἰδεῖν θελήσειε, πάντως 
εἰς τὴν τοῦ νικήσαντος χώραν καὶ ἀρχὴν παρελθών, ὁρᾷ τὸν 
ἄλλοις φοβερὸν ἀσθενῆ γενόμενον· οὕτως εἴ τίς ἐστιν 
ἄπιστος, καὶ ἀκμὴν μετὰ τοσαῦτα, καὶ μετὰ τοὺς τοσούτους 




Athanasius Theol., De incarnatione verbi 
Chapter 47, section 4, line 4

                  Ὅτι πρὶν μὲν ἐπιδημῆσαι τὸν Λόγον, 
ἴσχυε καὶ ἐνήργει παρ' Αἰγυπτίοις καὶ Χαλδαίοις καὶ 
Ἰνδοῖς αὕτη καὶ ἐξέπληττε τοὺς ὁρῶντας· τῇ δὲ παρουσίᾳ 
τῆς ἀληθείας καὶ τῇ ἐπιφανείᾳ τοῦ Λόγου διηλέγχθη 
καὶ αὕτη, καὶ κατηργήθη παντελῶς. 



Athanasius Theol., De incarnatione verbi 
Chapter 50, section 1, line 3

Πολλοὶ πρὸ τούτου γεγόνασι βασιλεῖς καὶ 
τύραννοι γῆς, πολλοὶ παρὰ Χαλδαίοις ἱστοροῦνται καὶ 
παρ' Αἰγυπτίοις καὶ Ἰνδοῖς γενόμενοι σοφοὶ καὶ μάγοι· 
τίς τούτων ποτέ, οὐ λέγω μετὰ θάνατον, ἀλλὰ καὶ ἔτι 
ζῶν ἠδυνήθη τοσοῦτον ἰσχῦσαι, ὥστε τὴν σύμπασαν 
αὐτὸν γῆν πληρῶσαι τῆς αὐτοῦ διδασκαλίας, καὶ τοσοῦτον 
πλῆθος παιδεῦσαι ἀπὸ τῆς τῶν εἰδώλων δεισιδαιμονίας, 
ὅσους ὁ ἡμέτερος Σωτὴρ εἰς ἑαυτὸν ἀπὸ τῶν εἰδώλων μετή-
νεγκεν; 



Athanasius Theol., De synodis Arimini in Italia et Seleuciae in Isauria (2035: 010)
“Athanasius Werke, vol. 2.1”, Ed. Opitz, H.G.
Berlin: De Gruyter, 1940.
Chapter 25, section 1, line 2

Ἦσαν δὲ οἱ συνελθόντες ἐν τοῖς Ἐγκαινίοις ἐπίσκοποι ἐνενήκοντα, ὑπατείᾳ Μαρ-
κελλίνου καὶ Προβίνου, ἰνδικτιῶνος ιδʹ, ἐκεῖ ὄντος Κωνσταντίου τοῦ ἀσεβεστάτου. 

\subsection{Epistula ad Afros episcopos}

Athanasius Theol., Epistula ad Afros episcopos (2035: 049); MPG 26.
Volume 26, page 1032, line 22

                   Ταύτην ἔγνωσαν καὶ Ἰνδοὶ, καὶ 
ὅσοι παρὰ τοῖς ἄλλοις βαρβάροις εἰσὶ Χριστιανοί. 

\subsection{Expositiones in Psalmos}
Athanasius Theol., Expositiones in Psalmos (2035: 061); MPG 27.
Volume 27, page 336, line 58

                                                   Λαοὺς 
δὲ ἐνταῦθα τὰς τῶν παρὰ Ἰνδοῖς ὀρνέων ἀγέλας 
φησίν· αἳ καὶ τῶν Αἰγυπτίων κατήσθιον τὰ σκηνώ-
ματα ἐκριφέντα ἀπὸ τῆς θαλάσσης ἐν τῇ γῇ. 


Athanasius Theol., Expositiones in Psalmos 
Volume 27, page 360, line 55

Ἐξέτεινε τὰ κλήματα αὐτῆς ἕως θαλάσσης. 
Κλήματα καὶ παραφυάδες τῆς ἀμπέλου τὸ πᾶν πλῆ-
θος τοῦ λαοῦ φησιν, ὃ κατέσχεν ἀπὸ ποταμοῦ Εὐ-
φράτου μέχρι θαλάσσης τῆς Ἰνδικῆς. 


\subsection{Quaestiones ad Antiochum ducem}
Athanasius Theol., Quaestiones ad Antiochum ducem [Sp.] (2035: 077); MPG 28.
Volume 28, page 628, line 36

                                                   Ὅθεν 
φασὶν ἱστορικοί τινες ἀκριβεῖς παῖδες, ὅτι τού-
του χάριν πάντα τὰ εὐώδη τῶν ἀρωμάτων περὶ τὰ 
ἀνατολικώτερα, ἤγουν τὰ Ἰνδικὰ μέρη, ὑπάρχουσιν 
ὡς πλησιόχωρα τυγχάνοντα τοῦ παραδείσου. 


Athanasius Theol., Quaestiones ad Antiochum ducem [Sp.] 
Volume 28, page 633, line 38

          Ἡ Θαρσὶς πόλις ἐστὶ χώρας τῆς Ἰνδικῆς, 
ὡς ἐν τῇ τρίτῃ τῶν Βασιλειῶν μανθάνομεν. 


Athanasius Theol., Quaestiones ad Antiochum ducem [Sp.] 
Volume 28, page 660, line 18

                           Οἷόν τι πολλάκις ὡς πνεύ-
ματα ὁρῶσι τοὺς ὄμβρους τοὺς Ἰνδικοὺς πολλοὺς 
γινομένους, καὶ προλαμβάνουσιν ἐν Αἰγύπτῳ, καὶ διὰ 
φαρμακειῶν ἢ ὀνειράτων, μεγάλην ἀνάβασιν τοῦ 
Νείλου ποταμοῦ μαντεύονται· καὶ ἄλλα δέ τινα 
ὅμοια τούτοις διαπράττονται. 

Athanasius Theol., Quaestiones ad Antiochum ducem [Sp.] 
Volume 28, page 676, line 34

          Ὡς ἡ τῶν πραγμάτων φύσις μαρτυρεῖ καὶ 
διδάσκει, θεωροῦμεν, ὅτι τὸ μὲν θερμὸν στοιχεῖον ἐκ 
τῆς ἀνατολῆς τίκτεται· καὶ μαρτυροῦσι τὰ πάσης 
τῆς γῆς ἀνατολικώτερα Ἰνδικὰ σώματα ὑπὸ τῆς 
ἄκρας θέρμης μέλανα γινόμενα. 


\subsection{Quaestiones in scripturam sacram}
Athanasius Theol., Quaestiones in scripturam sacram [Sp.] (2035: 080); MPG 28.
Volume 28, page 732, line 6

          Μία ἐστὶ τῶν ὑδάτων ἡ φύσις καὶ συναγωγή· 
πληθυντικῶς δὲ τὰς συναγωγὰς ὠνόμασεν, ἐπειδὴ 
ἄλλο μὲν τὸ Ἰνδικὸν πέλαγος, ἄλλο δὲ τὸ Ποντικὸν, 
καὶ τὸ Ἀτλαντικὸν ἕτερον, καὶ ἄλλο μὲν ἡ Προποντὶς, 
καὶ Αἰγαῖον ἕτερον, καὶ ἄλλος πάλιν ὁ Ἰώνιος κόλ-
πος. 

\end{greek}


\section{Aelius Aristides}
\blockquote[From Wikipedia\footnote{\url{http://en.wikipedia.org/wiki/Aelius_Aristides}}]{Aelius Aristides (AD 117 - 181) was a popular Greek orator, who lived during the Roman Empire. He is considered to be a prime example of the Second Sophistic, a group of showpiece orators who flourished from the reign of Nero until ca. 230 AD. His surname was Theodorus. He showed extraordinary talents even in his early youth, and devoted himself with remarkable zeal to the study of rhetoric, which appeared to him the worthiest occupation of a man, and along with it he cultivated poetry as an amusement. Besides the rhetorician Herodes Atticus, whom he heard at Athens, he also received instructions from Aristocles at Pergamum, from Polemon at Smyrna, and from the grammarian Alexander of Cotiaeum.[1]}

\begin{greek}

Aelius Aristides Rhet., Διόνυσος (0284: 004)
“Aristides, vol. 1”, Ed. Dindorf, W.
Leipzig: Reimer, 1829, Repr. 1964.
Jebb page 30, line 6

καὶ οὐ πολὺ τοὐμμέσῳ τοῦ τε ἀγῶνος καὶ τῶν ἐπινικίων. 
Ἰνδοὺς δὲ καὶ Τυρρηνοὺς λέγουσιν ὡς κατεστρέψατο, αἰ-
νιττόμενοι δοκεῖν ἐμοὶ διὰ μὲν τῶν Τυρρηνῶν τὰ πρὸς 
ἑσπέραν, διὰ δὲ τῶν ἑτέρων τὸν πρὸς ἕω τόπον τῆς γῆς 
ὡς ἁπάσης αὐτὸν ἄρχοντα. 



Aelius Aristides Rhet., Παναθηναϊκός (0284: 013)
“Aristides, vol. 1”, Ed. Dindorf, W.
Leipzig: Reimer, 1829, Repr. 1964.
Jebb page 102, line 9

                                                     ἄλλας μὲν 
γὰρ χώρας ἐλέφαντες καὶ λέοντες κοσμοῦσι, τὰς δὲ ἵπποι 
κύνες, τὰς δὲ ἃ τοὺς παῖδας ἀκούοντας ἐκπλήττει· τὴν δὲ 
ὑμετέραν χώραν κοσμεῖ τῶν ἐπὶ γῆς τὸ κάλλιστον, οὐ κατὰ 
τοὺς ἐν Ἰνδοῖς μύρμηκας ὑποπτέρους ἄξιον εἰπεῖν. 



Aelius Aristides Rhet., Παναθηναϊκός 
Jebb page 121, line 7

                                     καὶ πρῶτοι δὴ τῶν εἰς 
ἐκεῖνον τὸν χρόνον Ἑλλήνων ἀναβάντες εἰς Σάρδεις στρα-
τιᾷ κοινῇ πορθήσαντες ᾤχοντο· τέως δὲ ἐν Βαβυλῶνος 
τάξει καὶ τῶν ἐν Ἰνδοῖς Ἕλληνες Σάρδεις ἐθαύμαζον· λα-  
βὼν δὲ ταύτην πρόφασιν Δαρεῖος ἡσυχίαν ἄγειν οὐκ ἠδύ-
νατο, ἀλλ' ἐξήταζε τὴν ἀρχὴν καὶ τὰς δυνάμεις συνεκρό-
τει, καὶ πᾶν μικρὸν ἦν αὐτῷ. 



Aelius Aristides Rhet., Ῥώμης ἐγκώμιον (0284: 014)
“Aristides, vol. 1”, Ed. Dindorf, W.
Leipzig: Reimer, 1829, Repr. 1964.
Jebb page 200, line 30

                                          φόρτους μὲν ἀπ' Ἰν-
δῶν, εἰ βούλει δὲ, καὶ τῶν εὐδαιμόνων Ἀράβων, τοσού-
τους ὁρᾶν ἔξεστιν ὥστε εἰκάζειν γυμνὰ τὸ λοιπὸν τοῖς ἐκεῖ 
λελεῖφθαι τὰ δένδρα καὶ δεῦρο δεῖν ἐκείνους ἐλθεῖν, ἐάν 
του δέωνται, τῶν σφετέρων μεταιτήσοντας· ἐσθῆτας δὲ 
αὖ Βαβυλωνίους καὶ τοὺς ἐκ τῆς ἐπέκεινα βαρβάρου 
κόσμους πολὺ πλείους τε καὶ ῥᾷον εἰσαφικνουμένους ἢ εἰ 
ἐκ Νάξου ἢ Κύθνου Ἀθήναζε ἔδει κατᾶραι τῶν ἐκεῖ τι 
φέροντας. 



Aelius Aristides Rhet., Πανηγυρικὸς ἐν Κυζίκῳ περὶ τοῦ ναοῦ (0284: 016)
“Aristides, vol. 1”, Ed. Dindorf, W.
Leipzig: Reimer, 1829, Repr. 1964.
Jebb page 242, line 29

τοῦτο ὑπὲρ πᾶσαν μὲν Καυκάσου περίοδον, τοῦτο δ' ὑπὲρ 
Ἰνδοὺς ἁλισκομένους καὶ Κασπίας πύλας, τοῦτο ὑπὲρ 
πάντα τἀνθρώπεια, τοῦτο τοῖς καλλίστοις ἐστὶ νικώντων 
καὶ ἃ προσήκει τῇ φύσει. 



Aelius Aristides Rhet., Ἱεροὶ λόγοι γʹ (0284: 025)
“Aristides, vol. 1”, Ed. Dindorf, W.
Leipzig: Reimer, 1829, Repr. 1964.
Jebb page 317, line 18

                                πρῶτον μὲν ἦν λέοντος ὀδόντα 
καῦσαι καὶ κόψαντα χρῆσθαι σμήγματι, δεύτερον δὲ ὀπῷ 
διακλύζειν τούτῳ δὴ τῷ χρήματι· μετὰ ταῦτα πέπερι, καὶ 
προσέθηκε θέρμης οὕνεκα· ἐφ' ἅπασι δὲ στάχυς Ἰνδικὸς, 
σμῆγμα καὶ οὗτος. 



Aelius Aristides Rhet., Ἱεροὶ λόγοι δʹ (0284: 026)
“Aristides, vol. 1”, Ed. Dindorf, W.
Leipzig: Reimer, 1829, Repr. 1964.
Jebb page 325, line 6

      οὗτος καὶ τὸ πρόβλημα ἦν ὁ προβαλών· καὶ ἦν γε 
τὸ πρόβλημα τοιόνδε· μέμνημαι γὰρ, ἅτε καὶ πρῶτον λα-
βών· Ἀλεξάνδρου, φησὶν, ἐν Ἰνδοῖς ὄντος συμβουλεύει 
Δημοσθένης ἐπιθέσθαι τοῖς πράγμασιν. 



Aelius Aristides Rhet., Πρὸς Πλάτωνα περὶ ῥητορικῆς (0284: 045)
“Aristides, vol. 2”, Ed. Dindorf, W.
Leipzig: Reimer, 1829, Repr. 1964.
Jebb page 107, line 9

                            ὥσπερ γὰρ οἱ λέοντες καὶ ὅσα 
ἐντιμότερα τῶν ζωῶν σπανιώτερα τῶν ἄλλων ἐστὶ τῇ φύ-
σει, οὕτω καὶ κατ' ἀνθρώπους οὐδὲν οὕτω σπάνιον ὅσον 
ἄξιον προσειπεῖν ῥήτορα· εἷς δὲ ἀγαπητῶς καὶ δεύτερος 
ὥσπερ ὁ Ἰνδικὸς ὄρνις ἐν Αἰγυπτίοις ἡλίου περιόδοις φύε-
ται. 



Aelius Aristides Rhet., Αἰγύπτιος (0284: 048)
“Aristides, vol. 2”, Ed. Dindorf, W.
Leipzig: Reimer, 1829, Repr. 1964.
Jebb page 361, line 25

                       τὸ ἐν Συήνῃ μὲν καὶ Ἐλεφαντίνῃ 
ὀκτὼ καὶ εἴκοσιν αἴρεσθαι πήχεις, περὶ δ' αὖ τὸ Ἰνδικὸν 
καὶ Ἀράβιον ἐμπόριον τὴν Κόπτον ἕνα καὶ εἴκοσι, καὶ 
πάλιν τούτων ἀφαιρεῖν ἑπτὰ καὶ τέτταρας καὶ δέκα ἄγειν 
τοὺς κατὰ Μέμφιν γνωρίμους καὶ πρὸς οὓς Ἕλληνες ἤδη 
λογίζονται, κάτω δ' ἐν τοῖς ἕλεσιν εἰς ἑπτὰ καταβαίνειν, 
εἶτα δύ' ἤκουον. 

\end{greek}


\section{Ptolemaeus VIII Euergetes II <Hist.>}
\blockquote[(Same guy? )From Wikipedia\footnote{\url{http://en.wikipedia.org/wiki/Ptolemy_VIII_Physcon}}]{Ptolemy VIII Euergetes II[note 1] (Greek: Πτολεμαῖος Εὐεργέτης, Ptolemaĩos Euergétēs) (c. 182 BC – June 26, 116 BC), nicknamed Φύσκων, Physcon, was a king of the Ptolemaic dynasty in Egypt.}
\begin{greek}

Ptolemaeus VIII Euergetes II <Hist.>, Fragmenta (1645: 003)
“FHG 3”, Ed. Müller, K.
Paris: Didot, 1841–1870.
Fragment 1, line 5

E LIBRIS PRIMO ET QUINTO.


 Athenaeus X: Πολυπότης δὲ ἦν καὶ Ἀν-
τίοχος ὁ βασιλεὺς, ὁ κληθεὶς Ἐπιφανὴς, ὁ ὁμηρεύσας 
παρὰ Ῥωμαίοις, ὃν ἱστορεῖ Πτολεμαῖος ὁ Εὐεργέτης 
ἐν τῷ πρώτῳ τῶν Ὑπομνημάτων κἀν τῷ πέμπτῳ, φά-
σκων αὐτὸν εἰς τοὺς Ἰνδικοὺς κώμους καὶ μέθας τρα-
πέντα πολλὰ ἀναλίσκειν. 

\end{greek}


\end{document}

