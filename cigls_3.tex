\documentclass[12pt,letterpaper,twoside,final]{memoir}
%M-x shell
%latexmk -pdf -e '$pdflatex=q/xelatex %O %S/' *.tex
%latexmk -pdf -pvc -e '$pdflatex=q/xelatex %O %S/' *.tex
\usepackage[no-math]{fontspec}
\usepackage{xltxtra}
\defaultfontfeatures{Scale=MatchLowercase,Mapping=tex-text}
\setmainfont[Mapping=tex-text]{Junicode} %CMU Serif
\setsansfont[Mapping=tex-text]{Junicode} %CMU Sans Serif
\newfontfamily\greekfont{Linux Libertine} %CMU Serif,Linux Libertine O,Junicode
%\setmainfont[Mapping=tex-text]{CMU Serif}
\usepackage{xkeyval}
\usepackage{polyglossia}
\setdefaultlanguage[variant=american]{english}
\setotherlanguage[variant=ancient,numerals=arabic]{greek}
\setotherlanguage[spelling=new]{german}
\setotherlanguages{latin,french,italian,spanish,sanskrit}

\usepackage{xeindex}
\makeindex
\IndexList{mylist}{gold,silver,*Ethiopia=Ethiopia,*Ethiopian?=Ethiopians,*India=India,*Indian?=Indians,ψῆγμα=ψῆγμα,ψήγματος=ψήγμα}

\usepackage[babel=once,english=american,autostyle=tryonce,strict=true]{csquotes}
\usepackage[backend=biber,style=authoryear,sorting=debug,bibstyle=authoryear,citestyle=authoryear,useprefix=false,firstinits=false,url=false,usetranslator=true]{biblatex}%was: firstinits=true
%\DeclareAutoCiteCommand{plain}{\cite}{\cites}
\DeclareAutoCiteCommand{plain}{\textcite}{\textcites}
\DeclareAutoCiteCommand{inline}{\textcite}{\textcites}
%\DeclareAutoCiteCommand{footnote}[l]{\footcite}{\footcites}
%\DeclareAutoCiteCommand{footnote}[f]{\smartcite}{\smartcites}
\bibliography{classics_bib} %also avail india_bib
\usepackage[final]{hyperref}%?%hyperfootnotes=false
\hypersetup{bookmarks=false,        % show bookmarks bar?
    unicode=true,           % non-Latin characters in Acrobat’s bookmarks
    pdftoolbar=true,        % show Acrobat’s toolbar?
    pdfmenubar=true,        % show Acrobat’s menu?
    pdffitwindow=false,     % window fit to page when opened
    pdfstartview={FitH},    % fits the width of the page to the window
    pdftitle={Ethics of Leadership: Organization and Decision--Making in Caesar's \emph{Bellum~Gallicum}},
    pdfauthor={Kyle P. Johnson},     % author
    pdfsubject={Dissertation on organization and decision--making in Julius Caesar's Bellum Gallicum},   % subject of the document
    pdfcreator={Kyle P. Johnson},   % creator of the document
    pdfproducer={Kyle P. Johnson}, % producer of the document
    pdfkeywords={Julius Caesar, Bellum Gallicum, Gallic War, communication, deliberation, decision--making, leadership, organization, Xenophon, Anabasis}, % list of keywords
    pdfnewwindow=true,      % links in new window
    colorlinks=true,       % false: boxed links; true: colored links
    linkcolor=blue,          % color of internal links
    citecolor=blue,        % color of links to bibliography
    filecolor=blue,      % color of file links
    urlcolor=blue           % color of external links
}
\usepackage{microtype}
\usepackage{xecolor}
\definergbcolor{blue}{0000FF}
\definergbcolor{red}{FF0000} %\textxecolor{colorname}{text}
\XeTeXdashbreakstate=1
\usepackage{minitoc}
\usepackage{indentfirst}
\usepackage{outline}
\usepackage{verbatim}
\usepackage{enumerate}
%\usepackage{tikz}
%\usetikzlibrary{shapes,backgrounds}
\usepackage{longtable}
%\usepackage{lscape}
%\usepackage{verse}
%\usepackage{rotating}
\usepackage{ccicons}
\usepackage{bookmark}
\usepackage{eledmac}
\usepackage{eledpar}
\usepackage{etoolbox}
%!%%%%%%%%%%%%%%%%%%%%%for non-italicized headings%%%%%
% http://tex.stackexchange.com/questions/32655/remove-italic-from-memoir-headings-pagestyle
\makeevenhead{headings}{\leftmark}{}{}
\makeoddhead{headings}{\rightmark}{}{}
\makeevenfoot{headings}{}{\thepage}{}
\makeoddfoot{headings}{}{\thepage}{}
%!%%%%%%%%%%%%%%%%%%%%%%%%%%%%%%%%%%%%%%%%%%%%%%%%%%%%%%
%!%%%%%%%%%%%%%%%%%%%%%for margins, l=1.5; rest=1.0, maybe add 0.1in%%%%%
\setstocksize{11in}{8.5in}
\settrimmedsize{11.0in}{8.5in}{*} %\settrimmedsize{ height }{ width }{ ratio }
\settypeblocksize{7.75in}{5.8in}{*} %\settypeblocksize{ height }{ width }{ ratio } %note: 7.25h gives margins of 1.5 on top/bottom w/ ubratio of 1
\setlrmargins{1.5in}{*}{*} %\setlrmargins{ spine }{ edge }{ ratio } %spine = left, edge = right; only answer one or two of these values
%\setlrmarginsandblock{1.5in}{1.0in}{*} %\setlrmarginsandblock{ spine }{ edge }{ ratio } %
%\setulmargins{*}{*}{*} %\setulmargins{ upper }{ lower }{ ratio }
%\setulmarginsandblock{0.5in}{*}{*} %\setulmarginsandblock{ upper }{ lower }{ ratio }
%\setheadfoot{*}{*} %\setheadfoot{ headheight }{ footskip }
%\setheaderspaces{1.0in}{*}{*} %\setheaderspaces{ headdrop }{ headsep }{ ratio }
%\checkandfixthelayout[lines]
\flushbottom
%!%%%%%%%%%%%%%%%%%%%%%%%%%%%%%%%%%%%%%%%%%%%%%%%%%%%%%%
\apptocmd{\sloppy}{\hbadness 10000\relax}{}{}
%\renewcommand{\@pnumwidth}{3em} %taken from memman p. 153
%\renewcommand{\@tocrmarg}{4em} %taken from memman p. 153
\setsecnumdepth{subparagraph}
\makeatletter
\renewcommand\@makefntext{\hspace*{2em}\@thefnmark. }
\newenvironment*{singlespcquote}
        {\quote\SingleSpacing}
        {\endquote}
\SetBlockThreshold{0}
\SetBlockEnvironment{singlespcquote}
\SetCiteCommand{\parencite} %default is \cite
  %csquote + biblatex; see csquotes.pdf section 5 + 8.6
  %\textcquote[prenote][postnote]{key}[punct]{text}[tpunct]
  %\blockcquote[prenote][postnote]{key}[punct]{text}[tpunct]
  %usually:
  %\textcquote[page#]{key}{quote}
  %\blockcquote[page#]{key}[.]{quote}
  %\textcites(pre)(post)[pre][post]{key}...[pre][post]{key}
  %example: \textcites(and chapter 3)[35]{riggsby2006}[78]{hammond1996}[23]{levene2010}
\title{Eurasian Dharma Corpus}
\author{Kyle~P.~Johnson}
%\date{January 31, 2008}
\begin{document}
%\fussy
%\hyphenpenalty=5000   %1000 default=?
%\tolerance=1000        %1000 %200= default
%\setlength{\emergencystretch}{3em}
%\midsloppy
%\fussy
\sloppy
\vbadness=10000 % badness above which bad vboxes are shown. (Default = 10000?)
\frontmatter
\hyphenation{}

\SingleSpacing

%%%%%%%%%%for minitoc%%%%%%%%%%%%%%%%%%%%%
\dominitoc
\dominilof
\dominilot
%%%%%%%%%%%%%%%%%%%%%%%%%%%%%%%%%%%%%%%%%%%
\tableofcontents

\mainmatter


\chapter{Unsorted entries}

\begin{greek}

Vitae Dionysii Periegetae, Vita Dionysii (4173: 001)
“”Antimachos in der Vita Chisiana des Dionysios Periegetes””, Ed. Kassel, R.
Basel: Seminar für klassische Philologie, 1985; Catalepton. Festschrift für Bernhard Wyss.
Line 38

                 Ἀλέξανδρος μὲν οὖν ἐπὶ Σκύθας ἰὼν καὶ τοὺς Ἰνδοὺς 
ἔγνω τὸν ἑῶιον Ὠκεανόν, Ἀντίοχοί τε καὶ Πτολομαῖοι περιιόντες 
οὐκ ὀλίγην ἱστόρησαν γῆν. 



Apomasar Astrol., De revolutionibus nativitatum (4361: 008)
“Albumasaris de revolutionibus nativitatum”, Ed. Pingree, D.
Leipzig: Teubner, 1968.
Page 2, line 12

  καὶ τοῦ ἐπιμερίζοντος καὶ τοῦ κοινωνοῦντος αὐτῷ 
 βʹ Περὶ ὧν σημαίνουσιν οἵ τε ἀγαθοποιοὶ ἐπιμερίζοντες 
  καὶ οἱ κακοποιοὶ καὶ οἱ κοινωνοῦντες κατά τε ὅριον 
  καὶ κατὰ τὰς ἀκτῖνας 
 γʹ Καὶ περὶ τῆς παραδόσεως τοῦ ἑνὸς πρὸς τὸν ἕτερον 
 δʹ Περὶ τῆς σημασίας τῶν ἀστέρων ὅτε ὁ μὲν εἷς αὐτῶν 
  ἐπιμερίζει, ὁ δὲ κοινωνεῖ αὐτῷ 
 εʹ Περὶ τῆς σημασίας τοῦ χρονοκράτορος καὶ τοῦ 
  ἐπιμερίζοντος <καὶ> τοῦ ὡροσκόπου 
 ϛʹ Περὶ ὧν εἶπον οἱ Ἰνδοὶ περὶ τοῦ νουπεῦχρες καὶ περὶ 
  ὧν σημαίνει ἀγαθῶν καὶ κακῶν 
Τμῆμα τέταρτον περιέχον λόγους <ζ> 
 αʹ Περὶ τῆς περιόδου τῶν <ζ> ἀστέρων τῆς λεγομένης 
  φαρταρίας 
 βʹ Καὶ περὶ τοῦ Ἀναβιβάζοντος καὶ Καταβιβάζοντος 
Τμῆμα πέμπτον περιέχον λόγους <η> 
 αʹ Περὶ τῆς ἀποκατα<στά>σεως τῶν ἀστέρων ἐν ταῖς 
  τῶν χρόνων ἐναλλαγαῖς εἰς τοὺς ἰδίους τόπους καὶ 
  τῆς εἰς ἀλλήλους αὐτῶν ἐπεμβάσεως 




Apomasar Astrol., De revolutionibus nativitatum 
Page 12, line 8

              Περὶ ὧν συμβάλλεται πρὸς διά-
γνωσιν τῶν διαθέσεων ἡ τῶν ἐτῶν ἐναλλαγὴ 
ἤτοι περὶ τῆς ὠφελείας τῆς ἀπὸ τῶν ἐναλλαγῶν 
τῶν ἐτῶν


 Ἡ μὲν ἀρετὴ τῆς διαγνώσεως κατὰ τοὺς ἀνθρώπους 
πραγμάτων καὶ διαθέσεων ἀπὸ τῆς τῶν ἐτῶν ἐναλλαγῆς 
πρόδηλος· ἅπαντα γὰρ τὰ ἔθνη, οἵ τε Βαβυλώνιοι καὶ Πέρσαι 
καὶ Ἰνδοὶ καὶ Αἰγύπτιοι, καὶ οἱ τούτων βασιλεῖς καὶ ἰδιῶται 
οὐ πρότερον ἐπεχείρουν τινὶ πράγματι ἔν τινι <ἔτει> πρὶν 
ἢ ἰδεῖν τὴν ἐναλλαγὴν τοῦ χρόνου τοῦ γενεθλίου αὐτῶν. 



Apomasar Astrol., De revolutionibus nativitatum 
Page 170, line 12t

              Περὶ τῶν λεγομένων νουπαχρατῶν 
καὶ τῶν κυρίων αὐτῶν καὶ τοῦ περιπάτου αὐτῶν 
καὶ τῆς τρίτης διαιρέσεως αὐτῶν, καὶ περὶ δια-
γνώσεως παντὸς ἐπιμερίζοντος καὶ τῆς σημασίας 
αὐτοῦ καὶ τῆς σημασίας τοῦ συνεπιμερίζοντος 
αὐτῷ ἢ κατὰ σῶμα ἢ κατὰ ἀκτῖνα καθὼς οἱ 
Ἰνδοὶ ἐξέθεντο. 



Apomasar Astrol., De revolutionibus nativitatum 
Page 170, line 15

Τῷ μὲν ἐπιμερισμῷ ᾧ προείπομεν ἔναγχος 
χρῶνται οἵ τε Βαβυλώνιοι καὶ οἱ Πέρσαι καὶ οἱ Αἰγύπτιοι· 
οἱ μέντοι Ἰνδοὶ καὶ οἱ γειτνιάζοντες αὐτοῖς, ἰδόντες ἐν ἑνὶ 
ἐπιμερισμῷ διάφορα πάθη τοῖς ἀνθρώποις ἐπισυμβαίνον-
τα, οὐκ ἐποίησαν τὸν ἐπιμερισμὸν κατὰ τὰ ὅρια ὡς οἱ λοι-
ποί, ἀλλ' ἐποίησαν τοῦτον κατὰ τὰ νουπάχρατες ὡς ἂν 
ἀκριβέστερον προβαίνωσι τὰ ἀποτελέσματα. 



Apomasar Astrol., De revolutionibus nativitatum 
Page 171, line 3

Τὸ δὲ τοιοῦτον ὄνομα τῇ Ἰνδῶν διαλέκτῳ ἑρμηνεύεται 
τὸ ἔννατον, καὶ ἔστι διακοσίων λεπτῶν ἤτοι τριῶν μοιρῶν 
καὶ τρίτου μοίρας ὀρθῆς· ἔσται οὖν ἐν ἑκάστῳ ζῳδίῳ 
ἐννέα ἔννατα, ὧν ἕκαστον ἔχει ἴδιον ἐπικρατήτορα. 



Apomasar Astrol., De revolutionibus nativitatum 
Page 172, line 22

Τὸν δὲ περίπατον τῶν τοιούτων ἐννάτων ἔν τε τοῖς 
γενεθλίοις καὶ ἐν ἑτέροις καὶ τὴν διαίρεσιν αὐτῶν τὴν εἰς 
τὰ τρίτα ὡς ἐν μυστηρίῳ κατεῖχον οἱ Ἰνδοί, μὴ ἀπο-
καλύπτοντες τοῦτο εἰ μή γε μόνοις τοῖς ἐπ' ἄκρον ἐλη-
λακόσι τῆς ἐπιστήμης, ὁρκίζοντες αὐτοὺς πρότερον ἐν 
μυστηρίῳ ἔχειν τὴν τοιαύτην διδασκαλίαν καὶ μὴ ἀνα-
κοινοῦσθαι ταύτην τοῖς ἀγροικοτέροις, ἀλλὰ μόνοις τοῖς   
γινώσκουσι τὸ μέτρον τοῦ μαθήματος καὶ ὅσον ὑπερέχει 
ὁ ἐπιστήμων τῶν ἄλλων ἀνθρώπων· οἱ δὲ παραλαβόντες 
τὴν τοιαύτην μέθοδον μεγάλας ὡμολόγουν χάριτας τῷ 
διδάσκοντι. 



Apomasar Astrol., De revolutionibus nativitatum 
Page 178, line 20t

                    Περὶ διαγνώσεως τοῦ χρονο-
κράτορος ἐκ τῶν ἐννάτων κατὰ τὴν Ἰνδῶν δόξαν


 Ὅτε καταντήσει ὁ χρόνος εἴς τι ζῴδιον, λαμβάνουσιν οἱ 
Ἰνδοὶ χρονοκράτορα τὸν κύριον τοῦ πρώτου ἐννάτου, εἴτε   
εἰς τὴν ἀρχὴν τοῦ ζῳδίου κατήντησεν ἡ ἐναλλαγὴ εἴτε εἰς 
τὸ τέλος. 



Apomasar Astrol., De revolutionibus nativitatum 
Page 179, line 19

                                   εἰ δὲ καταντήσει τὸ ἔτος 
τυχὸν ἐπὶ τοὺς Διδύμους εἰς οἱονδήποτε τόπον αὐτῶν, 
ἔσται ἡ Ἀφροδίτη κυρία τοῦ ἔτους διὰ τὸ κυρίαν αὐτὴν 
εἶναι τοῦ πρώτου ἐννάτου τῶν Διδύμων· εἰ δὲ τὸ ἔτος 
καταντήσει ἐπὶ τὸν Καρκίνον, παραλαμβάνουσιν ὡσαύτως 
οἱ Ἰνδοὶ τὴν Σελήνην κυρίαν τοῦ ἔτους διὰ τὸ κυριεύειν 
αὐτὴν τοῦ πρώτου ἐννάτου τοῦ Καρκίνου. 



Apomasar Astrol., De revolutionibus nativitatum 
Page 180, line 10

                          περὶ δὲ τῶν κατὰ μῆνα καὶ τῶν καθ' 
ἡμέραν ἀποτελεσμάτων ἀπὸ τῶν ἐννάτων καθὼς ἐδό-
ξασαν οἱ Ἰνδοὶ διηγησόμεθα εἰς τὸ μετέπειτα, τοῦ Θεοῦ 
θέλοντος. 



Apomasar Astrol., De revolutionibus nativitatum 
Page 235, line 26

                           κατὰ τὸ μὲν [τὸ] ἀπὸ τοῦ ζῳδίου 
τοῦ καταντήματος ὀφείλεις ποιεῖν οὕτως· ἵνα ἐπιβλέπῃς 
πρὸς τὸν κύριον τοῦ κατα<ντήματος> ὅς ἐστιν κύριος τοῦ 
πρώτου ἐννατημορίου [τοῦ ἀπαντήματος] τοῦ ζῳδίου ὡς 
οἱ Ἰνδοὶ δοξάζουσι, καθὼς ἐδηλώσαμεν ἐν τῷ τρίτῳ 
τμήματι ταύτης τῆς βίβλου. 



Apomasar Astrol., De revolutionibus nativitatum 
Page 236, line 33

                                      ὁ αὐτὸς γοῦν ἀριθμὸς παρὰ τοῖς 
Ἰνδοῖς ἐστι ἑκάστου μηνός, διότι ποιοῦσι τὰς ἡμέρας τῶν 
ὅλων μηνῶν τῶν γενεθλιακῶν ἐτῶν ἴσας ὥστε μὴ εἶναι τὸν   
ἕνα μῆνα ἐλάττονα ἢ μείζονα τοῦ ἑτέρου. 




[Clitophon] Hist., Fragmenta (1281: 002)
“FHG 4”, Ed. Müller, K.
Paris: Didot, 1841–1870.
Fragment t1, line 1

ΙΝΔ*ιΚΑ. 




[Clitophon] Hist., Fragmenta 
Fragment 1, line 2

E LIBRO DECIMO.


 Plutarch. De fluv. 25, 3: Φύεται δὲ καὶ βοτάνη 
(sc. ἐν Ἰνδῷ), Καρπύκη καλουμένη, βουγλώσσῳ πα-
ρόμοιος· ποιεῖ δ' ἄριστα πρὸς ἰκτερικοὺς μεθ' ὕδατος 
χλιαροῦ διδομένη τοῖς πάσχουσι· καθὼς ἱστορεῖ Κλει-
τοφῶν ὁ Ῥόδιος ἐν ιʹ Ἰνδικῶν. 



Georgius Syncellus Chronogr., Ecloga chronographica (3045: 001)
“Georgius Syncellus. Ecloga chronographica”, Ed. Mosshammer, A.A.
Leipzig: Teubner, 1984.
Page 2, line 32

πρωτόκτιστος ἡμέρα, ἣν ὡς θεμέλιον ἀρραγῆ καὶ βάσιν ἄσειστον πηξάμενος 
τῆσδε τῆς συγγραφῆς, λιπαρῶ τὸν ἐν αὐτῇ καὶ κατ' αὐτὴν οὐ μόνον τὴν αἰ-
σθητὴν κτίσιν ὑποστησάμενον, ἀλλὰ καὶ τὴν ἐν αὐτῷ καινὴν κτίσιν Χριστὸν 
θεὸν ἡμῶν συνεργῆσαί μοι τῷ ἀμαθεστάτῳ, ὥστε σαφῶς ἀποδεῖξαι τῷ ͵εφʹ 
ἔτει τοῦ κόσμου τὴν ἔνσαρκον αὐτοῦ γεγενῆσθαι οἰκονομίαν, καὶ ὅσα ἐν τῷ 
μεταξὺ χρόνῳ γέγονεν ἐπίσημα πράγματα περί τε ἔθνη καὶ βασιλείας καὶ 
τῶν μετὰ ταῦτα ὀκτακοσίων δύο ἐτῶν, λγʹ μὲν ἐτῶν καὶ ἡμερῶν μʹ τῆς ἐπὶ 
γῆς οἰκονομίας, ἑπτακοσίων δὲ καὶ ξϛʹ καὶ μηνῶν ιʹ καὶ ἡμερῶν κʹ τῶν 
μετὰ τὴν ἁγίαν αὐτοῦ ἀνάληψιν, τοῦτ' ἔστιν ἀπὸ τῆς πρωτοκτίστου ἡμέρας 
ἕως τοῦ κοσμικοῦ καθολικοῦ ͵ϛτʹ ἔτους ἰνδικτίωνος αʹ, ὡς ὑποτέτακται. 



Georgius Syncellus Chronogr., Ecloga chronographica 
Page 6, line 12

μάτων ἀκολουθίας, διαφοράς φημι βασιλέων καὶ ἱερέων ἀριθμόν, προ-
φητῶν τε καὶ ἀποστόλων καὶ μαρτύρων καὶ διδασκάλων, καὶ τῶν παρ' Ἕλ-
λησιν ἢ ἄλλοις ἔθνεσιν ἐπὶ σοφίᾳ βεβοημένων ἢ ἄλλῃ τινὶ τέχνῃ ἢ ἀριστείᾳ 
πολεμικῇ καὶ ἐπὶ μοχθηρίᾳ προδήλῳ, πάντα ὡς οἷός τε ὢν ἀναλεξάμενος 
ἐκ τῶν προειρημένων ἱστορικῶν ἀνδρῶν· ἐπὶ πᾶσι τὴν κατὰ Χριστοῦ καὶ 
τοῦ γένους ἡμῶν θεοβδέλυκτον διαθήκην, ἣν διέθετο τὰ σκηνώματα τῶν 
Ἰδουμαίων καὶ οἱ Ἰσμαηλῖται διώκοντες τὸν κατὰ τὸ πνεῦμα λαόν, θείοις 
κρίμασι καὶ ἀποστασίαν μελετῶντες τὴν ἐπ' ἐσχάτων τῶν ἡμερῶν ὑπὸ 
τοῦ μακαρίου Παύλου προφητευθεῖσαν, διαγράψω κατὰ δύναμιν, ἕως τοῦ 
νῦν ἐνεστῶτος ͵ϛτʹ ἔτους ἀπὸ κτίσεως κόσμου ἰνδικτίωνος αʹ. 



Georgius Syncellus Chronogr., Ecloga chronographica 
Page 6, line 22

                                                              διὸ καλὸν ἡγούμενος 
τὰς ἐπισημοτέρας αὐτῶν διαφορὰς τρεῖς οὔσας καὶ ταῖς πατρικαῖς δι-
δασκαλίαις εὐδιαγνώστους εἶναι τοῖς φιλομαθέσιν ἐκθήσομαι τὰς λοιπὰς 
παρείς, ὡς ἀτρίπτους οἷον εἰπεῖν, Ἰνδῶν ἢ Χαλδαίων ἢ ἄλλων τινῶν 
ἀσυνήθων. 



Georgius Syncellus Chronogr., Ecloga chronographica 
Page 46, line 24

Σὴμ τῷ πρωτοτόκῳ αὐτοῦ υἱῷ κατάγοντι <ἔτος> υλαʹ ἔδωκεν ἀπὸ Περ-
σίδος καὶ Βάκτρων ἕως Ἰνδικῆς μῆκος, πλάτος δὲ ἀπὸ Ἰνδικῆς ἕως Ῥινο-  
κουρούρων τῆς Αἰγύπτου, ἤτοι τὰ ἀπὸ ἀνατολῆς ἕως μέρους τῆς μεσημ-
βρίας, τήν τε Συρίαν καὶ Μήδειαν καὶ ποταμὸν διορίζοντα αὐτοῦ τὰ ὅρια 
τὸν Εὐφράτην. 



Georgius Syncellus Chronogr., Ecloga chronographica 
Page 49, line 8

ιδʹ Ἐλμωδάμ, ἀφ' οὗ Ἰνδοί. 



Georgius Syncellus Chronogr., Ecloga chronographica 
Page 49, line 17

κγʹ Σαβάτ, ἀφ' οὗ Ἄραβες Ἰνδῶν. 



Georgius Syncellus Chronogr., Ecloga chronographica 
Page 49, line 21

Οὗτοι πάντες ἐκ τοῦ Σὴμ κατάγονται, ὧν ἡ κατοικία ἐστὶ κατὰ μῆκος 
μὲν ἀπὸ Βάκτρων καὶ Ἰνδικῆς ἕως Ῥινοκουρούρων τῆς ὁριζούσης Συρίαν   
καὶ Αἴγυπτον καὶ τὴν ἐρυθρὰν θάλασσαν ἀπὸ στόματος τοῦ κατὰ Ἀρσι-
νόην τῆς Ἰνδικῆς, κατὰ πλάτος δὲ ἀπὸ Περσίδος καὶ Βάκτρων ἕως Ἰν-
δικῆς. 



Georgius Syncellus Chronogr., Ecloga chronographica 
Page 50, line 9

                                      οἱ δὲ ἐπιστάμενοι αὐτῶν γράμματα 
Ἰουδαῖοι, Πέρσαι, Μῆδοι, Χαλδαῖοι, Ἰνδοί, Ἀσσύριοι. 



Georgius Syncellus Chronogr., Ecloga chronographica 
Page 52, line 20

                 αἱ δὲ χῶραι αὐτῶν Αἴγυπτος, Αἰθιοπία ἡ βλέπουσα κατὰ 
Ἰνδοὺς πρὸς εὐρόνοτον, ἄλλη Αἰθιοπία πρὸς νότον, ὄθεν ἐκπορεύεται ὁ 
Νεῖλος ποταμός, ἐρυθρὰ ἡ βλέπουσα κατὰ ἀνατολάς, Θηβαΐς, Λιβύη ἡ 
παρεκτείνουσα μέχρι Κορκυρίνης, Μαρμαρὶς καὶ τὰ περὶ αὐτὴν πάντα, 
Σύρτις ἔχουσα ἔθνη τρία, Νασαμῶνας, Μάκας, Ταυταμαίους. 



Georgius Syncellus Chronogr., Ecloga chronographica 
Page 56, line 2

          τοῦ δὲ Χὰμ πλεῖστα μέχρι καὶ νῦν ἔθνη ἐστὶν ἐν ἀποστασίᾳ κατά 
τε τὰς Ἰνδίας καὶ Αἰθιοπίας καὶ Μαυριτανίας, ἐν οἷς Ἀραβίαι καὶ Ἰδου-
μαῖαι κατὰ θεοῦ καὶ τῶν ἁγίων αὐτοῦ θρασύνονται τῇ τοῦ Χὰμ καὶ Χα-
ναὰν κατάρᾳ καθυποβεβλημένοι. 



Georgius Syncellus Chronogr., Ecloga chronographica 
Page 71, line 20

                                                  πρῶτος ἦρξε Νίνος ἁπάσης 
τῆς Ἀσίας, πλὴν Ἰνδῶν, ἔτεσι τριακοσίοις, οὐ πολὺ πρότερον Ὠγύγου. 



Georgius Syncellus Chronogr., Ecloga chronographica 
Page 177, line 11

                                                                         Αἰθίοπες ἀπὸ 
Ἰνδοῦ ποταμοῦ ἀναστάντες πρὸς τῇ Αἰγύπτῳ ᾤκησαν. 



Georgius Syncellus Chronogr., Ecloga chronographica 
Page 190, line 8

Διόνυσος ἐπὶ Ἰνδοὺς ἐστράτευσε καὶ Νῦσαν πόλιν ἔκτισε πρὸς τῷ Ἰνδῷ 
ποταμῷ. 



Georgius Syncellus Chronogr., Ecloga chronographica 
Page 190, line 16

Διονύσου πράξεις καὶ τὰ περὶ Ἰνδούς, Λυκοῦργόν τε καὶ Ἀκταίωνα καὶ 
Πενθέα, ὅπως τε Περσεῖ συστὰς εἰς μάχην ἀναιρεῖται, ὥς φησι Δείναρ-
χος ὁ ποιητής, οὐχ ὁ ῥήτωρ. 



Georgius Syncellus Chronogr., Ecloga chronographica 
Page 194, line 22

                                               καὶ ἐπάγει· αὕτη μὲν οὖν βασι-
λεύσασα τῆς Ἀσίας ἁπάσης πλὴν Ἰνδῶν ἐτελεύτησε τὸν προειρημένον 
τρόπον, βιώσασα μὲν ἔτη ξβʹ, βασιλεύσασα δὲ δύο πρὸς τοῖς μʹ. 



Georgius Syncellus Chronogr., Ecloga chronographica 
Page 195, line 29

       στρατείην τε αὐτῆς κατὰ τῶν Ἰνδῶν καὶ ἧτταν καὶ ὅτι τοὺς ἰδίους 
ἀνεῖλεν υἱοὺς καὶ ὑπὸ Νίνου τῶν παίδων ἑνὸς ἀνῃρέθη τοῦ διαδεξαμένου 
τὴν ἀρχήν. 



Georgius Syncellus Chronogr., Ecloga chronographica 
Page 259, line 22

καὶ ἀπὸ τῆς ἀρχῆς Ἰούδα ἀρχιερέως υἱοῦ Ματθίου ἕως λγʹ ἔτους Ἡρώδου, 
τεσσαρακοστοῦ δὲ τρίτου ἔτους Αὐγούστου Καίσαρος, ὅτε τὸ κατὰ σάρκα 
ἐκ τῆς ἁγίας παρθένου ὁ κύριος ἐτέχθη, ἔτη ρξζʹ, κόσμου δὲ ͵εφαʹ, μηνὶ 
Δεκεμβρίῳ ἰνδικτ. 



Georgius Syncellus Chronogr., Ecloga chronographica 
Page 264, line 1

Τὸν Ναβουχοδονόσωρ ὁ Μεγασθένης ἐν τῇ δʹ τῶν Ἰνδικῶν Ἡρακλέους 
ἀλκιμώτερον ἀποφαίνει, ὃς ἀνδρείᾳ μεγάλῃ Λιβύης τὸ πλεῖστον καὶ 
Ἰβηρίας κατεστρέψατο. 



Georgius Syncellus Chronogr., Ecloga chronographica 
Page 277, line 2

Ὁ αὐτὸς πάππος ἦν Κύρου, Δαρεῖος, Ἀσούηρος ὁ καὶ Ἀστυάγης. 
 Κατ' Ἰνδῶν καὶ ἄλλων ἐθνῶν ἑορτάζων ἐπινίκια Οὐασθὴν τὴν ἰδίαν 
γυναῖκα ὡς ὡραιοτάτην ὀφθῆναι τοῖς συνισταμένοις αὐτῷ μεγιστᾶσιν 
ἠξίου· τὴν δὲ μὴ βουληθεῖσαν ἠτίμωσε, προστάξας ἐπιλεγῆναι παρθένους 
ἐκ πάσης αὐτοῦ τῆς ἐξουσίας, ὧν εὑρέθη πασῶν εὐμορφοτάτη Ἔδεσα ἡ 
καὶ Ἐσθήρ, ἀνεψιὰ Μαρδοχαίου Ἑβραία ἀπὸ τῆς αἰχμαλωσίας τῆς ἐπὶ 
Σεδεκίου. 



Georgius Syncellus Chronogr., Ecloga chronographica 
Page 314, line 26

ὁ αὐτὸς τὴν Ἀορνὴν πέτραν ἐχειρώσατο καὶ Βερναβοᾶν ποταμὸν Ἰνδῶν 
πρὸς Δάνδαμιν διέβη Βραχμανόν. 



Georgius Syncellus Chronogr., Ecloga chronographica 
Page 318, line 19

κἀκεῖθεν μεταχωρήσας ἐπὶ τοὺς Ἰνδοὺς παντός τε κρατήσας ἔθνους 
Ἰνδικοῦ μέχρι ποταμοῦ Γάγγου πάλιν ἀναζεύγνυσι διὰ τοῦ Ἰνδοῦ ποταμοῦ 
μέχρι τῆς Ἰνδικῆς θαλάσσης, καὶ οὕτως εἰς τὴν Ἀσσυρίαν ἐπάνεισι 
Βαβυλῶνα, πᾶσαν ἔχων Εὐρώπην ὑπὸ χεῖρα καὶ τὴν Ἀσίαν. 



Georgius Syncellus Chronogr., Ecloga chronographica 
Page 368, line 10

Τῷ ͵ευξʹ ἔτει τοῦ κόσμου δεύτερον ἐχρημάτισεν Αὐγούστου Καίσαρος 
πλῆρες καὶ ἀρχὴ τοῦ τρίτου, ἐν ᾧ καὶ ἡ ἴνδικτος ὑπ' αὐτοῦ ἤτοι ἐπινέμησις 
ἐθεσπίσθη, ὡς μαρτυρεῖ ὁ μακάριος Μάξιμος ἐν τῷ περὶ τοῦ πάσχα λόγῳ. 



Georgius Syncellus Chronogr., Ecloga chronographica 
Page 376, line 6

Τότε καὶ Πανδίων ὁ τῶν Ἰνδῶν βασιλεὺς ἐπεκηρυκεύσατο φίλος Αὐγού-
στου γενέσθαι καὶ σύμμαχος. 



Georgius Syncellus Chronogr., Ecloga chronographica 
Page 376, line 15

Πανδίων ὁ τῶν Ἰνδῶν βασιλεὺς φίλος Αὐγούστου καὶ σύμμαχος εἶναι 
πρεσβεύεται. 



Georgius Syncellus Chronogr., Ecloga chronographica 
Page 416, line 22

                                                            τὸν γὰρ πρεσβύτε-
ρον υἱὸν Τίτον τὸν πρὸς Ἰουδαίους πόλεμον ἐκτελέσαντα καταλιπὼν αὐτὸς 
ἐπὶ τὴν Ἑλλάδα παραγίνεται, χαίρων, ὡς Ἕλληνες μυθεύονται, ἐφ' οἷς 
ἠκηκόει παρὰ τοῦ Τυανέως Ἀπολλωνίου περὶ τοῦ χρόνου τῆς βασιλείας 
καὶ τῶν λοιπῶν εὐδοκιμήσεων, κατὰ τὴν Αἴγυπτον συντυχὼν αὐτῷ ἐξ 
Ἰνδῶν καὶ Βραχμάνων ἐπανελθόντι τῶν ἐκεῖσε Γυμνοσοφιστῶν. 


Plutarchus Biogr., Phil., Lycurgus (0007: 004)
“Plutarch's lives, vol. 1”, Ed. Perrin, B.
Cambridge, Mass.: Harvard University Press, 1914, Repr. 1967.
Chapter 4, section 6, line 3

                                               ταῦτα 
μὲν οὖν Αἰγυπτίοις ἔνιοι καὶ τῶν Ἑλληνικῶν 
συγγραφέων μαρτυροῦσιν· ὅτι δὲ καὶ Λιβύην καὶ 
Ἰβηρίαν ἐπῆλθεν ὁ Λυκοῦργος καὶ περὶ τὴν 
Ἰνδικὴν πλανηθεὶς τοῖς Γυμνοσοφισταῖς ὡμίλη-
σεν, οὐδένα πλὴν Ἀριστοκράτη τὸν Ἱππάρχου 
Σπαρτιάτην εἰρηκότα γινώσκομεν. 



Plutarchus Biogr., Phil., Aemilius Paullus (0007: 019)
“Plutarchi vitae parallelae, vol. 2.1, 2nd edn.”, Ed. Ziegler, K.
Leipzig: Teubner, 1964.
Chapter 12, section 11, line 1

                                                Ἀλέξανδρος δὲ τῆς ἐπ' Ἰνδοὺς 
στρατείας ἁπτόμενος, καὶ βαρὺν ὁρῶν καὶ δύσογκον ἤδη τὸν Περσικὸν 
ἐφελκομένους πλοῦτον τοὺς Μακεδόνας, πρώτας ὑπέπρησε τὰς βασιλικὰς 
ἁμάξας, εἶτα τοὺς ἄλλους ἔπεισε ταὐτὸ ποιήσαντας ἐλαφροὺς ἀναζεῦξαι 
πρὸς τὸν πόλεμον ὥσπερ λελυμένους. 



Plutarchus Biogr., Phil., Crassus (0007: 039)
“Plutarchi vitae parallelae, vol. 1.2, 3rd edn.”, Ed. Ziegler, K.
Leipzig: Teubner, 1964.
Chapter 16, section 2, line 4

τότε δ' ἐπηρμένος κομιδῇ καὶ διεφθαρμένος, οὐ Συρίαν 
οὐδὲ Πάρθους ὅρον ἐποιεῖτο τῆς εὐπραξίας, ἀλλ' ὡς παι-
διὰν ἀποφανῶν τὰ Λευκόλλου πρὸς Τιγράνην καὶ Πομπηίου 
πρὸς Μιθριδάτην, ἄχρι Βακτρίων καὶ Ἰνδῶν καὶ τῆς ἔξω 
θαλάσσης ἀνῆγεν ἑαυτὸν ταῖς ἐλπίσι. 



Plutarchus Biogr., Phil., Comparatio Niciae et Crassi (0007: 040)
“Plutarchi vitae parallelae, vol. 1.2, 3rd edn.”, Ed. Ziegler, K.
Leipzig: Teubner, 1964.
Chapter 2, section 7, line 5

                                                          ὁ μὲν γὰρ 
τῆς εἰρήνης ἔρως θεῖος ἦν ὡς ἀληθῶς, καὶ τὸ λῦσαι τὸν 
πόλεμον ἑλληνικώτατον πολίτευμα, καὶ τῆς πράξεως 
ἕνεκα ταύτης οὐκ ἄξιον Νικίᾳ παραβαλεῖν Κράσσον, οὐδ' 
εἰ τὸ Κάσπιον φέρων πέλαγος ἢ τὸν Ἰνδῶν ὠκεανὸν τῇ 
Ῥωμαίων ἡγεμονίᾳ προσώρισε. 



Plutarchus Biogr., Phil., Comparatio Niciae et Crassi 
Chapter 4, section 2, line 4

                                                ὁ μὲν γὰρ ἐμπειρίᾳ 
καὶ λογισμῷ χρησάμενος ἡγεμόνος ἔμφρονος, οὐ συνη-
πατήθη ταῖς ἐλπίσι τῶν πολιτῶν, ἀλλ' ἔδεισε καὶ ἀπέγνω 
λήψεσθαι Σικελίαν· ὁ δ' ὡς ἐπὶ ῥᾷστον ἔργον τὸν Παρθι-
κὸν ὁρμήσας πόλεμον, ἥμαρτε <μέν>, ὠρέχθη δὲ μεγά-
λων, Καίσαρος τὰ ἑσπέρια καὶ Κελτοὺς καὶ Γερμανοὺς 
καταστρεφομένου καὶ Βρεττανίαν, αὐτὸς ἐπὶ τὴν ἕω καὶ 
τὴν Ἰνδικὴν ἐλάσαι θάλασσαν καὶ προς<κατ>εργάσασθαι 
τὴν Ἀσίαν, οἷς Πομπήιος ἐπῆλθε καὶ Λεύκολλος ἀντέσχεν, 
ἄνδρες εὐμενεῖς καὶ πρὸς πάντας ἀγαθοὶ διαμείναντες, 
προελόμενοι δ' ὅμοια Κράσσῳ καὶ τὰς αὐτὰς ὑποθέσεις 
λαβόντες· ἐπεὶ καὶ Πομπηίῳ τῆς ἀρχῆς διδομένης ἡ σύγ-
κλητος ἠναντιοῦτο, καὶ Καίσαρα μυριάδας τριάκοντα 
Γερμανῶν τρεψάμενον συνεβούλευεν ὁ Κάτων ἐκδοῦναι 
τοῖς ἡττημένοις καὶ τρέψαι τὸ μήνιμα τοῦ παρασπονδή-
ματος εἰς ἐκεῖνον, ὁ δὲ δῆμος ἐρρῶσθαι φράσας Κάτωνι 
πεντεκαίδεκα ἡμέρας ἔθυεν ἐπινίκια καὶ περιχαρὴς ἦν. 



Plutarchus Biogr., Phil., Eumenes (0007: 041)
“Plutarchi vitae parallelae, vol. 2.1, 2nd edn.”, Ed. Ziegler, K.
Leipzig: Teubner, 1964.
Chapter 1, section 5, line 1

                                                            μετὰ δὲ τὴν ἐκείνου 
τελευτὴν οὔτε συνέσει τινὸς οὔτε πίστει λείπεσθαι δοκῶν τῶν περὶ Ἀλέξ-
ανδρον, ἐκαλεῖτο μὲν ἀρχιγραμματεύς, τιμῆς δ' ἧσπερ οἱ μάλιστα φίλοι 
καὶ συνήθεις ἐτύγχανεν, ὥστε καὶ στρατηγὸς ἀποσταλῆναι κατὰ τὴν Ἰνδι-
κὴν ἐφ' ἑαυτοῦ μετὰ δυνάμεως, καὶ τὴν Περδίκκου παραλαβεῖν ἱππαρχίαν, 
ὅτε Περδίκκας ἀποθανόντος Ἡφαιστίωνος εἰς τὴν ἐκείνου προῆλθε τάξιν. 



Plutarchus Biogr., Phil., Pompeius (0007: 045)
“Plutarch's lives, vol. 5”, Ed. Perrin, B.
Cambridge, Mass.: Harvard University Press, 1917, Repr. 1968.
Chapter 70, section 3, line 1

πολὺ δὲ καὶ Σκυθία λειπόμενον ἔργον καὶ Ἰνδοί, 
καὶ πρόφασις οὐκ ἄδοξος ἐπὶ ταῦτα τῆς πλεον-
εξίας ἡμερῶσαι τὰ βαρβαρικά. 



Plutarchus Biogr., Phil., Pompeius 
Chapter 70, section 3, line 5

                                  τίς δ' ἂν ἢ 
Σκυθῶν ἵππος ἢ τοξεύματα Πάρθων ἢ πλοῦτος 
Ἰνδῶν ἐπέσχε μυριάδας ἑπτὰ Ῥωμαίων ἐν ὅπλοις 
ἐπερχομένας Πομπηΐου καὶ Καίσαρος ἡγουμένων, 
ὧν ὄνομα πολὺ πρότερον ἤκουσαν ἢ τὸ Ῥωμαίων; 



Plutarchus Biogr., Phil., Alexander (0007: 047)
“Plutarchi vitae parallelae, vol. 2.2, 2nd edn.”, Ed. Ziegler, K.
Leipzig: Teubner, 1968.
Chapter 13, section 4, line 2

                                       ὅλως δὲ καὶ τὸ περὶ 
Κλεῖτον ἔργον ἐν οἴνῳ γενόμενον, καὶ τὴν πρὸς Ἰνδοὺς 
τῶν Μακεδόνων ἀποδειλίασιν, ὥσπερ ἀτελῆ τὴν στρα-
τείαν καὶ τὴν δόξαν αὐτοῦ προεμένων, εἰς μῆνιν ἀνῆγε   
Διονύσου καὶ νέμεσιν. 



Plutarchus Biogr., Phil., Alexander 
Chapter 47, section 11, line 3

                                                          διὸ 
καὶ πρὸς ἀλλήλους ὑπούλως ἔχοντες, συνέκρουον πολλάκις, 
ἅπαξ δὲ περὶ τὴν Ἰνδικὴν καὶ εἰς χεῖρας ἦλθον σπασά-
μενοι τὰ ξίφη, καὶ τῶν φίλων ἑκατέρῳ παραβοηθούντων, 
προσελάσας <ὁ> Ἀλέξανδρος ἐλοιδόρει τὸν Ἡφαιστίωνα   
φανερῶς, ἔμπληκτον καλῶν καὶ μαινόμενον, εἰ μὴ συνίησιν 
ὡς ἐάν τις αὐτοῦ τὸν Ἀλέξανδρον ἀφέληται, μηδέν ἐστιν· 
ἰδίᾳ δὲ καὶ τοῦ Κρατεροῦ πικρῶς καθήψατο, καὶ συν-
αγαγὼν αὐτοὺς καὶ διαλλάξας, ἐπώμοσε τὸν Ἄμμωνα καὶ 
τοὺς ἄλλους θεούς, ἦ μὴν μάλιστα φιλεῖν ἀνθρώπων 
ἁπάντων ἐκείνους· ἂν δὲ πάλιν αἴσθηται διαφερομένους, 
ἀποκτενεῖν ἀμφοτέρους ἢ τὸν ἀρξάμενον. 



Plutarchus Biogr., Phil., Alexander 
Chapter 55, section 9, line 7

                                     ἀποθανεῖν δ' αὐτὸν οἱ 
μὲν ὑπ' Ἀλεξάνδρου κρεμασθέντα λέγουσιν, οἱ δ' ἐν πέ-
δαις δεδεμένον καὶ νοσήσαντα, Χάρης δὲ (FGrH 125 F 15) 
μετὰ τὴν σύλληψιν ἑπτὰ μῆνας φυλάττεσθαι δεδεμένον, 
ὡς ἐν τῷ συνεδρίῳ κριθείη παρόντος Ἀριστοτέλους· ἐν 
αἷς δ' ἡμέραις Ἀλέξανδρος [ἐν Μαλλοῖς Ὀξυδράκαις]   
ἐτρώθη περὶ τὴν Ἰνδίαν, ἀποθανεῖν ὑπέρπαχυν γενόμενον 
καὶ φθειριάσαντα. 



Plutarchus Biogr., Phil., Alexander 
Chapter 57, section 1, line 1

Μέλλων δ' ὑπερβάλλειν εἰς τὴν Ἰνδικὴν ὡς ἑώρα 
πλήθει λαφύρων τὴν στρατιὰν ἤδη βαρεῖαν καὶ δυσκίνητον 
οὖσαν, ἅμ' ἡμέρᾳ συνεσκευασμένων τῶν ἁμαξῶν πρώ-
τας μὲν ὑπέπρησε τὰς αὑτοῦ καὶ <τὰς> τῶν ἑταίρων, 
μετὰ δὲ ταύτας ἐκέλευσε καὶ ταῖς τῶν Μακεδόνων 
ἐνεῖναι πῦρ. 



Plutarchus Biogr., Phil., Alexander 
Chapter 59, section 1, line 1

Ὁ δὲ Ταξίλης λέγεται μὲν τῆς Ἰνδικῆς ἔχειν μοῖραν 
οὐκ ἀποδέουσαν Αἰγύπτου τὸ μέγεθος, εὔβοτον δὲ καὶ 
καλλίκαρπον ἐν τοῖς μάλιστα, σοφὸς δέ τις ἀνὴρ εἶναι 
καὶ τὸν Ἀλέξανδρον ἀσπασάμενος “τί δεῖ πολέμων” 
φάναι “καὶ μάχης ἡμῖν Ἀλέξανδρε πρὸς ἀλλήλους, εἰ 
μήθ' ὕδωρ ἀφαιρησόμενος ἡμῶν ἀφῖξαι, μήτε τροφὴν 
ἀναγκαίαν, ὑπὲρ ὧν μόνων ἀνάγκη διαμάχεσθαι νοῦν 
ἔχουσιν ἀνθρώποις; 



Plutarchus Biogr., Phil., Alexander 
Chapter 59, section 6, line 1

Ἐπεὶ δὲ τῶν Ἰνδῶν οἱ μαχιμώτατοι μισθοφοροῦντες ἐπε-
φοίτων ταῖς πόλεσιν ἐρρωμένως ἀμύνοντες, καὶ πολλὰ τὸν 
Ἀλέξανδρον ἐκακοποίουν, σπεισάμενος ἔν τινι πόλει πρὸς 
αὐτούς, ἀπιόντας ἐν ὁδῷ λαβὼν ἅπαντας ἀπέκτεινε. 



Plutarchus Biogr., Phil., Alexander 
Chapter 62, section 1, line 2

Τοὺς μέντοι Μακεδόνας ὁ πρὸς Πῶρον ἀγὼν 
ἀμβλυτέρους ἐποίησε, καὶ τοῦ πρόσω τῆς Ἰνδικῆς ἔτι 
προελθεῖν ἐπέσχε. 



Plutarchus Biogr., Phil., Alexander 
Chapter 62, section 4, line 4

                   Ἀνδρόκοττος γὰρ ὕστερον οὐ πολλῷ 
βασιλεύσας Σελεύκῳ πεντακοσίους ἐλέφαντας ἐδωρή-
σατο, καὶ στρατοῦ μυριάσιν ἑξήκοντα τὴν Ἰνδικὴν ἐπῆλθεν 
ἅπασαν καταστρεφόμενος. 



Plutarchus Biogr., Phil., Alexander 
Chapter 63, section 2, line 3

                                                    πρὸς δὲ 
τοῖς καλουμένοις Μαλλοῖς, οὕς φασιν Ἰνδῶν μαχιμωτά-
τους γενέσθαι, μικρὸν ἐδέησε κατακοπῆναι. 



Plutarchus Biogr., Phil., Alexander 
Chapter 65, section 5, line 2

          τὸν μέντοι Καλανὸν ἔπεισεν ὁ Ταξίλης ἐλθεῖν πρὸς 
Ἀλέξανδρον· ἐκαλεῖτο δὲ Σφίνης· ἐπεὶ δὲ κατ' Ἰνδικὴν 
γλῶτταν τῷ καλὲ προσαγορεύων ἀντὶ τοῦ χαίρειν τοὺς 
ἐντυγχάνοντας ἠσπάζετο, Καλανὸς ὑπὸ τῶν Ἑλλήνων 
ὠνομάσθη. 



Plutarchus Biogr., Phil., Alexander 
Chapter 66, section 3, line 2

                         καὶ τὰς μὲν ναῦς ἐκέλευσε παρα-
πλεῖν, ἐν δεξιᾷ τὴν Ἰνδικὴν ἐχούσας, ἡγεμόνα μὲν Νέαρχον 
ἀποδείξας, ἀρχικυβερνήτην δ' Ὀνησίκριτον· αὐτὸς δὲ πεζῇ 
δι' Ὠρειτῶν πορευόμενος, εἰς ἐσχάτην ἀπορίαν προήχθη, 
καὶ πλῆθος ἀνθρώπων ἀπώλεσε <τοσοῦτον>, ὥστε τῆς μα-
χίμου δυνάμεως μηδὲ τὸ τέταρτον ἐκ τῆς Ἰνδικῆς ἀπαγα-
γεῖν. 



Plutarchus Biogr., Phil., Alexander 
Chapter 69, section 8, line 2

                                                     (τοῦτο πολ-
λοῖς ἔτεσιν ὕστερον ἄλλος Ἰνδὸς ἐν Ἀθήναις Καίσαρι 
συνὼν ἐποίησε, καὶ δείκνυται μέχρι νῦν τὸ μνημεῖον, 
Ἰνδοῦ προσαγορευόμενον). 



Plutarchus Biogr., Phil., Demetrius (0007: 057)
“Plutarchi vitae parallelae, vol. 3.1, 2nd edn.”, Ed. Ziegler, K.
Leipzig: Teubner, 1971.
Chapter 7, section 2, line 4

Ἐπεὶ δὲ Σέλευκος ἐκπεσὼν μὲν ὑπ' Ἀντιγόνου τῆς 
Βαβυλωνίας πρότερον, ὕστερον δ' ἀναλαβὼν τὴν ἀρχὴν 
δι' αὑτοῦ καὶ κρατῶν ἀνέβη μετὰ δυνάμεως τὰ συν-
οροῦντα τοῖς Ἰνδοῖς ἔθνη καὶ τὰς περὶ Καύκασον ἐπαρ-
χίας προσαξόμενος, ἐλπίζων Δημήτριος ἔρημον εὑρήσειν 
τὴν Μεσοποταμίαν καὶ περάσας ἄφνω τὸν Εὐφράτην, εἰς 
τὴν Βαβυλῶνα παρεισπεσὼν ἔφθη, καὶ τῆς ἑτέρας ἄκρας 
– δύο γὰρ ἦσαν – ἐκκρούσας τὴν Σελεύκου φρουρὰν 
καὶ κρατήσας, ἰδίους ἐγκατέστησεν ἑπτακισχιλίους ἄνδρας. 



Plutarchus Biogr., Phil., Demetrius 
Chapter 32, section 7, line 5

                               Κιλικίαν δ' ἀξιῶν χρήματα 
λαβόντα παραδοῦναι Δημήτριον, ὡς <δ'> οὐκ ἔπειθε 
Σιδῶνα καὶ Τύρον ἀπαιτῶν πρὸς ὀργήν, ἐδόκει βίαιος 
εἶναι καὶ δεινὰ ποιεῖν, εἰ τὴν ἀπ' Ἰνδῶν ἄχρι τῆς κατὰ 
Συρίαν θαλάσσης ἅπασαν ὑφ' αὑτῷ πεποιημένος, οὕτως 
ἐνδεής ἐστιν ἔτι πραγμάτων καὶ πτωχός, ὡς ὑπὲρ δυεῖν 
πόλεων ἄνδρα κηδεστὴν καὶ μεταβολῇ τύχης κεχρημένον 
ἐλαύνειν, λαμπρὰν τῷ Πλάτωνι μαρτυρίαν διδούς, δια-  
κελευομένῳ (leg. 5, 736e) μὴ τὴν οὐσίαν πλείω, τὴν δ' 
ἀπληστίαν ποιεῖν ἐλάσσω τόν γε βουλόμενον ὡς ἀληθῶς 
εἶναι πλούσιον, ὡς ὅ γε μὴ παύων φιλοπλουτίαν [οὗτος] 
οὔτε πενίας οὔτ' ἀπορίας ἀπήλλακται. 



Plutarchus Biogr., Phil., Antonius (0007: 058)
“Plutarchi vitae parallelae, vol. 3.1, 2nd edn.”, Ed. Ziegler, K.
Leipzig: Teubner, 1971.
Chapter 37, section 5, line 2

         τοσαύτην μέντοι παρασκευὴν καὶ δύναμιν, ἣ καὶ 
τοὺς πέραν Βάκτρων Ἰνδοὺς ἐφόβησε καὶ πᾶσαν ἐκρά-
δανε τὴν Ἀσίαν, ἀνόνητον αὐτῷ διὰ Κλεοπάτραν γενέ-
σθαι λέγουσι. 



Plutarchus Biogr., Phil., Antonius 
Chapter 81, section 4, line 3

                              Καισαρίωνα δὲ τὸν ἐκ Καίσαρος 
γεγονέναι λεγόμενον ἡ μὲν μήτηρ ἐξέπεμψε μετὰ χρημά-
των πολλῶν εἰς τὴν Ἰνδικὴν δι' Αἰθιοπίας, ἕτερος δὲ 
παιδαγωγὸς ὅμοιος Θεοδώρῳ Ῥόδων ἀνέπεισεν ἐπανελ-  
θεῖν, ὡς Καίσαρος αὐτὸν ἐπὶ βασιλείαν καλοῦντος. 



Plutarchus Biogr., Phil., Comparatio Dionis et Bruti (0007: 062)
“Plutarchi vitae parallelae, vol. 2.1, 2nd edn.”, Ed. Ziegler, K.
Leipzig: Teubner, 1964.
Chapter 4, section 3, line 3

                             Διονυσίου μὲν γὰρ οὐδεὶς ὅστις οὐκ ἂν κατεφρό-
νησε τῶν συνήθων, ἐν μέθαις καὶ κύβοις καὶ γυναιξὶ τὰς πλείστας ποιου-
μένου διατριβάς· τὸ δὲ τὴν Καίσαρος κατάλυσιν εἰς νοῦν ἐμβαλέσθαι καὶ 
μὴ φοβηθῆναι τὴν δεινότητα καὶ δύναμιν καὶ τύχην, οὗ καὶ τοὔνομα τοὺς 
Παρθυαίων καὶ Ἰνδῶν βασιλεῖς οὐκ εἴα καθεύδειν, ὑπερφυοῦς ἦν ψυχῆς 
καὶ πρὸς μηθὲν ὑφίεσθαι φόβῳ τοῦ φρονήματος δυναμένης. 



Plutarchus Biogr., Phil., De tuenda sanitate praecepta (122b–137e) (0007: 077)
“Plutarch's moralia, vol. 2”, Ed. Babbitt, F.C.
Cambridge, Mass.: Harvard University Press, 1928, Repr. 1962.
Stephanus page 133, section C, line 1

                         ἀλειπτῶν δὲ φωνὰς καὶ 
παιδοτριβῶν λόγους ἑκάστοτε λεγόντων ὡς τὸ 
παρὰ δεῖπνον φιλολογεῖν τὴν τροφὴν διαφθείρει 
καὶ βαρύνει τὴν κεφαλὴν τότε φοβητέον, ὅταν 
τὸν Ἰνδὸν ἀναλύειν ἢ διαλέγεσθαι περὶ τοῦ 
Κυριεύοντος ἐν δείπνῳ μέλλωμεν. 



Plutarchus Biogr., Phil., Regum et imperatorum apophthegmata [Sp.?] (172b–208a) (0007: 081)
“Plutarchi moralia, vol. 2.1”, Ed. Nachstädt, W.
Leipzig: Teubner, 1935, Repr. 1971.
Stephanus page 181, section B, line 5

Τῶν δ' Ἰνδῶν τὸν ἄριστα τοξεύειν δοκοῦντα καὶ 
λεγόμενον διὰ δακτυλίου τὸν ὀιστὸν ἀφιέναι λαβὼν αἰχμά-
λωτον ἐκέλευσεν ἐπιδείξασθαι, καὶ μὴ βουλόμενον ὀργι-
σθεὶς ἀνελεῖν προσέταξε· ἐπεὶ δ' ἀγόμενος ὁ ἄνθρωπος 
ἔλεγε πρὸς τοὺς ἄγοντας ὅτι πολλῶν ἡμερῶν οὐ μεμελέ-
τηκε καὶ ἐφοβήθη διαπεσεῖν, ἀκούσας ὁ Ἀλέξανδρος 
ἐθαύμασε καὶ ἀπέλυσε μετὰ δώρων αὐτόν, ὅτι μᾶλλον 
ἀποθανεῖν ὑπέμεινεν ἢ τῆς δόξης ἀνάξιος φανῆναι. 



Plutarchus Biogr., Phil., Regum et imperatorum apophthegmata [Sp.?] (172b-208a) 
Stephanus page 181, section C, line 1

Ἐπεὶ δὲ Ταξίλης, εἷς τῶν Ἰνδῶν βασιλεὺς ὤν, 
ἀπαντήσας προυκαλεῖτο μὴ μάχεσθαι μηδὲ πολεμεῖν 
Ἀλέξανδρον, ἀλλ' εἰ μέν ἐστιν ἥττων, εὖ πάσχειν, εἰ δὲ 
βελτίων, εὖ ποιεῖν, ἀπεκρίνατο περὶ αὐτοῦ τούτου μαχη-
τέον εἶναι, πότερος εὖ ποιῶν περιγένηται. 



Plutarchus Biogr., Phil., Regum et imperatorum apophthegmata [Sp.?] (172b-208a) 
Stephanus page 181, section C, line 6

Περὶ δὲ τῆς λεγομένης Ἀόρνου πέτρας ἐν Ἰνδοῖς 
ἀκούσας, ὅτι τὸ μὲν χωρίον δυσάλωτόν ἐστιν ὁ δὲ ἔχων 
αὐτὸ δειλός ἐστι, ‘νῦν’ ἔφη ‘τὸ χωρίον εὐάλωτόν ἐστιν. 



Plutarchus Biogr., Phil., Apophthegmata Laconica [Sp.?] (208b–242d) (0007: 082)
“Plutarchi moralia, vol. 2.1”, Ed. Nachstädt, W.
Leipzig: Teubner, 1935, Repr. 1971.
Stephanus page 219, section E, line 8

           
ΔΑΜΙΣ


 Δᾶμις πρὸς τὰ ἐπισταλέντα περὶ τοῦ Ἀλέξανδρον θεὸν 
εἶναι ψηφίσασθαι, ‘συγχωρῶμεν’ ἔφη ‘Ἀλεξάνδρῳ, ἐὰν 
θέλῃ, θεὸς καλεῖσθαι.’   
          
ΔΑΜΙΝΔ*αΣ


 Δαμίνδας, Φιλίππου ἐμβαλόντος εἰς Πελοπόννησον καὶ 
εἰπόντος τινός ‘κινδυνεύουσι δεινὰ παθεῖν Λακεδαιμόνιοι, 
εἰ μὴ τὰς πρὸς αὐτὸν διαλλαγὰς ποιήσονται,’ ‘ἀνδρό-
γυνε’ εἶπε, ‘τί δ' ἂν πάθοιμεν δεινὸν θανάτου καταφρονή-
σαντες; 



Plutarchus Biogr., Phil., Apophthegmata Laconica [Sp.?] (208b-242d) 
Stephanus page 236, section A, line 8

                                               ὡς δ' ἐκεῖνος 
ἀγασθεὶς ἀπέλυσε τοὺς ἄνδρας καὶ ἠξίου μένειν παρ' 
αὐτῷ, ‘καὶ πῶς ἄν’ ἔφασαν ‘δυναίμεθα ζῆν ἐνταῦθα, πα-
τρίδα καταλιπόντες καὶ νόμους καὶ τούτους τοὺς ἄν-
δρας, ὑπὲρ ὧν τοσαύτην ἤλθομεν ὁδὸν ἀποθανούμενοι;’ 
Ἰνδάρνου δὲ τοῦ στρατηγοῦ ἐπὶ πλέον δεομένου καὶ λέ-
γοντος τεύξεσθαι αὐτοὺς τῆς ἴσης τιμῆς τοῖς μάλιστα ἐν 
προαγωγῇ φίλοις τοῦ βασιλέως, ἔφασαν ‘ἀγνοεῖν ἡμῖν 
δοκεῖς, ἡλίκον ἐστὶ τὸ τῆς ἐλευθερίας, ἧς οὐκ ἂν ἀλλά-
ξαιτό τις νοῦν ἔχων τὴν Περσῶν βασιλείαν. 



Plutarchus Biogr., Phil., De Alexandri magni fortuna aut virtute (326d–345b) (0007: 087)
“Plutarchi moralia, vol. 2.2”, Ed. Nachstädt, W.
Leipzig: Teubner, 1935, Repr. 1971.
Stephanus page 327, section A, line 10

                                            πρῶτον ἐν   
Ἰλλυριοῖς λίθῳ τὴν κεφαλὴν ὑπέρῳ δὲ τὸν τράχηλον 
ἠλοήθην· ἔπειτα περὶ Γράνικον τὴν κεφαλὴν βαρβαρικῇ 
μαχαίρᾳ διεκόπην, ἐν δ' Ἰσσῷ ξίφει τὸν μηρόν· πρὸς 
δὲ Γάζῃ τὸ μὲν σφυρὸν ἐτοξεύθην, τὸν δ' ὦμον ἐμπεσὼν 
βῶλος ἐξ ἕδρας περιεδίνησε· πρὸς δὲ Μαρακανδάνοις 
τοξεύμασι τὸ τῆς κνήμης ὀστέον διεσχίσθην· τὰ λοιπὰ 
δ' Ἰνδῶν πληγαὶ καὶ βίαι **** θυμῶν· ἐν Ἀσπασίοις 
ἐτοξεύθην τὸν ὦμον, ἐν δὲ Γανδρίδαις τὸ σκέλος· ἐν 
Μαλλοῖς βέλει μὲν ἀπὸ τόξου τὸ στέρνον ἐνερεισθέντι 
καὶ καταδύσαντι τὸν σίδηρον, ὑπέρου δὲ πληγῇ παρὰ τὸν 
τράχηλον, ὅτε προστεθεῖσαι τοῖς τείχεσιν αἱ κλίμακες 
ἐκλάσθησαν, ἐμὲ δ' ἡ Τύχη μόνον συνεῖρξεν οὐδὲ λαμπροῖς 
ἀνταγωνισταῖς, ἀλλὰ βαρβάροις ἀσήμοις χαριζομένη 
τηλικοῦτον ἔργον· εἰ δὲ μὴ Πτολεμαῖος ὑπερέσχε τὴν 
πέλτην, Λιμναῖος δὲ πρὸ ἐμοῦ τοῖς μυρίοις ἀπαντήσας   
βέλεσιν ἔπεσεν, ἤρειψαν δὲ θυμῷ καὶ βίᾳ Μακεδόνες 




Plutarchus Biogr., Phil., De Alexandri magni fortuna aut virtute (326d-345b) 
Stephanus page 328, section C, line 8

             ὢ θαυμαστῆς φιλοσοφίας, δι' ἣν Ἰνδοὶ 
θεοὺς Ἑλληνικοὺς προσκυνοῦσι, Σκύθαι θάπτουσι τοὺς 
ἀποθανόντας οὐ κατεσθίουσι. 



Plutarchus Biogr., Phil., De Alexandri magni fortuna aut virtute (326d-345b) 
Stephanus page 328, section F, line 6

οὐκ ἂν εἶχεν Ἀλεξάνδρειαν Αἴγυπτος οὐδὲ Μεσοποταμία 
Σελεύκειαν οὐδὲ Προφθασίαν Σογδιανὴ οὐδ' Ἰνδία   
Βουκεφαλίαν οὐδὲ πόλιν Ἑλλάδα Καύκασος παροι-
κοῦσαν, | αἷς ἐμπολισθείσαις ἐσβέσθη τὸ ἄγριον καὶ 
μετέβαλε τὸ χεῖρον ὑπὸ τοῦ κρείττονος ἐθιζόμενον. 



Plutarchus Biogr., Phil., De Alexandri magni fortuna aut virtute (326d-345b) 
Stephanus page 332, section B, line 1

             νῦν δὲ σύγγνωθι, Διόγενες, Ἡρακλέα μιμοῦμαι 
καὶ Περσέα ζηλῶ, καὶ τὰ Διονύσου μετιὼν ἴχνη, θεοῦ 
γενάρχου καὶ προπάτορος, βούλομαι πάλιν ἐν Ἰνδίᾳ 
νικῶντας Ἕλληνας ἐγχορεῦσαι καὶ τοὺς ὑπὲρ Καύκασον 
ὀρείους καὶ ἀγρίους τῶν βακχικῶν κώμων ἀναμνῆσαι. 



Plutarchus Biogr., Phil., De Alexandri magni fortuna aut virtute (326d-345b) 
Stephanus page 341, section B, line 6

ἰδοῦ κατατετρωμένον· ἐξ ἄκρας κεφαλῆς ἄχρι ποδῶν δια-
κέκοπται καὶ περιτέθλασται τυπτόμενον ὑπὸ τῶν πολε-
μίων (Hom. Λ 265. 541) ‘ἔγχεΐ τ' ἄορί τε μεγάλοισί τε 
χερμαδίοισιν·’ ἐπὶ Γρανίκου ξίφει διακοπεὶς τὸ κράνος 
ἄχρι τῶν τριχῶν, ἐν Γάζῃ βέλει πληγεὶς τὸν ὦμον, ἐν 
Μαρακάνδοις τοξεύματι τὴν κνήμην ὥστε τῆς κερκίδος 
τὸ ὀστέον ἀποκλασθὲν ὑπὸ τῆς πληγῆς ἐξαλέγθαι· περὶ 
τὴν Ὑρκανίαν λίθῳ τὸν τράχηλον, ἐξ οὗ καὶ τὰς ὄψεις 
ἀμαυρωθεὶς ἐφ' ἡμέρας πολλὰς ἐν φόβῳ πηρώσεως ἐγέ-
νετο· πρὸς Ἀσσακάνοις Ἰνδικῷ βέλει τὸ σφυρόν, ὅτε καὶ 
πρὸς τοὺς κόλακας εἶπεν ἐπιμειδιάσας ‘τουτὶ μὲν αἷμα, 
οὐκ (Hom. Ε 340) ‘ἰχώρ, οἷός πέρ τε ῥέει μακάρεσσι 
θεοῖσιν’·’ ἐν Ἰσσῷ ξίφει τὸν μηρόν, ὡς Χάρης (fr. 1) φησίν, 
ὑπὸ Δαρείου τοῦ βασιλέως εἰς χεῖρας αὐτῷ συνδραμόν-
τος· αὐτὸς δ' Ἀλέξανδρος ἁπλῶς γράφων καὶ μετὰ πάσης   
ἀληθείας πρὸς Ἀντίπατρον ‘συνέβη δέ μοι’ φησί ‘καὶ 
αὐτῷ ἐγχειριδίῳ πληγῆναι εἰς τὸν μηρόν· ἀλλ' οὐδὲν 
ἄτοπον οὔτε παραχρῆμα οὔθ' ὕστερον ἐκ τῆς πληγῆς 
ἀπήντησεν. 



Plutarchus Biogr., Phil., De Iside et Osiride (351c–384c) (0007: 089)
“Plutarchi moralia, vol. 2.3”, Ed. Sieveking, W.
Leipzig: Teubner, 1935, Repr. 1971.
Stephanus page 362, section B, line 11

Οὐ γὰρ ἄξιον προσέχειν τοῖς Φρυγίοις γράμμασιν, ἐν 
οἷς λέγεται † χαροπῶς τοὺς μὲν τοῦ Ἡρακλέους γενέσθαι 
θυγάτηρ, † ἰσαιακοῦ δὲ τοῦ Ἡρακλέους ὁ Τυφών, οὐδὲ 
Φυλάρχου μὴ καταφρονεῖν γράφοντος (FGrHist. 81 fr. 78), 
ὅτι πρῶτος εἰς Αἴγυπτον ἐξ Ἰνδῶν Διόνυσος ἤγαγε δύο 
βοῦς, ὧν ἦν τῷ μὲν Ἆπις ὄνομα τῷ δ' Ὄσιρις· Σάραπις 
δ' ὄνομα τοῦ τὸ πᾶν κοσμοῦντός ἐστι παρὰ τὸ σαί-
ρειν, ὃ καλλύνειν τινὲς καὶ κοσμεῖν λέγουσιν. 



Plutarchus Biogr., Phil., De defectu oraculorum (409e–438d) (0007: 092)
“Plutarchi moralia, vol. 3”, Ed. Sieveking, W.
Leipzig: Teubner, 1929, Repr. 1972.
Stephanus page 422, section D, line 8

                                 ἐλέγχει δ' αὐτὸν ὁ τῶν 
κόσμων ἀριθμὸς οὐκ ὢν Αἰγύπτιος οὐδ' Ἰνδὸς ἀλλὰ 
Δωριεὺς ἀπὸ Σικελίας, ἀνδρὸς Ἱμεραίου τοὔνομα Πέτρω-
νος. 



Plutarchus Biogr., Phil., An vitiositas ad infelicitatem sufficiat (498a–500a) (0007: 099)
“Plutarchi moralia, vol. 3”, Ed. Pohlenz, M.
Leipzig: Teubner, 1929, Repr. 1972.
Stephanus page 499, section C, line 3

καὶ μὴν τὸ πῦρ σου Δέκιος ὁ Ῥωμαίων στρατηγὸς προέ-
λαβεν, ὅτε τῶν στρατοπέδων ἐν μέσῳ πυρὰν νήσας τῷ 
Κρόνῳ κατ' εὐχὴν αὐτὸς ἑαυτὸν ἐκαλλιέρησεν ὑπὲρ τῆς 
ἡγεμονίας. Ἰνδῶν δὲ φίλανδροι καὶ σώφρονες γυναῖκες 
ὑπὲρ τοῦ πυρὸς ἐρίζουσι καὶ μάχονται πρὸς ἀλλήλας, τὴν 
δὲ νικήσασαν τεθνηκότι τῷ ἀνδρὶ συγκαταφλεγῆναι μακα-
ρίαν ᾄδουσιν αἱ λοιπαί· τῶν δ' ἐκεῖ σοφῶν οὐδεὶς ζηλωτὸς 
οὐδὲ μακαριστός ἐστιν, ἂν μὴ ζῶν ἔτι καὶ φρονῶν καὶ 
ὑγιαίνων τοῦ σώματος τὴν ψυχὴν πυρὶ διαστήσῃ καὶ 
καθαρὸς ἐκβῇ τῆς σαρκὸς ἐκνιψάμενος τὸ θνητόν. 



Plutarchus Biogr., Phil., De facie in orbe lunae (920b–945e) (0007: 126)
“Plutarchi moralia, vol. 5.3, 2nd edn.”, Ed. Pohlenz, M.
Leipzig: Teubner, 1960.
Stephanus page 921, section B, line 2

                         ὥσπερ οὖν τὴν ἶ<ριν> οἴεσθ' 
ὑμεῖς ἀνακλωμένης ἐπὶ τὸν ἥλιον τῆς ὄψεως ἐνορᾶσθαι 
τῷ νέφει λαβόντι νοτερὰν ἡσυχῇ λειότητα καὶ <πῆ>ξιν, 
οὕτως ἐκεῖνος ἐνορᾶσθαι τῇ σελήνῃ τὴν ἔξω θάλασσαν 
οὐκ ἐφ' ἧς ἐστι χώρας, ἀλλ' ὅθεν ἡ κλάσις ἐποίησε τῇ 
ὄψει τὴν ἐπαφὴν αὐτῆς καὶ τὴν ἀνταύγειαν· ὥς που πά-
λιν ὁ Ἀγησιάναξ εἴρηκεν (fr. 2) 
       
 ’ᾗ πόντου μέγα κῦμα καταντία κυμαίνοντος 
 δείκελον ἰνδάλλοιτο πυριφλεγέθοντος ἐσόπτρου. 



Plutarchus Biogr., Phil., De facie in orbe lunae (920b-945e) 
Stephanus page 938, section B, line 11

                                                  τὴν μὲν γὰρ 
Ἰνδικὴν ῥίζαν, ἥν φησι Μεγασθένης τοὺς <μήτ' ἐσθίον-
τας> μήτε πίνοντας ἀλλ' ἀστόμους ὄντας ὑποτύφειν καὶ 
θυμιᾶν καὶ τρέφεσθαι τῇ ὀσμῇ, πόθεν ἄν τις ἐκεῖ φυο-
μένην λάβοι μὴ βρεχομένης τῆς σελήνης; 



Plutarchus Biogr., Phil., Aquane an ignis sit utilior [Sp.] (955d–958e) (0007: 128)
“Plutarchi moralia, vol. 6.1”, Ed. Hubert, C.
Leipzig: Teubner, 1954, Repr. 1959.
Stephanus page 957, section A, line 9

                           νυνὶ δὲ τοῦτο μὲν παρ' Ἰνδῶν 
ἄμπελον τοῖς Ἕλλησιν, ἐκ δὲ τῆς Ἑλλάδος καρπῶν χρῆ-
σιν τοῖς ἐπέκεινα [ὁ] τῆς θαλάσσης ἔδωκεν, ἐκ Φοινίκης 
δὲ γράμματα μνημόσυνα λήθης ἐκόμισε, καὶ ἄοινον καὶ 
ἄκαρπον καὶ ἀπαίδευτον ἐκώλυσεν εἶναι τὸ πλεῖστον 
ἀνθρώπων γένος. 



Plutarchus Biogr., Phil., De sollertia animalium (959a–985c) (0007: 129)
“Plutarchi moralia, vol. 6.1”, Ed. Hubert, C.
Leipzig: Teubner, 1954, Repr. 1959.
Stephanus page 970, section F, line 5

                                  φασὶ δὲ καὶ τὸν πρωτεύοντα 
κύνα τῶν Ἰνδικῶν † καὶ μαχεσθέντα πρὸς Ἀλέξανδρον, 
ἐλάφου <μὲν> ἀφιεμένου καὶ κάπρου καὶ ἄρκτου, ἡσυχίαν 
ἔχοντα κεῖσθαι καὶ περιορᾶν, ὀφθέντος δὲ λέοντος εὐ-
θὺς ἐξαναστῆναι καὶ διακονίεσθαι | καὶ φανερὸν εἶναι 
αὑτοῦ ποιούμενον ἀνταγωνιστήν, τῶν δ' ἄλλων ὑπερφρο-
νοῦντα πάντων. 



Plutarchus Biogr., Phil., De sollertia animalium (959a-985c) 
Stephanus page 975, section D, line 9

ἀλλὰ μὴ φοβηθῆτε· χρήσομαι γὰρ αὐτῇ μετρίως, οὔτε 
δόξας φιλοσόφων οὔτ' Αἰγυπτίων μύθους οὔτ' ἀμαρτύ-
ρους Ἰνδῶν ἐπαγόμενος ἢ Λιβύων διηγήσεις· ἃ δὲ παν-
ταχοῦ μάρτυρας ἔχει τοὺς ἐργαζομένους τὴν θάλατταν 
ὁρώμενα καὶ δίδωσι τῇ ὄψει πίστιν, τούτων ὀλίγα παρα-
θήσομαι. 



Plutarchus Biogr., Phil., Fragmenta (0007: 145)
“Plutarchi moralia, vol. 7”, Ed. Sandbach, F.H.
Leipzig: Teubner, 1967.
Fragment 9, line t

                                      ξhaeroneus πlut-
archus nostrarum esse partium comprobatur, qui in 
Oetaeis verticibus Herculem post morborum comitialium 
ruinas dissolutum in cinerem prodidit. 
      
<ΠΙΝΔ*αΡΟΥ ΒΙΟΣ>


 Eustathius, Prooem. Comm. Pindaricorum, c. 25. Ἐπι-
μεμέληται ὑπὸ τῶν παλαιῶν καὶ εἰς γένους ἀναγραφὴν τὴν 
κατά τε Πλούταρχον καὶ ἑτέρους, παρ' οἷς φέρεται ὅτι 
κώμη Θηβαίων οἱ Κυνοσκέφαλοι. 



Herodotus Hist., Historiae (0016: 001)
“Hérodote. Histoires, 9 vols.”, Ed. Legrand, Ph.–E.
Paris: Les Belles Lettres, 1:1932; 2;1930; 3:1939; 4 (3rd edn.): 1960; 5:1946; 6:1948; 7:1951; 8:1953; 9:1954, Repr. 1:1970; 2:1963; 3:1967; 5:1968; 6:1963; 7:1963; 8:1964; 9:1968.
Book 1, section 192, line 18

                                                    Κυνῶν δὲ 
Ἰνδικῶν τοσοῦτο δή τι πλῆθος ἐτρέφετο ὥστε τέσσερες 
τῶν ἐν τῷ πεδίῳ κῶμαι μεγάλαι, τῶν ἄλλων ἐοῦσαι ἀτελέες, 
τοῖσι κυσὶ προσετετάχατο σιτία παρέχειν. 



Herodotus Hist., Historiae 
Book 3, section 38, line 15

                               Δαρεῖος δὲ μετὰ ταῦτα καλέσας 
Ἰνδῶν τοὺς καλεομένους Καλλατίας, οἳ τοὺς γονέας κατ-
εσθίουσι, εἴρετο, παρεόντων τῶν Ἑλλήνων καὶ δι' ἑρμηνέος 
μανθανόντων τὰ λεγόμενα, ἐπὶ τίνι χρήματι δεξαίατ' ἂν 
τελευτῶντας τοὺς πατέρας κατακαίειν πυρί· οἱ δὲ ἀμβώ-
σαντες μέγα εὐφημέειν μιν ἐκέλευον. 



Herodotus Hist., Historiae 
Book 3, section 94, line 7

               Μόσχοισι δὲ καὶ Τιβαρηνοῖσι καὶ Μάκρωσι 
καὶ Μοσσυνοίκοισι καὶ Μαρσὶ τριηκόσια τάλαντα προεί-
ρητο· νομὸς εἴνατος καὶ δέκατος οὗτος Ἰνδῶν δὲ πλῆθός 
τε πολλῷ πλεῖστόν ἐστι πάντων τῶν ἡμεῖς ἴδμεν 
ἀνθρώπων καὶ φόρον ἀπαγίνεον πρὸς πάντας τοὺς ἄλλους 
ἑξήκοντα καὶ τριηκόσια τάλαντα ψήγματος· νομὸς εἰκοστὸς 
οὗτος. 



Herodotus Hist., Historiae 
Book 3, section 97, line 9

                                        .. οἳ περί τε Νύσην 
τὴν ἱρὴν κατοίκηνται καὶ τῷ Διονύσῳ ἀνάγουσι τὰς ὁρτάς· 
[οὗτοι οἱ Αἰθίοπες καὶ οἱ πλησιόχωροι τούτοισι σπέρματι 
μὲν χρέωνται τῷ αὐτῷ τῷ καὶ οἱ Καλλαντίαι Ἰνδοί, οἰκή-
ματα δὲ ἔκτηνται κατάγαια]· οὗτοι συναμφότεροι διὰ τρί-
του ἔτεος ἀγίνεον, ἀγινέουσι δὲ καὶ τὸ μέχρις ἐμέο, δύο   
χοίνικας ἀπύρου χρυσίου καὶ διηκοσίας φάλαγγας ἐβένου 
καὶ πέντε παῖδας Αἰθίοπας καὶ ἐλέφαντος ὀδόντας μεγά-
λους εἴκοσι. 



Herodotus Hist., Historiae 
Book 3, section 98, line 1

Τὸν δὲ χρυσὸν τοῦτον τὸν πολλὸν οἱ Ἰνδοί, ἀπ' οὗ τὸ 
ψῆγμα τῷ βασιλέϊ τὸ εἰρημένον κομίζουσι, τρόπῳ τοιῷδε 
κτῶνται. 



Herodotus Hist., Historiae 
Book 3, section 98, line 3

          Ἔστι τῆς Ἰνδικῆς χώρης τὸ πρὸς ἥλιον ἀνί-
σχοντα ψάμμος· τῶν γὰρ ἡμεῖς ἴδμεν, τῶν καὶ πέρι ἀτρεκές 
τι λέγεται, πρῶτοι πρὸς ἠῶ καὶ ἡλίου ἀνατολὰς οἰκέουσι 
ἀνθρώπων τῶν ἐν τῇ Ἀσίῃ Ἰνδοί· Ἰνδῶν γὰρ τὸ πρὸς τὴν 
ἠῶ ἐρημίη ἐστὶ διὰ τὴν ψάμμον. 



Herodotus Hist., Historiae 
Book 3, section 98, line 8

Ἔστι δὲ πολλὰ ἔθνεα Ἰνδῶν καὶ οὐκ ὁμόφωνα σφίσι, καὶ 
οἱ μὲν αὐτῶν νομάδες εἰσί, οἱ δὲ οὔ, οἱ δὲ ἐν τοῖσι ἕλεσι 
οἰκέουσι τοῦ ποταμοῦ καὶ ἰχθῦς σιτέονται ὠμούς, τοὺς 
αἱρέουσι ἐκ πλοίων καλαμίνων ὁρμώμενοι· καλάμου δὲ ἓν 
γόνυ πλοῖον ἕκαστον ποιέεται. 



Herodotus Hist., Historiae 
Book 3, section 98, line 12

                                   Οὗτοι μὲν δὴ τῶν Ἰνδῶν 
φορέουσι ἐσθῆτα φλοΐνην· ἐπεὰν ἐκ τοῦ ποταμοῦ φλοῦν   
ἀμήσωνται καὶ κόψωσι, τὸ ἐνθεῦτεν φορμοῦ τρόπον κα-
ταπλέξαντες ὡς θώρηκα ἐνδύνουσι. 



Herodotus Hist., Historiae 
Book 3, section 99, line 1

                                      Ἄλλοι δὲ τῶν Ἰνδῶν 
πρὸς ἠῶ οἰκέοντες τούτων νομάδες εἰσί, κρεῶν ἐδεσταὶ 
ὠμῶν, καλέονται δὲ Παδαῖοι. 



Herodotus Hist., Historiae 
Book 3, section 100, line 1

                                     Ἑτέρων δέ ἐστι Ἰνδῶν 
ὅδε ἄλλος τρόπος· οὔτε κτείνουσι οὐδὲν ἔμψυχον οὔτε τι 
σπείρουσι οὔτε οἰκίας νομίζουσι ἐκτῆσθαι ποιηφαγέουσί τε, 
καὶ αὐτοῖσι <ὄσπριόν τι> ἔστι ὅσον κέγχρος τὸ μέγαθος ἐν 
κάλυκι, αὐτόματον ἐκ τῆς γῆς γινόμενον, τὸ συλλέγοντες 
αὐτῇ τῇ κάλυκι ἕψουσί τε καὶ σιτέονται. 



Herodotus Hist., Historiae 
Book 3, section 101, line 2

                                                     Μίξις δὲ 
τούτων τῶν Ἰνδῶν τῶν κατέλεξα πάντων ἐμφανής ἐστι   
κατά περ τῶν προβάτων, καὶ τὸ χρῶμα φορέουσι ὅμοιον 
πάντες καὶ παραπλήσιον Αἰθίοψι. 



Herodotus Hist., Historiae 
Book 3, section 101, line 7

                                        Οὗτοι μὲν τῶν Ἰνδῶν 
ἑκαστέρω τῶν Περσέων οἰκέουσι καὶ πρὸς νότου ἀνέμου 
καὶ Δαρείου βασιλέος οὐδαμὰ ὑπήκουσαν. 



Herodotus Hist., Historiae 
Book 3, section 102, line 2

                                               Ἄλλοι δὲ τῶν 
Ἰνδῶν Κασπατύρῳ τε πόλι καὶ τῇ Πακτυϊκῇ χώρῃ εἰσὶ 
πρόσοικοι, πρὸς ἄρκτου τε καὶ βορέω ἀνέμου κατοικημένοι 
τῶν ἄλλων Ἰνδῶν, οἳ Βακτρίοισι παραπλησίην ἔχουσι δίαι-
ταν. 



Herodotus Hist., Historiae 
Book 3, section 102, line 5

     Οὗτοι καὶ μαχιμώτατοί εἰσι Ἰνδῶν καὶ οἱ ἐπὶ τὸν 
χρυσὸν στελλόμενοί εἰσι οὗτοι· κατὰ γὰρ τοῦτό ἐστι ἐρημίη 
διὰ τὴν ψάμμον. 



Herodotus Hist., Historiae 
Book 3, section 102, line 16

           Ἐπὶ δὴ ταύτην τὴν ψάμμον στέλλονται ἐς τὴν 
ἔρημον οἱ Ἰνδοί, ζευξάμενος ἕκαστος καμήλους τρεῖς, σειρη-  
φόρον μὲν ἑκατέρωθεν ἔρσενα παρέλκειν, θήλεαν δὲ ἐς 
μέσον· ἐπὶ ταύτην δὴ αὐτὸς ἀναβαίνει, ἐπιτηδεύσας ὅκως 
ἀπὸ τέκνων ὡς νεωτάτων ἀποσπάσας ζεύξει· αἱ γάρ σφι 
κάμηλοι ἵππων οὐκ ἥσσονες ἐς ταχυτῆτά εἰσι· χωρὶς δὲ 
ἄχθεα δυνατώτεραι πολλὸν φέρειν. 



Herodotus Hist., Historiae 
Book 3, section 104, line 1

                          Οἱ δὲ δὴ Ἰνδοὶ τρόπῳ τοιούτῳ καὶ 
ζεύξι τοιαύτῃ χρεώμενοι ἐλαύνουσι ἐπὶ τὸν χρυσὸν λελογις-
μένως ὅκως [ἂν] καυμάτων τῶν θερμοτάτων ἐόντων ἔσον-
ται ἐν τῇ ἁρπαγῇ· ὑπὸ γὰρ τοῦ καύματος οἱ μύρμηκες 
ἀφανέες γίνονται ὑπὸ γῆν. 



Herodotus Hist., Historiae 
Book 3, section 104, line 12

                                Θερμότατος δέ ἐστι ὁ ἥλιος 
τούτοισι τοῖσι ἀνθρώποισι τὸ ἑωθινόν, οὐ κατά περ τοῖσι 
ἄλλοισι μεσαμβρίης, ἀλλ' ὑπερτείλας μέχρις οὗ ἀγορῆς δια-
λύσιος· τοῦτον δὲ τὸν χρόνον καίει πολλῷ μᾶλλον ἢ τῇ 
μεσαμβρίῃ τὴν Ἑλλάδα, οὕτω ὥστε ἐν ὕδατι λόγος αὐτούς 
ἐστι βρέχεσθαι τηνικαῦτα· μεσοῦσα δὲ ἡ ἡμέρη σχεδὸν 
παραπλησίως καίει τούς <τε> ἄλλους ἀνθρώπους καὶ τοὺς 
Ἰνδούς· ἀποκλινομένης δὲ τῆς μεσαμβρίης γίνεταί σφι ὁ   
ἥλιος κατά περ τοῖσι ἄλλοισι ὁ ἑωθινός· καὶ τὸ ἀπὸ τούτου 
ἀπιὼν ἐπὶ μᾶλλον ψύχει, ἐς ὃ ἐπὶ δυσμῇσι ἐὼν καὶ τὸ 
κάρτα ψύχει. 



Herodotus Hist., Historiae 
Book 3, section 105, line 1

               Ἐπεὰν δὲ ἔλθωσι ἐς τὸν χῶρον οἱ Ἰνδοὶ 
ἔχοντες θυλάκια, ἐμπλήσαντες ταῦτα τῆς ψάμμου τὴν 
ταχίστην ἐλαύνουσι ὀπίσω· αὐτίκα γὰρ οἱ μύρμηκες ὀδμῇ, 
ὡς δὴ λέγεται ὑπὸ Περσέων, μαθόντες διώκουσι. 



Herodotus Hist., Historiae 
Book 3, section 105, line 6

                                                       Εἶναι δὲ 
ταχυτῆτα οὐδενὶ ἑτέρῳ ὅμοιον, οὕτω ὥστε, εἰ μὴ προλαμ-
βάνειν τοὺς Ἰνδοὺς τῆς ὁδοῦ ἐν ᾧ τοὺς μύρμηκας συλλέγε-
σθαι, οὐδένα ἄν σφεων ἀποσῴζεσθαι. 



Herodotus Hist., Historiae 
Book 3, section 105, line 12

                  Τὸν μὲν δὴ πλέω τοῦ χρυσοῦ οὕτω οἱ 
Ἰνδοὶ κτῶνται, ὡς Πέρσαι φασί· ἄλλος δὲ σπανιώτερός 
ἐστι ἐν τῇ χώρῃ ὀρυσσόμενος. 



Herodotus Hist., Historiae 
Book 3, section 106, line 4

             Τοῦτο μὲν γὰρ πρὸς τὴν ἠῶ ἐσχάτη τῶν οἰκεο-
μένων ἡ Ἰνδική ἐστι, ὥσπερ ὀλίγῳ πρότερον εἴρηκα· ἐν 
ταύτῃ τοῦτο μὲν τὰ ἔμψυχα, <τὰ> τετράποδά τε καὶ τὰ 
πετεινά, πολλῷ μέζω ἢ ἐν τοῖσι ἄλλοισι χωρίοισί ἐστι, πά-
ρεξ τῶν ἵππων (οὗτοι δὲ ἑσσοῦνται ὑπὸ τῶν Μηδικῶν,   
Νησαίων δὲ καλεομένων ἵππων), τοῦτο δὲ χρυσὸς ἄπλετος 
αὐτόθι ἐστί, ὁ μὲν ὀρυσσόμενος, ὁ δὲ καταφορεόμενος ὑπὸ 
[τῶν] ποταμῶν, ὁ δὲ ὥσπερ ἐσήμηνα ἁρπαζόμενος. 



Herodotus Hist., Historiae 
Book 3, section 106, line 12

                                                         Τὰ δὲ 
δένδρεα τὰ ἄγρια αὐτόθι φέρει καρπὸν εἴρια καλλονῇ τε 
προφέροντα καὶ ἀρετῇ τῶν ἀπὸ τῶν ὀΐων· καὶ ἐσθῆτι Ἰνδοὶ 
ἀπὸ τούτων τῶν δενδρέων χρέωνται. 



Herodotus Hist., Historiae 
Book 4, section 40, line 7

       Μέχρι δὲ τῆς Ἰνδικῆς οἰκέεται [ἡ] Ἀσίη· τὸ δὲ   
ἀπὸ ταύτης ἔρημος ἤδη τὸ πρὸς τὴν ἠῶ, οὐδὲ ἔχει οὐδεὶς 
φράσαι οἷον δή τι ἐστί. 



Herodotus Hist., Historiae 
Book 4, section 44, line 2

Τῆς δὲ Ἀσίης τὰ πολλὰ ὑπὸ Δαρείου ἐξευρέθη, ὃς 
βουλόμενος Ἰνδὸν ποταμόν, ὃς κροκοδείλους δεύτερος 
οὗτος ποταμῶν πάντων παρέχεται, τοῦτον τὸν ποταμὸν 
εἰδέναι τῇ ἐς θάλασσαν ἐκδιδοῖ, πέμπει πλοίοισι ἄλλους τε   
τοῖσι ἐπίστευε τὴν ἀληθείην ἐρέειν καὶ δὴ καὶ Σκύλακα 
ἄνδρα Καρυανδέα. 



Herodotus Hist., Historiae 
Book 4, section 44, line 12

                                                         Μετὰ 
δὲ τούτους περιπλώσαντας Ἰνδούς τε κατεστρέψατο 
Δαρεῖος καὶ τῇ θαλάσσῃ ταύτῃ ἐχρᾶτο. 



Herodotus Hist., Historiae 
Book 5, section 3, line 1

Θρηίκων δὲ ἔθνος μέγιστόν ἐστι μετά γε Ἰνδοὺς πάντων 
ἀνθρώπων· εἰ δὲ ὑπ' ἑνὸς ἄρχοιτο ἢ φρονέοι κατὰ τὠυτό, 
ἄμαχόν τ' ἂν εἴη καὶ πολλῷ κράτιστον πάντων ἐθνέων κατὰ 
γνώμην τὴν ἐμήν· ἀλλὰ γὰρ τοῦτο ἄπορόν σφι καὶ ἀμήχανον 
μή κοτε ἐγγένηται· εἰσὶ δὴ κατὰ τοῦτο ἀσθενέες. 



Herodotus Hist., Historiae 
Book 7, section 9, line 6

                                                   Καὶ γὰρ δεινὸν 
ἂν εἴη πρῆγμα, εἰ Σάκας μὲν καὶ Ἰνδοὺς καὶ Αἰθίοπάς τε 
καὶ Ἀσσυρίους ἄλλα τε ἔθνεα πολλὰ καὶ μεγάλα, ἀδικήσαντα 
Πέρσας οὐδέν, ἀλλὰ δύναμιν προσκτᾶσθαι βουλόμενοι, 
καταστρεψάμενοι δούλους ἔχομεν, Ἕλληνας δὲ ὑπάρξαντας 
ἀδικίης οὐ τιμωρησόμεθα. 



Herodotus Hist., Historiae 
Book 7, section 65, line 1

                                                   Βακτρίων 
δὲ καὶ Σακέων ἦρχε Ὑστάσπης ὁ Δαρείου τε καὶ Ἀτόσσης 
τῆς Κύρου. Ἰνδοὶ δὲ εἵματα μὲν ἐνδεδυκότες ἀπὸ ξύλων 
πεποιημένα, τόξα δὲ καλάμινα εἶχον καὶ ὀϊστοὺς καλαμί-
νους· ἐπὶ δὲ σίδηρος ἦν· ἐσταλμένοι μὲν δὴ ἦσαν οὕτω 
Ἰνδοί, προσετετάχατο δὲ συστρατευόμενοι Φαρναζάθρῃ τῷ 
Ἀρταβάτεω. 



Herodotus Hist., Historiae 
Book 7, section 65, line 4

              Ἰνδοὶ δὲ εἵματα μὲν ἐνδεδυκότες ἀπὸ ξύλων 
πεποιημένα, τόξα δὲ καλάμινα εἶχον καὶ ὀϊστοὺς καλαμί-
νους· ἐπὶ δὲ σίδηρος ἦν· ἐσταλμένοι μὲν δὴ ἦσαν οὕτω 
Ἰνδοί, προσετετάχατο δὲ συστρατευόμενοι Φαρναζάθρῃ τῷ 
Ἀρταβάτεω. 



Herodotus Hist., Historiae 
Book 7, section 70, line 3

             Τῶν μὲν δὴ ὑπὲρ Αἰγύπτου Αἰθιόπων καὶ 
Ἀραβίων ἦρχε Ἀρσάμης, οἱ δὲ ἀπὸ ἡλίου ἀνατολέων 
Αἰθίοπες (διξοὶ γὰρ δὴ ἐστρατεύοντο) προσετετάχατο τοῖσι 
Ἰνδοῖσι, διαλλάσσοντες εἶδος μὲν οὐδὲν τοῖσι ἑτέροισι, 
φωνὴν δὲ καὶ τρίχωμα μοῦνον· οἱ μὲν γὰρ ἀπὸ ἡλίου Αἰ-
θίοπες ἰθύτριχές εἰσι, οἱ δ' ἐκ τῆς Λιβύης οὐλότατον τρί-
χωμα ἔχουσι πάντων ἀνθρώπων. 



Herodotus Hist., Historiae 
Book 7, section 70, line 7

                                  Οὗτοι δὲ οἱ ἐκ τῆς Ἀσίης 
Αἰθίοπες τὰ μὲν πλέω κατά περ Ἰνδοὶ ἐσεσάχατο, προμε-
τωπίδια δὲ ἵππων εἶχον ἐπὶ τῇσι κεφαλῇσι σύν τε τοῖσι 
ὠσὶ ἐκδεδαρμένα καὶ τῇ λοφιῇ· καὶ ἀντὶ μὲν λόφου ἡ λοφιὴ 
κατέχρα, τὰ δὲ ὦτα τῶν ἵππων ὀρθὰ πεπηγότα εἶχον· προ-
βλήματα δὲ ἀντ' ἀσπίδων ἐποιεῦντο γεράνων δοράς. 



Herodotus Hist., Historiae 
Book 7, section 86, line 2

                                                 Μῆδοι δὲ τήν 
περ ἐν τῷ πεζῷ εἶχον σκευήν· καὶ Κίσσιοι ὡσαύτως. Ἰνδοὶ 
δὲ σκευῇ μὲν ἐσεσάχατο τῇ αὐτῇ καὶ ἐν τῷ πεζῷ, ἤλαυνον 
δὲ κέλητας καὶ ἅρματα· ὑπὸ δὲ τοῖσι ἅρμασι ὑπῆσαν ἵπποι 
καὶ ὄνοι ἄγριοι. 



Herodotus Hist., Historiae 
Book 7, section 187, line 5

           Γυναικῶν δὲ σιτοποιῶν καὶ παλλακέων καὶ 
εὐνούχων οὐδεὶς ἂν εἴποι ἀτρεκέα ἀριθμόν· οὐδ' αὖ ὑποζυ-
γίων τε καὶ τῶν ἄλλων κτηνέων τῶν ἀχθοφόρων καὶ κυνῶν 
Ἰνδικῶν τῶν ἑπομένων, οὐδ' ἂν τούτων ὑπὸ πλήθεος οὐδεὶς 
ἂν εἴποι ἀριθμόν. 



Herodotus Hist., Historiae 
Book 8, section 113, line 12

           Ὡς δὲ ἀπίκατο ἐς τὴν Θεσσαλίην, ἐνθαῦτα 
Μαρδόνιος ἐξελέγετο πρώτους μὲν τοὺς μυρίους Πέρσας 
τοὺς Ἀθανάτους καλεομένους, πλὴν Ὑδάρνεος τοῦ στρατη-
γοῦ (οὗτος γὰρ οὐκ ἔφη λείψεσθαι βασιλέος), μετὰ δὲ τῶν 
ἄλλων Περσέων τοὺς θωρηκοφόρους καὶ τὴν ἵππον τὴν 
χιλίην, καὶ Μήδους τε καὶ Σάκας καὶ Βακτρίους [τε] καὶ 
Ἰνδούς, καὶ τὸν πεζὸν καὶ τὴν ἵππον. 



Herodotus Hist., Historiae 
Book 9, section 31, line 19

                                   Μετὰ δὲ Βακτρίους 
ἔστησε Ἰνδούς· οὗτοι δὲ ἐπέσχον Ἑρμιονέας τε καὶ Ἐρε-
τριέας καὶ Στυρέας τε καὶ Χαλκιδέας. 



Joannes Scylitzes Hist., Synopsis historiarum (3063: 001)
“Ioannis Scylitzae synopsis historiarum”, Ed. Thurn, J.
Berlin: De Gruyter, 1973; Corpus fontium historiae Byzantinae 5. Series Berolinensis.
Emperor life Leo5, section 2, line 5

      ἐξαπέστειλε γοῦν τινα τῶν οἱ πιστοτάτων, ἀναθήματά τε ὡς 
αὐτὸν κομίζοντα καὶ ἔπιπλα καὶ σκεύη ἀργύρεα καὶ χρύσεα καὶ εἴδη 
εὐώδη τῶν εἰς ἡμᾶς ἐξ Ἰνδίας κομιζομένων. 



Joannes Scylitzes Hist., Synopsis historiarum 
Emperor life Mich2, section 6, line 16

                                                       χεῖρα δὲ πολλὴν αὐτός   
τε συλλέγει καὶ παρὰ τῶν Ἀγαρηνῶν λαμβάνει, οὐ μόνον δὴ τῶν προς-
οίκων ἡμῖν, ἀλλὰ καὶ τῶν ἐκ τῆς περαίας Αἰγυπτίων, Ἰνδῶν, Περσῶν, 
Ἀσσυρίων, Ἀρμενίων, Χάλδων, Ἰβήρων, Ζεχῶν καὶ Καβείρων. 



Joannes Scylitzes Hist., Synopsis historiarum 
Emperor life Theoph, section 1, line 3

ΘΕΟΦΙΛΟΣ


Μετὰ δὲ τὸν τοῦ Μιχαὴλ θάνατον ὁ υἱὸς αὐτοῦ Θεόφιλος, ἀνδρὸς 
ἤδη ἡλικίαν ἔχων, διεδέξατο τὴν πατρῴαν ἀρχὴν κατὰ τὸν Ὀκτώβριον 
μῆνα τῆς ὀγδόης ἰνδικτιῶνος, λόγῳ μὲν τῆς δικαιοσύνης ἔμπυρος ἐραστὴς 
καλεῖσθαι βουλόμενος νόμων τε φύλαξ πολιτικῶν ἀκριβής, τῇ δ' ἀλη-
θείᾳ ἔξωθεν ἑαυτὸν τῶν ἐπιβουλευόντων διατηρῶν, ὡς ἂν μή τις κατ' 
αὐτοῦ νεανικόν τι τολμήσειε, ταῦτα ὑπεκρίνετο. 



Joannes Scylitzes Hist., Synopsis historiarum 
Emperor life Theoph, section 15, line 14

                                                                 ὡς δ' οὐκ ἐκ μόνων 
ἰνδαλμάτων, ἀλλὰ καὶ τῶν τῆς ψυχῆς καὶ σώματος γνωρισμάτων ὁ 
ζητούμενος ἐδηλοῦτό τε καὶ ἐγνωρίζετο, προσεμαρτύρει δὲ καί τις τῶν 
ἐκ γειτόνων τὴν γενομένην τῇ γυναικὶ πρὸς τὸν Πέρσην συνάφειαν (οὐ 
γάρ τι κρυπτόν, ὃ τοῖς πολλοῖς οὐ γνωσθήσεται), δήλους ἑαυτοὺς οἱ 
σταλέντες καθιστᾶσι τῷ βασιλεῖ καὶ τὰ τοῦ σκοποῦ σαφηνίζουσιν, 
εἰρήνην καὶ σπονδὰς καὶ παντὸς τοῦ ἔθνους ὑποταγὴν ὑπισχνούμενοι, 
εἰ τὸν Θεόφοβον αὐτοῖς ἀποδοῦναι οὐ παραιτήσεται. 



Joannes Scylitzes Hist., Synopsis historiarum 
Emperor life Const7, section 8, line 24

                                                   κατὰ δὲ τὴν ἕκτην τοῦ 
Αὐγούστου μηνός, τῆς πέμπτης ἰνδικτιῶνος, πολέμου συρραγέντος 
Ῥωμαίοις τε καὶ Βουλγάροις πρὸς τῷ Ἀχελῴῳ φρουρίῳ τρέπονται κατὰ 
κράτος οἱ Βούλγαροι, καὶ φόνος αὐτῶν ἐγένετο πολύς. 



Joannes Scylitzes Hist., Synopsis historiarum 
Emperor life Roman1, section 2, line 1

Ἰουλίῳ δὲ μηνὶ ἰνδικτιῶνος ὀγδόης ἡ τῆς ἐκκλησίας γέγονεν 
ἕνωσις, ἑνωθέντων τῶν διαφερομένων μητροπολιτῶν τε καὶ κληρικῶν 
τῶν ἀπὸ Νικολάου πατριάρχου καὶ Εὐθυμίου διεσχισμένων. 



Joannes Scylitzes Hist., Synopsis historiarum 
Emperor life Roman1, section 7, line 1

Εἰκάδι δὲ Φεβρουραρίου μηνός, ἰνδικτιῶνος δεκάτης, θνῄσκει 
Θεοδώρα ἡ σύμβιος Ῥωμανοῦ καὶ θάπτεται ἐν τῷ Μυρελαίῳ. 



Joannes Scylitzes Hist., Synopsis historiarum 
Emperor life Roman1, section 12, line 1

Σεπτεμβρίῳ δὲ μηνί, ἰνδικτιῶνος δευτέρας, ὁ ἄρχων Βουλγαρίας 
Συμεὼν πανστρατὶ κατὰ Κωνσταντινουπόλεως ἐκστρατεύει καὶ ληΐζεται 
μὲν Μακεδονίαν, ἐμπιπρᾷ δὲ τὰ ἐπὶ Θρᾴκης χωρία, καὶ πάντα καταστρέ-  
φει τὰ ἐν ποσίν. 



Joannes Scylitzes Hist., Synopsis historiarum 
Emperor life Roman1, section 13, line 1

Κατὰ δὲ τὴν ἑορτὴν τῶν Χριστουγέννων, ἰνδικτιῶνος δευτέρας, 
Ῥωμανὸς ὁ βασιλεὺς στέφει τοὺς δύο υἱοὺς αὐτοῦ, Στέφανον καὶ Κωνσταν-
τῖνον ἐν τῇ μεγάλῃ ἐκκλησίᾳ. 



Joannes Scylitzes Hist., Synopsis historiarum 
Emperor life Roman1, section 14, line 1

Πεντεκαιδεκάτῃ δὲ Μαΐου μηνός, ἰνδικτιῶνος τρίτης, τελευτᾷ 
ὁ πατριάρχης Νικόλαος, κρατήσας ἐν τῇ δευτέρᾳ ἀναρρήσει ἔτη τρισκαί-
δεκα. 



Joannes Scylitzes Hist., Synopsis historiarum 
Emperor life Roman1, section 16, line 1

Μαΐῳ δὲ μηνί, ἰνδικτιῶνος πεντεκαιδεκάτης, εἰσβολὴν Συμεὼν ὁ 
τῆς Βουλγαρίας ἄρχων ἐποιήσατο κατὰ Χρωβάτων, καὶ συμβαλὼν μετ' 
αὐτῶν καὶ ἡττηθεὶς ἐν ταῖς τῶν ὀρῶν δυσχωρίαις ἅπαν τὸ ἑαυτοῦ 
ἀπώλεσε στράτευμα. 



Joannes Scylitzes Hist., Synopsis historiarum 
Emperor life Roman1, section 21, line 1

Μηνὶ δὲ Ἰουλίῳ πεντεκαιδεκάτῃ, ἰνδικτιῶνος ἕκτης, ἐτελεύτησεν 
ὁ Ἀμασείας Στέφανος, πατριαρχήσας ἔτη δύο καὶ μῆνας ἕνδεκα. 



Joannes Scylitzes Hist., Synopsis historiarum 
Emperor life Roman1, section 25, line 2

Ἐτελεύτησε δὲ καὶ Χριστοφόρος ὁ βασιλεύς, μηνὶ Αὐγούστῳ, 
ἰνδικτιῶνος τετάρτης, καὶ ἐτάφη ἐν τῇ μονῇ τοῦ πατρὸς αὐτοῦ. 



Joannes Scylitzes Hist., Synopsis historiarum 
Emperor life Roman1, section 26, line 36

                                                       καὶ μετὰ χρόνον ἕνα 
καὶ μῆνας πέντε (τοσοῦτον γὰρ ὁ τῆς ἡλικίας τοῦ Θεοφυλάκτου ἐνέδει 
χρόνος πρὸς τελειότητα καὶ χειροθεσίαν ἀρχιερωσύνης) Φεβρουαρίῳ, 
δευτέρας ἰνδικτιῶνος, χειροτονεῖται πατριάρχης Θεοφύλακτος ὁ τοῦ 
βασιλέως υἱός. 



Joannes Scylitzes Hist., Synopsis historiarum 
Emperor life Roman1, section 29, line 2

Ἐγένετο δὲ καὶ εἰσβολὴ Τούρκων κατὰ Ῥωμαίων Ἀπριλλίῳ 
μηνί, ἰνδικτιῶνος ἑβδόμης, καὶ κατέδραμον πᾶσαν τὴν δύσιν μέχρι τῆς 
πόλεως. 



Joannes Scylitzes Hist., Synopsis historiarum 
Emperor life Roman1, section 31, line 1

Δεκάτῃ δὲ καὶ τετάρτῃ ἰνδικτιῶνι, Ἰουνίῳ μηνί, ἐπέλευσις κατὰ 
τῆς πόλεως ἐγένετο Ῥωσικοῦ στόλου πλοίων χιλιάδων δέκα. 



Joannes Scylitzes Hist., Synopsis historiarum 
Emperor life Roman1, section 34, line 1

Κατὰ δὲ τὴν πρώτην ἰνδικτιῶνα τῶν Τούρκων πάλιν ἐπιδρομὴν 
ποιησαμένων κατὰ Ῥωμαίων, ὁ παρακοιμώμενος Θεοφάνης ἐξελθὼν 
ἐσπείσατο μετ' αὐτῶν καὶ λαβὼν ὁμήρους ὑπέστρεψε. 



Joannes Scylitzes Hist., Synopsis historiarum 
Emperor life Roman1, section 35, line 1

Δευτέρᾳ δὲ ἰνδικτιῶνι Πασχάλιον πρωτοσπαθάριον καὶ στρατη-
γὸν Λογγιβαρδίας ἐξέπεμψεν ὁ βασιλεὺς πρὸς τὸν ῥῆγα Φραγγίας 
Οὔγωνα, τὴν αὐτοῦ θυγατέρα ἐπιζητῶν νυμφευθῆναι τῷ τοῦ Πορφυρο-
γεννήτου υἱῷ Ῥωμανῷ. 



Joannes Scylitzes Hist., Synopsis historiarum 
Emperor life Roman1, section 39, line 5

                                                              τῇ δὲ αὐτῇ ἰνδικτιῶνι 
κατήγαγον τὸν βασιλέα Ῥωμανὸν τοῦ παλατίου καὶ εἰς τὴν Πρώτην 
ἀγαγόντες νῆσον ἀπέκειραν μοναχόν. 



Joannes Scylitzes Hist., Synopsis historiarum 
Emperor life Const7 iterum, section 1, line 73

              προσεταιρισάμενος οὖν σὺν τῷ ῥηθέντι Βασιλείῳ καὶ τὸν 
μοναχὸν Μαριανὸν τὸν υἱὸν Λέοντος τοῦ Ἀργυροῦ, ὑπὸ τοῦ βασιλέως 
Ῥωμανοῦ λίαν καὶ τιμώμενον καὶ πιστευόμενον, καί τινας ἄλλους σὺν 
αὐτοῖς, εὐκαιρήσας κατασπᾷ τῆς ἀρχῆς τὸν αὐτοῦ πατέρα, μηνὶ Δεκεμ-
βρίῳ ἑξκαιδεκάτῃ, ἰνδικτιῶνος τρίτης, ἔτους ͵ϛυνγʹ, εἰκοστὸν ἕκτον 
ἀνύοντα ἐν τῇ βασιλείᾳ ἐνιαυτόν, καὶ τῇ νήσῳ Πρώτῃ περιορίζει, 
ἀποκείρας καὶ ἄκοντα τοῦτον μοναχόν. 



Joannes Scylitzes Hist., Synopsis historiarum 
Emperor life Const7 iterum, section 2, line 20

                                                        ἐκφήνας οὖν τὸ μυστή-
ριον τῷ εἰρημένῳ Βασιλείῳ τῷ Πετεινῷ, καὶ δι' αὐτοῦ προσκτησάμενος 
τὸν Μαριανόν, ἔτι δὲ Νικηφόρον καὶ Λέοντα τοὺς υἱοὺς Βάρδα τοῦ 
Φωκᾶ, Νικόλαόν τε καὶ Λέοντα τοὺς Τορνικίους, καὶ ἄλλους οὐκ ὀλίγους, 
μηδὲν ὑφορωμένους τὸν Στέφανον καὶ τὸν Κωνσταντῖνον, κατ' αὐτὸν 
τὸν καιρὸν τοῦ ἀρίστου συναριστοῦντας αὐτῷ, ἀναρπάστους τίθησι 
καὶ καταβιβάζει τῶν βασιλείων, τῇ εἰκάδι ἑβδόμῃ τοῦ Ἰαννουαρίου 
μηνός, τῆς αὐτῆς τρίτης ἰνδικτιῶνος, καὶ πλοιαρίοις ἐνθέμενος ὑπερο-
ρίζει τὸν μὲν ἐν τῇ Πανόρμῳ νήσῳ, τὸν Κωνσταντῖνον δὲ ἐν τῇ Τερε-
βίνθῳ. 



Joannes Scylitzes Hist., Synopsis historiarum 
Emperor life Const7 iterum, section 2, line 31

μηνὶ δὲ Ἰουλίῳ τῆς ἕκτης ἰνδικτιῶνος καὶ Ῥωμανὸς ὁ τούτων πατὴρ 
ἀπέτισε τὸ χρεών, καὶ ἐν τῷ Μυρελαίῳ θάπτεται. 



Joannes Scylitzes Hist., Synopsis historiarum 
Emperor life Const7 iterum, section 3, line 3

Ὁ δὲ Πορφυρογέννητος τὰ ὕποπτα περιελὼν ἐκ μέσου, καὶ 
μόνος τὴν αὐτοκράτορα περιζωσάμενος ἀρχήν, κατὰ τὸ θεῖον πάσχα 
τῆς αὐτῆς ἰνδικτιῶνος καὶ τῷ υἱῷ Ῥωμανῷ περιτίθησι διάδημα, τελέσαν-
τος δὴ τὰς εὐχὰς Θεοφυλάκτου τοῦ πατριάρχου. 



Joannes Scylitzes Hist., Synopsis historiarum 
Emperor life Const7 iterum, section 10, line 3

Ἔτει δὲ δωδεκάτῳ τῆς Κωνσταντίνου βασιλείας, τοῦ δὲ κόσμου 
ἑξακισχιλιοστῷ τετρακοσιοστῷ ἑξηκοστῷ τετάρτῳ, μηνὶ Φεβρουαρίῳ 
εἰκάδι ἑβδόμῃ, ἰνδικτιῶνος τεσσαρεσκαιδεκάτης, κατέλυσε τὸν βίον 
Θεοφύλακτος ὁ πατριάρχης, ἀρχιερατεύσας ἐπ' ἔτη εἴκοσι καὶ τρία, 
ἡμέρας εἰκοσιπέντε, ἑξκαίδεκα μὲν ἐτῶν ὤν, ὅτε τοὺς τῆς ἐκκλησίας ἀκα-
νονίστως παρείληφεν οἴακας, ὑπὸ παιδαγωγοὺς δέ, φεῦ μοι, ὁ ἀρχιερεὺς 
μέχρι τινὸς διετέλεσε. 



Joannes Scylitzes Hist., Synopsis historiarum 
Emperor life Const7 iterum, section 11, line 2

Καὶ χειροτονεῖται κατὰ τὴν τρίτην τοῦ Ἀπριλλίου μηνός, τῆς 
αὐτῆς ἰνδικτιῶνος, ἀντ' αὐτοῦ πατριάρχης Πολύευκτος μοναχός, τῆς 
Κωνσταντίνου καὶ θρέμμα τυγχάνων καὶ παίδευμα, ὑπὸ τῶν γονέων 
μὲν εὐνουχισθείς, ἐπὶ πολὺν δὲ χρόνον τῇ μοναδικῇ πολιτείᾳ ἐνδιαπρέ-
ψας. 



Joannes Scylitzes Hist., Synopsis historiarum 
Emperor life Const7 iterum, section 17, line 2

Πεντεκαιδεκάτῳ δὲ χρόνῳ τῆς αὐτοῦ βασιλείας, κατὰ τὸν 
Σεπτέμβριον μῆνα τῆς τρίτης ἰνδικτιῶνος, ἐν ἔτει κοσμικῷ ἑξακισχιλιοστῷ   
τετρακοσιοστῷ ἑξηκοστῷ ὀγδόῳ, ἔξεισι Κωνσταντῖνος ὁ βασιλεὺς ἐν 
τῷ τοῦ Ὀλύμπου ὄρει, τῷ μὲν δοκεῖν ταῖς τῶν ἐκεῖσε πατέρων εὐχαῖς 
θωρακισθῆναι καὶ μετ' αὐτῶν κατὰ Σαρακηνῶν ἐν Συρίᾳ στρατεῦσαι, 
ἀληθεῖ δὲ λόγῳ ἑνωθῆναι Θεοδώρῳ τῷ τῆς Κυζίκου προεδρεύοντι, 
ἐκεῖσε τότε τὰς διατριβὰς ποιουμένῳ, καὶ μετ' αὐτοῦ περὶ τῆς καθαιρέσεως 
τοῦ Πολυεύκτου βουλεύσασθαι. 



Joannes Scylitzes Hist., Synopsis historiarum 
Emperor life Roman2, section 1, line 5

καὶ ἄρχοντας προβαλλόμενος εὐνοϊκοὺς αὐτῷ καὶ θυμήρεις, καὶ τὴν 
βασιλείαν, ὡς ἐνῆν, κρατυνάμενος, κατὰ τὴν ἑορτὴν τοῦ πάσχα τῆς αὐτῆς 
τρίτης ἰνδικτιῶνος στέφει καὶ Βασίλειον τὸν υἱὸν αὐτοῦ διὰ τῶν χειρῶν 
Πολυεύκτου τοῦ πατριάρχου ἐν τῇ μεγάλῃ ἐκκλησίᾳ. 



Joannes Scylitzes Hist., Synopsis historiarum 
Emperor life Roman2, section 4, line 16

              τῇ ἑβδόμῃ δὲ τοῦ Μαρτίου μηνός, τῆς τετάρτης ἰνδικτιῶνος, 
καὶ τὴν πασῶν ὀχυρωτέραν πόλιν, ἣν ἐγχωρίως Χάνδακα ἐκάλουν, 
πεπορθηκώς, καὶ τὸν ἀμηρεύοντα τῆς νήσου Κουρούπην ὄνομα λαβὼν   
αἰχμάλωτον καὶ Ἀνεμᾶν τὸν μετ' αὐτὸν ἐν τῇ νήσῳ πρωτεύοντα, καὶ τὴν 
νῆσον ὅλην δουλωσάμενος, ἔμελλε μὲν ἐπὶ πλείονα προσμεῖναι χρόνον καὶ 
τὰ κατ' αὐτὴν καταστήσεσθαι, φήμης δὲ κρατούσης, ὡς ὁ μέλλων 
κατασχεῖν αὐτὴν Ῥωμαῖος ἀνὴρ ἐξ ἀνάγκης βασιλεύσει Ῥωμαίων, ἅμα 
τῷ γνωσθῆναι τὴν τῆς νήσου κατάσχεσιν ταῖς τοῦ Ἰωσὴφ ὑποθήκαις 
ἀποστείλας ὁ Ῥωμανὸς προσεκαλέσατο τὸν Νικηφόρον ἐκεῖθεν. 



Joannes Scylitzes Hist., Synopsis historiarum 
Emperor life Roman2, section 9, line 10

                              ἡ δὲ Ἑλένη τῇ τῶν θυγατέρων καταγωγῇ 
περιαλγήσασα, καὶ μικρὸν ἐπιζήσασα χρόνον, τῇ εἰκοστῇ τοῦ Σεπτεμ-
βρίου μηνὸς τῆς πέμπτης ἰνδικτιῶνος ἀπεβίω, καὶ βασιλικῶς ἐκκομι-
σθεῖσα ἐτάφη ἐν τῇ λάρνακι τοῦ πατρός. 



Joannes Scylitzes Hist., Synopsis historiarum 
Emperor life Roman2, section 11, line 1

Πεντεκαιδεκάτῃ δὲ Μαρτίου μηνός, τῆς ἕκτης ἰνδικτιῶνος, ἐν 
ἔτει ἑξακισχιλιοστῷ τετρακοσιοστῷ ἑβδομηκοστῷ πρώτῳ ἐτελεύτησε 
Ῥωμανὸς ὁ βασιλεύς, ἐτῶν ὑπάρχων εἰκοσιτεσσάρων, βασιλεύσας ἔτη 
τρισκαίδεκα, μῆνας τέσσαρας, καὶ ἡμέρας πέντε, ὡς μέν τινες, προκατανα-
λώσας τὸ ἑαυτοῦ σαρκίον ταῖς αἰσχίσταις καὶ φιληδόνοις πράξεσιν, ὡς 
δ' ἕτερος ἔχει λόγος, φαρμάκοις ἀναιρεθείς. 



Joannes Scylitzes Hist., Synopsis historiarum 
Emperor life Bas+Const, section 3, line 1

Ἀπριλλίῳ δὲ μηνί, τῆς αὐτῆς ἕκτης ἰνδικτιῶνος, εἴσεισιν ὁ Φωκᾶς 
Νικηφόρος κελεύσει τῆς δεσποίνης, τοῦ Ἰωσὴφ καθάπαξ καὶ πάλιν 
κωλύοντος, ἐν Κωνσταντινουπόλει. 



Joannes Scylitzes Hist., Synopsis historiarum 
Emperor life Bas+Const, section 6, line 22

             καὶ δευτέρᾳ τοῦ Ἰουλίου μηνός, τῆς αὐτῆς ἕκτης ἰνδικτιῶνος, 
ὑπὸ τῶν ἐν τῇ ἕῳ στρατευμάτων ἁπάντων παρακεκινημένων ὑπὸ τοῦ 
Τζιμισκῆ Ῥωμαίων ἀναγορεύεται βασιλεύς. 



Joannes Scylitzes Hist., Synopsis historiarum 
Emperor life Bas+Const, section 7, line 54

            στέφει οὖν τοῦτον ὁ Πολύευκτος ἐν τῷ ἄμβωνι τῆς τοῦ θεοῦ 
μεγάλης ἐκκλησίας, ἡμέρα δὲ ἦν κυριακή, ἑξκαιδεκάτην τοῦ Αὐγούστου 
μηνὸς ἄγοντος, τῆς ἕκτης ἰνδικτιῶνος. 



Joannes Scylitzes Hist., Synopsis historiarum 
Emperor life Niceph2, section 11, line 2

Ὁ δὲ Νικηφόρος κατὰ τὸ δεύτερον ἔτος τῆς αὐτοῦ βασιλείας, 
ἐν μηνὶ Ἰουλίῳ, ἰνδικτιῶνος ἑβδόμης, ἔξεισι κατὰ Κιλικίας σὺν βαρεῖ 
στρατῷ Ῥωμαίων καὶ συμμάχων Ἰβήρων καὶ Ἀρμενίων, ἔχων Θεο-
φανὼ τὴν γαμετὴν σὺν τοῖς τέκνοις αὐτῆς. 



Joannes Scylitzes Hist., Synopsis historiarum 
Emperor life Niceph2, section 14, line 2

Δῃώσας δὲ καὶ τεφρώσας καὶ τὰς λοιπὰς πόλεις τῆς Κιλικίας 
τῷ Ὀκτωβρίῳ μηνί, τῆς ἐννάτης ἰνδικτιῶνος, ὑπέστρεψεν εἰς Κωνσταν-
τινούπολιν, ἔχων μεθ' ἑαυτοῦ καὶ τὰς τῆς Ταρσοῦ πύλας καὶ τὰς τῆς 
Μόψου ἑστίας, ἃς καὶ χρυσῷ καταστίξας ἔξωθεν ἀνάθημα τῇ βασιλίδι 
διεκόμισε, τὰς μὲν κατὰ τὴν ἀκρόπολιν στήσας, τὰς δὲ κατὰ τὸ τῆς 
Χρυσῆς πόρτης τεῖχος. 



Joannes Scylitzes Hist., Synopsis historiarum 
Emperor life Niceph2, section 20, line 14

                               τετάρτῳ δὲ τῆς αὐτοῦ βασιλείας ἔτει, μηνὶ 
Ἰουνίῳ, τῆς δεκάτης ἰνδικτιῶνος, τὰς ἐν τῇ Θρᾴκῃ πόλεις ἐξῄει ἐπισκεψό-  
μενος, καὶ γενόμενος ἄχρι τῆς λεγομένης μεγάλης τάφρου ἔγραψε Πέτρῳ 
τῷ Βουλγαρίας ἄρχοντι, μὴ ἐᾶν τοὺς Τούρκους διαπερᾶν τὸν Ἴστρον 
καὶ τὰ Ῥωμαίων λυμαίνεσθαι. 



Joannes Scylitzes Hist., Synopsis historiarum 
Emperor life Niceph2, section 20, line 23

πεισθέντες οὖν οἱ Ῥῶς καὶ ἐπελθόντες τῇ Βουλγαρίᾳ κατὰ τὸν Αὔγουστον 
μῆνα, τῆς ἑνδεκάτης ἰνδικτιῶνος, πέμπτῳ τῆς βασιλείας ἔτει αὐτοῦ 
Νικηφόρου, πολλὰς πόλεις καὶ χώρας ἠδάφισαν τῶν Βουλγάρων, καὶ 
λείαν ὅτι πλείστην περιβαλλόμενοι ὑπέστρεψαν εἰς τὰ ἴδια. 



Joannes Scylitzes Hist., Synopsis historiarum 
Emperor life Niceph2, section 20, line 28

                                                                         δευτέρᾳ 
δὲ Σεπτεμβρίου μηνός, ὥρᾳ τῆς νυκτὸς δωδεκάτῃ, ἰνδικτιῶνος ἑνδεκάτης, 
γέγονε βρασμὸς καὶ κλόνος γῆς ἐξαίσιος, καὶ ἔπαθε κακῶς Ὁνωριὰς καὶ 
Παφλαγονία. 



Joannes Scylitzes Hist., Synopsis historiarum 
Emperor life Niceph2, section 20, line 31

              ἐγένοντο δὲ καὶ ἄνεμοι κατὰ τὸν Μάϊον μῆνα τῆς αὐτῆς 
ἰνδικτιῶνος σκληροὶ καὶ καυματώδεις, οἵτινες τοὺς καρποὺς διέφθειραν 
αὐταῖς ἀμπέλοις καὶ δένδρεσιν, ὡς ἐντεῦθεν κατὰ τὴν δωδεκάτην ἰνδι-
κτιῶνα σφοδρότατον ἐπενεχθῆναι λιμόν. 



Joannes Scylitzes Hist., Synopsis historiarum 
Emperor life Niceph2, section 22, line 9

                            καὶ τοῦ βασιλέως ὑπομνησθέντος, εἰ δεῖ τοῦτον 
εἰσελθεῖν ἐν τῇ βασιλίδι, κἀκείνου προσμένειν μικρὸν ἐπιτρέψαντος, 
ἐκείνη νυκτὸς ἑνδεκάτῃ τοῦ Δεκεμβρίου μηνός, ἰνδικτιῶνος τρισκαιδεκά-
της, ἔτους ἑξακισχιλιοστοῦ τετρακοσιοστοῦ ἑβδομηκοστοῦ ὀγδόου, 
ἀποστείλασα ἄγει τοῦτον πρὸς τὸν κάτωθεν τῶν παλατίων χειροποίη-
τον λιμένα, καὶ κοφίνῳ ἀνιμήσατο μετὰ πάντων τῶν περὶ αὐτόν. 



Joannes Scylitzes Hist., Synopsis historiarum 
Emperor life John1, section 20, line 1

Αὐγούστῳ δὲ μηνί, ἰνδικτιῶνος τρίτης, ἐφάνη καὶ κομήτης ὁ 
λεγόμενος πωγωνίας, καὶ ἐφαίνετο ἕως Ὀκτωβρίου μηνὸς τῆς τετάρ-
της ἰνδικτιῶνος. 



Joannes Scylitzes Hist., Synopsis historiarum 
Emperor life Bas2+Const8, section 1, line 4

                                                                                                ᾧ 
μᾶλλον πιστευτέον.   
ΒΑΣΙΛΕΙΟΣ ΚΑΙ ΚΩΝΣΤΑΝΤΙΝΟΣ


Τὸ μὲν οὖν Ἰωάννου τέλος τὸν εἰρημένον συνηνέχθη τρόπον, τὸ 
δὲ τῆς βασιλείας κράτος εἰς Βασίλειον μετήχθη καὶ Κωνσταντῖνον τοὺς 
υἱοὺς Ῥωμανοῦ, κατὰ τὸ ἑξακισχιλιοστὸν τετρακοσιοστὸν ὀγδοηκοστὸν 
τέταρτον ἔτος, ἰνδικτιῶνος τετάρτης, μηνὸς Δεκεμβρίου. 



Joannes Scylitzes Hist., Synopsis historiarum 
Emperor life Bas2+Const8, section 10, line 20

                                                       καὶ τούτους μὲν 
εἶχεν ἡ φρουρά, τῶν δὲ μὴ συνανελθόντων ἀποστατῶν τῷ Σκληρῷ   
Λέων μὲν ὁ αἰχμάλωτος καὶ οἱ τοῦ δουκὸς Ἀνδρονίκου τοῦ Λυδὸς 
παῖδες Χριστοφόρος ὁ Ἐπείκτης καὶ Βάρδας ὁ Μουγγός (ἔφθη γὰρ 
ἐκεῖνος ἀποθανεῖν) τὸ Ἀρμακούριον καὶ τὴν Πλατεῖαν πέτραν καὶ ἄλλα 
τινὰ φρούρια ἐρυμνὰ ἐν τῷ θέματι κείμενα τῶν Θρᾳκησίων κατεσχη-
κότες ἀντεῖχον ἕως ὀγδόης ἰνδικτιῶνος, καὶ ἐπεκδρομὰς ἐκ τούτων 
ποιούμενοι τὰ βασιλέως ἐλύπουν. 



Joannes Scylitzes Hist., Synopsis historiarum 
Emperor life Bas2+Const8, section 13, line 1

                                                                   οὗτος δὲ 
μὴ ἐνεγκὼν πράως τὴν ἐπιτίμησιν, ἀλλ' ἔτι μᾶλλον τραχυνόμενος καὶ 
δίκαια συμβουλεῦσαι ἰσχυριζόμενος, ἠνάγκασε τὸν βασιλέα διὰ τὴν 
ἀναίσχυντον ἰταμότητα ἀναπηδῆσαί τε τοῦ θρόνου καὶ τῶν τριχῶν 
αὐτοῦ καὶ τῆς γενειάδος λαβόμενον κατασπάσαι εἰς γῆν. 
Ἰνδικτιῶνος δὲ πεντεκαιδεκάτης, ἔτους ἑξακισχιλιοστοῦ τετρα-
κοσιοστοῦ ἐνενηκοστοῦ τετάρτου, Ὀκτωβρίῳ μηνί, ἐγένετο κλόνος μέγας,   
καὶ κατέπεσον οἰκίαι καὶ ναοὶ πολλοὶ καὶ μέρος τῆς σφαίρας τῆς τοῦ 
θεοῦ μεγάλης ἐκκλησίας. 



Joannes Scylitzes Hist., Synopsis historiarum 
Emperor life Bas2+Const8, section 14, line 7

Οἱ δὲ τῶν Ῥωμαίων μεγιστᾶνες μηνιῶντες τῷ βασιλεῖ, ὁ μὲν 
Φωκᾶς Βάρδας καί τινες σὺν αὐτῷ, ὅτι περ εἰς Βουλγαρίαν ἐκστρατεύ-
σας ὑπερεῖδεν αὐτούς, μηδ' ἐν Καρὸς λογισάμενος μοίρᾳ, ἄλλοι δὲ καὶ 
ἄλλοι δι' ἄλλους προπηλακισμούς τε καὶ παροινίας, ὁ δὲ μάγιστρος Εὐ-
στάθιος ὁ Μαλεῗνος διὰ τὸ ἀτίμως ἀπὸ τῆς εἰρημένης ἐκστρατείας ἀποπεμ-
φθῆναι, συναθροισθέντες ἐν τῷ Χαρσιανῷ κατὰ τὸν οἶκον τοῦ ῥηθέντος Μα-
λεΐνου, πεντεκαιδεκάτῃ τοῦ Αὐγούστου μηνός, τῆς πεντεκαιδεκάτης ἰνδικ-
τιῶνος, Βάρδαν τὸν Φωκᾶν ἀνεῖπον βασιλέα, διάδημά τε περιθέντες αὐτῷ 
καὶ τὰ λοιπὰ τῆς βασιλείας γνωρίσματα. 



Joannes Scylitzes Hist., Synopsis historiarum 
Emperor life Bas2+Const8, section 19, line 2

Ἄρτι δὲ τοῦ Φωκᾶ ἀποθανόντος κατὰ τὸν Ἀπρίλλιον μῆνα 
τῆς δευτέρας ἰνδικτιῶνος, τοῦ ἑξακισχιλιοστοῦ τετρακοσιοστοῦ ἐνενη-
κοστοῦ ἑβδόμου ἔτους, καὶ τῆς κατ' αὐτὸν ἀποστασίας διαλυθείσης, 
ἀδείας λαβόμενος ὁ Σκληρὸς πάλιν ἀνελάμβανεν ἑαυτὸν καὶ τὴν προ-
τέραν ἐσωμάσκει ἀποστασίαν. 



Joannes Scylitzes Hist., Synopsis historiarum 
Emperor life Bas2+Const8, section 22, line 4

Νικολάου δὲ τοῦ Χρυσοβέργη ἐπὶ χρόνοις δώδεκα καὶ μῆνας 
ὀκτὼ τὴν ἐκκλησίαν ἰθύναντος καὶ καταλύσαντος τὴν ζωήν, χειροτονεῖ-
ται Σισίνιος μάγιστρος, ἀνὴρ ἐλλόγιμος καὶ ἰατρικῆς τέχνης ἥκων εἰς τὸ 
ἀκρότατον, ἐν ἔτει ἑξακισχιλιοστῷ πεντακοσιοστῷ τρίτῳ, ἰνδικτιῶνος   
ὀγδόης· ὅστις καὶ τοὺς διακρινομένους ἥνωσε διὰ τὴν τετραγαμίαν 
Λέοντος τοῦ βασιλέως, ἐξ ὅτου τοῦτον Εὐθύμιος εἰς κοινωνίαν ἐδέξατο, ἥνωσεν. 
καὶ οὗτος δὲ ἐπὶ τρεῖς μόνους ἐνιαυτοὺς τὴν ἐκκλησίαν ποιμάνας ἐξέστη 
τῆς ζωῆς, καὶ προεβλήθη Σέργιος, ἡγούμενος ὢν τῆς μονῆς τοῦ Μανουὴλ 
καὶ τὸ γένος ἀναφέρων πρὸς Φώτιον τὸν πατριάρχην. 



Joannes Scylitzes Hist., Synopsis historiarum 
Emperor life Bas2+Const8, section 26, line 2

Τῷ δὲ ἑξακισχιλιοστῷ πεντακοσιοστῷ ὀγδόῳ ἔτει, τῆς τρισκαιδε-
κάτης ἰνδικτιῶνος, δύναμιν βαρεῖαν ἐκπέμψας ὁ βασιλεὺς κατὰ τῶν πέραν 
τοῦ Αἵμου Βουλγαρικῶν κάστρων, ἀρχηγοὺς ἔχουσαν τὸν πατρίκιον 
Θεοδωροκάνον καὶ Νικηφόρον πρωτοσπαθάριον τὸν Ξιφίαν, τήν τε   
μεγάλην εἷλε Περσθλάβαν καὶ τὴν μικρὰν καὶ τὴν Πλίσκοβαν, καὶ ἀσινὴς 
καὶ τροπαιοῦχος ἡ Ῥωμαϊκὴ ὑπενόστησε δύναμις. 



Joannes Scylitzes Hist., Synopsis historiarum 
Emperor life Bas2+Const8, section 30, line 1

Ὁ δὲ βασιλεὺς κατὰ τὸ ἐπιὸν ἔτος, ἰνδικτιῶνος πεντεκαιδε-
κάτης, ἐκστρατεύει κατὰ Βιδίνης, καὶ ἐφ' ὅλους ὀκτὼ μῆνας ἐμφιλοχω-
ρήσας τῇ προσεδρείᾳ αἱρεῖ κατὰ κράτος τὴν πόλιν. 



Joannes Scylitzes Hist., Synopsis historiarum 
Emperor life Bas2+Const8, section 32, line 1

Καὶ τῇ αὐτῇ ἰνδικτιῶνι δόγμα ἐξέθετο, τὰς τῶν ἀπολωλότων 
ταπεινῶν συντελείας τελεῖσθαι παρὰ τῶν δυνατῶν. 



Joannes Scylitzes Hist., Synopsis historiarum 
Emperor life Bas2+Const8, section 33, line 1

Ὀγδόῃ δὲ ἰνδικτιῶνι, ἐν ἔτει ἑξακισχιλιοστῷ πεντακοσιοστῷ 
ὀκτωκαιδεκάτω, ὁ τῆς Αἰγύπτου κατάρχων Ἀζίζιος διὰ μικρὰς αἰτίας 
καὶ μηδενὸς λόγου προσκρούσματα ἄξια τὰς πρὸς Ῥωμαίους λύσας 
σπονδὰς τόν τε ἐν Ἱεροσολύμοις ἐν τῷ τάφῳ τοῦ σωτῆρος Χριστοῦ 
ἀνεγηγερμένον πολυτελῶς θεῖον ναὸν κατεστρέψατο, καὶ τὰ εὐαγῆ 
ἐλυμήνατο μοναστήρια, καὶ τοὺς ἐν τούτοις ἀσκουμένους ἁπανταχοῦ 
τῆς γῆς ἐφυγάδευσε. 



Joannes Scylitzes Hist., Synopsis historiarum 
Emperor life Bas2+Const8, section 35, line 21

τούτου καὶ ἀποπειρασαμένου τῆς εἰσόδου, ἐπείπερ ἀντεῖχον γενναίως 
οἱ φυλάσσοντες καὶ τοὺς βιαζομένους ἐξ ὑπερδεξίων ἀνῄρουν βάλλοντες 
καὶ τιτρώσκοντες, καὶ ἤδη ἀπέγνωστο τῷ βασιλεῖ ἡ διάβασις, Νικη-
φόρος ὁ Ξιφίας τῆς Φιλιππουπόλεως τότε στρατηγῶν τῷ βασιλεῖ 
συνταξάμενος, καὶ αὐτὸν μὲν προσμένειν καὶ συνεχεῖς προσβολὰς ποιεῖσθαι 
τῷ δέματι παρεγγυήσας, αὐτὸς δὲ ἀπιέναι φήσας, εἴ πως καὶ δυνη-
θείη λυσιτελές τι διαπράξασθαι καὶ σωτήριον, τὸν περὶ αὐτὸν εἰληφὼς   
λαὸν ὑποστρέφει, καὶ περιοδεύσας τὸ πρὸς μεσημβρίαν τοῦ Κλειδίου 
κείμενον ὑψηλότερον ὄρος, ὃ Βαλασίτζαν κατονομάζουσι, καὶ τραχυ-
πορίαις καὶ ἀνοδίαις χρησάμενος, εἰκοστῇ ἐννάτῃ τοῦ Ἰουλίου μηνός, 
ἰνδικτιῶνος δωδεκάτης, ἄνωθεν ἐξαίφνης μετ' ἀλαλαγμοῦ καὶ δούπου 
κατὰ νώτου γίνεται τῶν Βουλγάρων. 



Joannes Scylitzes Hist., Synopsis historiarum 
Emperor life Bas2+Const8, section 35, line 44

                                   λαβὼν δὲ καὶ πιὼν ἐλήφθη καρδιωγμῷ, 
καὶ μετὰ δύο ἡμέρας θνῄσκει 
κατὰ τὴν ἕκτην τοῦ Ὀκτωβρίου μηνός. 
παραλαμβάνει δὲ τὴν ἀρχὴν τῶν Βουλγάρων ὁ υἱὸς αὐτοῦ Γαβριὴλ 
ὁ καὶ Ῥωμανός, ῥώμῃ μὲν καὶ ἰσχύϊ τοῦ πατρὸς ὑπερέχων, φρονήσει 
δὲ καὶ διανοίᾳ πολλῷ λειπόμενος, τεχθεὶς τῷ Σαμουὴλ ἀπό τινος 
αἰχμαλώτιδος Λαρισσαίας 
ἀπό τινος αἰχμαλώτιδος Λαρισσαίας] ἐξ Ἀγάθης τῆς θυγατρὸς Ἰωάννου τοῦ Χρυση-
λίου τοῦ ἐν Δυρραχίῳ πρωτεύοντος. 
ἦρξε δὲ κατὰ τὴν πεντεκαιδεκάτην τοῦ Σεπτεμβρίου μηνός, ἰνδικτιῶνος 
τρισκαιδεκάτης. 



Joannes Scylitzes Hist., Synopsis historiarum 
Emperor life Bas2+Const8, section 40, line 2

Ἑξακισχιλιοστῷ δὲ πεντακοσιοστῷ εἰκοστῷ τετάρτῳ ἔτει, 
ἰνδικτιῶνος τεσσαρεσκαιδεκάτης, ἀπάρας τῆς βασιλίδος ὁ βασιλεὺς 
ἄπεισιν εἰς Τριάδιτζαν, καὶ τὸ φρούριον Πέρνικον περικαθίσας ἐπολιόρκει, 
τῶν ἔνδον δὲ καρτερῶς καὶ ἐκθύμως ἀγωνιζομένων, καὶ πολλῶν Ῥω-
μαίων πιπτόντων. 



Joannes Scylitzes Hist., Synopsis historiarum 
Emperor life Bas2+Const8, section 40, line 42

                                                                           καὶ 
κτείνουσι μὲν πολλούς, συλλαμβάνουσι δὲ καὶ διακοσίους πανοπλίτας 
καὶ τοὺς ἵππους καὶ τὴν ἀποσκευὴν Ἰωάννου καὶ τὸν τούτου ἀνεψιόν· 
ὃν καὶ παραυτίκα στερεῖ τῶν ὀφθαλμῶν. ταῦτα δράσας ἐπάνεισι πρὸς τὰ Βο-
δηνά, καὶ πάντα τὰ ἐκεῖσε καταστησάμενος ἐπαναζευγνύει πρὸς τὸ 
Βυζάντιον, κατὰ τὴν ἐννάτην τοῦ Ἰαννουαρίου μηνός, ἰνδικτιῶνος πεντε-
καιδεκάτης, ἔτους ἑξακισχιλιοστοῦ πεντακοσιοστοῦ εἰκοστοῦ ἕκτου. 



Joannes Scylitzes Hist., Synopsis historiarum 
Emperor life Bas2+Const8, section 43, line 51

καὶ τῶν λοιπῶν Βουλγάρων καὶ τοῦ ἀρχιερέως Βουλγάρων. 
ἰνδικτιὼν ἦν δευτέρα, ἔτος ἑξακισχιλιοστὸν πεντακοσιοστὸν εἰκοστὸν 
ἕβδομον. 



Joannes Scylitzes Hist., Synopsis historiarum 
Emperor life Bas2+Const8, section 43, line 57

                                                     ὃς ἐπὶ εἴκοσιν ὅλους ἐνιαυ-
τοὺς τὴν τοῦ θεοῦ ποιμάνας ἐκκλησίαν Ἰουλίῳ μηνί, ἰνδικτιῶνος δευτέρας, 
ἐν ἔτει τῷ ἑξακισχιλιοστῷ πεντακοσιοστῷ εἰκοστῷ ἑβδόμῳ, πρὸς 
κύριον ἐξεδήμησε. 



Joannes Scylitzes Hist., Synopsis historiarum 
Emperor life Bas2+Const8, section 45, line 28

      μετὰ δὲ ταῦτα δευτέρας συμπλοκῆς γενομένης, τῇ ἑνδεκάτῃ τοῦ 
Σεπτεμβρίου μηνός, ἰνδικτιῶνος ἕκτης, κατὰ τὸ ἑξακισχιλιοστὸν πεντα-
κοσιοστὸν τριακοστὸν πρῶτον ἔτος, πίπτει μὲν ὁ Λιπαρίτης (οὗτος γὰρ 
ἦν ἀρχιστράτηγος τῷ Γεωργίῳ) καὶ σὺν αὐτῷ πᾶν τὸ κρατιστεῦον ἐν 
Ἀβασγοῖς, φεύγει δὲ καὶ ὁ Γεώργιος εἰς τὰ ἐνδότερα ὄρη τῆς Ἰβηρίας. 



Joannes Scylitzes Hist., Synopsis historiarum 
Emperor life Bas2+Const8, section 47, line 4

                                                          Δεκεμβρίῳ γὰρ μηνί, 
ἰνδικτιῶνος ἐννάτης, ἔτους ἑξακισχιλιοστοῦ πεντακοσιοστοῦ τριακοστοῦ 
τετάρτου, αἰφνιδίῳ νόσῳ ληφθεὶς ἀπεβίω, πρό τινων ἡμερῶν τῆς αὐτοῦ 
τελευτῆς καὶ Εὐσταθίου τοῦ πατριάρχου ἀποθανόντος, 
περὶ οὗ λέγεται, ὅτι ἐν μιᾷ τῶν ἐπισήμων ἑορτῶν τῆς ἱερουργίας τελουμένης – τοῦτο μὲν 
[γὰρ add. C] διὰ τὸ γῆρας· ἔξωρος γὰρ (γὰρ om. C) ἦν, τοῦτο δὲ διὰ μαλακίαν 
σωματικήν· ἐνοσηλεύετο γάρ – τῆς εἰσόδου τῶν ἁγίων ἀργῶς τε καὶ σχολαίως πλεῖον 
(πλείω EU) τοῦ (τοῦ] τῆς E) συνήθους τελουμένης (τελωμένης C), μετὰ τὸ τελέσαι τὴν 
εὐχὴν μὴ δυνάμενος ἵστασθαι, δόξαν αὐτῷ ἤχθη θρόνος, ἐφ' ᾧ ἀναπαύεσθαι (ἀναπεπαῦ-
σθαι A) εἴθιστο. 



Joannes Scylitzes Hist., Synopsis historiarum 
Emperor life Const8, section 3, line 1

Ἐννάτῃ δὲ Νοεμβρίου μηνός, τῆς δωδεκάτης ἰνδικτιῶνος, ἐν ἔτει 
ἑξακισχιλιοστῷ πεντακοσιοστῷ τριακοστῷ ἑβδόμῳ, αἰφνιδίῳ νόσῳ 
ληφθεὶς ὁ Κωνσταντῖνος καὶ παρὰ τῶν ἰατρῶν ἀπαγορευθεὶς ἐσκέπτετο, 
τίνα ἂν καταλίποι διάδοχον τῆς βασιλείας. 



Joannes Scylitzes Hist., Synopsis historiarum 
Emperor life Roman3, section 5, line 38

                                                            καὶ δὴ κατὰ τὴν 
δευτέραν τοῦ Αὐγούστου μηνός, ἰνδικτιῶνος τρισκαιδεκάτης, ἔτους 
ἑξακισχιλιοστοῦ πεντακοσιοστοῦ τριακοστοῦ ὀγδόου, κατὰ τὰ βεβου-
λευμένα τὰς τῆς στρατοπεδείας ἀνοίξαντες πύλας ἁπανταχῇ τῆς ἐπ' 
Ἀντιόχειαν εἴχοντο, οἱ πλείους μὲν κοιλιακῷ κατειργασμένοι νοσήματι, 
οἱ πλείους δὲ καὶ ὑπὸ τῆς ἀνάγκης τοῦ δίψους κατειργασμένοι. 



Joannes Scylitzes Hist., Synopsis historiarum 
Emperor life Roman3, section 8, line 10

                           τούτῳ τῷ ἔτει ἑξακισχιλιοστῷ πεντακοσιο-
στῷ τριακοστῷ ἐννάτῳ τυγχάνοντι, ἰνδικτιῶνος τεσσαρεσκαιδεκάτης, 
καὶ ὁ Προυσιάνος ἑκουσίως ἀποκείρεται μοναχός, καὶ ἡ μήτηρ αὐτοῦ 
ἐκ Μαντινείου τῆς ἐν Βουκελλαρίῳ μονῆς εἰς Θρᾳκησίους μεταβιβάζεται, 
καὶ Κωνσταντῖνος πατρίκιος ὁ Διογένης ἐκβληθεὶς τοῦ πύργου ἐν τῇ 
μονῇ τῶν Στουδίου ἀποκείρεται μοναχός. 



Joannes Scylitzes Hist., Synopsis historiarum 
Emperor life Roman3, section 9, line 1

Τῷ δὲ ἑξακισχιλιοστῷ πεντακοσιοστῷ τετρακοστῷ ἔτει, ἰνδι-
κτιῶνος πεντεκαιδεκάτης, μηνὶ Σεπτεμβρίῳ, ἦλθε πρὸς τὸν βασιλέα Ῥω-  
μανὸν μετὰ δώρων πολλῶν ὁ τοῦ Χαλεπίτου υἱὸς Ἄμερ, ἐξαιτῶν ἀνα-
νεῶσαι τὴν εἰρήνην καὶ τοὺς πρόσθεν παρέχειν φόρους. 



Joannes Scylitzes Hist., Synopsis historiarum 
Emperor life Roman3, section 17, line 37

                                            διήρκεσεν οὖν ἄχρι τῆς ἑνδεκάτης 
τοῦ Ἀπριλλίου μηνός, τῆς δευτέρας ἰνδικτιῶνος, τοῦ ͵ϛφμβʹ ἔτους. 



Joannes Scylitzes Hist., Synopsis historiarum 
Emperor life Mich4, section 5, line 22

                                         καὶ εὐθὺς ἄγεται εἰς τὰ βασίλεια, καὶ 
κατὰ τὴν τρίτην τοῦ Αὐγούστου μηνὸς τῆς δευτέρας ἰνδικτιῶνος ἐν τῇ 
νήσῳ Πλάτῃ περιορίζεται. 



Joannes Scylitzes Hist., Synopsis historiarum 
Emperor life Mich4, section 8, line 1

Τῷ δὲ ͵ϛφμγʹ ἔτει, ἰνδικτιῶνος τρίτης, Μαΐῳ μηνί, Ἄφροι καὶ 
Σικελοὶ καταδραμόντες τὰς Κυκλάδας καὶ τὰ τοῦ Θρᾳκησίου παράλια, 
τελευταῖον κατεπολεμήθησαν ὑπὸ τῶν ἐκεῖσε φυλαττόντων, καὶ πεντακό-
σιοι μὲν ζῶντες ὡς βασιλέα ἤχθησαν, οἱ δὲ λοιποὶ πάντες ἀνεσκολοπίσθη-
σαν ἐν τῇ παραλίῳ ἀπὸ Ἀτραμυτίου καὶ μέχρι Στροβίλου. 



Joannes Scylitzes Hist., Synopsis historiarum 
Emperor life Mich4, section 10, line 1

Τῷ δὲ ͵ϛφμδʹ ἔτει, ἰνδικτιῶνος τετάρτης, διὰ τοῦ ἔαρος τρεῖς 
εἰσβολὰς οἱ Πατζινάκαι ποιησάμενοι κατὰ Ῥωμαίων ἄρδην τὰ παρα-
τυχόντα ἠφάνισαν, ἡβηδὸν τοὺς ἁλισκομένους ἀναιροῦντες καὶ τιμωρίαις 
τοὺς αἰχμαλώτους ὑποβάλλοντες ἀνεκδιηγήτοις. 



Joannes Scylitzes Hist., Synopsis historiarum 
Emperor life Mich4, section 10, line 14

                                                                Δεκεμβρίῳ δὲ 
μηνί, ἰνδικτιῶνος πέμπτης, ἔτους ͵ϛφμεʹ, κατὰ τὴν ὀκτωκαιδεκάτην τοῦ 
μηνός, περὶ τετάρτην ὥραν τῆς νυκτός, γεγόνασι σεισμοὶ τρεῖς, δύο μικροὶ 
καὶ εἷς μέγας. 



Joannes Scylitzes Hist., Synopsis historiarum 
Emperor life Mich4, section 13, line 1

Τῷ δὲ ͵ϛφμϛʹ ἔτει, ἰνδικτιῶνος ἕκτης, Νοεμβρίου δευτέρᾳ, γέγονε 
σεισμὸς περὶ ὥραν δεκάτην τῆς ἡμέρας, καὶ διετέλεσεν ἡ γῆ σειομένη 
ἄχρι ὅλου τοῦ Ἰαννουαρίου μηνός. 



Joannes Scylitzes Hist., Synopsis historiarum 
Emperor life Mich4, section 17, line 1

Τῷ δὲ ͵ϛφμϛʹ ἔτει, ἰνδικτιῶνος ἕκτης, γέγονεν ἐπιβουλὴ κατὰ τῆς 
πόλεως Ἐδέσσης· καὶ μικροῦ ἂν ἑάλω, εἰ μὴ θεὸς διεσώσατο. 



Joannes Scylitzes Hist., Synopsis historiarum 
Emperor life Mich4, section 18, line 1

Τῷ δὲ ͵ϛφμζʹ ἔτει, ἰνδικτιῶνος ἑβδόμης, ἐπιτείνων τὸ πρὸς τὸν 
Δαλασσηνὸν ἔχθος ὁ Ἰωάννης ὑπερορίζει καὶ Θεοφύλακτον πατρίκιον 
τὸν αὐτοῦ ἀδελφόν, καὶ τὸν ἕτερον ἀδελφὸν αὐτοῦ τὸν πατρίκιον Ῥω-
μανόν, καὶ Ἀδριανὸν τὸν ἀνεψιὸν αὐτῶν καὶ τοὺς λοιποὺς τοὺς κατὰ 
γένος αὐτῷ ἐγγίζοντας. 



Joannes Scylitzes Hist., Synopsis historiarum 
Emperor life Mich4, section 19, line 1

Φεβρουαρίου δὲ μηνὸς δευτέρᾳ, ἰνδικτιῶνος ὀγδόης, ἔτους ͵ϛφμηʹ, 
γέγονε σεισμὸς φρικώδης, καὶ ἔπαθον μὲν καὶ ἄλλοι τόποι καὶ πόλεις, 
ἐγένετο δὲ ἡ Σμύρνα ἐλεεινὸν θέαμα, καταπεσόντων τῶν καλλίστων 
οἰκοδομημάτων αὐτῆς καὶ πολλοὺς τῶν οἰκητόρων ἀναλωσάντων. 



Joannes Scylitzes Hist., Synopsis historiarum 
Emperor life Mich4, section 21, line 14

                                                       Μαΐῳ δὲ μηνὶ τοῦ 
͵ϛφμηʹ ἔτους, ἰνδικτιῶνος ὀγδόης, ἡ ἀδελφὴ τοῦ βασιλέως Μαρία, μήτηρ 
δὲ τοῦ καίσαρος, ἄπεισιν ἐν Ἐφέσῳ εἰς προσκύνησιν τοῦ ἠγαπημένου. 



Joannes Scylitzes Hist., Synopsis historiarum 
Emperor life Mich4, section 27, line 1

Σεπτεμβρίου δὲ μηνός, ἰνδικτιῶνος ἐννάτης τοῦ ͵ϛφμθʹ ἔτους,   
Ἀλουσιάνος πατρίκιος καὶ στρατηγὸς Θεοδοσιουπόλεως, ὁ τοῦ Ἀαρὼν 
δεύτερος υἱός, ἄφνω τῆς πόλεως ἀποδρὰς τῷ Δελεάνῳ προσέρχεται ἐξ 
αἰτίας τοιαύτης. 



Joannes Scylitzes Hist., Synopsis historiarum 
Emperor life Mich4, section 28, line 1

Τούτῳ τῷ ἔτει, ἰνδικτιῶνος ἐννάτης, Ἰουνίου δεκάτῃ, περὶ ὥραν 
τῆς ἡμέρας δωδεκάτην, γέγονε σεισμός. 



Joannes Scylitzes Hist., Synopsis historiarum 
Emperor life Mich4, section 29, line 23

                                                     Δεκεμβρίου δὲ δεκάτῃ, 
τοῦ ͵ϛφνʹ ἔτους, ἰνδικτιῶνος δεκάτης, θνῄσκει ἐν μετανοίᾳ καὶ ἐξομολο-
γήσει, τὴν εἰς τὸν βασιλέα Ῥωμανὸν ἁμαρτάδα ἀποκλαιόμενος, βασι-
λεύσας ἐπὶ ἔτη ἑπτὰ καὶ μῆνας ὀκτώ, ἀνὴρ τἆλλα μὲν ἐπιεικὴς καὶ 
χρηστὸς καὶ εὐλαβῶς δόξας βιοῦν, πλὴν τοῦ εἰς τὸν βασιλέα Ῥωμανὸν 
ἁμαρτήματος. 



Joannes Scylitzes Hist., Synopsis historiarum 
Emperor life Mich5, section 2, line 27

                                               τυφλωθέντες οὖν ἐξορίζονται, ὁ 
μὲν Μιχαὴλ εἰς τὸ μοναστήριον τῶν Ἐλεγμῶν εἰκάδι πρώτῃ μηνὸς 
Ἀπριλλίου, ἰνδικτιῶνος δεκάτης τοῦ ͵ϛφνʹ ἔτους, βασιλεύσας ἐπὶ μῆνας 
τέσσαρας καὶ ἡμέρας πέντε, καὶ οἱ συγγενεῖς δὲ πάντες αὐτοῦ ἄλλος 
ἀλλαχοῦ διεσπάρησαν. 



Joannes Scylitzes Hist., Synopsis historiarum 
Emperor life Const9, section 2, line 10

                            καὶ ταῦτα μὲν ἐπράχθη ἐν προοιμίοις τῷ Μονομά-  
χῳ, κατὰ τὴν δεκάτην ἰνδικτιῶνα. 



Joannes Scylitzes Hist., Synopsis historiarum 
Emperor life Const9, section 2, line 11

                                       κατὰ δὲ τὴν ἕκτην τοῦ Ὀκτωβρίου 
μηνός, τῆς ἑνδεκάτης ἰνδικτιῶνος, τοῦ ͵ϛφναʹ ἔτους, ἐφάνη κομήτης ἀπὸ 
τῆς ἕω πρὸς δύσιν ποιούμενος τὴν πορείαν, καὶ ὡρᾶτο λάμπων παρ' 
ὅλον τοῦτον τὸν μῆνα. 



Joannes Scylitzes Hist., Synopsis historiarum 
Emperor life Const9, section 5, line 1

Τῇ δὲ εἰκοστῇ τοῦ Φεβρουαρίου μηνός, τῆς ἑνδεκάτης ἰνδικτιῶνος, 
κατέλυσε τὴν ζωὴν Ἀλέξιος ὁ πατριάρχης, καὶ ἀνάγεται εἰς τὸν αὐτοῦ 
θρόνον Μιχαὴλ ὁ Κηρουλάριος κατὰ τὴν ἡμέραν τοῦ εὐαγγελισμοῦ, 
μοναχὸς ὤν, ἐξ οὗπερ ὁ ὀρφανοτρόφος αὐτὸν διὰ τὴν ἐπιβουλὴν ἐξώρισε. 



Joannes Scylitzes Hist., Synopsis historiarum 
Emperor life Const9, section 5, line 7

                                  Μαΐου δὲ δευτέρᾳ, τῆς αὐτῆς ἰνδικτιῶνος, 
ἐκτυφλοῦται καὶ ὁ ὀρφανοτρόφος ἐν τῷ λεγομένῳ χωρίῳ τῶν Μαρυκά-
των, ὡς μέν τινές φασιν, ὑπὸ Θεοδώρας ἄκοντος τοῦ βασιλέως, ὡς δὲ ὁ 
τῶν πολλῶν κρατεῖ λόγος, παρ' αὐτοῦ τούτου ἐγκοτῶντος αὐτῷ διὰ 
τὰς ὑπερορίας, καὶ τῇ τρισκαιδεκάτῃ τοῦ αὐτοῦ μηνὸς ἀποθνῄσκει. 



Joannes Scylitzes Hist., Synopsis historiarum 
Emperor life Const9, section 5, line 12

Ἰουλίῳ δὲ μηνὶ τῆς αὐτῆς ἰνδικτιῶνος κατηγορήθη καὶ Στέφανος ὁ 
σεβαστοφόρος, ὡς ἐπιβουλεύων τῷ βασιλεῖ καὶ βασιλέα βουλόμενος ποιῆ-  
σαι Λέοντα πατρίκιον καὶ στρατηγὸν Μελιτηνῆς, τὸν υἱὸν τοῦ Λαμπροῦ. 



Joannes Scylitzes Hist., Synopsis historiarum 
Emperor life Const9, section 6, line 1

Ἐγένετο δὲ καὶ κατὰ τὸν Ἰούλιον μῆνα τῆς αὐτῆς ἰνδικτιῶνος καὶ 
ἡ τοῦ ἔθνους τῶν Ῥῶς κίνησις κατὰ τῆς βασιλίδος. 



Joannes Scylitzes Hist., Synopsis historiarum 
Emperor life Const9, section 7, line 1

Σεπτεμβρίῳ δὲ μηνί, ἰνδικτιῶνος δωδεκάτης, ἔτους ͵ϛρνβʹ, ἔπνευσεν 
ἄνεμος σφοδρός, ὡς διαφθαρῆναι σχεδὸν τοὺς τῶν ἀμπέλων καρπούς. 



Joannes Scylitzes Hist., Synopsis historiarum 
Emperor life Const9, section 8, line 1

                                                ἦν δὲ ὁ Στηθάτος οὗτος ἀρετῆς εἰς ἄκραν 
ἐπιμελούμενος καὶ νηστείᾳ καὶ σκληραγωγίᾳ καὶ πάσῃ ἄλλῃ ἀρετῇ ἐντήκων τὸ σῶμα 
ἑαυτοῦ, ὡς καί ποτε τεσσαράκοντα ἡμέρας ἄσιτος διατελέσαι, μηδενὸς τὸ παράπαν ἐν 
τῷ μέσῳ γευσάμενος.   
Ἰνδικτιῶνος δὲ τρισκαιδεκάτης ὁ κατὰ τοῦ Ἀνίου ἀρχὴν ἐλάμβανε 
πόλεμος. 



Joannes Scylitzes Hist., Synopsis historiarum 
Emperor life Const9, section 8, line 130

                                                                ἀλλὰ τοῦτο 
μὲν ὕστερον· πρὶν ἢ δὲ τὰς δυνάμεις ἐλθεῖν, ὁ Τορνίκιος κατὰ τὸν Σεπτέμ-
βριον μῆνα, τῆς πρώτης ἰνδικτιῶνος, βασιλεύς, ὡς ἐλέχθη, ἀναρρηθεὶς 
θᾶττον ἢ λόγος τὴν βασιλίδα καταλαμβάνει, ἐλπίζων αὐτοβοεὶ ταύτην 
αἱρήσειν διὰ τὸ μὴ εὐπορεῖν τὸν βασιλέα δυνάμεων. 



Joannes Scylitzes Hist., Synopsis historiarum 
Emperor life Const9, section 9, line 15

ἀρχῆς εἰς Σαρακηνοὺς διαλυθείσης, καὶ τῆς τῶν Σαρακηνῶν ἐπικρατείας 
μὴ μόνον Περσίδος καὶ Μηδίας καὶ Βαβυλῶνος καὶ Ἀσσυρίων κυριευούσης, 
ἤδη δὲ καὶ Αἰγύπτου καὶ Λιβύης καὶ μέρους οὐκ ὀλίγου τῆς Εὐρώπης, 
ἐπείπερ ἔτυχον ἐν διαφόροις καιροῖς ἀλλήλων καταστασιάσαντες καὶ ἡ 
μία καὶ μεγίστη αὕτη ἀρχὴ εἰς πολλὰ διῃρέθη μέρη, καὶ ἄλλον μὲν ἀρχηγὸν 
εἶχεν ἡ Ἱσπανία, ἄλλον δὲ ἡ Λιβύη, ἄλλον δὲ ἡ Αἴγυπτος, ἄλλον δὲ ἡ   
Βαβυλών, ἕτερον δὲ ἡ Περσίς, καὶ πρὸς ἀλλήλους μὲν οὐχ ὡμονόουν, 
μᾶλλον μὲν οὖν καὶ προσεπολέμουν οἱ γειτονοῦντες, ἀρχηγὸς Περσίδος 
καὶ Χωρασμίων καὶ Ὠρητανῶν καὶ Μηδίας ὑπάρχων Μουχούμετ κατὰ 
τοὺς χρόνους Βασιλείου τοῦ βασιλέως, ὁ τοῦ Ἰμβραήλ, καὶ πολεμῶν 
Ἰνδοῖς καὶ Βαβυλωνίοις καὶ κακῶς ἐν τῷ πολέμῳ φερόμενος, ἔγνω δεῖν 
πρὸς τὸν ἄρχοντα Τουρκίας διαπρεσβεύσασθαι καὶ συμμαχίαν ἐκεῖθεν 
αἰτήσασθαι. 



Joannes Scylitzes Hist., Synopsis historiarum 
Emperor life Const9, section 9, line 30

                                                     ὑποστρέψας δὲ εἰς 
τὴν ἑαυτοῦ ἠπείγετο καὶ πρὸς τοὺς πολεμοῦντας Ἰνδοὺς διαγωνίσασθαι 
μετ' αὐτῶν. 



Joannes Scylitzes Hist., Synopsis historiarum 
Emperor life Const9, section 14, line 7

                                                         ἀλλ' ὁ Λιπαρίτης οὐκ 
ἤθελε διὰ τὴν ἡμέραν· ἦν γὰρ σάββατον, ὀκτωκαιδεκάτην ἄγοντος τοῦ 
Σεπτεμβρίου μηνός, τῆς δευτέρας ἰνδικτιῶνος, ἐν ταῖς ἀποφράσι δὲ 
τῶν ἡμερῶν τῷ Λιπαρίτῃ τὸ σάββατον ἐνομίζετο, ὅπερ ἀποτροπια-
ζόμενος ἀνεδύετο πολεμεῖν. 



Joannes Scylitzes Hist., Synopsis historiarum 
Emperor life Const9, section 22, line 80

           τῆς δὲ τρίτης ἐπιστάσης ἰνδικτιῶνος τοῦ ͵ϛφνηʹ ἔτους, τὸν ἑται-
ρειάρχην Κωνσταντῖνον στρατηγὸν αὐτοκράτορα προχειρισάμενος ὁ   
βασιλεὺς ἐκπέμπει κατὰ τῶν Πατζινάκων. 



Joannes Scylitzes Hist., Synopsis historiarum 
Emperor life Const9, section 25, line 29

                                                                     διὸ καὶ τὰς 
ἐκδρομὰς οὐκ ἀνέτως, ὡς τὸ πρότερον, ἀλλὰ μετὰ φειδοῦς ἐποιοῦντο 
κατά τε τὴν τετάρτην καὶ πέμπτην ἰνδικτιῶνα. 



Joannes Scylitzes Hist., Synopsis historiarum 
Emperor life Const9, section 30, line 2

Ἐνέσκηψε δὲ καὶ λοιμικὴ νόσος τῇ βασιλίδι κατά τε τὴν ἑβδό-
μην καὶ ὀγδόην ἰνδικτιῶνα, ὡς μὴ ἐξισχύειν τοὺς ζῶντας ἐκφέρειν τοὺς 
τεθνεῶτας. 



Joannes Scylitzes Hist., Synopsis historiarum 
Emperor life Const9, section 30, line 3

            ἠνέχθη δὲ καὶ κατὰ τὸ θέρος τῆς ἑβδόμης ἰνδικτιῶνος μέγι-
στόν τι χρῆμα χαλάζης, καὶ πολλὰ ὑπ' αὐτῆς οὐ ζῷα μόνον, ἀλλὰ 
καὶ ἄνθρωποι διεφθάρησαν. 



Joannes Scylitzes Hist., Synopsis historiarum 
Emperor life Theod, section 1, line 27

               ὅλην οὖν τὴν θʹ ἰνδικτιῶνα, τοῦ ͵ϛφξδʹ ἔτους βιώσασα 
ἡ βασιλίς, καὶ περὶ τὰ τέλη τοῦ Αὐγούστου μηνὸς τῆς αὐτῆς ἰνδικτιῶνος 
ἰλεοῦ νοσήματι περιπεσοῦσα ἀπέθανεν. 



Joannes Scylitzes Hist., Synopsis historiarum 
Emperor life Mich6, section 1, line 2

ΜΙΧΑΗΛ Ο ΓΕΡΩΝ


Ἀναρρηθέντος δὲ τοῦ Μιχαὴλ αὐτοκράτορος κατὰ τὴν λαʹ ἡμέραν 
τοῦ Αὐγούστου μηνός, τῆς θʹ ἰνδικτιῶνος, Θεοδόσιος πρόεδρος ὁ τοῦ 
πατραδέλφου τοῦ βασιλέως Κωνσταντίνου τοῦ Μονομάχου υἱός, πυθό-
μενος τὴν ἀνάρρησιν καὶ δεινοπαθήσας, καὶ μὴ βουλευσάμενος, μηδὲ λογισά-
μενος τὴν τοῦ ἔργου δυσχέρειαν καὶ ἀπότευξιν, μηδ' οἷον κύβον μέλλει 
ἀναρριπτεῖν, ἀνειληφὼς τοὺς οἰκογενεῖς καὶ δούλους καὶ τοὺς ἄλλως 
ὑπηρετουμένους αὐτῷ, πολλοὺς δὲ καὶ τῶν γειτόνων καί τινας τῶν 
συνήθων, ὅσοι περ ἦσαν τὰς φρένας κουφότεροι, ἄρας περὶ δείλην 
ὀψίαν ἐκ τῆς οἰκίας (κεῖται δὲ αὕτη περὶ τὸ λεγόμενον Λεωμακέλλιον) 
προῄει διὰ τῆς πλατείας ὡς εἰς τὸ παλάτιον, ἀγανακτῶν καὶ δυσχεραί-
νων, καὶ τὴν ἀδικίαν ὡς ἂν τὰ ἔσχατα ἠδικημένος πρὸς τοὺς

παρατυ-



Joannes Scylitzes Hist., Synopsis historiarum 
Emperor life Mich6, section 6, line 45

                κἀκεῖσε τοὺς πλησιοχώρους ἀθροίσαντες στρατιώτας 
καὶ τοὺς ὅσοι πυθόμενοι τὴν κίνησιν ἐθελονταὶ παρεγένοντο, μετὰ πάν-
των αὐτῶν ἀναγορεύουσι τοῦτον αὐτοκράτορα βασιλέα Ῥωμαίων, 
ὀγδόην ἄγοντος τότε τοῦ Ἰουνίου μηνός, τῆς δεκάτης ἰνδικτιῶνος. 



Joannes Scylitzes Hist., Synopsis historiarum 
Emperor life Mich6, section 12, line 67

                                                                               τοῦ 
δὲ κατὰ τὴν ἐν ἀκροπόλει οἰκίαν αὐτοῦ γενομένου ἡμέρᾳ τετράδι, τρια-
κοστὴν πρώτην ἄγοντος τοῦ Αὐγούστου μηνός, τῆς δεκάτης ἰνδικτιῶνος, 
στέλλεται ὁ Κεκαυμένος κουροπαλάτης ὑπὸ τοῦ Κομνηνοῦ τιμηθεὶς 
μετά τινων οὐκ ὀλίγων εὐπατριδῶν τῇ πέμπτῃ πρωΐας διὰ δρόμωνος, 
καὶ εἰσελθὼν κρατεῖ τὰ ἀνάκτορα. 



Pytheas Perieg., Fragmenta (1650: 001)
“Pytheas von Massalia”, Ed. Mette, H.J.
Berlin: De Gruyter, 1952.
Fragment 5, line 17

            τί<ς> γὰρ ἡ πιθανότης πρῶτον μὲν τῆς κατὰ   
τὸν Ἰνδὸν περιπετείας; 



Pytheas Perieg., Fragmenta 
Fragment 6a, line 52

                 ὅτι μὲν γὰρ πλέον ἢ διπλάσιον τὸ γνώριμον 
μῆκός ἐστι τοῦ γνωρίμου πλάτους, ὁμολογοῦσι καὶ οἱ 
ὕστερον καὶ τῶν ἄλλων οἱ χαριέστατοι· λέγω δὲ <τὸ> ἀπὸ 
τῶν ἄκρων τῆς Ἰνδικῆς ἐπὶ τὰ ἄκρα τῆς Ἰβηρίας τοῦ 
<ἀπὸ τῶν> Αἰθιόπων ἕως τοῦ κατὰ Ἰέρνην κύκλου. 



Pytheas Perieg., Fragmenta 
Fragment 6a, line 57

                                                φησὶ δ' οὖν τὸ 
μὲν τῆς Ἰνδικῆς μέχρι τοῦ Ἰνδοῦ ποταμοῦ τὸ στενώτατον 
σταδίων μυρίων ἑξακισχιλίων – τὸ γὰρ ἐπὶ τὰ ἀκρωτήρια 
τεῖνον τρισχιλίοις εἶναι μεῖζον – , τὸ δὲ ἔνθεν ἐπὶ Κασπίους 
Πύλας μυρίων τετρακισχιλίων, εἶτ' ἐπὶ τὸν Εὐφράτην 
μυρίων, ἐπὶ δὲ τὸν Νεῖλον ἀπὸ τοῦ Εὐφράτου πεντακις-
χιλίων, ἄλλους δὲ χιλίους καὶ τριακοσίους μέχρι Κανωβικοῦ 
Στόματος, εἶτα μέχρι τῆς Καρχηδόνος μυρίους τρισχιλίους 
πεντακοσίους, εἶτα μέχρι Στηλῶν ὀκτακισχιλίους τοὐλά-
χιστον· ὑπεραίρειν δὴ τῶν ἑπτὰ μυριάδων ὀκτακοσίοις. 



Pytheas Perieg., Fragmenta 
Fragment 6b, line 3

                           ὅρα γάρ, εἰ τοῦτο μὲν μὴ κινοίη 
τις τὸ τὰ ἄκρα τῆς Ἰνδικῆς τὰ μεσημβρινὰ ἀνταίρειν τοῖς 
κατὰ Μερόην, μηδὲ τὸ διάστημα τὸ ἀπὸ Μερόης ἐπὶ τὸ 
στόμα τὸ κατὰ τὸ Βυζάντιον ὅτι ἐστὶ περὶ μυρίους 
σταδίους καὶ ὀκτακισχιλίους, ποιοίη δὲ τρισμυρίων τὸ 
ἀπὸ τῶν μεσημβρινῶν Ἰνδῶν μέχρι τῶν Ὀρῶν, ὅσα 
ἂν συμβαίη ἄτοπα. 



Pytheas Perieg., Fragmenta 
Fragment 9b, line 1

KOSMAS DER INDIENFAHRER Χριστιανικὴ 
τοπογραφία II p. 82, 18 E. O. Winstedt: <Πυθέας> δὲ ὁ 
<Μας<ς>αλιώτης> φησὶν <ἐν τοῖς Περὶ Ὠκεανοῦ> ὅτι   
παραγενομένωι αὐτῶι ἐν τοῖς βορειοτάτοις τόποις ἐδείκ-
νυον οἱ αὐτόθι βάρβαροι τὴν ἡλίου κοίτην, ὡς ἐκεῖ τῶν 
νυκτῶν ἀεὶ γενομένου παρ' αὐτοῖς. 



Maximus Confessor Theol., Quaestiones et dubia (2892: 002)
“Maximi confessoris quaestiones et dubia”, Ed. Declerck, J.H.
Turnhout: Brepols, 1982; Corpus Christianorum. Series Graeca 10.
Section 119, line 20

                                        Κατὰ δὲ τὴν θεωρίαν 
οὕτως ἐκληπτέον τὸν τόπον· ὁ γνωστικὸς νοῦς κατὰ τὸν 
Παῦλον ἔχει συνόντας αὐτῷ τοὺς λογισμούς· κατὰ οὖν τὴν 
πρώτην τοῦ λόγου πρὸς τὸν νοῦν γινομένην ἐμφάνειαν οἱ 
λογισμοὶ ἀπηχήματα μόνον καὶ ἰνδάλματα τῆς γνώσεως 
ἀκούουσιν, οὐδὲν δὲ τετρανωμένον θεωροῦσιν· προκόπτον-
τος δὲ τοῦ νοῦ καὶ ἐπὶ τῶν ἀναβαθμῶν γινομένου, 
τουτέστιν ἐπὶ τῆς ὑψηλῆς θεωρίας, οὐκέτι οἱ λογισμοὶ 
ἰνδαλμάτων ἀλλὰ τοῦ φωτὸς τῆς γνώσεως τέλειον ἐν 
μετουσίᾳ γίνονται. 



Maximus Confessor Theol., Scholia in Ecclesiasten (in catenis: catena trium patrum) (2892: 097)
“Anonymus in Ecclesiasten commentarius qui dicitur catena trium patrum”, Ed. Lucà, S.
Turnhout: Brepols, 1983; Corpus Christianorum. Series Graeca 11.
Section 1, line 65

                                            Τῶνδ' εἰρημένων 
 8. Πάντες οἱ λόγοι ἔγκοποι· οὐ δυνήσεται ἀνὴρ τοῦ 
λαλεῖν, καὶ οὐκ ἐμπλησθήσεται ὀφθαλμὸς τοῦ ὁρᾶν, καὶ οὐ 
πληρωθήσεται οὗς ἀπὸ ἀκροάσεως. 
 Δηλαδή, ὁ διὰ τελειότητα φρενῶν τῆς κατὰ τὴν ἄστατον 
φορὰν τῶν ὄντων ὑπεριδὼν ματαιότητος, καὶ τῆς τούτων 
φύσεως πάντας τοὺς καθόλου ἐρευνώμενος λόγους, πολλὰ 
καμών, μόλις νῷ μόνῳ ἐν ἄλλῃ ἐξ ἄλλου φαντασίᾳ ἕν τι 
συλλέξει τῆς ἀληθείας ἴνδαλμα, πρὶν κρατηθῆναι φεῦγον καὶ 
πρὶν νοηθῆναι διαδιδρᾶσκον. 

Socrates Scholasticus Hist., Historia ecclesiastica (2057: 001)
“Socrates' ecclesiastical history, 2nd edn.”, Ed. Bright, W.
Oxford: Clarendon Press, 1893.
Book 1, chapter 19, line 2

Τίνα τρόπον ἐπὶ τῶν χρόνων Κωνσταντίνου τὰ ἐνδοτέρω ἔθνη τῶν 
Ἰνδῶν ἐχριστιάνισαν.


 Αὖθις οὖν μνημονευτέον καὶ ὅπως ἐπὶ τῶν καιρῶν τοῦ βασιλέως 
ὁ Χριστιανισμὸς ἐπλατύνετο· τηνικαῦτα γὰρ Ἰνδῶν τε τῶν ἐνδο-
τέρω καὶ Ἰβήρων τὰ ἔθνη πρὸς τὸ Χριστιανίζειν ἐλάμβανε τὴν 
ἀρχήν. 



Socrates Scholasticus Hist., Historia ecclesiastica 
Book 1, chapter 19, line 10

                     Ἡνίκα οἱ ἀπόστολοι κλήρῳ τὴν εἰς τὰ ἔθνη 
πορείαν ἐποιοῦντο, Θωμᾶς μὲν τὴν Πάρθων ἀποστολὴν ὑπεδέχετο, 
Ματθαῖος δὲ τὴν Αἰθιοπίαν, Βαρθολομαῖος δὲ ἐκληροῦτο τὴν συνημ-
μένην ταύτῃ Ἰνδίαν· τὴν μέντοι ἐνδοτέρω Ἰνδίαν, ᾗ προσοικεῖ 
βαρβάρων ἔθνη πολλὰ, διαφόροις χρώμενα γλώσσαις, οὐδέ πω 
πρὸ τῶν Κωνσταντίνου χρόνων ὁ τοῦ Χριστιανισμοῦ λόγος ἐφώτιζε. 



Socrates Scholasticus Hist., Historia ecclesiastica 
Book 1, chapter 19, line 14

Μερόπιός τις φιλόσοφος, τῷ γένει Τύριος, ἱστορῆσαι τὴν Ἰνδῶν 
χώραν ἔσπευσεν, ἁμιλλησάμενος πρὸς τὸν φιλόσοφον Μητρόδωρον, 
ὃς πρὸ αὐτοῦ τὴν Ἰνδῶν χώραν ἱστόρησεν. 



Socrates Scholasticus Hist., Historia ecclesiastica 
Book 1, chapter 19, line 21

Συμβεβήκει δὲ τότε πρὸς ὀλίγον τὰς σπονδὰς διεσπᾶσθαι τὰς 
μεταξὺ Ῥωμαίων τε καὶ Ἰνδῶν. 



Socrates Scholasticus Hist., Historia ecclesiastica 
Book 1, chapter 19, line 21

                                     Συλλαβόντες δὲ οἱ Ἰνδοὶ τὸν 
φιλόσοφον καὶ τοὺς συμπλέοντας, πλὴν τῶν δύο συγγενῶν παιδα-
ρίων, ἅπαντας διεχρήσαντο· τοὺς δὲ δύο παῖδας, οἴκτῳ τῆς ἡλικίας 
διασώσαντες, δῶρον τῷ Ἰνδῶν βασιλεῖ προσκομίζουσιν. 



Socrates Scholasticus Hist., Historia ecclesiastica 
Book 1, chapter 19, line 38

                                                   Κατὰ βραχὺ δὲ προϊόντος 
τοῦ χρόνου, καὶ εὐκτήριον κατεσκεύασε, καί τινας τῶν Ἰνδῶν κατη-
χοῦντες συνεύχεσθαι αὐτοῖς παρεσκεύασαν. 



Socrates Scholasticus Hist., Historia ecclesiastica 
Book 1, chapter 19, line 48

                                Αἰδέσιος μὲν οὖν ὡς ἐπὶ τὴν Τύρον 
ἐσπουδάζεν, ὀψόμενος γονέας τε καὶ συγγενεῖς· Φουρμέντιος δὲ 
καταλαβὼν τὴν Ἀλεξάνδρειαν, τῷ ἐπισκόπῳ Ἀθανασίῳ τότε 
νεωστὶ τῆς ἐπισκοπῆς ἀξιωθέντι, ὑπαναφέρει τὸ πρᾶγμα, διδάξας 
τά τε τῆς ἑαυτοῦ ἀποδημίας, καὶ ὡς ἐλπίδας ἔχουσιν Ἰνδοὶ τὸν 
Χριστιανισμὸν παραδέξασθαι· ἐπίσκοπόν τε καὶ κλῆρον ἀποστέλ-
λειν, καὶ μηδαμῶς περιορᾷν τοὺς δυναμένους σωθῆναι. 



Socrates Scholasticus Hist., Historia ecclesiastica 
Book 1, chapter 19, line 54

                  Γίνεται δὴ τοῦτο· καὶ Φουρμέντιος ἀξιωθεὶς τῆς 
ἐπισκοπῆς αὖθις ἐπὶ τὴν Ἰνδῶν παραγίνεται χώραν, καὶ κῆρυξ 
τοῦ Χριστιανισμοῦ γίνεται, εὐκτήριά τε πολλὰ ἱδρύει, ἀξιωθείς τε 
θείας χάριτος, πολλὰ μὲν σημεῖα, πολλῶν δὲ σὺν τῇ ψυχῇ καὶ τὰ 
σώματα ἐθεράπευε. 



Sextus Empiricus Phil., Pyrrhoniae hypotyposes (0544: 001)
“Sexti Empirici opera, vol. 1”, Ed. Mutschmann, H.
Leipzig: Teubner, 1912.
Book 1, section 80, line 2

                   διαφέρει μὲν γὰρ κατὰ μορφὴν σῶμα 
Σκύθου Ἰνδοῦ σώματος, τὴν δὲ παραλλαγὴν ποιεῖ, καθάπερ 
φασίν, ἡ διάφορος τῶν χυμῶν ἐπικράτεια. 



Sextus Empiricus Phil., Pyrrhoniae hypotyposes 
Book 1, section 80, line 7

     ταῦτά τοι καὶ ἐν τῇ αἱρέσει καὶ φυγῇ τῶν ἐκτὸς διαφο-
ρὰ πολλὴ κατ' αὐτούς ἐστιν· ἄλλοις γὰρ χαίρουσιν Ἰνδοὶ 
καὶ ἄλλοις οἱ καθ' ἡμᾶς, τὸ δὲ διαφόροις χαίρειν τοῦ 
παρηλλαγμένας ἀπὸ τῶν ὑποκειμένων φαντασίας λαμβά-
νειν ἐστὶ μηνυτικόν. 



Sextus Empiricus Phil., Pyrrhoniae hypotyposes 
Book 1, section 148, line 6

οἷον ἔθος μὲν ἔθει οὕτως· τινὲς τῶν Αἰθιόπων στίζουσι τὰ 
βρέφη, ἡμεῖς δ' οὔ· καὶ Πέρσαι μὲν ἀνθοβαφεῖ ἐσθῆτι καὶ 
ποδήρει χρῆσθαι νομίζουσιν εὐπρεπὲς εἶναι, ἡμεῖς δὲ ἀπρε-
πές· καὶ οἱ μὲν Ἰνδοὶ ταῖς γυναιξὶ δημοσίᾳ μίγνυνται, οἱ δὲ 
πλεῖστοι τῶν ἄλλων αἰσχρὸν τοῦτο εἶναι ἡγοῦνται. 



Sextus Empiricus Phil., Pyrrhoniae hypotyposes 
Book 3, section 200, line 6

                       καὶ τὸ δημοσίᾳ γυναικὶ μίγνυσθαι, 
καίτοι παρ' ἡμῖν αἰσχρὸν εἶναι δοκοῦν, παρά τισι τῶν Ἰν-
δῶν οὐκ αἰσχρὸν εἶναι νομίζεται· μίγνυνται γοῦν ἀδια-
φόρως δημοσίᾳ, καθάπερ καὶ περὶ τοῦ φιλοσόφου Κράτητος 
ἀκηκόαμεν. 



Sextus Empiricus Phil., Pyrrhoniae hypotyposes 
Book 3, section 227, line 3

           Αἰθιόπων δὲ οἱ ἰχθυοφάγοι εἰς τὰς λίμνας ἐμ-
βάλλουσιν αὐτούς, ὑπὸ τῶν ἰχθύων βρωθησομένους· 
Ὑρκανοὶ δὲ κυσὶν αὐτοὺς ἐκτίθενται βοράν, Ἰνδῶν δὲ 
ἔνιοι γυψίν. 



Sextus Empiricus Phil., Adversus mathematicos (0544: 002)
“Sexti Empirici opera, vols. 2 \& 3 (2nd edn.)”, Ed. Mutschmann, H., Mau, J.
Leipzig: Teubner, 2:1914; 3:1961.
Book 7, section 249, line 7

                                                      δεύτερον δὲ τὸ 
καὶ ἀπὸ ὑπάρχοντος εἶναι καὶ κατ' αὐτὸ τὸ ὑπάρχον· 
ἔνιαι γὰρ πάλιν ἀπὸ ὑπάρχοντος μέν εἰσιν, οὐκ αὐτὸ δὲ 
τὸ ὑπάρχον ἰνδάλλονται, ὡς ἐπὶ τοῦ μεμηνότος Ὀρέστου 
μικρῷ πρότερον ἐδείκνυμεν. 



Sextus Empiricus Phil., Adversus mathematicos 
Book 7, section 425, line 4

               τοῦτο μέντοι τῶν ἀδυνάτων ἐστίν· καὶ 
γὰρ παρὰ τὰς διαφορὰς τῶν τόπων καὶ παρὰ τὰς τοῦ 
ἐκτὸς περιστάσεις καὶ παρ' ἄλλους πλείονας τρόπους οὔτε 
τὰ αὐτὰ οὔτε ὡσαύτως ἰνδάλλεται ἡμῖν τὰ πράγματα, κα-
θάπερ ἀνώτερον ἐπελογισάμεθα, ὥστε ὅτι μὲν 
φαίνεται πρὸς τῇδε τῇ αἰσθήσει καὶ τῇδε τῇ περιστάσει δύ-
νασθαι λέγειν, τὸ δ' εἰ ταῖς ἀληθείαις τοιοῦτόν ἐστιν οἷον 
καὶ φαίνεται, ἢ ἀλλοῖον μέν ἐστιν, ἀλλοῖον δὲ φαίνεται, μὴ 
ἔχειν ἡμᾶς διαυθεντεῖν, διὰ δὲ τοῦτο μηδεμίαν εἶναι φαν-
τασίαν χωρὶς ἐνστήματος. 



Sextus Empiricus Phil., Adversus mathematicos 
Book 9, section 45, line 3

Οἱ δὲ καὶ πρὸς τοῦτό φασιν, ὅτι ἡ μὲν ἀρχὴ τῆς νοή-
σεως τοῦ εἶναι θεὸν γέγονεν ἀπὸ τῶν κατὰ τοὺς ὕπνους 
ἰνδαλλομένων ἢ ἀπὸ τῶν κατὰ τὸν κόσμον θεωρουμένων, 
τὸ δὲ ἀίδιον εἶναι τὸν θεὸν καὶ ἄφθαρτον καὶ τέλειον ἐν 
εὐδαιμονίᾳ παρῆλθε κατὰ τὴν ἀπὸ τῶν ἀνθρώπων μετά-
βασιν. 



Sextus Empiricus Phil., Adversus mathematicos 
Book 11, section 15, line 5

Ἄλλοι δὲ κἀκείνως ἐνέστησαν· πᾶσα γάρ, φασίν, ὑγιὴς 
διαίρεσις γένους ἐστὶ τομὴ εἰς τὰ προσεχῆ εἴδη, καὶ διὰ 
τοῦτο μοχθηρὰ καθέστηκεν ἡ τοιαύτη διαίρεσις· “τῶν ἀν-
θρώπων οἱ μέν εἰσιν Ἕλληνες, οἱ δὲ Αἰγύπτιοι, οἱ δὲ Πέρ-
σαι, οἱ δὲ Ἰνδοί”. 



Sextus Empiricus Phil., Adversus mathematicos 
Book 11, section 16, line 1

                        τῷ γὰρ ἑτέρῳ τῶν προσεχῶν εἰδῶν οὐ 
τὸ συζυγοῦν καὶ προσεχὲς εἶδος ἀντιδιέζευκται, ἀλλὰ τὰ 
τούτου εἴδη, δέον οὕτως εἰπεῖν· “τῶν ἀνθρώπων οἱ μέν 
εἰσιν Ἕλληνες, οἱ δὲ βάρβαροι”, καὶ καθ' ὑποδιαίρεσιν λοι-
πόν “τῶν δὲ βαρβάρων οἱ μέν εἰσιν Αἰγύπτιοι, οἱ δὲ Πέρσαι, 
οἱ δὲ Ἰνδοί”. 



Sextus Empiricus Phil., Adversus mathematicos 
Book 11, section 17, line 4

                ἐῴκει γὰρ ἡ μὲν τοιαύτη διαίρεσις τῇ λε-
γούσῃ “τῶν ἀνθρώπων οἱ μέν εἰσιν Ἕλληνες, οἱ δὲ βάρ-
βαροι, τῶν δὲ βαρβάρων οἱ μὲν Αἰγύπτιοι, οἱ δὲ Πέρσαι, 
οἱ δὲ Ἰνδοί”· ἡ δὲ ἐκκειμένη ὡμοίωτο τῇ τοιουτοτρόπῳ· 
“τῶν ἀνθρώπων οἱ μέν εἰσιν Ἕλληνες, οἱ δὲ Αἰγύπτιοι, οἱ 
δὲ Πέρσαι, οἱ δὲ Ἰνδοί”. 



Sextus Empiricus Phil., Adversus mathematicos 
Book 11, section 20, line 3

                                  περὶ μὲν γὰρ τῆς πρὸς τὴν 
φύσιν ὑποστάσεως τῶν τε ἀγαθῶν καὶ κακῶν καὶ οὐδε-
τέρων ἱκανοί πώς εἰσιν ἡμῖν ἀγῶνες πρὸς τοὺς δογματι-
κούς· κατὰ δὲ τὸ φαινόμενον τούτων ἕκαστον ἔχομεν 
ἔθος ἀγαθὸν ἢ κακὸν ἢ ἀδιάφορον προσαγορεύειν, καθά-
περ καὶ ὁ Τίμων ἐν τοῖς ἰνδαλμοῖς ἔοικε 
δηλοῦν, ὅταν φῇ· 
  ἦ γὰρ ἐγὼν ἐρέω, ὥς μοι καταφαίνεται εἶναι, 
  μῦθον ἀληθείης ὀρθὸν ἔχων κανόνα, 
  ὡς ἡ τοῦ θείου τε φύσις καὶ τἀγαθοῦ αἰεί, 
  ἐξ ὧν ἰσότατος γίνεται ἀνδρὶ βίος. 



Sextus Empiricus Phil., Adversus mathematicos 
Book 11, section 122, line 10

                         ὁ ἄρα τὸν πλοῦτον μέγιστον ἀγα-
θὸν ἰνδαλλόμενος ἐν τῷ σπεύδειν ἐπὶ τοῦτον γίνεται φι-
λάργυρος. 



Eustathius Philol., Scr. Eccl., Commentarii ad Homeri Iliadem (4083: 001)
“Eustathii archiepiscopi Thessalonicensis commentarii ad Homeri Iliadem pertinentes, vols. 1–4”, Ed. van der Valk, M.
Leiden: Brill, 1:1971; 2:1976; 3:1979; 4:1987.
Volume 1, page 309, line 11

           Αὕτη ἡ ἀρχή, ἡ βασιλικὴ δηλαδή, καὶ τὰς θρυλλουμένας ποτὲ 
μεγάλας ἀρχὰς κατέστησε, τὴν τῶν Περσῶν, τὴν τῶν Ἰνδῶν, τὴν τῶν Ῥω-
μαίων· καὶ οὐδεμία τις ἀρχὴ ἐπὶ τοσοῦτον ἦλθε λόγου, ὅσον ἡ βασιλική. 



Eustathius Philol., Scr. Eccl., Commentarii ad Homeri Iliadem 
Volume 1, page 319, line 23

ὑπερφρονεῖ γὰρ τοῦ φαυλοτάτου Θερσίτου, καθὰ καὶ τὸν Ἀλεξάνδρου φασὶ 
κύνα τὰ μὲν μικρὰ θηρία παριέναι, ὁρμᾶν δὲ ἐπὶ τὰ κρείττω, ὅπερ ὁ Θεμίστιος 
ἐπὶ κυνῶν Ἰνδικῶν οὕτω φράζει· «κύων Ἰνδὸς λέοντι μὲν ἐπέξεισι καὶ παρδάλει, 
λύκους δὲ ὑπερορᾷ καὶ ἀλώπεκας». 



Eustathius Philol., Scr. Eccl., Commentarii ad Homeri Iliadem 
Volume 1, page 479, line 25

                Καὶ ἡ Βουκεφάλα δὲ τῆς τοιαύτης ἐφόδου ἐστί, πόλις Ἰνδικὴ ἐκείνη, 
κτισθεῖσα ὑπ' Ἀλεξάνδρου ἐπ' ἀμφοτέραις, φασί, ταῖς ὄχθαις τοῦ ποταμοῦ 
Ὑδάσπου, διότι διαβάντος Ἀλεξάνδρου ἐκεῖ καὶ μαχομένου ὁ Βουκεφάλας 
ἵππος ἀπέθανε. 



Eustathius Philol., Scr. Eccl., Commentarii ad Homeri Iliadem 
Volume 1, page 481, line 19

                                            Ὄρος δέ τι Ἰνδικὸν Μηρὸς ἐκλήθη, 
Διονύσῳ ἀνακείμενον, ὅθεν μηροτραφὴς μεμύθευται, φασίν, ὁ Διόνυσος. 



Eustathius Philol., Scr. Eccl., Commentarii ad Homeri Iliadem 
Volume 1, page 518, line 26

                                                         ἔστι δὲ καὶ Ἀστερουσία 
Κρήτης, φασίν, ὄρος καὶ πόλις δὲ περὶ τὸν Ἰνδικὸν Καύκασον. 



Eustathius Philol., Scr. Eccl., Commentarii ad Homeri Iliadem 
Volume 1, page 569, line 4

                ἐν δὲ τῇ Ἰνδικῇ πολύς που ὁ ἔβενος. 



Eustathius Philol., Scr. Eccl., Commentarii ad Homeri Iliadem 
Volume 1, page 606, line 24

                                     καὶ κύων δὲ εὐγενὴς ὑπηγάγετό τινα εἰς ὅμοιον 
πάθος, ὁποία τις ἡ τοῦ Τρωϊκοῦ Φοινοδάμαντος, ἣν Κρημισσὸς ὁ ποταμὸς 
λέγεται ζεῦξαι λέκτροις »ἰνδαλθεὶς κυνί». 



Eustathius Philol., Scr. Eccl., Commentarii ad Homeri Iliadem 
Volume 1, page 750, line 17

                                                               καὶ τοῦτό ἐστι τὸ ἐπὶ 
τριῶν ταχθῆναι τὴν φάλαγγα, ὃ δή φησιν Ἀρριανὸς ἐν τῇ Ἰνδικῇ, ἤγουν ἐπὶ 
τρεῖς ὡς εἰπεῖν στίχους ἢ ὀρδίνους. 



Eustathius Philol., Scr. Eccl., Commentarii ad Homeri Iliadem 
Volume 2, page 4, line 14

                                  κινδυνεύων γὰρ ἐκεῖνός που ἐν Ἰνδίᾳ ἐτινάξατο 
τοῖς ὅπλοις κατὰ τὴν ἱστορίαν. 



Eustathius Philol., Scr. Eccl., Commentarii ad Homeri Iliadem 
Volume 2, page 261, line 2

                                                                Εἰσὶ δὲ καὶ ἕτεραι 
Νύσσαι κατὰ τὰς ἱστορίας, αἱ μὲν ὄρη, ὡς ἐν Βοιωτίᾳ καὶ Ἀραβίᾳ καὶ Ἰνδικῇ 
καὶ Λιβύῃ, ἡ δέ τις καὶ νῆσος ἐν τῷ Καυκασίῳ ὄρει καὶ τῷ ποταμῷ Νείλῳ. 



Eustathius Philol., Scr. Eccl., Commentarii ad Homeri Iliadem 
Volume 2, page 561, line 12

                                                                          Παρὰ μέντοι 
τῷ Γεωγράφῳ δοκεῖ διαφορά τις εἶναι σίτου καὶ πυροῦ, ἔνθα λέγει γίνεσθαι 
ἐν Ἰνδίᾳ σῖτον αὐτοφυῆ ἐοικότα πυρῷ. 



Eustathius Philol., Scr. Eccl., Commentarii ad Homeri Iliadem 
Volume 2, page 744, line 12

                  Ὁμοίως καὶ τὸ «ὄστρακον φέρει τὸν θαυμαζόμενον μαργαρί-
την λίθον, ὃ καλοῦσιν Ἰνδοὶ βέρβερι. 



Eustathius Philol., Scr. Eccl., Commentarii ad Homeri Iliadem 
Volume 3, page 42, line 22

                                    ὁποῖοι καὶ οἱ παρ' Ἰνδοῖς κύνες, οὓς ἄλλοι 
τε ἱστοροῦσι καὶ ὁ σοφὸς Θεμίστιος, γράψας, ὡς καὶ ἀλλαχοῦ δηλοῦται, κύνα 
Ἰνδὸν λέοντι μὲν ἐπεξιέναι καὶ παρδάλει, λύκους δὲ ὑπερορᾶν καὶ ἀλώπεκας. 



Eustathius Philol., Scr. Eccl., Commentarii ad Homeri Iliadem 
Volume 3, page 225, line 2

           ὡς δὲ καὶ ἑτεροῖον ὄστρακον χαλαζᾷ κατὰ τὸν Κωμικὸν εἰπεῖν 
χαλάζαις τοιαύταις, καὶ ὅτι βέρβερι παρὰ Ἰνδοῖς ἐκεῖνο καλεῖται ἀλλαχοῦ 
δηλοῦται σαφῶς. 



Eustathius Philol., Scr. Eccl., Commentarii ad Homeri Iliadem 
Volume 3, page 621, line 5

                                                            Ὅρα δὲ ὡς Ἥρα μὲν 
ἁπλῶς οὕτω φιλότητα καὶ ἵμερον ἐξ Ἀφροδίτης ἐζήτησε τὰ κατὰ μῦθον ἰνδάλ-
ματα, Ὅμηρος δὲ καὶ τὸ αὐτοῖς ὑποκείμενον ἡρμήνευσε, τὸν κεστὸν ἱμάντα, 
καὶ πλεῖον δέ τι ἐδήλωσεν αὐτὸν ἔχειν, ὅπερ ἦν ἡ ὀαριστὺς πάρφασις. 



Eustathius Philol., Scr. Eccl., Commentarii ad Homeri Iliadem 
Volume 3, page 727, line 2

                                                                               Ἰδαῖα δὲ   
ὄρη τὴν Ἴδην φησίν, ἣν καὶ αὐτὴν πληθύνει τὸ ἐν αὐτοῖς μέγεθος, καθὰ 
καὶ τὰ Ἠμωδὰ Ἰνδίας ὄρη καὶ τὰ Ἀρμένια καὶ τὰς Ἄλπεις καὶ ἄλλα. 



Eustathius Philol., Scr. Eccl., Commentarii ad Homeri Iliadem 
Volume 3, page 804, line 17

               ἐκεῖνο δὲ καὶ ἀστεῖον, εἴπερ ἐνταῦθα μὲν Ἀχιλλέως εἰκασμὸς 
ἐκφοβεῖ τοὺς Τρῶας, ἐν τοῖς ἑξῆς δὲ τοὺς αὐτοὺς ὁ αὐτὸς εἰκασμὸς τέρπει, 
ὅτε δηλαδὴ αὐτὰ Ἕκτωρ φορῶν ἰνδάλλεται τῷ Ἀχιλλεῖ. 



Eustathius Philol., Scr. Eccl., Commentarii ad Homeri Iliadem 
Volume 4, page 40, line 20

                                      μετὰ δὲ κλειτοὺς ἐπικούρους ἔβη μέγα ἰάχων. 
ἰνδάλλετο δέ σφισι πᾶσι τεύχεσι λαμπόμενος μεγαθύμου Πηλείωνος», ἤγουν 
εἰκασμὸν ἑαυτοῦ ἔπεμπε τοῖς βλέπουσιν καὶ οὐκ ἐμφανῶς ἐγινώσκετο Ἕκτωρ 
εἶναι, οἷα μὴ τὰ ἑαυτοῦ φορῶν, ἀλλὰ τοῖς Ἀχιλλέως ὅπλοις λαμπόμενος. 



Eustathius Philol., Scr. Eccl., Commentarii ad Homeri Iliadem 
Volume 4, page 40, line 23

ἕτεροι δέ φασιν, ὅτι μεγαθύμῳ Πηλείωνι ἰνδάλλετο, ἀπὸ κοινοῦ τὴν τοιαύτην 
λαβόντες δοτικήν. 



Eustathius Philol., Scr. Eccl., Commentarii ad Homeri Iliadem 
Volume 4, page 47, line 19

                   (v. 262) Ἐπάγει γὰρ «Τρῶες δὲ προὔτυψαν ἀολλέες, ἦρχε δ' 
ἂρ Ἕκτωρ», ποθῶν τε, ὡς εἰκός, ἀντιστῆναι τῷ Αἴαντι, καὶ ἵνα μηδὲ τὸ πρὸς 
Ἀχιλλέα ἴνδαλμα αἰσχύνῃ ἀχρειώσας, εἰ μὴ προμάχοιτο. 



Eustathius Philol., Scr. Eccl., Commentarii ad Homeri Iliadem 
Volume 4, page 110, line 6

                                                               (v. 698 s.) Ὅτι 
Ἀντίλοχος θέων εἰς Ἀχιλλέα τὰ οἰκεῖα τεύχεα ἔδωκεν ἀμύμονι Λαοδόκῳ 
ἑταίρῳ, οὐ μόνον ἵνα ἐλαφρῶς ἔχῃ θέειν, ὡς καὶ Ὀδυσσεὺς ἐν τῇ βῆτα ῥαψῳδίᾳ 
τὴν χλαῖναν ἀποβαλών, ἀλλὰ καὶ ἵνα ἰνδαλλόμενος δοκῇ παραμένειν τῷ 
πολέμῳ, καθά φασιν οἱ παλαιοί. 



Eustathius Philol., Scr. Eccl., Commentarii ad Homeri Iliadem 
Volume 4, page 111, line 13

                                                                           πρὸς μέντοι ταῖς 
ναυσὶ πρὸ τῆς τοιαύτης ὁπλοποιΐας ἐπιφανεὶς μόνον καὶ οὐ πέμψας ἴνδαλμα 
ἑαυτοῦ ὡς ἐπὶ Πατρόκλου γέγονεν, ἀλλ' ἑαυτὸν ἐκφήνας, τρέψεται τοὺς Τρῶας 
εἰς φυγήν. 



Eustathius Philol., Scr. Eccl., Commentarii ad Homeri Iliadem 
Volume 4, page 216, line 5

                                                                         Σημείωσαι δὲ καὶ ὅτι 
τοσαῦτα ποιήσας Ἥφαιστος δαίδαλα οὐδαμοῦ ναυτιλίας ἴνδαλμα θέμενος 
ἔτευξεν. 



Eustathius Philol., Scr. Eccl., Commentarii ad Homeri Iliadem 
Volume 4, page 468, line 12

                  (v. 107 s.) Σημείωσαι δὲ καὶ ὅτι ἐν τῷ τόπῳ τούτῳ, καθὰ καί 
τινι σμικροτάτῳ ἐνόπτρῳ, ἴνδαλμά τι ὅλου ἐγκωμίου ἐμφαίνεται. 



Eustathius Philol., Scr. Eccl., Commentarii ad Homeri Iliadem 
Volume 4, page 484, line 4

                Σημείωσαι δὲ ὅτι ἔδει μὲν ἐν τοιούτῳ τόπῳ συγκρίσεως τὸν 
ποιητὴν μὴ Ἀχελῴου μνησθῆναι, ἀλλά τινος ἑτέρου τῶν μειζόνων ἢ 
μεγίστων, ὁποῖοι πολλοὶ ποταμοί, ὁ Ἰνδός, ὁ Ἴστρος, ὁ Νεῖλος, ὃν Αἴγυπτον ὁ 
ποιητὴς ἐν Ὀδυσσείᾳ καλεῖ. 



Eustathius Philol., Scr. Eccl., Commentarii ad Homeri Iliadem 
Volume 4, page 764, line 8

                                        (v. 459 – 64) «Ἄλλοι μοι δοκέουσι παροίτεροι 
ἔμμεναι ἵπποι, ἄλλος δ' ἡνίοχος ἰνδάλλεται· αἳ δέ που αὐτοῦ ἔβλαβεν ἐν 
πεδίῳ», ἤγουν ἐβλάβησαν, «αἳ κεῖσε φέρτεραι ἦσαν· ἤτοι γὰρ τάς», ἤγουν 
ταύτας, «πρῶτα ἴδον περὶ τέρμα βαλούσας, νῦν δ' οὔ πῃ δύναμαι ἰδέειν», καὶ 
ἑξῆς, ὡς προγέγραπται. 



Eustathius Philol., Scr. Eccl., Commentarii ad Homeri Iliadem 
Volume 4, page 765, line 9

                                                  (v. 459 s.) Ὅρα δὲ καὶ ὅτι τὸ 
δοκεῖν ταὐτὸν εἶναι δοξάζει Ὅμηρος τῷ ἰνδάλλεσθαι ἐν τῷ «ἄλλοι μοι 
δοκέουσι», καὶ «ἄλλος ἰνδάλλεται». 



Eustathius Philol., Scr. Eccl., Commentarii ad Homeri Iliadem 
Volume 4, page 765, line 11

                                            λέγεται δὲ ἐπὶ εἰκασμοῦ καὶ ὁμοιώσεως καὶ 
δοκήσεως τὸ ἰνδάλλεσθαι, ἀφ' οὗ καὶ ἐπὶ εἰδώλων τὸ ἴνδαλμα λέγεται. 



Eustathius Philol., Scr. Eccl., Commentarii ad Homeri Iliadem 
Volume 4, page 765, line 12

                                  ἀντίκειται δὲ πρὸς τὸ ἰνδάλλεσθαι καὶ τὸ δοκεῖν τὸ 
εὖ διαγινώσκειν. 



Eustathius Philol., Scr. Eccl., Commentarii ad Homeri Iliadem 
Volume 4, page 765, line 13

                    γίνεται δὲ τὸ ἰνδάλλω ἐκ τοῦ εἴδω, τὸ ὁμοιῶ, τραπέντος τοῦ <ι> 
εἰς <ν> ὡς ἐπὶ τοῦ αἰεί αἰέν, καὶ αὐτίκα τετραμμένου καὶ τοῦ <ε> εἰς <ι>, ὡς ἐπὶ τοῦ ἔχω 
ἴσχω καὶ τῶν ὁμοίων γίνεται. 



Eustathius Philol., Scr. Eccl., Commentarii ad Homeri Iliadem 
Volume 4, page 766, line 6

                                (v. 459 s. et v. 467) Οὐ μόνον γοῦν φησιν, 
Ἰδομενεύς, ὡς «ἄλλοι μοι δοκέουσι παροίτεροι ἔμμεναι ἵπποι», καὶ «ἄλλος 
ἡνίοχος ἰνδάλλεται», ἀλλὰ καὶ ὅτι ἐκπεσέειν ὀΐω τὸν Εὔμηλον καὶ ἑξῆς. 



Eustathius Philol., Scr. Eccl., Commentarii ad Homeri Odysseam (4083: 003)
“Eustathii archiepiscopi Thessalonicensis commentarii ad Homeri Odysseam, 2 vols. in 1”, Ed. Stallbaum, G.
Leipzig: Weigel, 1:1825; 2:1826, Repr. 1970.
Volume 1, page 25, line 41

                                                                    καθὰ καὶ κύων φασὶν Ἰνδὸς λέοντι ἐπε-
ξιὼν καὶ παρδάλει, λύκους ὑπερορᾷ καὶ ἀλώπεκας. 



Eustathius Philol., Scr. Eccl., Commentarii ad Homeri Odysseam 
Volume 1, page 124, line 8

                                                                                   ὥς τέ μοι ἀθανάτοις ἰνδάλ-
λεται εἰσοράασθαι. 



Eustathius Philol., Scr. Eccl., Commentarii ad Homeri Odysseam 
Volume 1, page 124, line 10

                                                       Καὶ ὅρα τὸ ἰνδάλλεται. 



Eustathius Philol., Scr. Eccl., Commentarii ad Homeri Odysseam 
Volume 1, page 124, line 11

                                                                                     οὐ γὰρ θεὸς ἄντικρυς ὁ ἐπὶ 
μακρὸν βιοὺς, ἀλλὰ θεοῦ ἴνδαλμα. 



Eustathius Philol., Scr. Eccl., Commentarii ad Homeri Odysseam 
Volume 1, page 124, line 12

         μακρόβιος δὲ ὁ Νέστωρ, ὥστε καὶ ἴνδαλμά τι ἔχει πρὸς θεόν. 



Eustathius Philol., Scr. Eccl., Commentarii ad Homeri Odysseam 
Volume 1, page 149, line 35

                                                    καὶ ὅτι Μενέλαος πρὸς Αἰθίοπας ἦλθε κατὰ μέν τινας, 
περιπλεύσας τὸν Ὠκεανὸν διὰ τῶν Γαδείρων μέχρι τῆς Ἰνδικῆς. 



Eustathius Philol., Scr. Eccl., Commentarii ad Homeri Odysseam 
Volume 1, page 150, line 14

                                                                                                  τινὲς δὲ Ἐρεμ-
βοὺς, τοὺς Ἰνδοὺς νοοῦσι παρὰ τὸ ἔρεβος. 



Eustathius Philol., Scr. Eccl., Commentarii ad Homeri Odysseam 
Volume 1, page 151, line 43

     καὶ τὰ μὲν Ἰνδικὰ, σιγάσθω. 



Eustathius Philol., Scr. Eccl., Commentarii ad Homeri Odysseam 
Volume 1, page 251, line 19

                                                                                                                      ὅ ἐστι 
μελαίνας κατὰ τὸν ὑάκινθον τὸ ἄνθος, ὁποίας καὶ τοῖς Ἰνδοῖς ὁ περιηγητὴς χρώζει τὰς κόμας. 



Eustathius Philol., Scr. Eccl., Commentarii ad Homeri Odysseam 
Volume 1, page 437, line 18

                                            ἦν δὲ αὕτη παρ' Ἰνδοῖς πόσις ἐν ποτηρίοις μεγάλοις, ἐξ ὧν 
ἐκεῖνοι καθεύδειν μέλλοντες ἔπινον, κύπτοντες ἐπὶ τὰς φιάλας ὡς λέγεται. 



Eustathius Philol., Scr. Eccl., Commentarii ad Homeri Odysseam 
Volume 1, page 437, line 20

                                                                                      φασὶ δὲ καὶ ὑπὲρ φιλότητος 
τὸ ποτὸν τοῦτο εὑρῆσθαι τοῖς Ἰνδοῖς. 



Eustathius Philol., Scr. Eccl., Commentarii ad Homeri Odysseam 
Volume 1, page 437, line 22

                                                                                                             Καὶ ὅρα τὸ 
περιώνυμον τοῦ Ταντάλου εἰ καὶ Ἰνδοῖς ἔγνωσται. 



Eustathius Philol., Scr. Eccl., Commentarii ad Homeri Odysseam 
Volume 2, page 148, line 7

                                                                                         Σημείωσαι δὲ, ὡς εἰ καὶ 
ἐπίσημα κυνῶν παραδέδονται γένη, Ἰνδοὶ, Ὑρκανοὶ, Μολοσσοὶ, πρὸς δὲ ἄλλοις καὶ Λακωνικοὶ, ὡς 
Πίνδαρος οἶδεν, εἰπὼν Λάκαιναν ἐπὶ θηρσὶ κύνα τρέφειν πυκινώτατον ἑρπετὸν, ὅμως ἡ φύσις καὶ ἐν 
τοῖς ἄλλοις ἅπασι τόποις διασπείρει κύνας ἀγαθοὺς, ὁποῖός τις καὶ ὁ Ἄργος. 



Eustathius Philol., Scr. Eccl., Commentarii ad Homeri Odysseam 
Volume 2, page 163, line 1

            ἄριστα οὖν τοῦτό γε φρονοῦσιν Ἰνδοί· παρ' οἷς, ὡς ἱστορεῖ Κτησίας, οὐκ ἔστι τῷ βασιλεῖ 
μεθυσθῆναι. 



Eustathius Philol., Scr. Eccl., Commentarii ad Homeri Odysseam 
Volume 2, page 175, line 27

                                       ἐκεῖνο μὲν γὰρ μέλαν ἐπιφαίνεται, καθὰ δηλοῖ καὶ ὁ εἰπὼν αὐτὸν 
Ἰνδὸν θῆρα κελαινόῤῥιον. 



Eustathius Philol., Scr. Eccl., Commentarii ad Homeri Odysseam 
Volume 2, page 199, line 46

                                                                            ἀκκισάμενος οὖν πρῶτον καὶ εἰπὼν, 
ὡς ἀργαλέον τόσσον χρόνον ἀμφὶς ἐόντα εἰπέμεν, ἤδη γάρ οἱ ἐεικοστὸν ἔτος ἐστὶν ἐκ τότε, εἶτα ὑποσχό-
μενος εἰπεῖν ἐν τῷ, (Vers. 224.) ἀλλὰ καὶ ὣς ἐρέω ὥς μοι ἰνδάλλεται ἦτορ, τουτέστι φαντάζεται   
ἀνειδωλοποιεῖται, ἅ περ εἴποι ἂν ὁ δυσχερῶς μεμνημένος τινὸς, ἐπιφέρει· χλαῖναν πορφυρέην οὔλην 
ἔχε δῖος Ὀδυσσεὺς, διπλῆν. 



Eustathius Philol., Scr. Eccl., Commentarii ad Homeri Odysseam 
Volume 2, page 240, line 41

                                                               ἐδήλου δὲ ταῦτα τῷ μαντικοῖς ὄμμασι βλέ-
ποντι χύσεις αἱμάτων καὶ γόους καὶ δάκρυα, ὧν πειράσονται οἱ κακοί· ὥς που καὶ ἡ παρὰ Λυκόφρονι 
Κασάνδρα οἰμωγὴν αὐτῇ λέγει ἰνδάλλεσθαι μαντευομένῃ τὰ ὕστερον τῶν Τρώων κακά. 



Eustathius Philol., Scr. Eccl., Commentarii ad Homeri Odysseam 
Volume 2, page 333, line 34

                                              sqq.) Ὅτι βραχύτατόν τι τῶν κατὰ τὴν Ἰλιάδα ἴνδαλμα 
σεμνῶς ἐμφαίνων ἐνταῦθα ὁ ποιητὴς πλάττει τὸν Ὀδυσσέα οὕτω σφοδρὸν ἐμπεσόντα τοῖς προμάχοις 
ἐπικουρούσης Ἀθηνᾶς, ὡς μὴ ἂν ἄλλως ἐκεῖνον παύσασθαι, εἰ μὴ Διὸς τὴν Ἀθηνᾶν ἐκφοβήσαντος 
ἐπεσχέθη αὐτὸς ὑπ' ἐκείνης τοῦ πολεμεῖν. 



Eustathius Philol., Scr. Eccl., De capta Thessalonica (4083: 004)
“Eustazio di Tessalonica. La espugnazione di Tessalonica”, Ed. Kyriakidis, S.
Palermo: Istituto Siciliano di Studi Bizantini e Neoellenici, 1961; Testi e Monumenti. Istituto Siciliano di Studi Bizantini e Neoellenici. Testi 5.
Page 64, line 9

Καὶ ἦσαν οὕτως αὐτῷ οἱ ἱππόται οἷοι ἀλαζονεύεσθαι κατὰ τὴν ἐν αὐτοῖς φύ-
σιν τριηκοσίων ἀνδρῶν ἕκαστος ἄντα κατὰ πόλεμον στήσεσθαι, οὐδὲν ἀπεοι-
κότες οὐδ' αὐτοὶ τοῦ Κομνηνοῦ, ὃς μόνος ἐδόξαζε τὴν τοσαύτην βασιλείαν 
ταχὺ καταλήψεσθαι, βραχὺ κατ' αὐτῆς παρακαλπάσας τὸν ἵππον, καὶ κατα-
κτήσεσθαι αὐτὴν χειρωσάμενος ὡσεὶ καὶ στρουθοῦ φωλεόν, λόγῳ μὲν τῷ 
Σικελῷ, ὃν καὶ γνησίως αὐθέντην ἐπεγράφετο, ψυχῇ δὲ ἑαυτῷ. Ἰνδάλλετο 
γάρ, οὐκ οἴδαμεν ὅπως, καθὰ καὶ προεξεθέμεθα, ἅμα τε ἐκφαίνεσθαί που καὶ 
πάντας εὐθὺς τοὺς Ῥωμαίων ὀφθαλμοὺς εἰς αὐτὸν ὡς ἥλιον ἐπιστρέφεσθαι 
καὶ αὐτοῦ μόνου γίνεσθαι. 



Eustathius Philol., Scr. Eccl., De capta Thessalonica 
Page 140, line 3

                                                                           Δοτέον 
μοι τοὺς αὐτοὺς καλέσαι καὶ χρυσωρύχους τὴν ἐπιβολὴν κατά γε τοὺς Ἰνδό-
θεν μύρμηκας· τοιαύτης γὰρ ὕλης ἔρωτι παρηνώχλουν τῇ γῇ· τοὺς δ' αὐτοὺς 
καὶ τυμβωρύχους μυριαχοῦ. 



Eustathius Philol., Scr. Eccl., Commentarium in Dionysii periegetae orbis descriptionem (4083: 006)
“Geographi Graeci minores, vol. 2”, Ed. Müller, K.
Paris: Didot, 1861, Repr. 1965.
Section 28, line 9

              Καὶ ὁ μὲν πρὸς ζέφυρον ὠκεανὸς Ἄτλας 
ἑσπέριος λέγεται, ἤγουν ἑσπέριον πέλαγος καὶ Ἀτλαν-
τικόν· ὁ δὲ πρὸς βορρᾶν πεπηγὼς λέγεται πόντος καὶ 
Κρόνιος καὶ νεκρὸς, ὡς μετ' ὀλίγα ῥηθήσεται· ὁ δὲ 
τῆς ἀνατολῆς ἠῷος καλεῖται καὶ Ἰνδικὸς, ὁ δὲ πρὸς 
νότον Ἐρυθραῖός τε ὀνομάζεται καὶ Αἰθιόπιος. 



Eustathius Philol., Scr. Eccl., Commentarium in Dionysii periegetae orbis descriptionem 
Section 28, line 12

                                                        Ση-
μειωτέον δὲ ὅτι ἐν οἷς μέρος τι λέγει τοῦ ὠκεανοῦ, 
πόντον Κρόνιον καὶ αὖθις Ἰνδικὸν κῦμα θαλάσσης, καὶ 
ἐν τῷ «οἵ εἰσιν ἀπὸ νοτίας ἁλός», καὶ «σκιδνάμενος 
ἐκ Κρονίας ἁλός», οὐ ποταμὸν νοεῖ τὸν ὠκεανὸν κατὰ 
τοὺς πολλοὺς τῶν ποιητῶν· τὸ γὰρ πόντος καὶ τὸ   
κῦμα θαλάσσης καὶ τὸ ἃλς οὐδὲν προσήκουσι ποτα-
μοῖς. 



Eustathius Philol., Scr. Eccl., Commentarium in Dionysii periegetae orbis descriptionem 
Section 36, line 3

Ὅτι τὴν ἀνατολὴν οὕτω περιφράζει, ὅπου 
πρώτιστα φαείνεται ἀνθρώποις ἥλιος, ἤτοι ὅπου ἐστὶν 
ἡ ἀνατολὴ, ἔνθα καὶ τὸν ἠῷον καὶ Ἰνδικὸν εἶναί φη-
σιν ὠκεανόν· ἠῷον μὲν λεγόμενον, διὰ τὸ ἐκεῖθεν ἀνα-
τέλλοντος ἡλίου τὴν ἡμέραν γίνεσθαι, ἢ καὶ, ὡς ἓν 
ἀνθ' ἑνὸς εἰπεῖν, ἠῷον ἀντὶ τοῦ ἀνατολικὸν, ἠὼς γὰρ 
πολλάκις καὶ ἡ ἀνατολή· Ἰνδικὸν δὲ διὰ τοὺς παροι-
κοῦντας Ἰνδούς. 



Eustathius Philol., Scr. Eccl., Commentarium in Dionysii periegetae orbis descriptionem 
Section 41, line 3

Ὅτι εἰπὼν τέσσαρα τὰ πελάγη τοῦ ὠκεανοῦ, 
τὸ ἑσπέριον Ἀτλαντικὸν, καὶ τὸ ἀρκτῶον Κρόνιον, καὶ 
τὸ κατὰ τὴν ἕω Ἰνδικὸν, καὶ τὸ νότιον Αἰθιοπικὸν, 
καὶ οὕτω τοῖς τοῦ κόσμου τέσσαρσι κέντροις ἐπεξελθὼν, 
ἐπάγει συμπληρωματικῶς ἅμα καὶ ἐπιφωνηματικῶς, 
ὅτι «Οὕτως ὠκεανὸς περιδέδρομε γαῖαν ἅπασαν»· ὡς ἤδη 
δείξας ὅτι πᾶσαν τὴν γῆν στεφανοῖ, ὡς προείρηται· εἰ 
καί τινες τῶν παλαιῶν ἀντιλέγειν ἐβούλοντο, καθὰ καὶ 
Ἡρόδοτος, ὃς οὐκ ἀποδέχεται τοὺς γράφοντας τὸν 
ὠκεανὸν ῥέειν πέριξ τῆς γῆς. 



Eustathius Philol., Scr. Eccl., Commentarium in Dionysii periegetae orbis descriptionem 
Section 142, line 53

Ἔστι δὲ καὶ Βόσπορος Ἰνδικός. 



Eustathius Philol., Scr. Eccl., Commentarium in Dionysii periegetae orbis descriptionem 
Section 239, line 12

                      Ἐκλήθη δέ ποτε κατὰ τὴν ἱστο-
ρίαν ἡ τοιαύτη χώρα καὶ Ἀερία καὶ Ποταμία καὶ Αἰ-
θιοπία διὰ τοὺς ἐκεῖ Αἰθίοπας, περὶ ὧν πολλοὶ τῶν 
παλαιῶν ἱστοροῦσι· ναὶ μὴν καὶ Ἀετία ἔκ τινος Ἰνδοῦ 
Ἀετοῦ καλουμένου, καὶ Ὠγυγία δὲ καὶ Μελάμβωλος 
καὶ Ἡφαιστία. 



Eustathius Philol., Scr. Eccl., Commentarium in Dionysii periegetae orbis descriptionem 
Section 323, line 6

                      Διὸ καὶ Ἡρόδοτος λέγει τοὺς 
Θρᾷκας ἐθνῶν μέγιστον μετὰ τοὺς Ἰνδούς. 



Eustathius Philol., Scr. Eccl., Commentarium in Dionysii periegetae orbis descriptionem 
Section 566, line 33

Εἶτα συγκρίνων ῥητορικῶς ταῦτα τὰ ἱερὰ πρὸς ἄλλα 
ὅμοιά φησιν, οὐχ οὕτως οἱ Ἀψίνθιοι Θρᾷκες, οὐδ' 
οὕτως οἱ Ἰνδοὶ μελανδίνην ἀνὰ ποταμὸν Γάγγην κῶ-
μον, ἤγουν κωμαστικὴν ἑορτὴν ἄγουσι τῷ Διονύσῳ, 
ὡς αἱ νησιώτιδες αὗται γυναῖκες ἀνευάζουσι, τουτέστιν 
ὑμνοῦσι τὸν Εὔιον Διόνυσον, εὐοῖ εὐᾶν ἀνακράζουσαι, 
ταῦτα δὴ τὰ ἐπὶ Διονύσῳ ἐνθουσιαστικὰ ἐπιφωνήματα. 



Eustathius Philol., Scr. Eccl., Commentarium in Dionysii periegetae orbis descriptionem 
Section 566, line 40

Λόγος γὰρ τὰς τῶν Ἀμνιτῶν γυναῖκας δι' ὅλης νυκτὸς   
ἐξαλλομένας χορεύειν, ὥστε ἐν τούτῳ καὶ τοὺς Θρᾷ-
κας εἴκειν αὐταῖς καὶ τοὺς Ἰνδοὺς, καίτοι καὶ αὐτοὺς 
κατόχους ὄντας τῷ Διονύσῳ, καὶ πάνυ ὀργιάζοντας 
αὐτῷ. 



Eustathius Philol., Scr. Eccl., Commentarium in Dionysii periegetae orbis descriptionem 
Section 568, line 5

Ὅτι τὸ μέγεθος τῶν Βρεττανίδων νήσων, ἃς 
ἄλλοι, ὡς προερρέθη, διὰ τοῦ π Πρεττανίδας καλοῦσιν, 
οὐ μόνον ὁ Διονύσιος ἐνέφηνεν, ὡς ἀνωτέρω ἐρρέθη, 
ἀλλὰ δηλοῖ καὶ ὁ Πτολεμαῖος ἐν τῇ γεωγραφικῇ ὑφη-
γήσει, λέγων ὅτι τῶν νήσων πρωτεύει ἡ Ἰνδικὴ Τα-
προβάνη μεγέθει καὶ δόξῃ, μεθ' ἣν ἡ Βρεττανικὴ, 
τρίτη ἡ Χρυσῆ χερρόνησος, τετάρτη ἑτέρα Βρεττανῶν   
ἡ Ἰουερνία, πέμπτη Πελοπόννησος, Σικελία μετ' 
αὐτὴν ἕκτη, ἑβδόμη Σαρδὼ, ὀγδόη Κύρνος, Κρήτη 
ἐννάτη· ἐπὶ δὲ ταύταις ἡ Κύπρος οὖσα δεκάτη γίνεται 
τοῦ καταλόγου κορωνίς. 



Eustathius Philol., Scr. Eccl., Commentarium in Dionysii periegetae orbis descriptionem 
Section 606, line 11

        Οἱ δὲ κατὰ ἐπίθετον ἐθνικὸν Ἐρυθραῖον βα-
σιλέα τὸν Δηριάδην νοοῦσιν αὐτόθι τεθαμμένον, οὗ 
καὶ πρὸ τούτων ἐμνήσθημεν, ὃς Ἐρυθραῖος μὲν ἦν τῷ 
γένει, χρόνῳ δὲ ὕστερον εἰς Ἰνδοὺς ἐλθὼν ἀντέστη 
λαμπρῶς τῷ τοῦ Διὸς Διονύσῳ στρατευσαμένῳ κατὰ 
τῶν Ἰνδῶν. 



Eustathius Philol., Scr. Eccl., Commentarium in Dionysii periegetae orbis descriptionem 
Section 623, line 4

                   Ὥσπερ γὰρ ἐν Γαδείροις περὶ τὸ 
ἑσπέριον ὀξὺ τοῦ κώνου Ἡράκλειαι ἑστᾶσι στῆλαι, 
οὕτω καὶ ἐν Ἰνδίᾳ πύματον περὶ ῥόον ὠκεανοῦ Ἰνδῶν 
ὑστατίοις ἐν οὔρεσι στῆλαι Θηβαίου Διονύσου, ἐλθόν-
τος μέχρι καὶ ἐκεῖ, ὅτε καὶ ὑφ' ἑαυτῷ τὴν χώραν 
ἐποίησεν, ὡς καὶ ἐν τῷ τέλει λεχθήσεται τῆς περιη-
γήσεως. 



Eustathius Philol., Scr. Eccl., Commentarium in Dionysii periegetae orbis descriptionem 
Section 625, line 1

Ὅτι ἐν οἷς λέγει ὡς ὁ Γάγγης ὁ Ἰνδικὸς πο-
ταμὸς λευκὸν ὕδωρ Νυσσαῖον ἐπὶ πλαταμῶνα κυλίει, 
οὐ κυρίως ἔοικε χρῆσθαι τῇ λέξει, ἀλλ' ἁπλῶς ἐπὶ γῆς 
πλάτους λέγει τὸν πλαταμῶνα. 



Eustathius Philol., Scr. Eccl., Commentarium in Dionysii periegetae orbis descriptionem 
Section 625, line 12

                     Φασὶ δὲ τοὺς περὶ τὴν Νυσσαίαν 
Ἰνδικὴν ταύτην ὁδὸν ὄντας καὶ τοὺς περὶ τὸ Νυσσαῖον 
ὄρος τοῦτο οἰκοῦντας ἀνθρωποφάγους εἶναι. 



Eustathius Philol., Scr. Eccl., Commentarium in Dionysii periegetae orbis descriptionem 
Section 638, line 2

Ὅτι ὁ Ταῦρος τὸ ὄρος ἀρξάμενος ἐκ τῆς Παμ-
φυλίας, ὡς οὗτος δοξάζει, ἄχρι καὶ Ἰνδῶν παρήκει, 
διὰ πάσης ἥκων Ἀσίας· ἄλλοτε μὲν, φησὶ, λοξὸς καὶ 
ἀγκύλος, καὶ, ὡς ἐρεῖ κατωτέρω, πολλὰς ἔχων στρο-
φάλιγγας, ἄλλοτε δὲ ὀρθότερος ἴχνεσιν. 



Eustathius Philol., Scr. Eccl., Commentarium in Dionysii periegetae orbis descriptionem 
Section 647, line 7

                     Ὁ δὲ Γεωγράφος οὕτω περὶ τοῦ 
Ταύρου φησί· «διέζωκεν ὁ Ταῦρος μέσην τὴν Ἀσίαν, 
ἀπὸ ἑσπέρας ἐπὶ τὴν ἕω τετραμμένος, οὗ τελευταῖον 
τὸ Ἰμάϊον τῇ Ἰνδικῇ θαλάσσῃ συνάπτον». 



Eustathius Philol., Scr. Eccl., Commentarium in Dionysii periegetae orbis descriptionem 
Section 700, line 7

                          Οὗτοι οἱ Καμαρῖται τὸν Βάκχον 
Ἰνδῶν ἐκ πολέμου, φησὶ, δεξάμενοι ἐξένισαν, καὶ ταῖς 
Λήναις, ὃ ἔστι ταῖς Βάκχαις, συνεχόρευσαν, τὰ ἐκείνων 
φορήματα, ζώματα δηλαδὴ καὶ νεβρῖδας, ἐπὶ στήθεσι 
βαλόντες, εὐοῖ Βάκχε λέγοντες. 



Eustathius Philol., Scr. Eccl., Commentarium in Dionysii periegetae orbis descriptionem 
Section 747, line 5

            Ὄρος δὲ τὸ Ἠμωδὸν Ἰνδικὸν πρὸς ταῖς 
τοῦ Διονύσου στήλαις, ὅπερ τινὲς προπαροξυτόνως 
Ἤμωδον λέγουσι, καθὰ καὶ τὸ Ἰμάϊον, ὅπερ ἐστὶ 
τμῆμα τοῦ Ταύρου. 



Eustathius Philol., Scr. Eccl., Commentarium in Dionysii periegetae orbis descriptionem 
Section 775, line 27

                                         Καὶ ὁ Γάγγης 
δὲ πεδίον τι ἐν Ἰνδίᾳ προσέχωσεν. 



Eustathius Philol., Scr. Eccl., Commentarium in Dionysii periegetae orbis descriptionem 
Section 875, line 21

                                   Ἰστέον δὲ ὅτι ἔστι 
καὶ ἔθνος μαχιμώτατον Ἰνδικὸν οἱ Μαλλοὶ πληθυντι-
κῶς, ἔνθα παρὰ βραχὺ ἐκινδύνευσεν ἂν εἰς ζωὴν ὁ μέ-
γας Ἀλέξανδρος. 



Eustathius Philol., Scr. Eccl., Commentarium in Dionysii periegetae orbis descriptionem 
Section 936, line 4

                                            Σκοπητέον δὲ 
εἴτε ὁ ἐνταῦθα εὐώδης κάλαμος ὁ αὐτός ἐστι ταῖς παρὰ 
τῷ Γεωγράφῳ Ἰνδικαῖς καλάμοις, ἃς μελιτοποιεῖν 
ἐκεῖνός φησιν, εἴτε καὶ διαφέρει μάλιστα ἐκείνων οὗτος 
ὡς εὐώδης μὴ τοιούτων. 



Eustathius Philol., Scr. Eccl., Commentarium in Dionysii periegetae orbis descriptionem 
Section 936, line 8

                            Ἰστέον δὲ ὅτι κατὰ τὴν πα-
λαιὰν ἱστορίαν, Ἀλέξανδρος διενοεῖτο μετὰ τὴν ἐξ 
Ἰνδῶν ἐπάνοδον βασίλειον ἑαυτοῦ τὴν Ἀραβίαν ποιῆ-
σαι, καὶ ὅτι εἰσὶ καὶ Σκηνῖται Ἄραβες περὶ τὰ πέραν 
Εὐφράτου ἕως Κοίλης Συρίας. 



Eustathius Philol., Scr. Eccl., Commentarium in Dionysii periegetae orbis descriptionem 
Section 988, line 15

Ὅτι δευτερεύουσι μετὰ τοὺς Ἰνδικοὺς ποταμοὺς 
Εὐφράτης καὶ Τίγρις, καὶ ὅτι μείζων Εὐφράτης Τίγριος 
ῥέοντος ἀπὸ Νιφάτα ὄρους, καὶ πλείω διέξεισι χώραν. 



Eustathius Philol., Scr. Eccl., Commentarium in Dionysii periegetae orbis descriptionem 
Section 1039, line 40

              Ἀφ' οὗ μέντοι εἰς τὴν Ἀσίαν διέβη, 
καὶ τετράπλευρόν τι χωρίον περιέγραψε πλευραῖς τῷ 
τε Ταύρῳ καὶ τῷ Νείλῳ καὶ τῷ Ἰνδικῷ ὠκεανῷ καὶ 
τῷ νοτίῳ, τότε δὴ πλατύτερον τοῖς ἱστορουμένοις ἐπ-
εξέρχεται, ὡς δῆλον ἔκ τε τῶν εἰς Ἀραβίαν λεχθέντων 
καὶ τῶν περὶ Μηδείας ἱστορηθέντων καὶ τῶν νῦν δὲ 
ἐκτεθέντων. 



Eustathius Philol., Scr. Eccl., Commentarium in Dionysii periegetae orbis descriptionem 
Section 1039, line 46

                                                Ἐν τοῖς 
ἑξῆς δὲ καὶ ἡ Ἰνδία ἱκανὴν αὐτῷ παρέξει λόγου τρι-
βήν. 



Eustathius Philol., Scr. Eccl., Commentarium in Dionysii periegetae orbis descriptionem 
Section 1069, line 4

                                             Ἐν δὲ τοῖς 
ἑξῆς καὶ Ἰνδικὸν ἔθνος οἱ Σάβαι εὑρίσκονται. 



Eustathius Philol., Scr. Eccl., Commentarium in Dionysii periegetae orbis descriptionem 
Section 1073, line 2

Ὅτι Κύρος ποταμὸς Περσικὸς, ἔτι δὲ καὶ 
Χοάσπης Ἰνδὸν μὲν ὕδωρ ἕλκων, ὡς ἐκ τοῦ Ἰνδοῦ σχι-
ζόμενος ποταμοῦ, παραρρέων δὲ τὰ Σοῦσα, ἐξ οὗ καὶ 
μόνου ἔπινεν ὁ τῶν Περσῶν βασιλεύς· καὶ ἦν βασιλι-
κὸν ὕδωρ τὸ Χοάσπειον. 



Eustathius Philol., Scr. Eccl., Commentarium in Dionysii periegetae orbis descriptionem 
Section 1088, line 1

Ὅτι πρὸς αὐγὰς τῆς Γεδρωσίας ὁ Ἰνδὸς μέ-
γιστος ποταμὸς, ὃν Σκύθαι νότιοι παροικοῦσιν, οἱ καὶ 
Ἰνδοσκύθαι συνθέτως λεγόμενοι. 



Eustathius Philol., Scr. Eccl., Commentarium in Dionysii periegetae orbis descriptionem 
Section 1088, line 5

         Ῥέειν δέ φασι τὸν Ἰνδὸν, καθὰ καὶ τὸν Γάγγην, 
ἀρξάμενον ἀπό τινος Καυκάσου ὄρους ἀρκτικοῦ ἀνατο-  
λικοῦ, λαβρότατον ῥόον ὀξὺν ἐπὶ νότον ὀρθὸν ἐλαύνοντα, 
κατεναντίον τῆς Ἐρυθρᾶς θαλάσσης, ἤγουν τοῦ Ἐρυ-
θραίου καὶ Αἰθιοπικοῦ ὠκεανοῦ, περὶ τὰ ἑῷα τοῦ Περ-
σικοῦ κόλπου, καὶ σχιζόμενον εἰς δύο στόματα διέχοντα 
ἀλλήλων στάδια ͵α καὶ ωʹ, ὡς καὶ Ἀρριανὸς ἱστορεῖ, ἃ 
δὴ στόματα ποιοῦσι τὴν Παταληνὴν οἷα νῆσον ἀπο-
λαμβανομένην, ἣν καὶ ἐπαναλαμβάνει ὡς λόγου ἀξίαν, 




Eustathius Philol., Scr. Eccl., Commentarium in Dionysii periegetae orbis descriptionem 
Section 1088, line 18

                                               Φησὶ γοῦν 
Ἀρριανός· «δίστομός ἐστιν ὁ Ἰνδὸς, καὶ Δέλτα ποιεῖ 
καὶ οὗτος ἐν τῇ τῶν Ἰνδῶν γῇ, τῷ Αἰγυπτίῳ Δέλτα 
παραπλήσιον. 



Eustathius Philol., Scr. Eccl., Commentarium in Dionysii periegetae orbis descriptionem 
Section 1088, line 24

                                 Τοῦ δὲ μεγέθους καὶ 
τῆς ὀξύτητος τοῦ Ἰνδοῦ ποταμοῦ δεῖγμα καὶ τοῦτό 
φασιν οἱ παλαιοὶ, ὅτι ἐκλιπὼν τὸ οἰκεῖον ῥεῖθρόν ποτε 
καὶ εἰς πεδία τινὰ κοῖλα ἐκτραπόμενος καὶ οἷον καταρ-
ράξας ἠρήμωσε χώραν πλειόνων ἢ χιλίων πόλεων σὺν 
ταῖς κώμαις. 



Eustathius Philol., Scr. Eccl., Commentarium in Dionysii periegetae orbis descriptionem 
Section 1095, line 1

Ὅτι πρὸς δύσιν τοῦ Ἰνδοῦ ποταμοῦ Ὠρῖται, 
διὰ διχρόνου γραφόμενοι, πρὸς ἀντιδιαστολὴν τῶν ἐν 
τῷ Ὠρεῷ τῆς Εὐβοίας Ὠρειτῶν. 



Eustathius Philol., Scr. Eccl., Commentarium in Dionysii periegetae orbis descriptionem 
Section 1095, line 9

                       Μέμνηται γὰρ τοιούτου ἔθνους Ἰνδικοῦ 
ὁ Γεωγράφος, λέγων· Ἄρβιες ὁμώνυμοι ποταμῷ Ἄρβει, 
ὁρίζοντι αὐτοὺς ἀπὸ τῶν Ὠριτῶν. 



Eustathius Philol., Scr. Eccl., Commentarium in Dionysii periegetae orbis descriptionem 
Section 1096, line 5

                                                 Τοὺς δὲ Ἀρα-
χώτας τινὲς Ἀραχωτούς φασιν ὀξυτόνως, λέγοντες ὅτι 
μετὰ τὸν Ἰνδὸν οἱ Παροπαμισάδαι, εἶτα πρὸς νότον 
Ἀραχωτοί· πρὸς ἐσπέραν δὲ τοῖς Παροπαμισάδαις πα-
ράκεινται Ἄριοι. 



Eustathius Philol., Scr. Eccl., Commentarium in Dionysii periegetae orbis descriptionem 
Section 1097, line 2

Ὅτι βούλονταί τινες Παρπάνισον προπαρο-
ξυτόνως λέγεσθαι ὄρος Ἰνδικὸν, οὐ μὴν Παρνησὸν, ὡς 
πολλὰ τῶν ἀντιγράφων ἔχουσι. 



Eustathius Philol., Scr. Eccl., Commentarium in Dionysii periegetae orbis descriptionem 
Section 1097, line 9

                           Καὶ ἡ χώρα αὐτῶν Ἀριανὴ κατὰ 
τὸν Γεωγράφον, λέγοντα ὅτι μετὰ τὴν Ἰνδικὴν ἡ Ἀ-
ριανὴ μερὶς, ἔνθα καὶ οἱ Ἄρβιες, εἶτα Ὠρῖται, οἷς 
ἐφεξῆς Ἰχθυοφάγοι. 



Eustathius Philol., Scr. Eccl., Commentarium in Dionysii periegetae orbis descriptionem 
Section 1107, line 1

Ὅτι ἡ τῶν Ἰνδῶν γῆ, ἔθνους μεγίστου πάντων 
καὶ εὐδαιμονεστάτου, πασῶν ἐσχάτη παρὰ τοῖς τοῦ 
ὠκεανοῦ κεῖται χείλεσιν, ἣν ὁ ἥλιος πρώταις ἐπιφλέγει 
ἀκτῖσιν, ἐπὶ τὴν γῆν ἀνερχόμενος, τουτέστιν ἀνίσχων, 
ἀνατέλλων· διὸ καὶ θερμότατος παρ' αὐτοῖς ὁ ἥλιος τὸ 
ἑωθινὸν, ὥς φασιν οἱ παλαιοὶ, καὶ πολλῷ μᾶλλον ἢ ἐν 
μεσημβρίᾳ κατὰ τὴν Ἑλλάδα. 



Eustathius Philol., Scr. Eccl., Commentarium in Dionysii periegetae orbis descriptionem 
Section 1107, line 10

         Ὁ δὲ Διονύσιος πολὺν ἔπαινον κατατείνων τῆς 
Ἰνδικῆς φησι, διὸ τόν τε χρόα οἱ ἐκεῖ κυανέουσι θε-
σπέσιον λιπόωντες, τρίχας τε πιοτάτας φοροῦσιν ὑακίνθῳ 
ὁμοίας, ὅπερ ὁ Ποιητής πού φησιν, ὑακινθίνῳ ἄνθει 
ὁμοίας, ἤτοι μελαίνας. 



Eustathius Philol., Scr. Eccl., Commentarium in Dionysii periegetae orbis descriptionem 
Section 1107, line 14

        Μελανότριχες δὲ καὶ τῶν Ἰνδῶν μάλιστα οἱ 
μεσημβρινοὶ, καὶ τὴν χρόαν ὅμοιοι τοῖς Αἰθίοψι, διὰ 
τὸ ἐπιλείπειν τὴν ἐπιπολῆς ἰκμάδα, ἐπικαίοντος τοῦ 
ἡλίου, ὅθεν οὐδὲ οὐλοτριχοῦσιν. 



Eustathius Philol., Scr. Eccl., Commentarium in Dionysii periegetae orbis descriptionem 
Section 1107, line 23

      Καὶ οἱ μὲν, φησὶ, χρυσοῦ μεταλλεύουσι γενέθλην, 
τὴν χρυσῖτιν ψάμμον μακέλλαις λαχαίνοντες, ὃ ἔστιν 
ἀνοίγοντες, σκάπτοντες· ἐξ οὗ καὶ τὸ λάχανον· οἱ δὲ 
ἱστοὺς ὑφαίνουσι λινεργέας, οἱ δὲ ἐλεφάντων λευκοὺς 
πρισθέντας ἀποξύουσιν ὀδόντας, οὓς ἕτεροι ἐλεφάντων 
κέρατά φασιν, ἄλλοι δὲ, φησὶ, τῶν Ἰνδῶν ἐπὶ ταῖς τῶν 
χειμάρρων προβολαῖς ἢ προμολαῖς, ἤγουν ἐκρύσεσιν, 
ἢ βηρύλλου γλαυκὴν λίθον ἰχνεύουσιν ἢ ἀδάμαντα 
μαρμαίροντα ἢ χλωρὰ διαυγάζουσαν ἴασπιν ἢ γλαυκὴν 
λίθον καθαροῦ τοπάζου καὶ γλυκερὰν ἀμέθυσον ὑπη-
ρέμα πορφύρουσαν. 



Eustathius Philol., Scr. Eccl., Commentarium in Dionysii periegetae orbis descriptionem 
Section 1107, line 36

                              Οὕτω τιμίοις λίθοις κα-
ταστρώσας τὴν Ἰνδικὴν ὁ Διονύσιος ἐπάγει ἐπιφωνη-
ματικῶς κατὰ συντομίαν, ὅτι παντοῖον ὄλβον αὔξει, 
ποταμοῖς οὖσα κατάρρυτος ἔνθα καὶ ἔνθα. 



Eustathius Philol., Scr. Eccl., Commentarium in Dionysii periegetae orbis descriptionem 
Section 1107, line 42

                                                 Ναὶ μὴν, φησὶ, 
καὶ οἱ λειμῶνες ἐκεῖ ἀεὶ κομῶσι πετάλοις· ἄλλοθι μὲν 
γὰρ κέγχρος αὔξεται, ἄλλοθι δὲ ὕλαι θάλλουσιν Ἐρυ-
θραίου καλάμου, τοῦ ἀρωματικοῦ δηλαδὴ, Ἐρυθραίου 
λεγομένου διὰ τὸ τὴν Ἰνδίαν ἕως καὶ εἰς τὸν Ἐρυθραῖον 
παρήκειν ὠκεανόν· ὅπερ δῆλόν ἐστι καὶ ἐκ τοῦ τὴν 
Γαγγῖτιν χώραν κατὰ τὸν Διονύσιον περὶ τὰ τέρματα 
ἕλκεσθαι τῆς Κωλιάδος γῆς, ἤτοι τῆς Ταπροβάνης, 
ὅπου τὰ μέγιστα κήτη, ἅπερ αὐτὸς Ἐρυθραίου πόντου 
εἶπε βοτά. 



Eustathius Philol., Scr. Eccl., Commentarium in Dionysii periegetae orbis descriptionem 
Section 1107, line 48

                                                Ἕτεροι δὲ καὶ 
πολυφάρμακον τὴν Ἰνδικὴν ἱστοροῦσι καὶ πολύρριζον 
καὶ πολυχρώματον καὶ δίκαρπον καὶ διφόρον, καὶ τοὺς 
ἐκεῖσε ἄνδρας εὐμεγέθεις καὶ πενταπήχεις τοὺς πολλοὺς 
ἢ ὀλίγον ἀποδέοντας· ὧν καὶ ὁ βασιλεὺς Πῶρος ὑπὲρ 
πεντάπηχυν ἱστόρηται. 



Eustathius Philol., Scr. Eccl., Commentarium in Dionysii periegetae orbis descriptionem 
Section 1107, line 52

                         Καὶ φιλῳδοὶ δὲ οἱ Ἰνδοὶ λέ-
γονται, καὶ φιλορχήμονες ἀπὸ Διονύσου. 



Eustathius Philol., Scr. Eccl., Commentarium in Dionysii periegetae orbis descriptionem 
Section 1107, line 54

                                             Ἐλεύθεροι δὲ 
πάντες Ἰνδῶν, καὶ οὔτις δοῦλος Ἰνδός. 



Eustathius Philol., Scr. Eccl., Commentarium in Dionysii periegetae orbis descriptionem 
Section 1107, line 57

                                Φασὶ δὲ καὶ Σηρικὰ παρ' 
Ἰνδοῖς γίνεσθαι ἔκ τινων φλοιῶν ξαινομένης βύσσου. 



Eustathius Philol., Scr. Eccl., Commentarium in Dionysii periegetae orbis descriptionem 
Section 1107, line 62

Λέγεται δὲ καὶ ὅτι ἰχθυοφαγοῦσι πολλοὶ τῶν Ἰνδῶν 
ὠμοὺς ἰχθύας σιτούμενοι, καὶ ὅτι τὰ ἑῷα τῆς Ἰνδίας 
ἐρημία ἐστὶ διὰ τὴν ψάμμον, καὶ ὅτι οἱ Ἰχθυοφάγοι 
πλοῖα ποιοῦνται καλάμινα, καὶ ὅτι ἓν γόνυ καλάμου 
πλοῖον ἀπαρτίζει, ὥς φησιν Ἡρόδοτος. 



Eustathius Philol., Scr. Eccl., Commentarium in Dionysii periegetae orbis descriptionem 
Section 1107, line 71

                                                          Καὶ 
ἄλλως δὲ κάλαμοι ἁπλῶς ἕτεροι παράσημοι ἐν Ἰνδίᾳ, 
ὡς ἐρρέθη, καθὰ καὶ παρὰ τοῖς ἑσπερίοις Αἰθίοψιν, ὧν 
καλάμων ἕκαστον γόνυ χωρεῖν λέγεται χοίνικας τρεῖς. 



Eustathius Philol., Scr. Eccl., Commentarium in Dionysii periegetae orbis descriptionem 
Section 1107, line 74

Λέγεται δὲ νόμον ἄριστον εἶναι παρ' Ἰνδοῖς, τὸν εὑ-
ρόντα τι ὀλέθριον ἀνάγκην ἔχειν ἐξευρίσκειν καὶ ἄκος 
τοῦ κακοῦ. 



Eustathius Philol., Scr. Eccl., Commentarium in Dionysii periegetae orbis descriptionem 
Section 1134, line 1

Ὅτι τὸ σχῆμα τῆς Ἰνδικῆς, ὡς καὶ ὁ Γεω-
γράφος φησὶ, τέσσαρσιν ἥρμοσται πλευραῖς λοξαῖς πά-
σαις, ῥόμβον ἀποτελούσαις. 



Eustathius Philol., Scr. Eccl., Commentarium in Dionysii periegetae orbis descriptionem 
Section 1134, line 4

                               Ὧν ἑσπερία μὲν ἡ πρὸς 
τῷ Ἰνδῷ ποταμῷ, νοτία δὲ ἡ πρὸς τὴν Ἐρυθρὰν 
νεύουσα, ὁ δὲ Γάγγης ἑῴα, εἰς δὲ πόλον ἄρκτων ὁ 
Καύκασος, ἤγουν κατὰ τὸν Γεωγράφον τὰ τοῦ Ταύρου 
ἔσχατα. 



Eustathius Philol., Scr. Eccl., Commentarium in Dionysii periegetae orbis descriptionem 
Section 1134, line 8

          Καὶ σημείωσαι ὅτι κατὰ τὸν Διονύσιον καὶ 
Ἰνδικός ἐστι Καύκασος, ὡς καὶ ἀνόπιν εἴρηται. 



Eustathius Philol., Scr. Eccl., Commentarium in Dionysii periegetae orbis descriptionem 
Section 1134, line 11

                         Ὅτι δὲ ὁ Γάγγης καὶ ὁ Ἰν-
δὸς μέγιστοι ποταμῶν, καὶ αὐτὸ προείρηται. 



Eustathius Philol., Scr. Eccl., Commentarium in Dionysii periegetae orbis descriptionem 
Section 1134, line 20

              Γάγγης. 
Ἰνδός. 



Eustathius Philol., Scr. Eccl., Commentarium in Dionysii periegetae orbis descriptionem 
Section 1138, line 1

Ὅτι οἱ Δαρδανεῖς Ἰνδικὸν ἔθνος· οἱ μέντοι 
Δάρδανοι Τρωϊκόν. 



Eustathius Philol., Scr. Eccl., Commentarium in Dionysii periegetae orbis descriptionem 
Section 1139, line 1

Ὅτι ὁ Ὑδάσπης (ποταμὸς δὲ καὶ οὗτος Ἰνδι-
κός) πλωτός ἐστι ναυσὶν, εἰς ὃν Ἀκεσίνης ἐμβάλλει, 
λοξὸς ἀπὸ σκοπέλων φερόμενος· περὶ οὗ ὁ Γεωγράφος 
φησὶν, ὅτι Ἀκεσίνης ἐν τροπαῖς θεριναῖς ἀναβαίνει 
πήχεις μʹ, ὧν οἱ μὲν κʹ πληροῦσι μέχρι χείλους τὸ 
ῥεῖθρον, οἱ δὲ κʹ ὑπερχέονται εἰς τὸ πεδίον, διὸ καὶ 
νησίζουσιν οἷον αἱ πόλεις, ὡς καὶ ἐν Αἰγύπτῳ, ἵδρυν-
ται γὰρ ἐπὶ χωμάτων. 



Eustathius Philol., Scr. Eccl., Commentarium in Dionysii periegetae orbis descriptionem 
Section 1140, line 1

Ὅτι καὶ ὁ Κώφης Ἰνδικός ἐστι ποταμὸς, ὃν 
Ἡρωδιανὸς μὲν καὶ αὐτὸν ἐν τῇ καθόλου βαρυτόνως 
ἐκφέρει καὶ μετὰ τοῦ σ, ὡς τὸ Χρύσης, Ἀριστοτέλης 
δὲ ὥς φασιν ἐν πέμπτῳ περὶ Ἀλεξάνδρου τὸν Κωφῆνά 
φησιν, ὡς τὸν σωλῆνα. 



Eustathius Philol., Scr. Eccl., Commentarium in Dionysii periegetae orbis descriptionem 
Section 1140, line 7

                         Ὁ Γεωγράφος δὲ ὁμοίως τῷ 
Διονυσίῳ ἐκφέρει, λέγων «Χοάσπης εἰς τὸν Κώφην 
ἐμβάλλει» καὶ πάλιν «μετὰ τὸν Κώφην ὁ Ἰνδὸς, εἶτα 
ὁ Ὑδάσπης, εἶτα ὁ Ἀκεσίνης, καὶ ὕστατος ὁ Ὕπανις. 



Eustathius Philol., Scr. Eccl., Commentarium in Dionysii periegetae orbis descriptionem 
Section 1141, line 2

Ὅτι εἰ καὶ πολλὰ τῶν ἀντιγράφων Τοξίλους 
γράφουσι τὸ Ἰνδικὸν ἔθνος, ἀλλ' ὁ Γεωγράφος Ταξίλους 
γράφει διὰ τοῦ α, καὶ Τάξιλα τὴν κατ' αὐτοὺς πόλιν, 
μεγάλην αὐτὴν λέγων καὶ εὐδαίμονα καὶ εὐνομωτά-
την, καί τινα ἄνδρα Ταξίλην ἱστορῶν αὐτῆς καὶ βασι-
λέα, καὶ λέγων ὅτι φιλανθρώπως αὐτὸς καὶ οἱ ἐκεῖ 
δεξάμενοι τὸν Ἀλέξανδρον ἔτυχον πλειόνων ἢ ἔδωκαν. 



Eustathius Philol., Scr. Eccl., Commentarium in Dionysii periegetae orbis descriptionem 
Section 1141, line 9

Διὸ καὶ φθονοῦντες οἱ Μακεδόνες ἔλεγον ὡς οὐκ εἶχεν 
Ἀλέξανδρος οὓς εὐεργετήσει, πρὶν διαβῆναι τὸν Ἰνδόν. 



Eustathius Philol., Scr. Eccl., Commentarium in Dionysii periegetae orbis descriptionem 
Section 1143, line 1

Ὅτι ἔθνος Ἰνδικὸν οἱ Πευκαλεῖς, ἄγρια φῦλα. 



Eustathius Philol., Scr. Eccl., Commentarium in Dionysii periegetae orbis descriptionem 
Section 1143, line 2

Τινὲς δὲ διὰ τοῦ ν γράφουσι Πευκανεῖς. Ἰνδοὶ δὲ καὶ 
οἱ Γαργαρίδαι, οὓς Διονύσου καλεῖ θεράποντας, ὅπου, 
φησὶ, χρυσὸν καταφέρουσιν ὁ Ὕπανίς τε ποταμὸς ὁ 
καὶ ἀνωτέρω ῥηθεὶς, καὶ ὁ Μάγαρσος, λαβρότατοι πο-
ταμῶν, ἀπὸ τοῦ Ἠμωδοῦ ὄρους προρρέοντες ἐπὶ τὴν 
Γαγγήτιδα χώραν, πρὸς νότον εἰς τὴν προγραφεῖσαν 
Κωλιάδα νῆσον τετρασυλλάβως, ἢ Κωλίδα τρισυλλά-
βως κατὰ συγκοπὴν, ἣν καὶ προνενευκέναι εἰς τὸν ὠκεα-
νόν φησιν, ὡς ὑπ' αὐτοῦ νησιζομένην, καὶ δυσέμβατον 
οἰωνοῖς εἶναι λέγει, διὸ καὶ καλεῖσθαι Ἄορνιν. 



Eustathius Philol., Scr. Eccl., Commentarium in Dionysii periegetae orbis descriptionem 
Section 1143, line 26

Λέγεται δὲ καὶ πέτρα τις Ἄορνος περὶ τὴν Ἰνδίαν, 
ᾗπερ ὁ Ἡρακλῆς μὲν προσβαλὼν εἰς τρὶς ἀπεκρού-
σθη, Ἀλέξανδρος δὲ κατὰ μίαν εἷλε προσβολήν. 



Eustathius Philol., Scr. Eccl., Commentarium in Dionysii periegetae orbis descriptionem 
Section 1143, line 29

                                                      Ταύ-
της δὲ τῆς Ἀόρνου τὰς ῥίζας ὁ Ἰνδὸς ποταμὸς ὑπο-
ῥέειν λέγεται. 



Eustathius Philol., Scr. Eccl., Commentarium in Dionysii periegetae orbis descriptionem 
Section 1143, line 35

                                  Πολὺς δὲ ἐν ταῖς ἱστο-
ρίαις ὁ Ἰνδικὸς Ὕπανις. 



Eustathius Philol., Scr. Eccl., Commentarium in Dionysii periegetae orbis descriptionem 
Section 1143, line 45

Μεθ' ὃν δεύτερον εἶναι λέγουσι τὸν Ἰνδὸν, καὶ τρίτον 
καὶ τέταρτον Ἴστρον καὶ Νεῖλον. 



Eustathius Philol., Scr. Eccl., Commentarium in Dionysii periegetae orbis descriptionem 
Section 1143, line 48

                                        Φασὶ δὲ καὶ ιεʹ 
ποταμοὺς τοὺς ἀξιολογωτάτους εἰσβάλλειν εἰς τὸν Ἰν-
δὸν, ὧν ὕστερον εἶναι τὸν Ὕπανιν· οὓς πάντας ὁ 
Ἰνδὸς παραλαβὼν, ὥς φησιν Ἀρριανὸς, καὶ τῇ ἐπωνυ-
μίᾳ κρατήσας ἐκδιδοῖ ἐς θάλασσαν. 



Eustathius Philol., Scr. Eccl., Commentarium in Dionysii periegetae orbis descriptionem 
Section 1143, line 61

                          Τὸν δὲ Ἰνδὸν ποταμὸν εὐρύ-
νεσθαί πού φασι καὶ ὑπὲρ πεντήκοντα σταδίους, ὅτε 
πληρωθείη, τὸ δὲ ἐλάχιστον εἰς ἑπτά. 



Eustathius Philol., Scr. Eccl., Commentarium in Dionysii periegetae orbis descriptionem 
Section 1143, line 66

        Ἀρριανὸς δέ φησιν ὅτι ἵνα μὲν στενότατος ὁ 
Ἰνδὸς, μʹ σταδίους διέχουσιν αὐτῷ αἱ ὄχθαι, ἵνα δὲ 
πλατύτατος καὶ ἑκατόν. 



Eustathius Philol., Scr. Eccl., Commentarium in Dionysii periegetae orbis descriptionem 
Section 1153, line 2

Ὅτι ὁ Θηβαῖος Διόνυσος τοὺς κελαινοὺς Ἰν-
δοὺς ὀλέσας ἀντιστάντας αὐτῷ δύο στήλας ἔστησε περὶ 
τὰ τῆς γῆς τέρματα, καὶ αὖθις καγχαλόων, ὃ ἔστι χαί-
ρων, εἰς Θήβας ἐπανῆλθε. 



Eustathius Philol., Scr. Eccl., Commentarium in Dionysii periegetae orbis descriptionem 
Section 1153, line 6

                              Καὶ ὅρα ὅτι ὥσπερ κυανέους 
ἀλλαχοῦ εἶπε τοὺς Αἰθίοπας, οὕτως ἐνταῦθα κελαι-
νοὺς εἶπε τοὺς Ἰνδούς. 



Eustathius Philol., Scr. Eccl., Commentarium in Dionysii periegetae orbis descriptionem 
Section 1153, line 21

                  Νύσσα δὲ κατὰ τὸν Γεωγράφον πό-
λις ἐν Ἰνδίᾳ, κτίσμα Διονύσου, καὶ ὄρος αὐτόθι Μηρὸς, 
ὅθεν παρὰ τοῖς μύθοις μηροτραφὴς ἐνομίσθη Διόνυσος. 



Eustathius Philol., Scr. Eccl., Commentarium in Dionysii periegetae orbis descriptionem 
Section 1153, line 36

                               Ἀρριανὸς δὲ λέγει καὶ ὅτι 
κισσὸς ἄλλῃ τῆς Ἰνδικῆς γῆς μὴ φυόμενος παρὰ 
Νυσσαίοις φύεται, καὶ ὅτι εἰς τὸ Νυσσαῖον ὄρος 
τὸν Μηρὸν ἐλθόντες οἱ Μακεδόνες ἡδέως τὸν κισσὸν εἶ-
δον, καὶ στεφάνους σπουδῇ ἀπ' αὐτοῦ ἐποιοῦντο, 
καὶ ὅτι Νυσσαίων πρέσβεις τὸν Ἀλέξανδρον ἐν ὅπλοις 
ἰδόντες κεκονιμένον ἐκ τῆς ὁδοῦ ἐθάμβησάν τε τὴν 
ὄψιν καὶ πεσόντες εἰς γῆν ἐπὶ πολὺ σιγὴν εἶχον. 



Eustathius Philol., Scr. Eccl., Commentarium in Dionysii periegetae orbis descriptionem 
Section 1153, line 53

                          Πάντων γὰρ, φασὶ, τὰ κατὰ 
τὸν δεσμὸν τοῦ Προμηθέως ὑποθεμένων ἐν τῷ ἀρκτώῳ 
γενέσθαι Καυκάσῳ, οἱ Μακεδόνες κολακείᾳ τῇ πρὸς 
τὸν Ἀλέξανδρον μετέθηκαν τῷ λόγῳ τὸν Καύκασον εἰς 
τὴν ἑῴαν θάλασσαν, καλέσαντες Καύκασον ὄρη τινὰ 
ἐκεῖ Ἰνδικά. 



Eustathius Philol., Scr. Eccl., Commentarium in Dionysii periegetae orbis descriptionem 
Section 1153, line 72

                                                      Οἱ δ' 
αὐτοὶ κόλακες κατὰ τοὺς παλαιοὺς Καύκασον ὠνόμα-
σαν ὁμοῦ τόν τε Παροπαμισὸν καὶ τὸ Ἠμωδὸν καὶ τὸ 
Ἰμάϊον, τὰ Ἰνδικὰ ὄρη. 



Eustathius Philol., Scr. Eccl., Commentarium in Dionysii periegetae orbis descriptionem 
Section 1153, line 76

                         Ἀρριανὸς δὲ τάδε γράφει περὶ 
τῆς εἰς Ἰνδοὺς τοῦ Διονύσου στρατείας· «Διόνυσον 
πολεμῆσαι Ἰνδοῖς λόγος, ὅστις δὴ οὗτος ὁ ἐπὶ Ἰν-
δοὺς στρατεύσας Διόνυσος. 



Scholia In Aristotelem, Scholia in Aristotelis sophisticos elenchos (scholia recentiora) (e cod. Vat. Urb. gr. 35) (5015: 001)
“”Vaticanus Urbinas Graecus 35. An edition of the scholia on Aristotle's sophistici elenchi””, Ed. Bülow–Jacobsen, A., Ebbesen, S., 1982; Cahiers de l'Institut du moyen–âge grec et latin 43.
Bekker page 5, line 167a7-8m, line of scholion 1

ὄν> 
<οἷον ὁ ἰνδός>) 
<λευκὸς καὶ 
οὐ λευκός 
οὐ λευκὸς μέν ἐστι κατὰ τὴν ἐπιφάνειαν· 
λευκὸς δὲ κατὰ τοὺς ὀδόντας 
ὁ αἰθίοψ>   
<λευκός μέλας 
αἰθίοψ> 
<ἄμφω>· 
<μέλας καὶ λευκός 




Uranius Hist., Fragmenta (2461: 003)
“FHG 4”, Ed. Müller, K.
Paris: Didot, 1841–1870.
Fragment 20, line 1

Idem: Σῆρες, ἔθνος Ἰνδικὸν, ἀπροσμιγὲς ἀνθρώ-
ποις, ὡς Οὐράνιος ἐν τρίτῳ Ἀραβικῶν. 



Sopater Rhet., Scholia ad Hermogenis status seu artem rhetoricam (2031: 002)
“Rhetores Graeci, vol. 5”, Ed. Walz, C.
Stuttgart: Cotta, 1833, Repr. 1968.
Volume 5, page 4, line 13

                          Διενήνοχε δὲ ἡ τέχνη τῆς ἐπι-
στήμης, τῷ μὴ ἀδιαπτώτῳ κεχρῆσθαι τῷ σκοπῷ, ἀλ-
λὰ μεθαρμόζεσθαι πρὸς πρόσωπα καὶ καιροὺς, ὥσπερ 
οἱ ἰώμενοι, ἵνα κατ' ἰατροὺς εἴπωμεν, ζητοῦσιν ὥραν, 
χώραν, ἡλικίαν, καὶ τῷ Σκύθῃ μὲν νοσοῦντι θερμὰ ἐπ-
άγουσι βοηθήματα, διὰ τὴν φύσιν τῆς χώρας καὶ τὸ 
πάνυ ψυχρὸν, τῷ δὲ Ἰνδῷ ψυχρὰ διὰ τὸ θερμὸν τῆς 
χώρας, καὶ πρεσβύτῃ καὶ νέῳ διαφόρως. 



Theognostus Gramm., Canones sive De orthographia (3128: 001)
“Anecdota Graeca e codd. manuscriptis bibliothecarum Oxoniensium, vol. 2”, Ed. Cramer, J.A.
Oxford: Oxford University Press, 1835, Repr. 1963.
Section 88, line 3

Ἡ ι συλλαβὴ εἴτε κατ' ἀρχὴν λέξεως, εἴτε κατὰ τὸ 
μέσον ἐν ἁπλῇ καὶ ἀκινήτῳ λέξει λήγουσα εἰς ν, ἐπαγομένου 
τοῦ δ, διὰ τοῦ ι γράφεται· οἷον, ἴνδικτος· ἴνδαλμα· ἴνδιξ· Ἰνδός· 
σινδών· σκινδαψός· Πίνδαρος· γίνδος· γινδαρις· καλινδοῦμαι· 
πινδηρα, ἄροτρον· Πίνδος ὄρος Θεσσαλίας· πίνδακας θραύ-
ματα σανίδων· σκινδαλαγμός· σκίνδιον τὸ λευκόν· φυγίνδα· 
βασιλίνδα· κρυπτίνδα· τὸ ἥνδανε ἐκ τοῦ ἁνδάνω γεγονὸς οὐ 
μάχεται, ὡς οὐδὲ τὸ ἤσχαλλον, εἰστήκειν, ἐκ τοῦ ἀσχάλλω, 
εἰστήκω, γεγονότα εἰς τὸν προλαβόντα κανόνα. 



Theognostus Gramm., Canones sive De orthographia 
Section 89, line 5

Ἡ ι συλλαβὴ εἴτε κατ' ἀρχὴν λέξεως, εἴτε κατὰ τὸ 
μέσον ἐν ἁπλῇ καὶ ἀκινήτῳ λέξει λήγουσα εἰς ρ, ἢ εἰς ἕν τι 
τῶν ἀμεταβόλων τῆς ἑξῆς συλλαβῆς ἀρχομένης ἐκ συμφώ-
νου, διὰ τοῦ ι γράφεται· καὶ κατ' ἀρχὴν μὲν λέξεως· ἴννος· 
Ἴμβρος· ἰνδάλω· ἴνδαλμα· Ἰνδός· ἴνδικτος· Συλιμβρία· 
Κιμμέριος· σκίνπους· στίλβων· στιλβανός· οἰκτίρμων· 
σκιρτῶ· κιρνῶ· κίρκος· θίῤῥον τὸ τρυφερόν· ἰλκαγλοιός, ῥύ-
πος· ἴλλον, πλάγιον, στραβόν· ἰλλάδας ἀγελαίας, διεστραμ-
μένας· Ἰλλυρίς· ἰλμηδεσμὸς, σειρά· ἴννος ὁ ἐξ ὄνου θηλείας 
καὶ ἵππου, ἢ τὸ ἐν τῇ κυήσει νοσῆς βρέφος· κιλλαγκτὴρ ὁ 
ὀνελάτης· κίλλοι ὄνοι τὸ ἑρμὸς ἐκ τοῦ εἵρω γεγονὸς τοῦ 
τάσσω ἐν κινήσει ὂν, οὐκ ἀντίκειται· σεσημείωται τὸ εἴργω 
τὸ κωλύω, ἐξ οὗ καὶ εἰρκτὴ διὰ τῆς ει διφθόγγου· καὶ μή-
ποτε παρὰ τὸ ἔργω ἐστὶν ἐν πλεονασμῷ τοῦ ι, καὶ οὐ

γνη-



Theognostus Gramm., Canones sive De orthographia 
Section 90, line 5

Ἡ ι συλλαβὴ ἐν ἁπλῇ καὶ ἀκινήτῳ λέξει, εἴτε κατ' ἀρ-
χὴν λέξεως, εἴτε κατὰ τὸ μέσον, πρὸ τοῦ κτ εὑρισκομένη διὰ 
τοῦ ι γράφεται· ἴκτινος, τὸ ὄρνεον· ἴκτερος, ἡ νόσος· ἱκτὸν 
τὸ ἐοικός· ἴκτης, οἰκέτης· ἴκταιον τὸ τρόφημον· ἵκταρ τὸ   
πρόσφατον· ἰκτὺς, ὁμοίωμα, εἰκών· ἴνδικτος· βίκτωρ· προ-
τίκτωρ· τὸ ἐπείκτης ἐκ τοῦ ἐπείγω γεγονὸς κεκινημένον ὂν, 
οὐκ ἀντίκειται. 



Theognostus Gramm., Canones sive De orthographia 
Section 157, line 4

Τὰ διὰ τοῦ ιων ὑπὲρ δύο συλλαβὰς, μὴ ἔχοντα πρωτοτύ-
που φωνῆς τὴν ει δίφθογγον, ὡς ἔχει τὸ Καδμεῖος Καδ-
μείων, καὶ Ἀργεῖος Ἀργείων, διὰ τοῦ ι γράφεται· οἷον, Ἀμ-
φίων· Ὑπερίων· Ἀμφικτίων· περικτίων· ἰνδικτίων. 



Theognostus Gramm., Canones sive De orthographia 
Section 293, line 1

Τὰ διὰ τοῦ ινδος, εἴτε δισύλλαβα, εἴτε ὑπὲρ δύο συλλα-
βὰς, διὰ τοῦ ι γράφονται· οἷον, Ἰνδός· Ἄλινδος· Ἴσινδος 
πόλις Μακεδονίας· Ἄρινδος ὄνομα ποταμοῦ· Βερέκινδος. 



Theognostus Gramm., Canones sive De orthographia 
Section 851, line 1

Τὰ διὰ τοῦ ινδῶ ῥήματα περισπώμενα διὰ τοῦ ι γρά-
φει τὴν παραλήγουσαν· κυλινδῶ· ἀλινδῶ· καλινδῶ. 



Theognostus Gramm., Canones sive De orthographia 
Section 999, line 1

Τὰ διὰ τοῦ ινδα ἐπιῤῥήματα παροξύνονται, καὶ διὰ τοῦ ι 
γράφονται, καὶ ἐπὶ παιδίων λαμβάνονται, καὶ πρὸς αἰτιατι-
κὴν συντάσσεται· οἷον, βασιλίνδα παιδίαν· χυτρίνδα· δρα-
πετίνδα· ποσίνδα· ἐπαιτίνδα· ξιφίνδα· δαληκίνδα· μυΐνδα, 
ἀπὸ τοῦ μύειν τοὺς ὀφθαλμοὺς, καὶ ἐρωτώμενον λέγειν τινὰ   
τάδε καὶ πόσα τάδε, ἐάν τις ἐπιτύχῃ· φυγίνδα· χυτρίνδα· 
ὀστρακίνδα, καὶ εἴτι ἕτερον. 



Pseudo-Zonaras Lexicogr., Lexicon (3136: 001)
“Iohannis Zonarae lexicon ex tribus codicibus manuscriptis, 2 vols.”, Ed. Tittmann, J.A.H.
Leipzig: Crusius, 1808, Repr. 1967.
Alphabetic letter alpha, page 231, line 18

               πέτρα ἐν τῇ Ἰνδικῇ, ἐν ᾗ οὐ κάθηται 
 ὄρνις διὰ τὸ εἶναι αὐτὴν ὑπερύψηλον. 



Pseudo-Zonaras Lexicogr., Lexicon 
Alphabetic letter alpha, page 327, line 8

                παρὰ τὸ ἄχω, ἀφ' οὗ ἄχομαι, 
 γίνεται ἀχάλλω, ὥσπερ ἄγω ἀγάλλω, εἴδω εἰ-
 δάλλω καὶ ἰνδάλλω. 



Pseudo-Zonaras Lexicogr., Lexicon 
Alphabetic letter delta, page 478, line 20

κατὰ γὰρ τὴν τῶν Ἰνδῶν φωνὴν δεῦνος ὁ βα-
 σιλεύς. 



Pseudo-Zonaras Lexicogr., Lexicon 
Alphabetic letter iota, page 1109, line 8

                 κύριον. 
[<Ἴνδακος>. 



Pseudo-Zonaras Lexicogr., Lexicon 
Alphabetic letter iota, page 1109, line 9

                 κύριον.] 
<Ἰνδοί>. 



Pseudo-Zonaras Lexicogr., Lexicon 
Alphabetic letter iota, page 1110, line 4

                  παρὰ Ῥωμαίοις τοῖς ἀσθενέσι διδό-
 μενον σιτίον, ὃ οὔτε ζῇν οὔτε ἀποθνήσκειν ποιεῖ. 
<Ἰνδικτίων>. 



Pseudo-Zonaras Lexicogr., Lexicon 
Alphabetic letter iota, page 1110, line 4

                  λέγεται καὶ ἴνδικτος. 



Pseudo-Zonaras Lexicogr., Lexicon 
Alphabetic letter iota, page 1110, line 12

            ὄνομα θεᾶς. 
<Ἴνδαλμα>. 



Pseudo-Zonaras Lexicogr., Lexicon 
Alphabetic letter iota, page 1110, line 14

                                              [παρὰ 
 τὸ εἴδω, τὸ ὁμοιῶ, εἰδάλλω καὶ ἀποβολῇ τοῦ <ε> 
 καὶ πλεονασμῷ τοῦ <ν>, ἴνδαλμα. 



Pseudo-Zonaras Lexicogr., Lexicon 
Alphabetic letter iota, page 1110, line 21

               ἰσχὺν παρέχω. 
<Ἰνδάλλεται>. 



Pseudo-Zonaras Lexicogr., Lexicon 
Alphabetic letter iota, page 1123, line 8

               ὁ ποταμὸς ὁ παρ' Ἑβραίοις Φεισσὼν, 
 παρὰ δὲ Ἰνδοῖς Γάγγης, παρὰ δὲ Αἰθίοψιν Ἰν-
 δὸς, παρ' Ἕλλησι δὲ Δανούβιος. 



Pseudo-Zonaras Lexicogr., Lexicon 
Alphabetic letter kappa, page 1209, line 2

                                  λέγει δὲ καὶ Διονύσιος, 
 ὅτι ἔθνος ἐστὶν Ἰνδικόν. 



Pseudo-Zonaras Lexicogr., Lexicon 
Alphabetic letter nu, page 1409, line 11

                οἱ μὲν τὸ Σίναιον ὄρος φασὶν, οἱ δὲ 
 ἕτερον ὄρος τοῦ Διονύσου, ἐνδότερον καὶ ἐπέ-
 κεινα τῆς Ἰνδικῆς. 



Pseudo-Zonaras Lexicogr., Lexicon 
Alphabetic letter sigma, page 1686, line 30

                          ἡ κειμένη πρὸς τῇ θαλάς-
 σῃ τῇ Ποντικῇ τῇ φερομένῃ ἐπὶ Περσίδα καὶ 
 Ἰνδίαν καλεῖται Συρία. 



Pseudo-Zonaras Lexicogr., Lexicon 
Alphabetic letter tau, page 1738, line 13

                λίθος διαυγέστατος καὶ ὑποχλωρίζων, 
 ἐν Τοπάζῳ, πόλει Ἰνδικῇ, εὑρισκόμενος· ὃς 
 τριβόμενος ἐν ἰατρικῇ ἀκόνῃ οὐκ ἐρυθρὸν ἀπο-
 δίδωσι κατὰ τὸ χρῶμα χυμὸν, ἀλλὰ γαλακτώ-
 δη· ἐμπίπλησι δὲ κρατῆρας, ὅσους ἂν ἐθέλῃ ὁ 
 ὑποτρίβων, καὶ οὔτε τῷ σταθμῷ, οὔτε τῇ πε-
 ριφερείᾳ ἐλαττοῦται, ὃ καὶ παράδοξον. 




Scholia In Lucianum, Scholia in Lucianum (scholia vetera et recentiora Arethae) (5029: 001)
“Scholia in Lucianum”, Ed. Rabe, H.
Leipzig: Teubner, 1906, Repr. 1971.
Lucianic work 14, section 19, line 2

                                            ~ VΓφOUΩ 
<μίσγονται μὲν ἀναφανδόν>] Ἡρόδοτον [3, 101] κωμῳδεῖ 
Ἰνδοῖς τοιοῦτον γίνεσθαι ἱστοροῦντα. 



Scholia In Lucianum, Scholia in Lucianum (scholia vetera et recentiora Arethae) 
Lucianic work 77,25, section 6, line 2

                      >] σὺ σωφρονέστερον διε⌊γέν⌋ου, Ἀμμω-  
νίδη, Ἰνδίας ⌊ἀπο⌋χωρῶν, ὃς πᾶσαν τὴν ⌊ἀπὸ⌋ ταύτης ἐπὶ 
Πέρσας ὁδὸν ⌊ἀσε⌋λγέστατα πάντων διῄεις αὐτῷ ⌊στρ⌋ατο-
πέδῳ μέθῃ καὶ γυναι⌊κείᾳ ὕ⌋βρει δουλεύων, καὶ τοῦτο 
⌊  ἐκ⌋εῖνο τὸν Λυαῖον Διόνυ⌊σον  ⌋ καὶ λυσιμελῆ μετιών, 
⌊καὶ⌋ ταῦτα φιλοσοφεῖν ἐ⌊π⌋ανῃρημένος. 



(H)eren(n)ius Philo Gramm., Hist., Fragmenta (1416: 006)
“FGrH \#790”.
Volume-Jacobyʹ-F 3c,790,F, fragment 26, line 4

                                        > ἔστι καὶ νῆσος μία τῶν Κυκλάδων· 
<καὶ <γ> Ἰνδικῆς, ἣν ἀναγράφει Φίλων> καὶ Δημοδάμας ὁ Μιλήσιος 
(428 F 3). 



(H)eren(n)ius Philo Gramm., Hist., Fragmenta (1416: 008)
“FHG 3”, Ed. Müller, K.
Paris: Didot, 1841–1870.
Fragment 16, line 11

                                               Ἔστι καὶ 
νῆσος μία τῶν Κυκλάδων (?), καὶ τρίτη Ἰνδικῆς, 
ἣν ἀναγράφει Φίλων καὶ Δημοδάμας ὁ Μιλήσιος. 



Hippocrates et Corpus Hippocraticum Med., De mulierum affectibus i–iii (0627: 036)
“Oeuvres complètes d'Hippocrate, vol. 8”, Ed. Littré, É.
Paris: Baillière, 1853, Repr. 1962.
Section 81, line 13

                  Ἢ κόκκους ἐκλέψαντα ὅσον τρεῖς ἰνδικοῦ φαρμάκου, 
τοῦ τῶν ὀφθαλμῶν, ὃ καλέεται πέπερι, καὶ τοῦ στρογγύλου, τρία 
ταῦτα λεῖα τρίβειν, καὶ οἴνῳ παλαιῷ χλιηρῷ διεὶς, βαλάνιον περὶ 
πτερὸν ὄρνιθος τιθέναι, καὶ ὧδε προσάγειν. 



Hippocrates et Corpus Hippocraticum Med., De mulierum affectibus i-iii 
Section 158, line 13

              Ἢ ἐκλέψας κόκκους πεντεκαίδεκα, ἔστω δὲ καὶ ἰν-
δικοῦ ποσὸν, ἢν δοκέῃ δεῖν, ἐν γάλακτι δὲ γυναικὸς κουροτρόφου 
τρίβειν, καὶ παραμίσγειν ἐλάφου μυελὸν καὶ τἄλλα ὁκόσα εἴρηται, 
καὶ μέλιτι ὀλίγῳ μίσγειν· τὸ δὲ εἴριον μαλθακὸν καθαρὸν ἔστω, καὶ 
προστίθεσθαι τὴν ἡμέρην· ἢν δὲ βούλῃ ἰσχυρότερον ποιέειν, σμύρ-
νης σμικρόν τι παραμίσγειν· ἄριστον δὲ ὠοῦ τὸ πυῤῥὸν καὶ αἰγὸς 
στέαρ καὶ μέλι καὶ ἔλαιον ῥόδινον, τουτέοισιν ἀναφυρῇν, παραχλιαί-
νειν δὲ παρὰ τὸ πῦρ καὶ τὸ ἀποστάζον εἰρίῳ ξυλλέγειν καὶ προστιθέ-
ναι. 



Hippocrates et Corpus Hippocraticum Med., De mulierum affectibus i-iii 
Section 185, line 15

                                                              Τοῦτο τὸ 
φάρμακον ὀδόντας καθαίρει καὶ εὐώδεας ποιέει· καλέεται δὲ ἰν-
δικὸν φάρμακον. 



Hippocrates et Corpus Hippocraticum Med., De mulierum affectibus i-iii 
Section 205, line 13

                                                 Ἕτερον προσθετόν· 
ἐκλέψας κόκκους τριήκοντα, τὸ ἰνδικὸν, ὃ καλέουσιν οἱ Πέρσαι πέ-
περι, καὶ ἐν τουτέῳ ἔνι στρογγύλον, ὃ καλέουσι μυρτίδανον, ξὺν γά-
λακτι γυναικείῳ ὁμοῦ τρίβειν καὶ μέλιτι διιέναι· ἔπειτα εἴριον μαλ-
θακὸν καὶ καθαρὸν ἀναφυρήσας, περὶ πτερὸν περιελίξας προσθεῖναι, 
καὶ τὴν ἡμέρην ἐῇν· ἢν δὲ ἰσχυρότερον βούλῃ ποιῆσαι, σμύρναν ὀλί-
γην παραμίσγειν ὅσον τριτημόριον, καὶ εἴριον μαλθακὸν καθαρὸν ἢ 
ἡμίῤῥυπον. 



Hippocrates et Corpus Hippocraticum Med., Epistulae (0627: 055)
“Oeuvres complètes d'Hippocrate, vol. 9”, Ed. Littré, É.
Paris: Baillière, 1861, Repr. 1962.
Epistle 18, line 8

                                                  Ὁκόσα γὰρ ἰνδαλμοῖσι 
διαλλάττοντα ἀνὰ τὸν ἠέρα πλάζει ἡμέας, ἃ δὴ κόσμῳ ξυνεώραται   
καὶ ἀμειψιρυσμέοντα τέτευχε, ταῦτα νόος ἐμὸς φύσιν ἐρευνήσας 
ἀτρεκέως ἐς φάος ἤγαγεν· μάρτυρες δὲ τουτέων βίβλοι ὑπ' ἐμοῖο 
γραφεῖσαι. 



Anonymi Medici Med., De urinis secundum Syros (0721: 012)
“Physici et medici Graeci minores, vol. 2”, Ed. Ideler, J.L.
Berlin: Reimer, 1842, Repr. 1963.
Section 1, line 10

                           πιέτω δὲ τὴν διὰ κρόκου ἢ μάννα 
μετὰ ὕδατος ἢ φοίνικας ἰνδικὰς καὶ τὰς βιόλας τὰς συν-
θέτους ἢ τὸ χεράβιν τῶν βιόλων καὶ εἰ ἔαρ ἐστὶ κἂν γέ-
ρων ὑπάρχῃ, κενωσάτω αἷμα διότι πλησμονὴ αἵματός ἐστιν. 



Anonymi Medici Med., De urinis in febribus (0721: 016)
“Physici et medici Graeci minores, vol. 2”, Ed. Ideler, J.L.
Berlin: Reimer, 1842, Repr. 1963.
Volume 2, page 324, line 21

                                             εἰ δὲ καὶ ὁ κάμνων 
ὑπάρχει ἐπίσημος, πρόσθες τὸ μέλι τοῦ μέλανος καλάμου 
τοῦ ἰνδικοῦ στάθμην οὐγγίας εʹ LL ἢ καὶ ςʹ καὶ πιέτω 
καὶ μετὰ τὴν κένωσιν λαβέτω δίαιταν ἐκ τῶν προειρημέ-
νων λαχάνων. 



Anonymi Medici Med., De urinis in febribus 
Volume 2, page 325, line 29

                                               πότιζε δὲ αὐτὸν 
καθ' ἡμέραν τὸ ὀξυσάχαρ μετὰ τῆς πτισάνης καὶ ἀπόφυγε 
τὰ θερμὰ ἀπό τε διαίτης καὶ θεραπείας, διότι ἐὰν θρέ-
ψῃς αὐτὸν μετὰ τῆς θερμῆς διαίτης καὶ τῆς ἰατρείας, 
ἐξέρχεται εἰς ἴκτερον ὁ λεγόμενος χρυσιασμὸς ἀσκὸς καὶ 
λύρα, βλάπτεται τὸ ὅλον σῶμα, καὶ οἱ ὀφθαλμοί, καὶ τὸ 
κλοκίον κίτρινον, καὶ ὁ ἀφρὸς αὐτοῦ, καὶ ἐὰν ἴδῃς τὸ τοι-
οῦτον κλοκίον, εἰπὲ ἴκτερον ἔχει καὶ ἁρμόζει ἰατρεύεσθαι 
ὡς τριταῖος, πινέτω δὲ τὸν χυλὸν τῶν ἰνδίβων ἐξαφρισμέ-
νον καὶ τοῦ στρύχνου μετὰ ὀξυμέλιτος καὶ πᾶν ὄξυνον 
φαγέτω, ἐκτὸς τῶν ἁλμυρῶν καὶ φλεβοτόμησον, εἰ πλη-
θωρικὸν σῶμα ἔχει ὁ κάμνων καὶ δύναμιν, καὶ ἡλικίαν 
καὶ δύναμιν, κενοῦται γὰρ ἡ χολή. 



Dionysius Hist., Fragmenta (2354: 002)
“FHG 2”, Ed. Müller, K.
Paris: Didot, 1841–1870.
Fragment 5, line 7

              πrotr. c. 4: 
Πολλοὶ δ' ἂν τάχα που θαυμάσειαν   
εἰ μάθοιεν τὸ Παλλάδιον τὸ διοπετὲς καλούμενον, ὃ 
Διομήδης καὶ Ὀδυσσεὺς ἱστοροῦνται μὲν ὑφελέσθαι ἀπὸ 
Ἰλίου, παρακαταθέσθαι δὲ Δημοφῶντι, ἐκ τοῦ Πέλο-
πος ὀστῶν κατεσκευάσθαι, καθάπερ τὸν Ὀλύμπιον ἐξ 
ἄλλων ὀστῶν Ἰνδικοῦ θηρίου. 



Hesychius Lexicogr., Lexicon (Α – Ο) (4085: 002)
“Hesychii Alexandrini lexicon, vols. 1–2”, Ed. Latte, K.
Copenhagen: Munksgaard, 1:1953; 2:1966.
Alphabetic letter alpha, entry 4350, line 1

                                   ἢ ἄνωθεν, ἐν ὕψει, ἄνω 
<ἀνάκης>· ὄρνεόν τι Ἰνδικόν, ὅμοιον ψάρῳ 
<ἀνακῆσαι>· ἡσυχάσαι 
<ἀνακηκίει>· ἀναφέρεται (Ν 705) 
<ἀνακείρει>· ἀποτέμνει 
*<ἀνακεφαλαιοῦται>· συμπληροῦται. 



Hesychius Lexicogr., Lexicon (Α – Ο) 
Alphabetic letter alpha, entry 6399, line 1

                                      ρegn. 9,24) αςn 
†<ἀποκολοκαύτωσις>· παρ' †Ἰνδοῖς ἡ συνουσία. 



Hesychius Lexicogr., Lexicon (Α – Ο) 
Alphabetic letter beta, entry 90, line 1

                                  καὶ <βαΐων> 
<βαισήνης>· παρ' Ἰνδοῖς στρατόπεδον 
<βαίσηνος>· ὁ στρατός 
†<βαισσόν>· βάθος 
<βαίταν>· Ἕλληνες   
<βαιτάς>· εὐτελὴς γυνή. 



Hesychius Lexicogr., Lexicon (Α – Ο) 
Alphabetic letter beta, entry 900, line 3

                     ) καὶ ὁ Ἀλεξάνδρου ἵππος, ἀφ' οὗ πόλιν ἐν 
 Ἰνδοῖς κτίσαι λέγεται 
*<βουκολέοντι>· <βοῦς νέμοντι> (Ε 313) Sn 
<βουκολητής>· ἀπατεών 
<βουκόλια>· ἀγέλη βοῶν (Hdt. 1,126,2? 1. 



Hesychius Lexicogr., Lexicon (Α – Ο) 
Alphabetic letter beta, entry 1076, line 1

καὶ θύεται αἴξ 
<βραυῶσα>· κεκραγυῖα hf 
<βράχαλον>· χρεμετισμόν 
*<βράχε>· ἐψόφησε (Μ 396 etc.) n 
(*)<βραχεῖν>· ἠχῆσαι (Sn) ψοφῆσαι (n) 
†<βραχιόνα>· τὸν τράχηλον 
*<βραχεῖς>· ὀλίγοι (Psalm. 104,12) vgAS 
<βράχιστον>· ἐλάχιστον (Soph. fr. 172) 
<βραχίων>· βραχύτατος 
<Βραχμᾶνες>· οἱ παρ' Ἰνδοῖς Γυμνοσοφισταὶ καλούμενοι 
<βραχύ>· ἀντὶ τοῦ οὐδέν (Eur. Tro. 1248) ὀλίγον, *μικρόν (Eur. 
 Phoen. 738) g 
[<βραχυτελόν>· μικρόν] 
a) <βραχύλον>· . 



Hesychius Lexicogr., Lexicon (Α – Ο) 
Alphabetic letter gamma, entry 142, line 1

                           ..> 
*<γαναυγέας>· τέλειος ἐν τῷ ὁρᾶν AS 
[<γανδᾶν ἢ] γανᾶν>· λάμπειν 
<γανδάνειν>· ἀρέσκειν 
<Γάνδαρος>· ὁ ταυροκέρατος, παρ' Ἰνδοῖς 
<γάνδιον>· κιβώτιον 
<γάνδομα>· πυροί 
<γανδόμην>· ἄλευρα 
<γάνδος>· ὁ πολλὰ εἰδὼς καὶ πανοῦργος. 



Hesychius Lexicogr., Lexicon (Α – Ο) 
Alphabetic letter gamma, entry 217, line 1

                           vgAS ἀκαταπλήκτῳ gAS 
<γαυσάδας>· ψευδής 
<γαυσαλίτης>· ὄρνεον, παρὰ Ἰνδοῖς 
<γαυσόν>· σκαμβόν, στρεβλόν (Hippocr. artic. 77 . 



Hesychius Lexicogr., Lexicon (Α – Ο) 
Alphabetic letter delta, entry 2248, line 1

            καὶ πρόσωπον <κωφόν> 
<δορυφοροῦντες>· προβαδίζοντες ἔνοπλοι r 
<δορυφόρους>· ἀναβαστάζοντας S τὰ ὅπλα 
[<δόρων>· τῶν δοράτων] 
<δορχελοί>· ἀστράγαλοι 
<δόσαν>· παρέσχον, ἔδοσαν (Α 162) 
†<Δορσάνης>· ὁ Ἡρακλῆς, παρ' Ἰνδοῖς 
<δοσείειν>· δοτικῶς ἔχειν 
*<δόσις>· ἡ δωρεά (Κ 213 . 



Hesychius Lexicogr., Lexicon (Α – Ο) 
Alphabetic letter epsilon, entry 5716, line 1

                            Αἰθίοπες, Ἄραβες, ⌊Ἰνδοὶ r Ἀράβιοι (δ 84) 
<ἐρεμνή>· σκοτεινή (AS). 



Hesychius Lexicogr., Lexicon (Α – Ο) 
Alphabetic letter epsilon, entry 6694, line 1

                          εὔμοιρος 
*†<εὐαλῶς>· εὐχερῶς θηρώμενος ASn 
<εὐαλδῆ>· εὐαυξῆ 
<Εὐαλωσία>· Δημήτηρ, ὅτι μεγάλας τὰς ἅλως ποιεῖ καὶ πληροῖ 
<εὐάλωτον>· εὐθήρατον (Prov. 24,63) r. g 
<εὐαμερία>· θεοσημία 
<εὐάν>· ὁ κισσός, ὑπὸ Ἰνδῶν 
<Εὐάνασσα>· ἡ Δημήτηρ 
*<εὐανδρείας>· καλῆς ἰσχύος (2. 



Hesychius Lexicogr., Lexicon (Α – Ο) 
Alphabetic letter iota, entry 665, line 1

                                              λέγονται δὲ καὶ *αἱ τῶν 
 ἀδελφῶν γυναῖκες <ἰνάτερες> n 
<ἳν αὐτῷ>· αὐτὸς αὐτῷ (Hes. fr. 11 Rz.) 
<Ἰνάχεια>· ἑορτὴ Λευκοθέας ἐν †Κρήτεσιν, ἀπὸ Ἰνάχου 
*<Ἴναχος>· ποταμός rA <Θεσσαλίας> A 
<ἰνδάλλεται>· *ὁμοιοῦται vg, φαίνεται (ψ 460) Avg, δοκεῖ. 



Hesychius Lexicogr., Lexicon (Α – Ο) 
Alphabetic letter iota, entry 666, line 1

                             σοφίζεται An 
*<ἰνδάλλετο>· ὡμοιοῦτο (Ρ 213) r. n 
*<ἰνδάλλονται>· φαίνονται An καὶ τὰ ὅμοια 
*<ἰνδάλματα>· φαντάσματα. 



Hesychius Lexicogr., Lexicon (Α – Ο) 
Alphabetic letter iota, entry 670, line 1

                              Μακεδόνες 
*<Ἰνδός>· ὁ τὸν ἐλέφαντα ἄγων ἀπὸ Αἰθιοπίας (1. 



Hesychius Lexicogr., Lexicon (Α – Ο) 
Alphabetic letter iota, entry 671, line 1

                                                                      μacc. 6,37) 
<ἰνδουρός>· ἀσπάλαξ r 
<ἵν' ἔκδηλος>· ἵν' ἐπίσημος ᾖ (Ε 2) 
<ἰνέκεσθαι>· μαθεῖν 
*<ἶνες>· νεῦρα (λ 219) ASgn   
<ἰνεύει>· τείνει 
<ἰνηθεῖσα>· καθαρθεῖσα, κενωθεῖσα (Hippocr.) 
[<ἱνία>· λῶρα] 
<ἰν ἱμίνᾳ>· ἐν ἡμίσει 
<ἰνίον>· τὸ ὄπισθεν τοῦ τραχήλου νεῦρον (Ε 73) r. καὶ ἡ συνα-
 γωγὴ τῶν χειρῶν πρὸς ἀλλήλας. 



Hesychius Lexicogr., Lexicon (Α – Ο) 
Alphabetic letter kappa, entry 28, line 1

                           ⌊ξηραίνει A 
<κάγκανα ξύλα>· ξηρά An. ἐλαφρά (Φ 364) 
<καγκαλέα>· κατακεκαυμένα 
<κάγκαμον>· παρ' Ἰνδοῖς ξύλου δάκρυον, καὶ θυμίαμα 
<καγκές>· πτύελος 
<καγκομένης>· ξηρᾶς τῷ φόβῳ 
<καγκύλας>· κηκῖδας. 



Hesychius Lexicogr., Lexicon (Α – Ο) 
Alphabetic letter kappa, entry 2328, line 1

                                        ἀπὸ τῶν εὑρόντων 
<Κερκέται>· ἔθνος Ἰνδικόν 
<κερκίδας>· δονακίνας. 



Hesychius Lexicogr., Lexicon (Α – Ο) 
Alphabetic letter kappa, entry 2730, line 1

                                                           καὶ Ἰνδοί 
<κίνδυνος ἡ ἐν πρῷρᾳ σελίς>· οἱ πολέμιοι γὰρ τὴν πρῴραν 
 εὐθέως ἐφάλλονται. 



Hesychius Lexicogr., Lexicon (Α – Ο) 
Alphabetic letter kappa, entry 2800, line 1

           (Σa) [ἢ] ἅπαξ <εἰρημένον> (ζ 76) 
<κιτρίον>· τὸ Ἰνδικὸν μῆλον r 
<κιτταί>· πρόγονοι 
<κιττᾶν>· γλίχεσθαι S· ἐπὶ τῶν γυναικῶν· ⌊ἐπιθυμεῖν S 
<κίτταλος>· μυῤῥίνη 
<κιττάναλον>· ἡ κρησέρα 
<κίττανος>· ἡ κονιακὴ τίτανος 
<κίτταρις>· διάδημα, ὃ φοροῦσι Κύπριοι. 



Hesychius Lexicogr., Lexicon (Α – Ο) 
Alphabetic letter mu, entry 63, line 1

                                    οἱ δὲ μέτρα, ὡς κύαθοι (Blaes. fr. 2 
 Kaib.) 
<μάθαμι>· ζητῶ 
<μάθας>· μαθήσεως 
<μαθήματα>· ἃ οἱ ὑποκριταὶ ἀνελάμβανον 
<μάθυιαι>· γνάθοι 
<μαΐ>· μέγα. Ἰνδοί 
<μαῖα>· πατρὸς καὶ μητρὸς μήτηρ. 



Hesychius Lexicogr., Lexicon (Α – Ο) 
Alphabetic letter mu, entry 95, line 1

                              Ταραντῖνοι δὲ <μαιριῆν>· τὸ κακῶς ἔχειν 
<μαίσωλος>· ζῷον τετράπουν, γενόμενον ἐν τῇ Ἰνδικῇ, ὅμοιον 
 μόσχῳ 
<μαίσων>· μάγειρον. 



Hesychius Lexicogr., Lexicon (Α – Ο) 
Alphabetic letter mu, entry 215, line 1

                                  Δόλοπες   
<Μάμερτος>· Ἄρης 
<μαμᾶτραι>· οἱ στρατηγοί, παρ' Ἰνδοῖς 
<μαμμάκυθος>· μωρός S. ἔστι δὲ καὶ δρᾶμα πεποιημένον Πλά-
 τωνι 
<Μαμβρή>· ἀπὸ ὁράσεως 
<μαμμᾶν>· ἐπὶ τῆς παιδικῆς φωνῆς. 



Hesychius Lexicogr., Lexicon (Α – Ο) 
Alphabetic letter mu, entry 1686, line 1

*μοῖρα τοῦ βίου (AS) 
<μοροπονοῦν>· κακοπαθοῦν 
<μόροττον>· ἐκ φλοιοῦ πλέγμα τι, ᾧ ἔτυπτον ἀλλήλους τοῖς 
 Δημητρίοις 
<μορσική>· †ἡ ἰνδική† 
<μόρσιμοι>· ἀναγκαῖαι, *εἱμαρμέναι, μεμοιραμέναι (ASvgb). 



Hesychius Lexicogr., Lexicon (Α – Ο) 
Alphabetic letter mu, entry 2067, line 1

                                                               .) (αvgn) 
<μώνυχα>· ἁπλῆν καὶ <μὴ> διεστῶσαν 
[<μῶρα>· συκάμινα] 
<μωραίνει>· ἀφραίνει, παρακόπτει, μαίνεται 
<μωρίαι>· ἁμαρτίαι 
<μωρίαι>· ἵπποι καὶ βοῦς ὑπὸ Ἀρκάδων 
<Μωριεῖς>· οἱ τῶν Ἰνδῶν βασιλεῖς (Euphor. fr. 168 Pow.) 
<μώριον>· πόα τις, ᾗ πρὸς φίλτρα χρῶνται r 
<μωρός>· ἄφρων, μάταιος 
<μωρόν>· ὀξύ. 



Hesychius Lexicogr., Lexicon (Α – Ο) 
Alphabetic letter nu, entry 742, line 3

                                                                   ἔστι γὰρ Ἀρα-
 βίας, Αἰθιοπίας, Αἰγύπτου, Βαβυλῶνος, Ἐρυθρᾶς, Θρᾴκης, 
 Θετταλίας, Κιλικίας, Ἰνδικῆς, Λιβύης, Λυδίας, Μακεδονίας, 
 Νάξου, περὶ τὸ Πάγγαιον, τόπος Συρίας 
<νύσσει>· παίει. 



Hesychius Lexicogr., Lexicon (Π – Ω) (4085: 003)
“Hesychii Alexandrini lexicon, vols. 3–4”, Ed. Schmidt, M.
Halle: *n.p., 3:1861; 4:1862, Repr. 1965.
Alphabetic letter pi, entry 4214, line 1

                                     ὅταν γὰρ τὰ μήπω δυνάμενα πέτεσθαι 
 τῶν ὀρνέων πειράζοντα ἐπιβάλληται καὶ κινῇ τὰς πτέρυγας <πτερυ-
 γίζειν> λέγεται 
<πτερύγιον>· ἀκρωτήριον, πτέρυξ 
<πτερυξαμένη>· διασείσασα 
<πτερύσσεται>· τὰ πτερὰ τινάσσει, πέτεται 
<πτερ[ε]υγοτύραννος>· ὄρνις ποιὸς ἐν Ἰνδικῇ Ἀλεξάνδρῳ δοθείς 
<πτέρων>· 
   ἀλλ' ἢ τρίορχος, ἢ πτέρων, ἢ στρουθίας 
 εἶδος ὀρνέου 
<πτερῶν ταρσοῦ>· τῶν πτερῶν 
<πτερωτοῖς>· πετεινοῖς 
<πτερωτός>· ἀναπτερωθείς. 



Hesychius Lexicogr., Lexicon (Π – Ω) 
Alphabetic letter sigma, entry 81, line 1

                                                       θέλει δὲ εἰπεῖν τὴν πήραν, 
 κατὰ μετάληψιν 
[<σακοφόροι>· ὁπλοφόροι] 
<σάκταρον>· τοῦτο ἐμφερές ἐστι κόμ(μ)ει, γεν(ν)ώμενον ἐν τῇ Ἰνδικῇ, 
 διαλυτικόν 
<σάκτας>· ὁ θύλακος 
<σακτῆρος>· θυλάκου. 



Hesychius Lexicogr., Lexicon (Π – Ω) 
Alphabetic letter sigma, entry 151, line 1

                                                        ἔστι δὲ καὶ ἑτέρα 
 ἱστορία, δι' ἣν <πολυγράμματον> ἔφη <δῆμον>· ἐπειδὴ Ἑλλήνων 
 Σάμιοι πολυγράμματοι ἐλέγοντο πρῶτοι καὶ χρησάμενοι καὶ δι(α)δόντες 
 εἰς τοὺς ἄλλους Ἕλληνας τὴν διὰ τῶν τεσσάρων καὶ εἴκοσι στοιχείων 
 χρῆσιν 
<σάμμα>· ὄργανον μουσικὸν παρὰ Ἰνδοῖς 
<Σάμου ὑληέσσης Θρηϊκίης>· τῆς Σαμοθρᾴκης. 



Hesychius Lexicogr., Lexicon (Π – Ω) 
Alphabetic letter sigma, entry 682, line 1

                             καὶ ἡ πόρνη 
<Σινδικὸν διάσφαγμα>· τὸ τῆς γυναικός 
<σίνδις>· γέρων 
<Σίνδοι>· ἔθνος Ἰνδικόν. 



Hesychius Lexicogr., Lexicon (Π – Ω) 
Alphabetic letter sigma, entry 1368, line 1

                                                 σώζεται 
<σοφία>· πᾶσα τέχνη, καὶ ἐπιστήμη 
*<Σουφείρ>· χώρα, ἐν ᾗ οἱ πολύτιμοι λίθοι, καὶ ὁ χρυσός, ἐν Ἰνδίᾳ 
<σοφίης>· τῆς περὶ τὴν τέχνην σχολῆς 
<σοφίζεται>· σοφόν τι λέγει. 



Hesychius Lexicogr., Lexicon (Π – Ω) 
Alphabetic letter sigma, entry 1994, line 1

                                          Δωριεῖς 
<Στρεφοῦραι>· τῶν Ἰνδῶν γένος τι, οἳ καλοῦνται Κοψίλοι 
<στρέφωσις>· κάλυψις ἀγγείων δέρματι γινομένη 
<Στρεψαῖοι>· ἔθνος περὶ Μακεδονίαν 
<στρεψίμαλ(λ)ος>· μεταφορικῶς λέγουσιν ἀπὸ τῶν ἐρίων. 



Hesychius Lexicogr., Lexicon (Π – Ω) 
Alphabetic letter chi, entry 582, line 1

                                               καὶ ἔλαιον 
<χοανεῦσαι>· χωνεῦσαι 
<χοάνη>· χώνη, τύπος, εἰς ὃν μεταχεῖται τὸ χωνευόμενον 
<χοάνοις>· τοῖς φυσητῆρσι, ταῖς χώναις, καὶ κοιλώμασιν, εἰς ἃ ἐγχεῖται 
 τὸ χωνευόμενον, ἢ τοῖς πηλίνοις τύποις 
<χοάρβηνα>· τὰ γράμματα 
<χοάς>· τὰς σπονδὰς τῶν νεκρῶν 
<χοᾶσθαι>· καυχᾶσθαι 
<χοάς[ς]ομαι>· ἐπίξομαι 
<Χοάσπης>· ποταμὸς Ἰνδίας 
<χοδιτεύειν>· ἀποπατεῖν 
<χόδανον>· τὴν ἕδραν 
<χόες>· χῶναι 
<χοΐ>· χώματι 
[<χόϊνον>· ποτήριον χαλκοῦν] 
<χοΐας>· τὸ ἀθροῖσθαι 
<χοϊκός>· πήλινος, γήϊνος 
<χοίνικες>· αἱ βαθεῖαι πέδαι. 


Achilles Tatius Scr. Erot., Leucippe et Clitophon (0532: 001)
“Achilles Tatius. Leucippe and Clitophon”, Ed. Vilborg, E.
Stockholm: Almqvist \& Wiksell, 1955.
Book 2, chapter 14, section 9, line 1

   ἀλλὰ καὶ λίμνη Λιβυκὴ μιμεῖται γῆν Ἰνδικήν, 
καὶ ἴσασιν αὐτῆς τὸ ἀπόρρητον αἱ Λιβύων παρθένοι, ὅτι τὸ ὕδωρ 
ἔχει πλούσιον. 



Achilles Tatius Scr. Erot., Leucippe et Clitophon 
Book 3, chapter 7, section 5, line 6

                                 ποδήρης χιτών, λευκὸς ὁ χιτών· τὸ 
ὕφασμα λεπτόν, ἀραχνίων ἐοικὸς πλοκῇ, οὐ κατὰ τὴν τῶν προβατείων 
τριχῶν, ἀλλὰ κατὰ τὴν τῶν ἐρίων τῶν πτηνῶν, οἷον ἀπὸ δένδρων 
ἕλκουσαι νήματα γυναῖκες ὑφαίνουσιν Ἰνδαί. 



Achilles Tatius Scr. Erot., Leucippe et Clitophon 
Book 3, chapter 9, section 2, line 5

                                         καὶ ἅμα πλήρης ἦν ἡ γῆ 
φοβερῶν καὶ ἀγρίων ἀνθρώπων· μεγάλοι μὲν πάντες, μέλανες δὲ τὴν   
χροιάν (οὐ κατὰ τὴν τῶν Ἰνδῶν τὴν ἄκρατον, ἀλλ' οἷος ἂν γένοιτο 
νόθος Αἰθίοψ), ψιλοὶ τὰς κεφαλάς, λεπτοὶ τοὺς πόδας, τὸ σῶμα 
παχεῖς· ἐβαρβάριζον δὲ πάντες. 



Achilles Tatius Scr. Erot., Leucippe et Clitophon 
Book 4, chapter 3, section 5, line 4

                                                                        καὶ γὰρ 
δεύτερος φαίνεται εἰς ἀλκὴν ἐλέφαντος Ἰνδοῦ. 



Achilles Tatius Scr. Erot., Leucippe et Clitophon 
Book 4, chapter 4, section 8, line 2

ὁ δὲ ἄνθρωπος ἔλεγεν ὅτι καὶ μισθὸν εἴη δεδωκὼς τῷ θηρίῳ· 
προσπνεῖν γὰρ αὐτῷ καὶ μόνον οὐκ ἀρωμάτων Ἰνδικῶν· εἶναι δὲ 
καὶ κεφαλῆς νοσούσης φάρμακον. 



Achilles Tatius Scr. Erot., Leucippe et Clitophon 
Book 4, chapter 5, section 1, line 3

                 Ὅτι,” ἔφη Χαρμίδης, “τοιαύτην ποιεῖται καὶ τὴν 
τροφήν. Ἰνδῶν γὰρ ἡ γῆ γείτων ἡλίου· πρῶτοι γὰρ ἀνατέλλοντα 
τὸν θεὸν ὁρῶσιν Ἰνδοί, καὶ αὐτοῖς θερμότερον τὸ φῶς ἐπικάθηται, 
καὶ τηρεῖ τὸ σῶμα τοῦ πυρὸς τὴν βαφήν. 



Achilles Tatius Scr. Erot., Leucippe et Clitophon 
Book 4, chapter 5, section 1, line 4

         Ἰνδῶν γὰρ ἡ γῆ γείτων ἡλίου· πρῶτοι γὰρ ἀνατέλλοντα 
τὸν θεὸν ὁρῶσιν Ἰνδοί, καὶ αὐτοῖς θερμότερον τὸ φῶς ἐπικάθηται, 
καὶ τηρεῖ τὸ σῶμα τοῦ πυρὸς τὴν βαφήν. 



Achilles Tatius Scr. Erot., Leucippe et Clitophon 
Book 4, chapter 5, section 2, line 2

   γίνεται δὲ παρὰ τοῖς 
Ἕλλησιν ἄνθος Αἰθίοπος χροιᾶς· ἔστι δὲ παρ' Ἰνδοῖς οὐκ ἄνθος 
ἀλλὰ πέταλον, οἷα παρ' ἡμῖν τὰ πέταλα τῶν φυτῶν· ὃ μὲν κλέπτον 
τὴν πνοὴν καὶ τὴν ὀδμὴν οὐκ ἐπιδείκνυται· ἢ γὰρ ἀλαζονεύεσθαι 
πρὸς τοὺς εἰδότας ὀκνεῖ τὴν ἡδονὴν ἢ τοῖς πολίταις φθονεῖ· ἂν δὲ 
τῆς γῆς μικρὸν ἐξοικήσῃ καὶ ὑπερβῇ τοὺς ὅρους, ἀνοίγει τῆς κλοπῆς 
τὴν ἡδονὴν καὶ ἄνθος ἀντὶ φύλλου γίνεται καὶ τὴν ὀδμὴν ἐνδύεται. 



Achilles Tatius Scr. Erot., Leucippe et Clitophon 
Book 4, chapter 5, section 3, line 1

μέλαν τοῦτο ῥόδον Ἰνδῶν· ἔστι δὲ τοῖς ἐλέφασι σιτίον, ὡς τοῖς 
βουσὶ παρ' ἡμῖν ἡ πόα. 



Antenor Hist., Fragmentum (2322: 003)
“FHG 4”, Ed. Müller, K.
Paris: Didot, 1841–1870.
Fragment 1, line 14

                Λέγει δὲ ὁ Ἀντήνωρ καὶ ἔτι κατὰ τὴν 
Ἴδην τὴν Κρῆσσαν ἐκείνου τοῦ γένους τῶν μελιττῶν 
εἶναι ἰνδάλματα, οὐ πολλὰ μὲν, εἶναι δ' οὖν καὶ πι-
κρὰς ἐντυχεῖν, ὡς ἐκεῖναι ἦσαν. 



Julianus Scr. Eccl., Commentarius in Job (4105: 001)
“Der Hiobkommentar des Arianers Julian”, Ed. Hagedorn, D.
Berlin: De Gruyter, 1973; Patristische Texte uns Studien 14.
Page 173, line 14

                  > 
 οὐδὲν τούτων τῶν ἐπὶ γῆς καλλίστων ἢ τιμίων τῇ σοφίᾳ συγκριθῆναι δύναται· 
οὐκ ἄβυσσος, {οὐ πλῆθος,} οὐ θαλάσσης σεσωρευμένα πελάγη, οὐχ ὁ ταύτης ὄγκος 
περιγράψαι ταύτην <δύναται>· οὐ στιλπνότης ἢ χρυσίου μαρμαρυγή, οὐ λίθων Ἰν-
δικῶν ξένη θέα σέλας ἐκπέμπουσα εἰς τὰς τῶν ὁρώντων ὄψεις, οὐδὲ τὰ κάλλιστα 
τῶν ὄντων, οὐδὲ τὰ ἐκλελεγμένα, οὐχ ὕψος οὐρανοῦ, οὐ πλάτος γῆς ἀντιταλαντεύ-
σει ποτὲ τῇ σοφίᾳ, οὐχ ὁ διειδὴς καὶ καθαρὸς ὕελος. 



Julianus Scr. Eccl., Commentarius in Job 
Page 236, line 1

       χρυσὸν καὶ ἄργυρον καὶ χαλκὸν καὶ σίδηρον μόλιβδον κασσίτερον ἐκ τῶν   
λαγόνων τῆς γῆς ἀνωρύξατο, λίθους Ἰνδικοὺς καὶ πέπλον τὸ ἐκ Σηρῶν σκωλήκων ὂν 
γένημα ἐξεῦρεν, καὶ τῆς θαλαττίας κόχλου τὸ αἷμα πολυπραγμονῶν ἔσχεν τὸ εὐαν-
θὲς τῆς πορφύρας. 



Julianus Scr. Eccl., Commentarius in Job 
Page 236, line 5

                     οὐκ ἔλαθεν αὐτὸν πῖνα, οὐ κατεκρύβη αὐτὸν ὁ βόμβυξ, καὶ οὐκ 
ἔφυγεν αὐτὸν τὰ διάφορα τῶν βυσσοποιῶν χρώματα οὐδὲ τὰ ποικίλα τῶν ἀρωμάτων, 
οὐκ Ἰνδικοὶ κόκκοι, οὐχὶ ποῶν τὰ ἄνθη, οὐ τῶν ῥιζῶν αἱ δυνάμεις, οὐ τῶν ἑρπε-
τῶν αἱ ἐνέργειαι, οὐχὶ ἰχθύων ἢ ὀρνίθων ποικιλία, οὐ τῶν βλαβερῶν ἡ ἀποφυγή, 
οὐ τῶν προσφόρων ἡ χρῆσις, οὐχ ἡ ἐξ ἰατρικῆς ὑγίεια, οὐχὶ τῶν μελιττῶν ἡ ἐργα-
σία, οὐ τοῦ γάλακτος τὸ προσηνές, οὐ τῶν ἀκροδρύων τὸ πολυσχιδές, οὐχ αἱ τού-
των ποιότητες. 



Julianus Scr. Eccl., Commentarius in Job 
Page 291, line 17

εἶδος δέ ἐστι λίθου οὕτω καλούμενον παρ' Ἰνδοῖς, ὅπερ σίδηρος οὐ δύναται 
διατεμεῖν, τοὐναντίον δὲ πρίν τι ἂν αὐτὸν διάθηται αὐτὸς φθάσας θρύπτεται διὰ 
τὸ ἀντιτυπές. 



Julianus Scr. Eccl., Commentarius in Job 
Page 311, line 19

               καὶ Ἡρόδοτος ὁ ἱστοριογράφος φησί· περιτέμνονται δὲ Ἰνδοὶ καὶ   
Αἰγύπτιοι καὶ Ἄραβες καὶ οἱ ἐν Παλαιστίνῃ Σύροι, τοὺς Ἰουδαίους καταλέγων. 



Lycophron Trag., Alexandra (0341: 002)
“Lycophronis Alexandra”, Ed. Mascialino, L.
Leipzig: Teubner, 1964.
Line 254

                          οἰμωγὴ δέ μοι 
ἐν ὠσὶ πύργων ἐξ ἄκρων ἰνδάλλεται, 
πρὸς αἰθέρος κυροῦσα νηνέμους ἕδρας, 
γόῳ γυναικῶν καὶ καταρραγαῖς πέπλων, 
ἄλλην ἐπ' ἄλλῃ συμφορὰν δεδεγμένων. 



Lycophron Trag., Alexandra 
Line 597

Ὁ δ' Ἀργυρίππαν Δαυνίων παγκληρίαν 
παρ' Αὐσονίτην Φυλαμὸν δωμήσεται,   
πικρὰν ἑταίρων ἐπτερωμένην ἰδὼν 
οἰωνόμικτον μοῖραν, οἳ θαλασσίαν 
δίαιταν αἰνήσουσι πορκέων δίκην, 
κύκνοισιν ἰνδαλθέντες εὐγλήνοις δομήν. 



Lycophron Trag., Alexandra 
Line 961

αἱ δ' αὖ παλαιστοῦ μητέρος Ζηρυνθίας 
σηκὸν μέγαν δείμαντο δωτίνην θεᾷ, 
μόρον φυγοῦσαι καὶ μονοικήτους ἕδρας, 
ὧν δὴ μίαν Κριμισός, ἰνδαλθεὶς κυνί, 
ἔζευξε λέκτροις ποταμός· ἡ δὲ δαίμονι 
τῷ θηρομίκτῳ σκύλακα γενναῖον τεκνοῖ, 
τρισσῶν συνοικιστῆρα καὶ κτίστην τόπων. 



Paulus Silentiarius Poeta, Descriptio Sanctae Sophiae (4039: 001)
“Prokop. Werke, vol. 5 [Die Bauten]”, Ed. Veh, O.
Munich: Heimeran, 1977.
Line 230

ἠρεμέει καὶ Μῆδος ἄναξ καὶ Κελτὶς ὁμοκλή, 
καὶ ξίφος ὑμετέροις φιλοτήσιον ὤπασε θώκοις 
Ἰνδὸς ἀνήρ, ἐλέφαντας ἄγων καὶ μάργαρα πόντου· 
Καρχηδὼν γόνυ δοῦλον ἐμοῖς ἔκλινε τροπαίοις. 



Paulus Silentiarius Poeta, Descriptio Sanctae Sophiae 
Line 694

οὐδὲ μὲν οὐ μούνοις ἐπὶ τείχεσιν, ὁππόσα μύστην 
ἄνδρα πολυγλώσσοιο διακρίνουσιν ὁμίλου, 
γυμνὰς ἀργυρέας ἔβαλε πλάκας, ἀλλὰ καὶ αὐτοὺς 
κίονας ἀργυρέοισιν ὅλους ἐκάλυψε μετάλλοις, 
τηλεβόλοις σελάεσσι λελαμπότας, ἑξάκι δοιούς· 
οἷς ἔπι, καλλιπόνοιο χερὸς τεχνήμονι ῥυθμῶι, 
ὀξυτέρους κύκλοιο χάλυψ κοιλήνατο δίσκους, 
ὧν μέσον ἀχράντοιο θεοῦ δείκηλα χαράξας   
ἄσπορα δυσαμένου βροτέης ἰνδάλματα μορφῆς, 
πῆι μὲν ἐϋπτερύγων στρατὸν ἔξεσεν ἀγγελιάων 
αὐχενίων ξυνοχῆα κατακλίνοντα τενόντων 
(οὐ γὰρ ἰδεῖν τέτληκε θεοῦ σέβας οὐδὲ καλύπτρηι 
ἀνδρομέηι κρυφθέντος, ἐπεὶ θεός ἐστιν ὁμοίως 
ἑσσάμενος καὶ σάρκα λυτήριον ἀμπλακιάων), 
πῆι δὲ θεοῦ κήρυκας ὀδοὺς ἤσκησε σιδήρου 
τοὺς προτέρους, πρὶν σάρκα λαβεῖν θεόν, ὧν ἀπὸ φωνῆς 
ἐσσομένου Χριστοῖο διέπτατο θέσπις ἀοιδή. 



Simplicius Phil., In Aristotelis quattuor libros de caelo commentaria (4013: 001)
“Simplicii in Aristotelis de caelo commentaria”, Ed. Heiberg, J.L.
Berlin: Reimer, 1894; Commentaria in Aristotelem Graeca 7.
Volume 7, page 548, line 2

                εἰ δὲ μὴ πάνυ μεγάλη, φησίν, ἐστὶν ἡ γῆ, οὐ χρὴ 
νομίζειν ἄπιστα λέγειν τοὺς ὑπολαμβάνοντας τὸν δυτικώτατον καὶ τὸν ἀνα-  
τολικώτατον τῶν ἐγνωσμένων ἡμῖν τόπων τόν τε περὶ τὰ Γάδειρα καὶ τὰς 
Ἡρακλείους στήλας, ὃν Ἡράκλειαν ἐκάλεσε, καὶ τὸν περὶ τὴν Ἰνδικὴν 
συνάπτειν ἀλλήλοις οὐ πόρρωθεν, καὶ οὕτως εἶναι τὴν θάλατταν μίαν τήν 
τε Ἐρυθρὰν καλουμένην καὶ τὴν παρ' ἡμῖν. 



Oppianus Epic., Halieutica (0023: 001)
“Oppian, Colluthus, Tryphiodorus”, Ed. Mair, A.W.
Cambridge, Mass.: Harvard University Press, 1928, Repr. 1963.
Book 2, line 233

Πουλυπόδων δ' οὔπω τιν' ὀΐομαι ἔμμεν' ἄπυστον   
τέχνης, οἳ πέτρῃσιν ὁμοίϊοι ἰνδάλλονται, 
τήν κε ποτιπτύξωσι περὶ σπείρῃς τε βάλωνται. 



Oppianus Epic., Halieutica 
Book 5, line 17

ὅσσους μὲν κατ' ὄρεσφι βίην ἄτρεστον ἔχοντας 
θῆρας ὑπερφιάλους βροτὸς ἔσβεσεν· ὅσσα δὲ φῦλα 
οἰωνῶν νεφέλῃσι καὶ ἠέρι δινεύοντα 
εἷλε, χαμαίζηλόν περ ἔχων δέμας· οὐδὲ λέοντα 
ῥύσατ' ἀγηνορίη δμηθήμεναι, οὐδ' ἐσάωσεν   
αἰετὸν ἠνεμόεις πτερύγων ῥόθος, ἀλλὰ καὶ Ἰνδὸν 
θῆρα κελαινόρινον ὑπέρβιον ἄχθος ἀνάγκῃ 
κλῖναν ἐπιβρίσαντες, ὑπὸ ζεύγλῃσι δ' ἔθηκαν 
οὐρήων ταλαεργὸν ἔχειν πόνον ἑλκυστῆρα. 



Pseudo-Symeon Hist., Chronographia (partim edita e cod. Paris. gr. 1712) (3182: 001)
“Theophanes Continuatus, Ioannes Cameniata, Symeon Magister, Georgius Monachus”, Ed. Bekker, I.
Bonn: Weber, 1838; Corpus scriptorum historiae Byzantinae.
Page 723, line 3

9. Τῷ δὲ Σεπτεμβρίῳ μηνί, ἰνδικτιῶνος γʹ, Παγκρατού-
κας ὁ Ἀρμένιος τὴν Ἀδριανούπολιν τῷ Συμεὼν προδέδωκεν. 



Pseudo-Symeon Hist., Chronographia (partim edita e cod. Paris. gr. 1712) 
Page 731, line 13

17. Τῇ κδʹ τοῦ Σεπτεμβρίου μηνὸς τιμᾶται Ῥωμανὸς τῇ 
τοῦ Καίσαρος ἀξίᾳ, καὶ τῇ ιζʹ τοῦ Δεκεμβρίου μηνὸς στέφεται 
παρὰ Κωνσταντίνου βασιλέως καὶ Νικολάου πατριάρχου ἐν ἔτει 
τῷ ͵ϛυνηʹ, ἰνδικτιῶνος ηʹ. 



Pseudo-Symeon Hist., Chronographia (partim edita e cod. Paris. gr. 1712) 
Page 731, line 15

                            καὶ τῇ αὐτῇ ἡμέρᾳ τοῦ αὐτοῦ μηνὸς τῆς 
εʹ ἰνδικτιῶνος Χριστοφόρος ὁ υἱὸς Ῥωμανοῦ διαγορεύεται βασι-
λεύς· στέφεται δὲ τῇ πεντηκοστῇ παρὰ τοῦ βασιλέως Κων-
σταντίνου, καὶ προέρχονται οὗτοι οἱ δύο μόνοι ἐν τῇ προε-
λεύσει. 



Pseudo-Symeon Hist., Chronographia (partim edita e cod. Paris. gr. 1712) 
Page 731, line 19

18. Τῷ δὲ Ἰουλίῳ μηνὶ τῆς αὐτῆς ἰνδικτιῶνος γίνεται 
ἕνωσις τῶν ἀποσχιστῶν. 



Pseudo-Symeon Hist., Chronographia (partim edita e cod. Paris. gr. 1712) 
Page 733, line 16

24. Τῷ δὲ Φεβρουαρίῳ μηνί, τῆς ιʹ ἰνδικτιῶνος, τῇ κʹ 
τοῦ μηνός, Θεοδώρα ἡ σύμβιος Ῥωμανοῦ τελευτᾷ, καὶ κατατί-
θεται εἰς τὸν οἶκον τοῦ βασιλέως Ῥωμανοῦ τὸν ὑπ' αὐτοῦ εἰς 
μοναστήριον γενόμενον· καὶ στέφεται Σοφία ἡ τοῦ βασιλέως 
Χριστοφόρου γυνή. 



Pseudo-Symeon Hist., Chronographia (partim edita e cod. Paris. gr. 1712) 
Page 735, line 14

29. Τῷ δὲ Σεπτεμβρίῳ μηνί, τῆς βʹ ἰνδικτιῶνος, πάλιν 
Συμεὼν ὁ Βούλγαρος πανστρατὶ κατὰ Κωνσταντινουπόλεως ἐκ-
στρατεύει, καὶ ληΐζεται μὲν Θρᾴκην καὶ Μακεδονίαν, ἐμπυρίζει 
δὲ πάντα καὶ καταστρέφει καὶ δενδροτομεῖ, μέχρι τῶν Βλαχερνῶν 
παραγενόμενος, καὶ δὴ ἐπιζητῶν ἀποσταλῆναι αὐτῷ τὸν πατριάρ-
χην Νικόλαον καί τινας τῶν μεγιστάνων, ὥστε περὶ εἰρήνης αὐτοῖς 
συντυχεῖν. 



Pseudo-Symeon Hist., Chronographia (partim edita e cod. Paris. gr. 1712) 
Page 739, line 13

32. Τῇ δὲ ιεʹ τοῦ Μαΐου μηνός, ἰνδικτιῶνος γʹ, τελευτᾷ 
ὁ πατριάρχης Νικόλαος, κρατήσας ἐν τῇ δευτέρᾳ αὐτοῦ ἀναβάσει 
τοῦ πατριαρχείου ἔτη ιγʹ, καὶ θάπτεται ἐν τῇ μονῇ αὐτοῦ τῶν 
Γαλακρηνῶν. 



Pseudo-Symeon Hist., Chronographia (partim edita e cod. Paris. gr. 1712) 
Page 740, line 4

33. Μηνὶ Μαΐῳ κζʹ, ἰνδικτιῶνος ιεʹ, Ἰωάννης ὁ ἀστρο-
νόμος εἶπεν Ῥωμανῷ τῷ βασιλεῖ ὅτι ἡ ἱσταμένη στήλη ἐπάνω τῆς 
καμάρας τοῦ Ξηρολόφου καὶ ἐπὶ δυσμὰς βλέπουσα τοῦ Συμεών 
ἐστι· καὶ εἰ ταύτης τὴν κεφαλὴν ἐκκόψεις, τῇ αὐτῇ ὥρᾳ ὁ Συ-
μεὼν τελευτᾷ. 



Pseudo-Symeon Hist., Chronographia (partim edita e cod. Paris. gr. 1712) 
Page 742, line 13

37. Τῇ ιηʹ τοῦ Ἰουνίου μηνός, τῆς ϛʹ ἰνδικτιῶνος, τελευτᾷ 
Στέφανος πατριάρχης, κρατήσας ἔτη βʹ μῆνας ιαʹ. 



Pseudo-Symeon Hist., Chronographia (partim edita e cod. Paris. gr. 1712) 
Page 744, line 21

41. Τῷ δὲ Αὐγούστῳ μηνί, τῆς δʹ ἰνδικτιῶνος, τελευτᾷ 
Χριστοφόρος ὁ βασιλεύς, καὶ ἐτάφη ἐν τῷ οἴκῳ τοῦ πατρὸς αὐ-
τοῦ. 



Pseudo-Symeon Hist., Chronographia (partim edita e cod. Paris. gr. 1712) 
Page 745, line 15

43. Τῷ δὲ Φεβρουαρίῳ μηνί, τῆς ϛʹ ἰνδικτιῶνος, τῇ 
ἑορτῇ τῆς ὑπαπαντῆς, χειροτονεῖται πατριάρχης ὁ ῥηθεὶς τοῦ 
βασιλέως υἱὸς Θεοφύλακτος, τοποτηρητῶν ἐκ Ῥώμης ἀνελθόντων 
καὶ τόμον συνοδικὸν ἐπιφερομένων περὶ τῆς αὐτοῦ χειροτονίας 
διαγορεύοντα· οἳ καὶ τῷ πατριαρχικῷ θρόνῳ τοῦτον ἐνίδρυσαν. 



Pseudo-Symeon Hist., Chronographia (partim edita e cod. Paris. gr. 1712) 
Page 746, line 1

44. Μηνὶ Ἀπριλλίῳ, ἰνδικτιῶνος ζʹ, γέγονε πρώτη ἐκ-
στρατεία τῶν Τούρκων κατὰ Ῥωμανίας· καὶ κατατρέχουσι μέχρι 
τῆς πόλεως, καὶ λεηλατοῦσι πᾶσαν Θρᾳκῴαν ψυχήν. 



Pseudo-Symeon Hist., Chronographia (partim edita e cod. Paris. gr. 1712) 
Page 748, line 1

47. Τῷ δὲ Ἀπριλλίῳ μηνί, τῆς αʹ ἰνδικτιῶνος, πάλιν 
ἦλθον οἱ Τοῦρκοι μετὰ πλείστης δυνάμεως· ὁ δὲ παρακοιμώμενος 
ἐξελθὼν σπονδὰς εἰρηνικὰς μετ' αὐτῶν ἐποιήσατο, ὅθεν καὶ ἐπὶ 
χρόνους εʹ ἡ εἰρήνη διεφυλάχθη. 



Pseudo-Symeon Hist., Chronographia (partim edita e cod. Paris. gr. 1712) 
Page 748, line 5

48. Τῇ δὲ βʹ ἰνδικτιῶνι, Πασχάλιος πρωτοσπαθάριος καὶ 
στρατηγὸς Λαγουβαρδίας ἀποστέλλεται πρὸς ῥῆγα τῶν Φράγγων, 
ἀγαγεῖν τὴν αὐτοῦ θυγατέρα Ῥωμανῷ τῷ Κωνσταντίνου τοῦ γαμ-
βροῦ αὐτοῦ υἱῷ εἰς νύμφην. 



Pseudo-Symeon Hist., Chronographia (partim edita e cod. Paris. gr. 1712) 
Page 748, line 9

                                   ἣν καὶ μετὰ πλούτου πολλοῦ ὁ Πα-
σχάλιος ἀνήγαγεν· καὶ γέγονεν ὁ γάμος Σεπτεμβρίῳ μηνί, ἰνδι-
κτιῶνος γʹ. 



Pseudo-Symeon Hist., Chronographia (partim edita e cod. Paris. gr. 1712) 
Page 748, line 13

49. Τῷ Δεκεμβρίῳ μηνί, τῆς δʹ ἰνδικτιῶνος, ἀνέμου 
βιαίου καὶ σφοδροῦ πνεύσαντος οἱ λεγόμενοι ἐν τῷ ἱππικῷ ἐξ ἐναν-
τίας τοῦ βασιλικοῦ θρόνου Δῆμοι κατέπεσον, καὶ συνέτριψαν τὰ 
ὑποκάτω αὐτῶν βάθρα καὶ στήθεα. 



Pseudo-Symeon Hist., Chronographia (partim edita e cod. Paris. gr. 1712) 
Page 753, line 2

Ὑπελείφθη οὖν αὐτοκράτωρ Κωνσταντῖνος ὁ τούτου γαμβρός, 
μηνὶ Δεκεμβρίῳ κʹ, ἰνδικτιῶνος γʹ, ἐν ἔτει ͵ϛυνδʹ. 



Pseudo-Symeon Hist., Chronographia (partim edita e cod. Paris. gr. 1712) 
Page 754, line 10

3. Τῷ Δεκεμβρίῳ μηνί, τῆς ϛʹ ἰνδικτιῶνος, ἐπιβουλῆς 
γενομένης εἰς τὸ τὸν Στέφανον ἐκ τῆς νήσου ἀγαγεῖν ἐν τῷ παλα-
τίῳ, ἐπεὶ παρὰ Μιχαὴλ τοῦ Διαβολίνου ἐμηνύθη, πάντων κρα-
τηθέντων τῶν μὲν τὰς ῥῖνας καὶ τὰ ὦτα ἀπέτεμε, τοὺς δὲ δαρμῷ 
ἀφορήτῳ ὑπέβαλεν καὶ ὄνοις ἐπικαθίσας καὶ θριαμβεύσας ἐξ-
ώρισεν. 



Pseudo-Symeon Hist., Chronographia (partim edita e cod. Paris. gr. 1712) 
Page 754, line 16

4. Τῇ δὲ κεʹ τοῦ Ἰουλίου μηνός, τῆς ϛʹ ἰνδικτιῶνος, Ῥω-
μανὸς ὁ βασιλεὺς ἐν τῇ νήσῳ τελευτᾷ, καὶ τὸ σῶμα αὐτοῦ ἐν τῇ 
πόλει διακομισθὲν ἐν τῇ αὐτοῦ ἀπετέθη μονῇ. 



Pseudo-Symeon Hist., Chronographia (partim edita e cod. Paris. gr. 1712) 
Page 756, line 17

                                                         ἔζησε δὲ 
χρόνους νεʹ, καὶ ἐτελεύτησε μηνὶ Νοεμβρίῳ ιεʹ, ἰνδικτιῶνος ϛʹ, 
ἔτους ͵ϛυνϛʹ, καταλείψας αὐτοκράτορα Ῥωμανὸν τὸν υἱὸν αὐτοῦ 
μεθ' Ἑλένης τῆς μητρὸς αὐτοῦ. 



Joannes Antiochenus Hist., Fragmenta (4394: 001)
“FHG 4”, Ed. Müller, K.
Paris: Didot, 1841–1870.
Fragment 24, line 16

3. Δίκτυς ὁ μετὰ Ἰδομενέως συστρατεύσας ἐπὶ 
Τροίαν φησὶν ὅτι Πρίαμος ἔπεμψε καὶ πρὸς τὸν Δαυὶδ 
πρεσβείαν, καὶ πρὸς Ταυτάνην βασιλέα Ἀσσυρίων· 
καὶ ὁ μὲν Δαυὶδ οὐ προσήκατο ταύτην, ὁ δὲ Ταυτάνης 
ἔπεμψε τὸν Τιθωνὸν καὶ τὸν Μέμνονα μετὰ πλήθους 
Ἰνδῶν. 



Joannes Antiochenus Hist., Fragmenta 
Fragment 41c, line 1

Suidas l. l: Ὕστερον δὲ εἰς Ἰνδίαν ἀφικόμενος ὑπὸ 
Κανδάκης τῆς βασιλίσσης συνελήφθη ἐν ἰδιώτου σχή-
ματι. 



Joannes Antiochenus Hist., Fragmenta 
Fragment 206, line 25

      – Ὁ δὲ τοῦ βασιλέως γαμβρὸς Ζήνων, τὴν 
ὕπατον ἔχων ἀρχὴν, ἔστελλε τοὺς τὸν Ἰνδακὸν ἀποστή-
σοντας ἀπὸ τοῦ λεγομένου Παπιρίου λόφου. 



Joannes Antiochenus Hist., Fragmenta 
Fragment 206, line 28

                                                Τοῦτον 
γὰρ πρῶτος Νέων ἐφώλευε· μεθ' ὃν Παπίριος καὶ 
ὁ τοῦδε παῖς Ἰνδακὸς, τοὺς προσοίκους ἅπαντας βια-
ζόμενοι, καὶ τοὺς διοδεύοντας ἀναιροῦντες. 



Joannes Antiochenus Hist., Fragmenta 
Fragment 214, line 71

                                          Ὁ δὲ Ἰλλοῦς, τὴν 
τοῦ φρουρίου φυλακὴν ἐπιτρέψας Ἰνδακῷ Κοττούνῃ, τὸ 
λοιπὸν ἐσχόλαζεν ἐν ἀναγνώσει βιβλίων. 



Cosmas Indicopleustes Geogr., Topographia Christiana (4061: 002)
“Cosmas Indicopleustès. Topographie chrétienne, 3 vols.”, Ed. Wolska–Conus, W.
Paris: Cerf, 1:1968; 2:1970; 3:1973; Sources chrétiennes 141, 159, 197.
Book pinax, section 3, line 33

Ἔτι ἔξωθεν τῆς βίβλου· Λόγος ιαʹ 
 Καταγραφὴ ζῴων ἰνδικῶν καὶ περὶ αὐτῶν διήγησις· ἔτι καὶ 
περὶ δένδρων καὶ τῆς Ταπροβάνης. 



Cosmas Indicopleustes Geogr., Topographia Christiana 
Book 2, section 27, line 15

                                                              Υἱοὶ δὲ 
Σήμ, Ἐλὰμ καὶ Ἀσούρ», τουτέστιν Ἐλαμίτας καὶ Ἀσσυ-
ρίους καὶ τὰ λοιπὰ ἔθνη καὶ ὅσα ἐξ αὐτῶν ἐπεκτάθησαν 
ἕως Ἀσίας καὶ ἐπὶ ἀνατολήν, Περσῶν, Οὔννων, Βάκτρων, 
Ἰνδῶν ἕως τοῦ Ὠκεανοῦ. 



Cosmas Indicopleustes Geogr., Topographia Christiana 
Book 2, section 29, line 9

                                                    Ἴσασι δὲ τὸ λεγόμε-
νον Ζίγγιον οἱ τὴν Ἰνδικὴν θάλατταν διαπερῶντες, περαιτέρω 
τυγχάνον τῆς λιβανωτοφόρου γῆς τῆς καλουμένης Βαρβαρίας, 
ἣν καὶ κυκλοῖ ὁ Ὠκεανὸς εἰσβάλλων ἐκεῖθεν εἰς ἀμφοτέρους 
τοὺς κόλπους. 



Cosmas Indicopleustes Geogr., Topographia Christiana 
Book 2, section 30, line 1

Ἐν οἷς ποτε πλεύσαντες ἐπὶ τὴν ἐσωτέραν Ἰνδίαν καὶ 
ὑπερβάντες μικρῷ πρὸς τὴν Βαρβαρίαν, ἔνθα περαιτέρω τὸ 
Ζίγγιον τυγχάνει – οὕτω γὰρ καλοῦσι τὸ στόμα τοῦ Ὠκεα-
νοῦ – , ἐκεῖ ἐθεωροῦμεν εἰς τὰ δεξιὰ εἰσερχομένων ἡμῶν πλῆ-
θος πετεινῶν πετομένων, ἃ καλοῦσι σοῦσφα· εἰσὶ δὲ ὡς διπλοῖ 
ἰκτῖνες, καὶ μείζους μικρόν· καὶ δυσαερίαν πολλὴν ἐν τῷ τόπῳ, 
ὡς καὶ δειλιᾶν πάντας. 



Cosmas Indicopleustes Geogr., Topographia Christiana 
Book 2, section 45, line 7

                                                Αὕτη δὲ ἡ 
χώρα τοῦ μεταξίου ἐστὶν ἐν τῇ ἐσωτέρᾳ πάντων Ἰνδίᾳ, κατὰ 
τὸ ἀριστερὸν μέρος εἰσιόντων τοῦ Ἰνδικοῦ πελάγους, περαι-
τέρω πολὺ τοῦ Περσικοῦ κόλπου καὶ τῆς νήσου τῆς καλου-
μένης παρὰ μὲν Ἰνδοῖς, Σελεδίβα, παρὰ δὲ τοῖς Ἕλλησι, 
Ταπροβάνη, Τζίνιστα οὕτω καλουμένη, κυκλουμένη πάλιν ἐξ 
ἀριστερῶν ὑπὸ τοῦ Ὠκεανοῦ, ὥσπερ καὶ ἡ Βαρβαρία κυκλοῦ-
ται ἐκ δεξιῶν ὑπ' αὐτοῦ. 



Cosmas Indicopleustes Geogr., Topographia Christiana 
Book 2, section 45, line 13

                              Καί φασιν οἱ Ἰνδοί, οἱ φιλόσοφοι, οἱ 
καλούμενοι Βραχμάνες, ὅτι ἐὰν βάλῃς ἀπὸ Τζίνιστα σπαρτίον 
διελθεῖν διὰ Περσίδος ἕως Ῥωμανίας ὡς ἀπὸ κανόνος τὸ 
μεσαίτατον τοῦ κόσμου ἐστί, καὶ τάχα ἀληθεύουσιν. 



Cosmas Indicopleustes Geogr., Topographia Christiana 
Book 2, section 46, line 9

                                                     Ὅσον γὰρ 
διάστημα ἔχει ὁ κόλπος ὁ Περσικὸς εἰσερχόμενος ἐν Περσίδι, 
τοσοῦτο διάστημα πάλιν ἀπὸ τῆς Ταπροβάνης καὶ περαιτέρω 
ποιεῖ ἐπὶ τὰ ἀριστερὰ εἰσερχόμενός τις ἐν αὐτῇ τῇ Τζίνιστα, 
μετὰ τὸ καὶ διαστήματα πάλιν ἱκανὰ ἔχειν ἀπὸ τῆς ἀρχῆς ἐξ 
αὐτοῦ τοῦ Περσικοῦ κόλπου, ὅλον τὸ Ἰνδικὸν πέλαγος ἕως   
Ταπροβάνης καὶ ἐπέκεινα. 



Cosmas Indicopleustes Geogr., Topographia Christiana 
Book 2, section 47, line 5

                                                       Μετρητέον δὲ 
οὕτως· ἀπὸ τῆς Τζίνιστα ἕως τῆς ἀρχῆς τῆς Περσίδος πᾶσα ἡ 
Οὔννια καὶ Ἰνδία καὶ ἡ Βάκτρων χώρα εἰσὶ περί που μοναὶ ρνʹ, 
εἰ μή τι πλείους, οὐκ ἔλαττον· καὶ πᾶσα ἡ Περσῶν χώρα μο-
ναὶ πʹ· καὶ ἀπὸ τοῦ Νίσιβι εἰς Σελεύκειαν μοναὶ ιγʹ· καὶ ἀπὸ 
Σελευκείας εἰς Ῥώμην καὶ Γάλλους καὶ Ἰβηρίαν, τοὺς νῦν 
λεγομένους Ἱσπανούς, ἕως Γαδείρων ἔξω εἰς τὸν Ὠκεανόν, 
μοναὶ ρνʹ καὶ πλέον, ὡς γίνεσθαι τὸ πᾶν μοναὶ υʹ πλέον 
ἔλαττον. 



Cosmas Indicopleustes Geogr., Topographia Christiana 
Book 2, section 49, line 8

Ἔστι δὲ ἡ χώρα ἡ λιβανωτοφόρος εἰς τὰ ἄκρα τῆς 
Αἰθιοπίας, μεσόγειος μὲν οὖσα, τὸν δὲ Ὠκεανὸν ἐπέκεινα   
ἔχουσα, ὅθεν καὶ οἱ τὴν Βαρβαρίαν οἰκοῦντες, ὡς ἐγγύθεν 
ὄντες, ἀνερχόμενοι εἰς τὰ μεσόγεια καὶ πραγματευόμενοι 
κομίζουσιν ἐξ αὐτῶν τὰ πλεῖστα τῶν ἡδυσμάτων, λίβανον, 
κασίαν, κάλαμον καὶ ἕτερα πολλά, καὶ αὐτὰ πάλιν διὰ θαλάς-
σης κομίζουσιν ἐν τῇ Ἀδούλῃ καὶ ἐν τῷ Ὁμηρίτῃ καὶ ἐν τῇ 
ἐσωτέρᾳ Ἰνδίᾳ καὶ ἐν τῇ Περσίδι. 



Cosmas Indicopleustes Geogr., Topographia Christiana 
Book 2, section 59, line 5

   Κυριεύσας δὲ τῆς τε 
ἐντὸς Εὐφράτου χώρας πάσης, καὶ Κιλικίας καὶ Παμφυλίας 
καὶ Ἰωνίας καὶ τοῦ Ἑλλησπόντου καὶ Θρᾴκης καὶ τῶν 
δυνάμεων τῶν ἐν ταῖς χώραις ταύταις πασῶν καὶ ἐλεφάντων 
ἰνδικῶν, καὶ τοὺς μονάρχους τοὺς ἐν τοῖς τόποις πάντας 
ὑπηκόους καταστήσας, διέβη τὸν Εὐφράτην ποταμὸν καί, 
τὴν Μεσοποταμίαν καὶ Βαβυλωνίαν καὶ Σουσιάνην καὶ 
Περσίδα καὶ Μηδείαν καὶ τὴν λοιπὴν πᾶσαν ἕως Βακτριανῆς 
ὑφ' ἑαυτῷ ποιησάμενος καὶ ἀναζητήσας ὅσα ὑπὸ τῶν Περσῶν 
ἱερὰ ἐξ Αἰγύπτου ἐξήχθη καὶ ἀνακομίσας μετὰ τῆς ἄλλης 
γάζης τῆς ἀπὸ τῶν τόπων εἰς Αἴγυπτον, δυνάμεις ἀπέστειλε 
διὰ τῶν ὀρυχθέντων ποταμῶν . 



Cosmas Indicopleustes Geogr., Topographia Christiana 
Book 2, section 79, line 2

Ἐφόρου ἐκ τῆς δʹ Ἱστορίας


 Τὸν μὲν γὰρ Ἀπηλιώτην καὶ τὸν ἐγγὺς ἀνατολῶν 
τόπον Ἰνδοὶ κατοικοῦσι, τὸν δὲ πρὸς Νότον καὶ μεσημβρίαν 
Αἰθίοπες νέμονται, τὸν δὲ ἀπὸ Ζεφύρου καὶ δυσμῶν Κελτοὶ 
κατέχουσι, τὸν δὲ κατὰ Βορρᾶν καὶ τὰς ἄρκτους Σκῦθαι 
κατοικοῦσιν. 



Cosmas Indicopleustes Geogr., Topographia Christiana 
Book 2, section 79, line 7

              Ἔστι μὲν οὖν οὐκ ἴσον ἕκαστον τούτων τῶν 
μερῶν, ἀλλὰ τὸ μὲν τῶν Σκυθῶν καὶ τῶν Αἰθιόπων μεῖζον, τὸ 
δὲ τῶν Ἰνδῶν καὶ τῶν Κελτῶν ἔλαττον. 



Cosmas Indicopleustes Geogr., Topographia Christiana 
Book 2, section 79, line 9

                                                       Οἱ μὲν γὰρ 
Ἰνδοί εἰσι μεταξὺ θερινῶν καὶ χειμερινῶν ἀνατολῶν, Κελτοὶ 
δὲ τὴν ἀπὸ θερινῶν μέχρι χειμερινῶν δυσμῶν χώραν κατέ-
χουσι. 



Cosmas Indicopleustes Geogr., Topographia Christiana 
Book 2, section 81, line 4

                                 Ὁ μὲν Φεισὼν ἐν τῇ Ἰνδικῇ χώρᾳ,   
ὃν καλοῦσί τινες Ἰνδὸν ἢ Γάγγην, ἐκ τῶν μεσογείων που 
κατερχόμενος πολλὰς ἐκροίας ἔχει ἐν τῷ Ἰνδικῷ πελάγει. 



Cosmas Indicopleustes Geogr., Topographia Christiana 
Book 3, section 65, line 1

Ἐν Ταπροβάνῃ νήσῳ ἐν τῇ ἐσωτέρᾳ Ἰνδίᾳ, ἔνθα τὸ 
Ἰνδικὸν πέλαγός ἐστι, καὶ Ἐκκλησία χριστιανῶν ἐστιν ἐκεῖ 
καὶ κληρικοὶ καὶ πιστοί, οὐκ οἶδα δὲ εἰ καὶ περαιτέρω. 



Cosmas Indicopleustes Geogr., Topographia Christiana 
Book 3, section 65, line 7

                                   Ὁμοίως καὶ ἐν τῇ νήσῳ τῇ 
καλουμένῃ Διοσκορίδους κατὰ τὸ αὐτὸ Ἰνδικὸν πέλαγος, ἔνθα 
καὶ οἱ παροικοῦντες ἑλληνιστὶ λαλοῦσι, πάροικοι τῶν Πτολε-
μαίων τῶν μετὰ Ἀλέξανδρον τὸν Μακεδόνα ὑπάρχοντες, καὶ   
κληρικοί εἰσιν ἐκ Περσίδος χειροτονούμενοι καὶ πεμπόμενοι 
ἐν τοῖς αὐτόθι καὶ χριστιανοὶ πλῆθος· ἣν νῆσον παρέπλευσα 
μέν, οὐ κατῆλθον δὲ ἐν αὐτῇ· συνέτυχον δὲ ἀνδράσι τῶν ἐκεῖ 
ἑλληνιστὶ λαλοῦσιν, ἐλθοῦσιν ἐν τῇ Αἰθιοπίᾳ. 



Cosmas Indicopleustes Geogr., Topographia Christiana 
Book 3, section 65, line 14

                                                         Ὁμοίως δὲ καὶ 
ἐπὶ Βάκτροις καὶ Οὔννοις καὶ Πέρσαις καὶ λοιποῖς Ἰνδοῖς καὶ 
Περσαρμενίοις καὶ Μήδοις καὶ Ἐλαμίταις καὶ πάσῃ τῇ χώρᾳ 
Περσίδος καὶ ἐκκλησίαι ἄπειροι καὶ ἐπίσκοποι καὶ χριστιανοὶ 
λαοὶ πάμπολλοι καὶ μάρτυρες πολλοὶ καὶ μονάζοντες ἡσυ-
χασταί. 



Cosmas Indicopleustes Geogr., Topographia Christiana 
Book 6, section 3, line 8

                                           Ὅθεν ἀπαιτηθεὶς τῷ 
Θὼθ μηνὶ τῆς παρούσης δεκάτης ἰνδικτιῶνος παρὰ ἀνδρὸς 
ἐπιστήμονος, Ἀναστασίου τοὔνομα, μηχανικοῦ ἀνδρὸς λογίου 
καὶ ὑπὲρ πολλοὺς ἐμπείρου, προειπεῖν ἔκλειψιν ἡλίου, ἔφη   
γενέσθαι ἐν αὐτῷ τῷ καιρῷ κατὰ τὴν δωδεκάτην τοῦ Μεχεὶρ 
μηνὸς καὶ σεληνιακὴν Μεσορὴ εἰκοστῇ τετάρτῃ πάλιν τῷ αὐτῷ 
καιρῷ. 



Cosmas Indicopleustes Geogr., Topographia Christiana 
Book 10, section 68, line 11t

Τοῦ αὐτοῦ 
εἰς τὰ Ἅγια Θεοφάνια, 
εἰς τὴν γενέθλιον τοῦ Χριστοῦ ἡμέραν, 
Χοιὰκ λʹ, ἰνδ. ιʹ


 Ἔτεκεν ἡ Παρθένος ἄνθρωπον τέλειον, ἀναμάρτητον. 



Cosmas Indicopleustes Geogr., Topographia Christiana 
Book 10, section 69, line 6t

Καὶ πάλιν, 
ἐν τῷ Ἁγίῳ Θεοδώρῳ, 
Τυβὶ ηʹ, ἰνδ. ιʹ


 Διὰ γὰρ τοῦ φαινομένου ἔδειξε τὴν δύναμιν τοῦ κρυπτο-
μένου. 



Cosmas Indicopleustes Geogr., Topographia Christiana 
Book 10, section 69, line 11t

Τοῦ αὐτοῦ 
ἐν τῇ Κυρίνου ἐκκλησίᾳ, Κυριακῆς οὔσης, 
Παχῶν κβʹ, ἰνδ. εʹ, 
ἀναγνωσθέντος τοῦ κατὰ Ἰωάννην Εὐαγγελίου 
»ὁ δὲ Ἰησοῦς κεκοπιακὼς ἐκ τῆς ὁδοιπορίας ἐκαθέζετο»


 Τοιγαροῦν ἐπειδὴ Θεός ἐστιν ἅμα καὶ ἄνθρωπος ὁ αὐτός, 
πιστοῦται διὰ τῶν ἔργων ἀμφότερα καὶ λανθάνειν τοὺς ὁρῶν-
τας οὐκ ἀνέχεται. 



Cosmas Indicopleustes Geogr., Topographia Christiana 
Book 10, section 72, line 13t

Τοῦ αὐτοῦ 
ἐν τῇ τεσσαρακοστῇ τῆς ἀναλήψεως τοῦ Κυρίου ἡμέρᾳ, 
Παχῶν κεʹ, ἰνδ. θʹ, 
εἰς τὸ ῥητὸν τοῦ κατὰ Ἰωάννην Εὐαγγελίου τὸ λέγον 
»συμφέρει ὑμῖν ἵνα ἐγὼ ἀπέλθω»


 Ἀλλ' οἷα καὶ νῦν τὰ παρ' αὐτοῦ πρὸς τοὺς ἀοιδίμους 
μαθητὰς εἰρημένα κατανοήσωμεν· «Συμφέρει ὑμῖν ἵνα ἐγὼ   
ἀπέλθω. 



Cosmas Indicopleustes Geogr., Topographia Christiana 
Book 11, section t, line 2

ΛΟΓΟΣ ΙΑʹ 
 Καταγραφὴ περὶ ζῴων ἰνδικῶν καὶ περὶ δένδρων ἰνδικῶν 
καὶ περὶ τῆς Ταπροβάνης νήσου. 




Cosmas Indicopleustes Geogr., Topographia Christiana 
Book 11, section 3, line 1

Τοῦτο τὸ ζῷον ὁ ταυρέλαφος καὶ ἐν τῇ Ἰνδίᾳ καὶ ἐν τῇ 
Αἰθιοπίᾳ εὑρίσκεται. 



Cosmas Indicopleustes Geogr., Topographia Christiana 
Book 11, section 3, line 2

                           Ἀλλὰ τὰ μὲν τῆς Ἰνδίας ἥμερά εἰσι, 
καὶ ἐν αὐτοῖς ποιοῦσιν ἐν δισακκίοις βασταγὰς πιπέρεως καὶ 
ἑτέρων φορτίων, καὶ γάλα ἀμέλγουσιν ἐξ αὐτῶν καὶ βούτυρον. 



Cosmas Indicopleustes Geogr., Topographia Christiana 
Book 11, section 5, line 1

Ἀγριόβους ἐστὶ μέγας, τῆς Ἰνδικῆς τοῦτο τὸ ζῷον, ἐξ 
οὗ ἐστιν ἡ λεγομένη τοῦφα, ᾗ κοσμοῦσι τοὺς ἵππους καὶ τὰ 
βάνδα οἱ ἄρχοντες εἰς τοὺς κάμπους. 



Cosmas Indicopleustes Geogr., Topographia Christiana 
Book 11, section 11, line 2

Τὸ δὲ ἄλλο τῶν ἀργελλίων ἐστὶ τῶν λεγομένων, 
τουτέστι τῶν μεγάλων καρύων τῶν ἰνδικῶν· παραλλάττει δὲ 
τοῦ φοίνικος οὐδέν, πλὴν ὅτι τελειότερόν ἐστι καὶ ἐν ὕψει καὶ 
ἐν πάχει καὶ ἐν τοῖς βαΐοις. 



Cosmas Indicopleustes Geogr., Topographia Christiana 
Book 11, section 11, line 8

                                    Οὐ βάλλει δὲ καρπόν, εἰ μὴ δύο 
ἢ τρία σπάθια ἀπὸ τριῶν ἀργελλίων· ἔστι δὲ ἡ γεῦσις γλυκεῖα 
πάνυ καὶ ἡδεῖα, ὡς τὰ κάρυα τὰ χλωρά· ἐξ ἀρχῆς μὲν τοῦ 
ὕδατος γέμει γλυκέος πάνυ, ὅθεν καὶ ἐξ αὐτῶν πίνουσιν οἱ 
Ἰνδοὶ ἀντὶ οἴνου· λέγεται δὲ τὸ πινόμενον ῥογχοσοῦρα ἡδὺ 
πάνυ· τρυγώμενον δὲ καὶ παραμένον αὐτὸ τὸ ἄργελλιν, 
πήγνυται τὸ ὕδωρ αὐτοῦ κατὰ πρόσβασιν τὸ ἐπὶ τὸ ὄστρακον 
αὐτοῦ, καὶ μένει τὸ ὕδωρ εἰς τὸ μέσον ἄπηκτον, μέχρις ὅτου 
καὶ αὐτὸ ἐκλίπῃ· ἐὰν δὲ καὶ πλέον παραμείνῃ, ταγγίζει 
ὁ καρπὸς αὐτοῦ ὁ πεπηγὼς καὶ οὐ δύναται ἔτι βρωθῆναι. 



Cosmas Indicopleustes Geogr., Topographia Christiana 
Book 11, section 13, line 2

Περὶ τῆς Ταπροβάνης νήσου


 Αὕτη ἐστὶν ἡ νῆσος ἡ μεγάλη ἐν τῷ Ὠκεανῷ, ἐν τῷ 
Ἰνδικῷ πελάγει κειμένη, παρὰ μὲν Ἰνδοῖς καλουμένη Σιελε-
δίβα, παρὰ δὲ Ἕλλησι Ταπροβάνη, ἐν ᾗ εὑρίσκεται ὁ λίθος 
ὁ ὑάκινθος· περαιτέρω δὲ κεῖται τῆς χώρας τοῦ πιπέρεως. 



Cosmas Indicopleustes Geogr., Topographia Christiana 
Book 11, section 15, line 1

Ἐξ ὅλης δὲ τῆς Ἰνδικῆς καὶ Περσίδος καὶ Αἰθιοπίας 
δέχεται ἡ νῆσος πλοῖα πολλά, μεσῖτις οὖσα, ὁμοίως καὶ 
ἐκπέμπει. 



Cosmas Indicopleustes Geogr., Topographia Christiana 
Book 11, section 16, line 1

Ἡ Σινδοῦ δέ ἐστιν ἀρχὴ τῆς Ἰνδικῆς. 



Cosmas Indicopleustes Geogr., Topographia Christiana 
Book 11, section 16, line 2

                                                 Διαιρεῖ γὰρ 
ὁ Ἰνδὸς ποταμός, τουτέστιν ὁ Φεισών, εἰς τὸν κόλπον τὸν 
Περσικὸν ἔχων τὰς ἐκροίας, τήν τε Περσίδα καὶ τὴν Ἰνδίαν. 



Cosmas Indicopleustes Geogr., Topographia Christiana 
Book 11, section 16, line 4

Εἰσὶν οὖν τὰ λαμπρὰ ἐμπόρια τῆς Ἰνδικῆς ταῦτα, Σινδοῦ, 
Ὀρροθᾶ, Καλλιανᾶ, Σιβώρ, ἡ Μαλέ, πέντε ἐμπόρια ἔχουσα 
βάλλοντα τὸ πέπερι, Πάρτι, Μαγγαρούθ, Σαλοπάτανα, Ναλο-
πάτανα, Πουδαπάτανα. 



Cosmas Indicopleustes Geogr., Topographia Christiana 
Book 11, section 16, line 14

      Αὕτη οὖν ἡ Σιελεδίβα μέση πως τυγχάνουσα τῆς Ἰνδικῆς, 
ἔχουσα δὲ καὶ τὸν ὑάκινθον, ἐξ ὅλων τῶν ἐμπορίων δέχεται καὶ 
ὅλοις μεταβάλλει, καὶ μέγα ἐμπόριον τυγχάνει. 



Cosmas Indicopleustes Geogr., Topographia Christiana 
Book 11, section 20, line 4

                   Ἀνώτεροι δέ, τουτέστι βορειότεροι, τῆς 
Ἰνδικῆς εἰσι λευκοὶ Οὔννοι, ὁ λεγόμενος Γολλᾶς ἐκβάλλων εἰς 
πόλεμον, ὥς φασιν, οὐκ ἔλαττον τῶν δισχιλίων ἐλεφάντων καὶ 
ἵππον πολλήν· κατακρατεῖ δὲ καὶ τῆς Ἰνδικῆς καταδυ-
ναστεύων καὶ φόρους ἀπαιτῶν. 



Cosmas Indicopleustes Geogr., Topographia Christiana 
Book 11, section 20, line 8

                                  Ποτὲ γοῦν, ὥς φασι, βουλό-
μενος πόλιν τῶν Ἰνδῶν μεσόγειον πορθῆσαι, τῆς δὲ πόλεως 
κύκλῳ ὕδατι φρουρουμένης, αὐτὸς ἱκανὰς ἡμέρας περικαθίσας   
καὶ φρουρήσας καὶ ἀναλώσας τὸ ὕδωρ διὰ τῶν ἐλεφάντων καὶ 
ἵππων καὶ τοῦ στρατοπέδου, ὕστερον διὰ ξηρᾶς περάσας τὴν 
πόλιν παρέλαβεν. 



Cosmas Indicopleustes Geogr., Topographia Christiana 
Book 11, section 21, line 4

                             Εἰσφέρουσι γὰρ οἱ Αἰθίοπες συναλ-
λαγὰς ποιοῦντες μετὰ τῶν Βλεμμύων ἐν τῇ Αἰθιοπίᾳ τὸν 
αὐτὸν λίθον ἕως εἰς τὴν Ἰνδίαν· καὶ αὐτοὶ τὰ καλλιστεύοντα 
ἀγοράζουσι. 



Cosmas Indicopleustes Geogr., Topographia Christiana 
Book 11, section 22, line 1

Οἱ δὲ κατὰ τόπον βασιλεῖς τῆς Ἰνδικῆς ἔχουσιν 
ἐλέφαντας, οἷον ὁ τῆς Ὀρροθᾶ καὶ ὁ Καλλιανῶν καὶ ὁ τῆς 
Σινδοῦ καὶ ὁ τῆς Σιβὼρ καὶ ὁ τῆς Μαλέ, ὁ μὲν ἑξακόσια, ὁ δὲ 
πεντακόσια, ἕκαστος πλέον ἢ ἔλαττον. 



Cosmas Indicopleustes Geogr., Topographia Christiana 
Book 11, section 23, line 8

                                           Ὀδόντας δὲ μεγάλους 
οἱ ἰνδικοὶ οὐκ ἔχουσιν, ἀλλὰ καὶ ἐὰν σχῶσι, πρίζουσιν αὐτοὺς 
διὰ τὸ βάρος, ἵνα μὴ βαρῇ αὐτοὺς ἐν τῷ πολέμῳ. 



Cosmas Indicopleustes Geogr., Topographia Christiana 
Book 11, section 23, line 13

                                                              Οἱ δὲ Αἰθίο-
πες οὐκ ἴσασιν ἡμερῶσαι ἐλέφαντας, ἀλλ' εἰ τύχοι θελῆσαι 
τὸν βασιλέα ἕνα ἢ δεύτερον πρὸς θέαν, μικροὺς πιάζουσι καὶ 
ἀνατρέφουσιν· ἔχει γὰρ ἡ χώρα αὐτῶν πλῆθος καὶ μεγάλους 
ὀδόντας ἔχοντας· ἐκ τῆς γὰρ Αἰθιοπίας καὶ εἰς Ἰνδίαν πλωΐ-
ζονται ὀδόντες καὶ ἐν Περσίδι καὶ ἐν τῷ Ὁμηρίτῃ καὶ ἐν τῇ 
Ῥωμανίᾳ. 



Cosmas Indicopleustes Geogr., Topographia Christiana 
Book 11, section 24, line 1

Πᾶσαν δὲ τὴν Ἰνδικὴν καὶ τὴν Οὐννίαν διαιρεῖ ὁ 
Φεισὼν ποταμός. 



Cosmas Indicopleustes Geogr., Topographia Christiana 
Book 11, section 24, line 3

                   Καλεῖται γὰρ παρὰ τῇ θείᾳ Γραφῇ ἡ γῆ τῆς 
Ἰνδικῆς χώρας «Εὐιλάτ»· οὕτως γὰρ γέγραπται ἐν τῇ 
Γενέσει· «Ποταμὸς δὲ ἐκπορεύεται ἐξ Ἐδὲμ ποτίζειν τὸν 
παράδεισον· ἐκεῖθεν ἀφορίζεται εἰς τέσσαρας ἀρχάς· ὄνομα 
τῷ ἑνὶ Φεισών, οὗτος ὁ κυκλῶν πᾶσαν τὴν γῆν Εὐιλάτ· ἐκεῖ 
οὖν ἐστι τὸ χρυσίον, τὸ δὲ χρυσίον τῆς γῆς ἐκείνης καλόν· ἐκεῖ 
ἐστιν ὁ ἄνθραξ καὶ ὁ λίθος ὁ πράσινος», γῆν Εὐιλὰτ σαφέστε-
ρον αὐτὴν ὀνομάσας. 



Cosmas Indicopleustes Geogr., Topographia Christiana 
Book 11, section 24, line 12

                        Οὗτος δὲ Εὐιλὰτ ἐκ τοῦ Χάμ ἐστιν· 
οὕτω γὰρ πάλιν γέγραπται· «Υἱοὶ Χάμ, Χοὺς καὶ Μεσραείμ, 
Φοὺδ καὶ Χαναάν· υἱοὶ δὲ Χούς, Σαβᾶ καὶ Εὐιλάτ», του-
τέστιν Ὁμηρῖται καὶ Ἰνδοί· ἡ Σαβᾶ γὰρ ἐν τῷ Ὁμηρίτῃ 
κεῖται, καὶ Εὐιλὰτ ἐν τῇ Ἰνδίᾳ ἐστί· τὰς δύο γὰρ ταύτας 
χώρας ὁ Περσικὸς κόλπος διαιρεῖ. 



Cyrillus Biogr., Vita Euthymii (2877: 001)
“Kyrillos von Skythopolis”, Ed. Schwartz, E.
Leipzig: Hinrichs, 1939; Texte und Untersuchungen 49.2.
Page 26, line 22

                καὶ κατελθὼν Ἰουβενάλιος ὁ ἀρχιεπίσκοπος εἰς τὴν 
λαύραν ἔχων μεθ' ἑαυτοῦ Πασσαρίωνα τὸν ἐν ἁγίοις, ὄντα τὸ τηνικαῦτα 
χωρεπίσκοπον καὶ τῶν μοναχῶν ἀρχιμανδρίτην καὶ τὸν πεφωτισμένον 
Ἡσύχιον τὸν πρεσβύτερον καὶ τῆς ἐκκλησίας διδάσκαλον ἐγκαινίζει 
τὴν τῆς λαύρας ἐκκλησίαν μηνὶ Μαίωι ἑβδόμηι τῆς ἑνδεκάτης 
ἰνδικτιόνος κατὰ τὸν πεντηκοστὸν δεύτερον τῆς τοῦ μεγάλου Εὐθυ-
μίου ἡλικίας χρόνον. 



Cyrillus Biogr., Vita Euthymii 
Page 54, line 10

τεσσάρων δὲ μηνῶν πληρωθέντων μετὰ τὸν ἐγκαινισμὸν εὐσεβῶς 
καὶ θεαρέστως διαθεμένη εἰς χεῖρας θεοῦ τὸ πνεῦμα παρέθετο μηνὶ 
Ὀκτωβρίωι εἰκάδι τῆς τεσσαρεσκαιδεκάτης ἰνδικτιόνος. 



Cyrillus Biogr., Vita Euthymii 
Page 54, line 13

Τῶι ἐνενηκοστῶι τῆς τοῦ μεγάλου Εὐθυμίου ἡλικίας χρόνωι ὁ 
μέγας πατὴρ ἡμῶν Θεόκτιστος ἠσθένησεν ἀσθένειαν μεγάλην, εἰς 
ἣν καὶ ἐκοιμήθη μηνὶ Σεπτεμβρίωι τρίτηι ἀρχῆι τῆς πέμπτης ἰνδικ-
τιόνος <πρεσβύτης καὶ πλήρης ἡμερῶν>. 



Cyrillus Biogr., Vita Euthymii 
Page 60, line 1

οὔτε γὰρ οἱ ὀφθαλμοὶ αὐτοῦ ἢ οἱ ὀδόντες τὸ παράπαν ἐβλάβησαν, 
ἀλλὰ στερρός τε καὶ πρόθυμος ὢν ἐτελειώθη, ἡ δὲ τελείωσις αὐτοῦ 
γέγονεν κατὰ τὴν εἰκάδα τοῦ Ἰαννουαρίου μηνὸς τῆς ἑνδεκάτης   
ἰνδικτιόνος ἀπὸ μὲν κτίσεως κόσμου, ἀφ' οὗπερ χρόνος ἤρξατο τῆι 
τοῦ ἡλίου φορᾶι μετρεῖσθαι, ἔτους πέμπτου ἑξηκοστοῦ ἐνακοσιοστοῦ 
πεντακισχιλιοστοῦ, ἀπὸ δὲ τῆς τοῦ θεοῦ λόγου ἐκ παρθένου ἐνανθρω-
πήσεως καὶ κατὰ σάρκα γεννήσεως ἔτους πέμπτου ἑξηκοστοῦ τετρα-
κοσιοστοῦ κατὰ τοὺς συγγραφέντας χρόνους ὑπὸ τῶν ἁγίων πατέρων 
Ἱππολύτου τε τοῦ παλαιοῦ καὶ γνωρίμου τῶν ἀποστόλων καὶ Ἐπι-
φανίου τοῦ Κυπριώτου καὶ Ἥρωνος τοῦ φιλοσόφου καὶ ὁμολογητοῦ. 



Cyrillus Biogr., Vita Euthymii 
Page 70, line 22

                             ὀκτὼ τοίνυν χρόνους τὴν Εὐθυμίου τοῦ 
μεγάλου κυβερνήσας μονὴν ὁ Θωμᾶς ἐτελεύτησεν μηνὶ Μαρτίωι εἰκάδι 
πέμπτηι τῆς πέμπτης ἰνδικτιόνος ἔτους ἑβδομηκοστοῦ τῆς τοῦ με-
γάλου Εὐθυμίου κοιμήσεως· Λεόντιος δὲ τὴν ἡγουμενίαν παρέ-
λαβεν ὁ ἐμὲ τὸν ἁμαρτωλὸν εἰς τὴν ἐμοὶ σεβασμίαν μονὴν εἰσδεξά-
μενος. 



Cyrillus Biogr., Vita Sabae (2877: 002)
“Kyrillos von Skythopolis”, Ed. Schwartz, E.
Leipzig: Hinrichs, 1939; Texte und Untersuchungen 49.2.
Page 93, line 14

Αὐτοῦ δὲ ἐν αὐτῶι τῶι κοινοβίωι ἤδη τὸν δέκατον διατελοῦντος 
ἐνιαυτὸν συνέβη τελευτῆσαι τὸν τρισμακάριον Θεόκτιστον τρίτηι 
τοῦ Σεπτεμβρίου μηνὸς τῆς τετάρτης ἰνδικτιόνος, ὁ δὲ μέγας Εὐθύ-
μιος ἐκεῖσε κατελθὼν καὶ τὸ νικηφόρον κηδεύσας σῶμα Μάριν τινὰ 
ἀξιοθαύμαστον διάδοχον τῆς ἡγουμενίας κατέστησεν. 



Cyrillus Biogr., Vita Sabae 
Page 103, line 10

Τοῦ δὲ ἐν ἁγίοις Μαρτυρίου τὸν ὄγδοον ἐν τῆι πατριαρχίαι διατε-
λοῦντος ἐνιαυτὸν πρὸς τὸν θεὸν ἐκδημήσαντος μηνὶ Ἀπριλίωι τρισκαι-
δεκάτηι τῆς ἐνάτης ἰνδικτιόνος καὶ Σαλουστίου τὴν ἱεραρχίαν δια-
δεξαμένου τῶι τεσσαρακοστῶι ὀγδόωι τῆς τοῦ πατρὸς ἡμῶν Σάβα 
ἡλικίας χρόνωι ἀνεφύησαν τινὲς ἐν τῆι αὐτοῦ λαύραι σαρκικοὶ τῶι 
φρονήματι καὶ κατὰ τὴν γραφὴν <πνεῦμα μὴ ἔχοντες>· οἵτινες 
ἐπὶ χρόνον ἱκανὸν συσκευὴν ποιησάμενοι ἔθλιβον αὐτὸν παντοίως. 



Cyrillus Biogr., Vita Sabae 
Page 104, line 24

                                                              καὶ ταῦτα 
εἰπὼν λαβὼν τόν τε μακαρίτην Σάβαν καὶ αὐτοὺς ἐκείνους κατῆλθεν 
εἰς τὴν λαύραν ἔχων μεθ' ἑαυτοῦ τὸν μνημονευθέντα σταυροφύλακα 
Κυρικὸν καὶ τὴν θεόκτιστον ἐκκλησίαν ἐγκαινίσας θυσιαστήριον 
ἡγιασμένον ἐν τῆι θεοκτίστωι κόγχηι κατέπηξεν πλεῖστα λείψανα 
ἁγίων καὶ καλλινίκων μαρτύρων ὑπὸ τὸ θυσιαστήριον καταθέμενος 
μηνὶ Δεκεμβρίωι δωδεκάτηι τῆς τεσσαρεσκαιδεκάτης ἰνδικτιόνος 
πεντηκοστῶι τρίτωι τῆς τοῦ μακαρίου Σάβα ἡλικίας χρόνωι, ἐν ὧι   
χρόνωι Ζήνωνος τοῦ βασιλέως τελευτήσαντος Ἀναστάσιος τὴν βασι-
λείαν παρέλαβεν. 



Cyrillus Biogr., Vita Sabae 
Page 110, line 4

Τῶι πεντηκοστῶι τετάρτωι τῆς τοῦ μεγάλου Σάβα ἡλικίας χρόνωι, 
δευτέρωι δὲ ἔτει τοῦ ἐγκαινισμοῦ τῆς θεοκτίστου ἐκκλησίας καὶ τῆς 
τοῦ ἐπισκόπου Ἰωάννου ἐν τῆι λαύραι παρουσίας τῆι εἰκάδι πρώτηι 
τοῦ Ἰαννουαρίου μηνὸς τῆς πεντεκαιδεκάτης ἰνδικτιόνος ἦλθεν ὁ ἐν 
ἁγίοις πατὴρ ἡμῶν Σάβας εἰς τὸν τοῦ Καστελλίου βουνὸν ὡς ἀπὸ 
εἴκοσι σταδίων ὄντα τῆς λαύρας κατὰ τὸ πρὸς ἀνατολὰς ἀρκτῶιον 
μέρος. 



Cyrillus Biogr., Vita Sabae 
Page 112, line 15

οὕστινας ἀδελφοὺς μετὰ τοῦ φόρτου ὁ θεῖος δεξάμενος πρεσβύτης 
τὰς εὐχαριστηρίους φωνὰς τοῦ τε Δαυὶδ καὶ τοῦ Δανιὴλ ἐπὶ τῆι τοῦ 
θεοῦ ἐπισκοπῆι ἐμελέτα προσφόρως καὶ γέγονεν περὶ τὴν τοῦ κοινο-
βίου οἰκοδομὴν προθυμότερος· ὁ μέντοι ἐν ἁγίοις Μαρκιανὸς μετὰ τὴν 
εἰρημένην ἀποκάλυψιν τέσσαρας μῆνας ἐπιβιώσας εἰς τὸν ἀγήρω καὶ 
ἄλυπον μετέστη βίον μηνὶ Νοεμβρίωι εἰκάδι τρίτηι τῆς πρώτης ἰνδικτιό-
νος. 



Cyrillus Biogr., Vita Sabae 
Page 116, line 1

Τοῦ μέντοι ἀρχιεπισκόπου Σαλουστίου ἐπὶ ὀκτὼ χρόνους καὶ 
μῆνας τρεῖς τοῦ Ἱεροσολύμων θρόνου κρατήσαντος καὶ ἐν Χριστῶι   
κοιμηθέντος μηνὶ Ἰουλίωι εἰκάδι τρίτηι τῆς δευτέρας ἰνδικτιόνος 
Ἡλίας ὁ πλειστάκις ἐν τῶι περὶ τοῦ ὁσίου Εὐθυμίου λόγωι μνη-
μονευθεὶς τὴν πατριαρχίαν διεδέξατο τῶι πεντηκοστῶι ἕκτωι τῆς 
τοῦ μακαρίου Σάβα ἡλικίας χρόνωι. 



Cyrillus Biogr., Vita Sabae 
Page 117, line 18

                                                                   τῆς οὖν 
μεγάλης ἐκκλησίας οἰκοδομηθείσης καὶ κόσμωι παντὶ διακοσμη-
θείσης κατελθὼν ὁ ἐν ἁγίοις Ἡλίας ἀρχιεπίσκοπος εἰς τὴν λαύραν 
ταύτην ἐνεκαίνισεν καὶ θυσιαστήριον ἡγιασμένον κατέπηξεν ἐν 
αὐτῆι μηνὶ Ἰουλίωι πρώτηι τῆς ἐνάτης ἰνδικτιόνος, ἑξηκοστῶι δὲ 
τρίτωι τῆς τοῦ μεγάλου Σάβα ἡλικίας χρόνωι. 



Cyrillus Biogr., Vita Sabae 
Page 146, line 23

          καὶ λαβὼν ἀπὸ χειρὸς τοῦ βασιλέως ἄλλα χίλια νομίσματα 
καὶ συνταξάμενος κατέπλευσεν ἐπὶ Παλαιστίνην περὶ τὸν Μάιον 
μῆνα τῆς πέμπτης ἰνδικτιόνος. 



Cyrillus Biogr., Vita Sabae 
Page 148, line 26

                                   καὶ πάλιν ἀποστέλλει τὰ αὐτὰ συνο-
δικὰ εἰς Ἱεροσόλυμα τῶι Μαίωι μηνὶ τῆς ἕκτης ἰνδικτιόνος μετά 
τινων κληρικῶν καὶ δυνάμεως βασιλικῆς. 



Cyrillus Biogr., Vita Sabae 
Page 150, line 11

           ὅστις Ὄλυμπος μετὰ δυνάμεως βασιλικῆς παραγενόμενος 
καὶ πολλοῖς τρόποις καὶ μηχανήμασιν χρησάμενος καὶ τὴν εἰρημένην 
ἐπιστολὴν ἐμφανίσας Ἡλίαν μὲν τῆς ἐπισκοπῆς ἐξέωσεν καὶ εἰς 
τὸν Ἀιλᾶν περιώρισεν, Ἰωάννην δὲ τὸν Μαρκιανοῦ υἱὸν συνθέμενον 
τόν τε Σευῆρον κοινωνικὸν εἰσδέξασθαι καὶ τὴν σύνοδον Χαλκηδόνος 
ἀναθεματίσαι ἐπίσκοπον Ἱεροσολύμων πεποίηκεν τῆι πρώτηι τοῦ 
Σεπτεμβρίου μηνὸς ἀρχῆι τῆς δεκάτης ἰνδικτιόνος. 



Cyrillus Biogr., Vita Sabae 
Page 161, line 4

Ὁ ἐν ἁγίοις πατὴρ ἡμῶν Σάβας τῶι ὀγδοηκοστῶι τῆς αὐτοῦ 
ἡλικίας χρόνωι περὶ τὰς θερινὰς τροπὰς τῆς ἑνδεκάτης ἰνδικτιόνος 
ἀπῆλθεν εἰς Ἀιλᾶ πρὸς τὸν ἀρχιεπίσκοπον Ἡλίαν ἐκ θεοῦ οἰκονο-
μίας ἔχων μεθ' ἑαυτοῦ Στέφανον τὸν ἡγούμενον τῆς μονῆς τοῦ μεγάλου 
Εὐθυμίου καὶ Εὐθάλιον τὸν ἡγούμενον τῶν ἐν Ἱεριχῶι μοναστηρίων 
αὐτοῦ τοῦ μακαρίου Ἡλία. 



Cyrillus Biogr., Vita Sabae 
Page 170, line 5

Τῶι ἀγδοηκοστῶι ἕκτωι τῆς τοῦ πατρὸς ἡμῶν Σάβα ἡλικίας 
χρόνωι ὁ ἀρχιεπίσκοπος Ἰωάννης τὸν ἕβδομον ἐνιαυτὸν καὶ ἕβδομον 
μῆνα ἐν τῆι πατριαρχίαι πληρώσας καὶ τὸν μακαριώτατον Πέτρον 
Ἐλευθεροπολίτην ὄντα τῶι γένει διάδοχον τῆς ἱεραρχίας καταλιπὼν 
ἐτελεύτησεν μηνὶ Ἀπριλλίωι εἰκάδι τῆς δευτέρας ἰνδικτιόνος. 



Cyrillus Biogr., Vita Sabae 
Page 170, line 14

                                                         προάγεται 
τοίνυν αὐτόν, ὡς εἴρηται, βασιλέα μηνὶ Ἀπριλλίωι τῆς πέμπτης 
ἰνδικτιόνος ἐν τῆι ἁγίαι πέμπτηι. 



Cyrillus Biogr., Vita Sabae 
Page 171, line 28

Ἀρχομένου τοῦ ἐνενηκοστοῦ πρώτου τῆς τοῦ ἁγίου πατρὸς ἡμῶν 
Σάβα ἡλικίας χρόνου ὁ ἐν ἁγίοις ἀββᾶς Θεοδόσιος τέλει τοῦ βίου 
ἐχρήσατο μηνὶ Ἰαννουαρίωι ἑνδεκάτηι τῆς ἑβδόμης ἰνδικτιόνος, 
<πρεσβύτης καὶ πλήρης ἡμερῶν>. 



Cyrillus Biogr., Vita Sabae 
Page 173, line 11

                              καὶ εἴξας ταῖς τῶν ἀρχιερέων παρακλή-
σεσιν ὁ πρεσβύτης ἀνέρχεται ἐν Κωνσταντινουπόλει περὶ τὸν Ἀπρίλ-
λιον μῆνα τῆς ὀγδόης ἰνδικτιόνος. 



Cyrillus Biogr., Vita Sabae 
Page 179, line 11

                                               καὶ ταῦτα μὲν ὕστερον· 
ὁ δὲ θεῖος πρεσβύτης ἀποστήσας, ὡς εἴρηται, τῆς ἑαυτοῦ συνοδίας 
Λεόντιόν τε τὸν Βυζάντιον καὶ τοὺς τῶι Θεοδώρωι τῶι Μομψουεστίας 
προσκειμένους καὶ τούτους ἐν Κωνσταντινουπόλει καταλιπὼν κατέπλευ-
σεν ἐπὶ Παλαιστίνην μηνὶ Σεπτεμβρίωι τῆς ἐνάτης ἰνδικτιόνος καὶ 
ἐλθὼν εἰς Ἱεροσόλυμα τὰς μὲν θείας κελεύσεις ἐνεφάνισεν, ὅπερ δὲ 
ἤγαγεν ἀπὸ τοῦ Βυζαντίου χρυσίον, διένειμεν τοῖς ὑπ' αὐτὸν μονα-
στηρίοις. 



Cyrillus Biogr., Vita Sabae 
Page 183, line 6

                    καὶ ἡ μὲν τελείωσις αὐτοῦ γέγονεν κατὰ τὴν 
πέμπτην τοῦ Δεκεμβρίου μηνὸς τῆς δεκάτης ἰνδικτιόνος, ἀπὸ μὲν 
κτίσεως κόσμου ἀφ' οὗπερ ἤρξατο χρόνος τῆι τοῦ ἡλίου φορᾶι μετρεῖ-
σθαι ἔτους τετάρτου εἰκοστοῦ ἑξακισχιλιοστοῦ, ἀπὸ δὲ τῆς τοῦ θεοῦ 
λόγου ἐκ παρθένου ἐνανθρωπήσεως καὶ κατὰ σάρκα γεννήσεως ἔτους 
τετάρτου εἰκοστοῦ πεντακοσιοστοῦ κατὰ τοὺς συγγραφέντας χρόνους 
ὑπὸ τῶν ἁγίων πατέρων Ἱππολύτου τε τοῦ παλαιοῦ καὶ γνωρίμου 
τῶν ἀποστόλων καὶ Ἐπιφανίου τοῦ τῶν Κυπρίων ἀρχιερέως καὶ 
Ἥρωνος τοῦ φιλοσόφου καὶ ὁμολογητοῦ· ὁ δὲ χρόνος τῆς αὐτοῦ 
ἐν σαρκὶ ζωῆς οὗτός ἐστιν. 



Cyrillus Biogr., Vita Sabae 
Page 184, line 13

                           τοῦτο γὰρ ἐγὼ αὐταῖς ὄψεσιν ἐθεασάμην 
ἐπὶ τῆς παρελθούσης δεκάτης ἰνδικτιόνος. 



Cyrillus Biogr., Vita Sabae 
Page 189, line 14

                                                      ὁ ἀδελφὸς αὐτοῦ 
Γελάσιος τὴν τοῦ ἀββᾶ Σάβα ἡγεμονίαν διεδέξατο ἐν ἀρχῆι τῆς πεντε-
καιδεκάτης ἰνδικτιόνος. 



Cyrillus Biogr., Vita Sabae 
Page 192, line 13

Τοῦ τοίνυν κατὰ Ὠριγένους ἰδίκτου ἐν Ἱεροσολύμοις ἐμφανισθέντος 
περὶ τὸν Φεβρουάριον μῆνα τῆς πέμπτης ἰνδικτιόνος τῶι ἑνδεκάτωι 
τῆς τοῦ πατρὸς ἡμῶν Σάβα κοιμήσεως χρόνωι καὶ πάντων τῶν 
κατὰ Παλαιστίνην ἐπισκόπων καὶ τῶν τῆς ἐρήμου ἡγουμένων τῶι 
αὐτῶι καθυπογραψάντων ἰδίκτωι παρεκτὸς Ἀλεξάνδρου τοῦ ἐπισκό-
που Ἀβίλης ἀγανακτήσαντες οἱ περὶ Νόννον καὶ Πέτρον καὶ Μηνᾶν 
καὶ Ἰωάννην καὶ Κάλλιστον καὶ Ἀναστάσιον καὶ λοιποὺς τῆς αἱρέσεως 
ἀρχηγοὺς τῆς τε καθολικῆς κοινωνίας ἀπέστησαν καὶ τῆς Νέας λαύρας 
ὑποχωρήσαντες ἔμειναν εἰς τὴν πεδιάδα. 



Cyrillus Biogr., Vita Sabae 
Page 195, line 6

καὶ δὴ φθάσας τὸ Ἀμόριον ἐτελεύτησεν μηνὶ Ὀκτοβρίωι τῆς ἐνάτης 
ἰνδικτιόνος. 



Cyrillus Biogr., Vita Sabae 
Page 195, line 19

                                              καὶ καταφέρουσιν αὐτὸν 
εἰς τὴν λαύραν δορυφορούμενον καὶ καθίζουσιν αὐτὸν ἐν τῶι θρόνωι 
τοῦ ἐν ἁγίοις πατρὸς ἡμῶν Σάβα μηνὶ Φεβρουαρίωι τῆς ἐνάτης ἰνδικ-
τιόνος. 



Cyrillus Biogr., Vita Sabae 
Page 196, line 17

                                                           ὅστις 
ἀββᾶς Κασιανὸς τὴν τοῦ θείου πρεσβύτου ἱερὰν ποίμνην ἐπὶ δέκα 
μῆνας ποιμάνας ἐν εἰρήνηι ἐπὶ τὸ αὐτὸ ἐκοιμήθη καὶ ὕπνωσεν μηνὶ 
Ἰουλίωι εἰκάδι τῆς δεκάτης ἰνδικτιόνος τῶι ἑξκαιδεκάτωι τῆς τοῦ 
μεγάλου Σάβα κοιμήσεως χρόνωι. 



Cyrillus Biogr., Vita Sabae 
Page 198, line 6

                                                       ὅστις Ἰσίδωρος   
μὴ ἰσχύων ἀντιστῆναι τῶι Ἀσκιδᾶι καὶ τοῖς Νεολαυρίταις προσερρύη 
τῶι τῆς ἡμετέρας ἀγέλης ποιμένι ἀββᾶι Κόνωνι καὶ δοὺς αὐτῶι 
λόγον εἰς τὴν ἁγίαν Σιὼν ὡς οὐκ ἀντιλαμβάνεται τοῦ τῆς προυπάρξεως 
δόγματος, ἀλλὰ καὶ πάσηι δυνάμει συναγωνίσηται κατὰ τῆς ἀσεβείας, 
ἀνέβη σὺν αὐτῶι ἐν Κωνσταντινουπόλει ἐν ἀρχῆι τῆς πεντεκαιδεκάτης 
ἰνδικτιόνος. 



Cyrillus Biogr., Vita Sabae 
Page 200, line 2

                                              τοιγαροῦν συναθροισθέντες 
εἰς τὴν ἁγίαν πόλιν ἐξήλθομεν μετὰ τοῦ πατριάρχου καὶ τοῦ νέου 
ἡγουμένου ἐπὶ Θεκῶαν τὴν κώμην καὶ τῶν Ὠριγενιαστῶν ὑπὸ   
Ἀναστασίου τοῦ δουκὸς διωχθέντων παρελάβομεν τὴν Νέαν λαύραν 
μηνὶ Φεβρουαρίωι εἰκάδι πρώτηι τῆς δευτέρας ἰνδικτιόνος τῶι 
εἰκοστῶι τρίτωι τῆς τοῦ μακαρίου Σάβα κοιμήσεως χρόνωι. 



Cyrillus Biogr., Vita Joannis Hesychastae (2877: 003)
“Kyrillos von Skythopolis”, Ed. Schwartz, E.
Leipzig: Hinrichs, 1939; Texte und Untersuchungen 49.2.
Page 201, line 17

Τοιγαροῦν ἐγεννήθη, ὡς αὐτός μοι διηγήσατο, κατὰ τὴν ὀγδόην 
τοῦ Ἰαννουαρίου μηνὸς τῆς ἑβδόμης ἰνδικτιόνος τῶι τετάρτωι ἔτει 
τῆς Μαρκιανοῦ τοῦ θεοφιλοῦς βασιλείας· καὶ Χριστιανῶν ὄντων 
τῶν γεγεννηκότων Χριστιανικῶς ἀνήγετο σὺν τοῖς ἑαυτοῦ ἀδελφοῖς, 
χρόνου δὲ τινὸς διελθόντος καὶ τῶν γονέων ἐν Χριστῶι τελειωθέντων 
καὶ τῆς γονικῆς οὐσίας διανεμηθείσης ὁ θεοφόρος οὗτος ἀνὴρ τῶι 
θεῶι ἑαυτὸν ἀφιέρωσεν καὶ οἰκοδομήσας ἐν αὐτῆι τῆι Νικοπόλει 
ἐκκλησίαν τῆι πανυμνήτωι θεοτόκωι καὶ ἀειπαρθένωι Μαρίαι ἀπε-  
τάξατο τοῖς τοῦ βίου πράγμασιν τῶι ὀκτωκαιδεκάτωι τῆς ἑαυτοῦ 
ἡλικίας χρόνωι καὶ προσλαβόμενος δέκα ἀδελφοὺς θέλοντας σωθῆναι 
κοινόβιον αὐτόθι συνεστήσατο. 



Cyrillus Biogr., Vita Joannis Hesychastae 
Page 204, line 23

καὶ πιστεύσας ἐξῆλθεν εὐθέως καὶ τῶι φωτὶ ἐκείνωι ἠκολούθει καὶ 
τοῦ φωτὸς ὁδηγοῦντος ἦλθεν εἰς τὴν Μεγίστην λαύραν τοῦ ἐν ἁγίοις 
πατρὸς ἡμῶν Σάβα Σαλουστίου τὸ τηνικαῦτα ἱεραρχοῦντος τῆς Ἱερο-
σολυμιτῶν ἐκκλησίας ἐπὶ τῆς τεσσαρεσκαιδεκάτης ἰνδικτιόνος τῶι 
τριακοστῶι ὀγδόωι τῆς αὐτοῦ ἡλικίας χρόνωι, ἐν ὧι χρόνωι ἡ μὲν 
θεόκτιστος ἐκκλησία τῆς Μεγίστης ἐνεκαινίσθη λαύρας, ὁ δὲ Ἀνα-  
στάσιος τὴν βασιλείαν Ζήνωνος τελευτήσαντος διεδέξατο, καθὼς τῆς 
αὐτοῦ ἀκήκοα διηγουμένης γλώσσης, γεγονὼς τοίνυν εἰς τὴν Μεγίστην 
λαύραν εὗρεν τὸν μακαρίτην Σάβαν συνοδίαν ἑκατὸν πεντήκοντα 
ἀναχωρητῶν περιποιησάμενον, ἐν πολλῆι μὲν τῶν σωματικῶν διαγόν-
των πτωχείαι, τοῖς δὲ πνευματικοῖς πλουτούντων χαρίσμασιν. 



Cyrillus Biogr., Vita Joannis Hesychastae 
Page 205, line 27

                                   τοῦ δὲ καιροῦ τῆς ἀλλαγῆς τῶν δια-
κονιῶν φθάσαντος ἐπὶ τῆς πρώτης ἰνδικτιόνος ὁ προβληθεὶς οἰκονόμος 
τοῦτον τὸν μέγαν φωστῆρα ξενοδόχον προβάλλεται καὶ μάγειρον, καὶ   
ταύτην μετὰ προθυμίας καὶ χαρᾶς τὴν διακονίαν δεξάμενος πάντας 
τοὺς πατέρας ταῖς ὑπουργίαις ἐθεράπευεν δουλεύων ἑκάστωι μετὰ 
πάσης ταπεινοφροσύνης καὶ ἐπιεικείας. 



Cyrillus Biogr., Vita Joannis Hesychastae 
Page 207, line 7

                               πληρώσαντα δὲ τὴν διακονίαν ἠβουλήθη 
ὁ μακάριος Σάβας χειροτονῆσαι αὐτὸν ὡς ἐνάρετον καὶ τέλειον μονα-
χὸν καὶ λαβὼν αὐτὸν εἰς τὴν ἁγίαν πόλιν ἐπὶ τῆς ἕκτης ἰνδικτιόνος 
προσήγαγεν τῶι μακαρίτηι Ἡλίαι τῶι ἀρχιεπισκόπωι καὶ τὰς αὐτοῦ 
ἀρετὰς διηγησάμενος παρεκάλει χειροτονηθῆναι αὐτὸν πρεσβύτερον. 



Cyrillus Biogr., Vita Joannis Hesychastae 
Page 208, line 27

ὁ δὲ γέρων ὑπέσχετο διὰ τοῦ λόγου τοῦ θεοῦ μηδενὶ τὸ σύνολον 
τοῦτο ἀπαγγεῖλαι, καὶ ἀπὸ τότε ἡσύχαζεν εἰς τὸ κελλίον αὐτοῦ μήτε 
εἰς τὴν ἐκκλησίαν προερχόμενος μήτε τινὶ τὸ σύνολον συντυγχάνων 
παρεκτὸς τοῦ διακονοῦντος αὐτῶι ἐπὶ τέσσαρας χρόνους πλὴν τῆς 
ἡμέρας τοῦ ἐγκαινισμοῦ τοῦ γεγονότος ἐν τῆι λαύραι ἐπὶ τῆς ἐνάτης 
ἰνδικτιόνος τοῦ σεβασμίου οἴκου τῆς παναγίας θεοτόκου καὶ ἀειπαρ-  
θένου Μαρίας. 



Cyrillus Biogr., Vita Joannis Hesychastae 
Page 209, line 11

Τοῦ δὲ τετραετοῦς χρόνου πληρωθέντος καὶ τοῦ μακαρίου Σάβα 
τῆς λαύρας ὑποχωρήσαντος ἐπὶ τὰ μέρη Σκυθοπόλεως διὰ τὴν ἀκα-
ταστασίαν τῶν εἰς ὕστερον τὴν Νέαν λαύραν οἰκησάντων ὁ τιμιώτατος 
οὗτος Ἰωάννης φεύγων τὸ τῆς ἀταξίας συνέδριον ἀνεχώρησεν εἰς 
τὴν ἔρημον τοῦ Ῥουβᾶ τῶι πεντηκοστῶι τῆς αὐτοῦ ἡλικίας χρόνωι 
ἐπὶ τῆς ἑνδεκάτης ἰνδικτιόνος, ἡσύχασεν δὲ αὐτόθι ἐν σπηλαίωι 
χρόνους ἓξ πάσης ἀνθρωπίνης συναναστροφῆς ἀφιστῶν ἑαυτόν, 
ὁμιλεῖν τῶι θεῶι ἐπιποθῶν ἐν ἡσυχίαι καὶ τὸ τῆς διανοίας ὀπτικὸν 
τῆι μακρᾶι φιλοσοφίαι ἐκκαθᾶραι πρὸς τὸ ἀνακεκαλυμμένωι προσώ-
πωι τὴν δόξαν κυρίου κατοπτρίζεσθαι, πᾶσαν σπουδὴν ποιούμενος 
ἀπὸ δόξης εἰς δόξαν προκόπτειν τῆι τῶν κρειττόνων ἐπιθυμίαι. 



Cyrillus Biogr., Vita Joannis Hesychastae 
Page 212, line 24

καὶ πολλαῖς ἑτέραις πρὸς αὐτὸν χρησάμενος παραινέσεσιν ἀνήγαγεν 
αὐτὸν εἰς τὴν Μεγίστην λαύραν ἐπὶ τῆς δευτέρας ἰνδικτιόνος καὶ 
καθεῖρξεν αὐτὸν εἰς κελλίον πεντηκοστὸν ἕκτον χρόνον τῆς ἑαυτοῦ 
ἡλικίας ἄγοντα μηδενὸς ἑτέρου τῆς συνοδίας γινώσκοντος ὅτι ἐπίσκο-  
πός ἐστιν. 



Cyrillus Biogr., Vita Joannis Hesychastae 
Page 214, line 7

Τῶι ἑβδομηκοστῶι ἐνάτωι τῆς τοῦ ὁσίου τούτου Ἰωάννου ἡλικίας 
χρόνωι, εἰκοστῶι τετάρτωι δὲ τῆς αὐτοῦ ἐν τῶι κελλίωι καθείρξεως 
ὁ ἐν ἁγίοις πατὴρ ἡμῶν Σάβας ἐν εἰρήνηι ἐπὶ τὸ αὐτὸ ἐκοιμήθη 
καὶ ὕπνωσεν μηνὶ Δεκεμβρίωι πέμπτηι τῆς δεκάτης ἰνδικτιόνος. 



Cyrillus Biogr., Vita Joannis Hesychastae 
Page 216, line 9

Τῶι ἐνενηκοστῶι τῆς τοῦ ὁσίου τούτου γέροντος ἡλικίας χρόνωι 
κατὰ τὸν Νοέμβριον μῆνα τῆς ἕκτης ἰνδικτιόνος ἐξερχόμενος ἐκ τῆς 
Σκυθοπολιτῶν μητροπόλεως, ὡς καὶ ἤδη μοι εἴρηται ἐν τῶι περὶ 
Εὐθυμίου τοῦ ἁγίου λόγωι, ἐντολὰς ἔλαβον παρὰ τῆς ἐμῆς φιλο-
χρίστου μητρὸς μηδὲν τῶν εἰς ψυχὴν συντεινόντων διαπράξασθαι 
χωρὶς γνώμης καὶ ἐπιτροπῆς τούτου τοῦ θεσπεσίου Ἰωάννου, μήποτε, 
λέγουσά μοι, τῆι τῶν Ὠριγενιαστῶν πλάνηι συναπαχθεὶς ἐκπέσηις 
ἐκ προοιμίων τοῦ ἰδίου στηριγμοῦ. 



Cyrillus Biogr., Vita Joannis Hesychastae 
Page 217, line 12

                                                              καὶ οὕτως 
ἀνελθὼν ἔμεινα εἰς τὴν τοῦ ἐν ἁγίοις Εὐθυμίου μονὴν μηνὶ Ἰουλίωι 
τῆς ἕκτης ἰνδικτιόνος. 



Cyrillus Biogr., Vita Cyriaci (2877: 004)
“Kyrillos von Skythopolis”, Ed. Schwartz, E.
Leipzig: Hinrichs, 1939; Texte und Untersuchungen 49.2.
Page 223, line 6

Οὗτος τοίνυν ὁ τὴν ψυχὴν ἔχων πεφωτισμένην Κυριακὸς γένος 
μὲν τῆς Ἑλλάδος ὑπῆρχεν πόλεως Κορίνθου, γονέων δὲ πατρὸς μὲν 
Ἰωάννου πρεσβυτέρου τῆς κατὰ Κόρινθον ἁγίας τοῦ θεοῦ καθολικῆς 
ἐκκλησίας, μητρὸς δὲ Εὐδοξίας, ὑφ' ὧν ἐγεννήθη περὶ τὸ τέλος τῆς 
Θεοδοσίου τοῦ νέου βασιλείας μηνὶ Ἰαννουαρίωι ἐνάτηι τῆς δευτέρας 
ἰνδικτιόνος. 



Cyrillus Biogr., Vita Cyriaci 
Page 224, line 12

                                                καὶ τοῖς τοιούτοις 
ἀδολεσχῶν λογισμοῖς ἤκουσεν ἐν μιᾶι κυριακῆι τοῦ εὐαγγελίου λέ-
γοντος <εἴ τις θέλει ὀπίσω μου ἐλθεῖν, ἀπαρνησάσθω 
ἑαυτὸν καὶ ἀράτω τὸν σταυρὸν αὐτοῦ καὶ ἀκολουθείτω 
μοι>, καὶ παραυτίκα ἐξελθὼν τῆς ἐκκλησίας καὶ μηδενὶ μηδὲν εἰρη-
κὼς ἦλθεν ἐπὶ Κεγχρεὰς καὶ ἐπιβὰς πλοίωι διαπερῶντι ἐν Παλαι-
στίνηι παρεγένετο εἰς Ἱεροσόλυμα ὀκτωκαιδέκατον μὲν ἔτος ἄγων 
τῆς ἡλικίας, ὀγδόωι δὲ ἔτει τῆς ἱεραρχίας ἐν Ἱεροσολύμοις Ἀναστα-
σίου, ἐνάτωι δὲ τῆς βασιλείας Λέοντος τοῦ μεγάλου βασιλέως ἐν 
ἀρχῆι τῆς πέμπτης ἰνδικτιόνος. 



Cyrillus Biogr., Vita Cyriaci 
Page 225, line 20

                                               τῶι μέντοι ἐνάτωι χρόνωι 
τῆς ἐν Παλαιστίνηι παρουσίας τοῦ ἀββᾶ Κυριακοῦ ὁ μέγας πατὴρ 
ἡμῶν Γεράσιμος ἐτελεύτησεν καὶ τῶι τῆς δικαιοσύνης κατεκοσμήθη 
στεφάνωι μηνὶ Μαρτίωι πέμπτηι τῆς τρισκαιδεκάτης ἰνδικτιόνος. 



Cyrillus Biogr., Vita Cyriaci 
Page 226, line 22

                          τότε οἱ μὲν τῆς μονῆς τοῦ μεγάλου Εὐθυμίου 
ἠγόρασαν ξενοδοχεῖον τῶν αὐτῶν διακοσίων νομισμάτων πλησίον 
τοῦ πύργου τοῦ Δαυὶδ παρὰ τῶν πατέρων τῆς λαύρας τοῦ Σουκᾶ· 
ὁ δὲ ἀββᾶς Κυριακὸς διὰ τὴν γενομένην τῶν μοναστηρίων διαίρεσιν 
ἀναχωρεῖ λυπηθεὶς κατὰ διάνοιαν καὶ ἔρχεται εἰς τὴν λαύραν τοῦ 
Σουκᾶ περὶ τὸ τέλος τῆς ὀγδόης ἰνδικτιόνος. 



Cyrillus Biogr., Vita Cyriaci 
Page 227, line 8

                                                  τῶι μέντοι ἑβδομηκοστῶι 
ἑβδόμωι τῆς αὐτοῦ ἡλικίας χρόνωι παραδίδωσι τὸ κειμηλιαρχεῖον 
ἐπὶ τῆς τρίτης ἰνδικτιόνος καὶ ἀναχωρεῖ ἐπὶ τὴν τοῦ Νατουφᾶ πανέρημον 
μαθητήν τινα ἔχων μεθ' ἑαυτοῦ. 



Cyrillus Biogr., Vita Theodosii (2877: 005)
“Kyrillos von Skythopolis”, Ed. Schwartz, E.
Leipzig: Hinrichs, 1939; Texte und Untersuchungen 49.2.
Page 239, line 28

                                      ἡ δὲ τελείωσις αὐτοῦ γέγονεν κατὰ 
τὴν ἑνδεκάτην τοῦ Ἰαννουαρίου μηνὸς τῆς ἑβδόμης ἰνδικτιόνος ἐν 
εἰκοστῶι δευτέρωι μηνὶ τῆς βασιλείας τοῦ θεοφυλάκτου βασιλέως 
ἡμῶν Ἰουστινιανοῦ. 



Cyrillus Biogr., Vita Theodosii 
Page 241, line 2

       διαλάμπουσιν γὰρ οἱ πόνοι Σωφρονίου καὶ τὰ τούτου κατορθώ-
ματα ἐν τῆι τοῦ μακαρίου ἀββᾶ Θεοδοσίου μονῆι· οὐ μόνον γὰρ κτή-
μασι καὶ προσόδοις ἐνιαυσιαίοις ἐπλούτισεν ταύτην, ἀλλὰ καὶ τὴν 
ἐν αὐτῆι ἐν Χριστῶι συνοδίαν ἐπλήθυνεν τριπλασίως καὶ ἁπλῶς εἰπεῖν   
ἐπὶ δέκα τέσσαρας χρόνους [καὶ δύο μῆνας] καλῶς αὐτὴν κυβερνήσας 
ἐτελεύτησεν ἐν χαρᾶι μηνὶ Μαρτίωι εἰκάδι πρώτηι τῆς πέμπτης ἰνδι-
κτιόνος. 



Cyrillus Biogr., Vita Abramii (2877: 007)
“Kyrillos von Skythopolis”, Ed. Schwartz, E.
Leipzig: Hinrichs, 1939; Texte und Untersuchungen 49.2.
Page 245, line 3

                                                           καὶ γενόμενος 
ἐν πολλῆι ἀθυμίαι ἐξῆλθεν λάθρα τὴν πόλιν καὶ φυγὰς ὤιχετο ἐπὶ   
τὴν ἁγίαν πόλιν μηδὲν τοῦ αἰῶνος τούτου ἐπιφερόμενος καὶ μετὰ 
πολλῆς ἐνδείας καὶ στενοχωρίας ἦλθεν εἰς Ἱεροσόλυμα ἐν ἀρχῆι 
τῆς πέμπτης ἰνδικτιόνος τὸν τριακοστὸν ἕβδομον τῆς ἡλικίας πλη-
ρώσας ἐνιαυτόν. 



Cyrillus Biogr., Vita Gerasimi [Sp.] (2877: 008)
“Ἀνάλεκτα Ἱεροσολυμιτικῆς σταχυολογίας, vol. 4”, Ed. Papadopoulos–Kerameus, A.
St. Petersburg: Kirschbaum, 1897, Repr. 1963.
Page 184, line 5

10 Ἡ μέντοι τελείωσις τοῦ πατρὸς ἡμῶν Γερασίμου γέγονε 
κατὰ τὴν πέμπτην τοῦ μαρτίου μηνὸς τῆς τρισκαιδεκάτης ἰνδικτιῶ-
νος, ἐν ἀρχῇ τῆς βασιλείας Ζήνωνος. 



Cyrillus Biogr., Vita Gerasimi [Sp.] 
Page 184, line 13

                                            Βασίλειον δὲ καὶ 
Στέφανον ἀδελφοὺς αὐτοῦ κατὰ σάρκα διαδόχους τῆς ἡγεμονίας κα-
τέλιπεν, οἵτινες ἐπὶ ἓξ ἔτη τὴν αὐτὴν ποιμάναντες συνοδίαν ἐτε-
λεύτησαν, καὶ οὕτως ψήφῳ πάντων τῶν ἁγίων πατέρων ὁ 
τρισμακάριος Εὐγένιος τῆς τοῦ μεγάλου Γερασίμου ἀγέλης ἐκράτησε, 
καὶ ταύτην καλῶς καὶ θεαρέστως ἐπὶ τεσσαράκοντα καὶ πέντε 
ἔτη καὶ μῆνας τέσσαρας ποιμάνας ἐκοιμήθη μηνὶ αὐγούστῳ ιθʹ 
τῆς τετάρτης ἰνδικτιῶνος. 




Clemens Romanus et Clementina Theol., Epistula i ad Corinthios (1271: 001)
“Clément de Rome. Épître aux Corinthiens”, Ed. Jaubert, A.
Paris: Cerf, 1971; Sources chrétiennes 167.
Chapter 23, section 2, line 1

   Διὸ μὴ διψυχῶμεν, μηδὲ ἰνδαλ-
λέσθω ἡ ψυχὴ ἡμῶν ἐπὶ ταῖς ὑπερβαλλούσαις καὶ ἐνδόξοις 
δωρεαῖς αὐτοῦ. 



Clemens Romanus et Clementina Theol., Homiliae [Sp.] (1271: 006)
“Die Pseudoklementinen I. Homilien, 2nd edn.”, Ed. Rehm, B., Irmscher, J., Paschke, F.
Berlin: Akademie–Verlag, 1969; Die griechischen christlichen Schriftsteller 42.
Homily 4, chapter 4, section 2, line 1

Ἡ Βερνίκη δὲ ἀξιωθεῖσα· Ταῦτα μὲν οὕτως, ἔφη, ἔχει ὡς ἠκούσατε, 
τὰ δὲ ἄλλα τὰ κατ' αὐτὸν τὸν Σίμωνα, ἅπερ ἴσως ἀγνοεῖτε, ἀκούσατε· 
φαντάσματά τε γὰρ καὶ ἰνδάλματα ἐν μέσῃ τῇ ἀγορᾷ φαίνεσθαι ποιῶν δι' 
ἡμέρας πᾶσαν ἐκπλήττει τὴν πόλιν καὶ προιόντος αὐτοῦ ἀνδριάντες κινοῦνται 
καὶ σκιαὶ πολλαὶ προηγοῦνται, ἅσπερ αὐτὰς ψυχὰς τῶν τεθνηκότων εἶναι 
λέγει. 

Clemens Romanus et Clementina Theol., Recognitiones (ex Eusebio) [Sp.] (1271: 008)
“Die Pseudoklementinen II. Rekognitionen”, Ed. Rehm, B., Paschke, F.
Berlin: Akademie–Verlag, 1965; Die griechischen christlichen Schriftsteller 51.
Book 9, chapter 20, line 1

Παρὰ Ἰνδοῖς καὶ Βάκτροις εἰσὶ χιλιάδες πολλαὶ τῶν λεγομένων Βραχ-
μάνων, οἵτινες κατὰ παράδοσιν τῶν προγόνων καὶ νόμων οὔτε φονεύουσιν 
οὔτε ξόανα σέβονται, οὐκ ἐμψύχου γεύονται, οὐ μεθύσκονταί ποτε, οἴνου 
καὶ σίκερος μὴ γευόμενοι,   
οὐ κακίᾳ τινὶ κοινωνοῦσι προσέχοντες τῷ θεῷ, τῶν ἄλλων Ἰνδῶν φονευ-
όντων καὶ ἑταιρευόντων καὶ μεθυσκομένων καὶ σεβομένων ξόανα καὶ 
πάντα σχεδὸν καθ' εἱμαρμένην φερομένων. 



Clemens Romanus et Clementina Theol., Recognitiones (ex Eusebio) [Sp.] 
Book 9, chapter 20, line 8

                                             ἔστι δὲ ἐν τῷ αὐτῷ κλίματι 
τῆς Ἰνδίας φυλή τις Ἰνδῶν, οἵτινες τοὺς ἐμπίπτοντας ξένους ἀγρεύοντες 
καὶ θύοντες ἐσθίουσι· καὶ οὔτε οἱ ἀγαθοποιοὶ τῶν ἀστέρων κεκωλύκασι τού-
τους μὴ μιαιφονεῖν καὶ μὴ ἀθεμιτογαμεῖν οὔτε οἱ κακοποιοὶ ἠνάγκασαν τοὺς 
Βραχμᾶνας κακουργεῖν. 



Clemens Romanus et Clementina Theol., Recognitiones (ex Eusebio) [Sp.] 
Book 9, chapter 25, line 4

Οἱ Μῆδοι πάντες τοῖς μετὰ σπουδῆς τρεφομένοις κυσὶ τοὺς νεκροὺς   
ἔτι ἐμπνέοντας παραβάλλουσι, καὶ οὐ πάντες σὺν τῇ Μήνῃ τὸν Ἄρεα ἐφ' 
ἡμερινῆς γενέσεως ἐν Καρκίνῳ ὑπὸ γῆν ἔχουσιν. 
 Ἰνδοὶ τοὺς νεκροὺς καίουσι, μεθ' ὧν συγκαίουσιν ἑκούσας τὰς γυναῖκας, 
καὶ οὐ δήπου πᾶσαι αἱ καιόμεναι ζῶσαι Ἰνδῶν γυναῖκες ἔχουσιν ὑπὸ γῆν 
ἐπὶ νυκτερινῆς γενέσεως σὺν Ἄρει τὸν Ἥλιον ἐν Λέοντι ὁρίοις Ἄρεος. 



Clemens Romanus et Clementina Theol., Recognitiones (ex Eusebio) [Sp.] 
Book 9, chapter 25, line 5

Ἰνδοὶ τοὺς νεκροὺς καίουσι, μεθ' ὧν συγκαίουσιν ἑκούσας τὰς γυναῖκας, 
καὶ οὐ δήπου πᾶσαι αἱ καιόμεναι ζῶσαι Ἰνδῶν γυναῖκες ἔχουσιν ὑπὸ γῆν 
ἐπὶ νυκτερινῆς γενέσεως σὺν Ἄρει τὸν Ἥλιον ἐν Λέοντι ὁρίοις Ἄρεος. 



Clemens Romanus et Clementina Theol., Recognitiones (ex Eusebio) [Sp.] 
Book 9, chapter 25, line 13

                                                           καὶ οὐκ ἀναγκάζει ἡ 
γένεσις τοὺς Σῆρας μὴ θέλοντας φονεύειν ἢ τοὺς Βραχμᾶνας κρεοφαγεῖν 
ἢ τοὺς Πέρσας ἀθεμίτως μὴ γαμεῖν ἢ τοὺς Ἰνδοὺς μὴ καίεσθαι ἢ τοὺς Μήδους   
μὴ ἐσθίεσθαι ὑπὸ κυνῶν ἢ τοὺς Πάρθους μὴ πολυγαμεῖν ἢ τὰς ἐν 
τῇ Μεσοποταμίᾳ γυναῖκας μὴ σωφρονεῖν ἢ τοὺς Ἕλληνας μὴ γυμνάζεσθαι 
γυμνοῖς τοῖς σώμασιν ἢ τοὺς Ῥωμαίους μὴ κρατεῖν ἢ τοὺς Γάλλους μὴ 
γαμεῖσθαι ἢ τὰ ἄλλα βάρβαρα ἔθνη ταῖς ὑπὸ τῶν Ἑλλήνων λεγομέναις 
Μούσαις κοινωνεῖν· ἀλλ', ὡς προεῖπον, ἕκαστον ἔθνος καὶ ἕκαστος τῶν 
ἀνθρώπων χρῆται τῇ ἑαυτοῦ ἐλευθερίᾳ ὡς βούλεται καὶ ὅτε βούλεται, καὶ 
δουλεύει τῇ γενέσει καὶ τῇ φύσει δι' ἣν περίκειται σάρκα, πῆ μὲν ὡς βούλεται, 
πῆ δὲ ὡς μὴ βούλεται. 

%ok

\end{greek}

%Something in the below breaks the run
\begin{comment}


Clemens Romanus et Clementina Theol., Recognitiones (ex Eusebio) [Sp.] Book 9, chapter 27, line 2

μνημονεύειν τε ὀφείλετε ὧν προεῖπον, ὅτι καὶ ἐν ἑνὶ κλίματι καὶ ἐν μιᾷ χώρᾳ   
τῶν Ἰνδῶν εἰσιν ἀνθρωποφάγοι Ἰνδοὶ καί εἰσιν οἱ ἐμψύχων ἀπεχόμενοι· 
καὶ ὅτι οἱ Μαγουσαῖοι οὐκ ἐν Περσίδι μόνῃ τὰς θυγατέρας γαμοῦσιν, ἀλλὰ 
καὶ ἐν παντὶ ἔθνει, ὅπου ἂν οἰκήσωσι, τοὺς τῶν προγόνων φυλάσσοντες 
νόμους καὶ τῶν μυστηρίων αὐτῶν τὰς τελετάς. 


%something in the above breaks the run 
\end{comment}

\begin{greek}



Clemens Romanus et Clementina Theol., Recognitiones (e Pseudo–Caesario) [Sp.] (1271: 009)
“Die Pseudoklementinen II. Rekognitionen”, Ed. Rehm, B., Paschke, F.
Berlin: Akademie–Verlag, 1965; Die griechischen christlichen Schriftsteller 51.
Book 9, chapter 20, line 3

Νόμος δὲ καὶ παρὰ Βακτριανοῖς ἤτοι Βραγμανοῖς ἡ ἐκ προγόνων παιδεία, 
μὴ μεθύειν μηδὲ ἐμψύχων ἀπογεύεσθαι, οὐκ οἴνου ἁπλοῦ ἢ νόθου μετέχειν,   
θεὸν τὸν ἐμὸν δεδοικότας, καίτοι τῶν παρακειμένων αὐτοῖς Ἰνδῶν μιαιφο-
νούντων καὶ οἰνοφλυγούντων καὶ μονιῶν ἀγρίων ἢ συῶν δίκην θηλυμανούν-
των καὶ τῷ πάθει κραδαινομένων. 



Clemens Romanus et Clementina Theol., Recognitiones (e Pseudo-Caesario) [Sp.] 
Book 9, chapter 20, line 6

                                     ἐν δὲ τοῖς ἑσπερίοις κλίμασιν ἐνδοτέρω τῶν 
ἐκεῖσε Ἰνδῶν ξενοβόροι τινὲς ὑπάρχοντες τοὺς ἐπήλυδας ἀναιροῦντες 
ἐσθίουσι, καὶ οὐδεὶς τῶν ἀγαθοποιῶν ἀστέρων τῆς μιαιφονίας αὐτοὺς 
ἀποστῆσαι ἴσχυσε μέχρι τήμερον. 

%ok

Clemens Romanus et Clementina Theol., Recognitiones (e Pseudo-Caesario) [Sp.] 
Book 9, chapter 25, line 6

Μῆδοι δὲ πάντες μετὰ σπουδῆς ἔτι ἐμπνέοντας τοὺς κάμνοντας κυσὶ βορὰν   
προτιθέασιν ἀναλγήτως· καὶ οὐ πάντως σὺν τῇ Μήνῃ, ὥς φατε, τὸν Ἄρεα 
ἐπὶ ἡμερινῆς γενέσεως ἐν Καρκίνῳ Μῆδοι ἔλαχον; 
 Ἰνδοὶ τοὺς νεκροὺς ἑαυτῶν τεφροποιοῦσι πυρί, μεθ' ὧν καταφλέγουσί 
τινων τὰς συμβίους. 






Clemens Romanus et Clementina Theol., Recognitiones (e Pseudo-Caesario) [Sp.] 
Book 9, chapter 25, line 7

                      καὶ οὐ δήπου πᾶσαι αἱ πυριάλωτοι Ἰνδῶν γυναῖκες [ἢ] 
αἱ ζῶσαι ἔλαχον ὑπὸ τῆς νυκτερινῆς συνελεύσεως τῶν γονέων σὺν Ἄρει 
τὸν Ἥλιον, ἐν νυκτὶ μὴ φαίνοντα ἐν μοίραις Ἄρεος. 



Clemens Romanus et Clementina Theol., Recognitiones (e Pseudo-Caesario) [Sp.] 
Book 9, chapter 25, line 17

                                                              οὐ γὰρ οἵα τε ἡ καθ' 
ὑμᾶς γένεσις ἀναγκάσαι Σῆρας ἀναιρεῖν ἢ Βραχμᾶνας κρεοβορεῖν καὶ σικερο-
ποτεῖν ἢ Πέρσας μὴ μητρογαμεῖν καὶ ἀδελφοφθορεῖν ἢ Ἰνδοὺς μὴ πυρὶ 
διδόναι τοὺς νεκροὺς ἢ Μήδους   
μὴ κυσὶ τοὺς θνησκομένους προτιθέναι ἢ Πάρθους μὴ πολυγαμεῖν ἢ τοὺς 
Μεσοποταμίτας μὴ ἄκρως σωφρονεῖν ἢ Ἕλληνας μὴ σωμασκεῖσθαι ἢ τὰ 
βάρβαρα ἔθνη ταῖς ὑφ' Ἑλλήνων προσαγορευομέναις <Μούσαις> κοινωνεῖν· 
ἀλλ', ὡς προέφην, ἕκαστος βροτῶν χρῆται τῇ τοῦ νόμου ἐλευθερίᾳ, τὰ 
ἐκ τῶν ἄστρων μυθουργούμενα καθ' Ἕλληνας παραπεμπόμενος, τῷ ἐκ τῶν 
νόμων δέει ἢ τῷ ἐξ ἔθνους ἔθει πατρίῳ τῶν φαύλων εἰργόμενος· αἱ μὲν γὰρ 
τῶν ἀρετῶν ὑπάρχουσι προαιρετικαί, αἱ δὲ περιστατικαί, ἀνάγκῃ ἐπὶ τὸ 
κρεῖττον χωροῦντος τοῦ ζητουμένου ὑπὸ τῶν νόμων. 



Clemens Romanus et Clementina Theol., Recognitiones (e Pseudo-Caesario) [Sp.] 
Book 9, chapter 27, line 1

                        πῶς δὲ ἐν ταὐτῷ τμήματι τοὺς ἀνθρωποβόρους Ἰνδοὺς 
καὶ τοὺς ἐμψύχων καὶ θοίνης ἁπάσης ἀπεχομένους Βραχμανοὺς οἰκοῦντας 
ὁρῶμεν; 



Clemens Romanus et Clementina Theol., Pseudo–Clementina (epitome altera auctore Symeone Metaphrasta) [Sp.] (1271: 011)
“Clementinorum epitomae duae, 2nd edn.”, Ed. Dressel, A.R.M.
Leipzig: Hinrichs, 1873.
Section 45, line 3

                                                      φαντάσματα γὰρ, ἔφη, μετὰ 
τὸ τῇδε αὐτὸν ἐπιστῆναι καὶ ἰνδάλματα καθ' ἑκάστην ἐν μέσῃ τῇ ἀγορᾷ 
ποιῶν πᾶσαν ἐκπλήττει τὴν πόλιν· ὡς ἀνδριάντας μὲν αὐτοῦ διερχομένου 
κινεῖσθαι, καὶ σκιὰς δὲ πολλάκις αὐτοῦ προηγεῖσθαι, ἃς καὶ τῶν τεθνηκό-
των ψυχὰς αὐτὸς ὀνομάζει. 



Clemens Romanus et Clementina Theol., Pseudo–Clementina (epitome de gestis Petri praemetaphrastica) [Sp.] (1271: 012)
“Clementinorum epitomae duae, 2nd edn.”, Ed. Dressel, A.R.M.
Leipzig: Hinrichs, 1873.
Section 45, line 2

Ἡ δὲ Βερνίκη εἶπεν· ταῦτα οὕτως ἔχει, τὰ δὲ ἄλλα τὰ κατὰ 
τὸν Σίμωνα ἀκούσατε· φαντάσματα καὶ ἰνδάλματα ἐν μέσῃ τῇ ἀγορᾷ ποιῶν 
καθ' ἡμέραν πᾶσαν ἐκπλήττει τὴν πόλιν· διερχομένου γὰρ αὐτοῦ ἀνδριάντες 
κινοῦνται καὶ σκιαὶ πολλαὶ προηγοῦνται, ἅσπερ αὐτὰς ψυχὰς τῶν τεθνη-
κότων λέγει. 


Philosophus Christianus Alchem., Τίς ἡ ἐν ἀποκρύφοις τῶν παλαιῶν ἐκδιδομένη τάξις (e cod. Venet. Marc. 299, fol. 124v) (4328: 012)
“Collection des anciens alchimistes grecs, vol. 2”, Ed. Berthelot, M., Ruelle, C.É.
Paris: Steinheil, 1888, Repr. 1963.
Volume 2, page 418, line 22

              Οὕτω γὰρ δεκτικὰ γίνεται τῶν χρωμάτων· 
ὥσπερ δὲ χοοποιηθεὶς ὅ ἐστιν λάχιον ὃ καλοῦσιν <λαχὰν> οἱ λαχωταὶ, 
τουτέστιν οἱ ἰνδικοβάφοι. 



Scholia In Dionysium Periegetam, Scholia in Dionysii periegetae orbis descriptionem (scholia vetera) (olim sub auctore Demetrio Lampsaceno) (5019: 001)
“Geographi Graeci minores, vol. 2”, Ed. Müller, K.
Paris: Didot, 1861, Repr. 1965.
Vita-verse of Orbis descriptio 1, line of scholion 56

                               Τό τε Ἰνδικὸν πέλαγος, 
ὅπερ ἐστὶ τῆς Ἐρυθρᾶς, ἄρχεται μὲν ἐκ τοῦ Ἀραβικοῦ 
κόλπου, παρατείνει δὲ μέσον ἔχον Ταπροβάνην νῆσον 
ἕως Μεγάλου τοῦ παρ' Ἰνδοῖς τε καὶ Σινδοῖς οὕτως 
καλουμένου ἑῴου κόλπου. 



Scholia In Dionysium Periegetam, Scholia in Dionysii periegetae orbis descriptionem (scholia vetera) (olim sub auctore Demetrio Lampsaceno) 
Vita-verse of Orbis descriptio 1, line of scholion 69

Φησὶ γὰρ ὁ ἀνὴρ (Ptol. 7, 5) διαιρεῖσθαι μὲν τὴν 
οἰκουμένην ἅπασαν εἰς τρεῖς ἠπείρους, Λιβύην, Εὐρώ-
πην, Ἀσίαν· καὶ ἐκ μὲν τῶν ἑῴων ἀγνώστῳ γῇ 
περιορίζεσθαι τῇ παρακειμένῃ τοῖς ἀνατολικοῖς ἔθνεσι 
τῆς Μεγάλης Ἀσίας, Σιναῖς τε καὶ Σηρικῇ· ἐκ δὲ 
μεσημβρίας ὁμοίως ἀγνώστῳ γῇ περικλειούσῃ τὸ Ἰν-
δικὸν πέλαγος· ἐκ δὲ τῶν δυτικῶν γῇ τε ἀγνώστῳ 
περιλαμβανούσῃ τὸν Αἰθιοπικὸν οὕτω λεγόμενον κόλ-
πον καὶ τῷ δυτικῷ ἐφεξῆς ὠκεανῷ καὶ πελάγει [τῷ 
περιέχοντι] τοῖς ἐκεῖ πέρασιν ὁμοῦ Λιβύης τε καὶ Εὐ-
ρώπης· ἐκ δὲ τῶν ἄρκτων αὐτὴν περιέχεσθαι [τῷ 
περιέχοντι] Θούλην ὠκεανῷ καὶ Βρεττανίαν καὶ τὰ 
βορειότερα μέρη τῆς Εὐρώπης· καλεῖσθαι δέ φησι τὸν 
ἀρκτῶον ὠκεανὸν Σαρματικόν τε καὶ Δουηκαληδόνιον· 
οὐ μὴν ἀλλὰ καὶ γῇ ἀγνώστῳ περιέχεσθαι, παρακει-
μένῃ ταῖς ἀρκτικωτάταις χώραις τῆς μεγάλης ἑῴας, 




Scholia In Dionysium Periegetam, Scholia in Dionysii periegetae orbis descriptionem (scholia vetera) (olim sub auctore Demetrio Lampsaceno) 
Vita-verse of Orbis descriptio 1, line of scholion 90

Τὰς δὲ θαλάσσας ὧδε ἔφη καθόλου κεῖσθαι· μίαν μὲν 
τὴν καθ' ἡμᾶς ἅμα τοῖς κόλποις, Ἀδριατικῷ τε καὶ τῷ 
Αἰγαίῳ πελάγει καὶ Προποντίδι καὶ Πόντῳ καὶ Μαιώ-
τιδι λίμνῃ, φέρεσθαι δὲ διὰ τῶν Ἡρακλέους στηλῶν, 
ἀρχομένην ἐκ τῶν Ἰβήρων· ἑτέραν δὲ διώνυμον οὕτω 
λέγεσθαι Κασπίαν τε καὶ Ὑρκανίαν, ἣν ἀποκλείουσά 
τις ἤπειρος ἐκ τοῦ ἐναντίου νησοποιεῖ τὴν θάλασσαν· 
τρίτην δὲ εἶναι τὴν Ἰνδικὴν, ἐπιφανῆ τοῖς ἐν αὐτῇ 
διαπρέπουσι κόλποις, Ἀραβίῳ τε καὶ Περσικῷ, Γαγ-
γητικῷ τε καὶ τῷ ἰδίως οὕτω καλουμένῳ Μεγάλῳ· 
περιλαμβάνεσθαι μέντοι καὶ αὐτὴν ὑπὸ ἀγνώστου, 
καὶ μείζονα μὲν ἔφη τῷ καταλόγῳ τῶν ἄλλων καὶ 
πρώτην, δευτέραν δὲ τὴν καθ' ἡμᾶς τῷ μεγέθει, ἐλατ-
τοῖ δὲ τὴν Ὑρκανίαν, καὶ ὑπὸ ταύτης * ὁρίζει. 



Scholia In Dionysium Periegetam, Scholia in Dionysii periegetae orbis descriptionem (scholia vetera) (olim sub auctore Demetrio Lampsaceno) 
Vita-verse of Orbis descriptio 1, line of scholion 108

           Πρωτεύει μὲν ἡ Ἰνδικὴ Ταπροβάνη με-
γέθει καὶ δόξῃ· μεθ' ἣν ἡ Βρεττανικὴ καλουμένη, παρὰ 
δὲ τοῖς οἰκοῦσιν αὐτὴν Ἀλουιωνίς, τρίτη δὲ ἡ Χρυσῆ, 
χερρόνησος Ἰνδική· τετάρτη τῶν Βρεττανῶν ἡ ἑτέρα, 
λεγομένη παρ' αὐτοῖς Ἰουερνία· πέμπτη τούτων ἡ 
Πελοπόννησος, καὶ ἡ Σικελία μετ' αὐτὴν ἕκτη· Σαρδὼ 
μετ' ἐκείνην ἑβδόμη· μετὰ δὲ ταύτην ἡ Κύρνος ὀγδόη· 
Κρήτη δὲ τάξεως ἐννάτης· μετὰ δὲ ταύτην ἡ Κύπρος 
οὖσα δεκάτη τῷ καταλόγῳ, εἰπεῖν δὲ κορωνίς. 



Scholia In Dionysium Periegetam, Scholia in Dionysii periegetae orbis descriptionem (scholia vetera) (olim sub auctore Demetrio Lampsaceno) 
Vita-verse of Orbis descriptio 37, line of scholion 1

Ὅτι τὸ πρὸς ἕω μέρος ἠῷον καὶ Ἰνδικόν. 



Scholia In Dionysium Periegetam, Scholia in Dionysii periegetae orbis descriptionem (scholia vetera) (olim sub auctore Demetrio Lampsaceno) 
Vita-verse of Orbis descriptio 37, line of scholion 2

                                                      Οἱ 
γὰρ Ἰνδοὶ ἀνατολικοί. 



Scholia In Dionysium Periegetam, Scholia in Dionysii periegetae orbis descriptionem (scholia vetera) (olim sub auctore Demetrio Lampsaceno) 
Vita-verse of Orbis descriptio 37, line of scholion 2

                            Ἔνθα καὶ ὁ Ἰνδὸς ποταμός. 



Scholia In Dionysium Periegetam, Scholia in Dionysii periegetae orbis descriptionem (scholia vetera) (olim sub auctore Demetrio Lampsaceno) 
Vita-verse of Orbis descriptio 180, line of scholion 6

                                          Εἰσὶ γὰρ καὶ 
ἀνατολικοὶ, Ἰνδοὶ καλούμενοι. 



Scholia In Dionysium Periegetam, Scholia in Dionysii periegetae orbis descriptionem (scholia vetera) (olim sub auctore Demetrio Lampsaceno) 
Vita-verse of Orbis descriptio 593, line of scholion 4

                                 Ταύτην δὲ Χρυσῆν χερ-
σόνησον ὁ Πτολεμαῖος φησί· κεῖται δὲ ἐν τῇ χώρᾳ 
τῶν Ἰνδῶν. 



Scholia In Dionysium Periegetam, Scholia in Dionysii periegetae orbis descriptionem (scholia vetera) (olim sub auctore Demetrio Lampsaceno) 
Vita-verse of Orbis descriptio 607, line of scholion 8

Ἐρύθρας δὲ βασιλεὺς, ἀφ' οὗ τὸ πέλαγος, ἦν τοῦ 
Δηριάδου, ὃς ἀντετάξατο Διονύσῳ ὑπὲρ Ἰνδῶν. 



Scholia In Dionysium Periegetam, Scholia in Dionysii periegetae orbis descriptionem (scholia vetera) (olim sub auctore Demetrio Lampsaceno) 
Vita-verse of Orbis descriptio 1074, line of scholion 1

<Ἕλκων Ἰνδὸν ὕδωρ>] ἀπὸ Ἰνδῶν ἐξιὼν ἐπὶ 
Σοῦσα δι' ἀδήλων. 



Scholia In Dionysium Periegetam, Scholia in Dionysii periegetae orbis descriptionem (scholia vetera) (olim sub auctore Demetrio Lampsaceno) 
Vita-verse of Orbis descriptio 1097, line of scholion 3

Παρνησοῖο] Τὰ κρείττονα τῶν ἀντιγράφων ἢ 
<Παρπάνισσον> γράφουσι προπαροξυτόνως, ὄρος ὂν 
Ἰνδικὸν, ἢ Παρπαμισσόν ὀξυτόνως διὰ τοῦ μ. 
<Ἀριηνούς>] οὕτω καλουμένους πάντας ὁμοίως 
Ἀριηνούς· εἶπε γὰρ ἄνω Ἐρυθραίων Ἀριηνῶν· πάντας 
τοὺς Ἀριηνοὺς καλουμένους ὁμοιῶς· ἀλλ' ὅμως ἀφορ-
μαὶ καὶ ὁδοὶ τοῦ βίου πολλαί εἰσι τοῖς ἀνθρώποις. 



Scholia In Dionysium Periegetam, Scholia in Dionysii periegetae orbis descriptionem (scholia vetera) (olim sub auctore Demetrio Lampsaceno) 
Vita-verse of Orbis descriptio 1139, line of scholion 8

Περὶ τῶν συμβεβηκότων οὖν ὁ βασιλεὺς ἀθυμήσας τὴν 
μὲν ἐνεδρεύσασαν αὐτὸν γραῦν ζῶσαν κατέχωσε, τὴν 
δὲ θυγατέρα σταυρώσας διὰ λυπῆς ὑπερβολὴν ἔρριψεν 
ἑαυτὸν εἰς ποταμὸν Ἰνδὸν, ὃς ἀπ' αὐτοῦ Ὑδάσπης με-
τωνομάσθη· ἔστι δὲ τῆς Ἰνδίας καταφερόμενος εἰς τὸν 
Σαρωνικὴν Ἀρότην. 



Scholia In Dionysium Periegetam, Scholia in Dionysii periegetae orbis descriptionem (scholia vetera) (olim sub auctore Demetrio Lampsaceno) 
Vita-verse of Orbis descriptio append, line of scholion 39

Ὅτι ὁ ὠκεανὸς εἰς τέσσαρα πελάγη μερίζεται· εἰς 
τὸ Ἀτλαντικὸν, Κρόνιον τὸ καὶ νεκρὸν, Ἰνδικὸν Ἐρυ-
θραῖόν τε ἢ Αἰθιοπικόν. 



Scholia In Dionysium Periegetam, Scholia in Dionysii periegetae orbis descriptionem (scholia vetera) (5019: 002)
“Aristarchs Homerische Textkritik nach den Fragmenten des Didymos, vol. 2”, Ed. Ludwich, A.
Leipzig: Teubner, 1885, Repr. 1971.
Vita-scholion 43, line 14

                                                BV. δύο δὲ νοτίους 
ἥττονας ἔχειν μὲν γένεσιν θαλάσσης Ἰνδικῆς, ὀνομάζεσθαι δὲ 
Περσικὸν καὶ Ἀραβικόν. 



Flavius Josephus Hist., Antiquitates Judaicae (0526: 001)
“Flavii Iosephi opera, vols. 1–4”, Ed. Niese, B.
Berlin: Weidmann, 1:1887; 2:1885: 3:1892; 4:1890, Repr. 1955.
Book 1, chapter 38, line 5

                                                             καὶ Φεισὼν 
μέν, σημαίνει δὲ πληθὺν τοὔνομα, ἐπὶ τὴν Ἰνδικὴν φερόμενος ἐκδί-
δωσιν εἰς τὸ πέλαγος ὑφ' Ἑλλήνων Γάγγης λεγόμενος, Εὐφράτης 
δὲ καὶ Τίγρις ἐπὶ τὴν Ἐρυθρὰν ἀπίασι θάλασσαν· καλεῖται δὲ ὁ 
μὲν Εὐφράτης Φοράς, σημαίνει δὲ ἤτοι σκεδασμὸν ἢ ἄνθος, Τίγρις 
δὲ Διγλάθ, ἐξ οὗ φράζεται τὸ μετὰ στενότητος ὀξύ· Γηὼν δὲ διὰ 
τῆς Αἰγύπτου ῥέων δηλοῖ τὸν ἀπὸ τῆς ἐναντίας ἀναδιδόμενον ἡμῖν, 
ὃν δὴ Νεῖλον Ἕλληνες προσαγορεύουσιν. 



Flavius Josephus Hist., Antiquitates Judaicae 
Book 1, chapter 143, line 2

Σημᾷ δὲ τῷ τρίτῳ τῶν Νώχου υἱῶν πέντε γίνονται παῖδες, 
οἳ τὴν μέχρι τοῦ κατ' Ἰνδίαν ὠκεανοῦ κατοικοῦσιν Ἀσίαν ἀπ' Εὐ-
φράτου τὴν ἀρχὴν πεποιημένοι. 



Flavius Josephus Hist., Antiquitates Judaicae 
Book 1, chapter 147, line 4

                                                     οὗτοι ἀπὸ Κωφῆνος 
ποταμοῦ τῆς Ἰνδικῆς καὶ τῆς πρὸς αὐτῇ Σηρίας τινὰ κατοικοῦσι. 



Flavius Josephus Hist., Antiquitates Judaicae 
Book 8, chapter 164, line 4

                                          ἔτυχε δὲ καὶ τῆς ἁρμοζούσης εἰς 
τὰς ναῦς δωρεᾶς παρ' Εἰρώμου τοῦ Τυρίων βασιλέως· ἄνδρας γὰρ 
αὐτῷ κυβερνήτας καὶ τῶν θαλασσίων ἐπιστήμονας ἔπεμψεν ἱκανούς, 
οἷς ἐκέλευσε πλεύσαντας μετὰ τῶν ἰδίων οἰκονόμων εἰς τὴν πάλαι 
μὲν Σώφειραν νῦν δὲ χρυσῆν γῆν καλουμένην, τῆς Ἰνδικῆς ἐστιν 
αὕτη, χρυσὸν αὐτῷ κομίσαι. 



Flavius Josephus Hist., Antiquitates Judaicae 
Book 10, chapter 227, line 1

           καὶ Μεγασθένης δὲ ἐν τῇ τετάρτῃ τῶν Ἰνδικῶν μνημο-
νεύει αὐτῶν δι' ἧς ἀποφαίνειν πειρᾶται τοῦτον τὸν βασιλέα τῇ 
ἀνδρείᾳ καὶ τῷ μεγέθει τῶν πράξεων ὑπερβεβηκότα τὸν Ἡρακλέα· 
καταστρέψασθαι γὰρ αὐτόν φησι Λιβύης τὴν πολλὴν καὶ Ἰβηρίαν. 



Flavius Josephus Hist., Antiquitates Judaicae 
Book 10, chapter 228, line 2

καὶ Διοκλῆς δ' ἐν τῇ δευτέρᾳ τῶν Περσικῶν μνημονεύει τούτου 
τοῦ βασιλέως καὶ Φιλόστρατος ἐν ταῖς Ἰνδικαῖς καὶ Φοινικικαῖς 
ἱστορίαις, ὅτι οὗτος ὁ βασιλεὺς ἐπολιόρκησε τὴν Τύρον ἔτεσι τρισὶ 
καὶ δέκα βασιλεύοντος κατ' ἐκεῖνον τὸν καιρὸν Ἰθωβάλου τῆς Τύ-
ρου. 



Flavius Josephus Hist., Antiquitates Judaicae 
Book 11, chapter 33, line 4

Τῷ δὲ πρώτῳ τῆς βασιλείας ἔτει Δαρεῖος ὑποδέχεται λαμ-
πρῶς καὶ μετὰ πολλῆς παρασκευῆς τούς τε περὶ αὐτὸν καὶ τοὺς 
οἴκοι γεγονότας καὶ τοὺς τῶν Μήδων ἡγεμόνας καὶ σατράπας τῆς 
Περσίδος καὶ τοπάρχας τῆς Ἰνδικῆς ἄχρι τῆς Αἰθιοπίας καὶ τοὺς 
στρατηγοὺς τῶν ἑκατὸν εἰκοσιεπτὰ σατραπειῶν. 



Flavius Josephus Hist., Antiquitates Judaicae 
Book 11, chapter 186, line 2

                                          παραλαβὼν γὰρ τὴν βασιλείαν 
ὁ Ἀρταξέρξης καὶ καταστήσας ἀπὸ Ἰνδίας ἄχρι Αἰθιοπίας τῶν 
σατραπειῶν ἑκατὸν καὶ εἰκοσιεπτὰ οὐσῶν ἄρχοντας, τῷ τρίτῳ τῆς 
βασιλείας ἔτει τούς τε φίλους καὶ τὰ Περσῶν ἔθνη καὶ τοὺς ἡγε-
μόνας αὐτῶν ὑποδεξάμενος ἑστιᾷ πολυτελῶς, οἷον εἰκὸς παρὰ 
βασιλεῖ τοῦ πλούτου παρασκευαζομένῳ τὴν ἐπίδειξιν ποιήσασθαι, 
ἐπὶ ἡμέρας ἑκατὸν ὀγδοήκοντα. 



Flavius Josephus Hist., Antiquitates Judaicae 
Book 11, chapter 216, line 1

       τυχὼν δὲ ὧν ἐπεθύμει Ἀμάνης παραχρῆμα πέμπει διά-
ταγμα ὡς τοῦ βασιλέως εἰς ἅπαντα τὰ ἔθνη περιέχον τοῦτον τὸν 
τρόπον· “βασιλεὺς μέγας Ἀρταξέρξης τοῖς ἀπὸ Ἰνδικῆς ἕως τῆς 
Αἰθιοπίας ἑπτὰ καὶ εἴκοσι καὶ ἑκατὸν σατραπειῶν ἄρχουσι τάδε 
γράφει. 



Flavius Josephus Hist., Antiquitates Judaicae 
Book 11, chapter 272, line 3

                            μεταπεμφθέντας οὖν τοὺς βασιλικοὺς 
γραμματεῖς ἐκέλευσε γράφειν τοῖς ἔθνεσι περὶ τῶν Ἰουδαίων τοῖς 
τε οἰκονόμοις καὶ ἄρχουσιν ἀπὸ Ἰνδικῆς ἕως τῆς Αἰθιοπίας ἑκατὸν 
εἰκοσιεπτὰ σατραπειῶν ἡγουμένοις. 



Flavius Josephus Hist., Antiquitates Judaicae 
Book 13, chapter 251, line 2

                                     μάρτυς δὲ τούτων ἡμῖν ἐστιν 
καὶ Νικόλαος ὁ Δαμασκηνὸς οὕτως ἱστορῶν· “τρόπαιον δὲ στήσας 
Ἀντίοχος ἐπὶ τῷ Λύκῳ ποταμῷ νικήσας Ἰνδάτην τὸν Πάρθων 
στρατηγὸν αὐτόθι ἔμεινεν ἡμέρας δύο δεηθέντος Ὑρκανοῦ τοῦ Ἰου-
δαίου διά τινα ἑορτὴν πάτριον, ἐν ᾗ τοῖς Ἰουδαίοις οὐκ ἦν νόμιμον 
ἐξοδεύειν. 



Flavius Josephus Hist., Contra Apionem (= De Judaeorum vetustate) (0526: 003)
“Flavii Iosephi opera, vol. 5”, Ed. Niese, B.
Berlin: Weidmann, 1889, Repr. 1955.
Book 1, section 144, line 3

                                  περὶ τούτων γοῦν συμφωνεῖ καὶ 
Φιλόστρατος ἐν ταῖς ἱστορίαις μεμνημένος τῆς Τύρου πολιορκίας, 
καὶ Μεγασθένης ἐν τῇ τετάρτῃ τῶν Ἰνδικῶν, δι' ἧς ἀποφαίνειν 
πειρᾶται τὸν προειρημένον βασιλέα τῶν Βαβυλωνίων Ἡρακλέους 
ἀνδρείᾳ καὶ μεγέθει πράξεων διενηνοχέναι· καταστρέψασθαι γὰρ 
αὐτόν φησι καὶ Λιβύης τὴν πολλὴν καὶ Ἰβηρίαν. 



Flavius Josephus Hist., Contra Apionem (= De Judaeorum vetustate) 
Book 1, section 179, line 2

          οὗτοι δέ εἰσιν ἀπόγονοι τῶν ἐν Ἰνδοῖς φιλοσόφων, κα-
λοῦνται δέ, ὥς φασιν, οἱ φιλόσοφοι παρὰ μὲν Ἰνδοῖς Καλανοί, 
παρὰ δὲ Σύροις Ἰουδαῖοι τοὔνομα λαβόντες ἀπὸ τοῦ τόπου· προς-
αγορεύεται γὰρ ὃν κατοικοῦσι τόπον Ἰουδαία. 



Flavius Josephus Hist., De bello Judaico libri vii (0526: 004)
“Flavii Iosephi opera, vol. 6”, Ed. Niese, B.
Berlin: Weidmann, 1895, Repr. 1955.
Book 2, section 385, line 2

                                                 καὶ τί δεῖ πόρρωθεν 
ὑμῖν τὴν Ῥωμαίων ὑποδεικνύναι δύναμιν παρὸν ἐξ Αἰγύπτου τῆς 
γειτνιώσης, ἥτις ἐκτεινομένη μέχρις Αἰθιόπων καὶ τῆς εὐδαίμονος 
Ἀραβίας ὅρμος τε οὖσα τῆς Ἰνδικῆς, πεντήκοντα πρὸς ταῖς 
ἑπτακοσίαις ἔχουσα μυριάδας ἀνθρώπων δίχα τῶν Ἀλεξάνδρειαν 
κατοικούντων, ὡς ἔνεστιν ἐκ τῆς καθ' ἑκάστην κεφαλὴν εἰσφορᾶς 
τεκμήρασθαι, τὴν Ῥωμαίων ἡγεμονίαν οὐκ ἀδοξεῖ, καίτοι πηλίκον 
ἀποστάσεως κέντρον ἔχουσα τὴν Ἀλεξάνδρειαν πλήθους τε ἀνδρῶν 
ἕνεκα καὶ πλούτου πρὸς δὲ μεγέθους· μῆκος μέν γε αὐτῆς τριά-
κοντα σταδίων, εὖρος δ' οὐκ ἔλαττον δέκα, τοῦ δὲ ἐνιαυσιαίου 
παρ' ὑμῶν φόρου καθ' ἕνα μῆνα πλέον Ῥωμαίοις παρέχει καὶ τῶν 
χρημάτων ἔξωθεν τῇ Ῥώμῃ σῖτον μηνῶν τεσσάρων· τετείχισται δὲ 
πάντοθεν ἢ δυσβάτοις ἐρημίαις ἢ θαλάσσαις ἀλιμένοις ἢ

ποτα-



Flavius Josephus Hist., De bello Judaico libri vii 
Book 7, section 351, line 4

                                                         ἔδει μὲν οὖν 
ἡμᾶς οἴκοθεν πεπαιδευμένους ἄλλοις εἶναι παράδειγμα τῆς πρὸς 
θάνατον ἑτοιμότητος· οὐ μὴν ἀλλ' εἰ καὶ τῆς παρὰ τῶν ἀλλοφύ-
λων δεόμεθα πίστεως, βλέψωμεν εἰς Ἰνδοὺς τοὺς σοφίαν ἀσκεῖν 
ὑπισχνουμένους. 



Flavius Josephus Hist., De bello Judaico libri vii 
Book 7, section 357, line 1

              ἆρ' οὖν οὐκ αἰδούμεθα χεῖρον Ἰνδῶν φρονοῦντες καὶ 
διὰ τῆς αὑτῶν ἀτολμίας τοὺς πατρίους νόμους, οἳ πᾶσιν ἀνθρώ-
ποις εἰς ζῆλον ἥκουσιν, αἰσχρῶς ὑβρίζοντες; 



Zosimus Hist., Historia nova (4084: 001)
“Zosime. Histoire nouvelle, vols. 1–3.2”, Ed. Paschoud, F.
Paris: Les Belles Lettres, 1:1971; 2.1–2.2:1979; 3.1:1986; 3.2:1989.
Book 1, chapter 5, section 1, line 2

Ἐπεὶ δὲ Δαρεῖον μὲν Βῆσος ἀνεῖλεν, Ἀλέξαν-
δρος δὲ μετὰ τὰς ἐν Ἰνδοῖς πράξεις ἐπανελθὼν εἰς 
Βαβυλῶνα τοῦ βίου μετέστη, τότε δὴ τῆς Μακεδόνων 
ἀρχῆς εἰς σατραπείας διαιρεθείσης, οὕτω τε τοῖς πρὸς 
ἀλλήλους συνεχέσι πολέμοις ἐλαττωθείσης, ἡ τύχη 
Ῥωμαίοις τὰ [τε] λειπόμενα τῆς Εὐρώπης πεποίηκεν 
ὑποχείρια. 



Hephaestion Astrol., Apotelesmatica (2043: 001)
“Hephaestionis Thebani apotelesmaticorum libri tres, vol. 1”, Ed. Pingree, D.
Leipzig: Teubner, 1973.
Page 9, line 4

                                                κατὰ δὲ μέρος   
τῷ μὲν βορείῳ Διδύμων ὑπόκειται κατὰ τοὺς πόδας Βοιω-
τία, παρὰ τὴν χεῖρα Θρᾴκη, ὑπὸ τὸν νῶτον Γαλατία, τοῦ 
δὲ νοτίου ὑπὸ τὸν γλουτὸν Πόντος, κατὰ νῶτον Κιλικία, 
κατὰ τὴν ὠμοπλάτην Φοινίκη, κατὰ τὴν κορυφὴν Ἰνδική. 



Hephaestion Astrol., Apotelesmatica 
Page 24, line 22

κατὰ δὲ Πτολεμαῖον Ἰνδική, Ἀριανή, Γεδρωσία, Θρᾴκη, 
Μακεδονία, Ἰλλυρίς. 



Hephaestion Astrol., Apotelesmatica 
Page 29, line 9

                                κατὰ μὲν τὸν Ὠδαψὸν τὰ 
ἐμπρόσθια Εὐφρατησία καὶ Τίγρις, καὶ τὰ μέσα Συρία καὶ 
Ἐρυθρὰ θάλασσα, Ἰνδική, μέση Περσίς, καὶ ὑπὸ τὸν 
νῶτον Ἀραβικὴ θάλασσα καὶ Βορυσθένης, κατὰ δὲ τὸν 
σύνδεσμον τοῦ βορείου Θρᾴκη, τοῦ νοτίου Ἀσία καὶ 
Σαρδώ. 



Hephaestion Astrol., Apotelesmatica 
Page 56, line 5

                                           τὴν δὲ Σελήνην τὸν 
δυναστεύοντα Συρίας πρὸς ἄλλον δυνάστην συγκρούσειν   
καί τινα μέγαν ἄνδρα ἀπολέσθαι, ὑπό τε τῶν ὄχλων 
προδοθήσεσθαι τὸν ἡγούμενον καὶ τόπους ἐπιφανεῖς 
ἀφανισθήσεσθαι ὑπὸ σεισμῶν καὶ ἀνθρώπους ἐνδόξους 
ἀναιρεθήσεσθαι, ἐν δὲ ταῖς <β> τριώροις ταῖς τελευταίαις 
Βαβυλῶνι καὶ Αἰθιοπίᾳ φθορὰν ἔσεσθαι, Ἰνδοῖς δὲ 
εὐστάθειαν, ἀφανισμὸν δὲ τοῖς ἁπανταχοῦ ζῴοις. 



Hephaestion Astrol., Apotelesmatica 
Page 60, line 12

Ἡλίου δὲ ἐν Αἰγοκέρωτι περὶ τὴν πρώτην τρίωρον 
ἐκλείποντος τοῖς πρὸς νότον οἰκοῦσι κακὰ σημαίνει, 
περὶ δὲ δευτέραν τρίωρον κακὰ Ἐλυμαίᾳ, Περσίδι, 
Μηδίᾳ, Γερμανίᾳ, Ἰνδίᾳ δηλοῖ καὶ τοῖς πρὸς ἀνατολὴν καὶ 
ἀπηλιώτην οἰκοῦσιν, ἐν δὲ τῇ τρίτῃ τριώρῳ τοῖς ἐν τῷ 
Πόντῳ πόλεμον καὶ τοῖς ἐν τῇ Ἀσίᾳ καὶ Κύπρῳ καὶ 
τοῖς πρὸς νότον οἰκοῦσι σημαίνει, ἔτι δὲ νόσους καὶ 
φθορὰς καρπῶν καὶ φυτῶν καὶ γῆς, ἐν δὲ τῇ τελευταίᾳ 
τριώρῳ τετραπόδων διαφθορὰν τῶν ἐν τῇ δύσει. 



Hephaestion Astrol., Apotelesmatica 
Page 64, line 5

           ἐὰν δὲ ἐν Καρκίνῳ αἱματώδης γένηται, ταράξει 
τὴν Ἰνδῶν καὶ Σύρων καὶ Αἰγυπτίων βασιλείαν. 



Hephaestion Astrol., Apotelesmatica 
Page 222, line 8

                                                     ἐν δὲ τοῖς τοῦ 
<Ἄρεως ὀμμάτων ἀσθενείας ποιεῖ, μ>άλιστα ἐπὶ ἡμέρας, 
καὶ αὐ<ξίφως οὖσα πυρετούς, αἱμαγμούς, κ]2ινδύνους, 
στομάχου πόνον <καὶ ἀνασκευὰς ποιεῖ, νυκτὸς> δὲ ἐν 
πᾶσιν ἀγαθὴ τοῖς ἐγ<χειριζομένοις, καὶ ὀξέως τὰς> πράξεις 
διαλύσει, καὶ <δοκοῦντες βλάπτεσθαι ἀπροσδο>κήτως 
ὠφελοῦνται. 



Hephaestion Astrol., Apotelesmatica (epitomae quattuor) (2043: 002)
“Hephaestionis Thebani apotelesmaticorum libri tres, vol. 2”, Ed. Pingree, D.
Leipzig: Teubner, 1974.
Page 128, line 17

                                   τῆς δὲ Σελήνης ὥρᾳ πρώτῃ, 
δευτέρᾳ καὶ τρίτῃ ἐκλειπούσης τὸν δυναστεύοντα τῆς 
Συρίας πρὸς ἄλλον δυνάστην συγκρούσειν καί τινα μέγαν 
ἄνδρα ἀπολέσθαι, ὑπό τε <τῶν ὄχλων> προδοθήσεσθαι 
τὸν ἡγούμενον καὶ τόπους ἐπιφανεῖς ἀφανισθήσεσθαι ὑπὸ 
σεισμῶν τε καὶ ἀστραπῶν καὶ ἀνθρώπους ἐνδόξους ἀν-
αιρεθήσεσθαι, ἐν δὲ τῇ βʹ τριώρῳ ἤτοι δʹ, εʹ, ϛʹ τῆς τε 
Βαβυλῶνος καὶ Αἰθιοπίας φθορὰν ἔσεσθαι, Ἰνδοῖς δὲ καὶ 
ἀφανισμὸν τοῖς ζῴοις, ζʹ δὲ καὶ ηʹ καὶ θʹ ἐν Περσίδι φθο-
ρὰν σημαίνει, ιʹ καὶ ιαʹ καὶ ιβʹ ὁμοίως ἐν Βαβυλῶνι φθορὰν 
ἔσεσθαι ἀνθρώπων καὶ προβάτων. 



Hephaestion Astrol., Apotelesmatica (epitomae quattuor) 
Page 132, line 10

Τοῦ Ἡλίου δὲ ἐν Αἰγοκέρωτι περὶ τὴν πρώτην τρίωρον 
ἐκλείποντος τοῖς πλησίον νότου οἰκοῦσι κακὰ σημαίνει, 
περὶ δὲ δευτέραν τρίωρον κακὰ Ἐλυμαίᾳ, Περσίδι, Μηδίᾳ, 
Γερμανίᾳ, Ἰνδίᾳ δηλοῖ καὶ τοῖς πρὸς ἀνατολὰς καὶ ἀπηλιώ-
την οἰκοῦσιν, ἐν δὲ τῇ τρίτῃ τριώρῳ τοῖς ἐν τῷ Πόντῳ 
πόλεμον καὶ τοῖς ἐν τῇ Ἀσίᾳ καὶ Κύπρῳ καὶ τοῖς πρὸς νότον 
οἰκοῦσι σημαίνει; 



Hephaestion Astrol., Apotelesmatica (epitomae quattuor) 
Page 140, line 13

                                                 ὡς δὲ κατὰ 
μέρος οἱ παλαιοὶ διωρίσαντο, τῷ μὲν βορείῳ Διδύμῳ κατὰ 
τοὺς πόδας ὑπόκειται Βοιωτία, τῇ χειρὶ Θρᾴκη, ὑπὸ τὸν 
νῶτον Γαλατία, τῷ δὲ νοτίῳ ὑπὸ τὸν γλουτὸν Πόντος, 
κατὰ δὲ τὸν νῶτον Κιλικία, κατὰ τὸν ὠμοπλάτην Φοι-
νίκη, κατὰ τὴν κορυφὴν Ἰνδική. 



Hephaestion Astrol., Apotelesmatica (epitomae quattuor) 
Page 154, line 10

Προσῳκείωνται δὲ αὐτῷ χῶραι κατὰ μὲν τοὺς παλαιοὺς 
Κιμμέριοί τε καὶ ἡ πανέρημος, κατὰ δὲ τὸν Πτολεμαῖον 
Ἰνδική, Ἀρ[ρ]ιανή, Γεδρωσία, Θρᾴκη, Μακεδονία, Ἰλλυρίς. 



Hephaestion Astrol., Apotelesmatica (epitomae quattuor) 
Page 158, line 4

ὡς δὲ κατὰ μέρος οἱ ἀρχαιότεροι διεῖλον τοῖς μὲν ἐμπρος-
θίοις Εὐφράτης καὶ Τίγρις, τοῖς δὲ μέσοις Συρία, Ἐρυ-
θρὰ θάλασσα, Ἰνδική, μέση Περσίς, τοῖς δ' ὑπὸ τὸν νῶτον 
Ἀραβικὴ θάλασσα καὶ Βορυσθένης, τῷ συνδέσμῳ τοῦ 
βορείου Ἰχθύος Θρᾴκη, τῷ δὲ τοῦ νοτίου Ἀσία καὶ Σαρδώ. 



Hephaestion Astrol., Apotelesmatica (epitomae quattuor) 
Page 171, line 26

                                   τῆς δὲ Σελήνης τὸν δυνα-
στεύοντα τῆς Συρίας πρὸς ἄλλον δυνάστην συγκρούσειν καί 
τινα μέγαν ἄνδρα ἀπολέσθαι, ὑπό τε τῶν ὄχλων προδο-
θήσεσθαι τὸν ἡγούμενον καὶ τόπους ἐπιφανεῖς ἀφανις-
θήσεσθαι, ἐν δὲ ταῖς δευτέραις τριώροις καὶ ταῖς τελευταί-
αις Βαβυλῶνι καὶ Αἰθιοπίᾳ φθορὰν ἔσεσθαι, Ἰνδοῖς δὲ εὐ-
στάθειαν καὶ ἀφανισμὸν τοῖς ἁπανταχοῦ ζῴοις. 



Hephaestion Astrol., Apotelesmatica (epitomae quattuor) 
Page 175, line 2

Ἐν δὲ Αἰγοκέρωτι Ἡλίου ἐκλείποντος κατὰ τὴν αʹ τρί-
ωρον τοῖς πρὸς νότον οἰκοῦσι κακὰ σημαίνει, περὶ δὲ τὴν   
βʹ τρίωρον τοῖς Ἐλυμαίοις, τῇ Περσίδι, τῇ Μηδικῇ, τῇ 
Γερμανίᾳ, τῇ Ἰνδίᾳ καὶ τοῖς πρὸς ἀνατολὴν καὶ ἀπηλι-
ώτην οἰκοῦσιν, κατὰ δὲ τὴν γʹ τρίωρον τοῖς ἐν τῷ Πόντῳ 
πόλεμον καὶ τοῖς ἐν τῇ Ἀσίᾳ καὶ Κύπρῳ καὶ πρὸς νότον 
οἰκοῦσιν, ἔτι δὲ νόσους καὶ φθορὰς καρπῶν καὶ φυτῶν 
γῆς, κατὰ δὲ τὴν τελευταίαν τρίωρον τετραπόδων δια-
φθορὰν τῶν ἐν τῇ δύσει. 



Hephaestion Astrol., Apotelesmatica (epitomae quattuor) 
Page 178, line 4

       ἐν Καρκίνῳ δὲ αἱματώδης γενόμενος ταράξει τὴν 
Ἰνδῶν καὶ Σύρων καὶ Αἰγυπτίων βασιλείαν. 



Hephaestion Astrol., Excerptum (e cod. Paris. gr. 2506) (2043: 003)
“Hephaestionis Thebani apotelesmaticorum libri tres, vol. 2”, Ed. Pingree, D.
Leipzig: Teubner, 1974.
Page vii, line 12

      καὶ ἐπεὶ ὁ Ἥλιος ὁ τὸ πατρικὸν ἐπέχων πρόσωπον 
πρὸς τὴν τοῦ Διὸς καὶ Ἑρμοῦ ἐπιμαρτυρίαν καὶ κόλλησιν 
φέρεται ἔστιν εὐτυχίας ἐλπίς· μετὰ γὰρ ἔτη <ιγ> πλήρη 
καὶ ἡμέρας <κδ> ὁ τοῦ Διὸς ἐπέρχεται τῷ ἡλιακῷ 
τόπῳ, καὶ ἔστιν ὁ καιρὸς τῆς ἐπερχομένης ἰνδικτιῶνος 
γʹ περὶ τὸν Ὀκτώβριον· εἴτε γὰρ εἰς κριτικὸν ἀξίωμα ἢ 
χαρτουλαρικὸν ἢ βασιλικὸν νοτάριον ἢ εἰς ἀναστροφὰς 
δημοσίων πραγμάτων ἤ τι τοιοῦτον προβληθείη. 



Hephaestion Astrol., Excerptum (e cod. Vat. gr. 1056) (2043: 005)
“Hephaestionis Thebani apotelesmaticorum libri tres, vol. 2”, Ed. Pingree, D.
Leipzig: Teubner, 1974.
Page xxii, line 2

                                                  ..>   
Ὑπόδειγμα ἕτερον


 Μηνὶ Ἰουνίῳ αʹ ἰνδ. εʹ ἦλθεν ἀγγελία ἀπὸ τῆς δύσεως 
ὡς τοῦ μεγίστου κάστρου τῶν Σερβίων παρὰ τοῦ βασιλέως 
κρατηθέντος. 



Antiphon Orat., Fragmenta (0028: 008)
“Antiphontis orationes et fragmenta”, Ed. Thalheim, T. (post F. Blass)
Leipzig: Teubner, 1914, Repr. 1982.
Fragment 25-33t, line 1

                                                   Harp. 
ΠΕΡΙ ΤΟΥ ΛΙΝΔ*ιΩΝ ΦΟΡΟΥ


<Ἀμφίπολις>· Ἀ. 



Pausanias Perieg., Graeciae descriptio (0525: 001)
“Pausaniae Graeciae descriptio, 3 vols.”, Ed. Spiro, F.
Leipzig: Teubner, 1903, Repr. 1:1967.
Book 1, chapter 12, section 3, line 5

                           ἐλέφαντας δὲ πρῶτος μὲν 
τῶν ἐκ τῆς Εὐρώπης Ἀλέξανδρος ἐκτήσατο Πῶρον καὶ 
τὴν δύναμιν καθελὼν τὴν Ἰνδῶν, ἀποθανόντος δὲ 
Ἀλεξάνδρου καὶ ἄλλοι τῶν βασιλέων καὶ πλείστους 
ἔσχεν Ἀντίγονος, Πύρρῳ δὲ ἐκ τῆς μάχης ἐγεγόνει 
τῆς πρὸς Δημήτριον τὰ θηρία αἰχμάλωτα· τότε δὲ 
ἐπιφανέντων αὐτῶν δεῖμα ἔλαβε Ῥωμαίους ἄλλο τι 
καὶ οὐ ζῷα εἶναι νομίσαντας. 



Pausanias Perieg., Graeciae descriptio 
Book 1, chapter 12, section 4, line 5

                                    ἐλέφαντα γάρ, ὅσος μὲν 
ἐς ἔργα καὶ ἀνδρῶν χεῖρας, εἰσὶν ἐκ παλαιοῦ δῆλοι 
πάντες εἰδότες· αὐτὰ δὲ τὰ θηρία, πρὶν ἢ διαβῆναι 
Μακεδόνας ἐπὶ τὴν Ἀσίαν, οὐδὲ ἑωράκεσαν ἀρχὴν πλὴν 
Ἰνδῶν τε αὐτῶν καὶ Λιβύων καὶ ὅσοι πλησιόχωροι 
τούτοις. 



Pausanias Perieg., Graeciae descriptio 
Book 2, chapter 28, section 1, line 8

                                 τὸ δὲ αὐτὸ εὑρίσκω καὶ 
ἄλλαις χώραις συμβεβηκός· Λιβύη μέν γε μόνη κρο-
κοδείλους τρέφει χερσαίους διπήχεων οὐκ ἐλάσσονας, 
παρὰ δὲ Ἰνδῶν μόνων ἄλλα τε κομίζεται καὶ ὄρνιθες 
οἱ ψιττακοί. 



Pausanias Perieg., Graeciae descriptio 
Book 2, chapter 28, section 1, line 11

               τοὺς δὲ ὄφεις οἱ Ἐπιδαύριοι τοὺς μεγά-  
λους ἐς πλέον πηχῶν καὶ τριάκοντα προήκοντας, οἷοι 
παρά τε Ἰνδοῖς τρέφονται καὶ ἐν Λιβύῃ, ἄλλο δή 
τι γένος φασὶν εἶναι καὶ οὐ δράκοντας. 



Pausanias Perieg., Graeciae descriptio 
Book 3, chapter 12, section 4, line 2

ἀργύρου γὰρ οὐκ ἦν πω τότε οὐδὲ χρυσοῦ νόμισμα, 
κατὰ τρόπον δὲ ἔτι τὸν ἀρχαῖον ἀντεδίδοσαν βοῦς καὶ 
ἀνδράποδα καὶ ἀργὸν τὸν ἄργυρον καὶ χρυσόν· οἱ δὲ 
ἐς τὴν Ἰνδικὴν ἐσπλέοντες φορτίων φασὶν Ἑλληνικῶν 
τοὺς Ἰνδοὺς ἀγώγιμα ἄλλα ἀνταλλάσσεσθαι, νόμισμα 
δὲ οὐκ ἐπίστασθαι, καὶ ταῦτα χρυσοῦ τε ἀφθόνου καὶ 
χαλκοῦ παρόντος σφίσι. 



Pausanias Perieg., Graeciae descriptio 
Book 4, chapter 32, section 4, line 5

          ἐγὼ δὲ Χαλδαίους καὶ Ἰνδῶν τοὺς μάγους 
πρώτους οἶδα εἰπόντας ὡς ἀθάνατός ἐστιν ἀνθρώπου 
ψυχή, καί σφισι καὶ Ἑλλήνων ἄλλοι τε ἐπείσθησαν 
καὶ οὐχ ἥκιστα Πλάτων ὁ Ἀρίστωνος· εἰ δὲ ἀποδέχε-
σθαι καὶ οἱ πάντες ἐθελήσουσιν, ἐκεῖνό γε ἀντειπεῖν 
οὐκ ἔνεστι μὴ οὐ τὸν πάντα αἰῶνα Ἀριστομένει τὸ 
μῖσος τὸ ἐς Λακεδαιμονίους ἐνεστάχθαι. 



Pausanias Perieg., Graeciae descriptio 
Book 4, chapter 34, section 2, line 7

                                      θηρία δὲ ἐς ὄλε-
θρον ἀνθρώπων οὐ πεφύκασιν οἱ Ἑλλήνων ποταμοὶ 
φέρειν, καθάπερ γε Ἰνδὸς καὶ Νεῖλος ὁ Αἰγύπτιος, 
ἔτι δὲ Ῥῆνος καὶ Ἴστρος Εὐφράτης τε καὶ Φᾶσις· οὗ-
τοι γὰρ δὴ θηρία ὅμοια τοῖς μάλιστα ἀνδροφάγα αὔ-
ξουσι, ταῖς ἐν Ἕρμῳ καὶ Μαιάνδρῳ γλάνισιν ἐοικότα 
ἰδέας πλὴν χρόας τε μελαντέρας καὶ ἀλκῆς· ταῦτα δὲ 
αἱ γλάνεις ἀποδέουσιν. 



Pausanias Perieg., Graeciae descriptio 
Book 4, chapter 34, section 3, line 1

                           ὁ δὲ Ἰνδὸς καὶ ὁ Νεῖλος κρο-
κοδείλους μὲν ἀμφότεροι, Νεῖλος δὲ παρέχεται καὶ 
ἵππους, οὐκ ἔλασσον ἢ ὁ κροκόδειλος κακὸν ἀνθρώ-
ποις. 



Pausanias Perieg., Graeciae descriptio 
Book 5, chapter 12, section 3, line 13

                                                   ὁ μὲν 
δὴ ἐλέφας παρὰ τὰ λοιπὰ ζῷα διάφορον καὶ τὴν ἔκ-
φυσιν παρέχεται τῶν κεράτων, ὥσπερ γε καὶ τὸ μέγε-
θός ἐστιν αὐτῷ καὶ εἶδος οὐδὲν ἐοικότα ἑτέρῳ θηρίῳ· 
φιλότιμοι δὲ ἐς τὰ μάλιστά μοι καὶ ἐς θεῶν τιμὴν οὐ 
φειδωλοὶ χρημάτων γενέσθαι δοκοῦσιν οἱ Ἕλληνες, 
οἷς γε παρὰ Ἰνδῶν ἤγετο καὶ ἐξ Αἰθιοπίας ἐλέφας ἐς 
ποίησιν ἀγαλμάτων. 



Pausanias Perieg., Graeciae descriptio 
Book 6, chapter 26, section 9, line 9

                      οὗτοι μὲν δὴ τοῦ Αἰθιόπων γένους 
αὐτοί τέ εἰσιν οἱ Σῆρες καὶ ὅσοι τὰς προσεχεῖς αὐτῇ 
νέμονται νήσους, Ἄβασαν καὶ Σακαίαν· οἱ δὲ αὐτοὺς 
οὐκ Αἰθίοπας, Σκύθας δὲ ἀναμεμιγμένους Ἰνδοῖς φα-
σὶν εἶναι. 



Pausanias Perieg., Graeciae descriptio 
Book 8, chapter 23, section 9, line 5

ἄγει μὲν δὴ ὁ Σόρων τὴν ἐπὶ Ψωφῖδος· θηρία δὲ 
οὗτός τε καὶ ὅσοι δρυμοὶ τοῖς Ἀρκάσιν εἰσὶν ἄλλοι 
παρέχονται τοσάδε, ἀγρίους ὗς καὶ ἄρκτους καὶ χελώ-
νας μεγίστας μεγέθει· λύρας ἂν ποιήσαιο ἐξ αὐτῶν 
χελώνης Ἰνδικῆς λύρᾳ παρισουμένας. 



Pausanias Perieg., Graeciae descriptio 
Book 8, chapter 29, section 4, line 7

       τοῦτον τὸν νεκρὸν <ὁ> ἐν Κλάρῳ [ὁ] θεός, 
ἀφικομένων ἐπὶ τὸ χρηστήριον τῶν Σύρων, εἶπεν Ὀρόν-
την εἶναι, γένους δὲ αὐτὸν εἶναι τοῦ Ἰνδῶν. 



Pausanias Perieg., Graeciae descriptio 
Book 8, chapter 29, section 4, line 11

                                                        εἰ δὲ 
τὴν γῆν τὸ ἀρχαῖον οὖσαν ὑγρὰν ἔτι καὶ ἀνάπλεων 
νοτίδος θερμαίνων ὁ ἥλιος τοὺς πρώτους ἐποίησεν 
ἀνθρώπους, ποίαν εἰκός ἐστιν ἄλλην χώραν ἢ προ-
τέραν τῆς Ἰνδῶν ἢ μείζονας ἀνεῖναι τοὺς ἀνθρώπους, 
ἥ γε καὶ ἐς ἡμᾶς ἔτι καὶ ὄψεως τῷ παραλόγῳ καὶ 
μεγέθει διάφορα ἐκτρέφει θηρία; 



Pausanias Perieg., Graeciae descriptio 
Book 9, chapter 21, section 3, line 1

                         εἶδον δὲ καὶ ταύρους τούς τε 
Αἰθιοπικούς, οὓς ἐπὶ τῷ συμβεβηκότι ὀνομάζουσι ῥινό-
κερως, ὅτι σφίσιν ἐπ' ἄκρᾳ τῇ ῥινὶ ἓν ἑκάστῳ κέρας 
καὶ ἄλλο ὑπὲρ αὐτὸ οὐ μέγα, ἐπὶ δὲ τῆς κεφαλῆς οὐδὲ 
ἀρχὴν κέρατά ἐστι, καὶ τοὺς ἐκ Παιόνων – οὗτοι δὲ 
οἱ ἐκ Παιόνων ἔς τε τὸ ἄλλο σῶμα δασεῖς καὶ ἀμφὶ 
τὸ στέρνον μάλιστά εἰσι καὶ τὴν γένυν – καμήλους 
τε Ἰνδικὰς χρῶμα εἰκασμένας παρδάλεσιν. 



Pausanias Perieg., Graeciae descriptio 
Book 9, chapter 21, section 4, line 2

                                             θηρίον δὲ 
<τὸ> ἐν τῷ Κτησίου λόγῳ τῷ ἐς Ἰνδοὺς – μαρτιχόρα   
ὑπὸ τῶν Ἰνδῶν, ὑπὸ δὲ Ἑλλήνων φησὶν ἀνδροφάγον 
λελέχθαι – εἶναι πείθομαι τὸν τίγριν· ὀδόντας δὲ 
αὐτὸ τριστοίχους καθ' ἑκατέραν τὴν γένυν καὶ κέντρα 
ἐπὶ ἄκρας ἔχειν τῆς οὐρᾶς, τούτοις δὲ τοῖς κέντροις 
ἐγγύθεν ἀμύνεσθαι καὶ ἀποπέμπειν ἐς τοὺς πορρωτέρω 
τοξότου ἀνδρὸς ὀιστῷ ἴσον, ταύτην οὐκ ἀληθῆ τὴν 
φήμην οἱ Ἰνδοὶ δέξασθαι δοκοῦσί μοι παρ' ἀλλήλων 
ὑπὸ τοῦ ἄγαν ἐς τὸ θηρίον δείματος. 



Pausanias Perieg., Graeciae descriptio 
Book 9, chapter 21, section 5, line 7

                                 δοκῶ δέ, εἰ καὶ Λιβύης 
τις ἢ τῆς Ἰνδῶν ἢ Ἀράβων γῆς ἐπέρχοιτο τὰ ἔσχατα 
ἐθέλων θηρία ὁπόσα παρ' Ἕλλησιν ἐξευρεῖν, τὰ μὲν 
οὐδὲ ἀρχὴν αὐτὸν εὑρήσειν, τὰ δὲ οὐ κατὰ ταὐτὰ ἔχειν 
φανεῖσθαί οἱ· οὐ γὰρ δὴ ἄνθρωπος μόνον ὁμοῦ τῷ 
ἀέρι καὶ τῇ γῇ διαφόροις οὖσι διάφορον κτᾶται καὶ 
τὸ εἶδος, ἀλλὰ καὶ τὰ λοιπὰ τὸ αὐτὸ ἂν πάσχοι τοῦτο, 
ἐπεὶ καὶ τὰ θηρία αἱ ἀσπίδες τοῦτο μὲν ἔχουσιν αἱ 
Λίβυσσαι παρὰ τὰς Αἰγυπτίας τὴν χρόαν, τοῦτο δὲ ἐν 
Αἰθιοπίᾳ μελαίνας τὰς ἀσπίδας οὐ μεῖον ἢ καὶ τοὺς 
ἀνθρώπους ἡ γῆ τρέφει. 



Pausanias Perieg., Graeciae descriptio 
Book 9, chapter 40, section 9, line 9

             μαρτυρεῖ δὲ τῷ λόγῳ καὶ Ἀλέξανδρος, οὐκ 
ἀναστήσας οὔτε ἐπὶ Δαρείῳ τρόπαια οὔτε ἐπὶ ταῖς 
Ἰνδικαῖς νίκαις. 



Pausanias Perieg., Graeciae descriptio 
Book 10, chapter 29, section 4, line 5

                                                     τὴν 
δὲ Ἀριάδνην ἢ κατά τινα ἐπιτυχὼν δαίμονα ἢ καὶ 
ἐπίτηδες αὐτὴν λοχήσας ἀφείλετο Θησέα ἐπιπλεύσας 
Διόνυσος στόλῳ μείζονι, οὐκ ἄλλος κατὰ ἐμὴν δόξαν, 
ἀλλὰ ὁ πρῶτος μὲν ἐλάσας ἐπὶ Ἰνδοὺς στρατείᾳ, πρῶ-
τος δὲ Εὐφράτην γεφυρώσας ποταμόν· Ζεῦγμά τε ὠνο-
μάσθη πόλις καθ' ὅ τι ἐζεύχθη τῆς χώρας ὁ Εὐφρά-
της, καὶ ἔστιν ἐνταῦθα ὁ κάλως καὶ ἐς ἡμᾶς ἐν ᾧ τὸν 
ποταμὸν ἔζευξεν, ἀμπελίνοις ὁμοῦ πεπλεγμένος καὶ 
κισσοῦ κλήμασι. 



Alexander Med., Therapeutica (0744: 003)
“Alexander von Tralles, vols. 1–2”, Ed. Puschmann, T.
Vienna: Braumüller, 1:1878; 2:1879, Repr. 1963.
Volume 2, page 19, line 12

                                     δʹ 
νάρδου Ἰνδικῆς . 



Alexander Med., Therapeutica 
Volume 2, page 21, line 17

                                       ηʹ 
λυκίου Ἰνδικοῦ . 



Alexander Med., Therapeutica 
Volume 2, page 41, line 3

                                                   ηʹ 
λυκίου Ἰνδικοῦ . 



Alexander Med., Therapeutica 
Volume 2, page 63, line 26

                                                   βʹ 
νάρδου Ἰνδικῆς . 



Alexander Med., Therapeutica 
Volume 2, page 91, line 14

                                          εʹ 
λυκίου Ἰνδικοῦ . 



Alexander Med., Therapeutica 
Volume 2, page 223, line 26

                                    ϛʹ 
νάρδου Ἰνδικῆς . 



Alexander Med., Therapeutica 
Volume 2, page 301, line 19

                                             κεʹ 
νάρδου Ἰνδικῆς . 



Alexander Med., Therapeutica 
Volume 2, page 543, line 9

                                     καλοῦσι δ' αὐτό τινες Ἰνδὸν, ἄλλοι δὲ Ἀσκληπιόν.


 Ξηρίον μετασυγκριτικὸν πώρων καὶ πρὸς τοὺς οἰδηματώδεις ὄγκους 
μετ' ὄξους ἐπιπλαττόμενον καλῶς ποιεῖ καὶ τοῖς κεφαλὰς ὑγρὰς ἔχουσι 
καὶ θώρακας καὶ λέπρας καὶ ἀλφοὺς καὶ δυσπνοίας καὶ ὅσα ἄλλα πάθη 
ἐκ τῆς κεφαλῆς ἢ ἐκ τοῦ θώρακος ἔχει τὴν αἰτίαν, ἀποκρουστικὸν ὑπάρχον 
βοήθημα πάσης ὕλης κακῆς καὶ πληθωρικῆς φοινίσσον σφόδρα καὶ διαφοροῦν 
τοὺς καχεκτικοὺς καὶ ἐμπεπηγότας χυμοὺς καὶ τοὺς θηριώδεις, ἀρθριτικοῖς 
καὶ στομαχικοῖς ἁρμόζον καὶ τὸν διὰ παντὸς αὐτῷ χρώμενον οὐ συγχωροῦν 
ποδαγριᾶν ἢ ἰσχίου πεῖραν ὀδύνης λαβεῖν. 



Alexander Med., Therapeutica 
Volume 2, page 543, line 21

     
Γραφὴ τοῦ Ἰνδοῦ.


 Ἁλῶν Καππαδοκικῶν, ἁλῶν κοινῶν, ἁλῶν πικρῶν, ἁλῶν νιτροπηγικῶν, 
ἁλῶν Τραγασαίων, νίτρου Ἀλεξανδρινοῦ καὶ ἀφρονίτρου, κισσήρεως, 
ἀδάρκης, ἀνὰ λιτ. 



Claudius Aelianus Soph., De natura animalium (0545: 001)
“Claudii Aeliani de natura animalium libri xvii, varia historia, epistolae, fragmenta, vol. 1”, Ed. Hercher, R.
Leipzig: Teubner, 1864, Repr. 1971.
Book 2, section 11, line 15

                      εἰ μὲν οὖν ἔμελλον τὴν ἐν Ἰν-
δοῖς αὐτῶν εὐπείθειαν καὶ εὐμάθειαν ἢ τὴν ἐν Αἰ-
θιοπίᾳ ἢ τὴν ἐν Λιβύῃ γράφειν, ἴσως ἄν τῳ καὶ 
μῦθον ἐδόκουν τινὰ συμπλάσας κομπάζειν, εἶτα ἐπὶ 
φήμῃ τοῦ θηρίου τῆς φύσεως καταψεύδεσθαι· ὅπερ 
ἐχρῆν δρᾶν φιλοσοφοῦντα ἄνδρα ἥκιστα καὶ ἀλη-
θείας ἐραστὴν διάπυρον. 



Claudius Aelianus Soph., De natura animalium 
Book 2, section 34, line 1

Εἰ σαφῆ ταῦτα καὶ μὴ ἀμφίλογα, Ἰνδῶν λόγοι 
πειθέτωσαν· ἃ δὲ νῦν ἐρῶ, τῆς ἐκεῖθεν φήμης δια-
κομιζούσης, ταῦτά ἐστιν. 



Claudius Aelianus Soph., De natura animalium 
Book 2, section 34, line 6

                 καὶ τὸν μὲν ὄρνιν κομίζειν τὸ φερώ-
νυμον τοῦτο δὴ φυτὸν ἐς Ἰνδούς, εἰδέναι δὲ ἄρα 
τοὺς ἀνθρώπους ὅπου τε καὶ ὅπως φύεται οὐδὲ ἕν. 



Claudius Aelianus Soph., De natura animalium 
Book 3, section 3, line 2

                                                    ὗν   
οὔτε ἄγριον οὔτε ἥμερον ἐν Ἰνδοῖς γίνεσθαι λέγει 
Κτησίας, πρόβατα δὲ τὰ ἐκείνων οὐρὰς πήχεως ἔχειν 
τὸ πλάτος πού φησιν. 



Claudius Aelianus Soph., De natura animalium 
Book 3, section 4, line 1

Οἱ μύρμηκες οἱ Ἰνδικοὶ οἱ τὸν χρυσὸν φυλάττον-
τες οὐκ ἂν διέλθοιεν τὸν καλούμενον Καμπύλινον 
ποταμόν. 



Claudius Aelianus Soph., De natura animalium 
Book 3, section 34, line 1

Πτολεμαίῳ τῷ δευτέρῳ φασὶν ἐξ Ἰνδῶν κέρας 
ἐκομίσθη, καὶ τρεῖς ἀμφορέας ἐχώρησεν. 



Claudius Aelianus Soph., De natura animalium 
Book 3, section 41, line 1

Ἵππους μονόκερως γῆ Ἰνδικὴ τίκτει, φασί, καὶ 
ὄνους μονόκερως ἡ αὐτὴ τρέφει, καὶ γίνεταί γε ἐκ 
τῶν κεράτων τῶνδε ἐκπώματα. 



Claudius Aelianus Soph., De natura animalium 
Book 3, section 46, line 2

Ἐλέφαντος πωλίῳ περιτυγχάνει λευκῷ πωλευτὴς 
Ἰνδός, καὶ παραλαβὼν ἔτρεφεν ἔτι νεαρόν, καὶ κατὰ 
μικρὰ ἀπέφηνε χειροήθη, καὶ ἐπωχεῖτο αὐτῷ, καὶ 
ἤρα τοῦ κτήματος καὶ ἀντηρᾶτο, ἀνθ' ὧν ἔθρεψε τὴν 
ἀμοιβὴν κομιζόμενος ἐκεῖνος. 



Claudius Aelianus Soph., De natura animalium 
Book 3, section 46, line 6

                                  ὁ τοίνυν βασιλεὺς τῶν 
Ἰνδῶν πυθόμενος ᾔτει λαβεῖν τὸν ἐλέφαντα. 



Claudius Aelianus Soph., De natura animalium 
Book 3, section 46, line 11

       ἀγανακτεῖ ὁ βασιλεύς, καὶ πέμπει κατ' αὐτοῦ 
τοὺς ἀφαιρησομένους καὶ ἅμα καὶ τὸν Ἰνδὸν ἐπὶ τὴν 
δίκην ἄξοντας. 



Claudius Aelianus Soph., De natura animalium 
Book 3, section 46, line 16

καὶ τὰ μὲν πρῶτα ἦν τοιαῦτα· ἐπεὶ δὲ βληθεὶς ὁ 
Ἰνδὸς κατώλισθε, περιβαίνει μὲν τὸν τροφέα ὁ ἐλέ-
φας κατὰ τοὺς ὑπερασπίζοντας ἐν τοῖς ὅπλοις, καὶ 
τῶν ἐπιόντων πολλοὺς ἀπέκτεινε, τοὺς δὲ ἄλλους 
ἐτρέψατο· περιβαλὼν δὲ τῷ τροφεῖ τὴν προβοσκίδα, 
αἴρει τε αὐτὸν καὶ ἐπὶ τὰ αὔλια κομίζει, καὶ παρέ-
μεινεν ὡς φίλῳ φίλος πιστός, καὶ τὴν εὔνοιαν ἐπε-
δείκνυτο. 



Claudius Aelianus Soph., De natura animalium 
Book 4, section 19, line 1

Κύνες Ἰνδικοί, θηρία καὶ οἵδε εἰσὶ καὶ ἀλκὴν 
ἄλκιμα καὶ ψυχὴν θυμοειδέστατα καὶ τῶν πανταχό-
θεν κυνῶν μέγιστοι. 



Claudius Aelianus Soph., De natura animalium 
Book 4, section 19, line 4

                      καὶ τῶν μὲν ἄλλων ζῴων ὑπερ-
φρονοῦσι, λέοντι δὲ ὁμόσε χωρεῖ κύων Ἰνδικός, καὶ 
ἐγκείμενον ὑπομένει, καὶ βρυχωμένῳ ἀνθυλακτεῖ, 
καὶ ἀντιδάκνει δάκνοντα· καὶ πολλὰ αὐτὸν λυπήσας 
καὶ κατατρώσας, τελευτῶν ἡττᾶται ὁ κύων. 



Claudius Aelianus Soph., De natura animalium 
Book 4, section 19, line 8

                                                εἴη δ' 
ἂν καὶ λέων ἡττηθεὶς ὑπὸ κυνὸς Ἰνδοῦ, καὶ μέν-
τοι καὶ δακὼν ὁ κύων ἔχεται καὶ μάλα ἐγκρατῶς. 



Claudius Aelianus Soph., De natura animalium 
Book 4, section 21, line 1

Θηρίον Ἰνδικὸν βίαιον τὴν ἀλκήν, μέγεθος κατὰ 
τὸν λέοντα τὸν μέγιστον, τὴν δὲ χρόαν ἐρυθρόν, ὡς 
κινναβάρινον εἶναι δοκεῖν, δασὺ δὲ ὡς κύνες, φωνῇ 
τῇ Ἰνδῶν μαρτιχόρας ὠνόμασται. 



Claudius Aelianus Soph., De natura animalium 
Book 4, section 21, line 25

                λέγει δὲ ἄρα Κτησίας καί φησιν ὁμολο-
γεῖν αὐτῷ τοὺς Ἰνδούς, ἐν ταῖς χώραις τῶν ἀπολυο-
μένων ἐκείνων κέντρων ὑπαναφύεσθαι ἄλλα, ὡς εἶναι 
τοῦ κακοῦ τοῦδε ἐπιγονήν. 



Claudius Aelianus Soph., De natura animalium 
Book 4, section 21, line 37

                                           τὰ βρέφη δὲ 
τῶνδε τῶν ζῴων Ἰνδοὶ θηρῶσιν ἀκέντρους τὰς οὐ-
ρὰς ἔχοντα, καὶ λίθῳ γε διαθλῶσιν αὐτάς, ἵνα ἀδυ-  
νατῶσι τὰ κέντρα ἀναφύειν. 



Claudius Aelianus Soph., De natura animalium 
Book 4, section 21, line 41

                                 λέγει δὲ καὶ ἑορακέναι 
τόδε τὸ ζῷον ἐν Πέρσαις Κτησίας ἐξ Ἰνδῶν κομισθὲν 
δῶρον τῷ Περσῶν βασιλεῖ, εἰ δή τῳ ἱκανὸς τεκμη-
ριῶσαι ὑπὲρ τῶν τοιούτων Κτησίας. 



Claudius Aelianus Soph., De natura animalium 
Book 4, section 24, line 1

Οἱ Ἰνδοὶ τέλειον μὲν ἐλέφαντα συλλαβεῖν ῥᾳδίως 
ἀδυνατοῦσιν, ἐς δὲ τὰ ἕλη φοιτῶντες τὰ γειτνιῶντα 
τῷ ποταμῷ εἶτα μέντοι λαμβάνουσιν αὐτῶν τὰ βρέφη. 



Claudius Aelianus Soph., De natura animalium 
Book 4, section 26, line 1

Τοὺς λαγὼς καὶ τὰς ἀλώπεκας θηρῶσιν οἱ Ἰνδοὶ 
τὸν τρόπον τοῦτον. 



Claudius Aelianus Soph., De natura animalium 
Book 4, section 27, line 1

Τὸν γρῦπα ἀκούω τὸ ζῷον τὸ Ἰνδικὸν τετράπουν 
εἶναι κατὰ τοὺς λέοντας, καὶ ἔχειν ὄνυχας καρτεροὺς 
ὡς ὅτι μάλιστα, καὶ τούτους μέντοι τοῖς τῶν λεόν-
των παραπλησίους· κατάπτερον δὲ εἶναι, καὶ τῶν 
μὲν νωτιαίων πτερῶν τὴν χρόαν μέλαιναν ᾄδουσι,   
τὰ δὲ πρόσθια ἐρυθρά φασι, τάς γε μὴν πτέρυγας 
αὐτὰς οὐκέτι τοιαύτας, ἀλλὰ λευκάς. 



Claudius Aelianus Soph., De natura animalium 
Book 4, section 27, line 14

                                            καὶ Βάκτριοι 
μὲν γειτνιῶντες Ἰνδοῖς λέγουσιν αὐτοὺς φύλακας 
εἶναι τοῦ χρυσοῦ τοῦ αὐτόθι, καὶ ὀρύττειν τε αὐτόν 
φασιν αὐτοὺς καὶ ἐκ τούτου τὰς καλιὰς ὑποπλέκειν, 
τὸ δὲ ἀπορρέον Ἰνδοὺς λαμβάνειν. 



Claudius Aelianus Soph., De natura animalium 
Book 4, section 32, line 1

Προβατεῖαι δὲ Ἰνδῶν ὁποῖαι μαθεῖν ἄξιον. 



Claudius Aelianus Soph., De natura animalium 
Book 4, section 32, line 4

                                                     τὰς 
αἶγας καὶ τὰς οἶς ὄνων τῶν μεγίστων μείζονας ἀκούω 
καὶ ἀποκύειν τέτταρα ἑκάστην· μείω γε μὴν τῶν 
τριῶν οὔτ' αἲξ Ἰνδικὴ οὔτ' ἂν οἶς ποτε τέκοι. 



Claudius Aelianus Soph., De natura animalium 
Book 4, section 36, line 1

Ἡ τῶν Ἰνδῶν γῆ, φασὶν αὐτὴν οἱ συγγραφεῖς 
πολυφάρμακόν τε καὶ τῶν βλαστημάτων τῶνδε δει-
νῶς πολύγονον εἶναι. 



Claudius Aelianus Soph., De natura animalium 
Book 4, section 36, line 14

                                               ὀδόντων 
δὲ ἄγονός ἐστιν ὁ ὄφις οὗτος· εὑρίσκεται δ' ἐν 
τοῖς πυρωδεστάτοις τῆς Ἰνδικῆς χωρίοις. 



Claudius Aelianus Soph., De natura animalium 
Book 4, section 41, line 1

Γένος ὀρνίθων Ἰνδικῶν βραχυτάτων καὶ τοῦτο 
εἴη ἄν. 



Claudius Aelianus Soph., De natura animalium 
Book 4, section 41, line 5

                                            καὶ Ἰνδοὶ 
μὲν αὐτὸ φωνῇ τῇ σφετέρᾳ δίκαιρον φιλοῦσιν ὀνο-
μάζειν, Ἕλληνες δὲ ὡς ἀκούω δίκαιον. 



Claudius Aelianus Soph., De natura animalium 
Book 4, section 41, line 15

      σπουδὴν δὲ ἄρα τὴν ἀνωτάτω τίθενται Ἰν-
δοὶ ἐς τὴν κτῆσιν αὐτοῦ· κακῶν γὰρ αὐτὸ ἐπί-
ληθον ἡγοῦνται τῷ ὄντι· καὶ οὖν καὶ ἐν τοῖς 
δώροις τοῖς μέγα τιμίοις τῷ Περσῶν βασιλεῖ ὁ Ἰν-
δῶν πέμπει καὶ τοῦτο. 



Claudius Aelianus Soph., De natura animalium 
Book 4, section 41, line 24

                                                         καὶ 
διὰ ταῦτα ἀντικρίνοντες βασανίσωμεν τῶν φαρμά-
κων τοῦ τε Ἰνδικοῦ καὶ τοῦ Αἰγυπτίου ὁπότερον ἦν 
προτιμότερον· ἐπεὶ τὸ μὲν ἐφ' ἡμέραν ἀνεῖργέ τε 
καὶ ἀνέστελλε τὰ δάκρυα τὸ Αἰγύπτιον, τὸ δὲ λήθην 
κακῶν παρεῖχεν αἰώνιον τὸ Ἰνδικόν· καὶ τὸ μὲν γυ-
ναικὸς δῶρον ἦν, τὸ δὲ ὄρνιθος ἢ ἀπορρήτου φύσεως 
δεσμῶν τῶν ὄντως βαρυτάτων ἀπολυούσης δι' ὑπη-
ρέτου τοῦ προειρημένου. 



Claudius Aelianus Soph., De natura animalium 
Book 4, section 41, line 30

                           καὶ Ἰνδοὺς κτήσασθαι αὐτὸ 
εὐτυχήσαντας, ὡς τῆς ἐνταυθοῖ φρουρᾶς ἀπολυθῆναι 
ὅταν ἐθέλωσιν. 



Claudius Aelianus Soph., De natura animalium 
Book 4, section 42, line 11

                                        ταῖς δὲ ἴνδαλμά τε 
καὶ σπέρμα τοῦ τότε πένθους ἐντακῆναι, καὶ ἐς νῦν 
ἔτι Μελέαγρόν τε ἀναμέλπειν, καὶ ὡς αὐτῷ προσή-
κουσιν ᾄδειν καὶ τοῦτο μέντοι. 



Claudius Aelianus Soph., De natura animalium 
Book 4, section 46, line 1

Ἐν Ἰνδοῖς γίνεται θηρία τὸ μέγεθος ὅσον γέ-
νοιντο ἂν οἱ κάνθαροι, καὶ ἔστιν ἐρυθρά· κινναβά-
ρει δὲ εἰκάσειας ἄν, εἰ πρῶτον θεάσαιο αὐτά. 



Claudius Aelianus Soph., De natura animalium 
Book 4, section 46, line 7

        θηρῶσι δὲ αὐτὰ οἱ Ἰνδοὶ καὶ ἀποθλίβουσι, 
καὶ ἐξ αὐτῶν βάπτουσι τάς τε φοινικίδας καὶ τοὺς 
ὑπ' αὐταῖς χιτῶνας καὶ πᾶν ὅ τι ἂν ἐθέλωσιν ἄλλο 
ἐς τήνδε τὴν χρόαν ἐκτρέψαι τε καὶ χρῶσαι. 



Claudius Aelianus Soph., De natura animalium 
Book 4, section 46, line 17

Γίνονται δὲ ἐνταῦθα τῆς Ἰνδικῆς, ἔνθα οἱ κάν-
θαροι, καὶ οἱ καλούμενοι κυνοκέφαλοι, οἷς τὸ ὄνομα 
ἔδωκεν ἡ τοῦ σώματος ὄψις τε καὶ φύσις· τὰ δὲ ἄλλα 
ἀνθρώπων ἔχουσι, καὶ ἠμφιεσμένοι βαδίζουσι δορὰς 
θηρίων. 



Claudius Aelianus Soph., De natura animalium 
Book 4, section 46, line 23

         καί εἰσι δίκαιοι, καὶ ἀνθρώπων λυποῦσιν 
οὐδένα, καὶ φθέγγονται μὲν οὐδὲ ἕν, ὠρύονται δέ, 
τῆς γε μὴν Ἰνδῶν φωνῆς ἐπαΐουσι. 



Claudius Aelianus Soph., De natura animalium 
Book 4, section 52, line 2

Ὄνους ἀγρίους οὐκ ἐλάττους ἵππων τὰ μεγέθη ἐν 
Ἰνδοῖς γίνεσθαι πέπυσμαι. 



Claudius Aelianus Soph., De natura animalium 
Book 4, section 52, line 9

                                   ἐκ δὴ τῶνδε τῶν ποι-
κίλων κεράτων πίνειν Ἰνδοὺς ἀκούω, καὶ ταῦτα οὐ 
πάντας, ἀλλὰ τοὺς τῶν Ἰνδῶν κρατίστους, ἐκ διαστη-
μάτων αὐτοῖς χρυσὸν περιχέαντας, οἱονεὶ ψελίοις τισὶ 
κοσμήσαντας βραχίονα ὡραῖον ἀγάλματος. 



Claudius Aelianus Soph., De natura animalium 
Book 4, section 52, line 22

                          πεπίστευται δὲ τοὺς ἄλλους 
τοὺς ἀνὰ πᾶσαν τὴν γῆν ὄνους καὶ ἡμέρους καὶ 
ἀγρίους καὶ τὰ ἄλλα μώνυχα θηρία ἀστραγάλους οὐκ 
ἔχειν, οὐδὲ μὴν ἐπὶ τῷ ἥπατι χολήν, ὄνους δὲ τοὺς 
Ἰνδοὺς λέγει Κτησίας τοὺς ἔχοντας τὸ κέρας ἀστρα-
γάλους φορεῖν, καὶ ἀχόλους μὴ εἶναι· λέγονται δὲ οἱ 
ἀστράγαλοι μέλανες εἶναι, καὶ εἴ τις αὐτοὺς συντρί-
ψειεν, εἶναι τοιοῦτοι καὶ τὰ ἔνδον. 



Claudius Aelianus Soph., De natura animalium 
Book 4, section 52, line 32

διατριβαὶ δὲ τοῖς ὄνοις τῶν Ἰνδικῶν πεδίων τὰ ἐρη-
μότατά ἐστιν. 



Claudius Aelianus Soph., De natura animalium 
Book 4, section 52, line 33

                 ἰόντων δὲ τῶν Ἰνδῶν ἐπὶ τὴν ἄγραν 
αὐτῶν, τὰ μὲν ἁπαλὰ καὶ ἔτι νεαρὰ ἑαυτῶν νέμεσθαι 
κατόπιν ἐῶσιν, αὐτοὶ δὲ ὑπερμαχοῦσι, καὶ ἴασι τοῖς 
ἱππεῦσιν ὁμόσε, καὶ τοῖς κέρασι παίουσι. 



Claudius Aelianus Soph., De natura animalium 
Book 4, section 52, line 48

                                    ζῶντα μὲν οὖν 
τέλειον οὐκ ἂν λάβοις, βάλλονται δὲ ἀκοντίοις καὶ 
οἰστοῖς, καὶ τὰ κέρατα ἐξ αὐτῶν Ἰνδοὶ νεκρῶν σκυ-
λεύσαντες ὡς εἶπον περιέπουσιν. 



Claudius Aelianus Soph., De natura animalium 
Book 4, section 52, line 49

                                     ὄνων δὲ Ἰνδῶν 
ἄβρωτόν ἐστι τὸ κρέας· τὸ δὲ αἴτιον, πέφυκεν εἶναι 
πικρότατον. 



Claudius Aelianus Soph., De natura animalium 
Book 5, section 3, line 1

Ὁ ποταμὸς ὁ Ἰνδὸς ἄθηρός ἐστι, μόνος δὲ ἐν αὐ-
τῷ τίκτεται σκώληξ φασί. 



Claudius Aelianus Soph., De natura animalium 
Book 5, section 3, line 41

                                τοῦτο δὴ τὸ ἔλαιον τῷ 
βασιλεῖ τῶν Ἰνδῶν κομίζουσι, σημεῖα ἐπιβαλόντες· 
ἔχειν γὰρ αὐτοῦ ἄλλον οὐδὲ ὅσον ῥανίδα ἐφεῖται. 



Claudius Aelianus Soph., De natura animalium 
Book 5, section 3, line 49

τούτῳ τοί φασι τὸν τῶν Ἰνδῶν βασιλέα καὶ τὰς πό-
λεις αἱρεῖν τὰς ἐς ἔχθραν προελθούσας οἱ, καὶ μήτε 
κριοὺς μήτε χελώνας μήτε τὰς ἄλλας ἑλεπόλεις ἀνα-
μένειν, ἐπεὶ καταπιμπρὰς ᾕρηκεν· ἀγγεῖα γὰρ κερα-
μεᾶ ὅσον κοτύλην ἕκαστον χωροῦντα ἐμπλήσας αὐτοῦ 
καὶ ἀποφράξας ἄνωθεν ἐς τὰς πύλας σφενδονᾷ. 



Claudius Aelianus Soph., De natura animalium 
Book 5, section 21, line 41

                                       Ἀλέξανδρος δὲ ὁ 
Μακεδὼν ἐν Ἰνδοῖς ἰδὼν τούσδε τοὺς ὄρνιθας ἐξε-
πλάγη, καὶ τοῦ κάλλους θαυμάσας ἠπείλησε τῷ κα-
ταθύσαντι ταῶν ἀπειλὰς βαρυτάτας. 



Claudius Aelianus Soph., De natura animalium 
Book 5, section 51, line 4

             ὁ γοῦν Σκύθης ἄλλως φθέγγεται καὶ 
ὁ Ἰνδὸς ἄλλως, καὶ ὁ Αἰθίοψ ἔχει φωνὴν συμφυᾶ 
καὶ οἱ Σάκαι· φωνὴ δὲ Ἑλλὰς ἄλλη, καὶ Ῥωμαία 
ἄλλη. 



Claudius Aelianus Soph., De natura animalium 
Book 5, section 55, line 1

Ἐν τοῖς Ἰνδοῖς οἱ ἐλέφαντες, ὅταν τι τῶν δέν-
δρων αὐτόρριζον ἀναγκάζωσιν αὐτοὺς οἱ Ἰνδοὶ ἐκ-
σπάσαι, οὐ πρότερον ἐμπηδῶσιν οὐδὲ ἐπιχειροῦσι τῷ 
ἔργῳ πρὶν ἢ διασεῖσαι αὐτὸ καὶ διασκέψασθαι ἆρά 
γε ἀνατραπῆναι οἷόν τέ ἐστιν ἢ παντελῶς ἀδύνατον. 



Claudius Aelianus Soph., De natura animalium 
Book 6, section 21, line 1

Ἐν Ἰνδοῖς, ὡς ἀκούω, ἐλέφας καὶ δράκων ἐστὶν 
ἔχθιστα. 



Claudius Aelianus Soph., De natura animalium 
Book 7, section 37, line 1

Πώρου τοῦ τῶν Ἰνδῶν βασιλέως ὁ ἐλέφας ἐν τῇ 
πρὸς Ἀλέξανδρον μάχῃ τετρωμένου πολλὰ ἡσυχῆ 
καὶ μετὰ φειδοῦς τῇ προβοσκίδι ἐξῄρει τὰ ἀκόντια, 
καὶ μέντοι καὶ αὐτὸς τετρωμένος πολλὰ οὐ πρότερον 
εἶξε πρὶν ἢ συνεῖναι ὅτι ἄρα ὁ δεσπότης αὐτῷ διὰ 
τὴν ῥοὴν τοῦ αἵματος τὴν πολλὴν παρεῖται καὶ ἐκ-
θνήσκει. 



Claudius Aelianus Soph., De natura animalium 
Book 8, section 1, line 1

                                                       καὶ   
τοῦτο ἀκουέτω Ἐρατοσθένους τε καὶ Εὐφορίωνος 
καὶ ἄλλων περιηγουμένων αὐτό. 
 Ἰνδικοὶ λόγοι διδάσκουσιν ἡμᾶς καὶ ἐκεῖνα. 



Claudius Aelianus Soph., De natura animalium 
Book 8, section 1, line 21

Ἀλεξάνδρῳ γοῦν τῷ Φιλίππου πεῖραν ἔδοσαν οἱ Ἰν-
δοὶ τῆς τῶν κυνῶν τῶνδε ἀλκῆς τὸν τρόπον τοῦτον. 



Claudius Aelianus Soph., De natura animalium 
Book 8, section 1, line 29

                                                         ὁ 
τοίνυν Ἰνδὸς ὁ τὴν θέαν τῷ βασιλεῖ τήνδε παρέχων 
κάλλιστα εἰδὼς τοῦ κυνὸς τὸ καρτερικόν, προσέταξέν 
οἱ τὴν οὐρὰν ἀποκοπῆναι. 



Claudius Aelianus Soph., De natura animalium 
Book 8, section 1, line 32

                      προσέταξεν οὖν ὁ Ἰνδὸς καὶ τῶν 
σκελῶν ἓν ἀποκόψαι, καὶ ἀπεκόπη· ὃ δὲ ὡς ἐξ ἀρ-
χῆς ἐνέφυ εἴχετο, καὶ οὐκ ἀνίει, ὥσπερ οὖν ἀλλο-
τρίου κοπτομένου σκέλους καὶ ὀθνείου. 



Claudius Aelianus Soph., De natura animalium 
Book 8, section 1, line 46

            ἰδὼν οὖν ὁ Ἰνδὸς αὐτὸν ἀνιώμενον, τέτ-
ταρας ὁμοίους ἐκείνῳ κύνας ἔδωκέν οἱ. 



Claudius Aelianus Soph., De natura animalium 
Book 8, section 7, line 2

Μεγασθένους ἀκούω λέγοντος περὶ τὴν τῶν Ἰν-
δῶν θάλατταν γίνεσθαί τι ἰχθύδιον, καὶ τοῦτο μὲν 
ὅταν ζῇ ἀθέατον εἶναι, κάτω που νηχόμενον καὶ ἐν 
βυθῷ, ἀποθανὸν δὲ ἀναπλεῖν. 



Claudius Aelianus Soph., De natura animalium 
Book 8, section 15, line 12

Ἔστι δὲ ἐν τοῖς Ἰνδοῖς ἄρουρα, καὶ κέκληται Φα-
λάκρα. 



Claudius Aelianus Soph., De natura animalium 
Book 9, section 30, line 2

Λέων ὅταν βαδίζῃ, οὐκ εὐθύωρον πρόεισιν, οὐδὲ 
ἐᾷ τῶν ἰχνῶν ἑαυτοῦ ἁπλᾶ εἶναι τὰ ἰνδάλματα, ἀλλὰ 
πῆ μὲν πρόεισι, πῆ δὲ ἐπάνεισι, καὶ αὖ πάλιν τοῦ 
πρόσω ἔχεται, καὶ μέντοι καὶ ἵεται ἐς τοὔμπαλιν. 



Claudius Aelianus Soph., De natura animalium 
Book 9, section 58, line 10

                    λέγει δὲ ὁ Ἰόβας γενέσθαι μὲν αὐ-
τοῦ τῷ πατρὶ πολυετῆ Λίβυν ἐλέφαντα κατιόντα ἐκ 
τῶν ἄνω τοῦ γένους· καὶ Πτολεμαίῳ δὲ τῷ Φιλα-
δέλφῳ Αἰθίοπα, καὶ ἐκεῖνον ἐκ πολλοῦ βιώσαντα 
γενέσθαι πραότατον καὶ ἡμερώτατον τὰ μὲν ἐκ τῆς 
πρὸς τοὺς ἀνθρώπους συντροφίας, τὰ δὲ πωλευθέντα· 
Σελεύκου τε τοῦ Νικάτορος κτῆμα ᾄδει Ἰνδὸν ἐλέ-
φαντα, καὶ μέντοι καὶ διαβιῶναι τοῦτον μέχρι τῆς 
τῶν Ἀντιόχων ἐπικρατείας φησίν. 



Claudius Aelianus Soph., De natura animalium 
Book 10, section 13, line 5

             αἵ τε γὰρ ἀκρίδες καὶ οἱ ὄφεις, χρυσοειδῆ 
ἰνδάλματα καὶ ἐπ' αὐτῶν κατέστικται· οἱ δὲ ἰχθῦς ἔτι 
καὶ πλέον τῆς πολυκόσμου χρόας μετειληχότες εἶτα 
ἰδεῖν ἐκπληκτικοί εἰσι. 



Claudius Aelianus Soph., De natura animalium 
Book 11, section 14, line 10

                  ταύτῃ τοίνυν ἡ τοῦ τρέφοντος αὐτὸν 
γυνὴ παιδίον, ὃ ἔτυχε τεκοῦσα πρὸ ἡμερῶν τριάκον-
τα, παρακατέθετο φωνῇ τῇ Ἰνδῶν, ἧς ἀκούουσιν 
ἐλέφαντες. 



Claudius Aelianus Soph., De natura animalium 
Book 11, section 15, line 12

καὶ τοῦτο μὲν Ἰνδικὸν τὸ ἔργον, ἐκεῖθεν δὲ ἐξεφοί-
τησε δεῦρο· ἀκούω δὲ καὶ ἐπὶ Τίτου ἀνδρὸς καλοῦ 
καὶ ἀγαθοῦ ἐν τῇ Ῥώμῃ ταὐτὸν γεγονέναι· προστι-
θέασι δὲ ὅτι ἄρα ὁ ἐνθάδε ἐλέφας ἀπέκτεινεν ἀμφο-
τέρους, καὶ ἱματίῳ κατεκάλυψε, καὶ ἐλθόντι τῷ 
τροφεῖ ἀποβαλὼν τὸ ἱμάτιον κειμένους ἀλλήλων 
πλησίον ἀπέδειξε, καὶ τὸ κέρας δέ, ᾧπερ οὖν διέπει-
ρεν αὐτούς, καὶ τοῦτο ᾑμαγμένον ἑωρᾶτο. 



Claudius Aelianus Soph., De natura animalium 
Book 11, section 25, line 4

ἐπεπίστευτο δὲ πρὸ τοῦδε τοῦ ζῴου τῆς Ἰνδῶν μόνης 
φωνῆς ἐπαΐειν τοὺς ἐλέφαντας. 



Claudius Aelianus Soph., De natura animalium 
Book 11, section 33, line 1

Ταῶν δὲ Ἰνδικὸν δῶρον λαβὼν ὁ τῶν Αἰγυπτίων 
βασιλεύς, ταώνων ἰδεῖν μέγιστόν τε καὶ ὡραιότατον, 
οὐκ ἀξιοῖ σὺν τοῖς ἀγελαίοις τρέφειν, ὡς οἰκίας 
ἄθυρμα αὐτὸν εἶναι ἢ γαστρὸς χάριν, ἀλλὰ ἀνάπτει 
τῷ Πολιεῖ Διί, κρίνας ἀνάθημα ἐπάξιον τῷ 
θεῷ τὸν ὄρνιν τὸν προειρημένον. 



Claudius Aelianus Soph., De natura animalium 
Book 12, section 32, line 1

Ἡ Ἰνδῶν γῆ φέρει πολλὰ καὶ ποικίλα. 



Claudius Aelianus Soph., De natura animalium 
Book 12, section 32, line 19

      καὶ ταῦτα μὲν αὐτοῖς ἐς ἐπικουρίαν τὴν 
ἀναγκαίαν καὶ μάλα εὐπόρως ἀνίησιν ἡ χώρα καὶ 
ἀφθόνως· ὄφις δὲ ὃς ἂν ἀποκτείνῃ ἄνθρωπον, ὡς 
Ἰνδοὶ λέγουσιν (καὶ μάρτυρας ἐπάγονται Λιβύων πολ-
λοὺς καὶ τοὺς περὶ Θήβας οἰκοῦντας Αἰγυπτίων), 
οὐκέτι καταδῦναι καὶ ἐσερπύσαι ἐς τὴν ἑαυτοῦ οἰκίαν 
ἔχει, τῆς γῆς αὐτὸν μὴ δεχομένης, ἀλλ' ἐκβαλλούσης 
τῶν οἰκείων ὡς ἂν εἴποις φυγάδα κόλπων. 



Claudius Aelianus Soph., De natura animalium 
Book 12, section 41, line 1

Ὁ Γάγγης ὁ παρὰ τοῖς Ἰνδοῖς ῥέων ὑπαρχόμενος 
μὲν ἐκ τῶν πηγῶν βαθύς ἐστιν ἐς ὀργυιὰς εἴκοσι, 
πλατὺς δὲ ἐς ὀγδοήκοντα σταδίους· ἔτι γὰρ αὐθιγε-
νεῖ τῷ ὕδατι πρόεισι καὶ ἀμιγεῖ πρὸς ἕτερον· προϊὼν 
δὲ τῶν ἄλλων ἐς αὐτὸν ἐμπιπτόντων καὶ ἀνακοινου-
μένων οἱ τὸ ὕδωρ ἐς βάθος μὲν ἥκει καὶ ἑξήκοντα 
ὀργυιῶν, πλατύνεται δὲ καὶ ὑπερεκχεῖται ἐς σταδίους 
τετρακοσίους. 



Claudius Aelianus Soph., De natura animalium 
Book 12, section 44, line 1

Λόγω δὲ ἄρα τώδε Ἰνδὸς καὶ Λίβυς τὸ γένος δια-
φόρω· ἐρεῖ δὲ ὁ μὲν Ἰνδὸς τὰ ἐπιχώρια, ὁ δὲ Λίβυς 
ὅσα οἶδε καὶ ἐκεῖνος· ἃ δ' οὖν ᾄδετον ἄμφω τὼ λόγω 
ἐστὶν ἐκεῖνα. 



Claudius Aelianus Soph., De natura animalium 
Book 12, section 44, line 4

                  ἐν Ἰνδοῖς ἐὰν ἁλῷ τέλειος ἐλέφας, 
ἡμερωθῆναι χαλεπός ἐστι, καὶ τὴν ἐλευθερίαν πο-
θῶν φονᾷ. 



Claudius Aelianus Soph., De natura animalium 
Book 12, section 44, line 8

                           ἀλλ' οἱ Ἰνδοὶ καὶ ταῖς τρο-
φαῖς κολακεύουσιν αὐτόν, καὶ ποικίλοις καὶ ἐφολκοῖς 
δελέασι πραΰνειν πειρῶνται, παρατιθέντες ὅσα πλη-
ροῖ τὴν γαστέρα καὶ θέλγει τὸν θυμόν. 



Claudius Aelianus Soph., De natura animalium 
Book 13, section 7, line 2

Τῶν τεθηραμένων ἐλεφάντων ἰῶνται τὰ τραύματα 
οἱ Ἰνδοὶ τὸν τρόπον τοῦτον. 



Claudius Aelianus Soph., De natura animalium 
Book 13, section 8, line 19

                                    κατασπείρει δὲ καὶ 
τοῦ χώρου ἔνθα αὐλίζεται τῶν ἀνθέων πολλά, ἡδυ-
σμένον αἱρεῖσθαι γλιχόμενος ὕπνον. Ἰνδοὶ δὲ ἐλέφαν-
τες ἦσαν ἄρα πήχεων ἐννέα τὸ ὕψος, πέντε δὲ τὸ 
εὖρος. 



Claudius Aelianus Soph., De natura animalium 
Book 13, section 9, line 1

Ἵππον δὲ ἄρα Ἰνδὸν κατασχεῖν καὶ ἀνακροῦσαι   
προπηδῶντα καὶ ἐκθέοντα οὐ παντὸς ἦν, ἀλλὰ τῶν 
ἐκ παιδὸς ἱππείαν πεπαιδευμένων. 



Claudius Aelianus Soph., De natura animalium 
Book 13, section 18, line 1

Ἐν δὲ τοῖς βασιλείοις τοῖς Ἰνδικοῖς, ἔνθα ὁ μέγι-
στος τῶν βασιλέων διαιτᾶται τῶν ἐκεῖθι, πολλὰ μὲν 
καὶ ἄλλα ἐστὶ θαυμάσαι ἄξια, ὡς μὴ αὐτοῖς ἀντικρί-
νειν μήτε τὰ Μεμνόνεια Σοῦσα καὶ τὴν ἐν αὐτοῖς 
πολυτέλειαν μήτε τὴν ἐν τοῖς Ἐκβατάνοις μεγαλουρ-
γίαν· ἔοικε γὰρ κόμπος εἶναι Περσικὸς ἐκεῖνα, εἰ 
πρὸς ταῦτα ἐξετάζοιτο. 



Claudius Aelianus Soph., De natura animalium 
Book 13, section 18, line 19

                              καὶ τὸ σεμνότερον τῆς   
ὥρας τῆς ἐκεῖθι, τὰ δένδρα αὐτὰ τῶν ἀειθαλῶν ἐστι, 
καὶ οὔποτε γηρᾷ καὶ ἀποῤῥεῖ τὰ φύλλα· καὶ τὰ μὲν 
ἐπιχώριά ἐστι, τὰ δὲ ἀλλαχόθεν σὺν πολλῇ κομισθέντα 
τῇ φροντίδι, ἅπερ οὖν κοσμεῖ τὸν χῶρον καὶ ἀγλαΐαν 
δίδωσι, πλὴν ἐλαίας· οὐ γὰρ αὐτὴν ἡ Ἰνδῶν φέρει, 
οὔτε αὐτή, οὔτε ἥκουσαν ἀλλαχόθεν τρέφει. 



Claudius Aelianus Soph., De natura animalium 
Book 13, section 18, line 24

                     σιτεῖται δὲ Ἰνδῶν οὐδὲ εἷς ψιτ-
τακόν, καίτοι παμπόλλων ὄντων τὸ πλῆθος· τὸ δὲ 
αἴτιον, ἱεροὺς αὐτοὺς εἶναι πεπιστεύκασιν οἱ Βρα-
χμᾶνες, καὶ μέντοι καὶ τῶν ὀρνίθων ἁπάντων προτι-
μῶσι. 



Claudius Aelianus Soph., De natura animalium 
Book 13, section 22, line 1

Τὸν Ἰνδῶν βασιλέα προϊόντα ἐπὶ δίκαις προσκυ-
νεῖ ὁ ἐλέφας πρῶτος, δεδιδαγμένος τοῦτο, καὶ μάλα 
γε δρῶν μνημόνως τε καὶ εὐπειθῶς αὐτό (παρέστηκε 
δὲ καὶ ἐκεῖνος, ὅσπερ οὖν ἐνδίδωσίν οἱ τοῦ παιδεύ-
ματος τὴν ὑπόμνησιν τῇ ἐκ τῆς ἅρπης κρούσει καὶ 
φωνῇ τινι ἐπιχωρίῳ, ἧσπερ οὖν ἐλέφαντες ἐπαΐειν 
εἰλήχασι φύσει τινὶ ἀπορρήτῳ καὶ μάλα γε ἰδίᾳ τοῦ 
ζῴου τοῦδε)· καὶ μέντοι καὶ κίνησίν τινα ὑποκινεῖται 
πολεμικήν, οἷον ἐνδεικνύμενος ὅτι καὶ τοῦτο τὸ μά-
θημα ἀποσώζει. 



Claudius Aelianus Soph., De natura animalium 
Book 13, section 22, line 14

                 τέτταρες δὲ καὶ εἴκοσι τῷ βασιλεῖ 
φρουροὶ παραμένουσιν ἐλέφαντες ἐκ διαδοχῆς, ὥσπερ 
οὖν οἱ φύλακες οἱ λοιποί, καὶ αὐτοῖς παίδευμα τὴν 
φρουρὰν ἔχειν οὐ κατανυστάζουσι· διδάσκονται γάρ 
τοι σοφίᾳ τινὶ Ἰνδικῇ καὶ τοῦτο. 



Claudius Aelianus Soph., De natura animalium 
Book 13, section 25, line 2

Ἵππους καὶ ἐλέφαντας ἅτε ζῷα καὶ ἐν ὅπλοις καὶ 
ἐν πολέμοις λυσιτελῆ τιμῶσιν Ἰνδοί, καὶ μάλα γε 
ἰσχυρῶς. 



Claudius Aelianus Soph., De natura animalium 
Book 13, section 25, line 9

      οὐκ ἀτιμάζει δὲ οὐδὲ τὰ ἄλλα ζῷα, ἀλλὰ καὶ   
ἐκεῖνα προσίεται δῶρά οἱ κομιζόμενα. Ἰνδοὶ γὰρ 
οὐκ ἐκφαυλίζουσι ζῷον οὔτε ἥμερον οὔτε μὴν ἄγριον 
οὐδέν. 



Claudius Aelianus Soph., De natura animalium 
Book 13, section 27, line 4

ταύτης οὖν τὴν δεξιὰν πτέρυγα εἰ ὑποθείης ἀνθρώ-
πῳ καθεύδοντι, εὖ μάλα ἐκταράξεις αὐτόν· δέα γάρ 
τινα καὶ ἰνδάλματα καὶ φάσματα ὄψεται, καὶ ἐνύπνια 
ἕτερα οὐδαμῶς εὐμενῆ καὶ φίλα. 



Claudius Aelianus Soph., De natura animalium 
Book 14, section 13, line 2

ὁ τῶν Ἰνδῶν βασιλεὺς ἐπιδόρπια σιτεῖται ταῦτα, οἷα 
δήπου Ἕλληνες ἐντραγεῖν αἰτοῦσι, φοινίκων τῶν 
χαμαιζήλων· ἐκεῖνος σκώληκά τινα ἐν τῷ φυτῷ τικτό-
μενον σταθευτὸν ἐπιδειπνεῖ γλύκιστον, ὡς Ἰνδῶν 
λέγουσι λόγοι, καί φασιν οἱ τὴν ἡδονὴν τὴν τοσαύτην 
ἐκ τοῦ σιτεῖσθαι ** καὶ ἐμέ γε αἱροῦσι λέγοντες. 



Claudius Aelianus Soph., De natura animalium 
Book 14, section 13, line 13

                                      τὰ μὲν οὖν ἄλλα   
οὐ μέμφομαι αὐτῷ, κύκνων γε μὴν Ἀπόλλωνι μὲν 
λατρευόντων ᾠδικωτάτων δὲ ὡς ἡ φήμη διαρρέουσα 
λέγει ἐπιβουλεύειν ἐκγόνοις καὶ διαφθείρειν τὰ ᾠά, 
ὦ Ἰνδοὶ φίλοι, οὐκέτι. 



Claudius Aelianus Soph., De natura animalium 
Book 15, section 7, line 1

Ὕεται ἡ Ἰνδῶν γῆ διὰ τοῦ ἦρος μέλιτι ὑγρῷ, καὶ 
ἔτι πλέον ἡ Πρασίων χώρα, ὅπερ οὖν ἐμπῖπτον ταῖς 
πόαις καὶ ταῖς τῶν ἑλείων καλάμων κόμαις, νομὰς 
τοῖς βουσὶ καὶ τοῖς προβάτοις παρέχει θαυμαστάς, καὶ 
τὰ μὲν ζῷα ἑστιᾶται ἡδίστην τήνδε ἑστίασιν (μάλιστα 
γὰρ ἐνταῦθα οἱ νομεῖς ἄγουσιν αὐτά, ἔνθα καὶ μᾶλλον 
ἡ δρόσος ἡ γλυκεῖα κάθηται πεσοῦσα), ἀνθεστιᾷ δὲ 
καὶ τὰ ζῷα τοὺς νομέας· ἀμέλγουσι γὰρ περιγλύκι-
στον γάλα, καὶ οὐ δέονται ἀναμῖξαι αὐτῷ μέλι, ὅπερ 
οὖν δρῶσιν Ἕλληνες. 



Claudius Aelianus Soph., De natura animalium 
Book 15, section 8, line 1

Ὁ δὲ Ἰνδὸς μάργαρος (ἄνω γὰρ εἶπον περὶ τοῦ   
Ἐρυθραίου) λαμβάνεται τρόπῳ τοιῷδε. 



Claudius Aelianus Soph., De natura animalium 
Book 15, section 8, line 23

                                                 ἄριστος 
δὲ ἄρα ὁ Ἰνδικὸς γίνεται καὶ ὁ τῆς θαλάττης τῆς 
Ἐρυθρᾶς. 



Claudius Aelianus Soph., De natura animalium 
Book 15, section 8, line 29

                             γίνεσθαι δέ φησιν Ἰόβας 
καὶ ἐν τῷ κατὰ Βόσπορον πορθμῷ, καὶ τοῦ Βρεττα-
νικοῦ ἡττᾶσθαι αὐτόν, τῷ δὲ Ἰνδῷ καὶ τῷ Ἐρυθραίῳ 
μηδὲ τὴν ἀρχὴν ἀντικρίνεσθαι. 



Claudius Aelianus Soph., De natura animalium 
Book 15, section 8, line 30

                                    ὁ δὲ ἐν Ἰνδίᾳ χερ-
σαῖος οὐ λέγεται φύσιν ἔχειν ἰδίαν, ἀλλὰ ἀπογέννημα 
εἶναι κρυστάλλου, οὐ τοῦ ἐκ τῶν παγετῶν συνιστα-
μένου, ἀλλὰ τοῦ ὀρυκτοῦ. 



Claudius Aelianus Soph., De natura animalium 
Book 15, section 14, line 1

Κομίζουσι δὲ ἄρα τῷ σφετέρῳ βασιλεῖ οἱ Ἰνδοὶ 
τίγρεις πεπωλευμένους καὶ τιθασοὺς πάνθηρας καὶ 
ὄρυγας τετράκερως, βοῶν δὲ γένη δύο, δρομικούς τε 
καὶ ἄλλους ἀγρίους δεινῶς. 



Claudius Aelianus Soph., De natura animalium 
Book 15, section 15, line 1

           καὶ περιστερὰς ὠχρὰς κομίζουσιν, ἅσπερ 
οὖν καὶ λέγουσι μήτε ἡμεροῦσθαι μήτε ποτὲ πραΰνε-
σθαι, καὶ ὄρνιθας δέ, οὓς κερκορώνους φιλοῦσιν ὀνο-
μάζειν, καὶ κύνας γενναίους, ὑπὲρ ὧν ἄνω μοι λέ-
λεκται, καὶ πιθήκους λευκοὺς καὶ μελαντάτους ἄλ-
λους· τοὺς γάρ τοι πυρροὺς ὡς γυναιμανεῖς ἐς τὰς 
πόλεις οὐκ ἄγουσιν, ἀλλὰ καί ποθεν ἐπιπηδήσαντες 
ἀναιροῦσιν, ὡς μοιχοὺς μεμισηκότες. 
 Ἰνδῶν δὲ ὁ μέγας βασιλεὺς μιᾶς ἡμέρας ἀνὰ πᾶν   
ἔτος ἀγωνίας προτίθησι τοῖς τε ἄλλοις ὅσοις εἶπον 
ἑτέρωθι, ἐν δὲ τοῖς καὶ ζῴοις ἀλόγοις, ἀλλὰ ἐκείνοις 
γε ὧν ἐκπέφυκε κέρατα. 



Claudius Aelianus Soph., De natura animalium 
Book 15, section 21, line 1

Ὅτε Ἀλέξανδρος τὰ μὲν ἐδόνει τῆς Ἰνδῶν γῆς 
τὰ δὲ ᾕρει, πολλοῖς μὲν καὶ ἄλλοις ζῴοις ἐνέτυχεν, 
ἐν δὲ τοῖς καὶ δράκοντι, ὅνπερ οὖν ἐν ἄντρῳ τινὶ 
νομίζοντες ἱερὸν Ἰνδοὶ μετὰ πολλοῦ τοῦ θειασμοῦ 
προσετρέποντο. 



Claudius Aelianus Soph., De natura animalium 
Book 15, section 21, line 5

                οὐκοῦν παντοῖοι ἐγένοντο οἱ Ἰνδοὶ δε-
όμενοι τοῦ Ἀλεξάνδρου μηδένα ἐπιθέσθαι τῷ ζῴῳ· ὃ 
δὲ κατένευσε. 



Claudius Aelianus Soph., De natura animalium 
Book 15, section 24, line 1

ἐπιφανεὶς δὲ ὁ Ἀπόλλων τὴν μὲν κόρην ἁρπάζει, τὴν 
δὲ ναῦν λίθον ἐργάζεται, τὸν δὲ Πομπίλον ἐς τὸν 
ἰχθὺν τοῦτον μετέβαλεν. 
 Ἰνδοὶ δὲ ἄρα καὶ περὶ τοὺς βοῦς τοὺς δρομικοὺς 
τίθενται σπουδήν. 



Claudius Aelianus Soph., De natura animalium 
Book 16, section 2, line 1

Ἐν Ἰνδοῖς μανθάνω σιττακοὺς ὄρνεις γίνεσθαι, 
ὧνπερ οὖν καὶ ἀνωτέρω μνήμην ἐποιησάμην· ἃ δὲ 
πρότερον ὑπὲρ αὐτῶν οὐκ εἶπον, ταῦτά μοι λεχθῆναι 
νῦν δοκεῖ πρεπωδέστατα. 



Claudius Aelianus Soph., De natura animalium 
Book 16, section 2, line 10

γίνονται δὲ καὶ ταῶς ἐν Ἰνδοῖς τῶν πανταχόθεν 
μέγιστοι, καὶ πελειάδες χλωρόπτιλοι· φαίη τις ἂν 
πρῶτον θεασάμενος καὶ οὐκ ἔχων ἐπιστήμην ὀρνι-
θογνώμονα, σιττακὸν εἶναι καὶ οὐ πελειάδα. 



Claudius Aelianus Soph., De natura animalium 
Book 16, section 2, line 22

                                          χρόαν δὲ ἔχει τὰ 
πτερὰ τῶν Ἰνδῶν ἀλεκτρυόνων χρυσωπόν τε καὶ κυα-
ναυγῆ κατὰ τὴν σμάραγδον λίθον. 



Claudius Aelianus Soph., De natura animalium 
Book 16, section 3, line 1

Γίνεται δὲ ἐν Ἰνδοῖς καὶ ἄλλο ὄρνεον, καὶ ἔχει 
τὸ μέγεθος κατὰ τοὺς ψᾶρας, καὶ ἔστι ποικίλον, καὶ 
μουσωθὲν ἀνθρώπου φωνὴν εἶτα μέντοι τῶν σιττα-
κῶν ἐστι λαλίστερόν τε καὶ θυμοσοφώτερον. 



Claudius Aelianus Soph., De natura animalium 
Book 16, section 3, line 8

               καλοῦσι δὲ αὐτὸ οἱ Μακεδόνων Ἰνδοῖς 
ἐποικήσαντες ἔν τε Βουκεφάλοις πόλει καὶ τῇ περὶ   
ταύτην καὶ τῇ καλουμένῃ Κύρου πόλει καὶ ταῖς ἄλλαις, 
ἃς ἀνέστησεν Ἀλέξανδρος ὁ Φιλίππου, κερκίωνα· 
ἔσχε δὲ ἄρα τὸ ὄνομα τήνδε τὴν γένεσιν, ἐπειδὴ καὶ 
αὐτὸ διασείει τὸν ὄρρον, ὥσπερ οὖν καὶ οἱ κίγκλοι. 



Claudius Aelianus Soph., De natura animalium 
Book 16, section 4, line 1

Γίνεσθαι δὲ ἐν Ἰνδοῖς καὶ κήλαν ἀκούω ὄρνιν· 
καὶ τὸ μέγεθος τριπλασίων ὠτίδος ἐστί, καὶ τὸ στόμα 
ἔχει γενναῖον δεινῶς καὶ μακρὰ τὰ σκέλη· φέρει δὲ 
καὶ πρηγορεῶνα καὶ ἐκεῖνον μέγιστον προσεμφερῆ 
κωρύκῳ, φθέγμα δὲ ἔχει καὶ μάλα ἀπηχές. 



Claudius Aelianus Soph., De natura animalium 
Book 16, section 5, line 1

Ἀκούω δὲ ἔγωγε καὶ Ἰνδὸν ἔποπα διπλασίονα 
τοῦ παρ' ἡμῖν καὶ ὡραιότερον ἰδεῖν. 



Claudius Aelianus Soph., De natura animalium 
Book 16, section 5, line 4

                                            καὶ Ὅμηρος 
μὲν λέγει βασιλεῖ κεῖσθαι ἄγαλμα Ἕλληνι χαλινὸν 
καὶ κόσμον ἵππου, ὁ δὲ ἔποψ οὗτος Ἰνδῶν βασιλεῖ 
ἄθυρμά ἐστι, καὶ διὰ χειρῶν αὐτὸν φέρει, καὶ ἥδεται 
αὐτῷ, καὶ συνεχὲς ἐνορᾷ τὴν ἀγλαΐαν τεθηπὼς τοῦ 
ὄρνιθος καὶ τὸ κάλλος τὸ αὐτοφυές. 



Claudius Aelianus Soph., De natura animalium 
Book 16, section 5, line 9

                                παῖς ἐγένετο Ἰνδῶν βασι-
λεῖ, καὶ ἀδελφοὺς εἶχεν, οἵπερ οὖν ἀνδρωθέντες ἐκ-
δικώτατοί τε γίνονται καὶ λεωργότατοι. 



Claudius Aelianus Soph., De natura animalium 
Book 16, section 5, line 39

ἔοικεν οὖν ἐξ Ἰνδῶν τὸ μυθολόγημα ἐπ' ἄλλου μὲν 
ὄρνιθος, ἐπιρρεῦσαι δ' οὖν καὶ τοῖς Ἕλλησιν. 



Claudius Aelianus Soph., De natura animalium 
Book 16, section 5, line 42

                                                        Ὠγύ-
γιον γάρ τι μῆκος χρόνου λέγουσι Βραχμᾶνες, ἐξ οὗ 
ταῦτα τῷ ἔποπι τῷ Ἰνδῷ ἔτι ἀνθρώπῳ ὄντι καὶ παιδὶ 
τήν γε ἡλικίαν ἐς τοὺς γειναμένους πέπρακται. 



Claudius Aelianus Soph., De natura animalium 
Book 16, section 6, line 1

Ἐν Ἰνδοῖς γίνεται ζῷον κροκοδείλῳ χερσαίῳ πα-
ραπλήσιον ἰδεῖν· μέγεθος δὲ αὐτῷ κυνιδίου Μελι-
ταίου εἴη ἄν. 



Claudius Aelianus Soph., De natura animalium 
Book 16, section 8, line 1

Ἡ δὲ Ἰνδῶν θάλαττα ὕδρους θαλαττίους τίκτει 
πλατεῖς τὰς οὐράς· τίκτουσι δὲ καὶ λίμναι μεγίστους 
ὕδρους. 



Claudius Aelianus Soph., De natura animalium 
Book 16, section 9, line 1

Ἐν Ἰνδοῖς ἵππων τε ἀγρίων καὶ ὄνων τοιούτων 
εἰσὶν ἀγέλαι. 



Claudius Aelianus Soph., De natura animalium 
Book 16, section 10, line 1

Ἐν Πρασίοις δὲ τοῖς Ἰνδικοῖς εἶναι γένος πιθή-
κων φασὶν ἀνθρωπόνουν, ἰδεῖν δέ εἰσι κατὰ τοὺς 
Ὑρκανοὺς κύνας τὸ μέγεθος, προκομία τε αὐτῶν ὁρᾶται 
συμφυής· εἴποι δ' ἂν ὁ μὴ τὸ ἀληθὲς εἰδὼς ἀσκητὰς 
εἶναι αὐτάς. 



Claudius Aelianus Soph., De natura animalium 
Book 16, section 10, line 12

                                                     φοι-
τῶσι δὲ ἀθρόοι ἐς τὰ τῆς Λατάγης προάστεια (πόλις 
δέ ἐστιν Ἰνδῶν ἡ Λατάγη), καὶ τὴν προτεθειμένην 
αὐτοῖς ἐκ βασιλέως ἑφθὴν ὄρυζαν σιτοῦνται· ἀνὰ 
πᾶσαν δὲ ἡμέραν ἥδε ἡ δαὶς αὐτοῖς εὐτρεπὴς πρό-  
κειται. 



Claudius Aelianus Soph., De natura animalium 
Book 16, section 11, line 1

Ποηφάγον ἐν Ἰνδοῖς ζῷόν ἐστι, καὶ πέφυκέ γε 
διπλάσιον ἵππου τὸ μέγεθος. 



Claudius Aelianus Soph., De natura animalium 
Book 16, section 11, line 5

                                 οὐρὰν δὲ ἔχει δασυ-
τάτην καὶ μελαίνης ἀκράτως χρόας, καὶ εἶεν αὗται 
αἱ τρίχες καὶ τῶν ἀνθρωπείων λεπτότεραι ἄν, καὶ ἐν 
μεγάλῳ τίθενται ταύτας ἔχειν Ἰνδῶν αἱ γυναῖκες· 
καὶ γάρ τοι παραπλέκονται ἐξ αὐτῶν καὶ κοσμοῦνται 
μάλα ὡραίως, ταῖς πλοκαμῖσι ταῖς συμφύτοις καὶ 
ταύτας ὑποδέουσαι. 



Claudius Aelianus Soph., De natura animalium 
Book 16, section 11, line 25

                          καὶ δείρας τὸ πᾶν σῶμα (ἀγα-
θὸν γὰρ καὶ ἡ δορά) ἀφῆκε τὸν νεκρόν· σαρκῶν γὰρ 
τῶν ἐκείνου δέονται Ἰνδοὶ οὐδὲ ἕν. 



Claudius Aelianus Soph., De natura animalium 
Book 16, section 12, line 1

Κήτη δὲ ἦν ἄρα ἐν τῇ τῶν Ἰνδῶν θαλάττῃ πεν-
ταπλασίονα τὸ μέγεθος ἐλέφαντος τοῦ μεγίστου. 



Claudius Aelianus Soph., De natura animalium 
Book 16, section 13, line 1

                καὶ αἱροῦσιν οἱ γεωργοῦντες αὐτοὺς 
ἀσθενεῖ τῇ νήξει χρωμένους, ἅτε μὴ ἐν βυθῷ φερο-
μένους ἀλλὰ ἐπιπολῆς, καὶ ἐκ τοῦ ὀλίγου ὕδατος 
ἀγαπητῶς καὶ μόλις ἀποζῶντας. 
 Ἰνδῶν δὲ ἰχθύων ἴδια καὶ ἐκεῖνα. 



Claudius Aelianus Soph., De natura animalium 
Book 16, section 13, line 3

                                             βατίδες γίνον-
ται παρ' αὐτοῖς οὐδέν τι μείους Ἀργολικῆς ἀσπίδος 
ἑκάστη, καρίδες δὲ καὶ μείζους καράβων αἱ Ἰνδῶν 
εἰσίν. 



Claudius Aelianus Soph., De natura animalium 
Book 16, section 13, line 7

         αἱ μὲν οὖν ἐκ τῆς θαλάττης ἀναθέουσαι διὰ 
τοῦ ποταμοῦ τοῦ Γάγγου χηλὰς μεγίστας ἔχουσι καὶ 
τραχείας θιγεῖν, τάς γε μὴν ἐκ τῆς Ἐρυθρᾶς ἐκπι-
πτούσας ἐς τὸν Ἰνδὸν λείας ἔχειν πέπυσμαι τὰς 
ἀκάνθας, προμήκεις γε μὴν καὶ βοστρυχώδεις τὰς 
ἀπηρτημένας ἕλικας. 



Claudius Aelianus Soph., De natura animalium 
Book 16, section 14, line 1


 Χελώνη δὲ ἐν Ἰνδοῖς ποταμία τὸ χελώνιον ἔχει 
σκάφης οὐ μεῖον τελείας. 



Claudius Aelianus Soph., De natura animalium 
Book 16, section 15, line 2

Θυμόσοφα δὲ καὶ παρ' ἡμῖν ζῷά ἐστιν, οὐ μὴν 
ὅσα ἐν Ἰνδοῖς ἀλλὰ ὀλίγα. 



Claudius Aelianus Soph., De natura animalium 
Book 16, section 15, line 5

                                    ἐκεῖ δὲ ὅ τε ἐλέφας τοι-
οῦτός ἐστι καὶ ὁ σιττακὸς καὶ αἱ σφίγγες καὶ οἱ κα-
λούμενοι σάτυροι· σοφὸν δὲ ἄρα ἦν καὶ ὁ μύρμηξ ὁ 
Ἰνδός. 



Claudius Aelianus Soph., De natura animalium 
Book 16, section 15, line 9

          οἱ μὲν οὖν ἡμεδαποὶ τὰς ἑαυτῶν χειὰς καὶ 
ὑποδρομὰς ὑπὸ τὴν γῆν ὀρύττουσι, καὶ φωλεούς 
τινας κρυπτοὺς ἀποφαίνουσι γεωρυχοῦντες, καὶ με-
ταλλείαις ὡς εἰπεῖν τισιν ἀπορρήτοις καὶ λανθανού-
σαις καταξαίνονται· ἀλλὰ οἵ γε Ἰνδοὶ μύρμηκες οἰκί-
σκους τινὰς συμφορητοὺς ἐργάζονται, καὶ τούτους γε 
οὐκ ἐν χωρίοις ὑπτίοις καὶ λείοις καὶ ἐπικλυζομένοις 
ῥᾷστα, ἀλλὰ μετεώροις καὶ ὑψηλοῖς. 



Claudius Aelianus Soph., De natura animalium 
Book 16, section 15, line 31

                           καὶ μυρμήκων μὲν Ἰνδῶν 
πέρι Ἰόβᾳ πάλαι, ἐμοὶ δὲ νῦν ἐς τοσοῦτον λελέχθω. 



Claudius Aelianus Soph., De natura animalium 
Book 16, section 16, line 1

Παρὰ τοῖς Ἀριανοῖς τοῖς Ἰνδικοῖς χάσμα Πλού-
τωνός ἐστι, καὶ κάτω τινὲς ἀπόρρητοι σύριγγες καὶ 
ὁδοὶ κρυπταὶ καὶ διαδρομαὶ ἀνθρώποις μὲν ἀθέατοι, 
βαθεῖαι δ' οὖν καὶ ἐπὶ μήκιστον προήκουσαι· γενό-
μεναι δὲ πῶς καὶ ὀρωρυγμέναι τρόπῳ τῷ, οὔτε Ἰνδοὶ 
λέγουσιν, οὔτε ἐγὼ μαθεῖν πολυπραγμονῶ. 



Claudius Aelianus Soph., De natura animalium 
Book 16, section 16, line 7

                                               ἄγουσιν 
οὖν Ἰνδοὶ καὶ ὑπὲρ τὰ τρισμύρια ἐνταῦθα κτήνη 
προβάτων τε καὶ αἰγῶν καὶ βοῶν καὶ ἵππων· καὶ 
ἕκαστος τῶν ἢ ἐνύπνιον ἢ ὄτταν τινὰ ἢ φήμην ἢ 
ὄρνιν οὐκ εὔεδρον ὑφορωμένων ἀντὶ τῆς ἑαυτοῦ 
ζωῆς ἐμβάλλει κατὰ τὴν οἴκοθεν δύναμιν, ἑαυτὸν 
λυτρούμενος καὶ διδοὺς ὑπὲρ τῆς ἑαυτοῦ ψυχῆς τὴν 
τοῦ ζῴου. 



Claudius Aelianus Soph., De natura animalium 
Book 16, section 20, line 1

Ἐν τοῖς χωρίοις τοῖς ἐν Ἰνδίᾳ (λέγω δὲ τοῖς ἐν-
δοτάτω) ὄρη φασὶν εἶναι δύσβατά τε καὶ ἔνθηρα, καὶ 
ἔχειν ζῷα ὅσα καὶ ἡ καθ' ἡμᾶς τρέφει γῆ, ἄγρια δέ· 
καὶ γάρ τοι καὶ τὰς οἶς τὰς ἐκεῖ φασιν εἶναι καὶ ταύ-
τας θηρία, καὶ κύνας καὶ αἶγας καὶ βοῦς, αὐτόνομά 
τε ἀλᾶσθαι καὶ ἐλεύθερα, ἀφειμένα νομευτικῆς ἀρχῆς. 



Claudius Aelianus Soph., De natura animalium 
Book 16, section 20, line 8

πλήθη δὲ αὐτῶν καὶ ἀριθμοῦ πλείω φασὶν οἱ τῶν 
Ἰνδῶν λόγιοι. 



Claudius Aelianus Soph., De natura animalium 
Book 16, section 20, line 34

         εἶτα ταύτης παραδραμούσης καὶ τῆς θηλείας   
κυούσης, ἐκθηριοῦται αὖθις, καὶ μονίας ἐστὶν ὅδε 
ὁ Ἰνδὸς καρτάζωνος. 



Claudius Aelianus Soph., De natura animalium 
Book 16, section 21, line 2

Ὑπερελθόντι τὰ ὄρη τὰ γειτνιῶντα τοῖς Ἰν-
δοῖς κατὰ τὴν ἐνδοτάτω πλευρὰν φανοῦνταί φασιν 
αὐλῶνες δασύτατοι, καὶ καλεῖταί γε ὑπ' Ἰνδῶν 
ὁ χῶρος Κόλουνδα. 



Claudius Aelianus Soph., De natura animalium 
Book 16, section 22, line 1

Ἔστι δὲ καὶ Σκιρᾶται πέραν Ἰνδῶν ἔθνος καὶ 
τοῦτο, καὶ εἰσὶ σιμοὶ τὰς ῥῖνας, εἴτε οὕτως ἐκ βρε-
φῶν ἁπαλῶν ἐνθλάσει τῇ τῆς ῥινὸς διαμείναντες, 
εἴτε καὶ τοῦτον τὸν τρόπον τίκτονται. 



Claudius Aelianus Soph., De natura animalium 
Book 16, section 31, line 1

Λέγει δὲ ἄρα Κτησίας ἐν λόγοις Ἰνδικοῖς τοὺς   
καλουμένους Κυναμολγοὺς τρέφειν κύνας πολλοὺς 
κατὰ τοὺς Ὑρκανοὺς τὸ μέγεθος, καὶ εἶναί γε ἰσχυ-
ρῶς κυνοτρόφους. 



Claudius Aelianus Soph., De natura animalium 
Book 16, section 33, line 5

                                 Λιβύων δὲ ἄρα τῶν 
γειτνιώντων Ἰνδοῖς ὀπισθονόμων βοῶν ἀγέλας εἶναί 
τινας ἀκούω. 



Claudius Aelianus Soph., De natura animalium 
Book 16, section 35, line 2

                                                      ὅσπερ   
οὖν ἐν Ἰνδοῖς λόγοις φησὶ Κώυθα μὲν οὕτως εἶναι 
κώμην τὸ ὄνομα λαβοῦσαν, ταῖς δὲ αἰξὶ ταῖς ἐπιχω-
ρίοις ἔνδον ἐν τοῖς σηκοῖς παραβάλλειν τοὺς νομέας 
ἰχθῦς ξηροὺς χιλόν. 



Claudius Aelianus Soph., De natura animalium 
Book 16, section 37, line 1

Παρά γε τοῖς Ψύλλοις καλουμένοις τῶν Ἰνδῶν 
(εἰσὶ γὰρ καὶ Λιβύων ἕτεροι) οἱ ἵπποι γίνονται τῶν 
κριῶν οὐ μείζους, καὶ τὰ πρόβατα ἰδεῖν μικρὰ κατὰ 
τοὺς ἄρνας, καὶ οἱ ὄνοι δὲ τοσοῦτοι γίνονται τὸ μέ-
γεθος καὶ οἱ ἡμίονοι καὶ οἱ βοῦς καὶ πᾶν κτῆνος 
ἕτερον ὅ τι οὖν. 



Claudius Aelianus Soph., De natura animalium 
Book 16, section 37, line 6

                       ὗν δὲ ἐν Ἰνδοῖς οὔ φασι γίνεσθαι 
οὔτε ἥμερον οὔτε ἄγριον· μυσάττονται δὲ καὶ ἐσθίειν 
τοῦδε τοῦ ζῴου Ἰνδοί, καὶ οὐκ ἂν γεύσαιντό ποτε 
ὑείων, ὥσπερ οὖν οὐδὲ ἀνθρωπείων οἱ αὐτοί. 



Claudius Aelianus Soph., De natura animalium 
Book 16, section 39, line 1

Ὀνησίκριτος ὁ Ἀστυπαλαιεὺς λέγει ἐν Ἰνδοῖς 
κατὰ τὴν Ἀλεξάνδρου τοῦ παιδὸς Φιλίππου ἀνάβασιν 
γενέσθαι δράκοντας δύο, οὓς Ἀβισάρης ὁ Ἰνδὸς ἔτρε-
φεν, ὧν ὃ μὲν ἦν πήχεων τετταράκοντα καὶ ἑκατόν, 
ὃ δὲ ὀγδοήκοντα· καί φησι ἐπιθυμῆσαι δεινῶς Ἀλέ-
ξανδρον θεάσασθαι αὐτούς. 



Claudius Aelianus Soph., De natura animalium 
Book 16, section 41, line 1

Μεγασθένης φησὶ κατὰ τὴν Ἰνδικὴν σκορπίους 
γίνεσθαι πτερωτοὺς μεγέθει μεγίστους, τὸ κέντρον 
δὲ ἐγχρίμπτειν τοῖς Εὐρωπαίοις παραπλησίως. 



Claudius Aelianus Soph., De natura animalium 
Book 17, section 2, line 1

Κλείταρχος ἐν τῇ περὶ τὴν Ἰνδικήν φησι γίνεσθαι 
ὄφεις πήχεων ἑκκαίδεκα. 



Claudius Aelianus Soph., De natura animalium 
Book 17, section 6, line 21

περὶ δὲ τὴν Γεδρωσίων χώραν (ἔστι δὲ μοῖρα τῆς 
γῆς τῆς Ἰνδικῆς οὐκ ἄδοξος) Ὀνησίκριτος λέγει καὶ 
Ὀρθαγόρας γίνεσθαι κήτη ἥμισυ ἔχοντα σταδίου τὸ 
μῆκος. 



Claudius Aelianus Soph., De natura animalium 
Book 17, section 22, line 2

                                   λέγει δὲ Κλεί-  
ταρχος ἐν Ἰνδοῖς γίνεσθαι ὄρνιν, καὶ εἶναι σφόδρα 
ἐρωτικόν, καὶ τὸ ὄνομα αὐτοῦ λέγει ὠρίωνα εἶναι. 



Claudius Aelianus Soph., De natura animalium 
Book 17, section 23, line 1

Κατρέα τὸ ὄνομα, Ἰνδὸν τὸ γένος, τῇ φύσει ὄρνιν 
λέγει Κλείταρχος εἶναι τὸ κάλλος ὑπερήφανον· τὸ μέ-
γεθος γὰρ εἴη ἂν κατὰ τὸν ταῶν, τὰ δὲ ἄκρα τῶν 
πτερῶν ἔοικε σμαράγδῳ. 



Claudius Aelianus Soph., De natura animalium 
Book 17, section 23, line 13

                           ἔχει δὲ καὶ φώνημα εὔμουσον 
καὶ κατὰ τὴν ἀηδόνα τορόν. Ἰνδοὶ δὲ ἄρα τὴν ἐξ 
ὀρνίθων τροφὴν ** εἶχον, ἵνα καὶ οἱ ὁρῶντες ἑστιᾶν 
τὴν ὄψιν δύνωνται. 



Claudius Aelianus Soph., De natura animalium 
Book 17, section 25, line 1

Λέγει δὲ Κλείταρχος πιθήκων ἐν Ἰνδοῖς εἶναι 
γένη ποικίλα τὴν χρόαν, μεγέθει δὲ μέγιστα. 



Claudius Aelianus Soph., De natura animalium 
Book 17, section 25, line 20

κατόπτρῳ δὲ χρησάμενος ὁ Ἰνδὸς ὁρώντων ἐκείνων, 
οὐκ εἰσὶ δ' ἔτι τὰ κάτοπτρα, ἀλλὰ ἕτερα προστιθέν-
τες· εἶτα καὶ τούτοις ἕρματα ἰσχυρὰ ὑποπλέκουσι· καὶ 
μέντοι καὶ τοιαῦτά ἐστιν. 



Claudius Aelianus Soph., De natura animalium 
Book 17, section 25, line 28

           εἴρηται μὲν ὑπὲρ πιθήκων καὶ ἄλλα, Ἰνδῶν 
τε καὶ οὐκ Ἰνδῶν· καὶ ταῦτα δὲ ἔχει τινὰ τῷ συνιέντι 
οὐκ ἀσπούδαστα, οὐ μὰ Δία. 



Claudius Aelianus Soph., De natura animalium 
Book 17, section 26, line 1

Λέοντας ἐν Ἰνδοῖς γίνεσθαι μεγίστους οὐ δια-
πορῶ· τὸ δὲ αἴτιον, τῶν ζῴων τῶν ἑτέρων ἥδε ἡ γῆ 
μήτηρ ἐστὶν ἀγαθή. 



Claudius Aelianus Soph., De natura animalium 
Book 17, section 29, line 1

Τοῦ Ἰνδῶν βασιλέως ἐλαύνοντος ἐπὶ τοὺς πολε-
μίους δέκα μυριάδες ἐλεφάντων προηγοῦνται μαχί-
μων. 



Claudius Aelianus Soph., De natura animalium 
Book 17, section 29, line 11

         ἰδεῖν δὲ ἐν Βαβυλῶνι ὁ αὐτὸς λέγει τοὺς 
φοίνικας αὐτορρίζους ἀνατρεπομένους ὑπὸ τῶν ἐλε-
φάντων τὸν αὐτὸν τρόπον, ἐμπιπτόντων τῶν θηρίων 
αὐτοῖς βιαιότατα· δρῶσι δὲ ἄρα, ἂν ὁ Ἰνδὸς ὁ πω-
λεύων αὐτοὺς κελεύσῃ δρᾶσαι τοῦτο αὐτοῖς. 



Claudius Aelianus Soph., De natura animalium 
Book 17, section 33, line 9

Κάσπιος δὲ ἄρα καὶ οὗτος ὄρνις ἢ Ἰνδὸς μᾶλλον 
(λέγεται γὰρ καὶ ἐκείνῃ τὸ γένος οἱ καὶ ταύτῃ), καὶ 
εἴη τὸ μέγεθος κατὰ χῆνα ἄν. 



Claudius Aelianus Soph., De natura animalium 
Book 17, section 35, line 13

     λέγει δὲ ὁ Ἀντήνωρ καὶ ἔτι κατὰ τὴν Ἴδην τὴν 
Κρῆσσαν ἐκείνου τοῦ γένους τῶν μελιττῶν εἶναι ἰν-
δάλματα, οὐ πολλὰ μέν, εἶναι δ' οὖν, καὶ πικρὰ ἐν-
τυχεῖν, ὡς ἐκεῖναι ἦσαν. 



Claudius Aelianus Soph., De natura animalium 
Book 17, section 39, line 1

Ἐν τῇ Πρασιακῇ χώρᾳ, (Ἰνδῶν δὲ αὕτη ἐστί) 
Μεγασθένης φησὶ πιθήκους εἶναι τῶν μεγίστων 
κυνῶν οὐ μείους, ἔχειν δὲ οὐρὰς πήχεων πέντε· προς-
πεφυκέναι δὲ ἄρα αὐτοῖς καὶ προκόμια καὶ πώγωνας 
καθειμένους καὶ βαθεῖς· καὶ τὸ μὲν πρόσωπον πᾶν 
εἶναι λευκούς, τὸ σῶμα δὲ μέλανας ἰδεῖν, ἡμέρους δὲ 
καὶ φιλανθρωποτάτους, καὶ τὸ τοῖς ἀλλαχόθι πιθήκοις 
συμφυὲς οὐκ ἔχειν τὸ κακόηθες. 



Claudius Aelianus Soph., De natura animalium 
Book 17, section 40, line 1

Ἐν Ἰνδοῖς ἐστι χώρα περὶ τὸν Ἀσταβόρραν πο-
ταμὸν ἐν τοῖς καλουμένοις Ῥιζοφάγοις. 



Claudius Aelianus Soph., De natura animalium 
Book 17, section 40, line 6

                          κατὰ μέντοι τὴν λίμνην τὴν 
καλουμένην Ἀορατίαν (Ἰνδῶν δὲ ἄρα καὶ αὕτη· 
πλησίον δέ ἐστι τοῦ προειρημένου ποταμοῦ) τοῦτο 
μὲν τὸ θηρίον τὸν κώνωπα ἐπιπολάζειν· ἔρημον δὲ 
καὶ εἶναι τὸν χῶρον καὶ καλεῖσθαι. 



Claudius Aelianus Soph., De natura animalium 
Book 17, section 40, line 10

                                          τὴν δὲ αἰτίαν 
ἐκείνην Ἰνδοί φασιν οἱ κύκλῳ περιοικοῦντες, τὸν 
χῶρον τὸν προειρημένον οὐκ ἄνωθεν οὐδὲ ἐξ ἀρχῆς 
ἄγονον ἀνθρώπων γενέσθαι, σκορπίους δὲ ἐπιπολάσαι 
πλῆθος ἄμαχον, καὶ φαλαγγίων τινὰ ἐπιφοιτῆσαι   
φοράν, φαλαγγίων δὲ ἃ καλοῦσι τετράγναθα. 



Claudius Aelianus Soph., Varia historia (0545: 002)
“Claudii Aeliani de natura animalium libri xvii, varia historia, epistolae, fragmenta, vol. 2”, Ed. Hercher, R.
Leipzig: Teubner, 1866, Repr. 1971.
Book 1, section 15, line 21

                                 εἰ δέ τι Καλλιμάχῳ χρὴ 
προσέχειν, φάτταν καὶ πυραλλίδα καὶ περιστερὰν 
καὶ τρυγόνα φησὶ μηδὲν ἀλλήλαις ἐοικέναι. Ἰνδοὶ δέ 
φασι λόγοι περιστερὰς ἐν Ἰνδοῖς γίνεσθαι μηλίνας 
τὴν χρόαν. 



Claudius Aelianus Soph., Varia historia 
Book 1, section 15, line 22

                                                  Ἰνδοὶ δέ 
φασι λόγοι περιστερὰς ἐν Ἰνδοῖς γίνεσθαι μηλίνας 
τὴν χρόαν. 



Claudius Aelianus Soph., Varia historia 
Book 2, section 31, line 7

                                              οὐδεὶς γοῦν 
ἔννοιαν ἔλαβε τοιαύτην, οἵαν Εὐήμερος ὁ Μεσσήνιος 
ἢ Διογένης ὁ Φρὺξ ἢ Ἵππων ἢ Διαγόρας ἢ Σωσίας 
ἢ Ἐπίκουρος οὔτε Ἰνδὸς οὔτε Κελτὸς οὔτε Αἰγύπτιος. 



Claudius Aelianus Soph., Varia historia 
Book 2, section 41, line 15

         καὶ Ἀλέξανδρος δὲ ὁ Μακεδὼν ἐπὶ Καλανῷ τῷ 
Βραχμᾶνι, τῷ Ἰνδῶν σοφιστῇ, ὅτε ἑαυτὸν ἐκεῖνος 
κατέπρησεν, ἀγῶνα μουσικῆς καὶ ἱππέων καὶ ἀθλη-
τῶν διέθηκε. 



Claudius Aelianus Soph., Varia historia 
Book 2, section 41, line 17

               χαριζόμενος δὲ τοῖς Ἰνδοῖς καί τι ἐπι-
χώριον αὐτῶν ἀγώνισμα ἐς τιμὴν τοῦ Καλανοῦ συγ-
κατηρίθμησε τοῖς ἄθλοις τοῖς προειρημένοις. 



Claudius Aelianus Soph., Varia historia 
Book 3, section 23, line 5

Καλὰ μὲν οὖν Ἀλεξάνδρου τὰ ἐπὶ Γρανίκῳ καὶ 
τὰ ἐπὶ Ἰσσῷ καὶ ἡ πρὸς Ἀρβήλοις μάχη καὶ Δαρεῖος 
ἡττημένος καὶ Πέρσαι δουλεύοντες Μακεδόσι· καλὰ 
δὲ καὶ τὰ τῆς ἄλλης ἁπάσης Ἀσίας νενικημένης, καὶ 
Ἰνδοὶ δὲ καὶ οὗτοι Ἀλεξάνδρῳ πειθόμενοι· καλὸν δὲ 
καὶ τὸ πρὸς τῇ Τύρῳ καὶ τὰ ἐν Ὀξυδράκαις καὶ τὰ 
ἄλλα αὐτοῦ. 



Claudius Aelianus Soph., Varia historia 
Book 3, section 29, line 8

καὶ ὅμως ἐπὶ τούτοις μέγα ἐφρόνει οὐδὲν ἧττον ἢ 
Ἀλέξανδρος ἐπὶ τῇ τῆς οἰκουμένης ἀρχῇ, ὅτε καὶ 
Ἰνδοὺς ἑλὼν ἐς Βαβυλῶνα ὑπέστρεψεν. 



Claudius Aelianus Soph., Varia historia 
Book 3, section 39, line 3

Ὅτι βαλάνους Ἀρκάδες, Ἀργεῖοι δ' ἀπίους, Ἀθη-
ναῖοι δὲ σῦκα, Τιρύνθιοι δὲ ἀχράδας δεῖπνον εἶχον, 
Ἰνδοὶ καλάμους, Καρμανοὶ φοίνικας, κέγχρον δὲ 
Μαιῶται καὶ Σαυρομάται, τέρμινθον δὲ καὶ κάρδα-
μον Πέρσαι. 



Claudius Aelianus Soph., Varia historia 
Book 4, section 1, line 9

Ὅτι Δαρδανεῖς τοὺς ἀπὸ τῆς Ἰλλυρίδος ἀκούω 
τρὶς μόνον λούεσθαι παρὰ πάντα τὸν ἑαυτῶν βίον, 
ἐξ ὠδίνων καὶ γαμοῦντας καὶ ἀποθανόντας. 
 Ἰνδοὶ οὔτε δανείζουσιν οὔτε ἴσασι δανείζεσθαι. 



Claudius Aelianus Soph., Varia historia 
Book 4, section 1, line 10

ἀλλ' οὐδὲ θέμις ἄνδρα Ἰνδὸν οὔτε ἀδικῆσαι οὔτε 
ἀδικηθῆναι. 



Claudius Aelianus Soph., Varia historia 
Book 4, section 20, line 6

                       ἧκεν οὖν καὶ πρὸς τοὺς Χαλ-
δαίους ἐς Βαβυλῶνα καὶ πρὸς τοὺς μάγους καὶ τοὺς 
σοφιστὰς τῶν Ἰνδῶν. 



Claudius Aelianus Soph., Varia historia 
Book 5, section 6, line 1

Ἄξιον δὲ καὶ τὸ Καλανοῦ τοῦ Ἰνδοῦ τέλος ἐπαινέ-
σαι· ἄλλος δ' ἂν εἶπεν ὅτι καὶ ἀγασθῆναι. 



Claudius Aelianus Soph., Varia historia 
Book 5, section 6, line 3

               Καλανὸς ὁ Ἰνδῶν σοφιστὴς μακρὰ χαί-
ρειν φράσας Ἀλεξάνδρῳ καὶ Μακεδόσι καὶ τῷ βίῳ, 
ὅτε ἐβουλήθη ἀπολῦσαι αὑτὸν ἐκ τῶν τοῦ σώματος 
δεσμῶν, ἐνένηστο μὲν ἡ πυρὰ ἐν τῷ καλλίστῳ προ-
αστείῳ τῆς Βαβυλῶνος, καὶ ἦν τὰ ξύλα αὖα καὶ πρὸς 
εὐωδίαν εὖ μάλα ἐπίλεκτα κέδρου καὶ θύου καὶ κυ-
παρίττου καὶ μυρσίνης καὶ δάφνης· αὐτὸς δὲ γυμνα-
σάμενος γυμνάσιον τὸ εἰωθός (ἦν δὲ αὐτὸ δρόμος), 
ἀνελθὼν ἐπὶ μέσης τῆς πυρᾶς ἔστη ἐστεφανωμένος 
καλάμου κόμῃ. 



Claudius Aelianus Soph., Varia historia 
Book 6, section 14, line 14

                          ὃ δὲ διέστησεν ἄλλους ἄλλῃ, 
καὶ τοὺς μὲν ἐπὶ τὰ τῆς Ἰνδικῆς ὅρια ἀπέπεμψε, τοὺς 
δὲ ἐπὶ τὰ Σκυθικά. 



Claudius Aelianus Soph., Varia historia 
Book 7, section 18, line 3

                                       παρὰ Ἰνδοῖς δὲ αἱ 
γυναῖκες τὸ αὐτὸ πῦρ ἀποθανοῦσι τοῖς ἀνδράσιν ὑπο-
μένουσι. 



Claudius Aelianus Soph., Varia historia 
Book 8, section 7, line 18

                      ἦσαν δὲ καὶ ἐκ τῆς Ἰνδικῆς θαυ-
ματοποιοὶ διαπρέποντες, καὶ ἔδοξαν δὲ αὐτοὶ κρατεῖν 
τῶν ἄλλων τῶν ἀλλαχόθεν. 



Claudius Aelianus Soph., Varia historia 
Book 10, section 14, line 3

                καὶ μαρτύριον ἔλεγεν ἀνδρειοτάτους καὶ 
ἐλευθεριωτάτους Ἰνδοὺς καὶ Πέρσας, ἀμφοτέρους δὲ 
πρὸς χρηματισμὸν ἀργοτάτους εἶναι· Φρύγας δὲ καὶ 
Λυδοὺς ἐργαστικωτάτους, δουλεύειν δέ. 



Claudius Aelianus Soph., Varia historia 
Book 12, section 48, line 1

Ὅτι Ἰνδοὶ τῇ παρά σφισιν ἐπιχωρίῳ φωνῇ τὰ 
Ὁμήρου μεταγράψαντες ᾄδουσιν οὐ μόνοι ἀλλὰ καὶ 
οἱ Περσῶν βασιλεῖς, εἴ τι χρὴ πιστεύειν τοῖς ὑπὲρ 
τούτων ἱστοροῦσιν. 



Claudius Aelianus Soph., Fragmenta (0545: 004)
“Claudii Aeliani de natura animalium libri xvii, varia historia, epistolae, fragmenta, vol. 2”, Ed. Hercher, R.
Leipzig: Teubner, 1866, Repr. 1971.
Fragment 85, line 1

οἱ ἄνεμοι οἱ σκληροί τε καὶ ἐχθροὶ παραχρῆμα 
ἐκόπασαν, καὶ τὸ κῦμα ἐστορέσθη· πνεῦμα δὲ κεκρι-
μένον κατὰ πρύμναν ἐπέρρει, καὶ τὰ ἱστία ἐπλήρου. 
 ἰνδάλματα εἰκασμένα κυσὶν ἔκ τινος θείας ὁρμῆς 
τοῦτον ἐπιπηδήσαντα, οἰκτρῶς, ὅσα ἰδεῖν, διέξυεν. 



Claudius Aelianus Soph., Fragmenta 
Fragment 99, line 8

                                                          καὶ 
ἔστι τὸ ἴνδαλμα τοῦ πάθους μάλα ἐναργές. 



Claudius Aelianus Soph., Fragmenta 
Fragment 106, line 18

ὄψις οὖν ἰνδάλματος ἱεροῦ ὄναρ ἐπιστᾶσα λέγει 
ποιήσασθαι μεταβολὴν βίου. 



Procopius Rhet., Scr. Eccl., Catena in Canticum canticorum (2598: 002); MPG 87.2.
Page 1561, line 54


μὲν διάνοιαν κατωτέραν τῆς ἀληθείας κατανοήσεως· 
πάντα δὲ λόγον ἑρμηνευτικὸν στιγμὴν βραχεῖαν δο-
κεῖν, μὴ δυνάμενον τῷ πλάτει τῆς διανοίας ἐπεκτεί-
νεσθαι· τὴν οὖν διὰ τῶν τοιούτων ῥημάτων χειρ-
αγωγουμένην ψυχὴν πρὸς τὴν τῶν ἀλήπτων περί-
νοιαν, διὰ μόνης πίστεως εἰσοικίζειν ἐν ἑαυτῇ λέγει 
δεῖν τὴν πάντα νοῦν ὑπερέχουσαν φύσιν· καὶ τοῦτό 
ἐστι τὸ παρὰ τῶν φίλων λεγόμενον, ὅτι Σοὶ ποιήσω-
μεν, ὦ ψυχὴ, τῇ καλῶς πρὸς τὸν ἵππον ἀπεικασθείσῃ, 
ἰνδάλματά τινα τῆς ἀληθείας καὶ ὁμοιώματα· τοι-
αύτη γὰρ καὶ τοῦ τῶν λόγων ἀργυρίου ἡ δύναμις   
ὡς ἐναυγάσματα σπινθηροειδῆ δοκεῖν εἶναι τὰ 
ῥήματα, μὴ δυνάμενα δι' ἀκριβείας ἐμφῆναι τὸ 
ἐγκείμενον νόημα· σὺ δὲ ταῦτα δεξαμένη, ὑποζύ-
γιόν τε καὶ οἰκητήριον γενήσῃ διὰ πίστεως, τοῦ ἐν 
σοὶ ἀνακλίνεσθαι μέλλοντος διὰ τῆς ἐν σοὶ κατοική-
σεως· τοῦ γὰρ αὐτοῦ καὶ θρόνος ἔσῃ καὶ οἶκος γε-
νήσῃ· ταῦτα τῶν φίλων τοῦ νυμφίου τῇ καθαρᾷ καὶ 
παρθένῳ χαρισαμένων ψυχῇ, εἶεν δ' ἂν οὗτοι τὰ 




Procopius Rhet., Scr. Eccl., Commentarii in Isaiam (2598: 004); MPG 87.2.
Page 2084, line 37

           Σουφεὶρ δὲ χώρα τῆς Ἰνδικῆς, ἔνθα γίνε-
σθαι τὰς τιμιωτάτας λίθους φησί. 



Procopius Rhet., Scr. Eccl., Commentarii in Isaiam 
Page 2628, line 19

                                   Θαυμαζούσης δὲ τῆς 
Ἐκκλησίας, τὴν τοῦ πράγματος αἰτίαν ὁ σαγηνεύων 
φησίν· Ἐμὲ νῆσοι ὑπέμειναν, τοὺς νησιώτας λέ-
γων, καὶ πλοῖα Θαρσεῖς. Τὰς Ἰνδικὰς δὲ χώρας 
Θαρσεῖς ἡ Γραφὴ καλεῖ. 



Procopius Rhet., Scr. Eccl., Commentarii in Isaiam 
Page 2629, line 19

Πλοῖα δὲ Θαρσεῖς καλεῖ, τοὺς ἀπὸ τῆς Θαρ-
σεῖς ἐρχομένους ἐν Ἰνδίᾳ κειμένης ἧς καὶ ἐν 
Ἰωνίᾳ μέμνηται. 



Procopius Rhet., Scr. Eccl., Epistulae 1–166 (2598: 005)
“Procopii Gazaei epistolae et declamationes”, Ed. Garzya, A., Loenertz, R.–J.
Ettal: Buch–Kunstverlag, 1963; Studia patristica et Byzantina 9.
Epistle 165, line 8

                               ὁ δὲ μισθὸς οὐ μὰ Δία χρυσὸς οὐδὲ λίθοι τινὲς 
Ἰνδικαί – οὔτε γὰρ τούτων πλουτῶ, οὔτε τούτων θηρεύσων ὁ νέος ἀφῖκται 
– ἀλλ' οὐδὲ λόγων κάλλος – οὐ γὰρ Μουσῶν εὔφορος ἐγώ, οὐδὲ τοῖς ἐξ 
Ἀττικῆς ἐναβρύνομαι, ταῦτα γὰρ εὐδαιμόνων εὐτύχησαν παῖδες – ἀλλ' εἰ 
τὴν ἐμὴν δόσιν ἥτις ἐστὶν ἐθέλεις σκοπεῖν, εὔνοιά τε καὶ προθυμία· τούτων 
γὰρ κύριος ἐγώ, Δημοσθένης φησίν, τῶν δὲ ἄλλων τύχη καὶ Μοῦσαι πρὸς 
τὸ δοκοῦν αὐταῖς τὴν δωρεὰν πρυτανεύουσαι. 



Procopius Rhet., Scr. Eccl., Descriptio imaginis (2598: 009)
“Spätantiker Gemäldezyklus in Gaza”, Ed. Friedländer, P.
Vatican City: Biblioteca Apostolica Vaticana, 1939; Studi e Testi 89.
Section 22, line 7

τοσοῦτον γὰρ ἡ κεκτημένη περίκειται κόσμον, ὅσον τῇ κατ' οἶκον 
ἁρμόσει διαίτῃ, ἀμφιδέτας τε καὶ περιβραχιόνια· ὅρμοι τε περὶ τῇ 
δέρῃ καὶ τοῖς ὠσὶν ἑλικτῆρες καὶ χρυσῆ ταινία τὴν κεφαλὴν περισφίγ-
γουσα Ἰνδικῶν λίθων διαδοχῇ περικλείεται. 



Menander Rhet., Διαίρεσις τῶν ἐπιδεικτικῶν (olim sub auctore Genethlio) (2586: 001)
“Menander rhetor”, Ed. Russell, D.A., Wilson, N.G.
Oxford: Clarendon Press, 1981.
Spengel page 358, line 14

                         ἐπὶ πένθει δὲ καὶ οἴκτῳ, <ὡς> 
ἱστοροῦσι Βουκέφαλον τὴν ἐν Ἰνδοῖς πόλιν ἐπὶ τῷ 
ἵππῳ τοῦ Ἀλεξάνδρου τῷ Βουκεφάλῳ ἀνοικισθῆναι· 
τὴν Ἀντινόου δὲ ἐν Αἰγύπτῳ <ἐπὶ τῷ> Ἀντινόου θανάτῳ 
ὑπὸ Ἀδριανοῦ. 



Etymologicum Genuinum, Etymologicum genuinum (ἀνάβλησις – βώτορες) (4097: 002)
“Etymologicum magnum genuinum. Symeonis etymologicum una cum magna grammatica. Etymologicum magnum auctum, vol. 2”, Ed. Lasserre, F., Livadaras, N.
Athens: Parnassos Literary Society, 1992.
Alphabetic letter alpha, entry 1332, line 4

                         πhilox. l. c.   
 <Ἀσχάλλων> (Eur. Or. 785)· ἀδημονῶν, λυπούμενος, 
χαλεπαίνων, ἢ ἀγανακτῶν· παρὰ τὸ ἄχω, ἀφ' οὗ ἄχομαι, οἷον (σ 256)· 
  νῦν δ' ἄχομαι· τόσα γάρ μοι ἐπέσευεν κακά, 
γίνεται ἀχάλλω, ὥσπερ ἄγω ἀγάλλω, εἴδω εἰδάλλω καὶ ἰνδάλλω, καὶ 
πλεονασμῷ τοῦ <σ> ἀσχάλλω· ἐκ δὲ τοῦ ἀσχάλλω γίνεται περισπώ-
μενον ῥῆμα ἀσχαλῶ, τὸ τρίτον πρόσωπον ἀσχαλᾷ, καὶ πλεονασμῷ   
τοῦ <α> ἀσχαλάᾳ, ὡς παρ' Ὁμήρῳ (Β 292 – 293)· 
  καὶ γάρ τίς θ' ἕνα μῆνα μένων ἀπὸ ἧς ἀλόχοιο 
  ἀσχαλάᾳ. 



Scholia In Pindarum, Scholia in Pindarum (scholia vetera) (5034: 001)
“Scholia vetera in Pindari carmina, 3 vols.”, Ed. Drachmann, A.B.
Leipzig: Teubner, 1:1903; 2:1910; 3:1927, Repr. 1:1969; 2:1967; 3:1966.
Ode O 3, scholion 52a, line 13

               ὅτι δὲ συνέβαινε καὶ εἰκός ἐστιν ἐνίας ἔχειν, 
ἐκεῖθεν δῆλον, ὅτι τῶν ἐλεφάντων οἱ μὲν ἐξ Αἰθιοπίας καὶ 
Λιβύης πάντες σὺν ταῖς θηλείαις ὀδόντας ἔχουσιν, ἢ κέρατα, 
ὥς τινες· καθὰ καὶ Ἀμυντιανὸς (Script. rer. Alex. M. p. 162 
M.) ἐν τῷ περὶ ἐλεφάντων φησί· τῶν δὲ Ἰνδικῶν αἱ θή-
λειαι χωρὶς ὀδόντων εἰσίν. 53eCDQ 54E 
<ἅν ποτε:> ἥν ποτε. 



Posidonius Phil., Fragmenta (1052: 001)
“Posidonios. Die Fragmente, vol. 1”, Ed. Theiler, W.
Berlin: De Gruyter, 1982.
Fragment 2, line 5

τὸ μὲν γὰρ ἑωθινὸν πλευρόν, τὸ κατὰ τοὺς Ἰνδούς, καὶ τὸ ἑσπέριον, τὸ 
κατὰ τοὺς Ἴβηρας καὶ τοὺς Μαυρουσίους, περιπλεῖται πᾶν ἐπὶ πολὺ τοῦ 
τε νοτίου μέρους καὶ τοῦ βορείου· τὸ δὲ λειπόμενον ἄπλουν ἡμῖν μέχρι 
νῦν τῷ μὴ συμμῖξαι μηδένας ἀλλήλοις τῶν ἀντιπεριπλεόντων οὐ πολύ, εἴ   
τις συντίθησιν ἐκ τῶν παραλλήλων διαστημάτων τῶν ἐφικτῶν ἡμῖν. 



Posidonius Phil., Fragmenta 
Fragment 3a, line 11

ὁμοίως δὲ καὶ τὸ παρ' Ἰνδοῖς οἰκεῖν ἢ παρ' Ἴβηρσιν· ὧν τοὺς μὲν ἑῴους 
μάλιστα, τοὺς δὲ ἑσπερίους, τρόπον δέ τινα καὶ ἀντίποδας ἀλλήλοις ἴσμεν. 



Posidonius Phil., Fragmenta 
Fragment 13, line 138

                                                               τυχεῖν δή τινα 
Ἰνδὸν κομισθέντα ὡς τὸν βασιλέα ὑπὸ τῶν φυλάκων τοῦ Ἀραβίου μυχοῦ, 
λεγόντων εὑρεῖν ἡμιθανῆ καταχθέντα μόνον ἐν νηί, τίς δ' εἴη καὶ πόθεν, 
ἀγνοεῖν, μὴ συνιέντας τὴν διάλεκτον. 



Posidonius Phil., Fragmenta 
Fragment 13, line 141

                                           τὸν δὲ παραδοῦναι τοῖς διδάξουσιν 
ἑλληνίζειν· ἐκμαθόντα δὲ διηγήσασθαι, διότι ἐκ τῆς Ἰνδικῆς πλέων 
περιπέσοι πλάνῃ καὶ σωθείη δεῦρο, τοὺς σύμπλους ἀποβαλὼν λιμῷ. 



Posidonius Phil., Fragmenta 
Fragment 13, line 143

ὑποληφθέντα δὲ ὑποσχέσθαι τὸν εἰς Ἰνδοὺς πλοῦν ἡγήσασθαι τοῖς ὑπὸ 
τοῦ βασιλέως προχειρισθεῖσι· τούτων δὲ γενέσθαι τὸν Εὔδοξον. 



Posidonius Phil., Fragmenta 
Fragment 13, line 173

καὶ πρῶτον μὲν εἰς Δικαιαρχίαν, εἶτ' εἰς Μασσαλίαν ἐλθεῖν, καὶ τὴν ἑξῆς 
παραλίαν μέχρι Γαδείρων, πανταχοῦ δὲ διακωδωνίζοντα ταῦτα καὶ χρη-
ματιζόμενον κατασκευάσασθαι πλοῖον μέγα καὶ ἐφόλκια δύο λέμβοις 
λῃστρικοῖς ὅμοια, ἐμβιβάσαι τε μουσικὰ παιδισκάρια καὶ ἰατροὺς καὶ 
ἄλλους τεχνίτας, ἔπειτα πλεῖν ἐπὶ τὴν Ἰνδικὴν μετέωρον ζεφύροις 
συνεχέσι. 



Posidonius Phil., Fragmenta 
Fragment 13, line 182

Ἀφέντα δὴ τὸν ἐπὶ Ἰνδοὺς πλοῦν ἀναστρέφειν· ἐν δὲ τῷ παράπλῳ 
νῆσον εὔυδρον καὶ εὔδενδρον ἐρήμην ἰδόντα σημειώσασθαι. 



Posidonius Phil., Fragmenta 
Fragment 13, line 209

Τίς γὰρ ἡ πιθανότης πρῶτον μὲν τῆς κατὰ τὸν Ἰνδὸν περιπετείας; 



Posidonius Phil., Fragmenta 
Fragment 13, line 213

οὐκ εἰκὸς δ' οὔτ' ἔξω που τὸν πλοῦν ἔχοντας εἰς τὸν κόλπον παρωσθῆναι 
τοὺς Ἰνδοὺς κατὰ πλάνην (τὰ γὰρ στενὰ ἀπὸ τοῦ στόματος δηλώσειν 
ἔμελλε τὴν πλάνην), οὔτ' εἰς τὸν κόλπον ἐπίτηδες καταχθεῖσιν ἔτι πλάνης 
ἦν πρόφασις καὶ ἀνέμων ἀστάτων. 



Posidonius Phil., Fragmenta 
Fragment 13, line 222

     ὁ δὲ δὴ σπονδοφόρος καὶ θεωρὸς τῶν Κυζικηνῶν πῶς ἀφεὶς τὴν 
πόλιν εἰς Ἰνδοὺς ἔπλει; 



Posidonius Phil., Fragmenta 
Fragment 13, line 270

Ὑπονοεῖ δὲ τὸ τῆς οἰκουμένης μῆκος ἑπτά που μυριάδων σταδίων 
ὑπάρχον ἥμισυ εἶναι τοῦ ὅλου κύκλου, καθ' ὃν εἴληπται, ὥστε, φησίν, 
ἀπὸ τῆς δύσεως εὔρῳ πλέων ἐν τοσαύταις μυριάσιν ἔλθοι ἂν εἰς Ἰνδούς. 



Posidonius Phil., Fragmenta 
Fragment 13, line 287

Ἐπαινῶν δὲ τὴν τοιαύτην διαίρεσιν τῶν ἠπείρων, οἵα νῦν ἐστι, παρα-
δείγματι χρῆται τῷ τοὺς Ἰνδοὺς τῶν Αἰθιόπων διαφέρειν τῶν ἐν τῇ 
Λιβύῃ· εὐερνεστέρους γὰρ εἶναι καὶ ἧττον ἕψεσθαι τῇ ξηρασίᾳ τοῦ περιέ-
χοντος. 



Posidonius Phil., Fragmenta 
Fragment 13, line 298

           ἔπειθ' Ὅμηρος οὐ διὰ τοῦτο διαιρεῖ τοὺς Αἰθίοπας, ὅτι τοὺς 
Ἰνδοὺς ᾔδει τοιούτους τινὰς τοῖς σώμασιν (οὐδὲ γὰρ ἀρχὴν εἰδέναι τοὺς 
Ἰνδοὺς εἰκὸς Ὅμηρον, ὅπου γε οὐδ' ὁ Εὐεργέτης κατὰ τὸν Εὐδόξειον 
μῦθον ᾔδει τὰ κατὰ τὴν Ἰνδικήν, οὐδὲ τὸν πλοῦν τὸν ἐπ' αὐτήν), ἀλλὰ 
μᾶλλον κατὰ τὴν λεχθεῖσαν ὑφ' ἡμῶν πρότερον διαίρεσιν. 



Posidonius Phil., Fragmenta 
Fragment 26, line 50

Ἀλέξανδρος δὲ τῆς Ἰνδικῆς στρατείας ὅρια βωμοὺς ἔθετο ἐν τοῖς 
τόποις εἰς οὓς ὑστάτους ἀφίκετο τῶν πρὸς ταῖς ἀνατολαῖς Ἰνδῶν, μιμού-
μενος τὸν Ἡρακλέα καὶ τὸν Διόνυσον· ἦν μὲν δὴ τὸ ἔθος τοῦτο. 



Posidonius Phil., Fragmenta 
Fragment 26, line 56

                                                                                 οὐ 
γὰρ νῦν οἱ Φιλαίνων βωμοὶ μένουσιν, ἀλλ' ὁ τόπος μετείληφε τὴν προση-
γορίαν· οὐδὲ ἐν τῇ Ἰνδικῇ στήλας φασὶν ὁραθῆναι κειμένας οὔθ' Ἡρακλέ-
ους, οὔτε Διονύσου, καὶ λεγομένων μέντοι καὶ δεικνυμένων τῶν τόπων   
τινῶν οἱ Μακεδόνες ἐπίστευον τούτους εἶναι στήλας, ἐν οἷς τι σημεῖον 
εὕρισκον ἢ τῶν περὶ τὸν Διόνυσον ἱστορουμένων ἢ τῶν περὶ τὸν Ἡρακλέα. 



Posidonius Phil., Fragmenta 
Fragment 26, line 82

                                                                  τὸ δὲ ἐπ' αὐτὰς 
ἀναφέρειν τὰς ἐν τῷ Ἡρακλείῳ στήλας τῷ ἐνθάδε ἧττον εὔλογον, ὡς 
ἐμοὶ φαίνεται· οὐ γὰρ ἐμπόρων, ἀλλ' ἡγεμόνων μᾶλλον ἀρξάντων τοῦ 
ὀνόματος τούτου, κρατῆσαι πιθανὸν τὴν δόξαν, καθάπερ καὶ ἐπὶ τῶν 
Ἰνδικῶν στηλῶν. 



Posidonius Phil., Fragmenta 
Fragment 37, line 4

Strabo 5,2,6 
Τοῦτό τε δὴ παράδοξον ἡ νῆσος ἔχει καὶ τὸ τὰ ὀρύγματα ἀναπληροῦσθαι 
πάλιν τῷ χρόνῳ τὰ μεταλλευθέντα, καθάπερ τοὺς πλαταμῶνάς φασι 
τοὺς ἐν Ῥόδῳ καὶ τὴν ἐν Πάρῳ πέτραν τὴν μάρμαρον καὶ τοὺς ἐν Ἰν-
δοῖς ἅλας, οὕς φησι Κλείταρχος. 



Posidonius Phil., Fragmenta 
Fragment 66, line 17

                                ὡς δὲ λέγεται πρὸς τὴν οἰκουμένην ὅλην καὶ 
τὰς ἐσχατιὰς τὰς τοιαύτας, οἵα καὶ ἡ Ἰνδικὴ καὶ ἡ Ἰβηρία, λέγοι ἄν, 
εἰ ἄρα, τὴν τοιαύτην ἀπόφασιν. 



Posidonius Phil., Fragmenta 
Fragment 68a, line 10

      πρῶτος δὲ Δημόκριτος (Vors. 68 B 15) πολύπειρος ἀνὴρ συνεῖδεν, 
ὅτι προμήκης ἐστὶν ἡ γῆ, ἡμιόλιον τὸ μῆκος τοῦ πλάτους ἔχουσα· συνῄνεσε 
τούτῳ καὶ Δικαίαρχος ὁ Περιπατητικός (109 Wehrli)· Εὔδοξος (276a 
Lasserre) δὲ τὸ μῆκος διπλοῦν τοῦ πλάτους, ὁ δὲ Ἐρατοσθένης πλεῖον 
τοῦ διπλοῦ· Κράτης (8a Mette) δὲ ὡς ἡμικύκλιον, Ἵππαρχος δὲ τραπε-
ζοειδῆ, ἄλλοι οὐροειδῆ, Ποσειδώνιος δὲ ὁ Στωϊκὸς σφενδονοειδῆ καὶ   
μεσόπλατον ἀπὸ νότου εἰς βορρᾶν, στενὴν πρὸς ἕω καὶ δύσιν, τὰ πρὸς 
εὖρον δ' ὅμως πλατύτερα <τὰ> πρὸς τὴν Ἰνδικήν. 



Posidonius Phil., Fragmenta 
Fragment 78, line 71

                 διὰ δὲ τὰς αὐτὰς αἰτίας κατὰ μὲν τὴν Αἴγυπτον τούς τε 
κροκοδείλους φύεσθαι καὶ τοὺς ποταμίους ἵππους, κατὰ δὲ τὴν Αἰθιοπίαν 
καὶ τὴν τῆς Λιβύης ἔρημον ἐλεφάντων τε πλῆθος καὶ παντοδαπῶν ὄφεών 
τε καὶ τῶν ἄλλων θηρίων καὶ δρακόντων ἐξηλλαγμένων τοῖς τε μεγέθεσι   
καὶ ταῖς ἀλκαῖς, ὁμοίως δὲ καὶ τοὺς περὶ τὴν Ἰνδικὴν ἐλέφαντας, ὑπερβάλ-
λοντας τοῖς τε ὄγκοις καὶ πλήθεσιν, ἔτι δὲ ταῖς ἀλκαῖς. 



Posidonius Phil., Fragmenta 
Fragment 78, line 116

                       ὁ δ' αὐτὸς λόγος καὶ κατὰ τὰς ἄλλας χώρας τῆς γῆς 
τὰς κατὰ τὴν ὁμοίαν κλίσιν κειμένας, λέγω δ' Ἰνδικὴν καὶ τὴν Ἐρυθρὰν 
θάλατταν, ἔτι δὲ Αἰθιοπίαν καί τινα μέρη τῆς Λιβύης. 



Posidonius Phil., Fragmenta 
Fragment 113, line 3

Diodor 33,18 
18. Ὅτι ὁ Ἀρσάκης ὁ βασιλεὺς ἐπιείκειαν καὶ φιλανθρωπίαν ζηλώσας 
αὐτομάτην ἔσχε τὴν ἐπίρροιαν τῶν ἀγαθῶν καὶ τὴν βασιλείαν ἐπὶ πλέον 
ηὔξησε· μέχρι <γὰρ> τῆς Ἰνδικῆς διατείνας τῆς ὑπὸ τὸν Πῶρον γενομένης 
χώρας ἐκυρίευσεν ἀκινδύνως. 



Posidonius Phil., Fragmenta 
Fragment 133, line 69

τοιοῦτος δὲ ὁ Ἀμφιάρεως καὶ ὁ Τροφώνιος καὶ <ὁ> Ὀρφεὺς καὶ ὁ Μουσαῖος 
καὶ ὁ παρὰ τοῖς Γέταις θεός, ⌈τὸ μὲν παλαιὸν⌉ Ζάμολξις Πυθαγόρειός τις, 
⌈καθ' ἡμᾶς δὲ ὁ τῷ Βυρεβίστᾳ θεσπίζων Δεκαίνεος⌉· παρὰ δὲ τοῖς Βοσπο-
ρηνοῖς Ἀχαίκαρος, παρὰ δὲ τοῖς Ἰνδοῖς οἱ γυμνοσοφισταί, παρὰ δὲ τοῖς 
Πέρσαις οἱ Μάγοι καὶ νεκυομάντεις καὶ ἔτι οἱ λεγόμενοι λεκανομάντεις 
καὶ ὑδρομάντεις, παρὰ δὲ τοῖς Ἀσσυρίοις οἱ Χαλδαῖοι, παρὰ δὲ τοῖς 
Ῥωμαίοις οἱ Τυρρηνικοὶ ἱεροσκόποι. 



Posidonius Phil., Fragmenta 
Fragment 310, line 59

                                                             ⌈τὸ παραπλήσιον 
μέντοι καὶ τοὺς κατὰ τὴν Ἰνδικὴν δράκοντάς φασι πάσχειν· ἀνέρποντας 
γὰρ ἐπὶ τὰ μέγιστα τῶν ζῴων, ἐλέφαντας, περὶ νῶτα καὶ νηδὺν ἅπασαν 
εἱλεῖσθαι, φλέβα δ' ἣν ἂν τύχῃ διελόντας ἐμπίνειν τοῦ αἵματος, ἀπλήστως 
ἐπισπωμένους βιαίῳ πνεύματι καὶ συντόνῳ ῥοίζῳ· μέχρι μὲν οὖν τινος   
ἐξαναλουμένους ἐκείνους ἀντέχειν ὑπ' ἀμηχανίας ἀνασκιρτῶντας καὶ τῇ 
προνομαίᾳ τὴν πλευρὰν τύπτοντας ὡς καθιξομένους τῶν δρακόντων, εἶτα 
ἀεὶ κενουμένου τοῦ ζωτικοῦ, πηδᾶν μὲν μηκέτι δύνασθαι, κραδαινομένους 
δ' ἑστάναι, μικρὸν δ' ὕστερον καὶ τῶν σκελῶν ἐξασθενησάντων, κατασει-
σθέντας ὑπὸ λιφαιμίας ἀποψύχειν, πεσόντας δὲ τοὺς αἰτίους τοῦ θανάτου 




Posidonius Phil., Fragmenta (1052: 003)
“FGrH \#87”.
Volume-Jacobyʹ-F 2a,87,F, fragment 28, line 106

                                                               τυχεῖν δή τινα 
Ἰνδὸν κομισθέντα ὡς τὸν βασιλέα ὑπὸ τῶν φυλάκων τοῦ Ἀραβίου μυχοῦ, 
λεγόντων εὑρεῖν ἡμιθανῆ καταχθέντα μόνον ἐν νηί, τίς δ' εἴη καὶ πόθεν, 
ἀγνοεῖν, μὴ συνιέντας τὴν διάλεκτον· τὸν δὲ παραδοῦναι τοῖς διδάξουσιν 
ἑλληνίζειν. 



Posidonius Phil., Fragmenta 
Volume-Jacobyʹ-F 2a,87,F, fragment 28, line 109

              ἐκμαθόντα δὲ διηγήσασθαι, διότι ἐκ τῆς Ἰνδικῆς πλέων περι-
πέσοι πλάνηι καὶ σωθείη δεῦρο, τοὺς σύμπλους ἀποβαλὼν λιμῶι. 



Posidonius Phil., Fragmenta 
Volume-Jacobyʹ-F 2a,87,F, fragment 28, line 111

                                                                       ὑπο-
ληφθέντα δὲ ὑποσχέσθαι τὸν εἰς Ἰνδοὺς πλοῦν ἡγήσασθαι τοῖς ὑπὸ τοῦ 
βασιλέως προχειρισθεῖσι τούτων δὲ γενέσθαι τὸν Εὔδοξον. 



Posidonius Phil., Fragmenta 
Volume-Jacobyʹ-F 2a,87,F, fragment 28, line 141

καὶ πρῶτον μὲν εἰς Δικαιαρχίαν, εἶτ' εἰς Μασσαλίαν ἐλθεῖν καὶ τὴν ἑξῆς 
παραλίαν μέχρι Γαδείρων· πανταχοῦ δὲ διακωδωνίζοντα ταῦτα καὶ χρηματι-
ζόμενον κατασκευάσασθαι πλοῖον μέγα καὶ ἐφόλκια δύο λέμβοις ληιστρι-
κοῖς ὅμοια, <οἷς> ἐμβιβάσασθαι μουσικὰ παιδισκάρια καὶ ἰατροὺς καὶ 
ἄλλους τεχνίτας, ἔπειτα πλεῖν ἐπὶ τὴν Ἰνδικὴν μετέωρον ζεφύροις συνε-
χέσι. 



Posidonius Phil., Fragmenta 
Volume-Jacobyʹ-F 2a,87,F, fragment 28, line 150

ἀφέντα δὴ τὸν ἐπὶ Ἰνδοὺς πλοῦν ἀναστρέφειν· ἐν δὲ τῶι παράπλωι νῆσον 
εὔυδρον καὶ εὔδενδρον ἐρήμην ἰδόντα σημειώσασθαι. 



Posidonius Phil., Fragmenta 
Volume-Jacobyʹ-F 2a,87,F, fragment 28, line 191

εἰκάζει δὲ καὶ τὴν τῶν Κίμβρων καὶ τῶν συγγενῶν ἐξανάστασιν ἐκ 
τῆς οἰκείας γενέσθαι κατὰ θαλάττης ἔφοδον, οὐκ ἀθρόαν συμβᾶσαν (F 31). 
 ὑπονοεῖ δὲ τὸ τῆς οἰκουμένης μῆκος ἑπτά που μυριάδων σταδίων 
ὑπάρχον ἥμισυ εἶναι τοῦ ὅλου κύκλου, καθ' ὃν εἴληπται, <«ὥστε»>, φησίν, 
<«ἀπὸ τῆς δύσεως εὔρωι πλέων ἐν τοσαύταις μυριάσιν 
ἔλθοι ἂν εἰς Ἰνδούς. 



Posidonius Phil., Fragmenta 
Volume-Jacobyʹ-F 2a,87,F, fragment 28, line 208

                                               οὐ γὰρ φύσει Ἀθηναῖοι μὲν φιλόλογοι, Λακε-
δαιμόνιοι δ' οὐ καὶ οἱ ἔτι ἐγγυτέρω Θηβαῖοι, ἀλλὰ μᾶλλον ἔθει· οὕτως οὐδὲ Βαβυ-
λώνιοι φιλόσοφοι φύσει καὶ Αἰγύπτιοι, ἀλλ' ἀσκήσει καὶ ἔθει· καὶ ἵππων τε καὶ 
βοῶν ἀρετὰς καὶ ἄλλων ζώιων οὐ τόποι μόνον ἀλλὰ καὶ ἀσκήσεις ποιοῦσιν· ὁ δὲ 
συγχεῖ ταῦτα. 
 ἐπαινῶν δὲ τὴν τοιαύτην διαίρεσιν τῶν ἠπείρων, οἵα νῦν ἐστι, 
παραδείγματι χρῆται τῶι τοὺς Ἰνδοὺς τῶν Αἰθιόπων διαφέρειν τῶν ἐν 
τῆι Λιβύηι· εὐερνεστέρους γὰρ εἶναι καὶ ἧττον ἕψεσθαι τῆι ξηρασίαι τοῦ 
περιέχοντος. 



Posidonius Phil., Fragmenta 
Volume-Jacobyʹ-F 2a,87,F, fragment 53, line 43

                                                                                  ...... Ἀλέξαν-
δρος δὲ τῆς Ἰνδικῆς στρατείας ὅρια βωμοὺς ἔθετο . 



Posidonius Phil., Fragmenta 
Volume-Jacobyʹ-F 2a,87,F, fragment 70, line 66

                                                (39) ταῦτα γὰρ ὅπως ποτὲ ἀλη-
θείας ἔχει, παρά γε τοῖς ἀνθρώποις ἐπεπίστευτο καὶ ἐνενόμιστο, καὶ διὰ τοῦτο καὶ 
οἱ μάντεις ἐτιμῶντο, ὥστε καὶ βασιλείας ἀξιοῦσθαι, ὡς τὰ παρὰ τῶν θεῶν ἡμῖν 
ἐκφέροντες παραγγέλματα καὶ ἐπανορθώματα καὶ ζῶντες καὶ ἀποθανόντες, καθάπερ 
καὶ ὁ Τειρεσίας, ‘τῶι καὶ τεθνεῶτι νόον πόρε Περσεφόνεια οἴωι πεπνῦσθαι· τοὶ δὲ 
σκιαὶ αἴσσουσι’ (Od. κ 495). τοιοῦτος δὲ ὁ Ἀμφιάρεως καὶ ὁ Τροφώνιος καὶ <ὁ> 
Ὀρφεὺς καὶ ὁ Μουσαῖος καὶ ὁ παρὰ τοῖς Γέταις θεός, ⟦τὸ μὲν παλαιὸν Ζάμολξις 
Πυθαγόρειός τις, καθ' ἡμᾶς δὲ ὁ τῶι Βυρεβίσται θεσπίζων Δεκαίνεος⟧· παρὰ δὲ 
τοῖς Βοσπορηνοῖς Ἀχαίκαρος, παρὰ δὲ τοῖς Ἰνδοῖς οἱ γυμνοσοφισταί, παρὰ δὲ τοῖς 
Πέρσαις οἱ Μάγοι καὶ νεκυομάντεις καὶ ἔτι οἱ λεγόμενοι λεκανομάντεις καὶ ὑδρο-
μάντεις, παρὰ δὲ τοῖς Ἀσσυρίοις οἱ Χαλδαῖοι, παρὰ δὲ τοῖς Ῥωμαίοις οἱ Τυρρη-
νικοὶ † ὡροσκόποι. 



Posidonius Phil., Fragmenta 
Volume-Jacobyʹ-F 2a,87,F, fragment 80, line 17

                                                                                           ὡς 
δὲ λέγεται πρὸς τὴν οἰκουμένην ὅλην καὶ τὰς ἐσχατιὰς τὰς τοιαύτας, οἵα καὶ ἡ 
Ἰνδικὴ καὶ ἡ Ἰβηρία, λέγοι ἄν, εἰ ἄρα, τὴν τοιαύτην ἀπόφασιν. 



Posidonius Phil., Fragmenta 
Volume-Jacobyʹ-F 2a,87,F, fragment 98a, line 8

                           ...· Εὔδοξος δὲ τὸ μῆκος διπλοῦν τοῦ πλάτους· 
ὁ δὲ Ἐρατοσθένης (V) πλεῖον τοῦ διπλοῦ· Κράτης δὲ ὡς ἡμικύκλιον· 
Ἵππαρχος δὲ τραπεζοειδῆ· ἄλλοι οὐροειδῆ· Ποσειδώνιος δὲ ὁ Στωικὸς 
σφενδονοειδῆ καὶ μεσόπλατον ἀπὸ νότου εἰς βορρᾶν, στενὴν πρὸς ἕω καὶ 
δύσιν, τὰ πρὸς εὖρον δ' ὅμως πλατύτερα <τὰ> πρὸς τὴν Ἰνδικήν. 



Posidonius Phil., Fragmenta 
Volume-Jacobyʹ-F 2a,87,F, fragment 114, line 64

                               (4) διὰ δὲ τὰς αὐτὰς αἰτίας κατὰ μὲν τὴν Αἴγυπτον 
τούς τε κροκοδείλους φύεσθαι καὶ τοὺς ποταμίους ἵππους· κατὰ δὲ τὴν Αἰθιοπίαν 
καὶ τὴν τῆς Λιβύης ἔρημον ἐλεφάντων τε πλῆθος καὶ παντοδαπῶν ὄφεών τε καὶ 
τῶν ἄλλων θηρίων καὶ δρακόντων ἐξηλλαγμένων τοῖς τε μεγέθεσι καὶ ταῖς ἀλκαῖς· 
ὁμοίως δὲ καὶ τοὺς περὶ τὴν Ἰνδικὴν ἐλέφαντας ὑπερβάλλοντας τοῖς τε ὄγκοις καὶ 
πλήθεσιν, ἔτι δὲ ταῖς ἀλκαῖς. 



Posidonius Phil., Fragmenta 
Volume-Jacobyʹ-F 2a,87,F, fragment 114, line 103

                       (3) ὁ δ' αὐτὸς λόγος καὶ κατὰ τὰς ἄλλας χώρας τῆς γῆς τὰς 
κατὰ τὴν ὁμοίαν κρᾶσιν κειμένας, λέγω δ' Ἰνδικὴν καὶ τὴν Ἐρυθρὰν θάλατταν, ἔτι 
δὲ Αἰθιοπίαν καί τινα μέρη τῆς Λιβύης. 



Eratosthenes et Eratosthenica Philol., Catasterismi (0222: 001)
“Pseudo–Eratosthenis catasterismi”, Ed. Olivieri, A.
Leipzig: Teubner, 1897; Mythographi Graeci 3.1.
Chapter 1, section 5R|:], line 19

Ἡφαίστου δὲ ἔργον εἶναί 
φασιν ἐκ χρυσοῦ πυρώδους 
καὶ λίθων ἰνδικῶν· ἱστο-
ρεῖται δὲ διὰ τούτου καὶ 
τὸν Θησέα σωθῆναι ἐκ τοῦ 
λαβυρίνθου ποιοῦντος τοῦ 
στεφάνου φέγγος· ἐν δὲ 
τοῖς ἄστροις ὕστερον αὐτὸν 
τεθεικέναι, ὅτε εἰς Νάξον 
ἦλθον ἀμφότεροι, σημεῖον 
τῆς αἱρέσεως· συνεδόκει δὲ 
καὶ τοῖς θεοῖς. 



Eratosthenes et Eratosthenica Philol., Catasterismi 
Chapter 1, section 5D, line 14

Ἡφαίστου δὲ ἔργον εἶναί 
φασιν ἐκ χρυσοῦ πυρώδους 
καὶ λίθων ἰνδικῶν· ἱστο-
ρεῖται δὲ καὶ διὰ τούτου 
τὸν Θησέα σεσῶσθαι ἐκ 
τοῦ λαβυρίνθου, φέγγος 
ποιοῦντος. 
\end{greek}


\section{Theodore (? Abu-Qurrah) | (? of Nicaea)}
\blockquote[From Wikipedia\footnote{\url{}}]{}
\begin{greek}
Theodorus Epist., Epistulae (3158: 001)
“Épistoliers byzantins du x<e> siècle”, Ed. Darrouzès, J.
Paris: Institut Français d'Études Byzantines, 1960; Archives de l'orient chrétien 6.
Epistle 7, line 21

Τὸν δὲ κοινωνοῦντά σοι καὶ τῆς στέγης καὶ τοῦ πρὸς ἐμὲ φίλτρου 
ὡς ἐξ ἡμῶν προσαγόρευσον· οὕτω γὰρ τὰς ὀλυμπιακὰς ἐδοξάμην θρίδακας 
ὑπὲρ τοὺς τῶν Ἰνδῶν λυχνίτας καὶ σμαράγδους· εἰ δὲ δι' ἐπιστολῆς ἡμᾶς 
οὗτος δεξιώσεται, τάχα ἂν τῶν αὐτοῦ σπερμάτων τοὺς καρποὺς φάγεται· 
ὅμως ἔρρωσο καὶ αὐτός· τοῦτο γὰρ φίλον ἐμοί. 
\end{greek}



\backmatter


\begin{comment}

\part{Raw, unsorted TLG data of ``Ινδ--''}

\section{}
\blockquote[From Wikipedia\footnote{\url{}}]{}
\begin{greek}
\end{greek}

\section{}
\blockquote[From Wikipedia\footnote{\url{}}]{}
\begin{greek}
\end{greek}

\section{}
\blockquote[From Wikipedia\footnote{\url{}}]{}
\begin{greek}
\end{greek}

\section{}
\blockquote[From Wikipedia\footnote{\url{}}]{}
\begin{greek}
\end{greek}

\section{}
\blockquote[From Wikipedia\footnote{\url{}}]{}
\begin{greek}
\end{greek}

\chapter{Raw PHI5 data of ``Ind--'}

Reject pattern: ind[e|u|ig|icat|oc|om|or|ol|on]

\printbibliography

\printindex

\end{comment}

\end{document}

