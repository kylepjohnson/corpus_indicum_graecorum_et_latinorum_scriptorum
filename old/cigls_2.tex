\documentclass[12pt,letterpaper,twoside,final]{memoir}
%M-x shell
%latexmk -pdf -e '$pdflatex=q/xelatex %O %S/' *.tex
%latexmk -pdf -pvc -e '$pdflatex=q/xelatex %O %S/' *.tex
\usepackage[no-math]{fontspec}
\usepackage{xltxtra}
\defaultfontfeatures{Scale=MatchLowercase,Mapping=tex-text}
\setmainfont[Mapping=tex-text]{Junicode} %CMU Serif
\setsansfont[Mapping=tex-text]{Junicode} %CMU Sans Serif
\newfontfamily\greekfont{Linux Libertine} %CMU Serif,Linux Libertine O,Junicode
%\setmainfont[Mapping=tex-text]{CMU Serif}
\usepackage{xkeyval}
\usepackage{polyglossia}
\setdefaultlanguage[variant=american]{english}
\setotherlanguage[variant=ancient,numerals=arabic]{greek}
\setotherlanguage[spelling=new]{german}
\setotherlanguages{latin,french,italian,spanish,sanskrit}

\usepackage{xeindex}
\makeindex
\IndexList{mylist}{gold,silver,*Ethiopia=Ethiopia,*Ethiopian?=Ethiopians,*India=India,*Indian?=Indians,ψῆγμα=ψῆγμα,ψήγματος=ψήγμα}

\usepackage[babel=once,english=american,autostyle=tryonce,strict=true]{csquotes}
\usepackage[backend=biber,style=authoryear,sorting=debug,bibstyle=authoryear,citestyle=authoryear,useprefix=false,firstinits=false,url=false,usetranslator=true]{biblatex}%was: firstinits=true
%\DeclareAutoCiteCommand{plain}{\cite}{\cites}
\DeclareAutoCiteCommand{plain}{\textcite}{\textcites}
\DeclareAutoCiteCommand{inline}{\textcite}{\textcites}
%\DeclareAutoCiteCommand{footnote}[l]{\footcite}{\footcites}
%\DeclareAutoCiteCommand{footnote}[f]{\smartcite}{\smartcites}
\bibliography{classics_bib} %also avail india_bib
\usepackage[final]{hyperref}%?%hyperfootnotes=false
\hypersetup{bookmarks=false,        % show bookmarks bar?
    unicode=true,           % non-Latin characters in Acrobat’s bookmarks
    pdftoolbar=true,        % show Acrobat’s toolbar?
    pdfmenubar=true,        % show Acrobat’s menu?
    pdffitwindow=false,     % window fit to page when opened
    pdfstartview={FitH},    % fits the width of the page to the window
    pdftitle={Ethics of Leadership: Organization and Decision--Making in Caesar's \emph{Bellum~Gallicum}},
    pdfauthor={Kyle P. Johnson},     % author
    pdfsubject={Dissertation on organization and decision--making in Julius Caesar's Bellum Gallicum},   % subject of the document
    pdfcreator={Kyle P. Johnson},   % creator of the document
    pdfproducer={Kyle P. Johnson}, % producer of the document
    pdfkeywords={Julius Caesar, Bellum Gallicum, Gallic War, communication, deliberation, decision--making, leadership, organization, Xenophon, Anabasis}, % list of keywords
    pdfnewwindow=true,      % links in new window
    colorlinks=true,       % false: boxed links; true: colored links
    linkcolor=blue,          % color of internal links
    citecolor=blue,        % color of links to bibliography
    filecolor=blue,      % color of file links
    urlcolor=blue           % color of external links
}
\usepackage{microtype}
\usepackage{xecolor}
\definergbcolor{blue}{0000FF}
\definergbcolor{red}{FF0000} %\textxecolor{colorname}{text}
\XeTeXdashbreakstate=1
\usepackage{minitoc}
\usepackage{indentfirst}
\usepackage{outline}
\usepackage{verbatim}
\usepackage{enumerate}
%\usepackage{tikz}
%\usetikzlibrary{shapes,backgrounds}
\usepackage{longtable}
%\usepackage{lscape}
%\usepackage{verse}
%\usepackage{rotating}
\usepackage{ccicons}
\usepackage{bookmark}
\usepackage{eledmac}
\usepackage{eledpar}
\usepackage{etoolbox}
%!%%%%%%%%%%%%%%%%%%%%%for non-italicized headings%%%%%
% http://tex.stackexchange.com/questions/32655/remove-italic-from-memoir-headings-pagestyle
\makeevenhead{headings}{\leftmark}{}{}
\makeoddhead{headings}{\rightmark}{}{}
\makeevenfoot{headings}{}{\thepage}{}
\makeoddfoot{headings}{}{\thepage}{}
%!%%%%%%%%%%%%%%%%%%%%%%%%%%%%%%%%%%%%%%%%%%%%%%%%%%%%%%
%!%%%%%%%%%%%%%%%%%%%%%for margins, l=1.5; rest=1.0, maybe add 0.1in%%%%%
\setstocksize{11in}{8.5in}
\settrimmedsize{11.0in}{8.5in}{*} %\settrimmedsize{ height }{ width }{ ratio }
\settypeblocksize{7.75in}{5.8in}{*} %\settypeblocksize{ height }{ width }{ ratio } %note: 7.25h gives margins of 1.5 on top/bottom w/ ubratio of 1
\setlrmargins{1.5in}{*}{*} %\setlrmargins{ spine }{ edge }{ ratio } %spine = left, edge = right; only answer one or two of these values
%\setlrmarginsandblock{1.5in}{1.0in}{*} %\setlrmarginsandblock{ spine }{ edge }{ ratio } %
%\setulmargins{*}{*}{*} %\setulmargins{ upper }{ lower }{ ratio }
%\setulmarginsandblock{0.5in}{*}{*} %\setulmarginsandblock{ upper }{ lower }{ ratio }
%\setheadfoot{*}{*} %\setheadfoot{ headheight }{ footskip }
%\setheaderspaces{1.0in}{*}{*} %\setheaderspaces{ headdrop }{ headsep }{ ratio }
%\checkandfixthelayout[lines]
\flushbottom
%!%%%%%%%%%%%%%%%%%%%%%%%%%%%%%%%%%%%%%%%%%%%%%%%%%%%%%%
\apptocmd{\sloppy}{\hbadness 10000\relax}{}{}
%\renewcommand{\@pnumwidth}{3em} %taken from memman p. 153
%\renewcommand{\@tocrmarg}{4em} %taken from memman p. 153
\setsecnumdepth{subparagraph}
\makeatletter
\renewcommand\@makefntext{\hspace*{2em}\@thefnmark. }
\newenvironment*{singlespcquote}
        {\quote\SingleSpacing}
        {\endquote}
\SetBlockThreshold{0}
\SetBlockEnvironment{singlespcquote}
\SetCiteCommand{\parencite} %default is \cite
  %csquote + biblatex; see csquotes.pdf section 5 + 8.6
  %\textcquote[prenote][postnote]{key}[punct]{text}[tpunct]
  %\blockcquote[prenote][postnote]{key}[punct]{text}[tpunct]
  %usually:
  %\textcquote[page#]{key}{quote}
  %\blockcquote[page#]{key}[.]{quote}
  %\textcites(pre)(post)[pre][post]{key}...[pre][post]{key}
  %example: \textcites(and chapter 3)[35]{riggsby2006}[78]{hammond1996}[23]{levene2010}
\title{Eurasian Dharma Corpus}
\author{Kyle~P.~Johnson}
%\date{January 31, 2008}
\begin{document}
%\fussy
%\hyphenpenalty=5000   %1000 default=?
%\tolerance=1000        %1000 %200= default
%\setlength{\emergencystretch}{3em}
%\midsloppy
%\fussy
\sloppy
\vbadness=10000 % badness above which bad vboxes are shown. (Default = 10000?)
\frontmatter
\hyphenation{}

\SingleSpacing

%%%%%%%%%%for minitoc%%%%%%%%%%%%%%%%%%%%%
\dominitoc
\dominilof
\dominilot
%%%%%%%%%%%%%%%%%%%%%%%%%%%%%%%%%%%%%%%%%%%
\tableofcontents

\mainmatter

\chapter{Late Antique}%late
\minitoc

\section{Nonnus}

\begin{greek}
Nonnus Epic., Dionysiaca (2045: 001)
“Nonni Panopolitani Dionysiaca, 2 vols.”, Ed. Keydell, R.
Berlin: Weidmann, 1959.
Book p1, line 28

ἐς <δέκατον δὲ τέταρτον> ἔχε φρένα· κεῖθι κορύσσει 
δαιμονίην στίχα πᾶσαν ἐς Ἰνδικὸν ἄρεα Ῥείη. 



Nonnus Epic., Dionysiaca 
Book p1, line 42

<εἰκοστὸν πρώτιστον> ἔχει χόλον Ἐννοσιγαίου 
καὶ μόθον Ἀμβροσίης ῥηξήνορα καὶ λόχον Ἰνδῶν. 



Nonnus Epic., Dionysiaca 
Book p1, line 45

<εἰκοστῷ τριτάτῳ> πεπερημένον Ἰνδὸν Ὑδάσπην 
καὶ κλόνον ὑδατόεντα καὶ αἰθαλόεντα λιγαίνω. 



Nonnus Epic., Dionysiaca 
Book p1, line 47

<εἰκοστὸν> δὲ <τέταρτον> ἔχει γόον ἄσπετον Ἰνδῶν 
κερκίδα θ' ἱστοπόνοιο καὶ ἠλακάτην Ἀφροδίτης. 



Nonnus Epic., Dionysiaca 
Book 1, line 24

εἰ γὰρ ἐφερπύσσειε δράκων κυκλούμενος ὁλκῷ, 
μέλψω θεῖον ἄεθλον, ὅπως κισσώδεϊ θύρσῳ 
φρικτὰ δρακοντοκόμων ἐδαΐζετο φῦλα Γιγάντων· 
εἰ δὲ λέων φρίξειεν ἐπαυχενίην τρίχα σείων, 
Βάκχον ἀνευάξω βλοσυρῆς ἐπὶ πήχεϊ Ῥείης 
μαζὸν ὑποκλέπτοντα λεοντοβότοιο θεαίνης· 
εἰ δὲ θυελλήεντι μετάρσιος ἅλματι ταρσῶν 
πόρδαλις ἀίξῃ πολυδαίδαλον εἶδος ἀμείβων, 
ὑμνήσω Διὸς υἷα, πόθεν γένος ἔκτανεν Ἰνδῶν 
πορδαλίων ὀχέεσσι καθιππεύσας ἐλεφάντων· 
εἰ δέμας ἰσάζοιτο τύπῳ συός, υἷα Θυώνης   
ἀείσω ποθέοντα συοκτόνον εὔγαμον Αὔρην, 
ὀψιγόνου τριτάτοιο Κυβηλίδα μητέρα Βάκχου· 
εἰ δὲ πέλοι μιμηλὸν ὕδωρ, Διόνυσον ἀείσω 
κόλπον ἁλὸς δύνοντα κορυσσομένοιο Λυκούργου· 
εἰ φυτὸν αἰθύσσοιτο νόθον ψιθύρισμα τιταίνων, 
μνήσομαι Ἰκαρίοιο, πόθεν παρὰ θυιάδι ληνῷ 
βότρυς ἁμιλλητῆρι ποδῶν ἐθλίβετο ταρσῷ. 



Nonnus Epic., Dionysiaca 
Book 1, line 196

οὐρανίου δὲ Δράκοντος ἐπεσκίρτησεν ἀκάνθῃ 
ἄρεα συρίζων· ὁ δὲ Κηφέος ἐγγύθι κούρης 
ἀστραίαις παλάμῃσιν ἰσόζυγα κύκλον ἑλίξας 
δέσμιον Ἀνδρομέδην ἑτέρῳ σφηκώσατο δεσμῷ 
λοξὸς ὑπὸ σπείρῃσιν· ὁ δὲ γλωχῖνι κεραίης 
ἰσοτύπου Ταύροιο δράκων κυκλοῦτο κεράστης, 
οἰστρήσας ἑλικηδὸν ὑπὲρ βοέοιο μετώπου 
ἀντιτύπους Ὑάδας, κεραῆς ἴνδαλμα Σελήνης, 
οἰγομέναις γενύεσσιν· ὁμοπλεκέων δὲ δρακόντων 
ἰοβόλοι τελαμῶνες ἐμιτρώσαντο Βοώτην· 
καὶ θρασὺς ἄλλος ὄρουσεν, ἰδὼν Ὄφιν ἄλλον Ὀλύμπου, 
πῆχυν ἐχιδνήεντα περισκαίρων Ὀφιούχου, 
καὶ στεφάνῳ στέφος ἄλλο περιπλέξας Ἀριάδνης, 
αὐχένα κυρτώσας, ἐλελίζετο γαστέρος ὁλκῷ. 



Nonnus Epic., Dionysiaca 
Book 1, line 280

θλιβομένη νεφέεσσι· Γιγαντείου δὲ καρήνου 
φρικτὸν ἀερσιλόφων ἀίων βρύχημα λεόντων   
πόντιος ἰλυόεντι λέων ἐκαλύπτετο κόλπῳ· 
πᾶσα δὲ κητώεσσα φάλαγξ ἐστείνετο πόντου, 
Γηγενέος πλήσαντος ὅλην ἅλα μείζονα γαίης 
ἀκλύστοις λαγόνεσσιν ἐμυκήσαντο δὲ φῶκαι, 
καὶ βυθίῃ δελφῖνες ἐνεκρύπτοντο θαλάσσῃ· 
καὶ σκολιαῖς ἑλίκεσσι περίπλοκον ὁλκὸν ὑφαίνων 
πούλυπος αἰολόμητις ἐθήμονι πήγνυτο πέτρῃ, 
καὶ μελέων ἴνδαλμα χαραδραίη πέλε μορφή. 



Nonnus Epic., Dionysiaca 
Book 3, line 280

Ζηνὸς ἀθηήτοιο, καὶ ἐς νομὸν ἤιε κούρη 
ὀφθαλμοὺς τρομέουσα πολυγλήνοιο νομῆος· 
γυιοβόρῳ δὲ μύωπι χαρασσομένη δέμας Ἰὼ 
Ἰονίης ἁλὸς οἶδμα κατέγραφε φοιτάδι χηλῇ· 
ἦλθε καὶ εἰς Αἴγυπτον, ἐμὸν ῥόον,  – ὃν πολιῆται   
Νεῖλον ἐφημίξαντο φερώνυμον, οὕνεκα γαίῃ 
εἰς ἔτος ἐξ ἔτεος πεφορημένος ὑγρὸς ἀκοίτης 
χεύματι πηλώεντι νέην περιβάλλεται ἰλύν,  –  
ἤλυθεν εἰς Αἴγυπτον, ὅπῃ βοέην μετὰ μορφὴν 
δαιμονίης ἴνδαλμα μεταλλάξασα κεραίης 
ἔσκε θεὰ φερέκαρπος· ἀναπτομένοιο δὲ καρποῦ 
Αἰγυπτίης Δήμητρος, ἐμῆς κεραελκέος Ἰοῦς, 
εὐόδμοις ὁμόφοιτος ἑλίσσεται ἀτμὸς ἀήταις. 



Nonnus Epic., Dionysiaca 
Book 4, line 120

οὐ ποθέω στίλβουσαν Ἐρυθραίην λίθον Ἰνδῶν, 
οὐ φυτὸν Ἑσπερίδων παγχρύσεον, οὐδέ με τέρπει 
Ἡλιάδων ἤλεκτρον, ὅσον μία νυκτὸς ὀμίχλη, 
τῇ ἔνι Πεισινόην προσπτύξεται οὗτος ἀλήτης. 



Nonnus Epic., Dionysiaca 
Book 4, line 420

                                         ἀμφὶ δὲ νεκρῷ 
θοῦρος Ἄρης βαρύμηνις ἀνέκραγε· χωομένου δὲ 
Κάδμος ἀμειβομένων μελέων ἑλικώδεϊ μορφῇ 
ἀλλοφυὴς ἤμελλε παρ' Ἰλλυρίδος σφυρὰ γαίης 
ξεῖνον ἔχειν ἴνδαλμα δρακοντείοιο προσώπου. 



Nonnus Epic., Dionysiaca 
Book 5, line 170

ὑψιφανὴς πτερύγων πισύρων τετράζυγι κόσμῳ· 
τῇ μὲν ξανθὸς ἴασπις ἐπέτρεχε, τῇ δὲ Σελήνης 
εἶχε λίθον πάνλευκον, ὃς εὐκεράοιο θεαίνης 
λειπομένης μινύθει καὶ ἀέξεται, ὁππότε Μήνη 
ἀρτιφαὴς σέλας ὑγρὸν ἀποστίλβουσα κεραίης 
Ἠελίου γενετῆρος ἀμέλγεται αὐτόγονον πῦρ· 
ἄλλη μάργαρον εἶχε φαεσφόρον, οὗ χάριν αἴγλης 
γλαυκὸν Ἐρυθραίης ἀμαρύσσεται οἶδμα θαλάσσης 
λαμπομένης· ἑτέρης δὲ μεσόμφαλος αἴθοπι κόσμῳ 
λεπτοφαὴς σέλας ὑγρὸν ἀπέπτυεν Ἰνδὸς ἀχάτης. 



Nonnus Epic., Dionysiaca 
Book 5, line 202

πρώτη δ' Αὐτονόη γονίμων ἀνεπήλατο κόλπων 
μητέρος ἐννεάμηνον ἀναπτύξασα λοχείην 
πρωτοτόκοις ὠδῖσιν· ὁμογνήτῳ δὲ γενέθλῃ 
καλλιφυὴς Ἀθάμαντος ἀέξετο σύγγαμος Ἰνώ, 
μήτηρ δισσοτόκος· τριτάτη δ' ἀνέτελλεν Ἀγαύη, 
ἥ ποτε νυμφευθεῖσα Γιγαντείοις ὑμεναίοις 
εἴκελον υἷα λόχευσεν ὀδοντοφύτῳ παρακοίτῃ· 
καὶ Χαρίτων ἴνδαλμα ποθοβλήτοιο προσώπου 
Ζηνὶ φυλασσομένη Σεμέλη βλάστησε τετάρτη 
θυγατέρων, μούνῃ δὲ καὶ ὁπλοτέρῃ περ ἐούσῃ 
δῶκεν ἀνικήτοιο φύσις πρεσβήια μορφῆς. 



Nonnus Epic., Dionysiaca 
Book 5, line 302

ἀλλά οἱ οὐ χραίσμησε ποδῶν δρόμος, οὐδὲ φαρέτρη 
ἤρκεσεν, οὐ βελέων σκοπὸς ὄρθιος, οὐ δόλος ἄγρης· 
ἀλλά μιν ὤλεσε Μοῖρα, κυνοσπάδα νεβρὸν ἀλήτην, 
Ἰνδῴην μετὰ δῆριν ἔτι πνείοντα κυδοιμοῦ, 
εὖτε τανυπρέμνοιο καθήμενος ὑψόθι φηγοῦ 
λουομένης ἐνόησεν ὅλον δέμας Ἰοχεαίρης, 
θηητὴρ δ' ἀκόρητος ἀθηήτοιο θεαίνης 
ἁγνὸν ἀνυμφεύτοιο δέμας διεμέτρεε κούρης 
ἀγχιφανής· καὶ τὸν μὲν ἀνείμονος εἶδος ἀνάσσης 
ὄμματι λαθριδίῳ δεδοκημένον † ὄμματι λοξῷ 
Νηιὰς ἀκρήδεμνος ἀπόπροθεν ἔδρακε Νύμφη, 
ταρβαλέη δ' ὀλόλυξεν, ἑῇ δ' ἤγγειλεν ἀνάσσῃ 
ἀνδρὸς ἐρωμανέος θράσος ἄγριον· ἡμιφανὴς δὲ 




Nonnus Epic., Dionysiaca 
Book 5, line 403

δύσμορον Αὐτονόην οὐ μέμφομαι· οἰχομένου γὰρ 
ὀφθαλμοὺς βροτέους οὐκ ἔδρακεν, οὐκ ἴδε μορφῆς 
ἀνδρομέης ἴνδαλμα, καὶ οὐκ ἐνόησεν ἰούλων 
ἄνθεϊ πορφυρέῳ κεχαραγμένον ἀνθερεῶνα. 



Nonnus Epic., Dionysiaca 
Book 6, line 187

ἔνθα λεοντείοιο λιπὼν ἴνδαλμα προσώπου 
ὑψιλόφῳ χρεμετισμὸν ὁμοίιον ἔβρεμεν ἵππῳ 
ἄζυγι, γαῦρον ὀδόντα μετοχμάζοντι χαλινοῦ, 
καὶ πολιῷ λεύκαινε περιτρίβων γένυν ἀφρῷ· 
ἄλλοτε ῥοιζήεντα χέων συριγμὸν ὑπήνης 
ἀμφιλαφὴς φολίδεσσι δράκων ἐλέλικτο κεράστης, 
γλῶσσαν ἔχων προβλῆτα κεχηνότος ἀνθερεῶνος, 
καὶ βλοσυρῷ Τιτῆνος ἐπεσκίρτησε καρήνῳ 
ὅρμον ἐχιδνήεντα περίπλοκον αὐχένι δήσας· 
καὶ δέμας ἑρπηστῆρος ἀειδίνητον ἐάσσας 




Nonnus Epic., Dionysiaca 
Book 6, line 215

ἀντολίην δ' ἔφλεξε, καὶ αἰθαλόεντι βελέμνῳ 
αἴθετο Βάκτριον οὖδας ἑώιον, ἀγχιπόροις δὲ 
κύμασιν Ἀσσυρίοισιν ἐδαίετο Κάσπιον ὕδωρ, 
Ἰνδῷοί τε τένοντες· Ἐρυθραίοιο δὲ κόλπου 
ἔμπυρα κυμαίνοντος Ἄραψ θερμαίνετο Νηρεύς. 



Nonnus Epic., Dionysiaca 
Book 7, line 98

τοῦτον ἀεθλεύσαντα μετὰ χθόνα σύνδρομον ἄστρων, 
Γηγενέων μετὰ δῆριν, ὁμοῦ μετὰ φύλοπιν Ἰνδῶν 
Ζηνὶ συναστράπτοντα δεδέξεται αἰόλος αἰθήρ. 



Nonnus Epic., Dionysiaca 
Book 7, line 164

καὶ τότε Τειρεσίαο δεδεγμένος ἔνθεον ὀμφὴν 
παῖδα πατὴρ προέηκεν ἐς ἠθάδα νειὸν Ἀθήνης 
Ζηνὶ θυηπολέουσαν ἀκοντιστῆρι κεραυνοῦ 
ταῦρον ὁμοκραίροιο φυῆς ἴνδαλμα Λυαίου, 
καὶ τράγον ἐσσομένης σταφυλητόμον ἐχθρὸν ὀπώρης. 



Nonnus Epic., Dionysiaca 
Book 9, line 27

καί μιν ἀχυτλώτοιο διαΐσσοντα λοχείης 
πήχεϊ κοῦρον ἄδακρυν ἐκούφισε σύγγονος Ἑρμῆς, 
καὶ βρέφος εὐκεράοιο φυῆς ἴνδαλμα Σελήνης 
ὤπασε θυγατέρεσσι Λάμου ποταμηίσι Νύμφαις, 
παῖδα Διὸς κομέειν σταφυληκόμον· αἱ δὲ λαβοῦσαι 
Βάκχον ἐπηχύναντο, καὶ εἰς στόμα παιδὸς ἑκάστη 
ἀθλιβέων γλαγόεσσαν ἀνέβλυεν ἰκμάδα μαζῶν. 



Nonnus Epic., Dionysiaca 
Book 9, line 149

ὑψόθεν ἀστήρικτος· ὁ δὲ δρόμον ἔφθασεν Ἥρης, 
πρωτογόνου δὲ Φάνητος ἀτέρμονα δύσατο μορφήν· 
καὶ θεὸν ἁζομένη πρωτόσπορον εἴκαθεν Ἥρη 
ψευδομένας ἀκτῖνας ὑποπτήσσουσα προσώπου, 
οὐδὲ νόθης ἐνόησε δολοπλόκον εἰκόνα μορφῆς· 
κουφοτέροις δὲ πόδεσσιν ὀρειάδα πέζαν ἀμείβων, 
χερσὶ περιπλεκέεσσι κερασφόρον υἷα κομίζων, 
μητρὶ Διὸς γενέταο λεοντοβότῳ πόρε Ῥείῃ, 
καί τινα μῦθον ἔειπεν ἀριστώδινι θεαίνῃ· 
 “δέξο, θεά, νέον υἷα τεοῦ Διός, ὃς μόθον Ἰνδῶν 
ἀθλεύσας μετὰ γαῖαν ἐλεύσεται εἰς πόλον ἄστρων. 



Nonnus Epic., Dionysiaca 
Book 9, line 190

καὶ χροῒ λαχνήεντας ἀνεχλαίνωσε χιτῶνας 
Εὔιος ἀρτιτέλεστον ἔχων παιδήιον ἥβην, 
δαιδαλέην ἐλάφοιο φέρων ὤμοισι καλύπτρην, 
αἰθερίων μιμηλὸν ἔχων τύπον αἰόλον ἄστρων· 
καὶ Φρυγίης ὑπὸ πέζαν ἐς αὔλια λύγκας ἐλάσσας 
στικτοῖς πορδαλίεσσιν ἑὴν ἔζευξεν ἀπήνην, 
οἷά τε πατρῴων δαπέδων ἴνδαλμα γεραίρων· 
πολλάκι δ' ἀθανάτης ἐποχημένος ἅρματι Ῥείης, 
βαιῇ χειρὶ φέρων ἁπαλόχροϊ κύκλα χαλινοῦ, 
κραιπνὸν ἐπειγομένων ἀνεσείρασεν ἅρμα λεόντων· 
καὶ Διὸς ὑψιμέδοντος ἐνὶ φρεσὶ θάρσος ἀέξων 
δεξιτερὴν ἐτίταινεν ἐπὶ στόμα λυσσάδος ἄρκτου, 
σμερδαλέαις γενύεσσιν ἀταρβέα δάκτυλα βάλλων. 



Nonnus Epic., Dionysiaca 
Book 10, line 18

Ταρταρίης δ' ὀφιῶδες ἰδὼν ἴνδαλμα θεαίνης 
πάλλετο δειμαίνων ἑτερόχροα φάσματα μορφῆς, 
ἀφρὸν ἀκοντίζων χιονώδεα, μάρτυρα λύσσης, 
ὀφθαλμοὺς μεθύοντας ἀπειλητῆρας ἑλίσσων. 



Nonnus Epic., Dionysiaca 
Book 10, line 216

γινώσκω τεὸν αἷμα, καὶ εἰ κρύπτειν μενεαίνεις· 
Ἠελίῳ σε λόχευσε παρευνηθεῖσα Σελήνη 
Ναρκίσσῳ χαρίεντι πανείκελον· αἰθέριον γὰρ 
θέσκελον εἶδος ἔχεις, κεραῆς ἴνδαλμα Σελήνης. 



Nonnus Epic., Dionysiaca 
Book 11, line 310

ὤμοι, ὅτ' οὐκ Ἀίδης πέλεν ἤπιος, οὐδ' ἐπὶ νεκρῷ 
δέχνυται ἀγλαὰ δῶρα βαθυπλούτοιο μετάλλου, 
Ἄμπελον ὄφρα θανόντα πάλιν ζώοντα τελέσσω· 
ὤμοι, ὅτ' οὐκ Ἀίδης ποτὲ πείθεται· ἢν δ' ἐθελήσῃ, 
ὄλβον ὅλον στίλβοντα χαρίζομαι Ἠριδανοῖο 
δένδρεα συλήσας ποταμήια, μαρμαρέην δὲ 
ἄξομαι ἀστράπτουσαν Ἐρυθραίην λίθον Ἰνδῶν 
ἀφνειῆς τ' Ἀλύβης ὅλον ἄργυρον, ἀντὶ δὲ νεκροῦ 
παιδὸς ἐμοῦ χρύσειον ὅλον Πακτωλὸν ὀπάσσω. 



Nonnus Epic., Dionysiaca 
Book 12, line 308

ἀγριὰς ἡβώουσα πολυγνάμπτοισιν ἑλίνοις 
οἰνοτόκων βλάστησε φυτῶν εὐάμπελος ὕλη, 
ὑγρὸν ἀναβλύζουσα βεβυσμένον ὄγκον ἐέρσης· 
καὶ πολὺς ὄρχατος ἦεν, ὅπῃ, στοιχηδὸν ἀνέρπων, 
σείετο φοινίσσων ἐπὶ βότρυϊ βότρυς ἀλήτης· 
ὧν ὁ μὲν ἡμιτέλεστος ἑὰς ὠδῖνας ἀέξων 
αἰόλα πορφύρων, ἑτερόχροϊ φαίνετο καρπῷ, 
ὃς δὲ φαληριόων ἐπεπαίνετο σύγχροος ἀφρῷ, 
καὶ πολὺς ὤθεεν ἄλλος ὁμόζυγα γείτονα γείτων 
ξανθοφυής, ἕτερος δὲ φυὴν ἰνδάλλετο πίσσῃ 
περκάζων ὅλον ἄνθος, ἀπ' οἰνοτόκων δὲ πετήλων 
σύμφυτον ἀγλαόκαρπον ὅλην ἐμέθυσσεν ἐλαίην· 
ἄλλου δ' ἀρτιχάρακτος ἐπέτρεχεν ὄμφακι καρπῷ 
βότρυος ἀργυφέοιο μέλας αὐτόσσυτος ἀήρ, 
ὄγκῳ βοτρυόεντι φέρων σφριγόωσαν ὀπώρην· 
καὶ πίτυν ἀγχικέλευθον ἕλιξ ἔστεψεν ὀπώρης   
συμφερτοῖς σκιόωσα περισκεπὲς ἔρνος ἰάμνοις, 
καὶ φρένα Πανὸς ἔτερπε· τινασσομένους δὲ Βορῆι 
ἀκρεμόνας πελάσασα παρ' ἀμπελόεντι κορύμβῳ 




Nonnus Epic., Dionysiaca 
Book 13, line 3

Ζεὺς δὲ πατὴρ προέηκεν ἐς αὔλια θέσκελα Ῥείης 
Ἶριν ἀπαγγέλλουσαν ἐγερσιμόθῳ Διονύσῳ, 
ὄφρα δίκης ἀδίδακτον ὑπερφιάλων γένος Ἰνδῶν 
Ἀσίδος ἐξελάσειεν ἑῷ ποινήτορι θύρσῳ, 
ναύμαχον ἀμήσας ποταμήιον υἷα κεράστην, 
Δηριάδην βασιλῆα, καὶ ἔθνεα πάντα διδάξῃ 
ὄργια νυκτιχόρευτα καὶ οἴνοπα καρπὸν ὀπώρης. 



Nonnus Epic., Dionysiaca 
Book 13, line 20

καὶ τὴν μὲν Κορύβαντες ἀμειδέι νεύματι Ῥείης 
θεσπεσίης ἀρέσαντο παρὰ κρητῆρι τραπέζης· 
θαμβαλέη δὲ πιοῦσα νεηγενέος χύσιν οἴνου 
τέρπετο βακχευθεῖσα· καρηβαρέουσα δὲ δαίμων 
παιδὶ Διὸς παρεόντι Διὸς μυκήσατο βουλήν· 
 “ἀλκήεις Διόνυσε, τεὸς γενέτης σε κελεύει   
εὐσεβίης ἀδίδακτον ἀιστῶσαι γένος Ἰνδῶν. 



Nonnus Epic., Dionysiaca 
Book 13, line 91

τοῖος ἐὼν ἔτι κοῦρος, ἔχων παιδήιον ἥβην, 
ἁβροκόμης Ὑμέναιος ἐδύσατο φύλοπιν Ἰνδῶν, 
δινεύων ἑκάτερθε παρηίδος ἥλικα χαίτην· 
καί οἱ ἐφωμάρτησαν ὁμήλυδες ἀσπιδιῶται, 
οἵ τ' Ἀσπληδόνος ἄστυ, καὶ ὃν Χάρις οὔ ποτε λείπει 
Ὀρχομενὸν Μινύαο, χοροίτυπον ἄλσος Ἐρώτων, 
οἵ θ' Ὑρίην ἐνέμοντο, θεηδόχον οὖδας ἀρούρης, 
ξεινοδόκου μεθέπουσαν ἐπωνυμίην Ὑριῆος, 
ἧχι Γίγας ἀπέλεθρος ἀπειρογάμων ἀπὸ λέκτρων   
Ὠρίων τριπάτωρ ἀπὸ μητέρος ἄνθορε Γαίης, 
εὖτε θεῶν τριγόνοισιν ἀεξηθεῖσα γενέθλαις 




Nonnus Epic., Dionysiaca 
Book 13, line 121

Βοιωτῶν τόσος ἦλθεν ἀμετρήτων στόλος ἀνδρῶν 
Ἰνδῴην ἐπὶ δῆριν ὁμαρτήσας Ὑμεναίῳ. 



Nonnus Epic., Dionysiaca 
Book 13, line 243


τοῖος ἀπὸ Κρήτης πρόμος ἤλυθεν· ἐρχομένῳ δὲ 
θερμοτέραις ἀκτῖσι χέων μαντήιον αἴγλην 
Ἀστερίῳ σελάγιζεν ὁμώνυμος Ἄρεος ἀστήρ, 
νίκης ἐσσομένης πρωτάγγελος· ἀλλ' ἐνὶ χάρμῃ   
νικήσας νόθον οἶστρον ἀήθεος ἔσχεν ἀρούρης 
νηλής· οὐ γὰρ ἔμελλεν ἰδεῖν μετὰ φύλοπιν Ἰνδῶν 
πάτριον Ἰδαίης κορυθαιόλον ἄντρον ἐρίπνης, 
ἀλλὰ βίον προβέβουλε λιπόπτολιν, ἀντὶ δὲ Δίκτης 
Κνώσσιος ἐν Σκυθίῃ μετανάστιος ἔσκε πολίτης, 
καὶ πολιὸν Μίνωα καὶ Ἀνδρογένειαν ἐάσσας 
ξεινοφόνων σοφὸς ἦλθεν ἐς ἔθνεα βάρβαρα Κόλχων 
Ἀστερίους τ' ἐκάλεσσε καὶ ὤπασεν οὔνομα λαοῖς 
Κρητικόν, οἷς ξένα θεσμὰ φύσις πόρε, παιδοκόμου δὲ 
πάτριον Ἀμνισοῖο ῥόον Κρηταῖον ἐάσσας 
αἰδομένοις στομάτεσσι νόθον πίε Φάσιδος ὕδωρ. 



Nonnus Epic., Dionysiaca 
Book 13, line 275

ἔνθεν Ἀρισταῖος βραδὺς ἤιεν εἰς μόθον Ἰνδῶν, 
ὄψιμος εὐνήσας πρότερον χόλον, Ἀρκάδα πέτρην, 
ἔνδιον Ἑρμείαο λιπὼν Κυλλήνιον ἕδρην· 
οὔ πω γὰρ προτέρῃ Μεροπηίδι νάσσατο νήσῳ,   
οὔ πω δ' ἀτμὸν ἔπαυσε πυρώδεα διψάδος ὥρης 
Ζηνὸς ἀλεξικάκοιο φέρων φυσίζοον αὔρην, 
οὐδὲ σιδηροχίτων δεδοκημένος ἀστέρος αἴγλην 
Σείριον αἰθαλόεντος ἀναστέλλων πυρετοῖο 
ἐννύχιος πρήυνε, τὸν εἰσέτι διψαλέον πῦρ 
θερμὸν ἀκοντίζοντα δι' αἰθέρος αἴθοπι λαιμῷ 




Nonnus Epic., Dionysiaca 
Book 13, line 372

κείνου μνῆστιν ἔχοντες ἐπεστρατόωντο μαχηταὶ 
μαρναμένου Βρομίοιο προασπιστῆρες ἐνυοῦς, 
τικτομένης ναίοντες ἐδέθλια γείτονα Μήνης 
καὶ Διὸς Ἀσβύσταο μεσημβρίζοντας ἐναύλους, 
μαντιπόλου κερόεντος, ὅπῃ ποτὲ πολλάκις Ἄμμων 
ἀρνειοῦ τριέλικτον ἔχων ἴνδαλμα κεραίης 
ὀμφαίοις στομάτεσσιν ἐθέσπισεν Ἑσπέριος Ζεύς· 
οἵ τε ῥόον Χρεμέταο καὶ οἳ παρὰ Κίνυφος ὕδωρ 
ᾤκεον ἀζαλέης ψαμαθώδεα πέζαν ἀρούρης, 
Αὐσχῖσαι Βάκαλές τε συνήλυδες, οὓς πλέον ἄλλων 
ἄρεϊ τερπομένους Ζεφυρήιος ἔτρεφεν ἀγκών. 



Nonnus Epic., Dionysiaca 
Book 13, line 418

τοῖσι κορυσσομένοισι σὺν εὐθύρσῳ Διονύσῳ 
Ἠλέκτρης ἀνέτελλε δι' αἰθέρος ἕβδομος ἀστὴρ 
δεξιὸν ὑσμίνης σημήιον, ἀμφὶ δὲ νίκῃ 
Πληιάδων κελάδησε βοῆς ἀντίθροος ἠχὼ 
γνωτῆς αἷμα φέροντι χαριζομένη Διονύσῳ, 
καὶ στρατιῇ πόρε θάρσος ὁμοίιον· ἐρχομένων δὲ 
Ὤγυρος ἡγεμόνευεν ἐς ἄρεα δεύτερος Ἄρης, 
Ὤγυρος ὑψικάρηνος, ἔχων ἴνδαλμα Γιγάντων· 
τοῦ μὲν ἔην ἄγναμπτον ὅλον δέμας, ἐκ δὲ καρήνου 
αὐχενίου τε τένοντος ὀπισθοκόμων ἐπὶ νώτων 
ἰσοφανεῖς πλοκαμῖδες ἀκανθοφόροισιν ἐχίνοις 
ἔρρεον ἰξύος ἄχρι κατήλυδες· εἶχε δὲ δειρὴν 
μηκεδανήν, περίμετρον, ὁμοίιον αὐχένι πέτρης, 
βάρβαρον ἦθος ἔχων πατρώιον· οὐδέ τις αὐτοῦ   
φέρτερος ἄλλος ἵκανεν Ἑώιον ἐς μόθον Ἰνδῶν 
νόσφι Διωνύσοιο· καὶ ὅρκιον ὤμοσε Νίκην 
Ἰνδῴην χθόνα πᾶσαν ἑῷ δορὶ μοῦνος ὀλέσσαι. 



Nonnus Epic., Dionysiaca 
Book 13, line 425

καὶ στρατιῇ πόρε θάρσος ὁμοίιον· ἐρχομένων δὲ 
Ὤγυρος ἡγεμόνευεν ἐς ἄρεα δεύτερος Ἄρης, 
Ὤγυρος ὑψικάρηνος, ἔχων ἴνδαλμα Γιγάντων· 
τοῦ μὲν ἔην ἄγναμπτον ὅλον δέμας, ἐκ δὲ καρήνου 
αὐχενίου τε τένοντος ὀπισθοκόμων ἐπὶ νώτων 
ἰσοφανεῖς πλοκαμῖδες ἀκανθοφόροισιν ἐχίνοις 
ἔρρεον ἰξύος ἄχρι κατήλυδες· εἶχε δὲ δειρὴν 
μηκεδανήν, περίμετρον, ὁμοίιον αὐχένι πέτρης, 
βάρβαρον ἦθος ἔχων πατρώιον· οὐδέ τις αὐτοῦ   
φέρτερος ἄλλος ἵκανεν Ἑώιον ἐς μόθον Ἰνδῶν 
νόσφι Διωνύσοιο· καὶ ὅρκιον ὤμοσε Νίκην 
Ἰνδῴην χθόνα πᾶσαν ἑῷ δορὶ μοῦνος ὀλέσσαι. 



Nonnus Epic., Dionysiaca 
Book 13, line 427

Ὤγυρος ἡγεμόνευεν ἐς ἄρεα δεύτερος Ἄρης, 
Ὤγυρος ὑψικάρηνος, ἔχων ἴνδαλμα Γιγάντων· 
τοῦ μὲν ἔην ἄγναμπτον ὅλον δέμας, ἐκ δὲ καρήνου 
αὐχενίου τε τένοντος ὀπισθοκόμων ἐπὶ νώτων 
ἰσοφανεῖς πλοκαμῖδες ἀκανθοφόροισιν ἐχίνοις 
ἔρρεον ἰξύος ἄχρι κατήλυδες· εἶχε δὲ δειρὴν 
μηκεδανήν, περίμετρον, ὁμοίιον αὐχένι πέτρης, 
βάρβαρον ἦθος ἔχων πατρώιον· οὐδέ τις αὐτοῦ   
φέρτερος ἄλλος ἵκανεν Ἑώιον ἐς μόθον Ἰνδῶν 
νόσφι Διωνύσοιο· καὶ ὅρκιον ὤμοσε Νίκην 
Ἰνδῴην χθόνα πᾶσαν ἑῷ δορὶ μοῦνος ὀλέσσαι. 



Nonnus Epic., Dionysiaca 
Book 13, line 500

τοὺς δὲ λίγα κροτέοντας ὑπ' εὐρύθμῳ χθόνα ταρσῷ 
καὶ Στάβιος καὶ Στάμνος ἐπὶ κλόνον ὥπλισαν Ἰνδῶν·   
καὶ στρατὸν ὀρχηστῆρα περισκαίροντα δοκεύων 
τοῖον ἔπος λέξειας, ὅτι πρόμος ἡγεμονεύει 
εἰς χορόν, οὐκ ἐπὶ δῆριν, ἐνόπλιον ἄνδρα κομίζων· 
τοῖσι γὰρ ἐρχομένοισιν ἀνακρούουσα χορείην 
Μυγδονὶς ἐγρεκύδοιμος ἐπὶ κλόνον ἔβρεμε φόρμιγξ, 
ἀντὶ χοροῦ πέμπουσα μόθου λαοσσόον ἠχώ· 
καὶ πολέμων σάλπιγγες ἔσαν σύριγγες Ἐρώτων, 
καὶ δίδυμοι Βερέκυντες ὁμόζυγες ἔκλαγον αὐλοί, 
καὶ κτύπον ἀμφιπλῆγα βαρυσμαράγων ἀπὸ χειρῶν 




Nonnus Epic., Dionysiaca 
Book 13, line 549

Ἀστερίου δ' ἀπάνευθεν ἑοῦ γενέταο μολόντος 
ἀρτιθαλὴς Μίλητος ὁμόστολος ἵκετο Βάκχῳ 
Καῦνον ἔχων συνάεθλον ἀδελφεόν, ὃς τότε Καρῶν 
λαὸν ἄγων ἔτι κοῦρος ἐδύσατο φύλοπιν Ἰνδῶν· 
οὔ πω γὰρ δυσέρωτα δολοπλόκον ἔπλεκε μολπὴν 
γνωτῆς οἶστρον ἔχων ἀδαήμονος, οὐδὲ καὶ αὐτὴν 
ἀντιτύπου φιλότητος ὁμοζήλων ἐπὶ λέκτρων 
Ζηνὶ συναπτομένην ἐμελίζετο σύγγονον Ἥρην 
Λάτμιον ἀμφὶ βόαυλον ἀκοιμήτοιο νομῆος, 
ὀλβίζων ὑπ' ἔρωτι μεμηλότα γείτονι πέτρῃ 
νυμφίον Ἐνδυμίωνα ποθοβλήτοιο Σελήνης· 
ἀλλ' ἔτι Βυβλὶς ἔην φιλοπάρθενος, ἀλλ' ἔτι θήρην 
Καῦνος ὁμογνήτων ἐδιδάσκετο νῆις ἐρώτων· 




Nonnus Epic., Dionysiaca 
Book 14, line 36

καὶ φθονεροὶ Τελχῖνες ἐπήλυδες ἐς μόθον Ἰνδῶν 
ἐκ βυθίου κενεῶνος ἀολλίζοντο θαλάσσης·   
καὶ δολιχῇ παλάμῃ δονέων περιμήκετον αἰχμὴν 
ἦλθε Λύκος, καὶ Σκέλμις ἐφέσπετο Δαμναμενῆι 
πάτριον ἰθύνων Ποσιδήιον ἅρμα θαλάσσης, 
Τληπολέμου μετὰ γαῖαν ἁλιπλανέες μετανάσται, 
δαίμονες ὑγρονόμοι μανιώδεες, οὓς πάρος αὐτοὶ 
πατρῴης ἀέκοντας ἀποτμήξαντες ἀρούρης 
Θρῖναξ σὺν Μακαρῆι καὶ ἀγλαὸς ἤλασεν Αὔγης, 
υἱέες Ἠελίοιο· διωκόμενοι δὲ τιθήνης 




Nonnus Epic., Dionysiaca 
Book 14, line 177

τοῖσι μὲν οὐατόεσσα φυῆς ἰνδάλλετο μορφή, 
ἱππείη δ' ἀνέτελλε δι' ἰξύος ὄρθιος οὐρὴ 
ἰσχία μαστίζουσα δασυστέρνοιο φορῆος, 
καὶ βοέη βλάστησε κατὰ κροτάφοιο κεραίη, 
ὄμματα δ' εὐρύνοντο τανυκραίροιο μετώπου, 
καὶ σκολιαὶ πλοκαμῖδες ἀνηέξηντο καρήνων, 
γναθμοὶ δ' ἀργιόδοντες ἐμηκύνοντο γενείων, 
ξείνη δ' αὐτοτέλεστος ἀπ' ἰξύος εἰς πόδας ἄκρους 
ἀμφιλαφὴς λασίοιο κατ' αὐχένος ἔρρεε χαίτη. 



Nonnus Epic., Dionysiaca 
Book 14, line 218

μηκεδανὸν ζώεσκον ἐπὶ χρόνον, αἱ μὲν ἐρίπνας 
γείτονες οἰονόμων ἐπιμηλίδες, αἱ δὲ λιποῦσαι 
ἄλσεα δενδρήεντα καὶ ἀγριάδος ῥάχιν ὕλης, 
συμφυέες Μελίαι δρυὸς ἥλικος· αἳ τότε πᾶσαι 
ἐς μόθον ἠπείγοντο συνήλυδες, αἱ μὲν ἑλοῦσαι 
τύμπανα χαλκεόνωτα, Κυβηλίδος ὄργανα Ῥείης,   
αἱ δὲ κατηρεφέες πλοκάμους ἑλικώδεϊ κισσῷ, 
ἄλλαι ἐμιτρώθησαν ἐχιδναίοισι κορύμβοις· 
χειρὶ δὲ θύρσον ἄειρον ἀκαχμένον, αἷς τότε Λυδαὶ 
Μαινάδες ὡμάρτησαν ἀταρβέες ἐς μόθον Ἰνδῶν· 
ὧν τότε Βασσαρίδες θιασώδεες ἴδμονι τέχνῃ 
κρείσσονες ἠπείγοντο Διωνύσοιο τιθῆναι, 
Αἴγλη Καλλιχόρη τε καὶ Εὐπετάλη καὶ Ἰώνη 
καὶ Καλύκη γελόωσα Βρύουσά τε, σύννομος Ὥραις, 
Σιλήνη τε Ῥόδη τε καὶ Ὠκυνόη καὶ Ἐρευθὼ 
Ἀκρήτη τε Μέθη τε, καὶ ἕσπετο σύννομος Ἅρπῃ 
Οἰνάνθη ῥοδόεσσα καὶ ἀργυρόπεζα Λυκάστη, 
Στησιχόρη Προθόη τε· φιλομμειδὴς δὲ γεραιὴ 
οἰνοβαρὴς Τρυγίη πυμάτη κεκόρυστο καὶ αὐτή. 



Nonnus Epic., Dionysiaca 
Book 14, line 272

καὶ θεὸς εὐόρπηκος ἐφήμενος ἄντυγι δίφρου 
Σαγγαρίου παρὰ χεῦμα, περὶ Φρύγα κόλπον ἀρούρης, 
λαϊνέης Νιόβης παρεμέτρεε πενθάδα πέτρην· 
καὶ λίθος Ἰνδὸν ὅμιλον ἐριδμαίνοντα Λυαίῳ 
δακρυόεις ὁρόων βροτέην πάλιν ἴαχε φωνήν· 
 “μὴ μόθον ἐντύνητε θεημάχον, ἄφρονες Ἰνδοί, 
παιδὶ Διός, μὴ Βάκχος ἀπειλείοντας ἐνυὼ   
λαϊνέους τελέσειε καὶ ὑμέας, ὥς περ Ἀπόλλων, 
μυρομένους τύπον ἶσον ἐμῇ πετρώδεϊ μορφῇ, 
μὴ ποταμοῦ παρὰ χεῦμα φερώνυμον Ἰνδὸν Ὀρόντην 
γαμβρὸν ἐσαθρήσητε δεδουπότα Δηριαδῆος. 



Nonnus Epic., Dionysiaca 
Book 14, line 278

Σαγγαρίου παρὰ χεῦμα, περὶ Φρύγα κόλπον ἀρούρης, 
λαϊνέης Νιόβης παρεμέτρεε πενθάδα πέτρην· 
καὶ λίθος Ἰνδὸν ὅμιλον ἐριδμαίνοντα Λυαίῳ 
δακρυόεις ὁρόων βροτέην πάλιν ἴαχε φωνήν· 
 “μὴ μόθον ἐντύνητε θεημάχον, ἄφρονες Ἰνδοί, 
παιδὶ Διός, μὴ Βάκχος ἀπειλείοντας ἐνυὼ   
λαϊνέους τελέσειε καὶ ὑμέας, ὥς περ Ἀπόλλων, 
μυρομένους τύπον ἶσον ἐμῇ πετρώδεϊ μορφῇ, 
μὴ ποταμοῦ παρὰ χεῦμα φερώνυμον Ἰνδὸν Ὀρόντην 
γαμβρὸν ἐσαθρήσητε δεδουπότα Δηριαδῆος. 



Nonnus Epic., Dionysiaca 
Book 14, line 282

Ῥείη χωομένη δύναται πλέον Ἰοχεαίρης· 
Φοίβου φεύγετε Βάκχον ἀδελφεόν· αἰδέομαι γὰρ 
Ἰνδῶν κτεινομένων ἀλλότρια δάκρυα λείβειν. 



Nonnus Epic., Dionysiaca 
Book 14, line 294

                         φιλαγρύπνῳ δὲ Λυαίῳ 
πάννυχος ἀστερόεντα πυρίτροχον ὁλκὸν ὑφαίνων 
οὐρανὸς ἐβρόντησεν, ἐπεὶ τότε μάρτυρι πυρσῷ 
νίκης Ἰνδοφόνοιο τέλος μαντεύσατο Ῥείη. 



Nonnus Epic., Dionysiaca 
Book 14, line 303

Ἥρη δ' ὠκυπέδιλος, ἐειδομένη δέμας Ἰνδῷ, 
οὐλοκόμῳ Μελανῆι, μὴ οἴνοπα θύρσον ἀείρειν 
Ἀστράεντα κέλευε, δορυσσόον ὄρχαμον ἀνδρῶν, 
μηδὲ φιλακρήτων Σατύρων ἀλάλαγμα γεραίρειν, 
ἀλλὰ μάχην ἄσπονδον ἀναστῆσαι Διονύσῳ· 
καί τινα μῦθον ἔειπε παραιφαμένη πρόμον Ἰνδῶν· 
 “ἡδὺς ὁ δειμαίνων ἁπαλὴν στίχα θηλυτεράων. 



Nonnus Epic., Dionysiaca 
Book 14, line 323

καὶ στρατὸν ὥπλισε Βάκχος ἐς ἀντιπόρων στίχας Ἰνδῶν. 



Nonnus Epic., Dionysiaca 
Book 14, line 325

οὐδὲ λάθε ζοφόεντα Κελαινέα θῆλυς ἐνυώ, 
ἀλλὰ θορὼν ἀκίχητος ὅλον στρατὸν ὥπλισεν Ἰνδῶν· 
καὶ θρασὺς Ἀστράεις, μενεδήιον οἶστρον ἀέξων, 
Ἀστακίδος κελάδοντα περὶ ῥόον ἵστατο λίμνης, 
δέγμενος ἀμπελόεντος ἐπηλυσίην Διονύσου. 



Nonnus Epic., Dionysiaca 
Book 14, line 331

ἀλλ' ὅτε δὴ διδύμης στρατιῆς ἑτερόζυγι λαῷ 
ἀμφοτέρων στίχα πᾶσαν ἐκόσμεον ἡγεμονῆες, 
κλαγγῇ μὲν ζοφόεντες ἐπὶ κλόνον ἤιον Ἰνδοί, 
Θρηικίοις γεράνοισιν ἐοικότες, εὖτε φυγοῦσαι 
χειμερίην μάστιγα καὶ ἠερίου χύσιν ὄμβρου 
Πυγμαίων ἀγεληδὸν ἐπαΐσσουσι καρήνοις 
Τηθύος ἀμφὶ ῥέεθρα, καὶ ὀξυόεντι γενείῳ 
οὐτιδανῆς ὀλέκουσι λιποσθενὲς αἷμα γενέθλης,   
ἱπτάμεναι νεφεληδὸν ὑπὲρ κέρας Ὠκεανοῖο· 
εἰς ἐνοπὴν δ' ἑτέρωθεν ἐβακχεύοντο μαχηταί, 
ἀκλινέες θεράποντες ἐγερσιμόθου Διονύσου. 



Nonnus Epic., Dionysiaca 
Book 14, line 387

π<ο>λλὴ δ' ἔνθα καὶ ἔν<θα παρ' Ἀστακίδος στόμα λίμν>η<ς 
Ἰνδῴη δ>εδάικτο γο<νὴ Κουρῆτι σιδήρῳ. 



Nonnus Epic., Dionysiaca 
Book 14, line 410

<καὶ πολὺς ἐσμ>ὸς ἔπιπτεν· ὅλη δ' ἐρυθαίνετο λύθρῳ 
<ὑγρῷ διψὰς ἄ>ρουρα, καὶ Ἀστακίδος στόμα λίμνης   
<αἱμοβαφὲ>ς κελάρυζε, φόνῳ κεκερασμένον Ἰνδῶν. 



Nonnus Epic., Dionysiaca 
Book 14, line 417

                                                     .. 
<ὄχθαι ἐφο̣ι̣>νίσσοντο· πιὼν δέ τις Ἰνδὸς ἀγήνωρ 
<τοί>ην ἐκ στομάτων πολυθαμβέα ῥήξατο φωνήν· 
 “<ξε>ῖνον ἴδον καὶ ἄπιστον ἐγὼ ποτόν· ὡς γλάγος αἰγῶν 
ἄργυφον οὐ πέλε τοῦτο, καὶ οὐ μέλαν οἷά περ ὕδωρ, 
οὐδέ μιν οἷον ὄπωπα πολυτρήτοις ἐνὶ σίμβλοις 
βομβήεσσα μέλισσα λοχεύεται ἡδέι κηρῷ· 
ἀλλὰ νόον τέρπουσαν ἔχει καλλίπνοον ὀδμήν. 



Nonnus Epic., Dionysiaca 
Book 15, line 1

<Ὣ>ς φα<μένου νεφεληδὸν ἐπέρρεον αἴθ>οπ<ες Ἰνδοὶ 
ἀ>μφὶ ῥό<ον ποταμοῖο μελίπνοον· ὧν ὁ μ>ὲν <αὐτῶν> 
ἀγχιβαθ<ὴς στατὸν ἴχν>ο<ς ἐπ' ἰλύι δισσὸν> ἐρεί<σας> 
ἡμιφανὴς ἕστηκε, καὶ ὀμ<φαλὸν ὕδατ>ι δεύω<ν>, 
κυρτὸς ἔσω ποταμοῖο κεκυφότα νῶτα τιταί<νων>, 
χερσὶ βαθυνομένῃσι μελισταγὲς ἤφυσεν <ὕδωρ>· 
ὃς δὲ παρὰ προχοῇσι, κατάσχετος αἴθοπι δίψ<ῃ>, 
πορφυρέῳ προβλῆτα γενειάδα κύματι βάπτων, 
στῆθος ἐφαπλώσας ποταμηίδος ὑψόθεν ὄχθης, 
οἰγομένοις στομάτεσσιν ἀνήφυσεν ἰκμάδα Βά<κχου>· 




Nonnus Epic., Dionysiaca 
Book 15, line 26

                                         ἔνθα τις ἀνὴρ 
Ἰνδὸς ἀμερσινόοιο μέθης δεδονημένος οἴστρῳ 
εἰς ἀγέλην ἤιξε, καὶ εὐπετάλῳ παρὰ λόχμῃ 
ταῦρον ἀπειλητῆρα μετήγαγε δέσμιον ἕλκων, 
διχθαδίων κεράων κεχαραγμένον ἄκρον ἐρύσσας 
τολμηραῖς παλάμαις, διδυμάονος οἷα κεραίης 
ταυροφυῆ Διόνυσον <ὑπὸ ζυγὰ δούλια σύρων· 
ἄλλος ἔχ>ων δασπλ<ῆτα σιδηρείης γένυν ἅρπης 
αἰγὸς ὀρε̣>σς<ινόμοιο διέθρισεν ἀ̣ν̣θερεῶνα, 
θηγαλέῳ δρεπάνῳ δεδαϊγμένον, οἷά τε δειρὴν 
Πανὸ>ς <ἐυκραίροιο ταμὼν γαμψώνυχι χαλκῷ· 




Nonnus Epic., Dionysiaca 
Book 15, line 43

αἰγὸς ὀρε̣>σς<ινόμοιο διέθρισεν ἀ̣ν̣θερεῶνα, 
θηγαλέῳ δρεπάνῳ δεδαϊγμένον, οἷά τε δειρὴν 
Πανὸ>ς <ἐυκραίροιο ταμὼν γαμψώνυχι χαλκῷ· 
ἄλλος ἀπηλοίησε βοῶν κεραελκέα φύτλην, 
οἷά περ ἀμώων Σα̣τύ̣ρων τα̣υρώπ̣>ιδα <μο̣>ρ<φή>ν,   
<ὃς δὲ τανυκραίρων ἐλάφ>ων ἐδίωκε γεν<έθλην 
στικτῆς εἰσορόων πολυδαί̣>δαλον εἶδος ὀ<πωπῆς, 
οἷά τε Βασσαρίδων ὀλέκω>ν στίχα· <δαιδαλέαις γὰρ 
νεβρίσιν ἰσοτύποισι παρεπλά̣>γχθ<ησαν ὀπωπαί· 
καὶ φονίαις λιβάδε̣>ς<σ̣ι̣ν ὅλον θώρη̣κα μιαίνων 
Ἰνδὸς ἀκοντιστῆ>ρι <μέλας ἐρυθαίνετο λύθρῳ. 



Nonnus Epic., Dionysiaca 
Book 15, line 75

ὄργια μιμήσαντο φερεσσακέων Κορυβάντων, 
ἴχνια δινεύοντες ἐνόπλιον ἀμφὶ χορείην· 
κ<αὶ π̣α̣>λάμης ἑλικηδὸν ἀμοιβαίῃσιν ἐρωαῖς 
<ἀσπίδ>ες ἐκρούο<ν̣το κ>υβιστητῆρι σιδήρῳ· 
<ἄλλος> ὀπιπεύω<ν θιας>ώδεος ὄργια Μούσης 
<μιμηλὴ>ν Σατύ<ροισι συ>νεσκίρτησε χορείην· 
<καί τις ἀ̣>ρασσο<μένης> ἀίων κελάδημα βοείης 
<μ>είλιχον ἦθος ἔδεκτο, φιλοσμαράγῳ δὲ μενοινῇ 
<ῥιγεδανὴν ἀνέμοισιν ἑὴν ἔρριψε> φαρέτρην, 
<λύσσα̣ν ἔχων· ἕτερος δὲ γυναιμανέω>ν πρόμο<σ̣ Ἰνδῶν   
ἀπλεκέος πλοκαμῖδος ἑλὼν ὑψαύχενα Βάκχ̣η̣ν, 
παρθενικὴν ἀδάμαστον ἀτάσθαλον εἰς γάμον ἕλκων, 
σφίγξεν ὑπὲρ δαπέδοιο, τανυσσάμενος δὲ κονίῃ 
χερσὶν ἐρωμανέεσσιν ἀπεσ̣φρηγίσσατο μίτρην, 
ἐλπίδι μαψι̣>δίῃ π<ε̣φ̣>ορημέν<ος· ἐξαπίνης γὰρ 
ὄρθιος εἷρπε δρ>άκων ὑποκόλπ<ιος ἰξύι γείτων, 
δυσμενέος δ' ἤι>ξε κατ' αὐχέν<ος, ἀμφὶ δὲ δειρῇ 
οὐραίαις ἑλίκεσσ̣>ιν ἀνέπλεκε <κ̣υ̣κλάδα μίτρην· 
ταρβαλέοις δὲ πόδες>σι φυ<γὼν μελανόχροος ἀνὴρ 




Nonnus Epic., Dionysiaca 
Book 15, line 87

ὄφ>ρα <μὲν οἰνωθέντες ἐν οὔρεσιν ἔτ̣>ρ<ε̣χον Ἰνδοί, 
τό>φρα <δ̣ὲ νήδυμος Ὕπνος ἑὸν πτερὸν οὖ̣>λο<ν ἑλίξας 
ἀκλινέων σφαλεροῖσιν ἐπέχραεν ὄ>μμ<ασιν Ἰνδῶν, 
εὔνασε δ̣' οἰστρηθέντας ἀμετρήτῳ ν>όον ο<ἴνῳ, 
Πα>σιθέης γε<ν̣ε̣>τῆ<ρι> χαρι<ζ̣ό̣μενος Δι>ονύσῳ· 
<ὧ>ν ὁ μὲν ὕπτιος εὗδεν ἄνω νεύον<τι προσώ>π<ῳ 
ὑ>πναλέῳ μυκτῆρι μεθυσφαλὲς <ἄσθμα τιταίνων>,   
ὃς δὲ βαρυνομένην κεφαλὴν <ἐπεθήκα>τ<ο πέτρῳ>, 
νωθρὸς ἐυκροκάλῳ ποταμη<ίδι κείμενος> ὄ<χθῃ>, 
ἠματίοις δ' ὀάριζε νοοπλαν<έεσσιν ὀνείροι>ς 




Nonnus Epic., Dionysiaca 
Book 15, line 121

κα>ὶ <δηί>ους κνώ<σ̣σ̣ο̣ν̣τ̣>ας ἰδ<ὼν γελόωντι προσώπῳ 
Βάκχος ἄν̣>α<ξ̣ ἀγόρευε, χέ>ων <σημάντορα φωνήν· 
 “Ἰνδοφόνοι θεράποντες ἀνικήτου Διονύσου, 
νόσφ>ι <μόθου σφίγξαντες ἀολλέας υἱέας Ἰνδῶν 
πάντας ἀναιμάκτῳ ζωγρήσατε δηιοτῆτι· 
καὶ βριαρῷ γόνυ δο̣>ῦ<λ̣>ον <ὑποκλίν>ας Διονύσῳ 
<Ἰνδὸς ὑποδρήσσειεν ἐμῇ> θιασώδεϊ Ῥείῃ, 
<σείων οἴνοπα θύρσον, ἀπορρί̣>ψας δὲ θυέλλαις 
<ἀργυρέην κνημῖδα πόδας σφ>ίγξειε κοθόρνοις, 
<καὶ κεφαλὴν> ς<τέψειεν ἐμ̣>ῷ <κ̣>ισσώδε<ϊ̣ δεσμῷ, 
γυμνώσας πλ̣οκαμῖδας ἀερσιλό>φου <τρυφαλείης, 




Nonnus Epic., Dionysiaca 
Book 15, line 142

<αὐχένι δυσμ>ενέων ὀ<φιώδεα δεσμὸν ἑλίξας   
εἷλκε> δρακοντείῃ πεπ<εδημένον ἀνέρα σειρῇ, 
ἄλλ̣>ος ἑλὼν λασίης κεχα<λασμέν>ο<ν ὁλκὸν ὑπήνη̣>ς 
<ἄ>νδ<ρ̣>α βαθυσμήριγγος <ἀνείρ>υσεν ἀνθ<ερεῶ>νος· 
<καί τις> ἑὰς παλάμας τανύσας σκολιότ<ριχι κ>όρσῃ 
<ἀνέρα δ>ουρί<κ̣τητον ἀ>δέσμιον εἷλκεν ἐθείρης· 
<ἄλλος ὁ>μοπ<λέκτους π>αλάμας περὶ νῶτα καθάψας 
<δήιον> εἱλικ<όεντι λύγ>ων μιτρώσατ<ο δες>μῷ 
<αὐχεν>ίῳ· τρ<ομερῷ δὲ Μ>άρων ἐλελί<ζετ>ο παλμῷ 
<ὤμ̣>ῳ γηραλ<έῳ βεβ>αρημένον Ἰνδὸ<ν ἀείρ>ων· 
<ἄλ>λος ἀκοντιστῆρα λαβὼν βεβιημ<ένον> ὕπνῳ, 
<δεσμ>ῷ βοτρυόεντι περίπλοκον αὐχένα <σύ>ρων, 
<στικ>τῶν πορδαλίων ὑπὲρ ἄντυγα θήκατο δίφρων· 
<ἄλλου κ>εκλιμένοιο φιλεύιος ἐσμὸς ἀλήτης 
<χεῖρας ὀ>πισθοτόνους ἀλύτῳ σφηκώσατο δεσμῷ, 
<καὶ λο>φιῆς ἐπέβησεν ἀκαμπτοπόδων ἐλεφάντων· 
καὶ πολὺς εὐκύκλοιο λαβὼν τελαμῶνα βοείης   
Ἰνδὸν ἐπωμαδίῳ πεπεδημένον εἶχεν ἱμάντι. 



Nonnus Epic., Dionysiaca 
Book 15, line 150

<δήιον> εἱλικ<όεντι λύγ>ων μιτρώσατ<ο δες>μῷ 
<αὐχεν>ίῳ· τρ<ομερῷ δὲ Μ>άρων ἐλελί<ζετ>ο παλμῷ 
<ὤμ̣>ῳ γηραλ<έῳ βεβ>αρημένον Ἰνδὸ<ν ἀείρ>ων· 
<ἄλ>λος ἀκοντιστῆρα λαβὼν βεβιημ<ένον> ὕπνῳ, 
<δεσμ>ῷ βοτρυόεντι περίπλοκον αὐχένα <σύ>ρων, 
<στικ>τῶν πορδαλίων ὑπὲρ ἄντυγα θήκατο δίφρων· 
<ἄλλου κ>εκλιμένοιο φιλεύιος ἐσμὸς ἀλήτης 
<χεῖρας ὀ>πισθοτόνους ἀλύτῳ σφηκώσατο δεσμῷ, 
<καὶ λο>φιῆς ἐπέβησεν ἀκαμπτοπόδων ἐλεφάντων· 
καὶ πολὺς εὐκύκλοιο λαβὼν τελαμῶνα βοείης   
Ἰνδὸν ἐπωμαδίῳ πεπεδημένον εἶχεν ἱμάντι. 



Nonnus Epic., Dionysiaca 
Book 15, line 153

καί τις ἀερτάζουσα καλαύροπα μηλοβοτῆρος 
Βασσαρίς, ἀφριόωσα λαθίφρονι κύματι λύσσης, 
Ἰνδὸν ἐρευνητῆρα βαθυπλούτοιο θαλάσσης 
τολμηρῇ παλάμῃ πολυκαμπέος εἷλκεν ἐθείρης 
δούλιον εἰς ζυγόδεσμον. 



Nonnus Epic., Dionysiaca 
Book 16, line 121

ληιδίην δ' ὀπάσαιμι γονὴν μελανόχροον Ἰνδῶν 
παστάδος ὑμετέρης θαλαμηπόλον· ἀλλὰ τί φύτλην 
κυανέην ὀνόμηνα τεῆς νυμφοστόλον εὐνῆς; 



Nonnus Epic., Dionysiaca 
Book 16, line 138

εἰ δὲ μόθου λάχες οἶστρον, ἅτε κλυτότοξος Ἀμαζών, 
ἵξεαι Ἰνδῴην ἐπὶ φύλοπιν, ὄφρα κεν εἴης 
Πειθὼ νόσφι μόθοιο καί, ὁππότε δῆρις, Ἀθήνη. 



Nonnus Epic., Dionysiaca 
Book 16, line 235

ἔννεπεν ἄγχι φυτοῖο· δι' εὐπετάλου δὲ κορύμβου   
φθογγῆς εἰσαΐουσα γυναιμανέος Διονύσου 
ἀρχαίη Μελίη φιλοκέρτομον ἴαχε φωνήν· 
 “ἄλλοι μέν, Διόνυσε, κυνοσσόοι Ἰοχεαίρῃ 
ἐνθάδε θηρεύουσι, σὺ δ' ἀγρώσσεις Ἀφροδίτῃ· 
ἡδὺς ὁ δειμαίνων ἁπαλόχροον ἄζυγα κούρην· 
Βάκχος ὁ τολμήεις ἱκέτης πέλε λάτρις Ἐρώτων· 
Ἰνδοφόνοις παλάμῃσιν ἀνάλκιδα λίσσετο κούρην. 



Nonnus Epic., Dionysiaca 
Book 16, line 254

καὶ φλογερῷ Φαέθοντος ἱμασσομένης χρόα πυρσῷ 
ἄβροχα διψαλέης τερσαίνετο χείλεα κούρης·   
καὶ δόλον ἀγνώσσουσα γυναιμανέος Διονύσου 
ξανθὸν ὕδωρ ἐνόησε φιλακρήτου ποταμοῖο, 
καὶ πίεν ἡδὺ ῥέεθρον, ὅθεν πίον αἴθοπες Ἰνδοί· 
καὶ φρένα δινηθεῖσα μέθῃ βακχεύετο κούρη, 
καὶ κεφαλὴν ἐλέλιζε μετήλυδα δίζυγι παλμῷ, 
καὶ διδύμην ἐδόκησεν ἰδεῖν πολυχανδέα λίμνην 
ὄμματα δινεύουσα· βαρυνομένου δὲ καρήνου 
δέρκετο θηροβότου διπλούμενα νῶτα κολώνης· 
καὶ τρομεροῖσι πόδεσσιν ὀλισθήσασα κονίῃ 
εἰς πτερὸν αὐτοκύλιστος ἐσύρετο γείτονος Ὕπνου· 
καὶ γαμίῳ βαρύγουνος ἐθέλγετο κώματι νύμφη. 



Nonnus Epic., Dionysiaca 
Book 16, line 405

καὶ πόλιν εὐλάιγγα φιλακρήτῳ παρὰ λίμνῃ 
τεῦξε θεὸς Νίκαιαν, ἐπώνυμον ἣν ἀπὸ νύμφης 
Ἀστακίης ἐκάλεσσε καὶ Ἰνδοφόνον μετὰ νίκην. 



Nonnus Epic., Dionysiaca 
Book 17, line 2

Οὐδὲ φιλακρήτοιο μέθης πεπεδημένον ὕπνῳ 
ζωγρήσας ἀτίνακτον ἀνουτήτων γένος Ἰνδῶν 
ληθαίοις Διόνυσος ἐπέτρεπε δῆριν ἀήταις· 
ἀλλὰ πάλιν Φρύγα θύρσον ἐκούφισεν· ὑψιλόφου γὰρ 
εἰς ἐνοπὴν καλέοντος ἐπείγετο Δηριαδῆος, 
παιδὸς Ἀμαζονίης δολίην ἄμνηστον ἐάσσας 
οἰνοβαρῆ φιλότητα καὶ ὑπναλέους ὑμεναίους. 



Nonnus Epic., Dionysiaca 
Book 17, line 26

καὶ Βρομίῳ συνάεθλος ὅλος στρατὸς ἔρρεε Βάκχων, 
θάρσος ἔχων προτέροιο μόθου χάριν, ὁππότε δισσῷ 
ἡδυμανὴς ἀσίδηρος ὁμόζυγι πήχεϊ μάρψας 
ἔμφρονα νεκρὸν ἄναυδον, ἐνόπλιον Ἰνδὸν ἀείρων, 
Σιληνὸς βαρύγουνος ἐχάζετο νωθρὸς ὁδίτης, 
ὁππότε κωμάζουσα ποδῶν διδυμάονι ῥυθμῷ 
Βακχιὰς ἀκρήδεμνος ἐπεκροτάλιζε Μιμαλλὼν . 



Nonnus Epic., Dionysiaca 
Book 17, line 88

καλλείψας δὲ νομῆα καὶ ἀγριάδος ῥάχιν ὕλης 
εἰς ἑτέρην ἔσπευδεν ὀρειάδα φύλοπιν Ἰνδῶν· 
καὶ Σατύρων ὁμόφοιτον ὀρίδρομον ἴχνος ἐπείγων 
ἀμφιπόλοις παλίνορσος ὁμίλεε θυιάσι Βάκχαις. 



Nonnus Epic., Dionysiaca 
Book 17, line 96

διψώων δὲ φόνοιο καὶ εὐθύρσοιο κυδοιμοῦ, 
Τυρσηνῆς βαρύδουπον ἔχων σάλπιγγα θαλάσσης, 
πομπὸν Ἐνυαλίοιο μέλος μυκήσατο κόχλῳ, 
λαὸν ἀολλίζων· βριαροὺς δ' ἐμέθυσσε μαχητάς, 
θερμοτέροις ἐς ἄρηα νοήμασιν ἀνέρας ἕλκων 
Ἰνδῴης ὀλετῆρας ἀβακχεύτοιο γενέθλης. 



Nonnus Epic., Dionysiaca 
Book 17, line 97

τοὺς μὲν ἄναξ Διόνυσος ἐκόσμεεν εἰς μόθον Ἰνδῶν· 
Ἀστράεις δ' ἀκίχητος ἰὼν ἤγγειλεν Ὀρόντῃ 
Ἰνδῶν δοῦλα γένεθλα καὶ ἴαχε πενθάδι φωνῇ· 
 “γαμβρὲ δοριθρασέος μενεδήιε Δηριαδῆος, 
κλῦθι καὶ εἰσαΐων μὴ χώεο· καί σε διδάξω 
νίκην φαρμακόεσσαν ἀθωρήκτου Διονύσου. 



Nonnus Epic., Dionysiaca 
Book 17, line 114

ἀστράπτων σακέεσσιν, ἀκοντοφόρους δὲ δοκεύων 
Λυδὸς ἀνὴρ πολύιδρις ἐμοὺς ἔφριξε μαχητάς· 
ἵστατο δ' ἀπτολέμων Σατύρων πρόμος, οὐ δόρυ χάρμης 
χειρὶ φέρων, οὐ γυμνὸν ἔχων ξίφος, οὐδ' ἐπὶ νευρῇ 
εἰς σκοπὸν ἰθυκέλευθον ὑπηνέμιον βέλος ἕλκων· 
ἀλλὰ κέρας βοὸς εἶχεν, ἐνὶ γλαφυρῇ δὲ κεραίῃ 
φάρμακον ὑγρὸν ἄειρε, καὶ ἀργυρέου ποταμοῖο 
εἰς προχοὰς δολόεσσαν ὅλην κατέχευεν ἐέρσην 
ἰκμάδι φοινίξας γλυκερὸν ῥόον· ἐκ δὲ κυδοιμοῦ 
καύματι διψώοντες ὅσοι πίον αἴθοπες Ἰνδοί, 
ἔμφρονα λύσσαν ἔχοντες ἀνεκρούσαντο χορείην· 
καί σφισι λοίγιος ὕπνος ἐπέχραεν, ἀκλινέες δὲ 
ἄσχετα βακχευθέντες ἐπευνάζοντο βοείαις·   
ἄλλοι δ' ἀστορέεσσι κατεκλίνοντο χαμεύναις 
νωθρὸν ἐπιτρέψαντες ἀκοιμήτῳ δέμας ὕπνῳ, 
Βάκχαις ἀδρανέεσσιν ἑλώρια καὶ Διονύσῳ. 



Nonnus Epic., Dionysiaca 
Book 17, line 132

ἀλλὰ ποτὸν πεφύλαξο, δορυσσόε, μὴ μετὰ νίκην 
κερδαλέην ἀσίδηρον ἀναιμάκτοιο Λυαίου 
ζωγρήσῃ δόλος ἄλλος ἐν ἄρεϊ λείψανον Ἰνδῶν. 



Nonnus Epic., Dionysiaca 
Book 17, line 136

ὄφρα μὲν Ἰνδὸν ὅμιλον ὀρίδρομος ὥπλισεν Ἄρης, 
τόφρα δὲ Βασσαρίδες πολυκαμπέος ὑψόθι Ταύρου 
εἰς μόθον ἠπείγοντο, συνεστρατόωντο δὲ Βάκχοι 
ὁπλοφόροι καὶ Φῆρες ἀτευχέες· οἱ μὲν ἐναύλων   
ῥηξάμενοι κρηπῖδας ἐκούφισαν, οἱ δὲ κολώνης 
ὑψιτενῆ πρηῶνα· καὶ ἀρχομένοιο κυδοιμοῦ 
ἔχραον ἀντιβίοισι· πολυσχιδέες δὲ χαράδραι 
Ἰνδῴοις ἑλικηδὸν ὀιστεύοντο καρήνοις. 



Nonnus Epic., Dionysiaca 
Book 17, line 150

καὶ ποσὶ λεπταλέοισιν ἐπισκαίροντες ἐρίπνῃ 
Πᾶνες ἐθωρήσσοντο μεμηνότες, ὧν ὁ μὲν αὐτῶν 
μάρψας εὐπαλάμῳ βεβιημένον αὐχένα δεσμῷ 
δήιον αἰγείῃσιν ἀνέσχισεν ἀνέρα χηλαῖς, 
σὺν βριαρῷ θώρηκι μέσον κενεῶνα χαράσσων· 
ὃς δὲ τανυπτόρθων κεράων εὐκαμπέσιν αἰχμαῖς 
ὄρθιον ἁρπάξας τετορημένον Ἰνδὸν ἀλήτην 
μεσσοπαγῆ κούφιζεν, ἐς ἠερίας δὲ κελεύθους 
δισσαῖς ὑψιπότητον ἀνηκόντιζε κεραίαις, 
κύμβαχον αὐτοκύλιστον· ἀμαλλοφόροιο δὲ Δηοῦς 
ἄλλος ἑῇ παλάμῃ δονέων καλαμητόμον ἅρπην, 
ὡς στάχυν ὑσμίνης, ὡς δράγματα δηιοτῆτος, 
δυσμενέων ἤμησε γονὰς γαμψώνυχι χαλκῷ, 
τεύχων κῶμον Ἄρηι θαλύσια καὶ Διονύσῳ, 
τέμνων ἐχθρὰ κάρηνα· καὶ ὤρεγε μάρτυρι Βάκχῳ 
καμπύλον ἀνδρομέῃ πεπαλαγμένον ἆορ ἐέρσῃ, 




Nonnus Epic., Dionysiaca 
Book 17, line 168

καὶ θρασὺς Ἰνδῴην στρατιὴν θάρσυνεν Ὀρόντης 
μῦθον ἀπειλητῆρα χέων ὑψήνορι φωνῇ· 
 “δεῦτε, φίλοι, Σατύροισιν ἀναστήσωμεν ἐνυώ· 
ἄρεα μὴ τρομέοιτε φυγοπτολέμου Διονύσου· 
μηδέ τις ὑμείων πιέτω ξανθόχροον ὕδωρ, 
μὴ γλυκερῆς δολόεντα μεμηνότα φάρμακα πηγῆς, 
Ἰνδῶν αἰνομόρων δεδαϊγμένα χειρὶ Λυαίου 
μὴ μετὰ τόσσα κάρηνα καὶ ἡμέας ὕπνος ὀλέσσῃ. 



Nonnus Epic., Dionysiaca 
Book 17, line 186

                                    ἀλλὰ μαχητὴν 
σφιγγόμενον βαρύδεσμον ἀνάλκιδα τοῦτον ἐρύσσω 
θηλυμανῆ Διόνυσον, ὀπάονα Δηριαδῆος·   
οὗτος ὁ θῆλυν ἔχων ἁπαλὸν χρόα, πάντας ἐάσσας 
Ἰνδοὺς τοσσατίους ἑνὶ μάρναο μοῦνον Ὀρόντῃ. 



Nonnus Epic., Dionysiaca 
Book 17, line 190

σὰς προπόλους Ἰνδοῖσι γυναιμανέεσσι συνάψω 
ἑλκομένας ἐπὶ λέκτρα δορικτήτων ὑμεναίων. 



Nonnus Epic., Dionysiaca 
Book 17, line 243

θηγαλέην Βρομίοιο μάτην ἤρασσε κεραίην· 
οὐ γὰρ ἄναξ Διόνυσος ἀδηλήτοιο καρήνου 
ταυροφυῆ τύπον εἶχε † Σεληναίοιο μετώπου 
τεμνόμενον βουπλῆγος ἀλοιητῆρι σιδήρῳ, 
ὡς κερόεις Ἀχελῷος ἀείδεται, οὗ ποτε κόψας 
Ἡρακλέης κέρας εἷλε γαμοστόλος· ἀλλὰ Λυαῖος 
οὐράνιον μίμημα βοώπιδος εἶχε Σελήνης, 
δαιμονίης ἄρρηκτον ἔχων βλάστημα κεραίης, 
ἀντιβίοις ἀτίνακτον· ὁ δὲ θρασὺς ἀντία Βάκχου 
ἠερίῃ βαρύδουπος ὁμοίιος Ἰνδὸς ἀέλλῃ 
δεύτερον ἠκόντιζεν, ἀνεγνάμφθη δέ οἱ αἰχμὴ 
νεβρίδος ἁψαμένη μολίβου τύπον. 



Nonnus Epic., Dionysiaca 
Book 17, line 254

ἵστασο δηριόων, καὶ γνώσεαι, οἷον ἀέξει 
ὄρχαμον ἀλκήεντα γέρων ἐμὸς Ἰνδὸς Ὑδάσπης. 



Nonnus Epic., Dionysiaca 
Book 17, line 274

τύψε κατὰ στέρνου πεφιδημένος· οὐτιδανῷ δὲ 
ἄνθεϊ βοτρυόεντι τυπεὶς ἐσχίζετο θώρηξ· 
οὐδὲ καλυπτομένου χροὸς ἥψατο Βακχιὰς αἰχμή, 
οὐ δέμας ἄκρον ἄμυξε· σιδηρείου δὲ χιτῶνος 
ῥηγνυμένου βαρύδουπος ἐχάζετο γυμνὸς Ὀρόντης· 
Ἠῴην δ' ἐπὶ πέζαν ἑὰς ἐτίταινεν ὀπωπὰς 
ἀντιπόρῳ Φαέθοντι καὶ ὑστατίην φάτο φωνήν· 
 “Ἠέλιε, φλογεροῖο δι' ἅρματος αἰθέρα τέμνων,   
γείτονα † καὶ κυθέην ὑπὲρ αὔλακα φέγγος ἰάλλων 
στῆσον ἐμοὶ σέο δίφρα, καὶ ἔννεπε Δηριαδῆι 
Ἰνδῶν δοῦλα γένεθλα καὶ αὐτοδάικτον Ὀρόντην 
καὶ θύρσους ὀλίγους ῥηξήνορας, εἰπὲ καὶ αὐτοῦ 
νίκην φαρμακόεσσαν ἀπειρομόθου Διονύσου, 
καὶ ῥόον οἰνωθέντα νοοσφαλέος ποταμοῖο· 
εἰπὲ δέ, πῶς ἀκάμαντα σιδηροφόρων στρατὸν Ἰνδῶν 
λεπταλέοις πετάλοισι διασχίζουσι γυναῖκες. 



Nonnus Epic., Dionysiaca 
Book 17, line 278

Ἠῴην δ' ἐπὶ πέζαν ἑὰς ἐτίταινεν ὀπωπὰς 
ἀντιπόρῳ Φαέθοντι καὶ ὑστατίην φάτο φωνήν· 
 “Ἠέλιε, φλογεροῖο δι' ἅρματος αἰθέρα τέμνων,   
γείτονα † καὶ κυθέην ὑπὲρ αὔλακα φέγγος ἰάλλων 
στῆσον ἐμοὶ σέο δίφρα, καὶ ἔννεπε Δηριαδῆι 
Ἰνδῶν δοῦλα γένεθλα καὶ αὐτοδάικτον Ὀρόντην 
καὶ θύρσους ὀλίγους ῥηξήνορας, εἰπὲ καὶ αὐτοῦ 
νίκην φαρμακόεσσαν ἀπειρομόθου Διονύσου, 
καὶ ῥόον οἰνωθέντα νοοσφαλέος ποταμοῖο· 
εἰπὲ δέ, πῶς ἀκάμαντα σιδηροφόρων στρατὸν Ἰνδῶν 
λεπταλέοις πετάλοισι διασχίζουσι γυναῖκες. 



Nonnus Epic., Dionysiaca 
Book 17, line 284

οὐ πιθόμην Βρομίῳ θηλύφρονι· μάρτυρας ἕλκω 
ἠέλιον καὶ γαῖαν ἀτέρμονα καὶ θεὸν Ἰνδῶν, 
ἁγνὸν ὕδωρ. 



Nonnus Epic., Dionysiaca 
Book 17, line 286

                σὺ δὲ χαῖρε, καὶ ἵλαος ἔσσο κυδοιμῷ 
Ἰνδῶν μαρναμένων, καὶ ὀλωλότα θάψον Ὀρόντην. 



Nonnus Epic., Dionysiaca 
Book 17, line 314

τὸν μὲν ἐταρχύσαντο καὶ ἔστενον αἴλινα Νύμφαι, 
Νύμφαι Ἁμαδρυάδες, χρυσέης παρὰ πυθμένα Δάφνης 
ἀμφὶ ῥοὰς ποταμοῖο, καὶ ἔγραφον ὑψόθι τύμβου· 
’Βάκχον ἀτιμήσας στρατιῆς πρόμος ἐνθάδε κεῖται, 
αὐτοφόνῳ παλάμῃ δεδαϊγμένος Ἰνδὸς Ὀρόντης. 



Nonnus Epic., Dionysiaca 
Book 17, line 317

οὐδὲ μόθου τέλος ἦεν ἀτερπέος· ἡμιτελὴς γὰρ 
ἦεν ἀγὼν καὶ δῆρις ἀνήνυτος· ὑψιφανὴς δὲ 
Ἰνδὸς ἄρης ἀλάλαζε· παλιννόστῳ δὲ κυδοιμῷ 
Λυδὸν ἐρευγομένη μανιώδεος ὄγκον ἀπειλῆς 
Βακχιὰς εἰς μόθον ἄλλον ἐκώμασε θυιὰς Ἐνυώ, 
δήιον ἀνδροφόνοισιν ἀκοντίζουσα κορύμβοις, 
ἄρεϊ βακχευθεῖσα· φιλοπτόρθου δὲ Λυαίου 
δυσμενέες δρυόεντι κατεκτείνοντο βελέμνῳ 
φοίνιον ἕλκος ἔχοντες· ἀθωρήκτοιο δὲ Βάκχης 
ἔγχεϊ βοτρυόεντι δαϊζομένοιο σιδήρου 
Ἰνδοὶ χαλκοχίτωνες ἐθάμβεον ὀξέι κισσῷ 
στήθεα γυμνωθέντα νεούτατα· ῥηίτεροι γὰρ 




Nonnus Epic., Dionysiaca 
Book 17, line 331

ἄλλων δ' ἄλλος ἔην φόνος ἄσπετος, ὧν ὑπὸ λύθρῳ 
σχιζόμενοι πετάλοισιν ἐφοινίσσοντο χιτῶνες 
μαρναμένων, ὅθι Ταῦρος· ἐκυκλώσαντο δὲ Βάκχαι 
ἀκλινέες στεφανηδὸν ὁμοζυγέων στίχας Ἰνδῶν. 



Nonnus Epic., Dionysiaca 
Book 17, line 338

καὶ θρασὺς αὐλὸς ἔμελπε φόνου μέλος· ἐν δὲ κυδοιμοῖς 
Βάκχοι μὲν θεράποντες ἑλινοφόρου Διονύσου 
τυπτόμενοι πελέκεσσι καὶ ἀμφιτόμοισι μαχαίραις 
πάντες ἔσαν πυργηδὸν ἀπήμονες· αἰνόμοροι δὲ   
δυσμενέες λεπτοῖσι κατεκτείνοντο πετήλοις· 
ἑξείης δ' ἐπέπηκτο τανυπτόρθοις ἐνὶ δένδροις 
Ἰνδῶν πυκνὰ βέλεμνα, καὶ ἔγχεϊ νύσσετο πεύκη 
τηλεπόρῳ, βέβλητο πίτυς, τοξεύετο δάφνη, 
Φοίβου δένδρον ἐοῦσα, καὶ αἰδομένοις ἐνὶ φύλλοις 
πεμπομένων ἐκάλυπτε τανυπτερύγων νέφος ἰῶν, 
μή μιν ἴδῃ βελέεσσιν ὀιστευθεῖσαν Ἀπόλλων. 



Nonnus Epic., Dionysiaca 
Book 17, line 347

καὶ γυμνῇ παλάμῃ σακέων δίχα, νόσφι σιδήρου, 
Βάκχη ῥόπτρα τίνασσε, καὶ ἤριπεν ἀσπιδιώτης· 
τύμπανα δ' ἐσμαράγησε, καὶ ὠρχήσαντο μαχηταί· 
κύμβαλα δ' ἐκροτάλιζε, καὶ αὐχένα κάμψε Λυαίῳ 
Ἰνδὸς ἀνὴρ ἱκέτης. 



Nonnus Epic., Dionysiaca 
Book 17, line 354

ἀθρήσας δὲ τάλαντα μάχης ἑτεραλκέι ῥιπῇ 
νίκην Ἰνδοφόνοιο προθεσπίζοντα Λυαίου 
Ἀστράεις ἀκίχητος ἐχάζετο, πότμον ἀλύξας, 
ἐγχείην τανύφυλλον ὑποπτήσσων Διονύσου. 



Nonnus Epic., Dionysiaca 
Book 17, line 376

                                           μαρναμένων δὲ 
ἤδη βαρβαρόφωνος † ἐπαύσατο Ἰνδὶς ἐνυώ. 



Nonnus Epic., Dionysiaca 
Book 17, line 380

καὶ πολέας ζώγρησαν ἀπὸ πτολέμοιο μαχητὰς 
Βασσαρίδες· πολλοὶ δὲ λελοιπότες οὔρεα Ταύρου   
δυσμενέες νόστησαν ἐς Ἰνδῷον κλίμα γαίης 
ἐλπίσιν ἀπρήκτοισιν ἐς οἰκία Δηριαδῆος, 
ἀμφιλαφεῖς ἐλατῆρες ἀμετροβίων ἐλεφάντων. 



Nonnus Epic., Dionysiaca 
Book 17, line 385

καὶ Βλέμυς οὐλοκάρηνος, Ἐρυθραίων πρόμος Ἰνδῶν, 
ἱκεσίης κούφιζεν ἀναίμονα θαλλὸν ἐλαίης, 
Ἰνδοφόνῳ γόνυ δοῦλον ὑποκλίνων Διονύσῳ. 



Nonnus Epic., Dionysiaca 
Book 17, line 390

καὶ θεός, ἀθρήσας κυρτούμενον ἀνέρα γαίῃ, 
χειρὶ λαβὼν ὤρθωσε, πολυγλώσσῳ δ' ἅμα λαῷ 
κυανέων πόμπευεν Ἐρυθραίων ἑκὰς Ἰνδῶν, 
κοιρανίην στυγέοντα καὶ ἤθεα Δηριαδῆος, 
Ἀρραβίης ἐπὶ πέζαν, ὅπῃ παρὰ γείτονι πόντῳ 
ὄλβιον οὖδας ἔναιε καὶ οὔνομα δῶκε πολίταις· 
καὶ Βλέμυς ὠκὺς ἵκανεν ἐς ἑπταπόρου στόμα Νείλου, 
ἐσσόμενος σκηπτοῦχος ὁμόχροος Αἰθιοπήων· 
καί μιν ἀειθερέος Μερόης ὑπεδέξατο πυθμήν, 
ὀψιγόνοις Βλεμύεσσι προώνυμον ἡγεμονῆα. 



Nonnus Epic., Dionysiaca 
Book 18, line 4

Ἤδη δὲ πτερόεσσα πολύστομος ἵπτατο Φήμη 
Ἀσσυρίης στίχα πᾶσαν ὑποτροχόωσα πολήων, 
οὔνομα κηρύσσουσα κορυμβοφόρου Διονύσου 
καὶ θρασὺν Ἰνδὸν ἄρηα καὶ ἀγλαόβοτρυν ὀπώρην. 



Nonnus Epic., Dionysiaca 
Book 18, line 170

ἀλλ' ὅτε δὴ ῥοδέοις ἀμαρύγμασιν ἄγγελος Ἠοῦς 
ἀκροφαὴς ἐχάραξε λιπόσκιον ὄρθρος ὀμίχλην, 
εὐχαίτης τότε Βάκχος ἑώιος ἄνθορεν εὐνῆς, 
ἐλπίδι νικαίῃ δεδονημένος· ἐννύχιος γὰρ 
Ἰνδῴην ἐδάιζε γονὴν κισσώδεϊ θύρσῳ,   
ὑπναλέης μεθέπων ἀπατήλιον εἰκόνα χάρμης. 



Nonnus Epic., Dionysiaca 
Book 18, line 176

καὶ κτύπον εἰσαΐων Σατύρων καὶ δοῦπον ἀκόντων 
φλοῖσβον ὀνειρείης ἀπεσείσατο δηιοτῆτος, 
ὕπνον ἀποσκεδάσας πολεμήιον· εἶχε δὲ θυμῷ 
μαντιπόλου φόβον ἄλλον ἀπειλητῆρος ὀνείρου· 
μιμηλῆς γὰρ ὄπωπε μάχης ἴνδαλμα Λυκούργου 
ἐσσομένων προκέλευθον, ὅτι θρασὺς ἔνδοθι λόχμης 
δύσμαχος ἐκ σκοπέλοιο λέων λυσσώδεϊ λαιμῷ 
Βάκχον ἔτι σκαίροντα καὶ οὐ ψαύοντα σιδήρου 
εἰς φόβον ἐπτοίησε, καὶ ἤλασεν ἄχρι θαλάσσης 
κρυπτόμενον πελάγεσσι, πεφυζότα θηρὸς ἀπειλήν· 
καὶ φόβον ἄλλον ὄπωπε, λέων θρασὺς ὅττι γυναῖκας 
θυρσοφόρους ἐδίωκε, κεχηνότος ἀνθερεῶνος, 
αἱμάσσων ὀνύχεσσι, χαρασσομένων δὲ γυναικῶν 
μύστιδος ἐκ παλάμης ἐκυλίνδετο θύσθλα κονίῃ, 




Nonnus Epic., Dionysiaca 
Book 18, line 197

τοῖον ὄναρ Διόνυσος ἐσέδρακεν· ἐκ λεχέων δὲ 
ὀρθὸς ἐὼν ἔνδυνε φόνῳ πεπαλαγμένον Ἰνδῶν 
χάλκεον ἀστερόεντα κατὰ στέρνοιο χιτῶνα, 
καὶ σκολιῷ μίτρωσε κόμην ὀφιώδεϊ δεσμῷ, 
καὶ πόδας ἐσφήκωσεν ἐρευθιόωντι κοθόρνῳ, 
χειρὶ δὲ θύρσον ἄειρε, φιλάνθεμον ἔγχος ἐνυοῦς· 
καὶ Σάτυρον κίκλησκεν ὀπάονα. 



Nonnus Epic., Dionysiaca 
Book 18, line 221

καὶ Βρομίῳ πολύδωρος ἄναξ ἐφθέγξατο φωνήν· 
 “μάρναό μοι, Διόνυσε, καὶ ἄξια ῥέζε τοκῆος· 
δεῖξον, ὅτι Κρονίδαο φέρεις γένος· ἀρτιθαλὴς γὰρ 
Γηγενέας Τιτῆνας ἀπεστυφέλιξεν Ὀλύμπου 
σὸς γενέτης ἔτι κοῦρος· ἐπείγεο καὶ σὺ κυδοιμῷ 
Γηγενέων ὑπέροπλον ἀιστῶσαι γένος Ἰνδῶν. 



Nonnus Epic., Dionysiaca 
Book 18, line 235

ὠμοβόρους δὲ λέοντας ἐπὶ κλόνον Ἰνδὸν ἱμάσσων, 
μὴ τρομέοις ἐλέφαντας, ἐπεὶ τεὸς ὑψιμέδων Ζεὺς 
Κάμπην ὑψικάρηνον ἀπηλοίησε κεραυνῷ, 
ἧς σκολιὸν πολύμορφον ὅλον δέμας· ἀλλοφυεῖς γὰρ 
λοξὴν αὐτοέλικτον ἀνερρίπιζον ἐνυὼ 
χίλιοι ἑρπηστῆρες ἐχιδναίων ἀπὸ ταρσῶν 
ἰὸν ἐρευγομένων δολιχόσκιον· ἀμφὶ δὲ δειρὴν 
ἤνθεε πεντήκοντα καρήατα ποικίλα θηρῶν· 
καὶ τὰ μὲν ἐβρυχᾶτο λεοντείοισι καρήνοις 




Nonnus Epic., Dionysiaca 
Book 18, line 267

γίνεο καὶ σὺ τοκῆι πανείκελος, ὄφρα καὶ αὐτὸν 
Γηγενέων ὀλετῆρα μετὰ Κρονίδην σὲ καλέσσω, 
δήιον ἀμήσαντα χαμαιγενέων στάχυν Ἰνδῶν. 



Nonnus Epic., Dionysiaca 
Book 18, line 271

σοὶ μόθος οὗτος ἔοικεν ὁμοίιος· ἀρχέγονον γὰρ 
σὸς γενέτης Κρονίοιο προασπιστῆρα κυδοιμοῦ 
ἠλιβάτοις μελέεσσι κεκασμένον υἱὸν ἀρούρης   
Ἰνδὸν ἀνεπρήνιξεν, ὅθεν γένος ἔλλαχον Ἰνδοί· 
Ἰνδῷ σὸς γενέτης, σὺ δὲ μάρναο Δηριαδῆι. 



Nonnus Epic., Dionysiaca 
Book 18, line 299

υἱὸν ἐγὼ Διὸς ἄλλον ἐμῷ ξείνισσα μελάθρῳ·   
χθιζὰ γὰρ εἰς ἐμὸν οἶκον ἐύπτερος ἤλυθε Περσεὺς 
γείτονα Κωρυκίοιο διαυγέα Κύδνον ἐάσσας, 
ὡς σύ, φίλος, καὶ ἔφασκεν ἐπώνυμον ὠκέι ταρσῷ 
ἀνδράσι πὰρ Κιλίκεσσι νεόκτιτον ἄστυ χαράξαι· 
ἀλλ' ὁ μὲν ἠέρταζεν ἀθηήτοιο Μεδούσης 
Γοργόνος ἄκρα κάρηνα, σὺ δ' οἴνοπα καρπὸν ἀείρεις, 
ἄγγελον εὐφροσύνης, βροτέης ἐπίληθον ἀνίης· 
Περσεὺς κῆτος ἔπεφνεν Ἐρυθραίῳ παρὰ πόντῳ, 
καὶ σὺ κατεπρήνιξας Ἐρυθραίων γένος Ἰνδῶν. 



Nonnus Epic., Dionysiaca 
Book 18, line 300

κτεῖνε δὲ Δηριάδην, ὡς ἔκτανες Ἰνδὸν Ὀρόντην, 
κήτεος εἰναλίοιο κακώτερον· ἀχνυμένην μὲν 
Περσεὺς Ἀνδρομέδην, σὺ δὲ ῥύεο μείζονι νίκῃ 
πικρὰ βιαζομένην ἀδίκων ὑπὸ νεύμασιν Ἰνδῶν 
Παρθένον ἀστερόεσσαν, ὅπως ἕνα κῶμον ἀνάψω 
Γοργοφόνῳ Περσῆι καὶ Ἰνδοφόνῳ Διονύσῳ. 



Nonnus Epic., Dionysiaca 
Book 18, line 312

ὣς εἰπὼν παλίνορσος ἑῷ νόστησε μελάθρῳ 
ἁβρὸς ἄναξ, Βρομίου ξεινηδόκος· εἰσαΐων δὲ 
φθεγγομένου βασιλῆος ἐτέρπετο κέντορι μύθῳ 
θυρσομανὴς Διόνυσος, ἐβακχεύθη δὲ κυδοιμῷ 
οὔασι θελγομένοισι μόθον πατρῷον ἀκούων· 
καὶ Κρονίδην νείκεσσε, καὶ ἤθελε μείζονα νίκην   
ἐσσομένην τριτάτην, διδύμην μετὰ φύλοπιν Ἰνδῶν, 
ζῆλον ἔχων Κρονίδαο. 



Nonnus Epic., Dionysiaca 
Book 18, line 366

ἐλπίδα δ' ἡμετέρην φθόνος ἥρπασεν· ὠισάμην γὰρ 
Ἰνδῴην μετὰ δῆριν ἅμα Σταφύλῳ βασιλῆι 
χερσὶν ἀερτάζειν θαλαμηπόλον ἑσπέριον πῦρ, 
Βότρυος ἀγχεμάχοιο τελειομένων ὑμεναίων. 



Nonnus Epic., Dionysiaca 
Book 19, line 37

Βότρυν ἔχεις θεράποντα· διδασκέσθω δὲ χορείας 
καὶ τελετὰς καὶ θύσθλα καί, ἢν ἐθέλῃς, μόθον Ἰνδῶν·   
καί μιν ἴδω γελόωντα φιλακρήτῳ παρὰ ληνῷ 
ποσσὶ περιθλίβοντα τεῆς ὠδῖνας ὀπώρης. 



Nonnus Epic., Dionysiaca 
Book 19, line 146

ἴδμονας ὀρχηθμοῖο καλέσσατο μάρτυρι φωνῇ· 
 “ὅς τις ἀεθλεύσει κυκλούμενος ἴδμονι ταρσῷ 
νικήσας τροχαλοῖο ποδὸς κρίσιν, οὗτος ἑλέσθω   
καὶ χρύσεον κρητῆρα καὶ ἡδυπότου χύσιν οἴνου· 
ὃς δὲ πέσῃ σφαλεροῖο ποδὸς δεδονημένος ὁλκῷ, 
ἥσσονα δ' ὀρχήσοιτο, καὶ ἥσσονα δῶρα δεχέσθω· 
οὐ γὰρ ἐγὼ πάντεσσιν ὁμοίιος· ἀθλοφόρῳ δὲ 
ἀνέρι νικήσαντι χοροίτυπον ἁβρὸν ἀγῶνα 
οὐ τρίποδα στίλβοντα καὶ οὐ ταχὺν ἵππον ὀπάσσω, 
οὐ δόρυ καὶ θώρηκα φόνῳ πεπαλαγμένον Ἰνδῶν, . 



Nonnus Epic., Dionysiaca 
Book 20, line 92

ἔστι καὶ εἰλαπίνη μετὰ φύλοπιν, ἔστι χορεύειν 
Ἰνδῴην μετὰ δῆριν ἔσω Σταφύλοιο μελάθρου· 
πηκτίδες † οὐ ψαύουσιν ἐνυαλίην μετὰ νίκην· 
νόσφι πόνων οὐκ ἔστιν ἀνέμβατον αἰθέρα ναίειν· 
οὐ πέλε ῥηιδίη μακάρων ὁδός· ἐξ ἀρετῆς δὲ 
ἀτραπὸς Οὐλύμποιο θεόσσυτος εἰς πόλον ἕλκει. 



Nonnus Epic., Dionysiaca 
Book 20, line 110

                  Σάτυροι δὲ δαφοινήεσσαν ἀπήνην 
πορδαλίων ἔζευξαν ἐπειγομένῳ Διονύσῳ· 
Σιληνοὶ δ' ἀλάλαζον· ἐμυκήσαντο δὲ Βάκχαι 
θυρσοφόροι· στρατιαὶ δὲ συνήλυδες εἰς μόθον Ἰνδῶν 
στοιχάδες ἐρρώοντο· καὶ ἔβρεμεν αὐλὸς Ἐνυοῦς· 
κεκριμένας δὲ φάλαγγας ἐκόσμεον ἡγεμονῆες. 



Nonnus Epic., Dionysiaca 
Book 20, line 283

ἔμπλεον ἡδυπότοιο καὶ ἠθάδα ῥάβδον ἀείρων, 
εὔια δῶρα τίταινε φιλοσταφύλῳ Λυκοόργῳ·   
ἄρτι δέμας κόσμησον ἀναιμάκτῳ σέο πέπλῳ, 
ἄρτι μέλος πλέξωμεν ἀθωρήκτοιο χορείης, 
καὶ στρατὸς ἠρεμέων μενέτω παρὰ δάσκιον ὕλην, 
μὴ μόθον ἐντύνειε γαληναίῳ βασιλῆι· 
ἀλλά, βαλὼν πλοκάμοισι φίλον στέφος, ἔρχεο χαίρων 
εἰς δόμον ἀκλήιστον ἑτοιμοτάτου Λυκοόργου, 
ἔρχεο κωμάζων ἅτε νυμφίος· Ἰνδοφόνους δὲ 
θύρσους σεῖο φύλαξον ἀπειθέι Δηριαδῆι. 



Nonnus Epic., Dionysiaca 
Book 21, line 201

ὄφρα μὲν ἄμφεπε Βάκχος ἁλίτροφα δεῖπνα τραπέζης, 
τόφρα δὲ Καυκασίοιο δι' οὔρεος εἰς πόλιν Ἰνδῶν 
οἰνοφύτου Βρομίοιο ποδήνεμος ἵκετο κῆρυξ 
ταυροφυής, νόθον εἶδος ἔχων κεραελκέι μορφῇ, 
ἀντίτυπον μίμημα Σεληναίῃσι κεραίαις, 
αἰγὸς ὀρεσσινόμοιο περὶ χροῒ δέρμα συνάψας, 
αὐχενίῃ κληῖδι καθειμένον ἐξ ἑνὸς ὤμου, 
δεξιτεροῦ πλευροῖο κατήορον εἰς πτύχα μηροῦ, 
ἀμφοτέρης ἑκάτερθε παρηίδος οὔατα σείων, 
ὡς ὄνος οὐατόεις, λάσιος δέμας· ἐκ μεσάτης δὲ 
ἰξύος αὐτοέλικτος ἐσύρετο σύγγονος οὐρή. 



Nonnus Epic., Dionysiaca 
Book 21, line 211

ἀμφὶ δέ μιν γελόωντες ἐπέρρεον αἴθοπες Ἰνδοί, 
εἰσόκεν ἐγγὺς ἵκανεν, ὅπῃ διδυμόζυγι δίφρῳ 
ἕζετο Δηριάδης περιμήκετος, ὄρχαμος ἀνδρῶν, 
ἠλιβάτων στατὸν ἴχνος ἀναστέλλων ἐλεφάντων. 



Nonnus Epic., Dionysiaca 
Book 21, line 225

καὶ πυρόεις σέο Βάκχος ἀκούεται, ὅττι τεκούσης 
ἐκ λαγόνων ἀνέτελλε Διοβλήτοιο Θυώνης· 
καὶ πυρός ἐστιν ὕδωρ πολὺ φέρτερον· ἢν ἐθελήσῃ, 
χεύματι παφλάζοντι πατὴρ ἐμός, Ἰνδὸς Ὑδάσπης, 
Ζηνὸς ἀποσβέσσειε πυρίπνοον ἄσθμα κεραυνοῦ. 



Nonnus Epic., Dionysiaca 
Book 21, line 234

καὶ βλοσυρῷ βασιλῆι τεθηπότα χείλεα λύσας 
ἀγγελίην Βρομίοιο ταχύδρομος ἔννεπε κῆρυξ· 
 “Δηριάδη, σκηπτοῦχε, θεὸς Διόνυσος ἀνώγει 
Ἰνδοὺς δεχνυμένους λαθικηδέος οἶνον ὀπώρης 
σπένδειν ἀθανάτοισι, δίχα πτολέμων, δίχα μόχθων· 
εἰ δέ κε μὴ δέξαιντο, κορύσσεται, εἰσόκε θύρσοις 
Βασσαρίδων γόνυ δοῦλον ὑποκλίνειεν Ὑδάσπης. 



Nonnus Epic., Dionysiaca 
Book 21, line 269

ταῦτα μολὼν ἀγόρευε φυγοπτολέμῳ Διονύσῳ· 
ἔρρε φυγὼν ἀκίχητος, ἕως ἔτι τόξον ἐρύκω, 
ἔρρε φυγὼν ἐμὸν ἔγχος· ἐς ὑσμίνην δὲ κορύσσας 
ἡμιτελεῖς σέο θῆρας ἀθωρήκτους τε γυναῖκας 
Δηριάδῃ πολέμιζε, καὶ Ἰνδῴην μετὰ νίκην 
σύνδρομον αὐερύσω σε δορικτήτῳ Διονύσῳ. 



Nonnus Epic., Dionysiaca 
Book 21, line 309

καὶ καλέσας Ῥαδαμᾶνας ἀλήμονας, οὕς ποτε γαίης 
Κρηταίης ἀέκοντας ἀπὸ χθονὸς ἤλασε Μίνως 
Ἀρραβίης ἐπὶ πέζαν, ἐπέφραδε νεύματι Ῥείης 
πῆξαι νήια δοῦρα θαλάσσιον εἰς μόθον Ἰνδῶν. 



Nonnus Epic., Dionysiaca 
Book 21, line 317

ὄφρα μὲν εὐθύρσοιο μάχης ἠκούετο φωνὴ 
καὶ στρατὸς ἀγχικέλευθος ὀρεσσινόμου Διονύσου,   
τόφρα δὲ Δηριάδης πυκινὸν λόχον ἵδρυσεν Ἰνδῶν, 
γαῖαν ἐς ἀντιπέραιαν ἑὸν στρατὸν ἄζυγα πέμπων, 
πᾶσαν ἐπιτρέψας δολομήχανον ἐλπίδα χάρμης 
Ἄρεϊ χαλκοχίτωνι· καὶ ἔπλεεν ὑψόθι νηῶν 
λαὸς ἐρετμώσας πεπερημένον Ἰνδὸν Ὑδάσπην. 



Nonnus Epic., Dionysiaca 
Book 21, line 322

καὶ στρατιαῖς διδύμῃσι μερίζετο φύλοπις Ἰνδῶν 
ἀμφοτέρην παρὰ πέζαν ἀκοντοφόρου ποταμοῖο· 
Θουρεὺς μὲν Ζεφύροιο παρὰ σφυρά, Δηριάδης δὲ 
ἀντιπόρου σχεδὸν ἦλθε παρὰ πτερὸν αἴθοπος Εὔρου. 



Nonnus Epic., Dionysiaca 
Book 22, line 3

Ἀλλ' ὅτε δὴ πόρον ἷξον ἐυκροκάλου ποταμοῖο 
Βάκχου πεζὸς ὅμιλος, ὅπῃ βαθυδίνεϊ κόλπῳ 
πλωτὸν ὕδωρ, ἅτε Νεῖλος, ἐρεύγεται Ἰνδὸς Ὑδάσπης, 
δὴ τότε Βασσαρίδων ἐμελίζετο θῆλυς ἀοιδὴ 
Νυκτελίῳ Φρύγα κῶμον ἀνακρούουσα Λυαίῳ, 
καὶ λασίων Σατύρων χορὸς ἔβρεμε μύστιδι φωνῇ· 
γαῖα δὲ πᾶσα γέλασσεν, ἐμυκήσαντο δὲ πέτραι, 
Νηιάδες δ' ὀλόλυξαν, ὑπὲρ ποταμοῖο δὲ Νύμφαι 
σιγαλέοις ἑλικηδὸν † ἐμυκήσαντο ῥεέθροις 
καὶ Σικελῆς ἐλίγαινον ὁμόζυγα ῥυθμὸν ἀοιδῆς, 
οἷον ἀνεκρούοντο μελιγλώσσων ἀπὸ λαιμῶν 
ὑμνοπόλοι Σειρῆνες· ὅλη δ' ἐλελίζετο λόχμη, 




Nonnus Epic., Dionysiaca 
Book 22, line 36

καὶ κύνας ὀρχηστῆρας ἐπηχύναντο λαγωοί· 
μηκεδανοὶ δὲ δράκοντες ἐβακχεύοντο χορείῃ 
ἴχνια λιχμώοντες ἐχιδνοκόμου Διονύσου, 
αὐχένα δοχμώσαντες, ἀνήρυγε δ' ἄλλος ἐπ' ἄλλῳ 
μειλίχιον σύριγμα γεγηθότος ἀνθερεῶνος· 
τερπομένου δὲ δράκοντος ἔην τότε ῥυθμὸς ἐχέφρων,   
καὶ δολιχῆς ἐλέλικτο περίπλοκος ὁλκὸς ἀκάνθης 
ποσσὶν ἀδειμάντοισι περισκαίρων Διονύσου· 
Ἰνδῴην δ' ἑλικηδὸν ἐπισκαίροντες ἐρίπνην 
τίγριδες ἑψιόωντο· πολὺς δέ τις ἔνδοθι λόχμης 
ἐσμὸς ἀνεσκίρτησεν ὀρεσσινόμων ἐλεφάντων. 



Nonnus Epic., Dionysiaca 
Book 22, line 47

καὶ δονέων πλοκαμῖδα παρήορον ἀνθερεῶνος 
σύννομος ἀντεχόρευε λέων βητάρμονι κάπρῳ· 
ἀνδρομέης δ' ὄρνιθες ἀνέκλαγον εἰκόνα μολπῆς 
μιμηλὴν ἀτέλεστον ὑποκλέπτοντες ἰωήν, 
νίκην Ἰνδοφόνοιο προθεσπίζοντες ἀγῶνος, . 



Nonnus Epic., Dionysiaca 
Book 22, line 63

καί τις ἐσαθρήσας ἑτερότροπα θαύματα Βάκχου, 
ὄμμα βαλὼν πυκινοῖο δι' ἀκροτάτοιο κορύμβου, 
φύλλα περιστείλας, θηήτορα κύκλον ὀπωπῆς 
τόσσον ἰδεῖν μεθέηκεν, ὅσον περιδέρκεται ἀνὴρ 
ὄμμασι ποιητοῖσι διοπτεύων τρυφαλείης, 
ἢ ὁπότε τραγικοῖο χοροῦ δεδαημένος ἀνήρ, 
φρικτὸν ἔχων μύκημα τανυφθόγγων ἀπὸ λαιμῶν, 
ἐνδόμυχον τυκτοῖο δι' ὄμματος ὄμμα τιταίνει, 
ψευδαλέον βροτέοιο φέρων ἴνδαλμα προσώπου· 
ὣς ὅ γε θαύματα πάντα λαθὼν ὑπὸ δάσκιον ὕλην 
ἀπροϊδὴς ἐδόκευεν ὑποκλέπτοντι προσώπῳ· 
ἀντιβίοις δ' ἤγγειλε· φόβῳ δ' ἐλελίζετο Θουρεὺς 
μεμφόμενος Μορρῆι καὶ ἄφρονι Δηριαδῆι. 



Nonnus Epic., Dionysiaca 
Book 22, line 68

ἔτρεμε δ' Ἰνδὸς ὅμιλος, ἀφειδήσας δὲ κυδοιμοῦ 
χάλκεα ταρβαλέων ἀπεσείσατο τεύχεα χειρῶν, 
δένδρεα παπταίνων δεδονημένα θυιάδι ῥιπῇ. 



Nonnus Epic., Dionysiaca 
Book 22, line 71

καί νύ κεν Ἰνδὸς ὅμιλος ἑλὼν ἀπὸ γείτονος ὄχθης 
μάρτυρον ἱκεσίης γλαυκόχροα θαλλὸν ἐλαίης 
αὐχένα δοῦλον ἔκαμψεν ἀδηρίτῳ Διονύσῳ·   
ἀλλὰ μεταλλάξασα δέμας πολυμήχανος Ἥρη 
δυσμενέας θάρσυνε καὶ ἤπαφεν ὄρχαμον Ἰνδῶν, 
Θεσσαλίδων μάγον ὕμνον ἐφαψαμένη Διονύσῳ 
καὶ Κίρκης κυκεῶνα, θεοκλήτοις ἐπαοιδαῖς 
οἷά τε φαρμακτῆρος ἀφαρμάκτου ποταμοῖο. 



Nonnus Epic., Dionysiaca 
Book 22, line 83

καί νύ κεν ἀφράστοιο διαθρῴσκοντες ἐναύλου 
δαινυμέναις στρατιῇσιν ἐπέχραον αἴθοπες Ἰνδοί· 
ἀλλά τις ἠνεμόεντος ὑπερκύψασα κορύμβου 
ἐκ λασίου κενεῶνος Ἁμαδρυὰς ἄνθορε Νύμφη· 
χειρὶ δὲ θύρσον ἔχουσα φυὴν ἰνδάλλετο Βάκχῃ, 
μιμηλὴν δρυόεντι πυκαζομένη τρίχα κισσῷ· 
δυσμενέων δ' ἐνέπουσα δόλον σημάντορι σιγῇ 
οὔασι βοτρυόεντος ἐπεψιθύριζε Λυαίου· 
 “ἀμπελόεις Διόνυσε, φυτηκόμε κοίρανε καρπῶν, 
σὸν φυτὸν Ἀδρυάδεσσι χάριν καὶ κάλλος ὀπάσσει· 
Βασσαρὶς οὐ γενόμην, οὐ σύνδρομός εἰμι Λυαίου, 




Nonnus Epic., Dionysiaca 
Book 22, line 99

   ἀμπελόεις Διόνυσε, φυτηκόμε κοίρανε καρπῶν, 
σὸν φυτὸν Ἀδρυάδεσσι χάριν καὶ κάλλος ὀπάσσει· 
Βασσαρὶς οὐ γενόμην, οὐ σύνδρομός εἰμι Λυαίου, 
μοῦνον ἐμῇ παλάμῃ ψευδήμονα θύρσον ἀείρω· 
οὐ πέλον ἐκ Φρυγίης, σέο πατρίδος, οὐ χθόνα Λυδῶν   
ναιετάω παρὰ χεῦμα ῥυηφενέος ποταμοῖο· 
εἰμὶ δὲ καλλιπέτηλος Ἁμαδρυάς, ἧχι μαχηταὶ 
δυσμενέες λοχόωσιν, ἀφειδήσασα δὲ πάτρης 
ῥύσομαι ἐκ θανάτοιο τεὸν στρατόν· ὑμετέροις γὰρ 
πιστὰ φέρω Σατύροισι, καὶ Ἰνδῴη περ ἐοῦσα, 
ἀντὶ δὲ Δηριαδῆος ὁμοφρονέω Διονύσῳ· 
σοὶ γὰρ ὀφειλομένην ὀπάσω χάριν, ὅττι ῥεέθρων 
ὑγροτόκους ὠδῖνας, ὅτι δρύας αἰὲν ἀέξει 
ὀμβρηρῇ ῥαθάμιγγι πατὴρ τεὸς ὑέτιος Ζεύς. 



Nonnus Epic., Dionysiaca 
Book 22, line 107

ἀλλά, φίλος, μὴ σπεῦδε ῥόον ποταμοῖο περῆσαι, 
μή σοι ἐπιβρίσωσιν ἐν ὕδασι γείτονες Ἰνδοί· 
εἰς δρύας ὄμμα τίταινε καὶ εὐπετάλῳ παρὰ λόχμῃ 
ἀπροϊδῆ σκοπίαζε καλυπτομένων λόχον ἀνδρῶν. 



Nonnus Epic., Dionysiaca 
Book 22, line 113

σιγὴ ἐφ' ἡμείων, μὴ δήιος ἐγγὺς ἀκούσῃ, 
μὴ κρυφίοις Ἰνδοῖσιν ἀπαγγείλειεν Ὑδάσπης. 



Nonnus Epic., Dionysiaca 
Book 22, line 122

                                  αὐτὰρ ὁ σιγῇ 
μίσγετο Βασσαρίδεσσιν, Ἁμαδρυάδος δὲ θεαίνης 
εἶπεν ἑοῖς προμάχοισιν ἐς οὔατα μῦθον ἑκάστου 
νεύμασι δενδίλλων, νοερῇ δ' ἐκέλευε σιωπῇ 
τεύχεσι θωρηχθέντας ἀνὰ δρύας εἰλαπινάζειν, 
καὶ κρυφίων ἀγόρευε δολορραφέων δόλον Ἰνδῶν, 
μή σφιν ἐπιβρίσωσιν ἀθωρήκτοισι μαχηταί, 
εἰσέτι δαινυμένοισι κατὰ στρατόν· οἱ δὲ Λυαίῳ 
κεκλομένῳ πείθοντο, καὶ εἰς μόθον ἦσαν ἑτοῖμοι 
σιγαλέον παρὰ δεῖπνον ἀκοντοφόροιο τραπέζης. 



Nonnus Epic., Dionysiaca 
Book 22, line 133

Ζεὺς δὲ πατὴρ δολόεντα μετατρέψας νόον Ἰνδῶν 
ἑσπερίην ἀνέκοψε μάχην μυκήτορι βόμβῳ, 
ὄμβρου παννυχίοιο χέων ἀπερείσιον ἠχώ. 



Nonnus Epic., Dionysiaca 
Book 22, line 140

ἀλλ' ὅτε χιονόπεζα χαραξαμένη ζόφον Ἠὼς 
ὄρθρον ἀμεργομένῃ δροσερῇ πορφύρετο πέτρῃ, 
ἄκρον ὑπερκύψαντες ἐγερσιμόθου σκέπας ὕλης 
δυσμενέες προὔτυψαν ἀολλέες· ἦρχε δὲ Θουρεύς, 
Ἰνδῴου πολέμοιο πέλωρ πρόμος, εἴκελος ὁρμὴν 
ἠλιβάτῳ Τυφῶνι καταΐσσοντι κεραυνοῦ. 



Nonnus Epic., Dionysiaca 
Book 22, line 144

καὶ στρατιαὶ πινυτοῖο δολόφρονι νεύματι Βάκχου 
ψευδαλέον φόβον εἶχον ἀταρβέες, ἐκ δὲ κυδοιμοῦ 
αὐτόματοι χάζοντο θελήμονες, εἰσόκεν Ἰνδοὶ 
ἐς πεδίον προχέοντο λελοιπότες ἔνδια λόχμης. 



Nonnus Epic., Dionysiaca 
Book 22, line 164

καὶ θεὸς ἀστήρικτος ὅλους ἐφόβησε μαχητὰς 
δυσμενέων, οὐ γυμνὸν ἔχων ξίφος, οὐ δόρυ πάλλων, 
ἀλλὰ μέσος προμάχων πεφορημένος εἴκελος αὔραις 
δεξιὸν ἐκ λαιοῖο κέρας κυκλώσατο χάρμης, 
θύρσον ἀκοντίζων δολιχόσκιον, ἄνθεϊ γαίης, 
ἔγχεϊ κισσήεντι διασχίζων νέφος Ἰνδῶν. 



Nonnus Epic., Dionysiaca 
Book 22, line 217

ὡς δ' ὅτε ῥιγαλέου σκιερὴν μετὰ χείματος ὥρην 
φαίνεται ἀσκεπέων νεφέων γυμνούμενος ἀήρ, 
φέγγεος εἰαρινοῖο δεδεγμένος αἴθριον αἴγλην·   
ὣς ὅ γε βακχεύων πυκινὰς στίχας ἄτρομος ἀνὴρ 
Ἰνδῶν σχιζομένων μεσάτην γυμνώσατο χάρμην. 



Nonnus Epic., Dionysiaca 
Book 22, line 250

οἱ δὲ βοῆς ἀίοντες ἐπὶ κλόνον ἔρρεον Ἰνδοί· 
θαρσαλέοι δ' ἥψαντο παλιννόστοιο κυδοιμοῦ. 



Nonnus Epic., Dionysiaca 
Book 22, line 266

οὐ πίσυνος σακέεσσι καὶ οὐ θώρηκι κυδοιμοῦ· 
ἀλλά ἑ πατρῴοις πεπυκασμένον ἀντὶ σιδήρου 
ἀρρήκτοις νεφέεσσιν ὅλον πύργωσεν Ἀθήνη, 
οἷς πάρος ἀβρέκτοιο κατέσβεσεν αὐχμὸν ἀρούρης 
διψαλέην ἐπὶ γαῖαν ἄγων βιοτήσιον ὕδωρ   
Ζηνὸς ἐπομβρήσαντος, ἀμαλλοτόκοιο δὲ γαίης 
αὔλακες εὐώδινες ἐνυμφεύθησαν ἀρότρῳ· 
καὶ μέσος ἀντιβίων κυκλούμενος ἔνθεος ἀνὴρ 
τοὺς μὲν ἀπηλοίησε θοῷ δορί, τοὺς δὲ μαχαίρῃ, 
τοὺς δὲ λίθοις κραναοῖσι· πέδον δ' ἐρυθαίνετο λύθρῳ 
Ἰνδῶν κτεινομένων, καὶ ἀκαμπέος ἀνέρος αἰχμῇ 
κεῖτο πολυσπερέων νεκύων χύσις, ὧν ὁ μὲν αὐτῶν 
ἡμιθανὴς ἤσπαιρεν, ὁ δὲ χθόνα ποσσὶν ἀράσσων 
ὕπτιος αὐτοκύλιστος ὁμίλεε γείτονι πότμῳ· 
καὶ δαπέδῳ στείνοντο, νέκυς δ' ἐπερείδετο νεκρῷ 
κεκλιμένῳ μετρηδόν, ἀπ' ἀρτιτόμοιο δὲ λαιμοῦ 
ψυχρὸν ἐρευθιόωντι δέμας θερμαίνετο λύθρῳ· 
καὶ φόνος ἄσπετος ἦεν, ἐπασσυτέρων δὲ πεσόντων 
Γαῖα κελαινιόωσα κατάρρυτος αἵματος ὁλκῷ, 
υἱέας οἰκτείρουσα, χαραδραίῃ φάτο φωνῇ· 




Nonnus Epic., Dionysiaca 
Book 22, line 279

καὶ δαπέδῳ στείνοντο, νέκυς δ' ἐπερείδετο νεκρῷ 
κεκλιμένῳ μετρηδόν, ἀπ' ἀρτιτόμοιο δὲ λαιμοῦ 
ψυχρὸν ἐρευθιόωντι δέμας θερμαίνετο λύθρῳ· 
καὶ φόνος ἄσπετος ἦεν, ἐπασσυτέρων δὲ πεσόντων 
Γαῖα κελαινιόωσα κατάρρυτος αἵματος ὁλκῷ, 
υἱέας οἰκτείρουσα, χαραδραίῃ φάτο φωνῇ· 
 “υἱὲ Διὸς ζείδωρε μιαιφόνε – καὶ γὰρ ἀνάσσεις 
ὄμβρου καρποτόκοιο καὶ αἱμαλέου νιφετοῖο – , 
ὄμβρῳ μὲν γονόεσσαν ὅλην ἐδίηνας ἀλωὴν 
Ἑλλάδος, Ἰνδῴην δὲ κατέκλυσας αὔλακα λύθρῳ, 
ὁ πρὶν ἀμαλλοφόρος, θανατηφόρος· ἀγρονόμοις μὲν 
σὸς νιφετὸς στάχυν εὗρε, σὺ δὲ στρατὸν ἔθρισας Ἰνδῶν 
ἀνέρας ἀμώων μετὰ λήιον· ἀμφότερον δὲ 
ἐκ Διὸς ὄμβρον ἄγεις, ἐξ Ἄρεος αἵματι νείφεις. 



Nonnus Epic., Dionysiaca 
Book 22, line 281

ψυχρὸν ἐρευθιόωντι δέμας θερμαίνετο λύθρῳ· 
καὶ φόνος ἄσπετος ἦεν, ἐπασσυτέρων δὲ πεσόντων 
Γαῖα κελαινιόωσα κατάρρυτος αἵματος ὁλκῷ, 
υἱέας οἰκτείρουσα, χαραδραίῃ φάτο φωνῇ· 
 “υἱὲ Διὸς ζείδωρε μιαιφόνε – καὶ γὰρ ἀνάσσεις 
ὄμβρου καρποτόκοιο καὶ αἱμαλέου νιφετοῖο – , 
ὄμβρῳ μὲν γονόεσσαν ὅλην ἐδίηνας ἀλωὴν 
Ἑλλάδος, Ἰνδῴην δὲ κατέκλυσας αὔλακα λύθρῳ, 
ὁ πρὶν ἀμαλλοφόρος, θανατηφόρος· ἀγρονόμοις μὲν 
σὸς νιφετὸς στάχυν εὗρε, σὺ δὲ στρατὸν ἔθρισας Ἰνδῶν 
ἀνέρας ἀμώων μετὰ λήιον· ἀμφότερον δὲ 
ἐκ Διὸς ὄμβρον ἄγεις, ἐξ Ἄρεος αἵματι νείφεις. 



Nonnus Epic., Dionysiaca 
Book 22, line 285

                                        ἀλλὰ Κρονίων 
οὐρανόθεν κελάδησε, καὶ Αἰακὸν εἰς φόνον Ἰνδῶν   
βρονταίοις πατάγοισι Διὸς προκαλίζετο σάλπιγξ. 



Nonnus Epic., Dionysiaca 
Book 22, line 290

μάρνατο δ' εἰσέτι μᾶλλον ἀνώδυνος εἰς μέσον Ἰνδῶν 
Αἰακὸς ἀστήρικτος, ἐπεὶ βέλος ἥπτετο μηροῦ 
λεπτὸς ὄνυξ ἅτε φωτός, ὅτε χροὸς ἄκρα χαράξῃ. 



Nonnus Epic., Dionysiaca 
Book 22, line 305

δύμεναι, ἧχι πάροιθεν ἐκεύθετο· τὸν δὲ διώκων 
εἰς δρόμον ἡνιόχευε ποδήνεμον ἵππον Ἐρεχθεύς· 
ἀλλ' ὅτε τόσσον ἔμαρψεν, ὅσον προμάχοιο βαλόντος 
ἔγχεος ἱπταμένοιο τιταίνεται ὄρθιος ὁρμή, 
δὴ τότε δὴ μετὰ νῶτα βαλὼν ἀντώπιος ἔστη 
πεζὸς ἀνὴρ ἱππῆα δεδεγμένος· αὐτὰρ ὁ κάμψας 
ὀκλαδὸν ἐστήριξεν ἀριστερὸν ἴχνος ἀρούρῃ 
λοξὸς ἐπὶ πλευρῇσιν, ὀπισθοτόνοιο δὲ ταρσοῦ 
ἴχνιον ἠέρταζε μετάρσιον, ὀρθὰ τιταίνων 
δεξιτεροῦ ποδὸς ἄκρα πεπηγότα δάκτυλα γαίῃ, 
Ἰνδικὸν ἑπταβόειον ἔχων σάκος, εἰκόνα πύργου, 
γυμνὸν ἔχων ξίφος ὀξύ· προϊσχόμενος δὲ προσώπου   
ἀσπίδα χαλκεόνωτον ἐπέδραμεν Ἰνδὸς ἀγήνωρ, 
ἢ θανέειν ἢ φῶτα βαλεῖν ἢ πῶλον ἐλάσσαι 
ἄορι τολμήεντι· καὶ ὀμφαλόεντι σιδήρῳ 
δόχμιος ἀντικέλευθον ἀνακρούσας γένυν ἵππου 
πεζὸς ἐὼν ἐτίναξεν ὑπέρτερον ἡνιοχῆα· 
καί νύ κεν ἐς χθόνα ῥῖψεν ἀμήτορος ἀστὸν Ἀθήνης, 
ἀλλά μιν ἔγχεϊ νύξε παρ' ὀμφαλὸν ἄκρον Ἐρεχθεὺς 
καὶ φονίῳ μέσον ἄνδρα πεπαρμένον ὀξέι χαλκῷ 




Nonnus Epic., Dionysiaca 
Book 22, line 307

ἔγχεος ἱπταμένοιο τιταίνεται ὄρθιος ὁρμή, 
δὴ τότε δὴ μετὰ νῶτα βαλὼν ἀντώπιος ἔστη 
πεζὸς ἀνὴρ ἱππῆα δεδεγμένος· αὐτὰρ ὁ κάμψας 
ὀκλαδὸν ἐστήριξεν ἀριστερὸν ἴχνος ἀρούρῃ 
λοξὸς ἐπὶ πλευρῇσιν, ὀπισθοτόνοιο δὲ ταρσοῦ 
ἴχνιον ἠέρταζε μετάρσιον, ὀρθὰ τιταίνων 
δεξιτεροῦ ποδὸς ἄκρα πεπηγότα δάκτυλα γαίῃ, 
Ἰνδικὸν ἑπταβόειον ἔχων σάκος, εἰκόνα πύργου, 
γυμνὸν ἔχων ξίφος ὀξύ· προϊσχόμενος δὲ προσώπου   
ἀσπίδα χαλκεόνωτον ἐπέδραμεν Ἰνδὸς ἀγήνωρ, 
ἢ θανέειν ἢ φῶτα βαλεῖν ἢ πῶλον ἐλάσσαι 
ἄορι τολμήεντι· καὶ ὀμφαλόεντι σιδήρῳ 
δόχμιος ἀντικέλευθον ἀνακρούσας γένυν ἵππου 
πεζὸς ἐὼν ἐτίναξεν ὑπέρτερον ἡνιοχῆα· 
καί νύ κεν ἐς χθόνα ῥῖψεν ἀμήτορος ἀστὸν Ἀθήνης, 
ἀλλά μιν ἔγχεϊ νύξε παρ' ὀμφαλὸν ἄκρον Ἐρεχθεὺς 
καὶ φονίῳ μέσον ἄνδρα πεπαρμένον ὀξέι χαλκῷ 
εἰς πέδον ἠκόντιζεν· ὁ δὲ στροφάδεσσιν ἐρωαῖς 
ἠερόθεν προκάρηνος ἐπωλίσθησε κονίῃ 




Nonnus Epic., Dionysiaca 
Book 22, line 344

ὡς δ' ὅτε χαλκείῳ τις ἐπ' ἄκμονι χαλκὸν ἐλαύνων 
ἀκαμάτῳ ῥαιστῆρι πυρίβρομον ἦχον ἰάλλει, 
τύπτων γείτονα μύδρον, ἀποθρῴσκουσι δὲ πολλοὶ 
ἁλλόμενοι σπινθῆρες ἀρασσομένοιο σιδήρου, 
ἠέρα θερμαίνοντες, ἀμοιβαίῃσι δὲ ῥιπαῖς 
ὃς μὲν ἔην προκέλευθος, ὁ δὲ σχεδόν, ἄλλος ὀρούσας 
ἄλλον ἔτι θρῴσκοντα κιχάνεται αἴθοπι παλμῷ· 
ὣς ὅ γε τοξεύων στρατιὴν ἀντώπιον Ἰνδῶν 
μαρναμένων ἐκέδασσεν, ἀλωφήτων ἀπὸ τόξων 
κτείνων ἄλλοθεν ἄλλον ἐπασσυτέροισι βελέμνοις. 



Nonnus Epic., Dionysiaca 
Book 22, line 348

μεσσατίης δὲ φάλαγγος ἀλευαμένης νέφος ἰῶν 
χῶρος ἐγυμνώθη, κεραῆς ἴνδαλμα Σελήνης,   
ἀμφιφαὴς ὅτε βαιὸν ἀποστίλβουσα κεραίης 
ἄκρα διαπλήσασα δύω νεοφεγγέος αἴγλης 
κεκλιμέναις ἀκτῖσι μέσον κύκλοιο χαράσσει, 
δίζυγι κεκριμένῳ μαλακῷ πυρί, μεσσατίης δὲ 
γυμνὰ χαρασσομένης ἔτι φαίνετο κύκλα Σελήνης. 



Nonnus Epic., Dionysiaca 
Book 22, line 394

ἄρκιον Ἰνδὸν ὄλεσσε τεὸν δόρυ· παύεο Νύμφαις 
δάκρυα Νηιάδεσσιν ἀδακρύτοισιν ἐγείρων· 
Νηιὰς ὑδατόεσσα καὶ ὑμετέρη πέλε μήτηρ· 
κούρην γὰρ ποταμοῖο τεὴν Αἴγιναν ἀκούω. 



Nonnus Epic., Dionysiaca 
Book 23, line 12

οὐδ' ἐπὶ δὴν παρὰ θῖνα φερεσσακέος ποταμοῖο 
πληθύι τοσσατίῃ φονίων κυκλούμενος Ἰνδῶν 
Αἰακὸς εἰσέτι μίμνεν, ἐπεὶ μογέοντι παρέστη 
Ἰνδοφόνος Διόνυσος ἀκαχμένα θύρσα τινάσσων. 



Nonnus Epic., Dionysiaca 
Book 23, line 21

                                         εἰ δέ τις ἀνὴρ 
νήχετο δαιδαλέης ὑπὲρ ἀσπίδος οἴδματα τέμνων, 
νηχομένου κεράιξε μετάφρενον· εἰ δέ τις Ἰνδῶν 
ἡμιφανὴς πολέμιζεν ἐπ' ἰλύι ταρσὸν ἐρείσας, 
θύρσῳ στῆθος ἔτυψεν ἢ αὐχένα, κύματα τέμνων, 
δυόμενος· βυθίων γὰρ ἐπίστατο κόλπον ἐναύλων, 
ἐξ ὅτε μιν φεύγοντα μόθον δασπλῆτα Λυκούργου 
δώματι κυμαίνοντι γέρων ὑπεδέξατο Νηρεύς. 



Nonnus Epic., Dionysiaca 
Book 23, line 52

καί τις ἑοὺς ἑτάρους δεδοκημένος Ἰνδὸς ἀγήνωρ 
τοὺς μὲν κτεινομένους δολιχῷ δορί, τοὺς δὲ μαχαίρῃ, 
ἄλλον ὀιστευθέντα χαραδρήεντι βελέμνῳ, 
τὸν δὲ πολυπλέκτῳ δεδαϊγμένον ὀξέι θύρσῳ, 
Θουρέι νεκρὸν ὅμιλον ἐδείκνυεν, ἀχνύμενος δὲ 
τίλλε κόμην, φλογερῷ δὲ χόλου βακχεύετο πυρσῷ, 
σφίγγων καρχαρόδοντι μεμυκότα χείλεα δεσμῷ· 
καὶ ταχὺς αὐτοφόνον μιμούμενος Ἰνδὸν Ὀρόντην, 
βάρβαρον αἷμα φέρων καὶ βάρβαρον ἦθος ἀέξων, 
ἆορ ἑὸν γύμνωσεν, ἀπορρίψας δὲ χιτῶνα,   




Nonnus Epic., Dionysiaca 
Book 23, line 117

Ἥρη δ' ὡς ἐνόησε δαϊκταμένων φόνον Ἰνδῶν, 
οὐρανόθεν πεπότητο, δι' ὑψιπόρου δὲ κελεύθου 
ἄστατος ἠνεμόεντι κατέγραφεν ἠέρα ταρσῷ. 



Nonnus Epic., Dionysiaca 
Book 23, line 120

Ἀντολίης δ' ἐπέβαινε, καὶ ὥπλισεν Ἰνδὸν Ὑδάσπην 
φύλοπιν ὑδατόεσσαν ἀναστῆσαι Διονύσῳ. 



Nonnus Epic., Dionysiaca 
Book 23, line 129

καὶ θεὸς ἡγεμόνευε, δι' οἴδματος ἡνιοχεύων 
ἅρμασι χερσαίοισι νόθον πλόον, ὑγροπόρων δὲ 
πορδαλίων ἀδίαντος ὄνυξ ἐχάραξεν Ὑδάσπην· 
καὶ στρατιαὶ πλόον εἶχον ἀκυμάντου ποταμοῖο, 
ὧν ὁ μὲν Ἰνδῴην σχεδίην πολύδεσμον ἐρέσσων, 
ἅμματι τεχνήεντι περίπλοκα δούρατα δήσας . 



Nonnus Epic., Dionysiaca 
Book 23, line 149

καὶ στρατὸς ἐγρεμόθων πρυλέων ἀκάτοιο χατίζων, 
ἀσκοῖς οἰδαλέοισι χέων ποιητὸν ἀήτην, 
δέρματι φυσαλέῳ διεμέτρεεν Ἰνδὸν Ὑδάσπην, 
ἐνδομύχων δ' ἀνέμων ἐγκύμονες ἔπλεον ἀσκοί. 



Nonnus Epic., Dionysiaca 
Book 23, line 188

οὔ ποτε τολμήεντες ἐμὸν ῥόον ἔπλεον Ἰνδοὶ 
ἅρμασιν ἠλιβάτοισι, καὶ οὐ πατρώιον ὕδωρ 
Δηριάδης ἐχάραξεν ἑῷ περιμήκεϊ δίφρῳ, 
ὑψιλόφων λοφιῇσιν ἐφεδρήσσων ἐλεφάντων. 



Nonnus Epic., Dionysiaca 
Book 23, line 276

Ὑδριάδων δὲ φάλαγγες ἀνάμπυκες ὠκέι ταρσῷ 
γυμναὶ κυματόεντος ἀπεπλάζοντο μελάθρου· 
καί τις ἀναινομένη φλογερὸν πατρώιον ὕδωρ 
Νηιὰς ἀκρήδεμνος ἀήθεα δύσατο Γάγγην· 
ἄλλη δ' Ἰνδὸν ἔναιεν ἐριβρεμέτην Ἀκεσίνην 
ἀζαλέοις μελέεσσιν· ἀλωομένην δὲ Χοάσπης 
ἄλλην οὐρεσίφοιτον ἀνάμπυκα Νηίδα Νύμφην 
παρθενικὴν ἀπέδιλον ἐδέξατο, Περσίδι γείτων. 



Nonnus Epic., Dionysiaca 
Book 23, line 319

Τηθύς, καὶ σύ, θάλασσα, κορύσσεο· ταυροφυῆ γὰρ 
Ζεὺς νόθον υἷα λόχευσεν, ἵνα ξύμπαντας ὀλέσσῃ 
καὶ ποταμοὺς καὶ φῶτας ἀμεμφέας· ἀμφότερον δὲ 
Ἰνδοὺς θύρσος ἔπεφνε καὶ ἔφλεγε πυρσὸς Ὑδάσπην. 



Nonnus Epic., Dionysiaca 
Book 24, line 17

ἀλλὰ πόθος τεκέων με βιήσατο· Δηριάδῃ γὰρ 
υἱέι πιστὰ φέρων ῥοθίων ἐλέλιζον ἀπειλήν, 
Ἰνδοῖς κτεινομένοισι βοηθόον οἶδμα κυλίνδων. 



Nonnus Epic., Dionysiaca 
Book 24, line 27

                                                    ... 
Νηιάδες φεύγουσιν ἐμὸν ῥόον· ἀμφὶ δὲ πηγὰς 
ἡ μὲν ναιετάει διερὸν δόμον, ἡ δ' ἐνὶ λόχμαις 
σύννομος Ἀδρυάδεσσι φυτὸν μετὰ πόντον ἀμείβει, 
ἄλλη δ' Ἰνδὸν ἔχει μετανάστιος, ἡ δὲ φυγοῦσα 
ποσσὶ κονιομένοισιν ἐδύσατο διψάδα πέτρην 
Καυκασίην, ἑτέρη δὲ μεταΐξασα Χοάσπην 
ναίει ξεῖνα ῥέεθρα καὶ οὐκέτι πάτριον ὕδωρ. 



Nonnus Epic., Dionysiaca 
Book 24, line 70

ὄφρα μὲν εἰσέτι Βάκχος ἐπέπλεεν ὑγρὸν Ὑδάσπην, 
τόφρα δέ, θάρσος ἄρηος ἔχων περιμήκεα μορφήν, 
Δηριάδης ἐπὶ δῆριν ἐπώνυμον ὥπλισεν Ἰνδούς, 
στήσας ἀμφὶ ῥέεθρον ἑὰς στίχας, ὄφρα μαχηταὶ 
λαὸν ἐρητύσωσιν ἀνερχομένων ἔτι Βάκχων. 



Nonnus Epic., Dionysiaca 
Book 24, line 82

καὶ σφετέροισιν † ἰόντες ἀρηγόνες, ἄλλος ἐπ' ἄλλῳ, 
σὺν Διὶ πάντες ἵκοντο θεοὶ ναετῆρες Ὀλύμπου 
ἅλματι πωτήεντι· καὶ Αἰγίνης χάριν εὐνῆς 
αἰετὸς ᾐώρητο τὸ δεύτερον ὑψιπέτης Ζεὺς 
Ἀσωποῦ μετὰ χεῦμα, καὶ Αἰακὸν ἠεροφοίτην 
φειδομένων ὀνύχων δεδραγμένος ἅρπαγι ταρσῷ 
κουφίζων ἐκόμισσεν ἐς ἄρεα Δηριαδῆος 
Ἰνδῴην ἐπὶ πέζαν· ἀπ' εὐρυπόροιο δὲ κόλπου 
υἱὸν Ἀρισταῖον γενέτης ἐσάωσεν Ἀπόλλων, 
φαιδρὸς ἀλεξικάκων πεφορημένος ἅρματι κύκνων, 
μνῆστιν ἔχων θαλάμοιο λεοντοφόνοιο Κυρήνης· 
καὶ κρατέων ἕο παῖδα τανύπτερος ἥρπασεν Ἑρμῆς, 
υἱέα Πηνελόπης, κεραελκέα Πᾶνα κομήτην· 
Οὐρανίη δ' Ὑμέναιον ἀνεζώγρησεν ὀλέθρου   
παιδὸς ἑοῦ γονόεντος ἐπώνυμον, ἠερίας δὲ 
ἀτραπιτοὺς ἐχάραξεν, ὁμοίιος ἀστέρος ὁλκῷ, 
γνωτῷ βοτρυόεντι χαριζομένη Διονύσῳ· 




Nonnus Epic., Dionysiaca 
Book 24, line 96

καὶ κρατέων ἕο παῖδα τανύπτερος ἥρπασεν Ἑρμῆς, 
υἱέα Πηνελόπης, κεραελκέα Πᾶνα κομήτην· 
Οὐρανίη δ' Ὑμέναιον ἀνεζώγρησεν ὀλέθρου   
παιδὸς ἑοῦ γονόεντος ἐπώνυμον, ἠερίας δὲ 
ἀτραπιτοὺς ἐχάραξεν, ὁμοίιος ἀστέρος ὁλκῷ, 
γνωτῷ βοτρυόεντι χαριζομένη Διονύσῳ· 
Καλλιόπη δ' Οἴαγρον ἑοῖς ἀνεκούφισεν ὤμοις· 
καὶ τεκέων Ἥφαιστος ἑῶν ἀλέγιζε Καβείρων, 
ἀμφοτέρους δ' ἥρπαξεν, ὁμοίιος ὀξέι πυρσῷ· 
Ἀκταίη δ' ἐσάωσεν Ἐρεχθέα Παλλὰς Ἀθήνη, 
Ἰνδοφόνον ναετῆρα θεοκρήπιδος Ἀθήνης· 
Νύμφας δ' Ἀδρυάδας ναέται ζώγρησαν Ὀλύμπου 
πάντες, ὅσοις μεμέληντο φίλαι δρύες, ἔξοχα δ' ἄλλων 
δαφναίας ἐσάωσε φανεὶς δαφναῖος Ἀπόλλων, 
καί σφιν ἅμα χραίσμησε συνέμπορος υἱέι μήτηρ, 
εἰσέτι κυδαίνουσα λεχώια δένδρεα Λητώ. 



Nonnus Epic., Dionysiaca 
Book 24, line 106

Βασσαρίδων δὲ φάλαγγα κορυμβοφόρους τε γυναῖκας 
ἐκ βυθίου ῥύσαντο πολυφλοίσβοιο κυδοιμοῦ 
θυγατέρες Κύδνοιο, φιλοζεφύρου ποταμοῖο, 
πλωτὸν ἐπιστάμεναι διερὸν δρόμον, ἃς ἐπὶ νίκῃ 
ἄρεος Ἰνδῴοιο πατὴρ δωρήσατο Βάκχῳ, 
Νηιάδας πολέμοιο δαήμονας, ἅς ποτε χάρμην 
μαρνάμενος Κρονίωνι Κίλιξ ἐδίδαξε Τυφωεύς. 



Nonnus Epic., Dionysiaca 
Book 24, line 123

ἄλλοι δ' ἦσαν ὄπισθεν, ἐπεσσεύοντο δὲ πορθμῷ 
ἐξ ἑτέρης ἀνιόντες ἀθηήτοιο κελεύθου, 
ἧχι θεὸς πόμπευεν· ἐπεὶ πτερὸν ἠρέμα πάλλων 
αἰετὸς ἡγεμόνευε δι' ἠέρος ἀντίτυπος Ζεύς, 
φειδομένοις ὀνύχεσσι μετάρσιον υἷα κομίζων, 
Αἰακὸν ἠερίῃ πεφορημένον ὕψι κελεύθῳ. 
 Ἰνδῴῃ δ' ἐχόρευον ἐπισκαίροντες ἐρίπνῃ, 
καὶ σκοπέλους ἐδίωκον, ἐναυλίζοντο δὲ λόχμαις, 
καὶ κλισίας πήξαντες † ἐς ἠρέμα δάσκιον ὕλην . 



Nonnus Epic., Dionysiaca 
Book 24, line 145

                                          ἀχνύμενος δὲ 
Δηριάδῃ βασιλῆι δυσάγγελος ἵκετο Θουρεύς, 
δάκρυσιν ἀφθόγγοισιν ἀπαγγέλλων φόνον Ἰνδῶν, 
καὶ μόγις ἐκ στομάτων ἀνενείκατο πενθάδα φωνήν· 
 “Δηριάδη σκηπτοῦχε, θεηγενὲς ἔρνος Ἐνυοῦς, 
ᾔομεν, ὡς ἐκέλευσας, ἐς ἀντιπέραιαν ἐρίπνην, 
εὕρομεν ἐν βήσσῃσιν ἐρημάδα γείτονα λόχμην· 
κεῖθι λόχον στήσαντες ἐμίμνομεν, εἰσόκεν ἔλθῃ 
θυρσομανὴς Διόνυσος· ἐπερχομένοιο δὲ Βάκχου 
αὐλὸς ἐπεσμαράγησεν, ἀδεψήτου δὲ βοείης   
τυπτομένης ἑκάτερθεν ἔην χαλκόκροτος ἠχὼ 
καὶ καναχὴ σύριγγος· ὅλη δ' ἐλελίζετο λόχμη 




Nonnus Epic., Dionysiaca 
Book 24, line 159

καὶ θεός, ὃν καλέουσιν, ἀκαχμένα θύρσα τινάσσων, 
οὐτιδανοῖς πετάλοισιν ὀιστεύων γένος Ἰνδῶν, 
κτεῖνε μὲν ἐν πεδίῳ στρατὸν ἄσπετον ὀξέι θύρσῳ 
βλήμενον, ἐν ῥοθίοις δὲ τὸ λείψανον ὤλεσεν Ἰνδῶν. 



Nonnus Epic., Dionysiaca 
Book 24, line 169

εἰ δὲ πόθος μεθέπει σε δυσαντήτοιο κυδοιμοῦ, 
σήμερον Ἰνδὸν ἔρυκε, καὶ αὔριον εἰς μόθον ἕλκεις. 



Nonnus Epic., Dionysiaca 
Book 24, line 173

ὣς εἰπὼν παρέπεισεν ἀπειθέα Δηριαδῆα, 
οὐ χάριν ἀδρανίης πειθήμονα, δυομένῳ δὲ 
μεμφόμενον Φαέθοντι καὶ οὐκ εἴκοντα Λυαίῳ. 
Ἰνδῴην δὲ φάλαγγα μεταστήσας ποταμοῖο 
Δηριάδης ὑπέροπλος ἐχάζετο πενθάδι λύσσῃ, 
ἑζόμενος λοφιῇσι παλιννόστων ἐλεφάντων. 



Nonnus Epic., Dionysiaca 
Book 24, line 176

Ἰνδῴην δὲ φάλαγγα μεταστήσας ποταμοῖο 
Δηριάδης ὑπέροπλος ἐχάζετο πενθάδι λύσσῃ, 
ἑζόμενος λοφιῇσι παλιννόστων ἐλεφάντων. 
Ἰνδοὶ δ' ἔνθα καὶ ἔνθα σὺν ἠλιβάτῳ βασιλῆι   
εἰς πόλιν ἐρρώοντο πεφυζότες, ἔνδοθι πύργων 
νίκην εἰσαΐοντες ἀρειμανέος Διονύσου. 



Nonnus Epic., Dionysiaca 
Book 24, line 180

ἤδη δὲ στονόεσσα δι' ἄστεος ἵπτατο Φήμη, 
σύγγονον ἀγγέλλουσα νεοσφαγέων φόνον Ἰνδῶν. 



Nonnus Epic., Dionysiaca 
Book 24, line 196

καί τις ἀμηχανέουσα δεδουπότος εὐνέτις Ἰνδοῦ, 
ἀγχιτόκους ὠδῖνας ἀναπλήσασα λοχείης 
καὶ δεκάτης ὁρόωσα λεχώια κύκλα Σελήνης, 
ὑδρηλῷ πολύδακρυς ἐπέστενεν ἀνδρὸς ὀλέθρῳ, 
καὶ ποταμῷ κοτέουσα γοήμονα ῥήξατο φωνήν· 
 “οὐ πίομαι πατρῷον ἐμόν ποτε πικρὸν Ὑδάσπην· 
οὐκέτι κεῖνα ῥέεθρα μετέρχομαι, οὐκέτι δειλὴ   
σεῖο νέκυν κρύψαντος ἐπιψαύσω ποταμοῖο, 
οὐ μὰ σὲ καὶ σέο φόρτον, ὃν ἔνδοθι γαστρὸς ἀείρω, 
οὐ μὰ σὲ καὶ τὸν ἔρωτα, τὸν οὐ χρόνος οἶδε μαραίνειν. 



Nonnus Epic., Dionysiaca 
Book 24, line 219

                                         ἀμφὶ δὲ λόχμας 
Βάκχος ἑοῖς Σατύροισι καὶ Ἰνδοφόνοισι μαχηταῖς 
εἰλαπίνην ἔστησεν· ἐδαιτρεύοντο δὲ ταῦροι, 
καὶ δαμάλαι στοιχηδὸν ἐμιστύλλοντο μαχαίρῃ 
θεινόμεναι πελέκεσσιν Ἐρυθραίης τ' ἀπὸ ποίμνης 
πυκνὰ δορικτήτων ἱερεύετο πώεα μήλων. 



Nonnus Epic., Dionysiaca 
Book 24, line 337

ἀλλ' ὅτε δὴ κόρος ἔσκε φιλακρήτοιο τραπέζης, 
οἶνον ἀναβλύζοντες ἐρημάδι κάππεσον εὐνῇ, 
οἱ μὲν δαιδαλέης ἐπὶ νεβρίδος, οἱ δ' ἐπὶ φύλλων   
πεπταμένων· ἕτεροι δὲ χυτῆς ἐφύπερθε κονίης 
δέρμασιν αἰγείοισιν ἐπετρέψαντο χαμεύνην· 
ἄλλοι δ' ἐγρεμόθοισιν ἐφωμίλησαν ὀνείροις 
χάλκεον ἁπλώσαντες ἐνυαλίῳ δέμας ὕπνῳ· 
ὧν ὁ μὲν Ἰνδὸν ἔβαλλε καθήμενον ὑψόθεν ἵππου, 
ἄλλος δ' ἵππον ἔνυξε κατ' αὐχένος, ὃς δὲ δαΐζων 
ἄορι πεζὸν ἔτυψεν, ὁ δ' οὔτασε Δηριαδῆα· 
ἄλλος δ' ἠερόφοιτον ἑὸν βέλος ὑψόσε πέμπων 
ἠλιβάτους ἐλέφαντας ὀνειρείῳ βάλεν ἰῷ. 



Nonnus Epic., Dionysiaca 
Book 24, line 346

πορδαλίων δὲ γένεθλα καὶ ἄγρια φῦλα λεόντων 
καὶ κύνες ἀγρευτῆρες ἐρημονόμου Διονύσου 
εἶχον ἀμοιβαίης φυλακῆς ἄγρυπνον ὀπωπήν, 
πάννυχον ἐγρήσσοντες ὀρειάδος ἔνδοθεν ὕλης, 
μή σφιν ἐπαΐξειε μελαρρίνων μόθος Ἰνδῶν· 
καὶ δαΐδες στοιχηδὸν ἐπαστράπτεσκον Ὀλύμπῳ, 
Βακχιάδος λαμπτῆρες ἀκοιμήτοιο χορείης. 



Nonnus Epic., Dionysiaca 
Book p2, line 52

<εἰκοστὸν> λάχεν <ἕκτον> ἐπίκλοπον εἶδος Ἀθήνης 
καὶ πολὺν ἐγρεκύδοιμον ἀγειρομένων στόλον Ἰνδῶν. 



Nonnus Epic., Dionysiaca 
Book p2, line 78

ἐν δὲ <τριηκοστῷ ἐνάτῳ> μετὰ κύματα λεύσσεις 
Δηριάδην φεύγοντα πυριφλεγέων στόλον Ἰνδῶν. 



Nonnus Epic., Dionysiaca 
Book p2, line 79

<τεσσαρακοστὸν> ἔχει δεδαϊγμένον ὄρχαμον Ἰνδῶν, 
πῶς δὲ Τύρον Διόνυσος ἐδύσατο, πατρίδα Κάδμου. 



Nonnus Epic., Dionysiaca 
Book 25, line 5

Μοῦσα, πάλιν πολέμιζε σοφὸν μόθον ἔμφρονι θύρσῳ· 
οὔ πω γὰρ γόνυ δοῦλον ὑποκλίνων Διονύσῳ 
φύλοπιν ἑπταέτηρον Ἑώιος εὔνασεν ἄρης· 
ἀλλὰ δρακοντείοιο τεθηπότες ἄκρα γενείου 
Ἰνδῴης πλατάνοιο πάλιν κλάζουσι νεοσσοί, 
Βακχείου πολέμοιο προμάντιες. 



Nonnus Epic., Dionysiaca 
Book 25, line 8

                                  οὐ μὲν ἀείσω 
πρώτους ἓξ λυκάβαντας, ὅτε στρατὸς ἔνδοθι πύργων 
Ἰνδὸς ἔην· τελέσας δὲ τύπον μιμηλὸν Ὁμήρου 
ὕστατον ὑμνήσω πολέμων ἔτος, ἑβδομάτης δὲ 
ὑσμίνην ἰσάριθμον ἐμῆς στρουθοῖο χαράξω· 
Θήβῃ δ' ἑπταπύλῳ κεράσω μέλος, ὅττι καὶ αὐτὴ 
ἀμφ' ἐμὲ βακχευθεῖσα περιτρέχει, οἷα δὲ νύμφη 
μαζὸν ἑὸν γύμνωσε κατηφέος ὑψόθι πέπλου, 
μνησαμένη Πενθῆος· ἐποτρύνων δέ με μέλπειν 
πενθαλέην ἕο χεῖρα γέρων ὤρεξε Κιθαιρὼν 
αἰδόμενος, μὴ λέκτρον ἀθέσμιον ἠὲ βοήσω 
πατροφόνον πόσιν υἷα παρευνάζοντα τεκούσῃ. 



Nonnus Epic., Dionysiaca 
Book 25, line 22

ἀλλὰ πάλιν κτείνωμεν Ἐρυθραίων γένος Ἰνδῶν· 
οὔ ποτε γὰρ μόθον ἄλλον ὁμοίιον ἔδρακεν Αἰὼν 
Ἠῴου πρὸ μόθοιο, καὶ οὐ μετὰ φύλοπιν Ἰνδῶν 
ἄλλην ὀψιτέλεστον ἰσόρροπον εἶδεν ἐνυώ, 
οὐδὲ τόσος στρατὸς ἦλθεν ἐς Ἴλιον, οὐ στόλος ἀνδρῶν 
τηλίκος. 



Nonnus Epic., Dionysiaca 
Book 25, line 71

φρουρὸν ἀκοιμήτοιο μετήλυδα κύκλον ὀπωπῆς 
Φορκίδος ἀλλοπρόσαλλον ἀμειβομένης πτερὸν Ὕπνου 
ἤνυσε θῆλυν ἄεθλον ἀθωρήκτοιο Μεδούσης· 
ἀλλὰ διατμήγων δηίων στίχα δίζυγι νίκῃ 
χερσαίου πολέμοιο καὶ ὑγροπόροιο κυδοιμοῦ 
λύθρῳ γαῖαν ἔδευσε, καὶ αἵματι κῦμα κεράσσας 
Νηρεΐδας φοίνιξεν ἐρευθιόωντι ῥεέθρῳ, 
κτείνων βάρβαρα φῦλα· πολὺς δ' ἐπὶ μητέρι Γαίῃ 
ὑψιλόφων ἀκάρηνος ἐτυμβεύθη στάχυς Ἰνδῶν, 
πολλοὶ δ' ἐν πελάγεσσιν ὀλωλότες ὀξέι θύρσῳ 
αὐτόματοι πλωτῆρες ἐπορθμεύοντο θαλάσσῃ, 
Ἰνδῶν νεκρὸς ὅμιλος. 



Nonnus Epic., Dionysiaca 
Book 25, line 74

ἤνυσε θῆλυν ἄεθλον ἀθωρήκτοιο Μεδούσης· 
ἀλλὰ διατμήγων δηίων στίχα δίζυγι νίκῃ 
χερσαίου πολέμοιο καὶ ὑγροπόροιο κυδοιμοῦ 
λύθρῳ γαῖαν ἔδευσε, καὶ αἵματι κῦμα κεράσσας 
Νηρεΐδας φοίνιξεν ἐρευθιόωντι ῥεέθρῳ, 
κτείνων βάρβαρα φῦλα· πολὺς δ' ἐπὶ μητέρι Γαίῃ 
ὑψιλόφων ἀκάρηνος ἐτυμβεύθη στάχυς Ἰνδῶν, 
πολλοὶ δ' ἐν πελάγεσσιν ὀλωλότες ὀξέι θύρσῳ 
αὐτόματοι πλωτῆρες ἐπορθμεύοντο θαλάσσῃ, 
Ἰνδῶν νεκρὸς ὅμιλος. 



Nonnus Epic., Dionysiaca 
Book 25, line 85

Βάκχου δ' Ἰνδοφόνου βριαροῦ πόνος οὐ μία Γοργώ, 
οὐ λίθος ἠερόφοιτος ἁλίκτυπος ἢ Πολυδέκτης· 
ἀλλὰ δρακοντοκόμων καλάμην ἤμησε Γιγάντων 
Βάκχος ἀριστεύων ὀλίγῳ ῥηξήνορι θύρσῳ, 
ὁππότε Πορφυρίωνι μαχήμονα κισσὸν ἰάλλων 
Ἐγκέλαδον στυφέλιξε καὶ ἤλασεν Ἀλκυονῆα 
αἰχμάζων πετάλοισιν· ὀιστεύοντο δὲ θύρσοι 
Γηγενέων ὀλετῆρες, ἀοσσητῆρες Ὀλύμπου, 
χερσὶ διηκοσίῃσιν ἕλιξ ὅτε λαὸς ἀρούρης 
θλίβων ἀστερόεσσαν ἴτυν πολυδειράδι κόρσῃ 




Nonnus Epic., Dionysiaca 
Book 25, line 99

ἀλλά, φίλοι, κρίνωμεν· ἐν ἀντολίῃ μὲν ἀρούρῃ 
Ἰνδοφόνους ἱδρῶτας ὀπιπεύων Διονύσου 
Ἠέλιος θάμβησεν, ὑπὲρ δυτικοῖο δὲ κόλπου 
Ἑσπερίη Περσῆα τανύπτερον εἶδε Σελήνη, 
βαιὸν ἀεθλεύσαντα πόνον γαμψώνυχι χαλκῷ· 
καὶ Φαέθων ὅσον εὖχος ὑπέρτερον ἔλλαχε Μήνης, 
τόσσον ἐγὼ Περσῆος ἀρείονα Βάκχον ἐνίψω. 



Nonnus Epic., Dionysiaca 
Book 25, line 168

Μίνως μὲν πτολίπορθος ἑῷ ποτε κάλλεϊ γυμνῷ 
ὑσμίνης τέλος εὗρε, καὶ οὐ νίκησε σιδήρῳ, 
ἀλλὰ πόθῳ καὶ ἔρωτι· κορυσσομένου δὲ Λυαίου 
οὐ Πόθος ἐπρήυνεν ἀκοντοφόρων μόθον Ἰνδῶν, 
οὐ Παφίη κεκόρυστο συναιχμάζουσα Λυαίῳ, 
κάλλεϊ νικήσασα μόθου τέλος, οὐ μία κούρη 
οἰστρομανὴς χραίσμησεν ἐρασσαμένη Διονύσου, 
οὐ δόλος ἱμερόεις, οὐ βόστρυχα Δηριαδῆος, 
ἀλλὰ πολυσπερέων ποταμῶν ἑτερότροπος Ἰνδὸς 
νίκης εὖχος ἔχων παλιναυξέος. 



Nonnus Epic., Dionysiaca 
Book 25, line 245

ἆθλα μὲν Ἡρακλῆος, ὃν ἤροσεν ἀθάνατος Ζεὺς 
Ἀλκμήνης τρισέληνον ἔχων παιδοσπόρον εὐνήν, 
οὐτιδανὸς πόνος ἦεν ὀρίτροφος· ἔργα δὲ Βάκχου 
ἠὲ Γίγας πολύπηχυς ἢ ὑψιλόφων πρόμος Ἰνδῶν, 
οὐ κεμάς, οὐ βοέης ἀγέλης στίχες, οὐ λάσιος σῦς, 
οὐδὲ κύων, ἢ ταῦρος, ἢ αὐτόπρεμνος ὀπώρη 
χρυσοφαής, ἢ κόπρος, ἢ ἄστατος ὄρνις ἀλήτης 
οὐτιδανὴν ἀσίδηρον ἔχων πτερόεσσαν ἀκωκήν, 
ἢ γένυς ἱππείη ξεινοκτόνος, οὐ μία μίτρη 
Ἱππολύτης ἐλάχεια· Διωνύσοιο δὲ νίκη 
Δηριάδης ἀπέλεθρος ἢ εἰκοσίπηχυς Ὀρόντης. 



Nonnus Epic., Dionysiaca 
Book 25, line 263

                                         ἀλλὰ λιγαίνειν   
πνεῦσον ἐμοὶ τεὸν ἄσθμα θεόσσυτον· ὑμετέρης γὰρ 
δεύομαι εὐεπίης, ὅτι τηλίκον ἄρεα μέλπων 
Ἰνδοφόνους ἱδρῶτας ἀμαλδύνω Διονύσου. 



Nonnus Epic., Dionysiaca 
Book 25, line 264

ἀλλὰ θεά με κόμιζε τὸ δεύτερον ἐς μόθον Ἰνδῶν, 
ἔμπνοον ἔγχος ἔχοντα καὶ ἀσπίδα πατρὸς Ὁμήρου, 
μαρνάμενον Μορρῆι καὶ ἄφρονι Δηριαδῆι 
σὺν Διὶ καὶ Βρομίῳ κεκορυθμένον· ἐν δὲ κυδοιμοῖς 
Βακχιάδος σύριγγος ἀγέστρατον ἦχον ἀκούσω 
καὶ κτύπον οὐ λήγοντα σοφῆς σάλπιγγος Ὁμήρου, 
ὄφρα κατακτείνω νοερῷ δορὶ λείψανον Ἰνδῶν. 



Nonnus Epic., Dionysiaca 
Book 25, line 271

ὣς ὁ μὲν Ἰνδῴοιο περὶ ῥάχιν εὔβοτον ὕλης 
ἕζετο Βάκχος ὅμιλος ἐρημάδος ἀστὸς ἐρίπνης, 
ἀμβολίῃ πολέμοιο· φόβῳ δ' ἐλελίζετο Γάγγης 
οἰκτείρων ἑὰ τέκνα· νεοφθιμένων δ' ἐπὶ πότμῳ 
πᾶσα πόλις δεδόνητο· φιλοθρήνων δὲ γυναικῶν 
πενθαλέοις πατάγοισιν ἐπεσμαράγησαν ἀγυιαί. 



Nonnus Epic., Dionysiaca 
Book 25, line 299

ἤδη δ' ἀμπελόεσσα δι' ἄστεος ἔτρεχεν ὀδμὴ 
καὶ λιγυροῖς ἀνέμοισιν ὅλας ἐμέθυσσεν ἀγυιάς, 
νίκην Ἰνδοφόνοιο προθεσπίζουσα Λυαίου. 



Nonnus Epic., Dionysiaca 
Book 25, line 304

                                  ἐν δὲ κολώναις 
ἀσχαλόων Διόνυσος ἐμέμφετο πολλάκις Ἥρῃ, 
ὅττι πάλιν φθονέουσα μάχην ἀνεσείρασεν Ἰνδῶν, 
νίκης δ' ἐλπίδα πᾶσαν ἀνερρίπιζον ἀῆται. 



Nonnus Epic., Dionysiaca 
Book 25, line 322

Ῥείης θεσπεσίης ταχὺς ἄγγελος, ὅς ποτε χαλκῷ 
φοινίξας γονόεντα τελεσσιγάμου στάχυν ἥβης 
ῥῖψεν ἀνυμφεύτων φιλοτήσιον ὄγμον ἀρότρων, 
ἄρσενος ἀμητοῖο θαλύσιον, αἱμαλέῃ δὲ 
παιδογόνῳ ῥαθάμιγγι περιρραίνων πτύχα μηροῦ 
θερμὸν ἀλοιητῆρι δέμας θήλυνε σιδήρῳ· 
ὃς τότε διφρεύων Κυβεληίδος ἅρμα θεαίνης 
ἄγγελος ἀσχαλόωντι παρήγορος ἦλθε Λυαίῳ· 
καί μιν ἰδὼν Διόνυσος ἀνέδραμε, μὴ σχεδὸν ἔλθῃ 
Ῥείην πανδαμάτειραν ἄγων ἐπὶ φύλοπιν Ἰνδῶν. 



Nonnus Epic., Dionysiaca 
Book 25, line 328

στήσας δ' ἄγριον ἅρμα, δι' ἄντυγος ἡνία τείνας, 
καὶ ῥοδέης ἀχάρακτα γενειάδος ἄκρα φαείνων 
Βάκχῳ μῦθον ἔλεξε, χέων ὀξεῖαν ἰωήν· 
 “ἀμπελόεις Διόνυσε, Διὸς τέκος, ἔγγονε Ῥείης, 
εἰπέ μοι εἰρομένῳ, πότε νόστιμος εἰς χθόνα Λυδῶν 
ἵξεαι οὐλοκάρηνον ἀιστώσας γένος Ἰνδῶν; 



Nonnus Epic., Dionysiaca 
Book 25, line 335

οὔ πω ληιδίας κυανόχροας ἔδρακε Ῥείη, 
οὔ πω σοὶ μετὰ δῆριν ὀρεσσαύλῳ παρὰ φάτνῃ   
Μυγδονίων ἔσμηξε τεῶν ἱδρῶτα λεόντων 
Πακτωλοῦ παρὰ χεῦμα ῥυηφενές· ἀλλὰ κυδοιμοῦ 
ἄψοφον ἀενάων ἐτέων στροφάλιγγα κυλίνδεις· 
οὔ πω θηροκόμῳ θεομήτορι σύμβολα νίκης 
Ἰνδῴων ἐκόμισσας ἑώια φῦλα λεόντων. 



Nonnus Epic., Dionysiaca 
Book 25, line 341

οὔ πω μῦθος ἔληγε, καὶ ἴαχε Βάκχος ἀγήνωρ· 
 “σχέτλιοί εἰσι θεοί, ζηλήμονες· ἐν πολέμοις μὲν 
εἰς μίαν ἠριγένειαν ἀιστῶσαι πόλιν Ἰνδῶν 
ἔγχεϊ κισσήεντι δυνήσομαι· ἀλλά με νίκης 
μητρυιῆς ἀέκοντα παραπλάζει φθόνος Ἥρης. 



Nonnus Epic., Dionysiaca 
Book 25, line 350

ἀλλὰ βαρυσμαράγων νεφέων κτύπον οὐράνιος Ζεὺς 
σήμερον εὐνήσειε, καὶ αὔριον Ἄρεα δήσω, 
εἰσόκεν εὐπήληκα διατμήξω στάχυν Ἰνδῶν. 



Nonnus Epic., Dionysiaca 
Book 25, line 367

θαρσήεις πολέμιζε τὸ δεύτερον, ὅττι κυδοιμοῦ 
νίκην ὀψιτέλεστον ἐμὴ μαντεύσατο Ῥείη· 
οὐ γὰρ πρὶν πολέμου τέλος ἔσσεται, εἰσόκε χάρμης 
ἕκτον ἀναπλήσωσιν ἔτος τετράζυγες Ὧραι· 
οὕτω γὰρ Διὸς ὄμμα καὶ ἀτρέπτου λίνα Μοίρης 
νεύμασιν Ἡραίοισιν ἐπέτρεπον· ἐσσομένῳ δὲ 
ἑβδομάτῳ λυκάβαντι διαρραίσεις πόλιν Ἰνδῶν. 



Nonnus Epic., Dionysiaca 
Book 26, line 29

οὐ ξεῖνος κατέπεφνεν ἀρειμανέων γένος Ἰνδῶν, 
ἀλλά μιν αὐτὸς ἔπεφνε πατὴρ τεός· ἐν πολέμοις γὰρ 
σοὺς προμάχους φεύγοντας ἰδὼν ἐδάμασσεν Ὑδάσπης. 



Nonnus Epic., Dionysiaca 
Book 26, line 48

πρῶτα μὲν ὡπλίζοντο κυβερνητῆρες ἐνυοῦς, 
Ἀγραῖος Φλογίος τε, συνήλυδες ἡγεμονῆες, 
ἀρτιτελὲς μετὰ σῆμα νεοφθιμένοιο τοκῆος, 
Εὐλαίου δύο τέκνα· συνεστρατόωντο δὲ λαοί, 
ὅσσοι Κῦρα νέμοντο καὶ Ἰνδῴου ποταμοῖο 
Βαίδιον Ὀμβηλοῖο παρὰ πλατὺ βάρβαρον ὕδωρ, 
καὶ Ῥοδόην εὔπυργον, ἀρειμανέων πέδον Ἰνδῶν, 
καὶ κραναὸν Προπάνισον, ὅσοι τ' ἔχον ἄντυγα νήσου 
Γραιάων, ὅθι παῖδες ἐθήμονος ἀντὶ τεκούσης 
ἄρσενα μαζὸν ἔχουσι γαλακτοφόρου γενετῆρος, 
χείλεσιν ἀκροτάτοισιν ὑποκλέπτοντες ἐέρσην· 
οἵ τε Σεσίνδιον αἰπύ, καὶ οἳ λινοερκέι κύκλῳ 
Γάζον ἐπυργώσαντο λινοπλέκτοισι δομαίοις, 
ἀρραγές, εὐποίητον ἐυκλώστοισι θεμέθλοις,   




Nonnus Epic., Dionysiaca 
Book 26, line 68

τοῖς δ' ἐπὶ θαρσήεντες ἐπεστρατόωντο μαχηταί, 
Δάρδαι καὶ Πρασίων στρατιαί, καὶ φῦλα Σαλαγγῶν 
χρυσοφόρων, οἷς πλοῦτος ὁμέστιος, οἷς θέμις αἰεὶ 
χέδροπα καρπὸν ἔδειν βιοτήσιον· ἀντὶ δὲ σίτου 
κεῖνον ἀλετρεύουσι μύλης τροχοειδέι κύκλῳ· 
καὶ σκολιοπλοκάμων Ζαβίων στίχες, οἷσιν ἐχέφρων 
Παλθάνωρ πρόμος ἦεν, ὃς ἔστυγε Δηριαδῆα 
ἤθεσιν εὐσεβέεσσιν ὁμοφρονέων Διονύσῳ· 
τὸν μὲν ἄναξ Διόνυσος ἄγων μετὰ φύλοπιν Ἰνδῶν 
ἀλλοδαπὸν ναετῆρα λυροδμήτῳ πόρε Θήβῃ· 
καὶ Δίρκῃ παρέμιμνε λιπὼν πατρῷον Ὑδάσπην, 
Ἀονίου ποταμοῖο πιὼν Ἰσμήνιον ὕδωρ. 



Nonnus Epic., Dionysiaca 
Book 26, line 78

τοῖς δ' ἐπὶ κυδιόων στρατὸν ἄσπετον ὥπλισε Μορρεὺς 
Τιδνασίδης, γενετῆρι συνέμπορος, ὃς τότε λυγρῷ 
γήραϊ πένθος ἔχων κεκερασμένον ἥψατο χάρμης, 
γηραλέῃ παλάμῃ πολυδαίδαλον ἀσπίδα πάλλων 
καὶ πολιῷ λειμῶνι κατάσκιον ἀνθερεῶνα 
αὐτόματον κήρυκα χρόνου δολιχοῖο τινάσσων, 
υἱὸν ἔτι στενάχων μινυώριον, Ἰνδὸν Ὀρόντην, 
Τίδνασος αἰολόδακρυς· ἄναξ δέ οἱ ἕσπετο Μορρεὺς 
ὄρθιον ἔγχος ἔχων τιμήορον, ὄφρα δαμάσσῃ   
λαὸν ὅλον Βρομίοιο, καὶ ἤθελε μοῦνος ἐρίζειν 
Βάκχῳ γνωτοφόνῳ, καὶ ἀνούτατον υἷα Θυώνης 
οὐτῆσαι μενέαινε κασιγνήτοιο φονῆα. 



Nonnus Epic., Dionysiaca 
Book 26, line 84

καί σφισιν ὡμάρτησε πολυγλώσσων γένος Ἰνδῶν, 
οἵ τ' ἔχον Ἠελίοιο πόλιν, καλλίκτιτον Αἴθρην, 
ἀννεφέλου δαπέδοιο θεμείλιον, οἵ τ' ἔχον ἄμφω, 
Ἀνθηνῆς λασιῶνα καὶ Ὠρυκίης δονακῆα, 
καὶ φλογερὴν Νήσαιαν ἀχειμάντους τε Μελαίνας, 
καὶ πέδον εὐδίνητον ἁλιστεφάνου Παταληνῆς· 
τοῖς ἔπι Δυσσαίων πυκιναὶ στίχες, οἷσι καὶ αὐτῶν 
φρικτὰ δασυστέρνων ἐκορύσσετο φῦλα Σαβείρων, 
τοῖσιν ἐπὶ κραδίῃ λάσιαι τρίχες, ὧν χάριν αἰεὶ 
ψυχῆς θάρσος ἔχουσι καὶ οὐ πτώσσουσιν ἐνυώ. 



Nonnus Epic., Dionysiaca 
Book 26, line 141

           Ἠερίης δὲ θεουδέος ἔργον ἀκούων 
Δηριάδης θάμβησε· περισσονόοιο δὲ κούρης 
εἴκελον εἰδώλῳ γενέτην ἀνελύσατο δεσμῶν· 
φήμη δ' ἀμφιβόητος ἀκούετο, καὶ στρατὸς Ἰνδῶν 
μαζὸν ἀλεξικάκοιο δολοπλόκον ᾔνεσε νύμφης. 



Nonnus Epic., Dionysiaca 
Book 26, line 157

καὶ στρατὸν ἀγκυλότοξον ἀολλίσσας ἐπικούρων 
Ἁβράθοος βραδὺς ἦλθε· νεοτμήτων δὲ κομάων 
αἰδόμενος κεκόρυστο, χόλον καὶ πένθος ἀέξων 
βουκεράου βασιλῆος, ἐπεί νύ οἱ ἄφρονι λύσσῃ 
Δηριάδης ὑπέροπλος ὅλην ἀπεκείρατο χαίτην, 
Ἰνδοῖς πικρὸν ὄνειδος. 



Nonnus Epic., Dionysiaca 
Book 26, line 220

τοῖς ἔπι θωρήσσοντο Σίβαι καὶ λαὸς Ὑδάρκης, 
καὶ στρατὸς ἄλλος ἵκανε πόλιν Καρμῖναν ἐάσσας· 
τῶν ἅμα Κύλλαρος ἦρχε καὶ Ἀστράεις, πρόμος Ἰνδῶν, 
Βρόγγου δίζυγα τέκνα τετιμένα Δηριαδῆι. 



Nonnus Epic., Dionysiaca 
Book 26, line 225

καὶ στόλος ἄλλος ἵκανε τριηκοσίων ἀπὸ νήσων, 
αἵ τε περιστιχόωσιν ἀμοιβάδες ἄλλυδις ἄλλαι 
γείτονες ἀλλήλῃσιν, ὅπῃ περιμήκεϊ πορθμῷ 
δίστομος Ἰνδὸς ἄγων μετανάστιον ἀγκύλον ὕδωρ, 
ἑρπύζων κατὰ βαιὸν ἀπ' Ἰνδῴου δονακῆος 
λοξὸς ὑπὲρ δαπέδοιο παρ' Ἠῴου στόμα πόντου, 
ἔρχεται αὐτοκύλιστος ὑπὲρ λόφον Αἰθιοπῆα· 
ἧχι θερειγενέων ὑδάτων ὑψούμενος ὁλκῷ 
χεύμασιν αὐτογόνοις ἐπὶ πήχεϊ πῆχυν ἀέξει· 
καὶ χθόνα πιαλέην ἀγκάζεται ὑγρὸς ἀκοίτης, 
τέρπων ἰκμαλέοισι φιλήμασι διψάδα νύμφην, 
οἶστρον ἔχων πολύπηχυν ἀμαλλοτόκων ὑμεναίων, 
μέτρῳ ἀμοιβαίῳ παλιναυξέα χεύματα τίκτων 




Nonnus Epic., Dionysiaca 
Book 26, line 235

ἑρπύζων κατὰ βαιὸν ἀπ' Ἰνδῴου δονακῆος 
λοξὸς ὑπὲρ δαπέδοιο παρ' Ἠῴου στόμα πόντου, 
ἔρχεται αὐτοκύλιστος ὑπὲρ λόφον Αἰθιοπῆα· 
ἧχι θερειγενέων ὑδάτων ὑψούμενος ὁλκῷ 
χεύμασιν αὐτογόνοις ἐπὶ πήχεϊ πῆχυν ἀέξει· 
καὶ χθόνα πιαλέην ἀγκάζεται ὑγρὸς ἀκοίτης, 
τέρπων ἰκμαλέοισι φιλήμασι διψάδα νύμφην, 
οἶστρον ἔχων πολύπηχυν ἀμαλλοτόκων ὑμεναίων, 
μέτρῳ ἀμοιβαίῳ παλιναυξέα χεύματα τίκτων 
Νεῖλος ἐν Αἰγύπτῳ καὶ ἑώιος Ἰνδὸς ἀκούων. 



Nonnus Epic., Dionysiaca 
Book 26, line 246

τοῖα μὲν ἑπταπόροιο φατίζεται εἵνεκα Νείλου, 
Ἰνδῴου ποταμοῖο φέρειν γένος. 



Nonnus Epic., Dionysiaca 
Book 26, line 247

                                     οἱ δὲ λιπόντες 
νήσων ἀγκύλα κύκλα καὶ ἕδρανα γείτονος Ἰνδοῦ 
ἄνδρες ἐθωρήσσοντο μαχήμονες, ὧν πρόμος ἀνὴρ 
Ῥίγβασος ἡγεμόνευεν, ἔχων ἴνδαλμα Γιγάντων. 



Nonnus Epic., Dionysiaca 
Book 26, line 281

μαντιπόλον δ' ἐρέεινε θεηγόρον· εἰρομένῳ δὲ   
ἐσσομένων θέσπιζεν ἀφωνήτων στίχα παίδων, 
εἰναλίης ἴνδαλμα λιπογλώσσοιο γενέθλης. 



Nonnus Epic., Dionysiaca 
Book 26, line 304

τοῖς ἔπι θωρήχθησαν, ὅσοι λάχον ἄντυγα † σοιτης, 
μητέρα δενδρήεσσαν ἀμετροβίων ἐλεφάντων, 
οἷς φύσις ὤπασε κύκλα διηκοσίων ἐνιαυτῶν 
ζώειν ἀενάοιο χρόνου πολυκαμπέι νύσσῃ, 
ἠὲ τριηκοσίων· καὶ βόσκεται ἄλλος ἐπ' ἄλλῳ, 
ἐκ ποδὸς ἀκροτάτου μελανόχροος ἄχρι καρήνου, 
γναθμοῖς μηκεδανοῖσιν ἔχων προβλῆτας ὀδόντας 
δίζυγας, ἀμητῆρι τύπῳ γαμψώνυχος ἅρπης,   
θηγαλέῳ τμητῆρι, διαστείχων στίχα δένδρων 
ποσσὶ τανυκνήμοισιν, ἔχων ἴνδαλμα καμήλων . 



Nonnus Epic., Dionysiaca 
Book 26, line 310

                                                  .. 
καὶ λοφιὴν ἐπίκυρτον, ἑῷ πολυχανδέι νώτῳ 
ἐσμὸν ἄγων νήριθμον ἐπασσυτέρων ἐλατήρων, 
δινεύων στατὸν ἴχνος ἀκαμπέι γούνατος ὁλκῷ, 
καὶ τύπον εὐρυμέτωπον ἐχιδναίοιο καρήνου, 
αὐχένα βαιὸν ἔχων κυρτούμενον· εἶχε δὲ λεπτὸν 
ὄμμασιν ἰσοτύποισι συῶν ἴνδαλμα προσώπου, 
ὑψιφανής, περίμετρος· ἑλισσομένου δὲ πορείῃ 
οὔατα μὲν λιπόσαρκα, παρήορα γείτονι κόρσῃ, 
λεπταλέων ἀνέμων ὀλίγῃ ῥιπίζεται αὔρῃ· 
πυκνὰ δὲ μαστίζουσα δέμας νωμήτορι παλμῷ 
λεπτοφυὴς ἐλάχεια τινάσσεται ἄστατος οὐρή. 



Nonnus Epic., Dionysiaca 
Book 26, line 329

τοὺς μὲν ἄναξ Διόνυσος ἄγων μετὰ φύλοπιν Ἰνδῶν 
Καυκασίην παρὰ πέζαν Ἀμαζονίου ποταμοῖο 
εἰς φόβον εὐπήληκας ἀνεπτοίησε γυναῖκας, 
ἠλιβάτων λοφιῇσιν ἐφεδρήσσων ἐλεφάντων. 



Nonnus Epic., Dionysiaca 
Book 26, line 351

πάντων δ' ἡγεμόνευεν ἐς ἄρεα κοίρανος Ἰνδῶν, 
ὃν διερῇ φιλότητι πατὴρ ἔσπειρεν Ὑδάσπης, 
Ἀστρίδος εὐώδινος ὁμιλήσας ὑμεναίοις, 
κούρης Ἠελίοιο. 



Nonnus Epic., Dionysiaca 
Book 27, line 14

                    Φαέθων δὲ πυριτρεφέων δρόμον ἵππων 
ἀενάων ἐτέων φλογόεις ἀνεσείρασε ποιμήν, 
γείτονος εἰσαΐων κορυθαιόλον ἄρεος ἠχώ, 
καὶ στρατὸν αἰχμάζειν προκαλίζετο μάρτυρι πυρσῷ, 
θερμὸν ἀκοντίζων ῥοδόεν βέλος· ἀμφὶ δὲ γαίῃ 
αἱμαλέης ξένον ὄμβρον ἀπ' ἰκμάδος ὑέτιος Ζεὺς 
οὐρανόθεν κατέχευε, φόνου πρωτάγγελον Ἰνδῶν. 



Nonnus Epic., Dionysiaca 
Book 27, line 17

καὶ φονίαις λιβάδεσσιν ἐνυαλίου νιφετοῖο 
δίψια κυανέης ἐρυθαίνετο νῶτα κονίης 
Ἰνδῴου δαπέδοιο· νεοσμήκτου δὲ σιδήρου 
Ἠελίου σελάγιζε βολαῖς ἀντίρροπος αἴγλη. 



Nonnus Epic., Dionysiaca 
Book 27, line 19

φαινομένας δὲ φάλαγγας ἐπὶ κλόνον ὥπλισεν Ἰνδῶν 
Δηριάδης ὑπέροπλος, ἐποτρύνων δὲ μαχητὰς 
μῦθον ἀπειλητῆρος ἀνήρυγεν ἀνθερεῶνος· 
 “δμῶες ἐμοί, μάρνασθε, πεποιθότες ἠθάδι Νίκῃ, 
καὶ θρασὺν ὃν καλέουσι κερασφόρον υἷα Θυώνης 
λάτριν ἰσοκραίροιο τελέσσατε Δηριαδῆος. 



Nonnus Epic., Dionysiaca 
Book 27, line 35

καί τις ἀνὴρ Φρυγίηθεν ὁμόστολος οἴνοπι Βάκχῳ 
Ἰνδῴου ποταμοῖο δέμας λούσειε ῥεέθροις, 
ἀντὶ δὲ Σαγγαρίου καλέσει πατρῷον Ὑδάσπην· 
ἄλλος ἀνὴρ Ἀλύβηθεν ὁμαρτήσας Διονύσῳ 
ἐνθάδε θητεύσειε, καὶ ἀργυρέου ποταμοῖο 
χεύματα καλλείψας πιέτω χρυσαυγέα Γάγγην. 



Nonnus Epic., Dionysiaca 
Book 27, line 46

χάζεό μοι, Διόνυσε, φυγὼν δόρυ Δηριαδῆος· 
ἔστι καὶ ἐνθάδε πόντος ἀπείριτος· ἀλλὰ θαλάσσης 
Ἀρραβίης μετὰ κῦμα καὶ ἡμετέρη σε δεχέσθω· 
εὐρύτερος βυθὸς οὗτος ἐρεύγεται ἄγριον ὕδωρ,   
καὶ Σατύρους καὶ Βάκχον ἐπάρκιός ἐστι καλύψαι 
καὶ στίχα Βασσαρίδων· οὐ μείλιχος ἐνθάδε Νηρεύς, 
οὐ Θέτις Ἰνδῴη σε δεδέξεται, οὐδέ σε κόλπῳ 
ξεινοδόκον μετὰ κῦμα πάλιν φεύγοντα σαώσει, 
αἰδομένη βαρύδουπον ἐμὸν πατρῷον Ὑδάσπην. 



Nonnus Epic., Dionysiaca 
Book 27, line 82

καὶ πολέας Κρονίδαο δεδουπότας υἷας ἀκούω· 
Δάρδανος ἐκ Διὸς ἔσκε καὶ ὤλετο, καὶ θάνε Μίνως, 
οὐδέ μιν ἐρρύσαντο Διὸς ταυρώπιδες εὐναί· 
εἰ δὲ θεμιστεύει καὶ ἐν Ἄιδι, τίς φθόνος Ἰνδοῖς, 
Αἰακὸς εἰ φθιμένοισι δικάζεται; 



Nonnus Epic., Dionysiaca 
Book 27, line 90

μὴ χθονίους Κύκλωπας ὀλέσσατε· καὶ γὰρ ἐκείνων 
δεύομαι· Ἰνδῴῳ δὲ παρήμενος ἐσχαρεῶνι   
Βρόντης μὲν βαρύδουπον ἐμοὶ σάλπιγγα τελέσσῃ 
βρονταίοις πατάγοισιν ἰσόκτυπον, ὄφρα κεν εἴην 
Ζεὺς χθόνιος, Στερόπης δὲ νέην ἀντίρροπον αἴγλην 
ἀστεροπῇ τεύξειε καὶ ἐνθάδε· καί μιν ἑλίξω 
μαρνάμενος Σατύροισιν, ἵνα φρένα μᾶλλον ἀμύξῃ 
Δηριάδην κτυπέοντα καὶ ἀστράπτοντα δοκεύων 
ζηλήμων Κρονίδης, πεφοβημένος ὄρχαμον Ἰνδῶν 
ὑψιγόνου φλογόεντος ἀκοντιστῆρα κεραυνοῦ. 



Nonnus Epic., Dionysiaca 
Book 27, line 116

καὶ ναέτην † βαρύδεσμον ἀπειρώδινος Ἀθήνης 
Ἡφαίστου πυρόεντος ἀπόσπορον αἴθοπι πυρσῷ   
φλέξατε, τὸν καλέουσιν Ἐρεχθέα· καὶ γὰρ ἐκείνου 
αἷμα φέρει περίπυστον Ἐρεχθέος, ὅν ποτε μαζῷ 
παρθενικὴ φυγόδεμνος ἀνέτρεφε Παλλὰς ἀμήτωρ, 
λάθριον ἀγρύπνῳ πεφυλαγμένον αἴθοπι λύχνῳ· 
μιμνέτω Ἰνδῴῃ κεκαλυμμένος αἴθοπι κίστῃ, 
καὶ κενεοῦ ζοφόεντος ἐν ἕρκεϊ παρθενεῶνος. 



Nonnus Epic., Dionysiaca 
Book 27, line 136

ὣς φαμένου βασιλῆος ἐπὶ κλόνον ἤιον Ἰνδοί, 
οἱ μὲν ὑπὲρ νώτοιο σιδηροφόρων ἐλεφάντων, 
οἱ δὲ συνεστρατόωντο θυελλοπόδων ὑπὲρ ἵππων. 



Nonnus Epic., Dionysiaca 
Book 27, line 144

καὶ μόθον ἐστήσαντο παρὰ στόμα γείτονος Ἰνδοῦ, 
ἐς πεδίον προθέοντες. 



Nonnus Epic., Dionysiaca 
Book 27, line 155

καὶ πισύρων ἀνέμων † φλογερῆς ἀντώπιον Ἠοῦς † 
τέτραχα τεμνομένην στρατιὴν ἐστήσατο Βάκχων· 
πρώτην μὲν βαθύδενδρα παρὰ σφυρὰ κυκλάδος Ἄρκτου, 
ἧχι πολυσπερέων ποταμῶν πεφορημένον ὁλκῷ 
Καυκασίου σκοπέλοιο Διιπετὲς ἔρχεται ὕδωρ·   
τὴν ἑτέρην δὲ φάλαγγα συνήρμοσεν, ὁππόθι γαίης 
μεσσατίης στεφανηδὸν ἐς ἑσπέριον κλίμα νεύων 
δίστομος οὐρεσίφοιτος ἑὸν ῥόον Ἰνδὸς ἑλίσσει, 
κύμασιν ἀμφίζωστον ἐπιστέψας Παταληνήν, 
τὴν αὐτὴν παρὰ πέζαν, ὅπῃ περιμήκεϊ πορθμῷ 
χεῦμα παλινδίνητον ἄγει βαρύδουπος Ὑδάσπης· 
καὶ τριτάτην κόσμησεν, ὅπῃ νοτίῳ παρὰ κόλπῳ 
κύματι πορφύροντι μεσημβριὰς ἕλκεται ἅλμη· 
καὶ στρατιὴν εὔχαλκον ἄναξ ἔστησε τετάρτην 
ἀντολίης ὑπὸ πέζαν, ὅθεν δονακῆα διαίνων 
στέλλεται εὐόδμοισι κατάρρυτος ὕδασι Γάγγης. 



Nonnus Epic., Dionysiaca 
Book 27, line 183

εἰ μὲν ἐμοὶ γόνυ δοῦλον ὑποκλίνειεν Ὑδάσπης 
μηδὲ πάλιν Βάκχοισι παλίγκοτον οἶδμα κορύσσῃ, 
ἔσσομαι εὐάντητος, ὅλον δέ οἱ ἀγλαὸν ὕδωρ   
χεύμασι ληναίοισιν ἐς εὔιον οἶνον ἀμείψω, 
τεύχων λαρὰ ῥέεθρα, καὶ ἀγριάδος λόφον ὕλης 
μιτρώσω πετάλοισι καὶ ἀμπελόεντα τελέσσω· 
εἰ δὲ πάλιν προχοῇσιν ἀλεξικάκοισιν ἀρήξει 
Ἰνδοῖς κτεινομένοισι καὶ υἱέι Δηριαδῆι, 
ἀνδροφυὴς κερόεσσαν ἔχων ποταμηίδα μορφήν, 
χεῦμα γεφυρώσαντες ὑπερφιάλου ποταμοῖο 
ἴχνεσιν ἀβρέκτοισιν ὁδεύσατε δίψιον ὕδωρ, 
καὶ γυμνῇ ψαμάθῳ πατέων αὐχμηρὸν Ὑδάσπην 
πεζὸς ὄνυξ εὔιππος ἐπιξύσειε κονίην. 



Nonnus Epic., Dionysiaca 
Book 27, line 189

εἰ δὲ πολυπτοίητος ἀρειμανέων πρόμος Ἰνδῶν 
αἰθερίου Φαέθοντος ἀπόσπορός ἐστι γενέθλης 
καὶ Φαέθων πυρόεσσαν ἐμοὶ στήσειεν ἐνυώ, 
θυγατέρος κερόεσσαν ἑῆς ὠδῖνα γεραίρων, 
γνωτὸν ἐμοῦ Κρονίδαο πάλιν Φαεθοντίδι χάρμῃ 
πόντιον ὑδατόεντα πυρὸς σβεστῆρα κορύσσω· 
Θρινακίην δ' ἐπὶ νῆσον ἐλεύσομαι, ὁππόθι ποῖμναι 
καὶ βόες αἰθερίοιο πυραυγέος ἡνιοχῆος, 
Ἠελίου δὲ θύγατρα, δορικτήτην ἅτε κούρην, 
Λαμπετίην ἀέκουσαν ἐπὶ ζυγὰ δούλια σύρω, 




Nonnus Epic., Dionysiaca 
Book 27, line 205

σπεύσατέ μοι καὶ κύκλα μελαρρίνοιο προσώπου 
Ἰνδῶν ληιδίων λευκαίνετε μύστιδι γύψῳ, 
καὶ θρασὺν ἀμπελόεντι περιπλεχθέντα κορύμβῳ . 



Nonnus Epic., Dionysiaca 
Book 27, line 209

                                                    .. 
νεβρίδα χαλκοχίτωνι καθάψατε Δηριαδῆι· 
καὶ Βρομίῳ γόνυ δοῦλον ὑποκλίνων μετὰ νίκην 
Ἰνδὸς ἄναξ ῥίψειεν ἑὸν θώρηκα θυέλλαις, 
κρείσσονι λαχνήεντι δέμας θώρηκι καλύπτων, 
καὶ πόδα πορφυρέοισι περισφίγξειε κοθόρνοις 
ἀργυρέας ἀνέμοισιν ἑὰς κνημῖδας ἐάσσας, 
καὶ μετὰ φοίνια τόξα καὶ ἠθάδος ἔργα κυδοιμοῦ 
ὄργια νυκτιχόρευτα διδασκέσθω Διονύσου, 
βάρβαρα δινεύων ἐπιλήνια βόστρυχα χαίτης. 



Nonnus Epic., Dionysiaca 
Book 27, line 218

δυσμενέων δὲ κάρηνα κομίσσατε σύμβολα νίκης 
Τμῶλον ἐς ἠνεμόεντα, πεπαρμένα μάρτυρι θύρσῳ· 
πολλὰς δ' ἐκ πολέμοιο μεταστήσω στίχας Ἰνδῶν 
ζωγρήσας μετ' ἄρηα, παρὰ προπύλαια δὲ Λυδῶν 
πήξω μαινομένοιο κεράατα Δηριαδῆος. 



Nonnus Epic., Dionysiaca 
Book 27, line 240

καί τις ἐπ' ἀντιβίοισι μεμηνότα τίγριν ἱμάσσων 
δίφρα διεπτοίησεν ὁμοζυγέων ἐλεφάντων· 
καὶ πολιὸς κεκόρυστο Μάρων ἑλικώδεϊ θαλλῷ, 
ἡμερίδων ὄρπηκι διασχίζων δέμας Ἰνδῶν 
μαρναμένων. 



Nonnus Epic., Dionysiaca 
Book 27, line 313

ὦ γένος ἀλλοπρόσαλλον Ὀλύμπιον· ἆ μέγα θαῦμα· 
ξείνῳ Δηριαδῆι παρίσταται Ἀργολὶς Ἥρη, 
Κεκροπίδας δὲ φάλαγγας ἀναίνεται Ἀτθὶς Ἀθήνη, 
μητρὶ δὲ πιστὰ φέρων, ἐμὸν υἱέα Βάκχον ἐάσσας 
καὶ στρατιὴν Θρήισσαν ἐφεσπομένην Διονύσῳ, 
ῥύεται Ἰνδὸν ὅμιλον ἐμὸς Θρηίκιος Ἄρης. 



Nonnus Epic., Dionysiaca 
Book 28, line 7

καὶ στρατιὴ κεκόρυστο πολύτροπος εἰς μόθον Ἰνδῶν 
σπερχομένων ἀγεληδόν· ὁ μὲν ταμεσίχροϊ κισσῷ 
κραιπνὸς ἐς ὑσμίνην πολυδαίδαλα δίφρα νομεύων 
πορδαλίων ἐπέβαινεν, ὁ δὲ φρίσσοντι λεπάδνῳ 
ζεῦξεν Ἐρυθραίων ὀρεσίδρομον ἅρμα λεόντων 
καὶ βλοσυρὴν ἴθυνε συνωρίδα, κυανέας δὲ 
ἄλλος ἐριπτοίητος ἀκοντίζων στίχας Ἰνδῶν 
ἀστεμφὴς ἀχάλινον ἐτέρπετο ταῦρον ἱμάσσων, 
καί τις ἀναΐξας Κυβεληίδος εἰς ῥάχιν ἄρκτου 
ἔχραε δυσμενέεσσι, καὶ οἴνοπα θύρσον ἑλίσσων 




Nonnus Epic., Dionysiaca 
Book 28, line 27

καί τις ὀρεσσινόμων Σατύρων, ἅτε πῶλον ἐλαύνων, 
ποσσὶ διχαζομένοισιν ὑπὲρ ῥάχιν ἧστο λεαίνης. 
 Ἰνδοὶ δ' ἀνταλάλαζον, ἀολλίζων δὲ μαχητὰς 
βάρβαρος ἐσμαράγησεν ἀγέστρατος αὐλὸς ἐνυοῦς· 
στέμματα μὲν κορύθεσσιν, ἐπέκτυπε δ' αἰγίδι θώρηξ, 
ἔγχεσι θύρσος † ἔθυσε, καὶ ἰσάζοντο κοθόρνοις 
ἀντίτυποι κνημῖδες. 



Nonnus Epic., Dionysiaca 
Book 28, line 69

αὐτίκα δ' ἐκ κολεοῖο Κορύμβασος ἆορ ἐρύσσας 
αὐχένα Δεξιόχοιο κατεπρήνιξε μαχαίρῃ· 
καὶ ταχὺς ἀσπαίροντι θορὼν περιδέδρομε νεκρῷ 
οἰστρομανὴς Κλυτίος, πρυλέων πρόμος· ὑψιλόφου δὲ 
κραιπνὸς ἐριπτοίητος ἀκόντισε Δηριαδῆος· 
ἀλλὰ δόρυ προμάχοιο παρακλιδὸν ἔτραπεν Ἥρη, 
καὶ Κλυτίῳ κοτέουσα καὶ Ἰνδοφόνῳ Διονύσῳ· 
ἔμπης δ' οὐκ ἀφάμαρτε ταχὺς πρόμος· ἀλλὰ τορήσας 
θηρὸς ἀμαιμακέτοιο πελώριον ἀνθερεῶνα 
ὀρθοπόδην ἐλέφαντα κατέκτανε Δηριαδῆος· 
καὶ μογέων ὀδύνῃσιν ὅλην ἐτίναξεν ἀπήνην 
αὐχένι κυανέῳ περιδέξιος ἠλίβατος θήρ· 
καὶ γένυν αἰθύσσων σκολιὴν προβλῆτα προσώπου 
αἱμοβαφῆ ζυγίων ἀνεσείρασε δεσμὰ λεπάδνων· 
ἀλλὰ πολυκλήιστον ὑπὸ ζυγὸν ἄορι κάμψας 
αὐχενίων ἀνέκοψεν ὁμόζυγον ὁλκὸν ἱμάντων 




Nonnus Epic., Dionysiaca 
Book 28, line 86

ὑμέας εἰς Φρυγίην ληίσσομαι, ἄστεα δ' Ἰνδῶν 
δῃώσει δόρυ τοῦτο, καὶ Ἰνδοφόνον μετὰ νίκην 
Δηριάδην θεράποντα Διωνύσοιο τελέσσω· 
παρθενικὴ δ' ἀνάεδνος ἑὴν λύσειε κορείην, 
δεχνυμένη Σατύροιο δασυστέρνους ὑμεναίους, 
Ἰνδὴ Μυγδονίοιο μιαινομένη σχεδὸν Ἕρμου. 



Nonnus Epic., Dionysiaca 
Book 28, line 97

καὶ νέκυν ὀρχηστῆρα παλινδίνητον ἐάσσας 
Σιληνοὺς ἐφόβησε Κορύμβασος, ἔξοχος Ἰνδῶν, 
ἔξοχος ἠνορέην μετὰ Μορρέα καὶ βασιλῆα. 



Nonnus Epic., Dionysiaca 
Book 28, line 176

καὶ βριαροὶ Κύκλωπες ἐκυκλώσαντο μαχητάς, 
Ζηνὸς ἀοσσητῆρες· ὀμιχλήεντι δὲ λαῷ 
Ἀργίλιπος σελάγιζε φεραυγέα δαλὸν ἀείρων, 
καὶ χθονίῳ κεκόρυστο πυριγλώχινι κεραυνῷ 
μαρνάμενος δαΐδεσσι· καὶ ἔτρεμον αἴθοπες Ἰνδοὶ 
οὐρανίῳ πρηστῆρι τεθηπότες ἀντίτυπον πῦρ· 
καὶ πυρόεις πρόμος ἦεν· ἐπ' ἀντιβίων δὲ καρήνοις 
Γηγενέος σπινθῆρες ἐτοξεύοντο κεραυνοῦ· 
καὶ μελίας νίκησε καὶ ἄσπετα φάσγανα Κύκλωψ, 
σείων θερμὰ βέλεμνα καὶ αἰθαλόεσσαν ἀκωκήν, 
δαλὸν ἔχων ἅτε τόξα· καὶ ἄσπετον ἄλλον ἐπ' ἄλλῳ 
Ἰνδὸν ὀιστευτῆρι κατέφλεγεν ἀνέρα πυρσῷ. 



Nonnus Epic., Dionysiaca 
Book 28, line 183

Ζηνὸς ἀοσσητῆρες· ὀμιχλήεντι δὲ λαῷ 
Ἀργίλιπος σελάγιζε φεραυγέα δαλὸν ἀείρων, 
καὶ χθονίῳ κεκόρυστο πυριγλώχινι κεραυνῷ 
μαρνάμενος δαΐδεσσι· καὶ ἔτρεμον αἴθοπες Ἰνδοὶ 
οὐρανίῳ πρηστῆρι τεθηπότες ἀντίτυπον πῦρ· 
καὶ πυρόεις πρόμος ἦεν· ἐπ' ἀντιβίων δὲ καρήνοις 
Γηγενέος σπινθῆρες ἐτοξεύοντο κεραυνοῦ· 
καὶ μελίας νίκησε καὶ ἄσπετα φάσγανα Κύκλωψ, 
σείων θερμὰ βέλεμνα καὶ αἰθαλόεσσαν ἀκωκήν, 
δαλὸν ἔχων ἅτε τόξα· καὶ ἄσπετον ἄλλον ἐπ' ἄλλῳ 
Ἰνδὸν ὀιστευτῆρι κατέφλεγεν ἀνέρα πυρσῷ. 



Nonnus Epic., Dionysiaca 
Book 28, line 221

ἀμφὶ δέ μιν προχυθέντες ἐς ἅρματα κούφισαν Ἰνδοί, 
δειδιότες Κύκλωπα δυσειδέα, μή τινι ῥιπῇ 
ὑψιτενῆ πάλιν ἄλλον ἑλὼν πρηῶνα κολώνης 
τρηχαλέῳ βασιλῆα κατακτείνειε βελέμνῳ, 
μῆκος ἔχων ἰσόμετρον ἀερσιλόφου Πολυφήμου. 



Nonnus Epic., Dionysiaca 
Book 28, line 229

καὶ βλοσυροῦ προμάχοιο μέσῳ σελάγιζε μετώπῳ 
μαρμαρυγὴ τροχόεσσα μονογλήνοιο προσώπου· 
καὶ βλοσυροῦ Κύκλωπος ὑποπτήσσοντες ὀπωπὴν 
θαμβαλέῳ δεδόνηντο φόβῳ κυανόχροες Ἰνδοί, 
οὐρανόθεν δοκέοντες Ὀλυμπιὰς ὅττι Σελήνη 
Γηγενέος Κύκλωπος ἐναντέλλουσα προσώπῳ 
πλησιφαὴς ἤστραπτε, προασπίζουσα Λυαίου. 



Nonnus Epic., Dionysiaca 
Book 28, line 244

                                             ... 
Εὐρύαλος κεκόρυστο· διατμήξας δὲ κυδοιμῷ 
ἐκ πεδίου φεύγοντα πολὺν στρατὸν ἄχρι θαλάσσης, 
κόλπον ἐς ἰχθυόεντα περικλείων στίχας Ἰνδῶν, 
δυσμενέας νίκησεν ἀκοντοφόρου διὰ πόντου, 
ὄρθιον εἰκοσίπηχυ δι' ὕδατος ἆορ ἑλίσσων· 
καὶ δολιχῷ βουπλῆγι ταμὼν ἁλιγείτονα πέτρην 
ῥῖψεν ἐπ' ἀντιβίοισιν· ἀτυμβεύτοιο δὲ πολλοὶ 
διχθαδίης ἐνόησαν ἁλιβρέκτου λίνα Μοίρης, 
ἄρεϊ κυματόεντι καὶ ὀκριόεντι βελέμνῳ. 



Nonnus Epic., Dionysiaca 
Book 28, line 261

καί μιν ἰδὼν Φλογίος κταμένων τιμήορος Ἰνδῶν 
τόξον ἑὸν κύκλωσε, καὶ ἠνεμόεν βέλος ἕλκων 
μεσσοφανῆ πτερόεντι βαλεῖν ἤμελλε βελέμνῳ . 



Nonnus Epic., Dionysiaca 
Book 28, line 299

εἱλιπόδην ἔστησε Μίμας εὔρυθμον ἐνυώ, 
καὶ στρατὸν ἐπτοίησε, χοροίτυπον ἆορ ἑλίσσων, 
σκαρθμὸν ἔχων ἀγέλαστον ἐνόπλιον ἴδμονι ταρσῷ,   
οἷον ὅτε Κρονίοισιν ἐπ' οὔασι δοῦπον ἐγείρων 
Πύρριχος Ἰδαίοισι σάκος ξιφέεσσιν ἀράσσων 
ψευδομένης ἀλάλαζε μέλος μενεδήιον Ἠχοῦς, 
Ζηνὸς ὑποκλέπτων παλιναυξέος ἔγκρυφον ἥβην· 
τοῖον ἔχων μιμηλὸν ἐνόπλιον ἅλμα χορείης 
χαλκοχίτων ἐλέλιζε Μίμας ἀνεμώδεα λόγχην· 
τέμνων δ' ἐχθρὰ κάρηνα, σιδήρεα λήια χάρμης, 
Ἰνδοφόνοις πελέκεσσι καὶ ἀμφιπλῆγι μαχαίρῃ 
δυσμενέων ἐτίταινε θαλύσια μάρτυρι Βάκχῳ, 
ἀντὶ θυηπολίης βοέης καὶ ἐθήμονος οἴνου 
λοιβὴν αἱματόεσσαν ἐπισπένδων Διονύσῳ. 



Nonnus Epic., Dionysiaca 
Book 28, line 305

ὀξυφαὴς δ' Ἰδαῖος ἐδύσατο κῶμον Ἐνυοῦς, 
ὀρχηστὴρ πολέμοιο πολύτροπον ἴχνος ἑλίσσων, 
ἄσχετος Ἰνδοφόνοιο μόθου δεδονημένος οἴστρῳ. 



Nonnus Epic., Dionysiaca 
Book 29, line 1

Ἥρη δ' ὡς ἐνόησε δαϊζομένων στίχας Ἰνδῶν, 
δύσμαχον ἔμβαλε θάρσος ἀγήνορι Δηριαδῆι. 



Nonnus Epic., Dionysiaca 
Book 29, line 9

καὶ θρασὺς ἔπλετο μᾶλλον· ὁμηγερέες δὲ καὶ αὐτοὶ 
κεκλομένου βασιλῆος ἐπὶ κλόνον ἔρρεον Ἰνδοί. 



Nonnus Epic., Dionysiaca 
Book 29, line 17

εὐχαίτης δ' Ὑμέναιος ἐμάρνατο φάσγανα σείων, 
Θεσσαλικῆς ἀκίχητος ὑπὲρ ῥάχιν ἥμενος ἵππου, 
Ἰνδοὺς κυανέους ῥοδοειδέι χειρὶ δαΐζων· 
ἀγλαΐῃ δ' ἤστραπτεν· ἴδοις δέ μιν εἰς μέσον Ἰνδῶν 
Φωσφόρον αἰγλήεντα δυσειδέι σύνδρομον ὄρφνῃ·   
καὶ δηίους ἐφόβησεν, ἐπεί νύ οἱ εἵνεκα μορφῆς 
μαρναμένῳ Διόνυσος ἐνέπνεεν ἔνθεον ἀλκήν. 



Nonnus Epic., Dionysiaca 
Book 29, line 25

εἴ ποτε πῶλον ἔλαυνεν ἀπόσσυτον εἰς μόθον Ἰνδῶν, 
δαιδαλέων Διόνυσος ἐμάστιεν αὐχένα θηρῶν, 
ἵππῳ δ' ἅρμα πέλαζε παρ' ἡβητῆρι θαμίζων, 
κοῦρον ἔχων, ἅτε Φοῖβος Ἀτύμνιον· ἵστατο δ' αἰεὶ 
ἀγχιφανής, ἐρόεις δὲ καὶ ἄλκιμος εἰν ἑνὶ θεσμῷ 
ἠιθέῳ μενέαινε φανήμεναι· ἐν δὲ κυδοιμοῖς 
καὶ νεφέων ἔψαυε συναιχμάζων Ὑμεναίῳ. 



Nonnus Epic., Dionysiaca 
Book 29, line 50

ὣς φαμένου Βρομίοιο πολὺ πλέον ἥψατο χάρμης 
ἱμερόεις Ὑμέναιος ἑκηβόλος, ᾧ ἔπι χαίρων 
οἰστρήεις Διόνυσος ἐδύσατο μᾶλλον ἐνυὼ 
καὶ ζοφερὴν προθέλυμνον ὅλην ἐφόβησε γενέθλην· 
καί τις ἰδὼν Διόνυσον ἀφειδέι λαίλαπι χάρμης 
Ἰνδῴων ἀκόρητον ὀιστευτῆρα καρήνων 
τοῖον ἔπος κατέλεξε φιλοκτεάνῳ Μελανῆι· 
 “τοξότα, πῇ σέο τόξα καὶ ἠνεμόεντες ὀιστοί; 



Nonnus Epic., Dionysiaca 
Book 29, line 125

                                           εἰ θέμις εἰπεῖν, 
Ἥρη δερκομένη ζηλήμονι Βάκχον ὀπωπῇ 
καὶ νέον ἀμητῆρα μελαρρίνοιο γενέθλης, 
ἠιθέῳ φθονέουσα καὶ ἱμείροντι Λυαίῳ 
ὥπλισε θοῦρον Ἄρηα βαλεῖν Ὑμέναιον ὀιστῷ, 
Ἰνδῴην μεθέποντα νόθην ἄγνωστον ὀπωπήν, 
ὄφρα νόον δυσέρωτος ἀνιήσειε Λυαίου. 



Nonnus Epic., Dionysiaca 
Book 29, line 178

οὐδὲ μάχης Διόνυσος ἐλώφεεν· ἀλλὰ τορήσας 
μεσσοπαγῆ κούφιζε πεπαρμένον ἀνέρα θύρσῳ 
ὄρθιον ὑψιπότητον, ἐν ἠερίῃ δὲ κελεύθῳ 
Ἰνδὸν ἐλαφρίζων ζηλήμονι δείκνυεν Ἥρῃ. 



Nonnus Epic., Dionysiaca 
Book 29, line 222

                       ὀρεσσαύλων δὲ νομήων 
Ἰνδῴη δεδάικτο γονὴ Κουρῆτι σιδήρῳ· 
καί τις ἀνὴρ προκάρηνος ἐπωλίσθησε κονίῃ, 
εἰσαΐων μύκημα βαρυγδούποιο βοείης. 



Nonnus Epic., Dionysiaca 
Book 29, line 236

ἔνθα μέλος πλέξασα καὶ Ἄρεϊ καὶ Διονύσῳ 
Εὐπετάλη κεκόρυστο, φιλοσταφύλῳ δὲ πετήλῳ 
κέντορα κισσὸν ἔπεμπεν ἀλοιητῆρα σιδήρου, 
Ἰνδῴην δρυόεντι γονὴν ὀλέκουσα κορύμβῳ. 



Nonnus Epic., Dionysiaca 
Book 29, line 241

καὶ δηίων κλονέουσα νέφος ῥηξήνορι θύρσῳ   
Στησιχόρη φιλόβοτρυς ἐπεσκίρτησε κυδοιμῷ, 
κύμβαλα δινεύουσα βαρύβρομα δίζυγι χαλκῷ· 
οὐ τόσον Ἡρακλέης Στυμφαλίδας ἤλασε βόμβῳ 
χαλκὸν ἔχων βαρύδουπον, ὅσον στρατὸν ἤλασεν Ἰνδῶν 
Στησιχόρη κτυπέουσα χοροῦ πολεμήιον ἠχώ. 



Nonnus Epic., Dionysiaca 
Book 29, line 279

καί τις ἀμερσινόοιο κατάσχετος ἅλματι λύσσης   
Βασσαρὶς Ἰνδὸν ἄρηα μετέστιχε θυιὰς ἐνυώ, 
ἀμφὶ σέ, Λύδιε δαῖμον· ἀπὸ πλοκάμοιο δὲ Βάκχης 
ἀφλεγέος σελάγιζε κατ' αὐχένος αὐτόματον πῦρ. 



Nonnus Epic., Dionysiaca 
Book 29, line 296

ἀλλ' ὅτε δὴ πόρον ἷξον, ὅπῃ πεφορημένος ὁλκῷ 
λευκὸν ὕδωρ μεθύοντι ῥόῳ φοίνιξεν Ὑδάσπης, 
δὴ τότε Βάκχος ἄυσε βαρυσμαράγων ἀπὸ λαιμῶν, 
ὁππόσον ἐννεάχιλος ἐπέβρεμεν ἐσμὸς ἐνυοῦς 
φρικτὸν ὁμογλώσσων στομάτων θρόον· ἀσταθέες δὲ 
ξανθὸν ἀλυσκάζοντες ἐπὶ ῥόον ὤκλασαν Ἰνδοί, 
ἄλλοι δ' ἐν πεδίῳ· στρατιὴ δ' ἐμερίζετο Βάκχου, 
δυσμενέας κτείνουσα καὶ ἐν δαπέδῳ καὶ Ὑδάσπῃ,   
δίψῃ καρχαλέῃ κεκαφηότας, ὁππότε γαίης 
ἠὼς μέσσον ἀνέσχε καὶ ἔτρεμε θερμὸς ὁδίτης 
αἴθοπος Ἠελίοιο μεσημβρίζουσαν ἱμάσθλην. 



Nonnus Epic., Dionysiaca 
Book 29, line 302

καὶ θεὸς ἀμπελόεις προκαλίζετο κοίρανον Ἰνδῶν, 
μῦθον ἀπειλητῆρα χέων λυσσώδεϊ λαιμῷ· 
 “τίς φόβος; 



Nonnus Epic., Dionysiaca 
Book 29, line 304

                εἰ ποταμοῖο φέρει γένος ὄρχαμος Ἰνδῶν, 
οὐρανόθεν λάχον αἷμα· χερειότερος δὲ Λυαίου 
Δηριάδης ὑπέροπλος, ὅσον Διός ἐστιν Ὑδάσπης. 



Nonnus Epic., Dionysiaca 
Book 29, line 324

τοῖσι δὲ μαρναμένοισιν ἐπήλυθεν Ἕσπερος ἀστήρ, 
λύων Ἰνδοφόνοιο θεμείλια δηιοτῆτος. 



Nonnus Epic., Dionysiaca 
Book 30, line 9

ὡς ἴδε Βάκχος Ἄρηα λελοιπότα φύλοπιν Ἰνδῶν· 
ἄλλῳ δ' ἄλλος ἔριζε. 



Nonnus Epic., Dionysiaca 
Book 30, line 229

καὶ φθονεροὶ Τελχῖνες ἐπεστρατόωντο κυδοιμῷ, 
ὃς μὲν ἔχων ἐλάτην περιμήκετον, ὃς δὲ κρανείης 
θάμνον ὅλον πρόρριζον, ὁ δὲ πρηῶνος ἀράξας 
ἄκρον ἀπηλοίησε καὶ ἐς μόθον ἤιεν Ἰνδῶν 
λᾶαν ἀκοντιστῆρα μεμηνότι πήχεϊ σείων. 



Nonnus Epic., Dionysiaca 
Book 30, line 235

Ἥρη δ' ἀλλοπρόσαλλος ἐπιβρίζουσα Λυαίῳ 
δῶκε μένος καὶ θάρσος ἀγήνορι Δηριαδῆι, 
καί οἱ ἀριστεύοντι σελασφόρον ὤπασεν αἴγλην 
εἰς φόβον ἀντιβίοισι· κορυσσομένου δὲ φορῆος 
ἀσπίδος Ἰνδῴης ἀμαρύσσετο φοίνιος αἴγλη, 
καὶ κυνέης σελάγιζεν ὑπὲρ λόφον ἁλλομένη φλόξ. 



Nonnus Epic., Dionysiaca 
Book 30, line 243

καὶ τότε θαρσήεντες ἐπὶ κλόνον ἤιον Ἰνδοί, 
ὑσμίνην Βρομίοιο λελοιπότος· εἰσορόων δὲ 
Δηριάδης ἐδάϊζεν ἐπασσυτέρων στίχα Βάκχων 
ἐγχείην ἑκάτερθε παλινδίνητον ἑλίσσων. 



Nonnus Epic., Dionysiaca 
Book 30, line 261

ποῖον ἴδον κατὰ δῆριν ὀλωλότα κοίρανον Ἰνδῶν; 



Nonnus Epic., Dionysiaca 
Book 30, line 279

Αἰακὸς ἀπτοίητος ὁμοίιος οὐ πέλε Βάκχῳ, 
οὐ φύγε Δηριάδην, οὐκ ἔτρεμε φύλοπιν Ἰνδῶν. 



Nonnus Epic., Dionysiaca 
Book 30, line 285

ποίην Ὀρσιβόην ληίσσαο δεσπότιν Ἰνδῶν; 



Nonnus Epic., Dionysiaca 
Book 30, line 288

ἱλήκοι Διὸς εὖχος, ἀδελφεὸν οὔ σε καλέσσω 
Δηριάδην φεύγοντα καὶ ἀπτολέμων γένος Ἰνδῶν. 



Nonnus Epic., Dionysiaca 
Book 30, line 325

Εὔιος ἐπτοίησεν ὅλον στρατὸν Οὐατοκοίτην· 
καὶ Λύγον αἱματόεντος ἀπεστυφέλιξε κυδοιμοῦ 
ἀλκήεις Ἰόβακχος· ἐφεδρήσσοντα δὲ δένδρῳ 
οὔτασε Μειλανίωνα δολοπλόκον οἴνοπι θύρσῳ, 
Βασσαρίδας κρυφίοισιν ὀιστεύοντα βελέμνοις· 
ἀλλά μιν ἐζώγρησεν ἀπήμονα δύσμαχος Ἥρη, 
ὅττι δόλῳ κεκόρυστο καὶ ἔχραε πολλάκι Βάκχαις 
κρυπταδίοις πολέμοισιν· ἀεὶ δέ μιν ἔκρυφε πέτρη 
ἢ φυτὸν ὑψικάρηνον ὑποκρυφθέντα πετήλοις, 
ἀνέρας ἀφράστοισιν ὀιστεύοντα βελέμνοις. 
 Ἰνδοὶ δ' ἀνδροφόνοιο μετεσσεύοντο κυδοιμοῦ 
ἠνορέην τρομέοντες ἀνικήτου Διονύσου. 



Nonnus Epic., Dionysiaca 
Book 31, line 1

Ὣς ὁ μὲν Ἰνδῴοιο τυπεὶς ἴυγγι κυδοιμοῦ 
Βάκχος Ἐρυθραίης περιδέδρομε κόλπον ἀρούρης, 
χρύσεα χιονέῃσι παρηίσι βόστρυχα σείων. 



Nonnus Epic., Dionysiaca 
Book 31, line 6

Ἥρη δὲ φθονεροῖσιν ἀνοιδαίνουσα μερίμναις 
† ἴκελον ἀπειλητῆρι κατέγραφεν ἠέρα ταρσῷ, 
αὐτόθι παπταίνουσα πολυσπερέων στρατὸν Ἰνδῶν 
θύρσοις ἀνδροφόνοισιν ἀλοιηθέντα Λυαίου. 



Nonnus Epic., Dionysiaca 
Book 31, line 62

ἀλλὰ τεὰς θώρηξον Ἐρινύας οἴνοπι Βάκχῳ, 
μὴ βροτὸν ἀθρήσαιμι νόθον σκηπτοῦχον Ὀλύμπου, 
αἴδεο λισσομένην Διὸς εὐνέτιν, αἴδεο Δηώ, 
αἴδεο † λισσομένην καθαρὴν Θέμιν, ὄφρα κεν Ἰνδοὶ 
βαιὸν ἀναπνεύσωσι τινασσομένου Διονύσου· 
ἔσσο μοι ἀχνυμένῃ τιμήορος, ὅττι Κρονίων   
Βάκχῳ νέκταρ ὄπασσε καὶ Ἄρεϊ λύθρον ἐνυοῦς. 



Nonnus Epic., Dionysiaca 
Book 31, line 77

ἡ δὲ θυελλήεντι διαΐξασα πεδίλῳ 
τρὶς μὲν ἀνηέρθη, τὸ δὲ τέτρατον ἵκετο Γάγγην· 
καὶ νέκυν Ἰνδὸν ὅμιλον ἀμειδέι δεῖξε Μεγαίρῃ 
καὶ στρατιῆς ἱδρῶτα καὶ ἠνορέην Διονύσου· 
Ἰνδοφόνους δὲ Μέγαιρα πόνους ὁρόωσα Λυαίου 
ζηλήμων ἐμέγηρε καὶ οὐρανίης πλέον Ἥρης. 



Nonnus Epic., Dionysiaca 
Book 31, line 86

ἡ δὲ νόῳ κεχάρητο· δρακοντοκόμῳ δὲ θεαίνῃ 
σαρδόνιον γελόωσα κατηφέα ῥήξατο φωνήν· 
 “οὕτω ἀριστεύουσι νέοι βασιλῆες Ὀλύμπου, 
οὕτω ἀκοντίζουσι νόθοι Διός· ἐκ Σεμέλης δὲ 
Ζεὺς ἕνα παῖδα λόχευσεν, ἵνα ξύμπαντας ὀλέσσῃ 
Ἰνδοὺς μειλιχίους καὶ ἀμεμφέας· ἀλλὰ δαείη   
Ζεὺς ἄδικος καὶ Βάκχος, ὅσον σθένος ἐστὶ Μεγαίρης. 



Nonnus Epic., Dionysiaca 
Book 31, line 93

ὦ πόποι, οἷον ἄθεσμον ἔχει νόον ὑψιμέδων Ζεύς· 
Τυρσηνοῖς ἀδίκοις οὐ μάρναται, ὅττι μαθόντες 
φώρια θεσμὰ βίαια κακοξείνων ἐπὶ νηῶν 
ἅρπαγες ἀλλοτρίων Σικελῇ πλώουσι θαλάσσῃ· 
οὐ κτάνε δυσσεβέων Δρυόπων γένος, οἷς βίος αἰχμαὶ 
καὶ φόνος· εὐσεβίῃ δὲ μεμηλότας ἔκτανεν Ἰνδούς, 
οὓς τάχα πασιμέλουσα Θέμις μαιώσατο μαζῷ. 



Nonnus Epic., Dionysiaca 
Book 31, line 115

τὴν δὲ καλεσσαμένη φιλίῳ μειλίξατο μύθῳ·   
 “Ἶρις, ἀεξιφύτου Ζεφύρου χρυσόπτερε νύμφη, 
εὔλοχε μῆτερ Ἔρωτος, ἀελλήεντι πεδίλῳ 
σπεῦδε μολεῖν ζοφόεντος ἐς Ἑσπέριον δόμον Ὕπνου· 
δίζεο καὶ περὶ Λῆμνον ἁλίκτυπον· εἰ δέ μιν εὕρῃς, 
λέξον, ἵνα Κρονίωνος ἀθελγέος ὄμματα θέλξῃ 
εἰς μίαν ἠριγένειαν, ὅπως Ἰνδοῖσιν ἀρήξω. 



Nonnus Epic., Dionysiaca 
Book 31, line 156

ἀλλὰ σύ μοι, φίλε κοῦρε, χολώεο δίζυγι θεσμῷ 
μυστιπόλοις Σατύροισι καὶ ἀγρύπνῳ Διονύσῳ· 
δὸς χάριν ἀχνυμένῃ σέο μητέρι, δὸς χάριν Ἥρῃ, 
καὶ Διὸς ὑψιμέδοντος ἀθελγέα θέλξον ὀπωπὴν 
ἐς μίαν ἠριγένειαν, ὅπως Ἰνδοῖσιν ἀρήξῃ, 
οὓς Σάτυροι κλονέουσι καὶ εἰσέτι Βάκχος ὀρίνει. 



Nonnus Epic., Dionysiaca 
Book 31, line 173

Γηγενέων δ' ἐλέαιρε γονὴν μελανόχροον Ἰνδῶν· 
δὸς χάριν· ὑμετέρης γὰρ ὁμόχροές εἰσι τεκούσης· 
ῥύεο κυανέους, κυανόπτερε· μηδὲ χαλέψῃς 
Γαῖαν ἐμοῦ γενετῆρος ὁμήλικα, τῆς ἄπο μούνης 
πάντες ἀνεβλάστησαν, ὅσοι ναετῆρες Ὀλύμπου. 



Nonnus Epic., Dionysiaca 
Book 31, line 188

εἰ δὲ σὺ ναιετάεις παρὰ Τηθύι Λευκάδα πέτρην, 
Δηριάδῃ χραίσμησον, ὃν ἤροσεν Ἰνδὸς Ὑδάσπης· 
γείτονι πιστὰ φύλαξον, ἐπεὶ τεὸς ἠχέτα γείτων 
Ὠκεανὸς κελάδων προπάτωρ πέλε Δηριαδῆος. 



Nonnus Epic., Dionysiaca 
Book 31, line 207

καὶ Παφίην μάστευεν· ὑπὲρ Λιβάνοιο δὲ μούνην 
Ἀσσυρίην ἐκίχησεν ἐρημαίην Ἀφροδίτην 
ἑζομένην· Χάριτες γὰρ ἐς ἄνθεα ποικίλα κήπων 
εἰαριναὶ στέλλοντο, χορίτιδες Ὀρχομενοῖο,   
ἡ μὲν ἀμεργομένη Κίλικα κρόκον, ἡ δὲ κομίζειν 
βάλσαμον ἱμείρουσα καὶ Ἰνδῴου δονακῆος 
φυταλιήν, ἑτέρη δὲ ῥόδων εὐώδεα ποίην. 



Nonnus Epic., Dionysiaca 
Book 31, line 274

ἀλλά, θεά, χραίσμησον, ἐμῆς δ' ἐπίκουρον ἀνίης 
δός μοι κεστὸν ἱμάντα, τεὴν πανθελγέα μίτρην, 
εἰς μίαν ἠριγένειαν, ὅπως Διὸς ὄμματα θέλξω, 
καὶ Διὸς ὑπνώοντος ἐμοῖς Ἰνδοῖσιν ἀρήξω· 
δὸς χάριν ὀψιτέλεστον, ἐπεὶ κυανόχροες Ἰνδοὶ 
ξεινοδόκοι γεγάασιν Ἐρυθραίης Ἀφροδίτης,   
οἷς κοτέων Διόνυσος ἐπέχραεν, οἷσι καὶ αὐτὸς 
θηλυμανὴς ἄστοργος ἐχώσατο παιδοτόκος Ζεύς, 
καὶ στεροπὴν ἐλέλιξε συναιχμάζων Διονύσῳ· 
δός μοι κεστὸν ἱμάντα βοηθόον, ᾧ ἔνι μούνῳ 
θέλγεις εἰν ἑνὶ πάντα· καὶ ἄξιός εἰμι φορέσσαι, 
ὡς ζυγίη γεγαυῖα καὶ ὡς συνάεθλος Ἐρώτων. 



Nonnus Epic., Dionysiaca 
Book 32, line 25

καὶ κεφαλῇ στέφος εἶχε παναίολον, ᾧ ἔνι πολλαὶ 
λυχνίδες ἦσαν, Ἔρωτος ὁμόστολοι, ὧν ἄπο πέμπει 
φαιδρὰ τινασσομένων ἀμαρύγματα Κυπριδίη φλόξ· 
εἶχε δὲ πέτρον ἐκεῖνον, ὃς ἀνέρας εἰς πόθον ἕλκει, 
οὔνομα φαιδρὸν ἔχοντα ποθοβλήτοιο Σελήνης, 
καὶ λίθον ἱμείρουσαν ἐρωτοτόκοιο σιδήρου, 
καὶ λίθον Ἰνδῴην φιλοτήσιον, ὅττι καὶ αὐτὴ 
ἐξ ὑδάτων βλάστησεν ὁμόγνιος Ἀφρογενείης, 
κυανέην θ' ὑάκινθον, ἐράσμιον εἰσέτι Φοίβῳ· 
ἀμφὶ δέ οἱ πλοκάμοισιν ἐρωτίδα δήσατο ποίην, 
ἣν φιλέει Κυθέρεια καὶ ὡς ῥόδον, ὡς ἀνεμώνην, 
καὶ φορέει μέλλουσα μιγήμεναι υἱέι Μύρρης· 
καὶ λαγόνας στεφανηδὸν ἀήθεϊ δήσατο κεστῷ· 
εἶχε δὲ ποικίλον εἷμα παλαίτατον, ᾧ χύτο νύμφης 
κρυπταδίῃ φιλότητι κασιγνήτων ὑμεναίων 




Nonnus Epic., Dionysiaca 
Book 32, line 45

ἦ ῥα πάλιν κοτέουσα κορύσσεαι οἴνοπι Βάκχῳ, 
καὶ ποθέεις Ἰνδοῖσιν ὑπερφιάλοισιν ἀρῆξαι; 



Nonnus Epic., Dionysiaca 
Book 32, line 49

ἔννεπε· καὶ γελόωντι νόῳ πολυμήχανος Ἥρη 
ζηλομανὴς ἀγόρευε παραιφαμένη παρακοίτην· 
 “Ζεῦ πάτερ, ἄλλος ἔχει με φίλος δρόμος· οὐ γὰρ ἱκάνω 
ἄρεος Ἰνδῴοιο καὶ Ἰνδοφόνου Διονύσου 
ἀλλοτρίας μεθέπουσα μεληδόνας, ἀντολίης δὲ 
γείτονας Ἠελίοιο μετέρχομαι αἴθοπας αὐλὰς 
σπερχομένη· πτερόεις γὰρ Ἔρως παρὰ Τηθύος ὕδωρ 
Ὠκεανηιάδος Ῥοδόπης δεδονημένος οἴστρῳ 
συζυγίην ἀπέειπε· καὶ ἔπλετο κόσμος ἀλήτης, 
καὶ βίος ἀχρήιστος ἀποιχομένων ὑμεναίων· 
τοῦτον ἐγὼ καλέουσα παλίνδρομος ἐνθάδε βαίνω· 
οἶσθα γάρ, ὡς Ζυγίη κικλήσκομαι, ὅττι καὶ αὐτῆς 
παῖδες ἐμαὶ κρατέουσι τελεσσιγόνου τοκετοῖο. 



Nonnus Epic., Dionysiaca 
Book 32, line 61

τοῖον ἔπος βοόωσαν ἀμείβετο θερμὸς ἀκοίτης·   
 “νύμφα φίλη, λίπε δῆριν· ἐμὸς Διόνυσος ἀγήνωρ 
ἀμώων προθέλυμνον ἀβακχεύτων γένος Ἰνδῶν 
χαιρέτω· ἀμφοτέρους δὲ γαμήλια λέκτρα δεχέσθω· 
οὐ γὰρ ἐπιχθονίης ἀλόχου πόθος οὐδὲ θεαίνης 
θυμὸν ἐμὸν θελκτῆρι τόσον βακχεύσατο κεστῷ . 



Nonnus Epic., Dionysiaca 
Book 32, line 161

ὡς δ' ὅτε χειμερίων ῥοθίων μυκώμενος ὁλκῷ 
ἄπλοος ἀντιπόροις βακχεύετο πόντος ἀέλλαις, 
κύμασιν ἠλιβάτοισι κατάρρυτον ἠέρα νείφων, 
πρυμναίους δὲ κάλωας ἀφειδέι κύματος ὁρμῇ 
λαίλαπες ἐρρήξαντο, καὶ ἄσθματι λαῖφος ἑλίξας 
ἱστὸν ἀνεχλαίνωσε κεκυφότα λάβρος ἀήτης 
λαίφεσιν ἀμφίζωστον, ἐδοχμώθη δὲ κεραίη, 
ναῦται δ' ἀσχαλόωντες ἐπέτρεπον ἐλπίδα πόντῳ· 
ὣς τότε Βάκχον ὄρινεν ὅλον στρατὸν Ἰνδικὸς ἄρης. 



Nonnus Epic., Dionysiaca 
Book 32, line 169

ἔνθα τις οὐ κατὰ κόσμον ἔην ἔρις, οὐ κλόνος ἀνδρῶν 
ἶσος ἔην, οὐ δῆρις ὁμοίιος· ἀκάματος γὰρ 
νόστιμος ἐγρεκύδοιμος ἐπέβρεμε χάλκεος Ἄρης, 
Μωδαίου προμάχοιο φέρων τύπον, ὃς πλέον ἄλλων 
ὑσμίνης ἀκόρητος ἀτερπέι τέρπετο λύθρῳ,   
ᾧ πλέον εἰλαπίνης φόνος εὔαδεν· ἐν δὲ βοείῃ, 
οἷά τε Γοργείων πλοκάμων ὀφιώδεας ὁλκούς, 
γραπτὸν ἐυσμήριγγος ἔχων ἴνδαλμα Μεδούσης 
Δηριάδῃ πέλεν ἶσος, ὁμόχροος· οὗ τότε μορφῆς 
ῥιγεδανῆς ἀγέλαστον ἔχων μίμημα προσώπου, 
καὶ σκολιὴν πλοκαμῖδα φέρων καὶ σῆμα βοείης, 
αἰνομανὴς πεφόρητο μόθῳ λαοσσόος Ἄρης, 
καὶ προμάχους θάρσυνεν. 



Nonnus Epic., Dionysiaca 
Book 32, line 175

                           ὁμογλώσσῳ δ' ἀλαλητῷ 
Βάκχου μὴ παρεόντος ἀταρβέες ἔβρεμον Ἰνδοί, 
καὶ κτύπον ἐννεάχιλον ἐπέκτυπε λοίγιος Ἄρης, 
φοιταλέην συνάεθλον ἔχων Ἔριν· ἐν δὲ κυδοιμοῖς 
στῆσε Φόβον καὶ Δεῖμον ὀπάονα Δηριαδῆος. 



Nonnus Epic., Dionysiaca 
Book 32, line 275

καὶ βλοσυροὶ Κύκλωπες ἀναιδέες εὔποδι ταρσῷ 
εἰς φόβον ἠπείγοντο τεθηπότες, οἷς ἅμα φεύγων 
Ἰνδῴην ἀδόνητος ἐλίμπανε Φαῦνος ἐνυώ. 



Nonnus Epic., Dionysiaca 
Book 32, line 287

                             ἀπὸ σκοπέλοιο δὲ Νύμφαι 
Νηιάδες βυθίοισιν ἐνεκρύπτοντο μελάθροις· 
αἱ μὲν Ὑδασπιάδεσσιν ὁμήλυδες, αἱ δὲ φυγοῦσαι 
Ἰνδὸν ἐς ἀγχικέλευθον ἐναυλίζοντο ῥεέθροις, 
ἄλλαι Συδριάδεσσιν ὁμόστολοι, αἱ δ' ἐνὶ Γάγγῃ 
λύθρον ἀπεσμήξαντο νεόσσυτον, ἃς τότε πολλὰς 
ἐρχομένας ἀγεληδὸν ἐς ὑδατόεντας ἐναύλους 
Νηιὰς ἀργυρόπεζα φιλοξείνῳ πυλεῶνι 
δέξατο κυματόεντος ἐς αὔλια παρθενεῶνος. 



Nonnus Epic., Dionysiaca 
Book 33, line 8

καὶ Χάρις ὠκυπέδιλος Ἐρυθραίῳ παρὰ κήπῳ 
φυταλιὴν εὔοδμον ἀμεργομένη δονακήων, 
ὄφρα πυριπνεύστων Παφίων ἔντοσθε λεβήτων 
Ἀσσυρίου μίξασα χυτὰς ὠδῖνας ἐλαίου 
ἄνθεσιν Ἰνδῴοισι μύρον τεύξειεν ἀνάσσῃ, 
ὁππότε παντοίην δροσερὴν ἐδρέψατο ποίην, 
χῶρον ὅλον θηεῖτο· καὶ ἀγχιπόρῳ παρὰ λόχμῃ 
λύσσαν ἑοῦ γενετῆρος ὀπιπεύουσα Λυαίου 
ἀχνυμένη δάκρυσε, φιλοστόργῳ δὲ μενοινῇ 
πενθαλέοις ὀνύχεσσιν ἑὰς ἐχάραξε παρειάς· 
καὶ Σατύρους σκοπίαζεν ὑποπτήσσοντας ἐνυώ, 
Κωδώνην δ' ἐνόησε μινυνθαδίην τε Γιγαρτὼ 
κεκλιμένας ἐφύπερθεν ἀτυμβεύτοιο κονίης· 
Χαλκομέδην δ' ἐλέαιρε θυελλήεντι πεδίλῳ 




Nonnus Epic., Dionysiaca 
Book 33, line 159

τοῦτό με μᾶλλον ὄρινεν, ὅτι βροτοειδέι μορφῇ 
Ἄρης ἐγρεκύδοιμος ἔχων συνάεθλον Ἐνυώ, 
ἀρχαίης φιλότητος ἀφειδήσας Ἀφροδίτης, 
νεύμασιν Ἡραίοισιν ἐθωρήχθη Διονύσῳ, 
Ἰνδῴῳ βασιλῆι συνέμπορος. 



Nonnus Epic., Dionysiaca 
Book 33, line 167

ἀλλὰ μολὼν ἀκίχητος ἑώιον εἰς κλίμα γαίης 
Ἰνδῴην παρὰ πέζαν, ὅπῃ θεράπαινα Λυαίου 
ἔστι τις ἐν Βάκχῃσιν, ὑπέρτερος ἥλικος ἥβης, 
οὔνομα Χαλκομέδη φιλοπάρθενος – εἰ δέ κεν ἄμφω 
Χαλκομέδην καὶ Κύπριν ἔσω Λιβάνοιο νοήσῃς, 
οὐ δύνασαι, φίλε κοῦρε, διακρίνειν Ἀφροδίτην – · 
κεῖθι μολὼν χραίσμησον ἐρημονόμῳ Διονύσῳ, 
Μορρέα τοξεύσας ἐπὶ κάλλεϊ Χαλκομεδείης· 
σεῖο δὲ τοξοσύνης γέρας ἄξιον ἐγγυαλίξω   
Λήμνιον εὐποίητον ἐγὼ στέφος, εἴκελον αἴγλαις 
Ἠελίου φλογεροῖο· σὺ δὲ γλυκὺν ἰὸν ἰάλλων 




Nonnus Epic., Dionysiaca 
Book 33, line 188

καὶ ταχὺς Ἰνδῴοιο μολὼν κατὰ μέσσον ὁμίλου 
τόξον ἑὸν στήριξεν ἐπ' αὐχένι Χαλκομεδείης· 
καὶ βέλος ἰθύνων ῥοδέης περὶ κύκλα παρειῆς 
Μορρέος εἰς φρένα πέμψεν. 



Nonnus Epic., Dionysiaca 
Book 33, line 194

                               ἐρετμώσας δὲ πορείην 
νηχομένων πτερύγων ἑτερόζυγι σύνδρομος ὁλκῷ 
πατρῴους ἀνέβαινεν ἐς ἀστερόεντας ὀχῆας, 
καλλείψας πυρόεντι πεπαρμένον Ἰνδὸν ὀιστῷ. 



Nonnus Epic., Dionysiaca 
Book 33, line 201

ἡ δὲ δολοφρονέουσα παρήπαφεν ὄρχαμον Ἰνδῶν, 
οἷά περ ἱμείρουσα, πόθου δ' ἀπεμάξατο κούρη 
ψευδαλέον μίμημα· καὶ αἰθέρος ἥπτετο Μορρεύς, 
ἐλπίδι μαψιδίῃ πεφορημένος· ἐν κραδίῃ γὰρ 
παρθενικὴν ἐδόκησεν ἔχειν βέλος ἶσον ἐρώτων, 
κοῦφος ἀνήρ, ὅτι παῖδα σαόφρονα δίζετο θέλγειν 
κυανέοις μελέεσσι, καὶ οὐκ ἐμνήσατο μορφῆς. 



Nonnus Epic., Dionysiaca 
Book 33, line 256

                                                       ... 
εἰς Φρυγίην Διόνυσος ὀπάονα † Δηριαδῆα 
δουλοσύνης ἐρύσειεν ὑπὸ ζυγόν, ἀντὶ δὲ πάτρης 
Μαιονίη πολύολβος ἑὸν ναέτην με δεχέσθω· 
Τμῶλον ἔχειν ἐθέλω μετὰ Καύκασον· ἀρχέγονον δὲ 
Ἰνδὸν ἀπορρίψας ἐμὸν οὔνομα Λυδὸς ἀκούσω, 
αὐχένα δοῦλον Ἔρωτος ὑποκλίνων Διονύσῳ· 
Πακτωλὸς φερέτω με· τί μοι πατρῷος Ὑδάσπης; 



Nonnus Epic., Dionysiaca 
Book 33, line 269

ἤδη γὰρ σκιόεντι θορὼν αὐτόχθονι παλμῷ 
ἄψοφος ἀννεφέλοιο μελαίνετο κῶνος ὀμίχλης,   
καὶ τρομερῇ ξύμπαντα μιῇ ξύνωσε σιωπῇ· 
οὐδέ τις ἴχνος ἔπειγε δι' ἄστεος Ἰνδὸς ὁδίτης, 
οὐδὲ γυνὴ χερνῆτις ἐθήμονος ἥπτετο τέχνης, 
οὐδέ οἱ ἐν παλάμῃσι φιληλακάτῳ παρὰ λύχνῳ 
κύκλον ἐς αὐτοέλικτον ἰὼν ἄτρακτος ἀλήτης 
ἄστατος ὀρχηστῆρι τιταίνετο νήματος ὁλκῷ, 
ἀλλὰ καρηβαρέουσα φιλαγρύπνῳ παρὰ λύχνῳ 
εὗδε γυνὴ ταλαεργός· ὄφις δέ τις ἥσυχος ἕρπων 
κεῖτο πεσών, κεφαλῇ † δὲ λύων παλινάγρετον οὐρὴν 
γαστέρος ὑπναλέης ἀνεσείρασεν ὁλκὸν ἀκάνθης· 
καί τις ἀερσιπόδης ἐλέφας παρὰ γείτονι λόχμῃ 




Nonnus Epic., Dionysiaca 
Book 33, line 307

οἶδα, πόθεν, Κυθέρεια, χολώεαι υἱάσιν Ἰνδῶν· 
γείτονας Ἠελίοιο τεοὶ κλονέουσιν ὀιστοί· 
οὔ πω μνῆστιν ὄλεσσας ἐλεγχομένων σέο δεσμῶν. 



Nonnus Epic., Dionysiaca 
Book 33, line 363

γίνεό μοι δολόεσσα φερέσβιος· αὐτοφόνος γὰρ   
αἴ κε θάνῃς ἀδίδακτος ἀνυμφεύτων ὑμεναίων, 
Βασσαρίδων στίχα πᾶσαν ἀνάρσιος Ἰνδὸς ὀλέσσει· 
ἀλλά μιν ἠπερόπευε, καὶ ἐκ θανάτοιο σαώσεις 
σὴν στρατιὴν φύξηλιν ἱμασσομένου Διονύσου, 
ψευδομένη Παφίης κενεὸν πόθον· εἰ δέ σε Μορρεὺς 
εἰς εὐνὴν ἐρύσειεν ἀναινομένην ὑμεναίους, 
οὐ χατέεις ἐπὶ Κύπριν ἀρηγόνος· ὑμετέρης γὰρ 
φρουρὸν ἔχεις ἀπέλεθρον ὄφιν χραισμήτορα μίτρης· 
ὑμέτερον δὲ δράκοντα λαβὼν μετὰ φύλοπιν Ἰνδῶν 
στηρίξει Διόνυσος ἐν ἀστεροφεγγέι κύκλῳ, 
ἄγγελον οὐ λήγοντα τεῆς ἀλύτοιο κορείης, 




Nonnus Epic., Dionysiaca 
Book 34, line 125

ἀλλ' ὅτε φοινίσσοντι σέλας πέμπουσα προσώπῳ 
ὑσμίνης προκέλευθος ἑκηβόλος ἄνθορεν Ἠώς, 
Ἰνδῴην ἐκόρυσσε γονὴν λαοσσόος Ἄρης· 
καὶ τότε θωρηχθέντες ἐυτροχάλων ἐπὶ δίφρων 
ἅρματι Δηριάδαο συνήλυδες ἔρρεον Ἰνδοί. 



Nonnus Epic., Dionysiaca 
Book 34, line 162

ἔνθα διατμήξας Χαρίτων ἴνδαλμα προσώπου 
Βασσαρίδας ζώγρησεν ἀνάλκιδας ἕνδεκα Μορρεύς, 
ἃς μετὰ Χαλκομέδην ἐκρίνατο· Μαιναλίδων δὲ 
χεῖρας ὀπισθοτόνους ἀλύτῳ σφηκώσατο δεσμῷ, 
καὶ στίχα λυσιέθειραν ἐπὶ ζυγὰ δούλια σύρων   
ληίδας ἀμφιπόλους ἑκυρῷ πόρε Δηριαδῆι, 
ἕδνον ἑῆς ἀλόχοιο τὸ δεύτερον, ἧς χάριν εὐνῆς 
νυμφοκόμον μόθον εἶχεν ἀερσιλόφῳ παρὰ Ταύρῳ, 
ὁππότε Δηριάδαο νέην βασιληίδα κούρην, 
ἥλικα Χειροβίην, ζυγίῳ σφηκώσατο δεσμῷ· 




Nonnus Epic., Dionysiaca 
Book 34, line 172

Βασσαρίδας ζώγρησεν ἀνάλκιδας ἕνδεκα Μορρεύς, 
ἃς μετὰ Χαλκομέδην ἐκρίνατο· Μαιναλίδων δὲ 
χεῖρας ὀπισθοτόνους ἀλύτῳ σφηκώσατο δεσμῷ, 
καὶ στίχα λυσιέθειραν ἐπὶ ζυγὰ δούλια σύρων   
ληίδας ἀμφιπόλους ἑκυρῷ πόρε Δηριαδῆι, 
ἕδνον ἑῆς ἀλόχοιο τὸ δεύτερον, ἧς χάριν εὐνῆς 
νυμφοκόμον μόθον εἶχεν ἀερσιλόφῳ παρὰ Ταύρῳ, 
ὁππότε Δηριάδαο νέην βασιληίδα κούρην, 
ἥλικα Χειροβίην, ζυγίῳ σφηκώσατο δεσμῷ· 
οὐ γὰρ δῶρον ἔδεκτο γαμήλιον ὄρχαμος Ἰνδῶν 
παιδὸς ἑῆς, οὐ χρυσὸν ἀπείριτον, οὐ λίθον ἅλμης 
μαρμαρέην, ἀγέλας δὲ βοῶν καὶ πώεα μήλων 
Δηριάδης ἀπέειπε, καὶ ἐγρεμόθοισι μαχηταῖς 
θυγατέρων ἔζευξεν ἀδωροδόκους ὑμεναίους, 
γαμβρὸν ἔχων Μορρῆα καὶ ἐννεάπηχυν Ὀρόντην· 
καὶ διδύμοις προμάχοισιν ἑὴν νύμφευσε γενέθλην, 
Μορρέι Χειροβίην καὶ Πρωτονόειαν Ὀρόντῃ. 



Nonnus Epic., Dionysiaca 
Book 34, line 182

οὐ γὰρ ἐπιχθονίοισιν ὁμοίιος ἔπλετο Μορρεύς, 
ἀλλὰ Γιγαντείων μελέων ὑψαύχενι μορφῇ 
Ἰνδῶν Γηγενέων μιμήσατο πάτριον ἀλκήν, 
ἠλιβάτου Τυφῶνος ἔχων αὐτόχθονα φύτλην, 
εὖτε πυριτρεφέων Ἀρίμων παρὰ γείτονι πέτρῃ 
σύγγονον ἠνορέην ἐπεδείκνυε μάρτυρι Κύδνῳ, 
ἕδνα φέρων θαλάμων Κιλίκων ἱδρῶτας ἀέθλων, 
νυμφίος ἀκτήμων, ἀρετῇ δ' ἐκτήσατο νύμφην, 
ὁππότε Μορρηνοῖο γάμων μνηστῆρι σιδήρῳ   
Ἀσσυρίη γόνυ κάμψε, καὶ εἰς ζυγὰ Δηριαδῆος 
αὐχένα πετρήεντα Κίλιξ δοχμώσατο Ταῦρος, 
καὶ θρασὺς ὤκλασε Κύδνος, ὅθεν Κιλίκων ἐνὶ γαίῃ 




Nonnus Epic., Dionysiaca 
Book 34, line 198

ὣς φαμένου Μορρῆος ἀμείβετο κοίρανος Ἰνδῶν· 
 “Χειροβίην ἀνάεδνον ἔχων, κορυθαιόλε Μορρεῦ, 
ἄξιά μοι πόρες ἕδνα φερεσσακέων ὑμεναίων, 
ἄστεα δουλώσας Κιλίκων ὑψήνορι νίκῃ. 



Nonnus Epic., Dionysiaca 
Book 34, line 208

μοῦνον ἐμοὶ πεφύλαξο δορικτήτης πόθον εὐνῆς, 
μή σε γυναιμανέεσσιν ἴδω πανομοίιον Ἰνδοῖς· 
ὄμματα μὴ σκοπίαζε καὶ ἄργυφον αὐχένα Βάκχης, 
μὴ ποθέων τελέσειας ἐμὴν ζηλήμονα κούρην. 



Nonnus Epic., Dionysiaca 
Book 34, line 236

                                    ὑψιτενεῖς δὲ 
αἱ μὲν ἐυγλυφάνοιο παρὰ προπύλαια μελάθρου 
ἀγχονίῳ θλίβοντο περίπλοκον αὐχένα δεσμῷ· 
ἄλλαις θερμὸν ὄπασσε μόρον πυρόεντι ῥεέθρῳ· 
αἱ δὲ πεδοσκαφέεσσιν ἐτυμβεύοντο βερέθροις 
φρείατος ἐν γυάλοισιν, ὅπῃ βυθίων ἀπὸ κόλπων 
χερσὶν ἀμοιβαίαις βεβιημένον ἕλκεται ὕδωρ·   
καί τις ἔσω διεροῖο βαθυνομένου κενεῶνος 
ἡμιθανὴς ἀτίνακτος ἀμοιβαίῃ φάτο φωνῇ· 
 “ἔκλυον, ὡς Ἰνδοῖσι θεὸς πέλε γαῖα καὶ ὕδωρ· 
οὐδὲ μάτην ποτὲ τοῦτο φατίζεται· ἀμφότεροι γὰρ 
εἰς ἐμὲ θωρήχθησαν ὁμόφρονες, εἰμὶ δὲ μέσση 
καὶ χθονίου θανάτοιο καὶ ὑδατόεντος ὀλέθρου, 
καὶ μόρον ἐγγὺς ἔχω διδυμόζυγον· ἰλυόεις γὰρ 
ξεῖνος δεσμὸς ἔχει με, καὶ οὐκέτι ταρσὸν ἀείρω, 
ὑγρὰ δὲ ῥιζώσασα πεπηγότα γούνατα πηλῷ 
ἵσταμαι ἀστυφέλικτος ἐγὼ Μοίρῃσιν ἑτοίμη. 



Nonnus Epic., Dionysiaca 
Book 34, line 328

δέξο με σοῖς Σατύροισιν ὁμόστολον· ἐν πολέμοις γὰρ 
Ἰνδοὶ ἀριστεύσουσιν, ἕως ἔτι χεῖρα κορύσσω. 



Nonnus Epic., Dionysiaca 
Book 35, line 2

Δηριάδης δ' ἀπέλεθρος ἐμάρνατο θυιάδι χάρμῃ, 
καὶ Βρομίου προπόλοισιν ἐπέχραε κοίρανος Ἰνδῶν, 
πῇ μὲν ἀκοντίζων δολιχῷ δορί, πῇ δὲ δαΐζων 
ἄορι κωπήεντι, χαραδραίοις δὲ βελέμνοις 
τοξεύων πεφόρητο καὶ ὀξυτέροισιν ὀιστοῖς. 



Nonnus Epic., Dionysiaca 
Book 35, line 130

οὐκέτι Μαιονίης ἐπιβήσομαι· οὐδ' ἐνὶ παστῷ 
δέξομαι, ἢν ἐθέλῃς, μετὰ Μορρέα Βάκχον ἀκοίτην· 
ἔσσομαι Ἰνδῴη καὶ ἐγώ, φίλος· ἀντὶ δὲ Λυδῆς 
κυδαίνω θυέεσσιν Ἐρυθραίην Ἀφροδίτην 
κρυπταδίη Μορρῆος ὁμευνέτις· ἐν δὲ κυδοιμοῖς 
Ἰνδὸς ἀνὴρ ἐχέτω με συναιχμάζουσαν ἀκοίτῃ· 
εἰς σὲ γὰρ ἶσα βέλεμνα καὶ εἰς ἐμὲ διπλόα πέμπων 
Ἵμερος ἀμφοτέροισι μίαν ξύνωσεν ἀνάγκην, 
εἰς κραδίην Μορρῆι καὶ εἰς φρένα Χαλκομεδείῃ. 



Nonnus Epic., Dionysiaca 
Book 35, line 151

κούρην Δηριαδῆος ἀναίνομαι· αὐτὸς ἐλάσσω 
ἐκ μεγάρων ἀέκουσαν ἐμὴν ζηλήμονα νύμφην· 
οὐκέτι Βασσαρίδεσσι κορύσσομαι, εἴ με κελεύεις, 
ἀλλὰ φίλοις ναέτῃσι μαχέσσομαι· Ἰνδὸν ὀλέσσω   
οἴνοπα θύρσον ἔχων, οὐ χάλκεον ἔγχος ἀείρων· 
ῥίψω δ' ἔντεα πάντα καὶ ἄνθεα λεπτὰ τινάξω, 
ὑμετέρῳ βασιλῆι συναιχμάζων Διονύσῳ. 



Nonnus Epic., Dionysiaca 
Book 35, line 190

                                  ἄγχι δὲ πόντου 
καλλείψας ἀκόμιστον ἐπ' αἰγιαλοῖο χιτῶνα 
θαλπόμενος γλυκερῇσι μεληδόσι λούσατο Μορρεύς, 
γυμνὸς ἐών· ψυχρῇ δὲ δέμας φαίδρυνε θαλάσσῃ, 
θερμὸν ἔχων Παφίης ὀλίγον βέλος· ἐν δὲ ῥεέθροις 
Ἰνδῴην ἱκέτευεν Ἐρυθραίην Ἀφροδίτην, 
εἰσαΐων, ὅτι Κύπρις ἀπόσπορός ἐστι θαλάσσης· 
λουσάμενος δ' ἀνέβαινε μέλας πάλιν· εἶχε δὲ μορφήν, 
ὡς φύσις ἐβλάστησε, καὶ ἀνέρος οὐ δέμας ἅλμη, 
οὐ χροιὴν μετάμειψεν, ἐρευθαλέη περ ἐοῦσα. 



Nonnus Epic., Dionysiaca 
Book 35, line 228

καὶ γὰρ ἀπ' Οὐλύμποιο θορὼν ὠκύπτερος Ἑρμῆς, 
ἀντίτυπον Βρομίοιο φέρων ἴνδαλμα προσώπου, 
Βακχείην ἐκάλεσσεν ὅλην στίχα μύστιδι φωνῇ· 
δαιμονίην δὲ γυναῖκες ὅτ' ἔκλυον εὔιον ἠχώ, 
εἰς ἕνα χῶρον ἵκανον· ἀπὸ τριόδων δὲ κομίζων 
Μαιναλίδων ὅλον ἔθνος ἐς ἀγκύλα κύκλα κελεύθου 
ἤγαγεν ὠκυπέδιλος, ἕως σχεδὸν ἤιε πύργων· 
καὶ φυλάκων στοιχηδὸν ἀκοιμήτοισιν ὀπωπαῖς 
νήδυμον ὕπνον ἔχευεν ἑῇ πανθελγέι ῥάβδῳ 
φώριος Ἑρμείας, πρόμος ἔννυχος· ἐξαπίνης δὲ 
Ἰνδοῖς μὲν ζόφος ἦεν, ἀθηήτοισι δὲ Βάκχαις 




Nonnus Epic., Dionysiaca 
Book 35, line 268

                                       ἔγρετο δὲ Ζεὺς 
Καυκάσου ἐν κορυφῇσιν ἀπορρίψας πτερὸν Ὕπνου· 
καὶ δόλον ἠπεροπῆα μαθὼν κακοεργέος Ἥρης 
Σιληνοὺς ἐδόκευε πεφυζότας, ἔδρακε Βάκχας 
σπερχομένας ἀγεληδὸν ἀπὸ τριόδων, ἀπὸ πύργων, 
καὶ Σατύρους κείροντα καὶ ἀμώοντα γυναῖκας 
Δηριάδην ἐνόησεν ὀπίστερον, ὄρχαμον Ἰνδῶν, 
υἱέα δ' ἐν δαπέδῳ κατακείμενον· ἀμφὶ δὲ νύμφαι   
ἐγγὺς ἔσαν στεφανηδόν· ὁ δὲ στροφάλιγγι κονίης 
κεῖτο καρηβαρέων, ὀλιγοδρανὲς ἄσθμα τιταίνων, 
ἀφρὸν ἀκοντίζων χιονώδεα, μάρτυρα λύσσης. 



Nonnus Epic., Dionysiaca 
Book 35, line 283

                                         οὐδὲ καὶ αὐτὴ 
σὸν κότον ἐπρήυνεν ἀτέρμονα νυμφιδίη φλόξ, 
λέκτρα διασκεδάσασα Διοβλήτοιο Θυώνης; 
Ἰνδοφόνῳ τέο μέχρις ἐπιβρίζεις Διονύσῳ; 



Nonnus Epic., Dionysiaca 
Book 35, line 297

δήσω σὰς παλάμας χρυσέῳ πάλιν ἠθάδι δεσμῷ· 
Ἄρεα δ' ἀρραγέεσσιν ἀλυκτοπέδῃσι πεδήσω   
καί μιν ἀναλθήτοισιν ὅλον πληγῇσιν ἱμάσσω, 
εἰς τροχὸν αὐτοκύλιστον ὁμόδρομον, οἷος ἀλήτης 
Τάνταλος ἠερόφοιτος ἢ Ἰξίων μετανάστης, 
εἰσόκε νικήσειεν ἐμὸς πάις υἱέας Ἰνδῶν. 



Nonnus Epic., Dionysiaca 
Book 35, line 301

ἀλλὰ τεῷ Κρονίωνι χαρίζεαι, αἴ κεν ἐλάσσῃς 
λύσσαν ἐριπτοίητον ἱμασσομένου Διονύσου· 
μηδὲ λίπῃς κοτέοντα τεὸν πόσιν, ἀλλὰ μολοῦσα 
Ἰνδῴης ἀκίχητος ὑπὸ κλέτας εὔβοτον ὕλης 
Βάκχῳ μαζὸν ὄρεξον ἐμὴν μετὰ μητέρα Ῥείην, 
ὄφρα τελειοτέροισιν ἑοῖς στομάτεσσιν ἀφύσσῃ 
σὴν ἱερὴν ῥαθάμιγγα προηγήτειραν Ὀλύμπου, 
καὶ βατὸν αἰθέρα τεῦξον ἐπιχθονίῳ Διονύσῳ· 
ὑμετέρῳ δὲ γάλακτι δέμας χρίσασα Λυαίου 
σβέσσον ἀμερσινόοιο δυσειδέα λύματα νούσου. 



Nonnus Epic., Dionysiaca 
Book 35, line 352

ἶσος ἐμῷ γενετῆρι φανήσομαι· ἐν πολέμοις γὰρ 
Γηγενέας Τιτῆνας ἐμὸς νίκησε Κρονίων, 
νικήσω καὶ ἔγωγε χαμαιγενέων γένος Ἰνδῶν. 



Nonnus Epic., Dionysiaca 
Book 35, line 354

σήμερον ἀθρήσητε κορυμβοφόρον μετὰ νίκην 
Δηριάδην ἱκέτην βραδυπειθέα, καὶ χορὸν Ἰνδῶν 
αὐχένα δοχμώσαντα γαληναίῳ Διονύσῳ, 
καὶ ποταμὸν μεθέποντα μεθυσφαλὲς Εὔιον ὕδωρ· 
ἀντιβίους δ' ὄψεσθε παρὰ κρητῆρι Λυαίου 
ξανθὸν ὕδωρ πίνοντας ἀπ' οἰνοπόρου ποταμοῖο, 
καὶ θρασὺν Ἰνδὸν ἄνακτα κατάσχετον οἴνοπι κισσῷ, 
ἰλλόμενον πετάλοισι καὶ ἀμπελόεντι κορύμβῳ, 
εἴκελα δεσμὰ φέροντα, τά περ μετὰ κύματα λύσσης 
Νυσιάδες βοόωσι θεουδέες εἰσέτι νύμφαι, 
ἀλκῆς ἡμετέρης ἐπιμάρτυρες, ὁππότε κισσοῦ 




Nonnus Epic., Dionysiaca 
Book 36, line 141

καὶ τότε λυσσήεις παλινάγρετον ἄμφεπε χάρμην   
Δηριάδης βαρύμηνις, ἀπήμονας ὡς ἴδε Βάκχας· 
καὶ μόθον ἀρτεμέοντος ὀπιπεύων Διονύσου 
εἰς ἐνοπὴν οἴστρησε πεφυζότας ἡγεμονῆας· 
καὶ ξυνὴν πρυλέεσσι καὶ ἱππήεσσιν ἀπειλὴν 
βάρβαρον ἐσμαράγησε βαρυφθόγγων ἀπὸ λαιμῶν· 
 “σήμερον ἢ Διόνυσον ἐγὼ πλοκαμῖδος ἐρύσσω, 
ἠὲ μόθος Βακχεῖος ἀιστώσει γένος Ἰνδῶν. 



Nonnus Epic., Dionysiaca 
Book 36, line 157

θαρσαλέοι δὲ γένεσθε, καὶ Ἰνδῴην μετὰ χάρμην 
νίκην κυδιάνειραν ἀείσατε Δηριαδῆος,   
ὄφρα τις ἐρρίγῃσι καὶ ὀψιγόνων στρατὸς ἀνδρῶν 
Ἰνδοῖς Γηγενέεσσιν ἀνικήτοισιν ἐρίζειν. 



Nonnus Epic., Dionysiaca 
Book 36, line 174

θυρσομανὴς Διόνυσος ἐρημονόμων στίχα θηρῶν 
εἰς ἐνοπὴν βάκχευεν· ὀριτρεφέες δὲ μαχηταὶ 
δαιμονίῃ βρυχηδὸν ἐβακχεύθησαν ἱμάσθλῃ· 
καὶ πολὺς ἐκ στομάτων ἐκορύσσετο μαινόμενος θὴρ 
ὠμοβόρων τε δράκοντες ἀποπτύοντες ὀδόντων 
τηλεβόλους πόμπευον ἐς ἠέρα πίδακας ἰοῦ 
χάσματι συρίζοντι μεμυκότος ἀνθερεῶνος, 
λοξὰ παρασκαίροντες· ἐς ἀντιβίους δὲ θορόντες 
αὐτόματον σκοπὸν εἶχον ἐχιδνήεντες ὀιστοί· 
καὶ σκολιαῖς ἑλίκεσσιν ἐμιτρώθη δέμας Ἰνδῶν 
ἰλλομένων, βροτέους δὲ πόδας σφηκώσατο σειρὴ 
εἰς δρόμον ἀίσσοντας. 



Nonnus Epic., Dionysiaca 
Book 36, line 197

καὶ πολὺς ἐσμὸς ἔπιπτε, βαρυσμαράγων ἀπὸ λαιμῶν 
φρικτὸν ἐρημονόμων ἀίων βρύχημα λεόντων· 
καί τις ἐνικήθη τρομέων μυκήματα ταύρου, 
καὶ βοὸς εἰσορόων βλοσυρῆς γλωχῖνα κεραίης 
λοξὸν ἀκοντίζουσαν ἐς ἠέρα· φοιταλέος δὲ 
εἰς φόβον ἄλλος ὄρουσεν ὑποφρίσσων γένυν ἄρκτου·   
θηρείαις δ' ἰαχῇσιν ὁμόκτυπος ἄλλος ἐπ' ἄλλῳ 
Πανὸς ἀνικήτοιο κύων συνυλάκτεε λαιμῷ, 
καὶ μόθον ὑλακόμωρον ἐδείδεσαν αἴθοπες Ἰνδοί. 



Nonnus Epic., Dionysiaca 
Book 36, line 253

Βακχείης κατὰ μέσσον ἐμαίνετο δηιοτῆτος· 
Βασσαρίδων δὲ φάλαγγα μετὰ κλόνον ἤθελεν ἕλκειν 
εἰς εὐνὴν ἀνάεδνον ἀναγκαίων ὑμεναίων, 
καὶ κενεῇ πολέμιζεν ἐπ' ἐλπίδι, τηλίκος ἀνήρ, 
οἷος ἔην θρασὺς Ὦτος ἀνέμβατον αἰθέρα βαίνων, 
ἁγνὸν ἀνυμφεύτου ποθέων λέχος Ἰοχεαίρης, 
οἷος ἔην φιλέων καθαρῆς ὑμέναιον Ἀθήνης 
ὑψινεφὴς ἐς Ὄλυμπον ἀκοντίζων Ἐφιάλτης· 
Κολλήτης πέλε τοῖος ὑπέρτερος, αἰθέρι γείτων, 
Γηγενέος προγόνοιο θεημάχον αἷμα κομίζων, 
Ἰνδοῦ πρωτογόνοιο· καὶ ἄρκιος ἔπλετο μορφῇ 
δῆσαι θοῦρον Ἄρηα μεθ' υἱέας Ἰφιμεδείης· 
ἀλλὰ τόσον περ ἐόντα γυνὴ κτάνεν ὀξέι πέτρῳ, 
Βακχιάδος Χαρόπεια κυβερνήτειρα χορείης. 



Nonnus Epic., Dionysiaca 
Book 36, line 283

καὶ κοτέων ἑτάροιο δεδουπότος ἀρχὸς Ἀβάντων 
Καρμίνων βασιλῆα κατεπρήνιξε Μελισσεύς, 
Κύλλαρον, ὀξυόεντι κατ' αὐχένος ἄορι τύψας, 
Λωγασίδην, ὃς μοῦνος, ἐπεὶ σοφὸς ἔσκε μαχητής, 
Δηριάδῃ μεμέλητο δοριθρασέων πλέον Ἰνδῶν . 



Nonnus Epic., Dionysiaca 
Book 36, line 373

καὶ τόσον Ἰνδὸν ἄνακτα, τὸν οὐ κτάνεν ἄσπετος αἰχμή, 
ἀμπελόεις νίκησεν ἕλιξ πρόμος· ἀμφιέπων δὲ 
ἡμερίδων ὄρπηκι κατάσχετον ἀνθερεῶνα 
πνίγετο Δηριάδης σκολιῷ τεθλιμμένος ὁλκῷ. 



Nonnus Epic., Dionysiaca 
Book 36, line 416

                          ἐπασχαλόων δὲ κυδοιμῷ 
μαντοσύνης Διόνυσος ἑῆς ἐμνήσατο Ῥείης, 
ὅττι τέλος πολέμοιο φανήσεται, ὁππότε Βάκχοι 
εἰναλίην Ἰνδοῖσιν ἀναστήσωσιν ἐνυώ. 



Nonnus Epic., Dionysiaca 
Book 36, line 424

                                                       .. 
εἰς ἀγορὴν ἐκάλεσσε μελαρρίνων γένος Ἰνδῶν 
Δηριάδης σκηπτοῦχος· ἐπειγομένῳ δὲ πεδίλῳ 
λαὸν ἀολλίζων ἑτερόθροον ἤιε κῆρυξ. 



Nonnus Epic., Dionysiaca 
Book 36, line 427

αὐτίκα δ' ἠγερέθοντο πολυσπερέων στίχες Ἰνδῶν, 
ἑζόμενοι στοιχηδὸν ἀμοιβαίων ἐπὶ βάθρων· 
λαοῖς δ' ἀγρομένοισιν ἄναξ ἀγορήσατο Μορρεύς· 
 “ἴστε, φίλοι, τάχα πάντες, ἅ περ κάμον ὑψόθι Ταύρου, 
εἰσόκε γαῖα Κίλισσα καὶ Ἀσσυρίων γένος ἀνδρῶν 
αὐχένα δοῦλον ἔκαμψεν ὑπὸ ζυγὰ Δηριαδῆος· 
ἴστε καί, ὅσσα τέλεσσα καταιχμάζων Διονύσου, 
μαρνάμενος Σατύροισι καὶ ἀμητῆρι σιδήρῳ 
τέμνων ἐχθρὰ κάρηνα βοοκραίροιο γενέθλης, 
ὁππότε Βασσαρίδων πεπεδημένον ἐσμὸν ἐρύσσας 




Nonnus Epic., Dionysiaca 
Book 36, line 450

ἢ πότε λυσσώων ὀρεσίδρομος ὑψίκερως Πὰν 
θηγαλέοις ὀνύχεσσι διατμήξει νέας Ἰνδῶν; 



Nonnus Epic., Dionysiaca 
Book 36, line 464

ἀλλά, φίλοι, μάρνασθε πεποιθότες· ἀντιβίων δὲ 
μή τις ὑποπτήσσειεν ὀπιπεύων στίχα νηῶν 
Βακχιάδων· Ἰνδοὶ γὰρ ἐθήμονές εἰσι κυδοιμοῦ 
εἰναλίου, καὶ μᾶλλον ἀριστεύουσι θαλάσσῃ 
ἢ χθονὶ δηριόωντες. 



Nonnus Epic., Dionysiaca 
Book 37, line 1

Ὣς οἱ μὲν φιλότητι μεμηλότες ἔμφρονες Ἰνδοί, 
Βακχείην ἀνέμοισιν ἐπιτρέψαντες ἐνυώ, 
ὄμμασιν ἀκλαύτοισιν ἐταρχύσαντο θανόντας, 
οἷα βίου βροτέου γαιήια δεσμὰ φυγόντας 
ψυχῆς πεμπομένης, ὅθεν ἤλυθε, κυκλάδι σειρῇ 
νύσσαν ἐς ἀρχαίην· στρατιὴ δ' ἀμπαύετο Βάκχου. 



Nonnus Epic., Dionysiaca 
Book 37, line 48

                                          ἀμφὶ δὲ νεκρῷ 
Ἀστέριος Δικταῖος ἐπήορον ἆορ ἐρύσσας 
Ἰνδοὺς κυανέους δυοκαίδεκα δειροτομήσας 
θῆκεν ἄγων στεφανηδὸν ἐπασσυτέρῳ τινὶ κόσμῳ· 
ἐν δ' ἐτίθει μέλιτος καὶ ἀλείφατος ἀμφιφορῆας. 



Nonnus Epic., Dionysiaca 
Book 37, line 102

καὶ τροχαλοὶ Κορύβαντες, ἐπεὶ λάχον ἔνδιον Ἴδης, 
τύμβον ἐτορνώσαντο· βαθυνομένων δὲ θεμέθλων 
νεκρὸν ἐταρχύσαντο πεδοσκαφέος διὰ κόλπου, 
Κρήτης γνήσιον αἷμα, μιῆς οἰκήτορα πάτρης, 
καὶ κόνιν ὀθνείην πυμάτην ἐπέχευαν Ὀφέλτῃ· 
καὶ τάφον αἰπυτέροισιν ἀνεστήσαντο δομαίοις, 
τοῖον ἐπιγράψαντες ἔπος νεοπενθέι τύμβῳ· 
’νεκρὸς Ἀρεστορίδης μινυώριος ἐνθάδε κεῖται,   
Κνώσσιος, Ἰνδοφόνος, Βρομίου συνάεθλος, Ὀφέλτης. 



Nonnus Epic., Dionysiaca 
Book 37, line 115

ποικίλα δ' ἦεν ἄεθλα, λέβης, τρίπος, ἀσπίδες, ἵπποι, 
ἄργυρος, Ἰνδὰ μέταλλα, βόες, Πακτώλιος ἰλύς. 



Nonnus Epic., Dionysiaca 
Book 37, line 120

καὶ θεὸς ἱππήεσσιν ἀέθλια θήκατο νίκης· 
πρώτῳ μὲν θέτο τόξον Ἀμαζονίην τε φαρέτρην 
καὶ σάκος ἡμιτέλεστον Ἀρηιφίλην τε γυναῖκα, 
τήν ποτε Θερμώδοντος ἐπ' ὀφρύσι πεζὸς ὁδεύων 
λουομένην ζώγρησε, καὶ ἤγαγεν εἰς πόλιν Ἰνδῶν· 
δευτέρῳ ἵππον ἔθηκε Βορειάδι σύνδρομον αὔρῃ, 
ξανθοφυῆ, δολιχῇσι κατάσκιον αὐχένα χαίταις, 
ἡμιτελὲς κυέουσαν ἔτι βρέφος, ἧς ἔτι γαστὴρ   
ἵππιον ὄγκον ἔχουσα γονῆς οἰδαίνετο φόρτῳ· 
καὶ τριτάτῳ θώρηκα, καὶ ἀσπίδα θῆκε τετάρτῳ· 
τὸν μὲν ἀριστοπόνος τεχνήσατο Λήμνιος ἄκμων 
ἀσκήσας χρυσέῳ δαιδάλματι, τῆς δ' ἐνὶ μέσσῳ 
ὀμφαλὸς ἀργυρέῳ τροχόεις ποικίλλετο κόσμῳ· 
πέμπτῳ δοιὰ τάλαντα, γέρας Πακτωλίδος ὄχθης. 



Nonnus Epic., Dionysiaca 
Book 37, line 486

αὐτὰρ ὁ πυγμαχίης χαροπῆς ἔστησεν ἀγῶνα· 
πρώτῳ μὲν θέτο ταῦρον ἀπ' Ἰνδῴοιο βοαύλου 
δῶρον ἔχειν, ἑτέρῳ δὲ μελαρρίνων κτέρας Ἰνδῶν 
βάρβαρον αἰολόνωτον ἄγων κατέθηκε βοείην. 



Nonnus Epic., Dionysiaca 
Book 37, line 545

                                   ἐσσύμενος δὲ 
Ἰνδῴην περίμετρον ἀνηέρταζε βοείην. 



Nonnus Epic., Dionysiaca 
Book 37, line 751

καὶ φιλίην ἐπὶ δῆριν ἀκοντιστῆρας ἐπείγων 
Ἰνδικὰ Βάκχος ἄεθλα φέρων παρέθηκεν ἀγῶνι, 
διχθαδίην κνημῖδα καὶ Ἰνδῴης λίθον ἅλμης. 



Nonnus Epic., Dionysiaca 
Book 37, line 778

ἔννεπεν· ἐγρεμόθου δὲ λαβὼν πρεσβήια νίκης 
Αἰακὸς αὐχήεις χρυσέας κνημῖδας ἀείρων 
δῶκεν ἑῷ θεράποντι· καὶ ὕστερα δῶρα κομίζων 
Ἀστέριος κούφιζε δορικτήτην λίθον Ἰνδῶν. 



Nonnus Epic., Dionysiaca 
Book 38, line 10

Λῦτο δ' ἀγών· λαοὶ δὲ μετήιον ἔνδια λόχμης, 
καὶ σφετέραις κλισίῃσιν ὁμίλεον· ἀγρονόμοι δὲ 
Πᾶνες ἐναυλίζοντο χαραδραίοισι μελάθροις, 
αὐτοπαγῆ ναίοντες ἐρημάδος ἄντρα λεαίνης 
ἑσπέριοι· Σάτυροι δὲ δεδυκότες εἰς σπέος ἄρκτου 
θηγαλέοις ὀνύχεσσι καὶ οὐ τμητῆρι σιδήρῳ 
πετραίην ἐλάχειαν ἐκοιλαίνοντο χαμεύνην, 
εἰσόκεν ὄρθρος ἔλαμψε σελασφόρος, ἀρτιφανὲς δὲ 
ἀμφοτέροις ἀνέτελλε γαληναίης φάος Ἠοῦς, 
Ἰνδοῖς καὶ Σατύροισιν· ἐπεὶ τότε κυκλάδι νύσσῃ 
Μυγδονίου πολέμοιο καὶ Ἰνδῴοιο κυδοιμοῦ 
ἀμβολίην ἐτάνυσσεν ἕλιξ χρόνος· οὐδέ τις αὐτοῖς 
οὐ φόνος, οὐ τότε δῆρις· ἔκειτο δὲ τηλόθι χάρμης 
Βακχιὰς ἑξαέτηρος ἀραχνιόωσα βοείη. 



Nonnus Epic., Dionysiaca 
Book 38, line 11

καὶ σφετέραις κλισίῃσιν ὁμίλεον· ἀγρονόμοι δὲ 
Πᾶνες ἐναυλίζοντο χαραδραίοισι μελάθροις, 
αὐτοπαγῆ ναίοντες ἐρημάδος ἄντρα λεαίνης 
ἑσπέριοι· Σάτυροι δὲ δεδυκότες εἰς σπέος ἄρκτου 
θηγαλέοις ὀνύχεσσι καὶ οὐ τμητῆρι σιδήρῳ 
πετραίην ἐλάχειαν ἐκοιλαίνοντο χαμεύνην, 
εἰσόκεν ὄρθρος ἔλαμψε σελασφόρος, ἀρτιφανὲς δὲ 
ἀμφοτέροις ἀνέτελλε γαληναίης φάος Ἠοῦς, 
Ἰνδοῖς καὶ Σατύροισιν· ἐπεὶ τότε κυκλάδι νύσσῃ 
Μυγδονίου πολέμοιο καὶ Ἰνδῴοιο κυδοιμοῦ 
ἀμβολίην ἐτάνυσσεν ἕλιξ χρόνος· οὐδέ τις αὐτοῖς 
οὐ φόνος, οὐ τότε δῆρις· ἔκειτο δὲ τηλόθι χάρμης 
Βακχιὰς ἑξαέτηρος ἀραχνιόωσα βοείη. 



Nonnus Epic., Dionysiaca 
Book 38, line 48

καὶ Φρύγιον πολύιδριν ἀνείρετο μάντιν Ἐρεχθεύς, 
σύμβολα παπταίνων ὑπάτου Διός, εἰ πέλε χάρμης 
αἴσια δυσμενέεσσιν ἢ Ἰνδοφόνῳ Διονύσῳ, 
οὐ τόσον ὑσμίνης ποθέων τέλος, ὅσσον ἀκοῦσαι 
μυστιπόλοις ὀάροισι μεμηλότα μῦθον Ὀλύμπου, 
καὶ στίχας ἀστραίων ἑλίκων καὶ κυκλάδα μήνην, 
καὶ δύσιν ἠματίην Φαεθοντίδος ἄμμορον αἴγλης 
κλεπτομένης. 



Nonnus Epic., Dionysiaca 
Book 38, line 80

καὶ τότε μουνωθέντι φιλοσκοπέλῳ Διονύσῳ 
σύγγονος οὐρανόθεν Διὸς ἄγγελος ἤλυθεν Ἑρμῆς, 
καί τινα μῦθον ἔειπε παρηγορέων ἐπὶ νίκῃ· 
 “μὴ τρομέοις τόδε σῆμα, καὶ εἰ πέλεν ἠματίη νύξ· 
τοῦτό σοι, ἄτρομε Βάκχε, πατὴρ ἀνέφηνε Κρονίων 
νίκης Ἰνδοφόνοιο προάγγελον· ἠελίῳ γὰρ 
δεύτερον ἀστράπτοντι φεραυγέα Βάκχον ἐίσκω, 
καὶ θρασὺν ὀρφναίῃ μελανόχροον Ἰνδὸν ὀμίχλῃ· 
αἰθέρι γὰρ τύπος οὗτος ὁμοίιος· εὐφαέος δὲ 
ὡς ζόφος ἠμάλδυνε καλυπτομένης φάος ἠοῦς, 
καὶ πάλιν ἀντέλλων πυριφεγγέος ὑψόθι δίφρου 
Ἠέλιος ζοφόεσσαν ἀπηκόντιζεν ὀμίχλην, 
οὕτω σῶν βλεφάρων μάλα τηλόθι καὶ σὺ τινάξας 
Ταρταρίης ζοφόεσσαν Ἐρινύος ἄσκοπον ἀχλὺν 
ἀστράψεις κατ' ἄρηα τὸ δεύτερον ὡς Ὑπερίων. 



Nonnus Epic., Dionysiaca 
Book 39, line 7

ὄφρα μὲν εἰσέτι Βάκχος ἀκοσμήτων χύσιν ἄστρων 
θάμβεε καὶ Φαέθοντα δεδουπότα, πῶς παρὰ Κελτοῖς 
Ἑσπερίῳ πυρίκαυτος ἐπωλίσθησε ῥεέθρῳ, 
τόφρα δὲ νῆες ἵκανον ἐπήλυδες, ἃς ἐνὶ πόντῳ 
στοιχάδας ἰθύνοντες ἐς ἄρεα ναύμαχον Ἰνδῶν 
ἀκλύστῳ Ῥαδαμᾶνες ἐναυτίλλοντο θαλάσσῃ, 
πόντον ἀμοιβαίῃσιν ἐπιρρήσσοντες ἐρωαῖς 
ὑσμίνης ἐλατῆρες· ἐπειγομένῳ δὲ Λυαίῳ 
ὁλκάσιν ἀρτιτύποις ἐπεσύρισε πομπὸς ἀήτης. 



Nonnus Epic., Dionysiaca 
Book 39, line 21

καὶ στόλον ἀθρήσαντες ἀταρβέες ἔτρεμον Ἰνδοί, 
ἄρεα παπταίνοντες ἁλίκτυπον, ἄχρι καὶ αὐτοῦ 
γούνατα τολμήεντος ἐλύετο Δηριαδῆος. 



Nonnus Epic., Dionysiaca 
Book 39, line 25

ποιητῷ δὲ γέλωτι γαληναίοιο προσώπου 
Ἰνδὸς ἄναξ ἐκέλευσε τριηκοσίων ἀπὸ νήσων 
τῆς ἐλεφαντοβότοιο παρὰ σφυρὰ δύσβατα γαίης 
λαὸν ἄγειν· καὶ κραιπνὸς ἐς ἀτραπὸν ἤιε κῆρυξ, 
ποσσὶ πολυγνάμπτοισιν ἀπὸ χθονὸς εἰς χθόνα βαίνων 
καὶ στόλος ὀξὺς ἵκανε πολυσπερέων ἀπὸ νήσων 
κεκλομένου βασιλῆος· ὁ δὲ θρασὺς αὐχένα τείνων, 
ὁλκάδας εὐπήληκας ἐς ἄρεα πόντιον ἕλκων, 
λαὸν ὅλον θάρσυνε, καὶ ὑψινόῳ φάτο φωνῇ· 
 “ἀνέρες, οὓς ἀτίταλλεν ἐμὸς μενέχαρμος Ὑδάσπης, 
ἄρτι πάλιν μάρνασθε πεποιθότες· αἰθόμενον δὲ 




Nonnus Epic., Dionysiaca 
Book 39, line 45

εἰ γὰρ ἔην ῥόος οὗτος ἀπ' ἀλλοτρίου ποταμοῖο, 
μηδὲ πατὴρ ἐμὸς ἦεν ἀρήιος Ἰνδὸς Ὑδάσπης, 
καί κεν ἐγὼ τόδε χεῦμα χυτῆς ἔπλησα κονίης 
ὀδμὴν βοτρυόεσσαν ἀμαλδύνων Διονύσου, 
καὶ προχοὴν μεθύουσαν ἐμοῦ γενετῆρος ὁδεύων 
ποσσὶ κονιομένοισι διέτρεχον ἄβροχον ὕδωρ, 
οἷα παρ' Ἀργείοισι φατίζεται, ὡς Ἐνοσίχθων 
ξηρὸν ὕδωρ ποίησε, καὶ αὐσταλέου ποταμοῖο 
Ἰναχίην ἵππειος ὄνυξ ἐχάραξε κονίην. 



Nonnus Epic., Dionysiaca 
Book 39, line 80


καὶ προμάχοις Διόνυσος ἐκέκλετο θυιάδι φωνῇ· 
 “Ἄρεος ἄλκιμα τέκνα καὶ εὐθώρηκος Ἀθήνης, 
οἷς βίος ἔργα μόθοιο καὶ ἐλπίδες εἰσὶν ἀγῶνες, 
σπεύσατε καὶ κατὰ πόντον ἀιστῶσαι γένος Ἰνδῶν, 
εἰναλίην τελέσαντες ἐπιχθονίην μετὰ νίκην. 



Nonnus Epic., Dionysiaca 
Book 39, line 90

νόσφι φόβου μάρνασθε, Μιμαλλόνες· ὑγρομόθων γὰρ 
ἐλπίδες ἀντιβίων κενεαυχέες· εἰ δὲ μογήσας 
φύλοπιν οὐκ ἐτέλεσσεν ἐπὶ χθονὸς ὄρχαμος Ἰνδῶν, 
ἠλιβάτων λοφιῇσιν ἐφεδρήσσων ἐλεφάντων, 
ἀγχινεφής, ἀκίχητος, ἀνούτατος, ἠέρι γείτων . 



Nonnus Epic., Dionysiaca 
Book 39, line 97

                                                      .. 
οὐ μὲν ἐγὼ προμάχων ποτὲ δεύομαι, οὐδὲ καλέσσω 
ἄλλον ἀοσσητῆρα μετὰ Κρονίωνα τοκῆα, 
ἡνίοχον πόντοιο καὶ αἰθέρος· ἢν δ' ἐθελήσω, 
γνωτὸν ἐμοῦ Κρονίδαο Ποσειδάωνα κορύσσω 
Ἰνδῴην στίχα πᾶσαν ἀμαλδύνοντα τριαίνῃ· 
καὶ πρόμον εὐρυγένειον, ἀπόσπορον Ἐννοσιγαίου, 
Γλαῦκον ἔχω συνάεθλον, ἐμῆς ἅτε γείτονα Θήβης, 
πόντιον Ἀονίης Ἀνθηδόνος ἀστὸν ἀρούρης· 
Γλαῦκον ἔχω καὶ Φόρκυν· ἱμασσομένην δὲ θαλάσσῃ 
ὁλκάδα Δηριάδαο κατακρύψει Μελικέρτης, 
κυδαίνων Διόνυσον ὁμόγνιον, οὗ ποτε μήτηρ 
νήπιον ἔτρεφε Βάκχον, ἐπεὶ πόρε ποντιὰς Ἰνὼ 
ἓν γλάγος ἀμφοτέροισι, Παλαίμονι καὶ Διονύσῳ· 
μαντιπόλου δὲ γέροντος, ὃς ἡμετέρην ποτὲ νίκην 




Nonnus Epic., Dionysiaca 
Book 39, line 118

                                           ἀλλὰ σιωπῇ 
ἔκτοθεν εὐθύρσοιο καὶ Ἰνδῴοιο κυδοιμοῦ 
μιμνέτω ἠρεμέων θρασὺς Αἴολος, ἠθάδι δεσμῷ 
ἀσκὸν ἐπισφίγξας ἀνεμώδεα, μηδ' ἐνὶ πόντῳ 
ἄσθμασιν Ἰνδοφόνοισιν ἀριστεύσωσιν ἀῆται· 
ἀλλὰ μόθον τελέσω νηοφθόρα θύρσα τιταίνων. 



Nonnus Epic., Dionysiaca 
Book 39, line 134

τοῖσι δὲ μαρναμένοισιν ἔην κλόνος, ὦρτο δ' ἰωὴ 
κεκλομένων· καὶ λαὸς ἐθήμονι μάρνατο τέχνῃ 
κυκλώσας στεφανηδὸν ὅλον στρατόν, ἐν δ' ἄρα μέσσῳ 
νηυσὶν ὁμοζυγέεσσιν ἐμιτρώθη στόλος Ἰνδῶν 
εἰς λίνον ἐργομένων νεπόδων τύπον· Αἰακίδαις δὲ 
Αἰακὸς ὑγρὸν ἄρηα προθεσπίζων Σαλαμῖνος 
ἀρχόμενος πολέμοιο θεουδέα ῥήξατο φωνήν· 
 “εἰ πάρος ἡμετέρην ἀίων ἱκετήσιον ἠχὼ 
ἄσπορον εὐρυάλωος ἀπήλασας αὐχμὸν ἀρούρης, 
διψαλέην ἐπὶ γαῖαν ἄγων βιοτήσιον ὕδωρ, 
δὸς πάλιν ὀψιτέλεστον ἴσην χάριν, ὑέτιε Ζεῦ, 
ὕδατι κυδαίνων με καὶ ἐνθάδε· καί τις ἐνίψῃ 
νίκην ἡμετέρην δεδοκημένος· ‘ὡς ἐνὶ γαίῃ 




Nonnus Epic., Dionysiaca 
Book 39, line 146

ἄλλος ἀνὴρ λέξειεν Ἀχαιικός· ‘εἰν ἑνὶ θεσμῷ 
Αἰακὸς Ἰνδοφόνος φυσίζοος· ἀμφότερον γάρ, 
κείρων ἐχθρὰ κάρηνα καὶ αὔλακι καρπὸν ὀπάσσας 
χάρμα πόρεν Δήμητρι καὶ εὐφροσύνην Διονύσῳ. 



Nonnus Epic., Dionysiaca 
Book 39, line 163

πέμπε μοι αἰετὸν ὄρνιν ἐμῆς κήρυκα γενέθλης 
δεξιτερὸν προμάχοισι καὶ ὑμετέρῳ Διονύσῳ· 
ἄλλος δ' ἀντιβίοισιν ἀριστερὸς ὄρνις ἱκέσθω· 
σύμβολα δ' ἀμφοτέροις ἑτερότροπα ταῦτα γενέσθω· 
τὸν μὲν ἐσαθρήσω πεφορημένον ἅρπαγι ταρσῷ 
θηγαλέων ὀνύχων κεχαραγμένον ὀξέι κέντρῳ 
νεκρὸν ὄφιν περίμετρον ἀερτάζοντα κεράστην, 
δυσμενέος κερόεντος ἀπαγγέλλοντα τελευτήν· 
λαῷ δ' ἀντιβίων ἕτερος μελανόχροος ἔλθῃ 
κυανέαις πτερύγεσσι προθεσπίζων φόνον Ἰνδῶν, 
αὐτόματον θανάτοιο φέρων τύπον· ἢν δ' ἐθελήσῃς, 
βρονταίοις πατάγοισιν ἐμὴν μαντεύεο νίκην, 
καὶ στεροπὴν Βρομίοιο λεχώια φέγγεα πέμπων 
υἱέα σεῖο γέραιρε πάλιν πυρί, δυσμενέων δὲ 
ὁλκάδας εὐπήληκας ὀιστεύσωσι κεραυνοί. 



Nonnus Epic., Dionysiaca 
Book 39, line 224

        –   –   –  
 καὶ διδύμαις στρατιῇσιν ἐπέκτυπε πόντιος ἄρης   
χερσαίην μετὰ δῆριν, ἁλιρροίζῳ δ' ἀλαλητῷ 
ὁλκάσι Βακχείῃσιν ἐπέρρεον ὁλκάδες Ἰνδῶν· 
καὶ φόνος ἦν ἑκάτερθε, καὶ ἔζεε κύματα λύθρῳ, 
καὶ πολὺς ἀμφοτέρων στρατὸς ἤριπεν· ἀρτιχύτῳ δὲ 
αἵματι κυανέης ἐρυθαίνετο νῶτα θαλάσσης. 



Nonnus Epic., Dionysiaca 
Book 39, line 252

        –   –   –  
 καὶ φονίαις λιβάδεσσιν ἐφοινίχθη Μελικέρτης· 
Λευκοθέη δ' ὀλόλυξε, τιθηνήτειρα Λυαίου, 
αὐχένα γαῦρον ἔχουσα, καὶ Ἰνδοφόνου περὶ νίκης 
ἄνθεϊ φυκιόεντι κόμην ἐστέψατο Νύμφη· 
καὶ Θέτις ἀκρήδεμνος ὑπερκύψασα θαλάσσης 
χεῖρας ἐρεισαμένη καὶ Δωρίδι καὶ Πανοπείῃ 
ἄσμενον ὄμμα τίταινεν ἐπ' εὐθύρσῳ Διονύσῳ. 



Nonnus Epic., Dionysiaca 
Book 39, line 261

καὶ βυθίη Γαλάτεια θαλασσαίου διὰ κόλπου 
ἡμιφανὴς πεφόρητο διαξύουσα γαλήνην, 
καὶ φονίου Κύκλωπος ἁλιπτοίητον ἐνυὼ 
δερκομένη δεδόνητο, φόβῳ δ' ἤμειψε παρειάς· 
ἔλπετο γὰρ Πολύφημον ἰδεῖν κατὰ φύλοπιν Ἰνδῶν 
ἀντία Δηριάδαο συναιχμάζοντα Λυαίῳ· 
ταρβαλέη δ' ἱκέτευε θαλασσαίην Ἀφροδίτην 
υἷα Ποσειδάωνος ἀριστεύοντα σαῶσαι, 
καὶ γενέτην φιλότιμον ἐφ' υἱέι Κυανοχαίτην 
μαρναμένου λιτάνευε προασπίζειν Πολυφήμου. 



Nonnus Epic., Dionysiaca 
Book 39, line 283

εἰς χρόνον ἑπταέτηρον ἔχεις πολύκυκλον ἀγῶνα, 
βόσκων ἀλλοπρόσαλλον ἀτέρμονος ἐλπίδα χάρμης, 
ὅττι τεοῦ μεγάλοιο προασπιστῆρες ἀγῶνος 
πάντες ἑνὸς χατέουσιν ἀνικήτου Πολυφήμου· 
εἰ δὲ τεὴν ἐπὶ δῆριν ἐμὸς πάις ἵκετο Κύκλωψ, 
πατρῴην δ' ἐλέλιζεν ἐμῆς γλωχῖνα τριαίνης, 
καί κεν ὑπὲρ πεδίοιο συναιχμάζων Διονύσῳ 
στήθεα βουκεράοιο διέθλασε Δηριαδῆος, 
καὶ πολὺν ἄλλον ὅμιλον ἐμῷ τριόδοντι δαΐζων 
εἰς μίαν ἠριγένειαν ὅλον γένος ἔκτανεν Ἰνδῶν. 



Nonnus Epic., Dionysiaca 
Book 39, line 386

        –   –   –  
 καὶ στόλον ἰθύνουσα μαχήμονα Δηριαδῆος   
ὑσμίνης Ἔρις ἦρχε· Διωνύσοιο δὲ νηῶν 
Ἰνδοφόνῳ παλάμῃ κολπώσατο λαίφεα Νίκη. 



Nonnus Epic., Dionysiaca 
Book 39, line 402

χάζετο δ' Ἰνδὸς ὅμιλος ἐπὶ χθόνα, πόντον ἐάσσας· 
καὶ Φαέθων ἐγέλασσεν, ὅτι προτέρους μετὰ δεσμοὺς 
ἐκ πυρὸς Ἡφαίστοιο πάλιν φύγε ναύμαχος Ἄρης. 



Nonnus Epic., Dionysiaca 
Book 40, line 5

Οὐ δὲ Δίκην ἀλέεινε πανόψιον, οὐδὲ καὶ αὐτῆς 
ἀρραγέος κλωστῆρος ἀκαμπέα νήματα Μοίρης· 
ἀλλά μιν ἀθρήσασα πεφυζότα Παλλὰς Ἀθήνη –  
ἕζετο γὰρ κατὰ πόντον ἐπὶ προβλῆτος ἐρίπνης, 
ναύμαχον εἰσορόωσα κορυσσομένων μόθον Ἰνδῶν –  
ἐκ σκοπιῆς ἀνέπαλτο, καὶ ἄρσενα δύσατο μορφήν· 
κλεψινόοις δ' ὀάροισι παρήπαφεν ὄρχαμον Ἰνδῶν, 
Μορρέος εἶδος ἔχουσα, χαριζομένη δὲ Λυαίῳ 
Δηριάδην ἀνέκοψε, καὶ ὡς ἀλέγουσα κυδοιμοῦ 
φρικτὸν ἀπερροίβδησεν ἔπος πολυμεμφέι φωνῇ· 
 “φεύγεις, Δηριάδη; 



Nonnus Epic., Dionysiaca 
Book 40, line 96

             χρονίην δὲ θεοὶ μετὰ φύλοπιν Ἰνδῶν 
σὺν Διὶ παμμεδέοντι πάλιν νόστησαν Ὀλύμπῳ. 



Nonnus Epic., Dionysiaca 
Book 40, line 171

                                       εἴπατε, Μοῖραι· 
τίς φθόνος Ἰνδῴην πόλιν ἔπραθε; 



Nonnus Epic., Dionysiaca 
Book 40, line 182

εἰς ἐμὲ θωρήχθη καὶ ἐμὸς γάμος· ἡμετέρου γὰρ 
Μορρέος ἱμείροντος ἐσυλήθη πόλις Ἰνδῶν· 
πατρὸς ἐνοσφίσθην χάριν ἀνέρος· ἡ πρὶν ἀγήνωρ 
καὶ θυγάτηρ βασιλῆος, ἐγώ ποτε δεσπότις Ἰνδῶν, 
ἔσσομαι ἀμφιπόλων καὶ ἐγὼ μία· καὶ τάχα δειλὴ 
δμωίδα Χαλκομέδειαν ἐμὴν δέσποιναν ἐνίψω. 



Nonnus Epic., Dionysiaca 
Book 40, line 187

σήμερον Ἰνδὸν ἔδεθλον ἔχεις, ἀπατήλιε Μορρεῦ· 
αὔριον αὐτοκέλευστος ἐλεύσεαι εἰς χθόνα Λυδῶν, 
Χαλκομέδης διὰ κάλλος ὑποδρήσσων Διονύσῳ. 



Nonnus Epic., Dionysiaca 
Book 40, line 199

Δηριάδης τέθνηκεν· ἐσυλήθη πόλις Ἰνδῶν, 
ἀρραγὲς ἤριπε τεῖχος ἐμῆς χθονός· αἴθε καὶ αὐτὴν 
Βάκχος ἑλὼν ὀλέσῃ με σὺν ὀλλυμένῳ παρακοίτῃ, 
καί με λαβὼν ῥίψειεν ἐς ὠκυρέεθρον Ὑδάσπην, 
γαῖαν ἀναινομένην· ἐχέτω δέ με πενθερὸν ὕδωρ, 
Δηριάδην δ' ἐσίδω καὶ ἐν ὕδασι· μηδὲ νοήσω 
Πρωτονόην ἀέκουσαν ἐφεσπομένην Διονύσῳ· 
μή ποτε Χειροβίης ἕτερον γόον οἰκτρὸν ἀκούσω 
ἑλκομένης ἐς ἔρωτα δορικτήτων ὑμεναίων· 
μὴ πόσιν ἄλλον ἴδοιμι μετ' ἀνέρα Δηριαδῆα. 



Nonnus Epic., Dionysiaca 
Book 40, line 217

Βάκχοι δ' ἐκροτάλιζον ἀπορρίψαντες ἐνυώ, 
τοῖον ἔπος βοόωντες ὁμογλώσσων ἀπὸ λαιμῶν· 
 “ἠράμεθα μέγα κῦδος· ἐπέφνομεν ὄρχαμον Ἰνδῶν. 



Nonnus Epic., Dionysiaca 
Book 40, line 235

παυσάμενος δὲ πόνοιο, καὶ ὕδατι γυῖα καθήρας, 
ὤπασε λισσομένοισι θεουδέα κοίρανον Ἰνδοῖς, 
κρινάμενος Μωδαῖον· ἐπὶ ξυνῷ δὲ κυπέλλῳ 
Βάκχοις δαινυμένοισι μιῆς ἥψαντο τραπέζης 
ξανθὸν ὕδωρ πίνοντες ἀπ' οἰνοπόρου ποταμοῖο. 



Nonnus Epic., Dionysiaca 
Book 40, line 252

ἀλλ' ὅτε λυσιπόνοιο παρήλυθε κῶμος ἑορτῆς, 
νίκης ληίδα πᾶσαν ἑλὼν μετὰ φύλοπιν Ἰνδῶν 
ἀρχαίης Διόνυσος ἑῆς ἐμνήσατο πάτρης, 
λύσας ἑπταέτηρα θεμείλια δηιοτῆτος. 



Nonnus Epic., Dionysiaca 
Book 40, line 256

καὶ δηίων ὅλον ὄλβον ἐληίζοντο μαχηταί, 
ὧν ὁ μὲν Ἰνδὸν ἴασπιν, ὁ δὲ γραπτῆς ὑακίνθου 
Φοιβάδος εἶχε μέταλλα καὶ ἔγχλοα νῶτα μαράγδου· 
ἄλλος ἐυκρήπιδος ὑπὸ σκοπιῇσιν Ἰμαίου 
ὄρθιον ἴχνος ἔπειγε δορικτήτων ἐλεφάντων, 
ὃς δὲ παρ' Ἡμωδοῖο βαθυσπήλυγγι κολώνῃ 
ἤλασεν Ἰνδῴων μετανάστιον ἅρμα λεόντων 
κυδιόων, ἕτερος δὲ κατ' αὐχένος ἅμμα πεδήσας 
Μυγδονίην ἔσπευδεν ἐς ᾐόνα πόρδαλιν ἕλκειν· 
καὶ Σάτυρος πεφόρητο, φιλακρήτῳ δὲ πετήλῳ 
στικτὸν ἔχων προκέλευθον ἐκώμασε τίγριν ἱμάσσων· 




Nonnus Epic., Dionysiaca 
Book 40, line 277

καὶ στρατιῇ Διόνυσος ἐδάσσατο ληίδα χάρμης 
λαὸν ὅλον συνάεθλον ὑπότροπον οἴκαδε πέμπων 
Ἰνδῴην μετὰ δῆριν· ἀπεσσεύοντο δὲ λαοὶ 
μάρμαρα κουφίζοντες ἑώια δῶρα θαλάσσης 
ὄρνεά τ' αἰολόμορφα· παλιννόστῳ δὲ πορείῃ 
κῶμον ἀνευάζοντες ἀνικήτῳ Διονύσῳ 
πάντες ἐβακχεύοντο, πολυκμήτοιο λιπόντες 
μνῆστιν ὅλην πολέμοιο, Βορειάδι σύνδρομον αὔρῃ 
σκιδναμένην· καὶ ἕκαστος ἔχων ἀναθήματα νίκης 
ὄψιμον εἰς δόμον ἦλθε παλίνδρομος. 



Nonnus Epic., Dionysiaca 
Book 40, line 292

                                           αὐτὰρ ὁ μούνοις 
Βάκχος ἑοῖς Σατύροισι καὶ Ἰνδοφόνοις ἅμα Βάκχαις 
Καυκασίην μετὰ δῆριν Ἀμαζονίου ποταμοῖο 
Ἀρραβίης ἐπέβαινε τὸ δεύτερον, ἧχι θαμίζων 
λαὸν ἀβακχεύτων Ἀράβων ἐδίδαξεν ἀείρειν 
μυστιπόλους νάρθηκας· ἀεξιφύτοιο δὲ λόχμης 
Νύσια βοτρυόεντι κατέστεφεν οὔρεα καρπῷ. 



Nonnus Epic., Dionysiaca 
Book 40, line 441

ἁγνὸν ἀνυμφεύτοιο γένος χθονός, ὧν τότε μορφὴν 
αὐτομάτην ὤδινεν ἀνήροτος ἄσπορος ἰλύς· 
οἳ πόλιν ἰσοτύπων δαπέδων αὐτόχθονι τέχνῃ 
πετραίοις ἀτίνακτον ἐπυργώσαντο θεμέθλοις, 
ὁππότε πηγαίῃσι παρ' εὐύδροισι χαμεύναις 
ἠελίου πυρόεντος ἱμασσομένης χθονὸς ἀτμῷ 
τερψινόου ληθαῖον ἀμεργόμενοι πτερὸν Ὕπνου 
εὗδον ὁμοῦ, κραδίῃ δὲ φιλόπτολιν οἶστρον ἀέξων 
Γηγενέων στατὸν ἴχνος ἐπῃώρησα καρήνῳ, 
καὶ βροτέου σκιοειδὲς ἔχων ἴνδαλμα προσώπου 
θέσφατον ὀμφήεντος ἀνήρυγον ἀνθερεῶνος· 
’ὕπνον ἀποσκεδάσαντες ἀεργέα, παῖδες ἀρούρης,   
τεύξατέ μοι ξένον ἅρμα βατῆς ἁλός· ὀξυτόμοις δὲ 
κόψατέ μοι πελέκεσσι ῥάχιν πιτυώδεος ὕλης· 
τεύξατέ μοι σοφὸν ἔργον· ὑπὸ σταμίνεσσι δὲ πυκνοῖς 
ἴκρια γομφώσαντες ἐπασσυτέρῳ τινὶ κόσμῳ 
συμφερτὴν ἀτίνακτον ἀρηρότι δήσατε δεσμῷ 
δίφρον ἁλός, σχεδίην πρωτόπλοον, ἣ διὰ πόντου 
ὑμέας ὀχλίζειε· καὶ ἀγκύλον ἄκρον ἀπ' ἄκρου 




Nonnus Epic., Dionysiaca 
Book 41, line 65

σύζυγα μορφώσασα σοφὸν τόκον ἄσπορος ὠδὶς 
ἔμπνοον ἐψύχωσε γονὴν ἐγκύμονι πηλῷ,   
οἷς Φύσις εἶδος ὄπασσε τελεσφόρον· ἀρχεγόνου γὰρ 
Κέκροπος οὐ τύπον εἶχον, ὃς ἰοβόλῳ ποδὸς ὁλκῷ 
γαῖαν ἐπιξύων ὀφιώδεϊ σύρετο ταρσῷ, 
νέρθε δράκων, καὶ ὕπερθεν ἀπ' ἰξύος ἄχρι καρήνου 
ἀλλοφυὴς ἀτέλεστος ἐφαίνετο δίχροος ἀνήρ· 
οὐ τύπον ἄγριον εἶχον Ἐρεχθέος, ὃν τέκε γαίης 
αὔλακι νυμφεύσας γαμίην Ἥφαιστος ἐέρσην· 
ἀλλὰ θεῶν ἴνδαλμα γονῆς αὐτόχθονι ῥίζῃ 
πρωτοφανὴς χρύσειος ἐμαιώθη στάχυς ἀνδρῶν. 



Nonnus Epic., Dionysiaca 
Book 42, line 145

                                    ἆ μέγα θαῦμα, 
παρθένον ἔτρεμε Βάκχος, ὃν ἔτρεμε φῦλα Γιγάντων· 
Γηγενέων ὀλετῆρα φόβος νίκησεν Ἐρώτων· 
τοσσατίων δ' ἤμησεν ἀρειμανέων γένος Ἰνδῶν, 
καὶ μίαν ἱμερόεσσαν ἀνάλκιδα δείδιε κούρην, 
δείδιε θηλυτέρην ἁπαλόχροον· ἐν δὲ κολώναις 
θηρονόμῳ νάρθηκι κατεπρήυνε λεόντων 
φρικαλέον μύκημα, καὶ ἔτρεμε θῆλυν ἀπειλήν. 



Nonnus Epic., Dionysiaca 
Book 42, line 239

ἕδνα δὲ σεῖο πόθοιο, τεῆς κειμήλια νύμφης, 
μὴ λίθον Ἰνδῴην, μὴ μάργαρα χειρὶ τινάξῃς,   
οἷα γυναιμανέοντι πέλει θέμις· εἰς Παφίην γὰρ 
ἀμφιέπεις τεὸν εἶδος ἐπάρκιον, εὐαφέος δὲ 
κάλλεος ἱμείρουσι καὶ οὐ χρυσοῖο γυναῖκες. 



Nonnus Epic., Dionysiaca 
Book 42, line 333

οὕτω καὶ Διόνυσος, ἔχων ἰνδάλματα μόχθων, 
μιμηλῷ πτερόεντα νόον πόμπευεν ὀνείρῳ, 
καὶ σκιεροῖσι γάμοισιν ὁμίλεεν. 



Nonnus Epic., Dionysiaca 
Book 42, line 358

καί ποτε μουνωθεῖσαν Ἀδώνιδος ἄζυγα κούρην 
ἀθρήσας σχεδὸν ἦλθε, καὶ ἀνδρομέης ἀπὸ μορφῆς 
εἶδος ἑὸν μετάμειψε, καὶ ὡς θεὸς ἵστατο κούρῃ· 
καί οἱ ἑὸν γένος εἶπε καὶ οὔνομα, καὶ φόνον Ἰνδῶν, 
καὶ χορὸν ἀμπελόεντα, καὶ ἡδυπότου χύσιν οἴνου, 
ὅττι μιν ἀνδράσιν εὗρε· φιλοστόργῳ δὲ μενοινῇ 
θάρσος ἀναιδείῃ κεράσας ἀλλότριον αἰδοῦς 
τοίην ποικιλόμυθον ὑποσσαίνων φάτο φωνήν· 
 “παρθένε, σὸν δι' ἔρωτα καὶ οὐρανὸν οὐκέτι ναίω· 
σῶν πατέρων σπήλυγγες ἀρείονές εἰσιν Ὀλύμπου·   
πατρίδα σὴν φιλέω πλέον αἰθέρος· οὐ μενεαίνω 
σκῆπτρα Διὸς γενετῆρος, ὅσον Βερόης ὑμεναίους· 
ἀμβροσίης σέο κάλλος ὑπέρτερον· αἰθερίου δὲ 




Nonnus Epic., Dionysiaca 
Book 43, line 137

ἀλλὰ πάλιν μάρνασθε, Μιμαλλόνες, ἠθάδι νίκῃ 
θαρσαλέαι· κταμένων δὲ νεόρρυτον αἷμα Γιγάντων 
νεβρὶς ἐμὴ μεθέπουσα μελαίνεται· εἰσέτι δ' αὐτὴ 
ἀντολίη τρομέει με, καὶ εἰς πέδον αὐχένα κάμπτει 
Ἰνδὸς ἄρης, Βρομίῳ δὲ λιτήσια δάκρυα λείβων, 
δάκρυα κυματόεντα, γέρων ἔφριξεν Ὑδάσπης. 



Nonnus Epic., Dionysiaca 
Book 43, line 165

Αἰθιόπων δὲ φάλαγγας ἐρύσσατε καὶ στίχας Ἰνδῶν, 
ληίδα Νηρεΐδεσσι, κακογλώσσοιο δὲ νύμφης 
Δωρίδι δούλια τέκνα κομίσσατε Κασσιεπείης, 
ποινὴν ὀψιτέλεστον· ἀμαιμακέτῳ δὲ ῥεέθρῳ 
Ὠκεανὸς πυρόεντα λελουμένον ἀστέρα Μαίρης, 
ληναίης προκέλευθον ἀκοιμήτοιο χορείης, 
Σείριον ἀμπελόεντα μεταστήσειεν Ὀλύμπου. 



Nonnus Epic., Dionysiaca 
Book 43, line 227

Πρωτεὺς δ' Ἴσθμιον οἶδμα λιπὼν Παλληνίδος ἅλμης 
εἰναλίῳ θώρηκι κορύσσετο, δέρματι φώκης· 
ἀμφὶ δέ μιν στεφανηδὸν ἐπέρρεον αἴθοπες Ἰνδοὶ 
Βάκχου κεκλομένοιο, καὶ οὐλοκόμων στίχες ἀνδρῶν 
φωκάων πολύμορφον ἐπηχύναντο νομῆα. 



Nonnus Epic., Dionysiaca 
Book 43, line 243

δενδρώσας ἑὰ γυῖα, τινασσομένων δὲ πετήλων 
ψευδαλέον ψιθύρισμα Βορειάδι σύρισεν αὔρῃ· 
καὶ γραπταῖς φολίδεσσι κεκασμένα νῶτα χαράξας 
εἷρπε δράκων, μεσάτου δὲ πιεζομένου κενεῶνος 
σπεῖραν ἀνῃώρησεν, ὑπ' ὀρχηστῆρι δὲ παλμῷ 
ἄκρα τιταινομένης ἐλελίζετο κυκλάδος οὐρῆς, 
καὶ κεφαλὴν ὤρθωσεν, ἀποπτύων δὲ γενείων 
ἰὸν ἀκοντιστῆρα κεχηνότι σύρισε λαιμῷ· 
καὶ δέμας ἀλλοπρόσαλλον ἔχων σκιοειδέι μορφῇ 
φρῖξε λέων, σύτο κάπρος, ὕδωρ ῥέε· καὶ χορὸς Ἰνδῶν 
ὑγρὸν ἀπειλητῆρι ῥόον σφηκώσατο δεσμῷ 
χερσὶν ὀλισθηρῇσιν ἔχων ἀπατήλιον ὕδωρ· 
κερδαλέος δὲ γέρων πολυδαίδαλον εἶδος ἀμείβων   
εἶχε Περικλυμένοιο πολύτροπα δαίδαλα μορφῆς, 
ὃν κτάνεν Ἡρακλέης, ὅτε δάκτυλα δισσὰ συνάψας 
ψευδαλέον μίμημα νόθης ἔθραυσε μελίσσης. 



Nonnus Epic., Dionysiaca 
Book 43, line 445

                                 ἀπ' Ἀσσυρίοιο δὲ κόλπου 
ἁβροχίτων Διόνυσος ἀνήιεν εἰς χθόνα Λυδῶν 
Πακτωλοῦ παρὰ πέζαν, ὅπῃ χρυσαυγέι πηλῷ   
ἀφνειαῖς λιβάδεσσι μέλαν φοινίσσεται ὕδωρ· 
Μαιονίης δ' ἐπέβαινε, καὶ ἵστατο μητέρι Ῥείῃ 
Ἰνδῴης ὀρέγων βασιλήια δῶρα θαλάσσης. 



Nonnus Epic., Dionysiaca 
Book 44, line 236

ἤδη δ' ἀμφὶ τένοντας Ἐρυθραίων δονακήων 
κέκλιται ἔνθα καὶ ἔνθα, τεῆς αὐτάγγελος ἀλκῆς,   
Ἰνδῶν νεκρὸς ὅμιλος, ἀναινομένῳ δὲ ῥεέθρῳ 
ἄφρονα Δηριαδῆα πατὴρ ἔκρυψεν Ὑδάσπης 
ἔγχεϊ κισσήεντι τετυμμένον· αὐτὰρ ὁ φεύγων 
πατρῴῳ βαρύθοντι κατηφέι πῖπτε ῥεέθρῳ· 
Τυρσηνοὶ δεδάασι τεὸν σθένος, ὁππότε νηῶν 
ὄρθιος ἱστὸς ἄμειπτο καὶ ἀμπελόεις πέλεν ὄρπηξ 
αὐτοτελής, τὸ δὲ λαῖφος ὑπὸ σκιεροῖσι πετήλοις 
ἡμερίδων εὔβοτρυς ἀνηέξητο καλύπτρη, 
καὶ πρότονοι σύριζον ἐχιδνήεντι κορύμβῳ 




Nonnus Epic., Dionysiaca 
Book 44, line 251

καὶ νέκυς ὑμετέρῳ βεβολημένος ὀξέι θύρσῳ 
χεύμασιν Ἀσσυρίοισι καλύπτεται Ἰνδὸς Ὀρόντης, 
εἰσέτι δειμαίνων καὶ ἐν ὕδασιν οὔνομα Βάκχου. 



Nonnus Epic., Dionysiaca 
Book 45, line 125

ἀλλὰ δόλῳ Διόνυσος ἐπίκλοπον εἶδος ἀμείψας 
Τυρσηνοὺς ἀπάφησε· νόθην δ' ὑπεδύσατο μορφήν, 
ἱμερόεις ἅτε κοῦρος ἔχων ἀχάρακτον ὑπήνην, 
αὐχένι κόσμον ἔχων χρυσήλατον· ἀμφὶ δὲ κόρσην 
στέμματος ἀστράπτοντος ἔην αὐτόσσυτος αἴγλη 
λυχνίδος ἀσβέστοιο, καὶ ἔγχλοα νῶτα μαράγδου, 
καὶ λίθος Ἰνδῴη χαροπῆς ἀμάρυγμα θαλάσσης· 
καὶ χροῒ δύσατο πέπλα φαάντερα κυκλάδος Ἠοῦς 
ἄρτι χαρασσομένης, Τυρίῃ πεπαλαγμένα κόχλῳ. 



Nonnus Epic., Dionysiaca 
Book 46, line 71

μοῦνον ἐμῆς κύδαινε μελισταγὲς ἄνθος ὀπώρης· 
μὴ ποτὸν ἀμπελόεντος ἀτιμήσῃς Διονύσου. 
Ἰνδοφόνῳ Βρομίῳ μὴ μάρναο, θηλυτέρῃ δέ, 
εἰ δύνασαι, πολέμιζε μιῇ ῥηξήνορι Βάκχῃ. 



Nonnus Epic., Dionysiaca 
Book 47, line 505

μὴ κισσῷ δρεπάνην ἰσάζετε· καὶ γὰρ ἀρείων 
Βάκχου θυρσοφόρου δρεπανηφόρος ἔπλετο Περσεύς· 
εἰ στρατὸν Ἰνδὸν ἔπεφνεν, ἀέθλιον ἶσον ἐνίψω 
Γοργοφόνῳ Περσῆι καὶ Ἰνδοφόνῳ Διονύσῳ. 



Nonnus Epic., Dionysiaca 
Book 47, line 624

καὶ σὺ μέγα φρονέων δρεπανηφόρε παύεο Περσεῦ· 
Γοργόνος οὐ μόθος οὗτος ὀλίζονος, οὐ μία νύμφη 
Ἀνδρομέδη βαρύδεσμος ἀέθλιον· ἀλλὰ Λυαίῳ 
δῆριν ἄγεις, ὃς Ζηνὸς ἔχει γένος, ᾧ ποτε μούνῳ 
Ῥείη μαζὸν ὄρεξε φερέσβιον, ὅν ποτε πυρσῷ 
ἀστεροπῆς γαμίης μαιώσατο μειλιχίη φλόξ, 
ὃν δύσις, ὃν θάμβησεν Ἑωσφόρος, ᾧ στίχες Ἰνδῶν 
εἴκαθον, ὃν τρομέων καὶ Δηριάδης καὶ Ὀρόντης 
ἠλιβάτων ἀπέλεθρον ἔχων ἴνδαλμα Γιγάντων 
ἤριπεν, ᾧ θρασὺς Ἄλπος ὑπώκλασεν, υἱὸς ἀρούρης, 
ἀγχινεφὲς περίμετρον ἔχων δέμας, ᾧ γόνυ κάμπτει 
λαὸς Ἄραψ, Σικελὸς δὲ μελίζεται εἰσέτι ναύτης 
Τυρσηνῶν νόθον εἶδος ἁλίδρομον, ὧν ποτε μορφὴν 
ἀνδρομέην ἤμειψα μετάτροπον, ἀντὶ δὲ φωτῶν 
ἰχθύες ὀρχηστῆρες ἐπισκαίρουσι θαλάσσῃ. 



Nonnus Epic., Dionysiaca 
Book 48, line 9

καὶ δολίας ἀνέφαινε λιτὰς παμμήτορι Γαίῃ, 
ἔργα Διὸς βοόωσα καὶ ἠνορέην Διονύσου 
Γηγενέων ὀλέσαντος ἀμετρήτων νέφος Ἰνδῶν· 
καὶ Σεμέλης ὅτε παῖδα φερέσβιος ἔκλυε μήτηρ 
Ἰνδῴην ταχύποτμον ἀιστώσαντα γενέθλην, 
μνησαμένη τεκέων πλέον ἔστενεν· ἀμφὶ δὲ Βάκχῳ 
αὐτογόνων θώρηξεν † ὀρίδρομα φῦλα Γιγάντων, 
ὑψιλόφους ἕο παῖδας ἀνοιστρήσασα κυδοιμῷ· 
 “παῖδες ἐμοί, μάρνασθε κορυμβοφόρῳ Διονύσῳ 
ἠλιβάτοις σκοπέλοισιν, ἐμῆς δ' ὀλετῆρα γενέθλης 
Ἰνδοφόνον Διὸς υἷα κιχήσατε· μηδὲ νοήσω   
σὺν Διὶ κοιρανέοντα νόθον σκηπτοῦχον Ὀλύμπου. 



Nonnus Epic., Paraphrasis sancti evangelii Joannei (fort. auctore Nonno alio) (2045: 002)
“Paraphrasis s. evangelii Ioannei”, Ed. Scheindler, A.
Leipzig: Teubner, 1881.
Demonstratio 19, line 207

ἦλθε δὲ καὶ Νικόδημος, ὃς ἤλυθε νυκτὸς ὁδίτης 
εἰς μέγαρον Χριστοῖο φυλασσομένῳ ποδὶ βαίνων, 
σμύρναν ἄγων θυόεσσαν, Ἐρυθραίοιο δὲ κόλπου 
Ἰνδῴης ἀλόην δονακοτρεφὲς ἔρνος ἀρούρης, 
λίτρας τὰς καλέουσι φατιζομένῳ τινὶ μέτρῳ 
ἄχρι μιῆς ζαθέης ἑκατοντάδος· ὧν ἅμα καρπῷ 
λεπταλέαις ὀθόνῃσιν ἐμιτρώσαντο θανόντος 
σῶμα πολυπλέκτων ἑλίκων εὐώδεϊ δεσμῷ, 
ὡς ἔθος Ἑβραίοις ἐπιτύμβια θεσμὰ φυλάσσειν. 

\end{greek}

\section{Asclepius of Tralles}
\blockquote[From Wikipedia\footnote{\url{http://en.wikipedia.org/wiki/Asclepius_of_Tralles}}]{

Asclepius of Tralles (Greek: Ἀσκληπιός; died c. 560-570) was a student of Ammonius Hermiae. Two works of his survive:

    Commentary on Aristotle's Metaphysics, books I-VII (In Aristotelis metaphysicorum libros Α - Ζ (1 - 7) commentaria, ed. Michael Hayduck, Commentaria in Aristotelem Graeca, VI.2, Berin: Reiner, 1888).
    Commentary on Nicomachus' Introduction to Arithmetic (Leonardo Tarán, Asclepius of Tralles, Commentary to Nicomachus' Introduction to Arithmetic, Transactions of the American Philosophical Society (n.s.), 59: 4. Philadelphia, 1969.

Both works seem to be notes on the lectures conducted by Ammonius.
}
\begin{greek}

Asclepius Phil., In Aristotelis metaphysicorum libros A–Z commentaria (4018: 001)
“Asclepii in Aristotelis metaphysicorum libros A–Z commentaria”, Ed. Hayduck, M.
Berlin: Reimer, 1888; Commentaria in Aristotelem Graeca 6.2.
Page 262, line 8

                                                                             οὐκ ἄρα 
δυνατὸν τὸ αὐτὸ εἶναι καὶ μὴ εἶναι, εἰ μήτι γε ὁμωνύμως, εἰ τύχοι· ὁ 
γὰρ ἡμῖν ἄνθρωπος καλούμενος παρὰ Ἰνδοῖς, εἰ τύχοι, καλείσθω μὴ ἄν-
θρωπος. 

\end{greek}


\section{Acta Philippi}
\blockquote[From Wikipedia\footnote{\url{http://en.wikipedia.org/wiki/Acts_of_Philip}}]{The Greek Acts of Philip (Acta Philippi) is an unorthodox episodic apocryphal mid-to late fourth-century[1] narrative, originally in fifteen separate acta,[2] that gives an accounting of the miraculous acts performed by the Apostle Philip, with overtones of the heroic romance.
Courtyard of the Xenophontos monastery on Mount Athos where the complete text of the Acts of Philip was discovered

Some of these episodes are identifiable as belonging to more closely related "cycles".[3] Two episodes recounting events of Philip's commission (3 and 8) have survived in both shorter and longer versions. There is no commission narrative in the surviving texts: Philip's authority rests on the prayers and benediction of Peter and John and is explicitly bolstered by a divine epiphany, in which the voice of Jesus urges "Hurry Philip! Behold, my angel is with you, do not neglect your task" and "Jesus is secretly walking with him".(ch. 3).

}
\begin{greek}
Acta Philippi, Acta Philippi (2948: 001)
“Acta apostolorum apocrypha, vol. 2.2”, Ed. Bonnet, M.
Leipzig: Mendelssohn, 1903, Repr. 1972.
Section 32, line 5

Ἦν δὲ ἐκεῖ καὶ ὁ μακάριος Ἰωάννης, καὶ λέγει τῷ 
Φιλίππῳ· Ἀδελφέ μου καὶ συναπόστολε, εἰ καὶ μακρὰν ἔχεις 
τὴν ἀποδημίαν, γνώριζε ὅτι καὶ ὁ ἀδελφὸς Ἀνδρέας ἐπο-
ρεύθη εἰς τὴν Ἀχαίαν καὶ ὅλην τὴν Θρᾴκην, καὶ ὁ Θωμᾶς 
εἰς τὴν Ἰνδικὴν καὶ εἰς τοὺς σαρκοφάγους παλαμναίους, καὶ 
ὁ Ματθαῖος εἰς τοὺς τρωγλοδύτας καὶ ἀνηλεεῖς· ἡ γὰρ 
φύσις αὐτῶν ἐστιν ἠγριωμένη· καὶ ὁ κύριος ἐστι μετ' αὐτῶν. 

\end{greek}

\section{Calani Epistula (date ?)}
\blockquote[\footnote{\url{http://catalog.perseus.org/catalog/Atlg0040Calan}}]{?}
\begin{greek}


Calani Epistula, Epistula (0040: 001)
“Epistolographi Graeci”, Ed. Hercher, R.
Paris: Didot, 1873, Repr. 1965.
Line 2

Φίλοι πείθουσι χεῖρας καὶ ἀνάγκην προσφέρειν 
Ἰνδῶν φιλοσόφοις, οὐδ' ἐν ὕπνῳ ἑορακότες ἡμέτερα 
ἔργα. 

\end{greek}


\section{Corpus Hermeticum}
\blockquote[From Wikipedia\footnote{\url{http://en.wikipedia.org/wiki/Corpus_Hermeticum}}]{The Hermetica are Egyptian-Greek wisdom texts from the 2nd and 3rd centuries CE,[1] which are mostly presented as dialogues in which a teacher, generally identified as Hermes Trismegistus ("thrice-greatest Hermes"), enlightens a disciple. The texts form the basis of Hermeticism. They discuss the divine, the cosmos, mind, and nature. Some touch upon alchemy, astrology, and related concepts.}
\begin{greek}

Corpus Hermeticum, Νοῦς πρὸς Ἑρμῆν (1286: 011)
“Corpus Hermeticum, vol. 1”, Ed. Nock, A.D., Festugière, A.–J.
Paris: Les Belles Lettres, 1946, Repr. 1972.
Section 19, line 2

καὶ οὕτω νόησον ἀπὸ σεαυτοῦ, καὶ κέλευσόν σου τῇ 
ψυχῇ εἰς Ἰνδικὴν πορευθῆναι, καὶ ταχύτερόν σου τῆς 
κελεύσεως ἐκεῖ ἔσται. 

\end{greek}


\section{Hippiatrica}
\blockquote[From Wikipedia\footnote{\url{http://en.wikipedia.org/wiki/Hippiatrica}}]{The Hippiatrica (Greek: Ιππιατρικά) is a Byzantine compilation of ancient Greek texts, mainly excerpts, dedicated to the care and healing of the horse.[1] The texts were probably compiled in the 5th or 6th century AD by an unknown editor.[1] Seven texts from Late Antiquity constitute the main sources of the Hippiatrica: the veterinary manuals of Apsyrtus, Eumelus (a veterinary practitioner in Thebes, Greece[2]), Hierocles, Hippocrates, and Theomnestus, as well as the work of Pelagonius (originally a Latin text translated into Greek), and the chapter on horses from the agricultural compilation of Anatolius.[3] Although the aforementioned authors allude to their classical Greek veterinary predecessors (i.e. Xenophon and Simon of Athens), the roots of their tradition mainly lie in Hellenistic agricultural literature derived from Mago of Carthage.[3] In the 10th century AD, two more sources from Late Antiquity were added to the Hippiatrica: a work by Tiberius and an anonymous set of Prognoseis and iaseis (Greek: Πρόγνωσεις και ιάσεις).[4] Content-wise, the sources in the Hippiatrica provide no systematic exposition of veterinary art and emphasize practical treatment rather than on aetiology or medical theory.[5] However, the compilation contains a wide variety of literary forms and styles: proverbs, poetry, incantations, letters, instructions, prooimia, medical definitions, recipes, and reminiscences.[6] In the entire Hippiatrica, the name of Cheiron, the Greek centaur associated with healing and linked with veterinary medicine, appears twice (as a deity) in the form of a rhetorical invocation and in the form of a spell; a remedy called a cheironeion (Greek: χειρώνειον) is named after the mythological figure.[7] Currently, the compilation is preserved in five recensions in twenty-two manuscripts (containing twenty-five copies) ranging in date from the 10th to the 16th century AD.[1]}
\begin{greek}


Hippiatrica, Hippiatrica Berolinensia (0738: 001)
“Corpus hippiatricorum Graecorum, vol. 1”, Ed. Oder, E., Hoppe, K.
Leipzig: Teubner, 1924, Repr. 1971.
Chapter 4, section 14, line 3

Ἐκ τῶν ποδῶν ἀφαίμαξον τὸ ζῷον, εἶτα σμύρνης τρω-
γλίτιδος γ<ο> δʹ, κρόκου γ<ο> ἕξ, κενταυρίου γράμματα δʹ, νάρ-  
δου στάχυος Ἰνδικῆς γ<ο> αʹ, πεπέρεως λευκοῦ γ<ο> γʹ, σελινο-
σπέρμου κοχλιάρια εʹ, μήκωνος γ<ο> αʹ, προπόλεως γ<ο> αʹ, μέ-
λιτος ξέστην ἕνα, καὶ νίτρου τὸ ἀρκοῦν | συμμίξας, καὶ ποιή-
σας ὡς μέγεθος λεπτοκαρύου, καὶ λύσας ἐν ὕδατι χλιαρῷ ἑνὸς 
ξέστου, δίδου τῷ κάμνοντι ζῴῳ. 



Hippiatrica, Hippiatrica Berolinensia 
Chapter 11, section 2, line 3

Ἐὰν λευκωθῇ ὀφθαλμὸς ἀπὸ πληγῆς ἢ παρα|τρίψεως, 
σηπίας χρὴ ὄστρακον ξέσαντα καὶ τρίψαντα μετὰ σμύρνης καὶ 
μέλιτος ὑποχρίειν· ἢ ἅλας ὀρυκτὸν ἢ Ἰνδικὸν λεάναντα μετὰ 
κρόκου καὶ μέλιτος ἡψημένου ὑποχρίειν· ἢ ψιμυθίου καὶ 
μέλιτος Ἀττικοῦ μίξαντα ὑποχρίειν· ἢ σταφυλίνου ἀγρίου τὸ 
ἄνθος καὶ ἀνεμώνης τρίψας, ἔνσταζε δὶς τῆς ἡμέρας. 



Hippiatrica, Hippiatrica Berolinensia 
Chapter 22, section 16, line 1

Λυκίου Ἰνδικοῦ 𐆄 δʹ, ναρδοστάχυος 𐆄 δʹ, σμύρνης, στύ-
ρακος, κρόκου, πεπέρεως λευκοῦ, ἀμώμου βότρυος, πεπέρεως 
μακροῦ ἀνὰ 𐆄 εʹ, κινναμώμου 𐆄 αʹ, πάνακος 𐆄 αʹ, ἀκόρου 
𐆄 βʹ, μέλιτος 𐆄 βʹ, σελίνου σπέρματος 𐆄 αʹ, κασίας δαφνίτι-
δος 𐆄 βʹ, λιβάνου ἀρρενικοῦ, καλάμου ἀρωματικοῦ, νάρδου 
Κελτικῆς ἀνὰ 𐆄 γʹ, ῥόδων ξηρῶν λίτραν μίαν, ἀννήσου, πε-
τροσελίνου ἀνὰ ξέστην ἕνα, κροκομάγματος λίτραν μίαν, καὶ 
Λημνίας 𐆄 δʹ. 



Hippiatrica, Hippiatrica Berolinensia 
Chapter 45, section 1, line 8

         εἰ δὲ μή, ὀποπάνακος ἢ σμύρνης ἢ σελίνου καρπὸν 
ἢ τὸ βρύον τὸ ἀπὸ τῶν Ἰνδῶν ἢ χελιδόνιον, ὅ τι ἂν τούτων 
σχῇς. 



Hippiatrica, Hippiatrica Berolinensia 
Chapter 129, section 28, line 1

                     > κασίας γ<ο> αʹ, νάρδου Ἰνδικῆς γ<ο> δʹ, 
κρόκου, κόστου, ἴρεως Ἰλλυρικῆς, καρδαμώμου, πετροσελίνου, 
ἀργίου, πεπέρεως λευκοῦ, πρασίου, κενταυρίου, πάνακος, <φοῦ 
Ποντικοῦ>, σχοινάνθης, ἀμώμου, ἑκάστου ἀνὰ γ<ο> αʹ. 



Hippiatrica, Hippiatrica Berolinensia 
Chapter 130, section 173, line 2

        > νάρδου Συριακῆς ἤτοι Ἰνδικῆς, κρόκου Σικελικοῦ, 
σμύρνης τρωγλίτιδος, σχοινάνθης, πεπέρεως μέλανος, πεπέ-  
ρεως λευκοῦ, κασίας μελαίνης, χαμαίδρυος, νάρδου Κελτικῆς, 
κινναμώμου, κρομύων Ἰνδικῶν, ἀγαρικοῦ Ποντικοῦ, λιβάνου 
ἄρρενος, ἴρεως λευκῆς, καλαμίνθης, λασάρου Ποντικοῦ, γεν-
τιανῆς, πετροσελίνου ξηροῦ, κασίας σύριγγος· πάντα ἐπ' ἴσης 
σταθμῷ τῷ δοκοῦντί σοι κόψας καὶ σήσας χρῶ. 



Hippiatrica, Appendices ad hippiatrica Berolinensia (0738: 002)
“Corpus hippiatricorum Graecorum, vol. 1”, Ed. Oder, E., Hoppe, K.
Leipzig: Teubner, 1924, Repr. 1971.
Appendix 8, line 30

μέλανος 𐆄 αςʹ, ἑλενίου 𐆄 αςʹ, ἐπιθύμου 𐆄 αςʹ, εὐφορβίου 𐆄 αςʹ, 
ἑρπύλου κόμεως 𐆄 αςʹ, ἐρίφου αἵματος ξηροῦ 𐆄 αςʹ, εὐζώμου 
σπέρματος 𐆄 αςʹ, ἐλελισφάκ[τ]ου 𐆄 αςʹ, ἐλέφαντος σπέρματος 𐆄 αςʹ, 
ζαδώριον 𐆄 αςʹ, ζιγγιβέρεως 𐆄 αςʹ, ζάνης 𐆄 αςʹ, ἡλιοτροπίου 
ῥίζης 𐆄 αςʹ, ἡδυχρόου μαλάγματος 𐆄 αςʹ, θύμου Κρητικοῦ 
𐆄 αςʹ, θείου ἀπύρου καθαροῦ 𐆄 αςʹ, θλάσπεως 𐆄 αςʹ, θέρμων 
𐆄 αςʹ, ἰσχάδων 𐆄 αςʹ, ἴρεως Ἰλλυρικῆς 𐆄 αςʹ, ἰρυγγίου 𐆄 αςʹ, 
ἰξοῦ κεδρέας 𐆄 αςʹ, κόστου 𐆄 αςʹ, κενταυρίου 𐆄 αςʹ, κόμιος 
𐆄 αςʹ, κρομύου Ἰνδικοῦ 𐆄 αςʹ, κασίας ναρδίνης 𐆄 αςʹ, καλά-
μου ἀρωματικοῦ 𐆄 αςʹ, κολοκύνθης Ἀλεξανδρίνης 𐆄 αςʹ, καλα-
μίνθης ὀρεινῆς 𐆄 αςʹ, κέρατος ἐλαφείου 𐆄 αςʹ, καπάρεως 
ῥίζης φλοιοῦ 𐆄 αςʹ, καστορίου 𐆄 αςʹ, κρόκου 𐆄 αςʹ, κυκλα-
μίνου 𐆄 αςʹ, καδμείας βοτρυΐτιδος 𐆄 αςʹ, κασίας σύριγγος 
𐆄 αςʹ, καρδάμου 𐆄 αςʹ, κινναμώμου 𐆄 αςʹ, κρομύου σπέρ-
ματος 𐆄 αςʹ, καλάμου Συριακοῦ 𐆄 αςʹ, κασίας ἀσμαλίτιδος 
οὐγκία αςʹ. 



Hippiatrica, Fragmenta Timothei Gazaei (0738: 005)
“Corpus hippiatricorum Graecorum, vol. 2”, Ed. Oder, E., Hoppe, K.
Leipzig: Teubner, 1927, Repr. 1971.
Section 2, line 2

   <Ἄρ[ρ]αβες> δὲ οἱ πρὸς τῷ ὄρει τῆς 
Ἰνδικῆς εὐμεγέθεις, φοίνικες, ὡς ἐπίπαν εἰπεῖν, τὴν χροιάν, 
ὑψαύχενες, τὴν προτομὴν σύμμετρον καὶ εὔρυθμον ἔχοντες, 
τὴν κεφαλὴν [καὶ] ἡνομένην σχεδὸν τῷ τοῦ ἱππέως μετώπῳ, 
κυδροὶ καὶ δυσγαργάλιστοι, πολὺ τὸ τῆς ἱπποτυφίας ἔχοντες 
ὑπερήφανον, ὀξεῖς σφόδρα, ποδώκεις, γόνατα ὑγροί, τῇ ὁρμῇ 
τοῦ δρόμου ὅλους ἑαυτοὺς ἐπιδιδόντες, πηδῶντες κούφως 
μᾶλλον ἢ τρέχοντες, γοργοὶ τὸ βλέμμα, τὴν ἶριν κεκραμένην 
καὶ ὑδαρεστέραν ἔχοντες, τοὺς κενεῶνας συνεσταλμένοι καὶ τὸ 
ἄλλο σῶμα ἰσχνοὶ καὶ οὐ κεχυμένοι, <τὴν> ῥάχιν κοίλην ἔχοντες, 
πρὸς τὸ καῦμα μὴ ἀπαγορεύοντες, τῷ ἡλίῳ μᾶλλον χαίροντες, 




Hippiatrica, Additamenta Londinensia ad hippiatrica Cantabrigiensia (0738: 007)
“Corpus hippiatricorum Graecorum, vol. 2”, Ed. Oder, E., Hoppe, K.
Leipzig: Teubner, 1927, Repr. 1971.
Section 18, line 3

                                                   αςʹ, λυκίου 
Ἰνδικοῦ ἑξαγ. 



Hippiatrica, Additamenta Londinensia ad hippiatrica Cantabrigiensia 
Section 27, line 5

           > Χαλκοῦ κεκαυμένου <γο> ϛʹ, καδμείας κεκαυμένης 
πεπλυμένης, ναρδοστάχυος, λίθου αἱματίτου, λυκίου Ἰνδικοῦ, 
κρόκου ἀνὰ ἑξαγ. 



Hippiatrica, Excerpta Lugdunensia (0738: 008)
“Corpus hippiatricorum Graecorum, vol. 2”, Ed. Oder, E., Hoppe, K.
Leipzig: Teubner, 1927, Repr. 1971.
Section 141, line 4

                                   > Περιστερ(ε)ῶνος ὀρθοῦ 
τοῦ χυλοῦ 𐆆 ϛʹ καὶ τῆς ῥίζης αὐτοῦ ξηρᾶς κεκομμένης καὶ 
σεσησμένης 𐆆 αʹ, λίθου αἱματί[ς]του 𐆆 δʹ, λυκίου Ἰνδικοῦ 
𐆆 βʹ, ὀποβαλσάμου 𐆆 βʹ, χολῆς αἰγὸς θηλείας 𐆆 δʹ, μέλιτος 
𐆆 γʹ. 



\end{greek}


\section{Cyril of Alexandria}
\blockquote[From Wikipedia\footnote{\url{http://en.wikipedia.org/wiki/Cyril_of_Alexandria}}]{Cyril of Alexandria (Greek: Κύριλλος Ἀλεξανδρείας; c. 376 – 444) was the Patriarch of Alexandria from 412 to 444. He was enthroned when the city was at the height of its influence and power within the Roman Empire. Cyril wrote extensively and was a leading protagonist in the Christological controversies of the later 4th and 5th centuries. He was a central figure in the First Council of Ephesus in 431, which led to the deposition of Nestorius as Patriarch of Constantinople.

Cyril was a scholarly archbishop and a prolific writer. In the early years of his active life in the Church he wrote several exegesis. Among these were: Commentaries on the Old Testament,[25] Thesaurus, Discourse Against Arians Commentary on St. John's Gospel,[26] and Dialogues on the Trinity. In 429 as the Christological controversies increased, his output of writings was that which his opponents could not match. His writings and his theology have remained central to tradition of the Fathers and to all Orthodox to this day.}
\begin{greek}


Cyrillus Theol., Commentarius in xii prophetas minores (4090: 001)
“Sancti patris nostri Cyrilli archiepiscopi Alexandrini in xii prophetas, 2 vols.”, Ed. Pusey, P.E.
Oxford: Clarendon Press, 1868, Repr. 1965.


Cyrillus Theol., Commentarius in xii prophetas minores 
Volume 1, page 328, line 19

ἐν μὲν γὰρ τοῖς νοτίοις μέρεσι τῆς Ἱερουσαλὴμ βαθεῖά τις 
ἔρημος ἐξευρύνεται, τερματίζεται γεμὴν πρὸς ἠῶ τε καὶ νότον 
πελάγεσιν Ἰνδικοῖς· πρὸς δύσιν δ' αὖ καὶ βορειότερα, τῇ 
Παλαιστινῶν γείτονι θαλάσσῃ, καὶ αὐτῇ δὲ προσκλυζούσῃ 
τῇ Αἰγυπτίων. 



Cyrillus Theol., Commentarius in xii prophetas minores 
Volume 1, page 366, line 9

ΑΜΩΣ γέγονεν αἰπόλος ἀνὴρ, καὶ ποιμενικοῖς ἔθεσί τε 
καὶ νόμοις ἐντεθραμμένος· ἐποιεῖτο δὲ τὰς διατριβὰς ἐν 
ἐρήμῳ τῇ πρὸς νότον τῆς Ἰουδαίων χώρας, ἣ μέχρις ὅρων 
διήκει θαλάσσης τῆς Ἰνδικῆς, καὶ εἰς τὴν Περσῶν ἐκτείνεται 
γῆν, μυρία δὲ ὅσα βαρβάρων αὐτὴν καταβόσκεται γένη, ἔχει 
δὲ λίαν ἐπιτηδείως καὶ εἰς τὸ φέρβειν δύνασθαι τὰς οἰῶν 
ἀγέλας· εὔβοτος γὰρ καὶ πλατεῖα καὶ πολυειδεῖ τῇ πόᾳ 
κατεστεμμένη. 



Cyrillus Theol., Commentarius in xii prophetas minores 
Volume 1, page 558, line 15

                                     ἔοικε δὲ διὰ τούτων ἡμῖν ὁ 
λόγος τὰ Ἰνδικὰ κατασημαίνειν ἔθνη· νοτιώτατοι γὰρ οἱ 
Ἰνδοὶ καὶ αἱ τούτων χῶραι. 



Cyrillus Theol., Commentarius in xii prophetas minores 
Volume 1, page 568, line 24

                                                         τινὲς μὲν 
οὖν οἴονται πόλιν διὰ τούτου κατασημαίνεσθαι τὴν παρ' 
Αἰθίοψι καὶ Ἰνδοῖς, καὶ ἔστι μὲν ὁμολογουμένως παρ' ἐκεί-
νοις Θαρσεῖς· ἡ γοῦν σύμπασα τῶν Ἰνδῶν διὰ τοῦ Θαρσεῖς 
σημαίνεται χώρα· πλὴν εἴς γε τὸ παρὸν οὐκ ἐκεῖνο οἶμαι   
βούλεσθαι δηλοῦν τὸν λόγον· ὅτι τοῖς ἀποπλεῖν ἐθέλου-
σιν ἐπὶ τὰ Ἰνδῶν ἔθνη, γένοιτ' ἂν εἰκότως οὐ διά γε τῆς 
Ἰόππης ὁ πλοῦς, ἀλλὰ διὰ θαλάσσης μᾶλλον τῆς Ἐρυθρᾶς, 
εἰ μὴ ἄρα τις οἴοιτο τυχὸν βεβουλῆσθαι τὸν Προφήτην διὰ 
Περσῶν τε καὶ Ἀσσυρίων εἰς Αἰθίοπας τοὺς ἐσωτάτω 
ποιεῖσθαι τὴν ἀποδρομήν. 



Cyrillus Theol., Commentarius in xii prophetas minores 
Volume 2, page 44, line 7

                                                         διήρπαζον 
τὸ ἀργύριον καὶ διήρπαζον τὸ χρυσίον· ταῦτα δὲ ἦν ἡ ὑπό-
στασις· Καὶ οὐκ ἦν πέρας τοῦ κόσμου αὐτῶν, ἔοικε δὲ καὶ 
λίθους ἐν τούτοις ὀνομάζειν τὰς Ἰνδικὰς, ἐφ' αἷς δὴ μάλιστα 
βεβαρύνθαι φησὶν αὐτὴν, καίτοι καὶ ἐφ' ἑτέροις σκεύεσιν 
ἀθύμως διακειμένην. 



Cyrillus Theol., Commentarius in xii prophetas minores 
Volume 2, page 142, line 7

                                                      οὗτοι δὲ 
ἦσαν οἵ τε πρὸς ἠῶ καὶ νότον Αἰθίοπες τὴν Ἰνδικὴν προσοι-
κοῦντες θάλασσαν, καὶ μέντοι καὶ Μαδιηναῖοι, καὶ αὐτοὶ τὴν 
ὅμορον οἰκοῦντες ἔρημον. 



Cyrillus Theol., Commentarius in xii prophetas minores 
Volume 2, page 229, line 14

                                                    καὶ γοῦν 
ἐκκλησίαι πανταχοῦ, ποιμένες καὶ διδάσκαλοι, καθηγηταὶ 
καὶ μυσταγωγοὶ καὶ θεῖα θυσιαστήρια, θύεται δὲ νοητῶς ὁ 
ἀμνὸς παρὰ τῶν ἁγίων ἱερουργῶν καὶ παρ' Ἰνδοῖς καὶ 
Αἰθίοψι. 



Cyrillus Theol., Commentarius in xii prophetas minores 
Volume 2, page 419, line 4

                                 τερματίζεται γὰρ ἡ τῶν 
Ἰουδαίων χώρα θαλάσσῃ τε τῇ πρὸς νότον καὶ Ἰνδικῇ, καὶ 
τῇ καλουμένῃ Μέσῃ τῶν ποταμῶν. 



Cyrillus Theol., Fragmenta in sancti Pauli epistulam ad Hebraeos (4090: 006)
“Sancti patris nostri Cyrilli archiepiscopi Alexandrini in D. Joannis evangelium, vol. 3”, Ed. Pusey, P.E.
Oxford: Clarendon Press, 1872, Repr. 1965.
Page 396, line 1

                                                            καὶ 
τοῦτο ἰδὼν ὁ προφήτης Δαυεὶδ ψάλλει που καί φησι πρὸς 
αὐτὸν περὶ τῶν ἐξ Ἰσραήλ “Ἄμπελον ἐξ Αἰγύπτου μετῆ-
“ρας, ἐξέβαλες ἔθνη καὶ κατεφύτευσας αὐτήν·” προσεπάγει 
δὲ τούτοις, ποταμῷ Εὐφράτῃ καὶ μὴν καὶ θαλάσσῃ τῇ πρὸς   
νότον καὶ Ἰνδικῇ τερματίζων τὴν Ἰουδαίων “Ἐξέτεινε τὰ 
“κλήματα αὐτῆς ἕως θαλάσσης καὶ ἕως ποταμῶν τὰς παρα-
“φυάδας αὐτῆς. 




Cyrillus Theol., De adoratione et cultu in spiritu et veritate 
Volume 68, page 484, line 12

                                      λίθῳ γὰρ οἶμαι τῇ 
Ἰνδικῇ, τῷ ὑακινθίνῳ φημὶ, τὸ αἰθέριόν πως εἰκάζε-
ται σῶμα, αὐγῇ τε καὶ σκότῳ συμμιγὲς, καὶ ἐν βάθει 
πως ἔχον τὸ ὑδαροειδὲς, ὑποτρέμουσάν τε καὶ εὐ-
διάχυτον ὑποφαῖνον τὴν ὄψιν. 



Cyrillus Theol., Expositio in Psalmos (4090: 100); MPG 69.
Volume 69, page 1181, line 43

                      Σαββᾶ πόλις τῆς Ἰνδίας, ἀφ' 
ἧς ἦλθεν ἡ βασίλισσα Νότου πρὸς Σολομῶντα. 



Cyrillus Theol., Commentarius in Isaiam prophetam (4090: 103); MPG 70.
Volume 70, page 88, line 44

Οὐκοῦν ἢ τοῦτό φησιν ὁ προφητικὸς ἡμῖν ἐν τούτοις 
λόγος, ἢ ἐκεῖνό που τάχα· Πλεῖστοι γὰρ τῶν βεβα-
σιλευκότων τοῦ Ἰσραὴλ, ναῦς εἶχον ἐμπορικὰς, τὰ[ς] 
ἐκ τῆς Ἰνδῶν χώρας τε καὶ γῆς ἀποφερούσας αὐτοῖς, 
καὶ τὰ[ς] ἐξ ἑτέρων χωρῶν καὶ πόλεων· ὅθεν αὐτοῖς 
καὶ ὁ πολὺς συναγήγερτο πλοῦτος· ἐξ οὗ γεγόνασι 
ὑψηλοὶ, κέδροις τε καὶ δρυσὶν ἐν ἴσῳ, ὄρεσί τε καὶ 
πύργοις. 



Cyrillus Theol., Commentarius in Isaiam prophetam 
Volume 70, page 357, line 22

                                        Ὀλιγανδρή-
σειν τοίνυν φησὶν οὕτω τὴν Βαβυλωνίων, καὶ ἅπασαν 
δὲ τὴν Ἀσσυρίων χώραν, ὡς δυσευρέτους γενέσθαι 
τοὺς ζῶντας ἐν αὐτῇ, καὶ τοσαύτην τῶν περιλελειμ-
μένων τὴν σπάνιν, ὅση καὶ λίθων Ἰνδικῶν, καὶ χρυ-
σίου τοῦ δοκιμωτάτου. 



Cyrillus Theol., Commentarius in Isaiam prophetam 
Volume 70, page 969, line 57

                                            Αἰγύπτιοι τοί-
νυν, καὶ τὰ τῶν Αἰθιόπων ἐμπορεῖα, τοῦτ' ἔστιν, αἱ 
Θηβαίων πόλεις, αἷς εἰσι πρόσοικοί τε καὶ ἀγχιτέρμο-
νες οἱ καλούμενοι Σαβαῒμ, τοῦτ' ἔστι, τὰ τῶν Αἰθιό-
πων, ἤγουν τῶν Ἰνδῶν ἔθνη (τάχα που τὴν κλῆσιν   
λαχόντες ὡς ἀπό γε τῆς Σαβᾶ τῆς βασιλευσάσης κατὰ 
καιροὺς τῆς αὐτῶν χώρας τε καὶ γῆς), ἐκοπίασαν, 
φησί· τὸ δὲ, Ἐκοπίασαν, διχῇ νοητέον· ἢ γὰρ ἐκεῖνο 
βούλεται δηλοῦν, ὅτι κεκμήκασιν οὐ φορητῶς, κατά 
γε τὸν τοῦ πλανᾶσθαι καιρὸν ὠμῷ τυράννῳ κατεζευ-
γμένοι τῷ Σατανᾷ, καὶ ταῖς τῶν δαιμονίων ἀγέλαις 
δουλεύοντες ἐξαιτοῦντι θυσίας, καὶ αὐτὰ τὰ αὐτῶν 
γεννήματα, υἱούς τέ φημι καὶ θυγατέρας· ἤγουν ὅτι 
πάλαι μὲν ἦσαν δεινοὶ καὶ ἄθραυστοι, καὶ οἷον

ὑπ-



Cyrillus Theol., Commentarius in Isaiam prophetam 
Volume 70, page 972, line 28

                                       τὸ] Χριστοῦ παρα-
τείνει σέβας, καὶ μέχρις αὐτῶν τῶν Σαβαῒμ, ἤτοι 
τῶν Ἰνδικῶν ἐθνῶν. 



Cyrillus Theol., Commentarius in Isaiam prophetam 
Volume 70, page 1072, line 14

                                         Διὰ γάρ τοι 
τοῦτο τὴν Ἐκκλησίαν ἱματισμῷ διαχρύσῳ καὶ πε-
ποικιλμένῳ κατακοσμεῖν ἔθος τῇ θεοπνεύστῳ Γραφῇ· 
ὅνπερ γὰρ αἱ πολυειδεῖς καὶ πολυτελέστατοι τῶν 
λίθων, φημὶ δὴ τῶν Ἰνδικῶν, χρυσοῖς κανόσιν 
ἐνισχημέναι, θαυμαστόν τι καὶ ἀξιοθέατον ἐπιτελοῦσι 
κόσμημα· οὕτω καὶ αἱ τῶν ἁγίων ψυχαὶ τοῖς ἐξ 
ἀρετῶν αὐχήμασιν ἐξωραϊσμέναι, φαιδρὸν ἀποστίλ-
βουσι κάλλος τοῖς τῆς Θεότητος ὀφθαλμοῖς, ὥστε καὶ 
ἕκαστον εὐχαριστοῦντα τῷ Χριστῷ λέγειν· Ἀγαλλιά-
σθω ἡ ψυχή μου ἐπὶ τῷ Κυρίῳ. 



Cyrillus Theol., Commentarius in Isaiam prophetam 
Volume 70, page 1329, line 21

                                              Γονιμωτάτη 
δὲ ἡ χώρα λιβανωτοῦ, χρυσοῦ τε καὶ λίθων τῶν 
Ἰνδικῶν. 



Cyrillus Theol., Commentarius in Isaiam prophetam 
Volume 70, page 1332, line 40

                                                          Θαρ-
σεῖς δὲ ἡ θεία Γραφὴ τὰς Ἰνδικὰς ὀνομάζει πλειστα-
χοῦ· φασὶ δὲ εἶναι καὶ ἐν τῇ Κύπρῳ Θαρσεῖς πόλιν 
ὠνομασμένην. 



Cyrillus Theol., Commentarius in Isaiam prophetam 
Volume 70, page 1332, line 43

               Οὐκοῦν ἐπειδήπερ αἵ τε νῆσοι, φησὶ, 
καὶ οἱ ἐκ τῆς Ἰνδῶν ἥκοντες χώρας ἐμὲ ὑπέμειναν, 
τουτέστιν, ὑπέστησαν τὸν ἐπενεχθέντα αὐτοῖς παρ' 
ἐμοῦ ζυγὸν, διὰ τοῦτο συῤῥεῖ πανταχόθεν ἡ τῶν πι-
στευόντων πληθύς. 



Cyrillus Theol., Commentarius in Isaiam prophetam 
Volume 70, page 1369, line 44

       Ἑκάστην γὰρ ἁγίαν ψυχὴν, καὶ συλλήβδην τὴν 
Ἐκκλησίαν, τουτέστι, τὰ τῶν ἁγίων συστήματα, 
στεφάνῳ παρεικαστέον ἐκ πολλῶν ἀνθέων συντεθει-
μένῳ, ἤγουν διαδήματι βασιλικῷ, λίθοις Ἰνδικοῖς 
ἐκλάμποντι, καὶ πολυειδῶς ἔχοντι τὸ διαπρεπές· 
πλεῖστα δὲ ὅσα καὶ τὰ τῶν ἁγίων ἀνδραγαθήματα, 
καὶ οὐχ εἷς μᾶλλον τῶν αὐχημάτων ὁ τρόπος, πολὺς 
δὲ καὶ διάφορος. 

\end{greek}




\section{Acta Thomae}
\blockquote[From Wikipedia\footnote{\url{http://en.wikipedia.org/wiki/Acts_of_Thomas}}]{The early 3rd-century text called Acts of Thomas is one of the New Testament apocrypha, portraying Christ as the "Heavenly Redeemer", independent of and beyond creation, who can free souls from the darkness of the world. References to the work by Epiphanius of Salamis show that it was in circulation in the 4th century. The complete versions that survive are Syriac and Greek. There are many surviving fragments of the text. Scholars detect from the Greek that its original was written in Syriac, which places the Acts of Thomas in Syria. The surviving Syriac manuscripts, however, have been edited to purge them of the most unorthodox overtly gnostic passages, so that the Greek versions reflect the earlier tradition.

Acts of Thomas is a series of episodic Acts (Latin passio) that occurred during the evangelistic mission of Judas Thomas ("Judas the Twin") to India. It ends with his martyrdom: he dies pierced with spears, having earned the ire of the monarch Misdaeus (Vasudeva I) because of his conversion of Misdaeus' wives and a relative, Charisius. He was imprisoned while converting Indian followers won through the performing of miracles.

Summary: The Acts of Thomas[1][2] connects Thomas, the apostle's Indian ministry with two kings, one in the north and the other in the south. According to one of the legends in the Acts, Thomas was at first reluctant to accept this mission, but the Lord appeared to him in a night vision and said, “Fear not, Thomas. Go away to India and proclaim the Word, for my grace shall be with you.” But the Apostle still demurred, so the Lord overruled the stubborn disciple by ordering circumstances so compelling that he was forced to accompany an Indian merchant, Abbanes, to his native place in northwest India, where he found himself in the service of the Indo-Parthian king Gundaphorus. The apostle's ministry resulted in many conversions throughout the kingdom, including the king and his brother.[1]

According to the legend, Thomas was a skilled carpenter and was bidden to build a palace for the king. However, the Apostle decided to teach the king a lesson by devoting the royal grant to acts of charity and thereby laying up treasure for the heavenly abode. Although little is known of the immediate growth of the church, Bar-Daisan (154–223) reports that in his time there were Christian tribes in North India which claimed to have been converted by Thomas and to have books and relics to prove it.[2] But at least by the year of the establishment of the Second Persian Empire (226), there were bishops of the Church of the East in northwest India, Afghanistan and Baluchistan, with laymen and clergy alike engaging in missionary activity.[3]

The Acts of Thomas identifies his second mission in India with a kingdom ruled by King Mahadeva, one of the rulers of a 1st-century dynasty in southern India. It is most significant that, aside from a small remnant of the Church of the East in Kurdistan, the only other church to maintain a distinctive identity is the Mar Thoma or “Church of Thomas” congregations along the Malabar Coast of Kerala State in southwest India. According to the most ancient tradition of this church, Thomas evangelized this area and then crossed to the Coromandel Coast of southeast India, where, after carrying out a second mission, he died in Mylapore near Madras. Throughout the period under review, the church in India was under the jurisdiction of Edessa, which was then under the Mesopotamian patriarchate at Seleucia-Ctesiphon and later at Baghdad and Mosul. Historian Vincent A. Smith says, “It must be admitted that a personal visit of the Apostle Thomas to South India was easily feasible in the traditional belief that he came by way of Socotra, where an ancient Christian settlement undoubtedly existed. I am now satisfied that the Christian church of South India is extremely ancient... ”.[4]

Although there was a lively trade between the Near East and India via Mesopotamia and the Persian Gulf, the most direct route to India in the 1st century was via Alexandria and the Red Sea, taking advantage of the Monsoon winds, which could carry ships directly to and from the Malabar coast. The discovery of large hoards of Roman coins of 1st-century Caesars and the remains of Roman trading posts testify to the frequency of that trade. In addition, thriving Jewish colonies were to be found at the various trading centers, thereby furnishing obvious bases for the apostolic witness.

Piecing together the various traditions, one may conclude that Thomas left northwest India when invasion threatened and traveled by vessel to the Malabar coast, possibly visiting southeast Arabia and Socotra en route and landing at the former flourishing port of Muziris on an island near Cochin (c. AD 51–52). From there he is said to have preached the gospel throughout the Malabar coast, though the various churches he founded were located mainly on the Periyar River and its tributaries and along the coast, where there were Jewish colonies. He reputedly preached to all classes of people and had about seventeen thousand converts, including members of the four principal castes. Later, stone crosses were erected at the places where churches were founded, and they became pilgrimage centres. In accordance with apostolic custom, Thomas ordained teachers and leaders or elders, who were reported to be the earliest ministry of the Malabar church.}
\begin{greek}



Acta Thomae, Acta Thomae (2038: 001)
“Acta apostolorum apocrypha, vol. 2.2”, Ed. Bonnet, M.
Leipzig: Mendelssohn, 1903, Repr. 1972.
Section 1, line 9

         κατὰ κλῆρον οὖν ἔλαχεν ἡ Ἰνδία Ἰούδᾳ Θωμᾷ τῷ καὶ 
Διδύμῳ· οὐκ ἐβούλετο δὲ ἀπελθεῖν, λέγων μὴ δύνασθαι μήτε 
χωρεῖν διὰ τὴν ἀσθένειαν τῆς σαρκός, καὶ ὅτι Ἄνθρωπος ὢν 
Ἑβραῖος πῶς δύναμαι πορευθῆναι ἐν τοῖς Ἰνδοῖς κηρύξαι τὴν 
ἀλήθειαν; 



Acta Thomae, Acta Thomae 
Section 1, line 15

            Καὶ ταῦτα αὐτοῦ διαλογιζομένου καὶ λέγοντος 
ὤφθη αὐτῷ ὁ σωτὴρ διὰ τῆς νυκτός, καὶ λέγει αὐτῷ· Μὴ 
φοβοῦ Θωμᾶ, ἄπελθε εἰς τὴν Ἰνδίαν καὶ κήρυξον ἐκεῖ τὸν 
λόγον· ἡ γὰρ χάρις μού ἐστιν μετὰ σοῦ. 



Acta Thomae, Acta Thomae 
Section 1, line 18

                                               Ὃ δὲ οὐκ ἐπείθετο,   
λέγων· Ὅπου βούλει με ἀποστεῖλαι ἀπόστειλον ἀλλαχοῦ· εἰς 
Ἰνδοὺς γὰρ οὐκ ἀπέρχομαι. 



Acta Thomae, Acta Thomae 
Section 2, line 2

Καὶ ταῦτα αὐτοῦ λέγοντος καὶ ἐνθυμουμένου ἔτυχεν 
ἔμπορόν τινα εἶναι ἐκεῖ ἀπὸ τῆς Ἰνδίας ἐλθόντα ᾧ ὄνομα 
Ἀββάνης, ἀπὸ τοῦ βασιλέως Γουνδαφόρου ἀποσταλέντα καὶ 
ἐντολὴν παρ' αὐτοῦ εἰληφότα τέκτονα πριάμενον ἀγαγεῖν 
αὐτῷ. 



Acta Thomae, Acta Thomae 
Section 2, line 13

                                          Καὶ ταῦτα εἰπὼν ὑπέ-
δειξεν αὐτῷ τὸν Θωμᾶν ἀπὸ μακρόθεν, καὶ συνεφώνησεν   
μετ' αὐτοῦ τριῶν λιτρῶν ἀσήμου, καὶ ἔγραψεν ὠνὴν λέγων· 
Ἐγὼ Ἰησοῦς υἱὸς Ἰωσὴφ τοῦ τέκτονος ὁμολογῶ πεπρακέναι 
ἐμὸν δοῦλον Ἰούδαν ὀνόματι σοὶ τῷ Ἀββάνῃ ἐμπόρῳ Γουνδα-
φόρου τοῦ βασιλέως τῶν Ἰνδῶν. 



Acta Thomae, Acta Thomae 
Section 16, line 19

                                                            πολλοὶ δὲ 
καὶ τῶν ἀδελφῶν συνηθροίζοντο ἐκεῖ, ἕως ὅτε φήμης ἤκουσαν   
τοῦ ἀποστόλου, ὅτι ἐν ταῖς πόλεσιν τῆς Ἰνδίας κατήχθη καὶ 
ἐκεῖ διδάσκει. 



Acta Thomae, Acta Thomae 
Section 17, line 2

Ὅτε δὲ εἰσῆλθεν ὁ ἀπόστολος εἰς τὰς πόλεις τῆς Ἰν-
δίας μετὰ Ἀββάνη τοῦ ἐμπόρου, ἀπῆλθεν ὁ Ἀββάνης εἰς 
ἀσπασμὸν Γουνδαφόρου τοῦ βασιλέως, προσανήνεγκεν δὲ 
αὐτῷ περὶ τοῦ τέκτονος ὃν μετ' αὐτοῦ ἤγαγεν. 



Acta Thomae, Acta Thomae 
Section 39, line 10

διαλεγομένου τῷ πλήθει πῶλος ὀνάδος ἦλθεν καὶ ἔστη ἔμ-
προσθεν αὐτοῦ, καὶ ἀνοίξας τὸ στόμα αὐτοῦ εἶπεν· Ὁ δίδυμος 
τοῦ Χριστοῦ, ὁ ἀπόστολος τοῦ ὑψίστου καὶ συμμύστης τοῦ 
λόγου τοῦ Χριστοῦ τοῦ ἀποκρύφου, ὁ δεχόμενος αὐτοῦ τὰ 
ἀπόκρυφα λόγια, ὁ συνεργὸς τοῦ υἱοῦ τοῦ θεοῦ, ὃς ἐλεύθερος 
ὢν γέγονας δοῦλος καὶ πραθεὶς πολλοὺς εἰς ἐλευθερίαν εἰσή-
γαγες· ὁ συγγενὴς τοῦ μεγάλου γένους τοῦ τὸν ἐχθρὸν κατα-
δικάσαντος καὶ τοὺς ἰδίους λυτρωσαμένου, ὁ πρόφασις τῆς 
ζωῆς πολλοῖς γενόμενος ἐν τῇ χώρᾳ τῶν Ἰνδῶν· ἦλθες γὰρ 
πρὸς τοὺς πλανωμένους ἀνθρώπους, καὶ διὰ τῆς σῆς ἐπιφα-
νείας καὶ τῶν λόγων σου τῶν θεϊκῶν νῦν ἐπιστρέφονται πρὸς 
τὸν ἀποστείλαντά σε θεὸν τῆς ἀληθείας· ἀνελθὼν ἐπικαθέ-
σθητί μοι καὶ ἀναπάηθι ἕως ἂν εἰς τὴν πόλιν εἰσέλθῃς. 



Acta Thomae, Acta Thomae 
Section 42, line 7

                                                     γυνὴ δέ τις πάνυ 
ὡραία αἰφνιδίως φωνὴν ἀφῆκε μεγίστην λέγουσα· 
 Ἀπόστολε τοῦ νέου θεοῦ ὁ ἐλθὼν εἰς τὴν Ἰνδίαν, καὶ 
δοῦλε τοῦ ἁγίου ἐκείνου καὶ μόνου ἀγαθοῦ θεοῦ· διὰ σοῦ 
γὰρ οὗτος κηρύσσεται ὁ σωτὴρ τῶν ψυχῶν τῶν πρὸς αὐτὸν 
ἐρχομένων, καὶ διὰ σοῦ ἰατρεύεται τὰ σώματα τῶν ὑπὸ τοῦ 
ἐχθροῦ κολαζομένων, καὶ σὺ εἶ ὁ γεγονὼς πρόφασις τῆς ζωῆς 
πάντων τῶν ἐπ' αὐτὸν ἐπιστρεφόντων· κέλευσόν με ἀχθῆναι 
ἔμπροσθέν σου ἵνα σοι ἀφηγήσωμαι τὰ συμβάντα μοι καὶ 
τάχα ἐκ σοῦ γένηταί μοι ἐλπίς, καὶ οὗτοι δὲ οἱ παρεστῶτές   
σοι εὐέλπιδες γένωνται μᾶλλον εἰς τὸν θεὸν ὃν κηρύσσεις. 



Acta Thomae, Acta Thomae 
Section 62, line 2

Τοῦ δὲ ἀποστόλου Ἰούδα Θωμᾶ καταγγέλλοντος ἐν 
πάσῃ τῇ Ἰνδίᾳ τὸν λόγον τοῦ θεοῦ στρατηλάτης τις τοῦ 
βασιλέως Μισδαίου ἦλθεν πρὸς αὐτόν, καὶ ἔλεγεν αὐτῷ· 
Ἀκήκοα περὶ σοῦ ὅτι μισθὸν παρά τινος οὐ λαμβάνεις, 
ἀλλ' ὅπερ καὶ ἔχεις τοῖς δεομένοις παρέχεις· εἰ γὰρ μισθοὺς 
ἐλάμβανες, ἀπέστειλα ἂν χρῆμα ἱκανόν, καὶ αὐτὸς ἐνθάδε οὐ 
παρεγενόμην· ὁ γὰρ βασιλεὺς ἐκτὸς ἐμοῦ οὐδὲν διαπράττεται· 
πολλὰ γὰρ ὑπάρχοντά μοί εἰσιν καὶ πλούσιός εἰμι, εἷς τῶν   
πλουτούντων ἐν τῇ Ἰνδίᾳ· καὶ οὐδ' ὅλως ἠδίκησά ποτέ τινα· 
τὸ δὲ ἐναντίον μοι συνέβη· γαμετὴν ἔχω, καὶ ἔσχον ἐξ αὐτῆς 
θυγατέρα, καὶ πάνυ διάκειμαι πρὸς αὐτήν, ὡς καὶ ἡ φύσις 




Acta Thomae, Acta Thomae 
Section 101, line 8

                Ὁ δὲ Χαρίσιος λέγει πρὸς τὸν βασιλέα· Καινόν 
σοι ἔχω ὑφηγήσασθαι πρᾶγμα καὶ ἐρημίαν νέαν, ἣν Σιφὼρ   
ἤγαγεν ἐν τῇ Ἰνδίᾳ, ἄνδρα τινὰ Ἑβραῖον μάγον, ὃν ἔχει κα-
θεζόμενον ἐν τῷ ἰδίῳ οἴκῳ, ὃς οὐκ ἀφίσταται αὐτοῦ· πολλοὶ 
δέ εἰσιν οἱ εἰσιόντες πρὸς αὐτόν· οὓς καὶ διδάσκει νέον θεὸν 
καὶ νόμους νέους ἐντίθησιν αὐτοῖς τοὺς μή πω ἀκουσθέντας, 
λέγων· Ἀδύνατόν ἐστιν ὑμᾶς εἰς τὴν αἰώνιον ζωὴν εἰσελθεῖν 
ἣν ἐγὼ καταγγέλλω ὑμῖν, ἐὰν μὴ ἀπαλλαγῆτε ὑμεῖς τῶν ἰδίων 
γυναικῶν, ὁμοίως καὶ αἱ γυναῖκες τῶν ἰδίων ἀνδρῶν. 



Acta Thomae, Acta Thomae 
Section 108, line 10

                                                  προσευξάμενος δὲ 
καὶ καθεσθεὶς ἤρξατο λέγειν ψαλμὸν τοιοῦτον· Ὅτε ἤμην 
βρέφος ἄλαλον ἐν τοῖς τοῦ πατρός μου βασιλείοις ἐν πλούτῳ 
καὶ τρυφῇ τῶν τροφέων ἀναπαυόμενος, ἐξ Ἀνατολῆς τῆς 
πατρίδος ἡμῶν ἐφοδιάσαντές με οἱ γονεῖς ἀπέστειλάν με· 
ἀπὸ δὲ πλούτου τῶν θησαυρῶν τούτων φόρτον συνέθηκαν 
μέγαν τε καὶ ἐλαφρόν, ὅπως αὐτὸν μόνος βαστάσαι δυνηθῶ· 
χρυσός ἐστιν ὁ φόρτος τῶν ἄνω, καὶ ἄσημος τῶν μεγάλων   
θησαυρῶν, καὶ λίθοι ἐξ Ἰνδῶν οἱ χαλκεδόνιοι, καὶ μαργαρῖται 
ἐκ Κοσάνων· καὶ ὥπλισάν με τῷ ἀδάμαντι· καὶ ἐνέδυσάν με 
ἐσθῆτα διάλιθον χρυσόπαστον, ἣν ἐποίησαν στέργοντές με, 
καὶ στολὴν τὸ χρῶμα ξανθὴν πρὸς τὴν ἐμὴν ἡλικίαν. 



Acta Thomae, Acta Thomae 
Section 116, line 4

Λέγοντος δὲ τοῦ Χαρισίου ταῦτα μετὰ δακρύων 
ἐκαθέζετο ἡ Μυγδονία σιωπῶσα καὶ εἰς τὸ ἔδαφος ἀφορῶσα· 
ὃ δὲ αὖθις προσελθὼν εἶπεν· Κυρία μου ποθεινοτάτη Μυγ-
δονία, ὑπομνήσθητι ὅτι σὲ ἐκ πάντων τῶν ἐν τῇ Ἰνδίᾳ γυ-
ναικῶν ὡς καλλίστην ἐπελεξάμην καὶ ἔλαβον, δυνηθεὶς ἑτέρας 
πολλῷ σου καλλίω εἰς γάμον συνάψαι ἐμαυτῷ. 



Acta Thomae, Acta Thomae 
Section 116, line 8

                                                    μᾶλλον δὲ 
ψεύδομαι Μυγδονία· μὰ τοὺς γὰρ θεοὺς οὐκ ἂν ἔσται ἑτέραν 
κατὰ σὲ ἐν τῇ τῶν Ἰνδῶν εὑρεθῆναι χώρᾳ· οὐαὶ δέ μοι διὰ 
παντός, ὅτι οὐδὲ   
λόγῳ ἀμείψασθαι θέλεις· 
ὕβριζε δέ μοι εἰ δοκεῖ σοι, ἵνα 
λόγον μόνον παρὰ σοῦ κατα-
ξιωθῶ. 



Acta Thomae, Acta Thomae 
Section 117, line 26

                                μὴ οὖν ἀντὶ μηδενὸς θῇς τοὺς 
ἐμοὺς λόγους καὶ ποιήσῃς με ὄνειδος ἐν τοῖς Ἰνδοῖς. 



Acta Thomae, Acta Thomae 
Section 123, line 4

Ὁ δὲ Χαρίσιος ἅμα ἕωθεν πρὸς τὴν Μυγδονίαν 
ἤρχετο· εὗρεν δὲ αὐτὰς εὐχομένας καὶ λεγούσας· Νέε θεὲ ὃς 
ἦλθες διὰ τοῦ ξένου εἰς ἡμᾶς ὧδε· θεὲ ἐναπόκρυφε τῆς τῶν 
ἐν Ἰνδίᾳ οἰκητόρων· ὁ θεὸς ὁ δείξας τὴν σὴν δόξαν διὰ τοῦ 
ἀποστόλου σου Θωμᾶ· ὁ θεὸς οὗ τῆς φήμης ἀκούσασαι εἰς 
σὲ ἐπιστεύσαμεν· ὁ θεὸς πρὸς ὃν ἤλθομεν σωθῆναι· ὁ θεὸς 
ὁ διὰ φιλανθρωπίαν καὶ οἰκτιρμοὺς κατελθὼν πρὸς τὴν 
ἡμετέραν σμικρότητα· ὁ θεὸς ὁ ἐπιζητήσας ἡμᾶς ὅτε αὐτὸν 
ἠγνοοῦμεν· ὁ θεὸς ὁ τὰ ὕψη ἔχων καὶ τὰ βάθη μὴ λανθά-
νων· σὺ ἀπόστρεψον τὴν μανίαν Χαρισίου ἀφ' ἡμῶν. 



Acta Thomae, Acta Thomae 
Section 134, line 7

Μισδαῖος δὲ ὁ βασιλεὺς ἀπολύσας Ἰούδαν δειπνήσας 
ἀπῄει οἴκαδε, διηγεῖτο δὲ τῇ γυναικὶ τὰ συμβεβηκότα τῷ οἰκείῳ 
αὐτῶν Χαρισίῳ λέγων· Ὅρα φησὶν τί γέγονεν ἐν τῷ ἀθλίῳ 
ἐκείνῳ· οἶδας δὲ καὶ αὐτὴ ὦ ἀδελφή μου Τερτία ὅτι οὐδέν   
ἐστιν ἀνδρὶ καλλίω τῆς γυναικὸς τῆς ἰδίας ἐφ' ἣν ἀναπέπαυ-
ται· συνέβη δὲ τὴν γυναῖκα αὐτοῦ ἀπελθεῖν πρὸς τὸν φαρ-
μακὸν ἐκεῖνον ὃν ἤκουσας τῇ Ἰνδῶν ἐπιδημήσαντα χώρᾳ, τοῖς 
αὐτοῦ περιπεσεῖν φαρμάκοις καὶ τοῦ ἰδίου ἀνδρὸς δια-
ζευχθῆναι· καὶ ἀπορεῖ ὃ πράξειεν. 



Acta Thomae, Acta Thomae 
Section 136, line 8

                            Ἡ δὲ Μυγδονία λέγει αὐτῇ· Ἐν τῷ 
οἴκῳ ἐστὶν Σιφόρου τοῦ στρατηλάτου· καὶ γὰρ αὐτὸς γέγονεν 
πρόφασις πᾶσιν τοῖς ἐν τῇ Ἰνδίᾳ σῳζομένοις. 



Acta Thomae, Acta Thomae 
Section 171, line 2

Ἐπληρώθησαν αἱ πράξεις Ἰούδα Θωμᾶ τοῦ ἀπο-
στόλου ἃς ἔπραξεν εἰς τὴν Ἰνδῶν, πληρῶν τὸ πρόσταγμα τοῦ   
πέμψαντος αὐτόν· ᾧ ἡ δόξα εἰς τοὺς αἰῶνας τῶν αἰώνων. 



Acta Thomae, Carmen animae (De margarita) (cod. Rom. vallicellanus B 35) (2038: 002)
“L'hymne de la perle des actes de Thomas”, Ed. Poirier, P.–H.
Louvain–La–Neuve: Université Catholique de Louvain, 1981; Homo religiosus 8.
Line 7

μου βασιλείοις 
ἐν πλούτῳ καὶ τρυφῇ τῶν τροφέων ἀναπαυόμενος 
ἐξ Ἀνατολῆς τῆς πατρίδος ἡμῶν ἐφοδιάσαντές με οἱ 
γονεῖς ἀπέστειλάν με· 
ἀπὸ δὲ πλούτου τῶν θησαυρῶν τούτων φόρτον συνέθηκαν 
μέγαν τε καὶ ἐλαφρὸν ὅπως αὐτὸν μόνος βαστάσαι 
δυνηθῶ· 
χρυσός ἐστιν ὁ φόρτος τῶν ἄνω καὶ ἄσημος τῶν μεγάλων 
θησαυρῶν 
καὶ λίθοι ἐξ Ἰνδῶν οἱ χαλκεδόνιοι καὶ μαργαρῖται ἐκ 
Κοσάνων· 
καὶ ὥπλισάν με τῷ ἀδάμαντι· 
καὶ <ἐξέδυσάν> με ἐσθῆτα διάλιθον χρυσόπαστον * ἣν 
ἐποίησαν στέργοντές με 
καὶ στολὴν τὸ χρῶμα ξανθὴν πρὸς τὴν ἐμὴν ἡλικίαν· 
σύμφωνα δὲ πρὸς ἐμὲ πεποιήκασιν ἐγκαταγράψαντες τῇ 
διανοίᾳ μου <τοῦ μὴ> ἐπιλαθέσθαι με· ἔφησάν τε· 
ἐὰν κατελθὼν εἰς Αἴγυπτον κομίσῃς ἐκεῖθεν τὸν ἕνα 
μαργαρίτην 




Acta Thomae, Acta Thomae (recensio) (2038: 004)
“Acta apostolorum apocrypha, vol. 2.2”, Ed. Bonnet, M.
Leipzig: Mendelssohn, 1903, Repr. 1972.
Section 16, line 14

                                             μετὰ δὲ χρόνον ἤκου-
σαν περὶ τοῦ ἀποστόλου ὅτι ἐν Ἰνδίᾳ ἐστὶν καὶ διδάσκει· καὶ 
ἀπελθόντες ἐβαπτίσθησαν ἀμφότεροι. 



Acta Thomae, Acta Thomae (recensio) 
Section 16, line 17t

Πράξεις τοῦ ἁγίου ἀποστόλου Θωμᾶ ὅτε εἰσῆλθεν ἐν τῇ 
Ἰνδίᾳ καὶ τὸ ἐν οὐρανοῖς παλάτιον ᾠκοδόμησεν. 




Acta Thomae, Acta Thomae (recensio) 
Section 17, line 1

Ὅτε δὲ εἰσῆλθεν ὁ ἀπόστολος ἐν τῇ Ἰνδίᾳ μετὰ Ἀβ-
βάνη τοῦ ἐμπόρου, εὐθέως ἀνήγαγεν τῷ βασιλεῖ περὶ τοῦ οἰκο-
δόμου· ὁ δὲ βασιλεὺς χαρᾶς πλησθεὶς ἐκέλευσεν εἰσελθεῖν τὸν 
Θωμᾶν. 


\end{greek}

\section{Pseudo-Nonnus}
\blockquote[From Wikipedia\footnote{\url{}}]{?}
\begin{greek}

Pseudo-Nonnus, Scholia mythologica (3127: 001)
“Pseudo–Nonniani in iv orationes Gregorii Nazianzeni commentarii”, Ed. Nimmo Smith, J.
Turnhout: Brepols, 1992; Corpus Christianorum. Series Graeca 27.
Oration 4, historia 74, line 3

Πᾶσα πόλις κατ' ἐξαίρετόν τι εἶχεν ἰδίωμα, οἷον ἡ Θετταλῶν 
χώρα ἔσχε τοὺς ἵππους, ἡ Ἀθηναίων τὰ μέταλλα τοῦ ἀργύρου, 
ἡ Ἰνδία τὴν χρυσίτιδα ψάμμον, ὁμοίως καὶ ἡ Λακεδαίμων 
κύνας θηρευτικοὺς (ἔνθεν καὶ οἱ Λακωνικοὶ κύνες) καὶ γυναῖ-
κας ἀνδρείας καὶ ἀπτοήτους. 

\end{greek}



\section{Joannes Malalas}
\blockquote[From Wikipedia\footnote{\url{http://en.wikipedia.org/wiki/Joannes_Malalas}}]{John Malalas or Ioannes Malalas (or Malelas) (Greek: Ἰωάννης Μαλάλας) (c. 491 – 578) was a Greek chronicler from Antioch. Malalas is probably a Syriac word for "rhetor", "orator"; it is first applied to him by John of Damascus (the form Malelas is later, first appearing in Constantine VII).[1]



He wrote a Chronographia (Χρονογραφία) in 18 books, the beginning and the end of which are lost. In its present state it begins with the mythical history of Egypt and ends with the expedition to Roman Africa under the tribune Marcianus, Justinian's nephew, in 563 (his editor Thurn believes it originally ended with Justinian's death[4]); it is focused largely on Antioch and (in the later books) Constantinople. Except for the history of Justinian and his immediate predecessors, it possesses little historical value; the author, "relying on Eusebius of Caesarea and other compilers, confidently strung together myths, biblical stories, and real history."[5] The eighteenth book, dealing with Justinian's reign, is well acquainted with, and colored by, official propaganda. The writer is a supporter of Church and State, an upholder of monarchical principles. (However, the theory identifying him with the patriarch John Scholasticus is almost certainly incorrect.[6])}
\begin{greek}

Joannes Malalas Chronogr., Chronographia (2871: 001)
“Ioannis Malalae chronographia”, Ed. Dindorf, L.
Bonn: Weber, 1831; Corpus scriptorum historiae Byzantinae.
Page 42, line 5

                                                                  ὅστις 
καὶ πρὸς Πέρσας καὶ πρὸς Ἰνδοὺς καὶ εἰς πολλὰς χώρας ἀπῆλθε, 
καὶ πολεμῶν φαντασίας τινὰς θαυμάτων ἐδείκνυεν, ἔχων καὶ 
στρατὸν μεθ' ἑαυτοῦ πολύν. 



Joannes Malalas Chronogr., Chronographia 
Page 127, line 10

Μετὰ δὲ ὀλίγας ἡμέρας ὁ Τιθών τις ὀνόματι ὑπὸ τοῦ Πριά-
μου παρακληθεὶς παραγίνεται, ἄγων Ἰνδοὺς ἐφίππους καὶ πεζοὺς 
καὶ Φοίνικας μαχιμωτάτους μετ' αὐτῶν καὶ τὸν βασιλέα αὐτῶν 
Πολυδάμαντα. 



Joannes Malalas Chronogr., Chronographia 
Page 127, line 14

                                      καὶ διὰ ναυτικοῦ στόλου ἦλθον 
πολλοὶ Ἰνδοὶ καὶ οἱ αὐτῶν βασιλεῖς· ἐδιῳκοῦντο δὲ τασσόμενοι 
πάντες οἱ βασιλεῖς καὶ πᾶς ὁ στρατὸς ὑπὸ τοῦ δυνατοῦ Μέμνο-
νος, βασιλέως Ἰνδῶν. 



Joannes Malalas Chronogr., Chronographia 
Page 128, line 14

                                                                    καὶ 
πρὶν ἢ ἥλιον ἀνελθεῖν ἐξερχόμεθα οἱ Ἕλληνες ὁπλισάμενοι πάν-
τες, ὁμοίως δὲ καὶ οἱ Τρῶες καὶ ὁ Μέμνων, βασιλεὺς Ἰνδῶν, 
καὶ πάντα τὰ πλήθη αὐτῶν. 



Joannes Malalas Chronogr., Chronographia 
Page 128, line 17

                                καὶ συμβολῆς γενομένης καὶ πολ-
λῶν πεσόντων, ὁ ἐμὸς ἀδελφὸς Αἴας κελεύσας τοῖς βασιλεῦσι τῶν 
Ἑλλήνων τοὺς ἄλλους ἀμύνασθαι Ἰνδοὺς καὶ Τρῶας, ὁρμᾷ κατὰ 
τοῦ Μέμνονος, βασιλέως Ἰνδῶν, τοῦ ἥρωος Ἀχιλλέως, τοῦ σοῦ 
γενέτου, ὄπισθεν συνεπισχύοντος τῷ Αἴαντι, ἑαυτὸν ἀποκρύ-
πτων. 



Joannes Malalas Chronogr., Chronographia 
Page 154, line 5

                                                           καὶ 
ἀδημονῶν ἠβούλετο φυγεῖν ἐπὶ τὴν Ἰνδικὴν χώραν. 



Joannes Malalas Chronogr., Chronographia 
Page 194, line 13

               παρέλαβε δὲ καὶ πάντα τὰ Ἰνδικὰ μέρη καὶ τὰ 
βασίλεια αὐτῶν ὁ αὐτὸς Ἀλέξανδρος, λαβὼν καὶ Πῶρον τὸν βα-
σιλέα Ἰνδῶν αἰχμάλωτον· καὶ τὰ ἄλλα πάντα τὰ τῶν ἐθνῶν βα-
σίλεια δίχα τῆς βασιλείας τῆς Κανδάκης τῆς χήρας τῆς βασι-
λευούσης τῶν ἐνδοτέρων Ἰνδῶν· ἥτις συνελάβετο τὸν αὐτὸν 
Ἀλέξανδρον τῷ τρόπῳ τούτῳ. 


Joannes Malalas Chronogr., Chronographia 
Page 429, line 14

Ἐν δὲ τοῖς χρόνοις τούτοις, ὡς προεῖπον, ἐβασίλευσεν ὁ θειό-
τατος Ἰουστινιανός, τῶν δὲ Περσῶν βασιλεὺς Κωάδης ὁ Δα-
ράσθενος, ὁ υἱὸς Περόζου, ἐν δὲ Ῥώμῃ Ἀλλάριχος, ἔκγονος τοῦ 
Οὐαλεμεριακοῦ, τῆς δὲ Ἀφρικῆς ῥὴξ Γιλδέριχ ὁ ἔκγονος Γινζι-
ρίχου, τῶν δὲ Ἰνδῶν Αὐξουμιτῶν ἐβασίλευσεν Ἄνδας ὁ γεγονὼς 
χριστιανός, τῶν δὲ Ἰβήρων Σαμαναζός. 



Joannes Malalas Chronogr., Chronographia 
Page 433, line 3

Ἐν αὐτῷ δὲ τῷ χρόνῳ συνέβη Ἰνδοὺς πολεμῆσαι πρὸς ἑαυ-
τοὺς οἱ ὀνομαζόμενοι Αὐξουμῖται καὶ οἱ Ὁμηρῖται· ἡ δὲ αἰτία 
τοῦ πολέμου αὕτη. 



Joannes Malalas Chronogr., Chronographia 
Page 433, line 9

                                                          οἱ δὲ πραγμα-
τευταὶ Ῥωμαίων διὰ τῶν Ὁμηριτῶν εἰσέρχονται εἰς τὴν Αὐξού-
μην καὶ ἐπὶ τὰ ἐνδότερα βασίλεια τῶν Ἰνδῶν. 



Joannes Malalas Chronogr., Chronographia 
Page 433, line 9

                                                       εἰσὶ γὰρ Ἰνδῶν 
καὶ Αἰθιόπων βασίλεια ἑπτά, τρία μὲν Ἰνδῶν, τέσσαρα δὲ Αἰ-
θιόπων, τὰ πλησίον ὄντα τοῦ Ὠκεανοῦ ἐπὶ τὰ ἀνατολικὰ μέρη. 



Joannes Malalas Chronogr., Chronographia 
Page 434, line 10

Καὶ μετὰ τὴν νίκην ἔπεμψε συγκλητικοὺς αὐτοῦ δύο καὶ 
μετ' αὐτῶν διακοσίους ἐν Ἀλεξανδρείᾳ, δεόμενος τοῦ βασιλέως 
Ἰουστινιανοῦ ὥστε λαβεῖν αὐτὸν ἐπίσκοπον καὶ κληρικοὺς καὶ 
κατηχηθῆναι καὶ διδαχθῆναι τὰ χριστιανῶν μυστήρια καὶ φω-
τισθῆναι καὶ πᾶσαν τὴν Ἰνδικὴν χώραν ὑπὸ Ῥωμαίους γενέσθαι. 



Joannes Malalas Chronogr., Chronographia 
Page 434, line 14

                                              καὶ ἐπελέξαντο οἱ αὐ-
τοὶ πρεσβευταὶ Ἰνδοὶ τὸν παραμονάριον τοῦ ἁγίου Ἰωάννου τοῦ 
ἐν Ἀλεξανδρείᾳ, ἄνδρα εὐλαβῆ, παρθένον, ὀνόματι Ἰωάννην, 
ὄντα ἐνιαυτῶν ὡς ἑξήκοντα δύο. 



Joannes Malalas Chronogr., Chronographia 
Page 434, line 18

                                       καὶ λαβόντες τὸν ἐπίσκοπον 
καὶ τοὺς κληρικούς, οὓς αὐτὸς ἐπελέξατο, ἀπήγαγον εἰς τὴν 
Ἰνδικὴν χώραν πρὸς Ἄνδαν τὸν βασιλέα αὐτῶν. 



Joannes Malalas Chronogr., Chronographia 
Page 434, line 22

    ὁ δὲ Ἀρέθας φοβηθεὶς εἰσῆλθεν εἰς τὸ ἐνδότερον λίμιτον 
ἐπὶ τὰ Ἰνδικά. 



Joannes Malalas Chronogr., Chronographia 
Page 435, line 9

                     καὶ εὐθέως ἀπελθόντες Ἀρέθας ὁ φύλαρχος 
καὶ Γνούφας καὶ Νααμὰν καὶ Διονύσιος ὁ δοὺξ Φοινίκης καὶ 
Ἰωάννης ὁ τῆς Εὐφρατησίας καὶ Σεβαστιανὸς ὁ χιλίαρχος μετὰ 
τῆς στρατιωτικῆς βοηθείας· καὶ μαθὼν Ἀλαμούνδαρος ὁ Σα-
ρακηνὸς ἔφυγεν εἰς τὰ Ἰνδικὰ μέρη μεθ' ἧς εἶχε βοηθείας Σα-
ρακηνικῆς. 



Joannes Malalas Chronogr., Chronographia 
Page 444, line 9

Ἐν δὲ τῷ αὐτῷ χρόνῳ ἀνεφάνη ἐν τοῖς Περσικοῖς μέρεσι 
δόγμα Μανιχαϊκόν· καὶ μαθὼν ὁ τῶν Περσῶν βασιλεὺς ἠγα-
νάκτησεν· ὡσαύτως δὲ καὶ οἱ ἀρχιμάγοι τῶν Περσῶν· ἦσαν 
γὰρ ποιήσαντες οἱ αὐτοὶ Μανιχαῖοι καὶ ἐπίσκοπον, ὀνόματι Ἰν-
δαράζαρ. 



Joannes Malalas Chronogr., Chronographia 
Page 447, line 12

        ἔλαβε δὲ καὶ ὁ φύλαρχος Σαρακηνὸς ὁ τῶν Ῥωμαίων 
πραῖδαν ἐξ αὐτῶν χιλιάδας εἴκοσι παίδων καὶ κορασίων· οὕς-
τινας λαβὼν αἰχμαλώτους ἐπώλησεν ἐν τοῖς Περσικοῖς καὶ 
Ἰνδικοῖς μέρεσιν. 



Joannes Malalas Chronogr., Chronographia 
Page 457, line 3

Ὁ δὲ βασιλεὺς Ῥωμαίων ἀκούσας παρὰ τοῦ πατρικίου Ῥου-  
φίνου τὴν παρὰ Κωάδου, βασιλέως Περσῶν, παράβασιν, ποιήσας 
θείας κελεύσεις κατέπεμψε πρὸς τὸν βασιλέα τῶν Αὐξουμιτῶν· 
ὅστις βασιλεὺς Ἰνδῶν συμβολὴν ποιήσας μετὰ τοῦ βασιλέως τῶν 
Ἀμεριτῶν Ἰνδῶν, κατὰ κράτος νικήσας παρέλαβε τὰ βασίλεια 
αὐτοῦ καὶ τὴν χώραν αὐτοῦ πᾶσαν, καὶ ἐποίησεν ἀντ' αὐτοῦ βα-
σιλέα τῶν Ἀμεριτῶν Ἰνδῶν ἐκ τοῦ ἰδίου γένους Ἀγγάνην διὰ 
τὸ εἶναι καὶ τὸ τῶν Ἀμεριτῶν Ἰνδῶν βασίλειον ὑπ' αὐτόν. 



Joannes Malalas Chronogr., Chronographia 
Page 457, line 9

                                                                        καὶ 
ἀποπλεύσας ὁ πρεσβευτὴς Ῥωμαίων ἐπὶ Ἀλεξάνδρειαν διὰ τοῦ 
Νείλου ποταμοῦ καὶ τῆς Ἰνδικῆς θαλάσσης κατέφθασε τὰ Ἰνδικὰ 
μέρη. 



Joannes Malalas Chronogr., Chronographia 
Page 457, line 10

       καὶ εἰσελθὼν παρὰ τῷ βασιλεῖ τῶν Ἰνδῶν, μετὰ χαρᾶς 
πολλῆς ἐξενίσθη ὁ βασιλεὺς Ἰνδῶν, ὅτι διὰ πολλῶν χρόνων ἠξιώ-
θη μετὰ τοῦ βασιλέως Ῥωμαίων κτήσασθαι φιλίαν. 



Joannes Malalas Chronogr., Chronographia 
Page 457, line 13

                                                       ὡς δὲ 
ἐξηγήσατο ὁ αὐτὸς πρεσβευτής, ὅτε ἐδέξατο αὐτὸν ὁ τῶν Ἰνδῶν 
βασιλεύς, ὑφηγήσατο τὸ σχῆμα τῆς βασιλικῆς τῶν Ἰνδῶν κατα-
στάσεως ὅτι γυμνὸς ὑπῆρχε καὶ κατὰ τοῦ ζώσματος εἰς τὰς ψύας 
αὐτοῦ λινόχρυσα ἱμάτια, κατὰ δὲ τῆς γαστρὸς καὶ τῶν ὤμων 
φορῶν σχιαστὰς διὰ μαργαριτῶν καὶ κλαβία ἀνὰ πέντε καὶ 
χρυσᾶ ψέλια εἰς τὰς χεῖρας αὐτοῦ, ἐν δὲ τῇ κεφαλῇ αὐτοῦ λινό-
χρυσον φακιόλιν ἐσφενδονισμένον, ἔχον ἐξ ἀμφοτέρων τῶν μερῶν 
σειρὰς τέσσαρας, καὶ μανιάκιν χρυσοῦν ἐν τῷ τραχήλῳ αὐτοῦ, 
καὶ ἵστατο ὑπεράνω τεσσάρων ἐλεφάντων ἐχόντων ζυγὸν καὶ 
τροχοὺς δʹ, καὶ ἐπάνω, ὡς ὄχημα ὑψηλὸν ἠμφιεσμένον χρυσέοις 




Joannes Malalas Chronogr., Chronographia 
Page 458, line 1

                        καὶ ἵστατο ἐπάνω ὁ βασιλεὺς τῶν Ἰνδῶν 
βαστάζων σκουτάριον μικρὸν κεχρυσωμένον καὶ δύο λαγκίδια 
καὶ αὐτὰ κεχρυσωμένα κατέχων ἐν ταῖς χερσὶν αὐτοῦ. 



Joannes Malalas Chronogr., Chronographia 
Page 458, line 7

Καὶ εἰσενεχθεὶς ὁ πρεσβευτὴς Ῥωμαίων, κλίνας τὸ γόνυ 
προσεκύνησε· καὶ ἐκέλευσεν ὁ βασιλεὺς Ἰνδῶν ἀναστῆναί με καὶ 
ἀναχθῆναι πρὸς αὐτόν. 



Joannes Malalas Chronogr., Chronographia 
Page 458, line 15

                                                      λύσας δὲ καὶ 
ἀναγνοὺς δι' ἑρμηνέως τὰ γράμματα, εὗρε περιέχοντα ὥστε 
ὁπλίσασθαι αὐτὸν κατὰ Κωάδου, βασιλέως Περσῶν, καὶ τὴν 
πλησιάζουσαν αὐτῷ χώραν ἀπολέσαι καὶ τοῦ λοιποῦ μηκέτι συν-
άλλαγμα ποιῆσαι μετ' αὐτοῦ, ἀλλὰ δι' ἧς ὑπέταξε χάρας τῶν 
Ἀμεριτῶν Ἰνδῶν διὰ τοῦ Νείλου ἐπὶ τὴν Αἴγυπτον ἐν Ἀλεξαν-
δρείᾳ τὴν πραγματείαν ποιεῖσθαι. 



Joannes Malalas Chronogr., Chronographia 
Page 458, line 17

                                      καὶ εὐθέως ὁ βασιλεὺς Ἰν-
δῶν Ἐλεσβόας ἐπ' ὄψεσι τοῦ πρεσβευτοῦ Ῥωμαίων ἐκίνησε 
πόλεμον κατὰ Περσῶν, προπέμψας καὶ τοὺς ὑπ' αὐτὸν Ἰνδοὺς 
Σαρακηνούς, ἐπῆλθε τῇ Περσικῇ χώρᾳ ὑπὲρ Ῥωμαίων, δηλώσας 
τῷ βασιλεῖ Περσῶν τοῦ δέξασθαι τὸν βασιλέα Ἰνδῶν πολεμοῦντα 
αὐτῷ καὶ ἐκπορθῆσαι πᾶσαν τὴν ὑπ' αὐτοῦ βασιλευομένην γῆν. 



Joannes Malalas Chronogr., Chronographia 
Page 458, line 22

καὶ πάντων οὕτως προβάντων ὁ βασιλεὺς Ἰνδῶν κρατήσας τὴν   
κεφαλὴν τοῦ πρεσβευτοῦ Ῥωμαίων, δεδωκὼς εἰρήνης φίλημα, 
ἀπέλυσεν ἐν πολλῇ θεραπείᾳ. 



Joannes Malalas Chronogr., Chronographia 
Page 459, line 3

                                   κατέπεμψε γὰρ καὶ σάκρας διὰ 
Ἰνδοῦ πρεσβευτοῦ καὶ δῶρα τῷ βασιλεῖ Ῥωμαίων. 




Joannes Malalas Chronogr., Chronographia 
Page 477, line 7

Ἐν αὐτῷ δὲ τῷ χρόνῳ καὶ πρεσβευτὴς Ἰνδῶν μετὰ δώρων 
κατεπέμφθη ἐν Κωνσταντινουπόλει. 



Joannes Malalas Chronogr., Chronographia 
Page 478, line 22

Ἐν αὐτῷ δὲ τῷ χρόνῳ ἔῤῥιψε καὶ τὴν τρίτην ὑπατείαν ὁ αὐ-
τὸς βασιλεύς, ἀνακαλεσάμενος τοὺς πατρικίους Ὀλύβριον καὶ 
Πρόβον, ὄντας ἐν ἐξορίᾳ, δεδωκὼς αὐτοῖς πάντα τὰ ὑπάρχοντα 
αὐτῶν. 
 Ἰνδικτιῶνος ιβʹ παρελήφθη ὁ ῥὴξ Ἀφρικῆς μετὰ τῆς αὐτοῦ   
γυναικὸς ὑπὸ Βελισαρίου· καὶ εἰσηνέχθησαν ἐν Κωνσταντινουπό-
λει. 



Joannes Malalas Chronogr., Chronographia 
Page 484, line 9

Ἰνδικτιῶνος ιγʹ πρεσβευτὴς Ἰνδῶν κατεπέμφθη μετὰ καὶ 
ἐλέφαντος ἐν Κωνσταντινουπόλει. 



Joannes Malalas Chronogr., Chronologica (fort. auctore anonymo excerptorum chronologicorum) (2871: 002)
“Ioannis Malalae chronographia”, Ed. Dindorf, L.
Bonn: Weber, 1831; Corpus scriptorum historiae Byzantinae.
Page 14, line 5

Μετὰ γοῦν τὴν σύγχυσιν καὶ τὴν τοῦ πύργου διάλυσιν μετα-
στέλλονται οἱ τρεῖς υἱοὶ τοῦ Νῶε πάντας τοὺς ἐξ αὐτῶν γενομέ-
νους, καὶ διδόασιν αὐτοῖς ἔγγραφον τῶν τόπων τὴν κατανέμησιν,   
ἥνπερ ἐκ τοῦ πατρὸς Νῶε παρειλήφασι, καὶ λαγχάνουσιν ἑκά-
στῳ καὶ ταῖς ἑκάστου φυλαῖς καὶ πατριαῖς τόπον καὶ κλίματα 
καὶ χώρας καὶ νήσους καὶ ποταμοὺς κατὰ τὴν ὑποκειμένην ἔκθε-
σιν, καὶ κατακληροῦνται τῷ μὲν πρωτοτόκῳ υἱῷ Νῶε Σὴμ ἀπὸ 
Περσίδος καὶ Βάκτρων ἕως Ἰνδικῆς καὶ Ῥινοκουρούρων τὰ πρὸς 
ἀνατολήν, τῷ δὲ Χὰμ ἀπὸ Ῥινοκουρούρων ἕως Γαδείρων τὰ 
πρὸς νότον, τῷ δὲ Ἰάφεθ ἀπὸ Μηδίας ἕως Γαδείρων τὰ πρὸς 
βοῤῥᾶν. 



Joannes Malalas Chronogr., Chronologica (fort. auctore anonymo excerptorum chronologicorum) 
Page 14, line 11

Αἱ δὲ λαχοῦσαι χῶραι τῷ μὲν Σὴμ εἰσὶν αὗται, Περσίς, 
Βακτριανή, Ὑρκανία, Βαβυλωνία, Κορδυαία, Ἀσσυρία, Με-
σοποταμία, Ἀραβία ἡ ἀρχαία, Ἐλυμαΐς, Ἰνδική, Ἀραβία ἡ 
εὐδαίμων, κοίλη Συρία, Κομμαγηνὴ καὶ Φοινίκη πᾶσα καὶ πο-
ταμὸς Εὐφράτης· τῷ δὲ Χὰμ Αἴγυπτος, Αἰθιοπία ἡ βλέπουσα 
κατ' Ἰνδούς, ἑτέρα Αἰθιοπία, ὅθεν ἐκπορεύεται ὁ ποταμὸς τῶν 
Αἰθιόπων, ἐρυθρὰ βλέπουσα κατ' ἀνατολάς, Θηβαΐς, Λιβύη 
ἡ παρεκτείνουσα μέχρι Κυρήνης, Μαρμαρίς, Σύρτις, Λιβύη 
ἄλλη, Νουμιδία, Μασσυρίς, Μαυριτανία ἡ κατέναντι Γαδεί-
ρων· ἐν δὲ τοῖς κατὰ βοῤῥᾶν τὰ παρὰ θάλασσαν ἔχει Κιλικίαν,   
Παμφυλίαν, Πισιδίαν, Μυσίαν, Λυκαονίαν, Φρυγίαν, Κα-
μαλίαν, Λυκίαν, Καρίαν, Λυδίαν, Μυσίαν ἄλλην, Τρωάδα, 




Joannes Malalas Chronogr., Chronographia (eclogae e cod. Paris. gr. 1336) (2871: 003)
“Anecdota Graeca e codd. manuscriptis bibliothecae regiae Parisiensis, vol. 2”, Ed. Cramer, J.A.
Oxford: Oxford University Press, 1839, Repr. 1967.
Page 234, line 6

Ἐν δὲ τοῖς ἀνωτέρω χρόνοις ἐκ τοῦ γένους τοῦ Ἀρφαξὰδ ἀνε-
φύει τίς Ἰνδὸς σοφὸς ἀνὴρ ἀστρονόμος, ὀνόματι Γανδουβάριος, 
ὅστις συνεγράψατο πρῶτος ἀστρονομίαν Ἰνδοῖς· ἐκ δὲ τῆς αὐτῆς 
φυλῆς τοῦ Σὴμ, τῆς κρατησάσης τὴν Ἀσσυρίαν καὶ τὴν Περ-
σῖδα, καὶ τὰ μέρη τῆς ἀνατολῆς, ἀνεφάνη ἄνθρωπος γίγας τὸ 
γένος, ὀνόματι Κρόνος, ἐπικληθεὶς ὑπὸ τοῦ ἰδίου πατρὸς Δόμνος, 
εἰς τὴν ἐπωνυμίαν τοῦ πλανήτου ἀστέρος Κρόνου. 

\end{greek}


\section{Alexander romance}
\blockquote[From Wikipedia\footnote{\url{http://en.wikipedia.org/wiki/Alexander_romance}}]{Alexander romance is any of several collections of legends concerning the mythical exploits of Alexander the Great. The earliest version is in Greek, dating to the 3rd century. Several late manuscripts attribute the work to Alexander's court historian Callisthenes, but the historical figure died before Alexander and could not have written a full account of his life. The unknown author is still sometimes called Pseudo-Callisthenes.}
\begin{greek}


Anonymi Historici (FGrH), Alexandri historia (e cod. Sabbaitico 29) (1139: 006)
“FGrH \#151”.
Volume-Jacobyʹ-F 2b,151,F, fragment 1, line 96

ἦσαν δὲ αὐτῶι καὶ ἐλέφαντες ἠγμένοι ἀπὸ τῆς Ἰνδικῆς, ὧν ἡ παρασκευὴ 
τοῦτον εἶχεν τὸν τρόπον· πύργοι ξύλινοι κατεσκευασμένοι ἐπετίθεντο 
τοῖς νώτοις αὐτῶν, ἀφ' ὧν ἄνδρες ἀπεμάχοντο ἐν ὅπλοις, ὡς συμβαίνειν 
τοὺς ἀντιπαρατασσομένους διαφθείρεσθαι καὶ ὑπὸ τῶν ἀπομαχομένων 
ἀνδρῶν διαφθειρομένους καὶ ὑπὸ τῶν ἐλεφάντων πατουμένους. 



Anonymi Historici (FGrH), De historia Alexandri (1139: 007)
“FGrH \#153”.
Volume-Jacobyʹ-T+F 2b,153,T, fragment 4, line 3

STRABON XI 5, 5: (Kleitarch. 137 F 16) καὶ τὰ πρὸς τὸ ἔνδοξον θρυλη-
θέντα κἂν ὡμολόγηται παρὰ πάντων, οἵ γε πλάσαντες ἦσαν οἱ κολακείας μᾶλλον ἢ 
ἀληθείας φροντίζοντες· οἷον τὸ τὸν Καύκασον μετενεγκεῖν εἰς τὰ Ἰνδικὰ ὄρη καὶ 
τὴν πλησιάζουσαν ἐκείνοις ἑῶιαν θάλατταν ἀπὸ τῶν ὑπερκειμένων τῆς Κολχίδος καὶ 
τοῦ Εὐξείνου ὀρῶν. 



Anonymi Historici (FGrH), De historia Alexandri 
Volume-Jacobyʹ-T+F 2b,153,T, fragment 4, line 6

                         ταῦτα γὰρ οἱ Ἕλληνες [καὶ] Καύκασον ὠνόμαζον, διέχοντα τῆς 
Ἰνδικῆς πλείους ἢ τρισμυρίους σταδίους, καὶ ἐνταῦθα ἐμύθευσαν τὰ περὶ Προμηθέα 
καὶ τὸν δεσμὸν αὐτοῦ· ταῦτα γὰρ τὰ ὕστατα πρὸς ἕω ἐγνώριζον οἱ τότε. 



Anonymi Historici (FGrH), De historia Alexandri 
Volume-Jacobyʹ-T+F 2b,153,T, fragment 4, line 8

                                                                                      ἡ δὲ ἐπὶ 
Ἰνδοὺς στρατεία Διονύσου καὶ Ἡρακλέους ὑστερογενῆ τὴν μυθοποιίαν ἐμφαίνει, ἅτε 
τοῦ Ἡρακλέους καὶ τὸν Προμηθέα λῦσαι λεγομένου χιλιάσιν ἐτῶν ὕστερον. 



Anonymi Historici (FGrH), De historia Alexandri 
Volume-Jacobyʹ-T+F 2b,153,T, fragment 4, line 10

                                                                                     καὶ ἦν 
μὲν ἐνδοξότερον τὸ τὸν Ἀλέξανδρον μέχρι τῶν Ἰνδικῶν ὀρῶν καταστρέψασθαι τὴν 
Ἀσίαν ἢ μέχρι τοῦ μυχοῦ τοῦ Εὐξείνου καὶ τοῦ Καυκάσου· ἀλλ' ἡ δόξα τοῦ ὄρους 
καὶ τοὔνομα καὶ τὸ τοὺς περὶ Ἰάσονα δοκεῖν μακροτάτην στρατείαν τελέσαι τὴν 
μέχρι τῶν πλησίον Καυκάσου καὶ τὸ τὸν Προμηθέα παραδεδόσθαι δεδεμένον ἐπὶ 
τοῖς ἐσχάτοις τῆς γῆς ἐν τῶι Καυκάσωι, χαριεῖσθαί τι τῶι βασιλεῖ ὑπέλαβον, τοὔνομα 
τοῦ ὄρους μετενέγκαντες εἰς τὴν Ἰνδικήν. 



Anonymi Historici (FGrH), De historia Alexandri 
Volume-Jacobyʹ-T+F 2b,153,F, fragment 2, line 6

                              ὅτι μὲν γὰρ μέγιστος τῶν μνημονευομένων κατὰ τὰς τρεῖς 
ἠπείρους, καὶ μετ' αὐτὸν ὁ Ἰνδός, τρίτος δὲ καὶ τέταρτος ὁ Ἴστρος καὶ ὁ Νεῖλος, 
ἱκανῶς συμφωνεῖται· τὰ καθ' ἕκαστα δ' ἄλλοι ἄλλως περὶ αὐτοῦ λέγουσιν, οἱ μὲν 
τριάκοντα σταδίων τοὐλάχιστον πλάτος, οἱ δὲ καὶ † τριῶν, Μεγασθένης (III) δέ, 
ὅταν ἦι μέτριος καὶ εἰς ἑκατὸν εὐρύνεσθαι, βάθος δὲ εἴκοσι ὀργυιῶν τοὐλάχιστον. 



Anonymi Historici (FGrH), De historia Alexandri 
Volume-Jacobyʹ-T+F 2b,153,F, fragment 7, line 26

                                                  τὸ γὰ̣ρ αὐτ̣[ὸ γέγρ]α̣φα (?) καὶ ἐ̣ν το[......] 
δ̣ο̣[......]α̣ρ̣ι̣ν̣δ̣ε καταλ̣[....]οις γυν[αι]κὸς [....]ε̣ν̣ | [........]α̣ κ̣αὶ ἡ Στερ̣ο̣π̣ὴ 
Ἀλέξανδρον αὐτόθ̣[..... | .......] ἀ̣ν̣έτειλε τῶν Φιλιππε̣[ίω]ν̣ υ̣ἱῶν σ̣ο̣φ̣ω̣[. | ........].. 
κεκλῆσθαι μᾶλλον [ἢ] ε̣ἶ̣ν̣αι Ἀντίπ[α]τρ̣οσ̣ | [ποιεῖ αὐ]τ̣ὴν βασιλίδα’ Ἀντίπατρος. 



Anonymi Historici (FGrH), De historia Alexandri 
Volume-Jacobyʹ-T+F 2b,153,F, fragment 9, line 15

                                   διαπορουμένου δὲ τοῦ Ἀλεξάνδρου περὶ τού<των> 
νοήσαντα τὸν Ἰνδὸν εἰπεῖν, ὅτι τοῖς ἀπόροις τῶν ἐρωτημά|των ἀπόρους εἶναι καὶ 
τὰς ἀποκρίσεις συμβαίνει. 



Anonymi Historici (FGrH), De historia Alexandri 
Volume-Jacobyʹ-T+F 2b,153,F, fragment 15c, line 1

          Σάνεια· πόλις Ἰνδική, ὡς Ἀδριανὸς Ἀλεξανδρειάδος <ζ>. 



\end{greek}



\section{Joannes Stobaeus}
\blockquote[From Wikipedia\footnote{\url{http://en.wikipedia.org/wiki/Joannes_Stobaeus}}]{Joannes Stobaeus (/dʒoʊˈænɨs stoʊˈbiːəs/;[1] Greek: Ἰωάννης ὁ Στοβαῖος; 5th-century CE), from Stobi in Macedonia, was the compiler of a valuable series of extracts from Greek authors. The work was originally divided into two volumes containing two books each. The two volumes became separated in the manuscript tradition, and the first volume became known as the Extracts (Eclogues) and the second volume became known as the Anthology (Florilegium). Modern editions now refer to both volumes as the Anthology. The Anthology contains extracts from hundreds of writers, especially poets, historians, orators, philosophers and physicians. The subjects covered range from natural philosophy, dialectics, and ethics, to politics, economics, and maxims of practical wisdom. The work preserves fragments of many authors and works who otherwise might be unknown today.


Books 1 and 2

The first two books ("Eclogues") consist for the most part of extracts conveying the views of earlier poets and prose writers on points of physics, dialectics, and ethics.[3] We learn from Photius that the first book was preceded by a dissertation on the advantages of philosophy, an account of the different schools of philosophy, and a collection of the opinions of ancient writers on geometry, music, and arithmetic.[3] The greater part of this introduction is lost. The close of it only, where arithmetic is spoken of, is still extant. The first book was divided into sixty chapters, the second into forty-six, of which the manuscripts preserve only the first nine.[3] Some of the missing parts of the second book (chapters 15, 31, 33, and 46) have, however, been recovered from a 14th-century gnomology.[5] His knowledge of physics — in the wide sense which the Greeks assigned to this term — is often untrustworthy.[2] Stobaeus betrays a tendency to confound the dogmas of the early Ionian philosophers, and he occasionally mixes up Platonism with Pythagoreanism.[2] For part of the first book and much of the second, it is clear that he depended on the (lost) works of the Peripatetic philosopher Aetius and the Stoic philosopher Arius Didymus.[2]
Books 3 and 4

The third and fourth books ("Florilegium") are devoted to subjects of a moral, political, and economic kind, and maxims of practical wisdom.[3] The third book originally consisted of forty-two chapters, and the fourth of fifty-eight.[3] These two books, like the larger part of the second, treat of ethics; the third, of virtues and vices, in pairs; the fourth, of more general ethical and political subjects, frequently citing extracts to illustrate the pros and cons of a question in two successive chapters.[2]}

\begin{greek}


Joannes Stobaeus Anthologus, Anthologium (2037: 001)
“Ioannis Stobaei anthologium, 5 vols.”, Ed. Wachsmuth, C., Hense, O.
Berlin: Weidmann, 1–2:1884; 3:1894; 4:1909; 5:1912, Repr. 1958.
Book 1, chapter 3, section 56, line 3

<Πορφυρίου ἐκ τοῦ Περὶ Στυγός> (ap. Holsten., 
opusc. Porph. p. 148). 
 Ἰνδοὶ οἱ ἐπὶ τῆς βασιλείας τῆς Ἀντωνίνου, τοῦ ἐξ   
Ἐμίσων ἐν τῇ Συρίᾳ [ἀφικομένου] Βαρδισάνῃ τῷ ἐκ τῆς 
Μεσοποταμίας εἰς λόγους ἀφικόμενοι, ἐξηγήσαντο, ὡς ὁ 
Βαρδισάνης ἀνέγραψεν, εἶναί τινα λίμνην ἔτι καὶ νῦν παρ' 
Ἰνδοῖς δοκιμαστήριον λεγομένην, εἰς ἥν, ἄν τις τῶν Ἰνδῶν 
αἰτίαν ἔχων τινὸς ἁμαρτίας ἀρνῆται, <εἰσάγεται>. 



Joannes Stobaeus Anthologus, Anthologium 
Book 1, chapter 3, section 56, line 7

Ἰνδοὶ οἱ ἐπὶ τῆς βασιλείας τῆς Ἀντωνίνου, τοῦ ἐξ   
Ἐμίσων ἐν τῇ Συρίᾳ [ἀφικομένου] Βαρδισάνῃ τῷ ἐκ τῆς 
Μεσοποταμίας εἰς λόγους ἀφικόμενοι, ἐξηγήσαντο, ὡς ὁ 
Βαρδισάνης ἀνέγραψεν, εἶναί τινα λίμνην ἔτι καὶ νῦν παρ' 
Ἰνδοῖς δοκιμαστήριον λεγομένην, εἰς ἥν, ἄν τις τῶν Ἰνδῶν 
αἰτίαν ἔχων τινὸς ἁμαρτίας ἀρνῆται, <εἰσάγεται>. 



Joannes Stobaeus Anthologus, Anthologium 
Book 1, chapter 3, section 56, line 26

                                     Ἑκουσίων τοίνυν ἁμαρ-
τημάτων δοκιμαστήριον Ἰνδοὺς τοῦτ' ἔχειν τὸ ὕδωρ· ἀκου-
σίων δὲ ὁμοῦ καὶ ἑκουσίων καὶ ὅλως ὀρθοῦ βίου ἕτερον 
εἶναι, περὶ οὗ ὁ Βαρδισάνης τάδε γράφει· θήσω γὰρ τἀ-
κείνου κατὰ λέξιν· “Ἔλεγον δὲ καὶ σπήλαιον εἶναι αὐτό-
ματον, μέγα, ἐν ὄρει ὑψηλοτάτῳ σχεδὸν κατὰ μέσον τῆς 
γῆς, ἐν ᾧ σπηλαίῳ ἐστὶν ἀνδριάς, ὃν εἰκάζουσι πηχῶν 
δέκα ἢ δώδεκα, ἑστὼς ὀρθός, ἔχων τὰς χεῖρας ἡπλωμένας   
ἐν τύπῳ σταυροῦ· καὶ τὸ μὲν δεξιὸν τῆς ὄψεως αὐτοῦ 
ἐστιν ἀνδρικόν, τὸ δ' εὐώνυμον θηλυκόν· ὁμοίως δὲ καὶ 
ὁ βραχίων ὁ δεξιὸς καὶ ὁ δεξιὸς ποῦς καὶ ὅλον τὸ μέρος 




Joannes Stobaeus Anthologus, Anthologium 
Book 1, chapter 3, section 56, line 86

            Ἃ μὲν οὖν Ἰνδοὶ ἱστοροῦσι περὶ τοῦ παρ' 
αὐτοῖς δοκιμαστηρίου ὕδατος, ἔστι ταῦτα. 



Joannes Stobaeus Anthologus, Anthologium 
Book 1, chapter 40, section 1, line 122

                                          Πρός γε μὴν ταῖς 
ἀνασχέσεσι τοῦ ἡλίου πάλιν εἰσρέων ὁ Ὠκεανός, τὸν Ἰν-
δικόν τε καὶ Περσικὸν διανοίξας κόλπον, ἀναφαίνει τὴν 
Ἐρυθρὰν θάλασσαν διειληφώς. 



Joannes Stobaeus Anthologus, Anthologium 
Book 1, chapter 40, section 1, line 134

                                              Ἐν τούτῳ γε μὴν 
νῆσοι μεγάλαι τυγχάνουσιν, αἵ [τε] δύο Βρετανικαὶ λεγό-
μεναι [καὶ] Ἄλβιον καὶ Ἰέρνη, τῶν προἱστορημένων μεί-  
ζους, ὑπὲρ Κελτοὺς κείμεναι· τούτων δὲ ἐλάττους ἥ τε 
Ταπροβάνη πέραν Ἰνδῶν, λοξὴ πρὸς τὴν οἰκουμένην, καὶ 
ἡ Φοβέα [Εὔβοια] καλουμένη, κατὰ τὸν Ἀράβιον κειμένη 
κόλπον. 



Joannes Stobaeus Anthologus, Anthologium 
Book 3, chapter 9, section 49, line 3

Ἐν Παιδαλίοις, Ἰνδικῷ ἔθνει, οὐχ ὁ θύων, ἀλλ' ὁ 
συνετώτατος τῶν παρόντων κατάρχεται τῶν ἱερῶν. 



Joannes Stobaeus Anthologus, Anthologium 
Book 4, chapter 2, section 25, line 98

                                       οἱ δὲ παῖδες παρ' 
αὐτοῖς ὥσπερ μάθημα τὸ ἀληθεύειν διδάσκονται. 
 Παρ' Ἰνδοῖς ἐάν τις ἀποστερηθῇ δανείου ἢ παρα-
καταθήκης, οὐκ ἔστι κρίσις, ἀλλ' αὑτὸν αἰτιᾶται ὁ πις-
τεύσας. 



Joannes Stobaeus Anthologus, Anthologium 
Book 4, chapter 22d, section 102, line 21

     ἐπειδὴ δὲ οὐκ εὔδηλόν ἐστιν ἄρτι γαμοῦντι, ὁποῖαι 
δή τινες τοῖς τρόποις αἱ γυναῖκες ἀναφανήσονται, Ἰνδοὶ 
οὕτω γαμοῦσι καὶ οἱ σοφοὶ αὐτῶν, καὶ οὐδεπώποτε 
ψευσθῆναι λέγονται. 



Joannes Stobaeus Anthologus, Anthologium 
Book 4, chapter 22d, section 102, line 23

                      ἐκεῖνοι τοίνυν οἱ Ἰνδοὶ οὐχ ὅπως 
πλούτῳ καὶ δόξῃ ἔγημαν ἐνδ<όξ>ου ἀνδρὸς καὶ πλουσίου 
θυγατέρα, ἀλλὰ αὐτὴν τὴν κόρην τήν τε ὄψιν αὐτῆς καὶ 
τὸ κάλλος ἐπολυπραγμόνησαν, σοφίᾳ δή τινι τοῦτό γε, 
οὐκ ἀκολασίᾳ οὐδεμιᾷ, οὐδὲ κατὰ τὰ αὐτὰ ἡμῖν. 



Joannes Stobaeus Anthologus, Anthologium 
Book 4, chapter 22d, section 102, line 39

                                 περὶ μὲν οὖν τῶν Ἰνδῶν 
οὗτος ὁ λόγος κρατεῖ, ὡς καταμαντεύονται περὶ τῶν γυ-
ναικῶν πόρρωθεν, ὁποῖαί τινες ἔσονται. 



Joannes Stobaeus Anthologus, Anthologium 
Book 4, chapter 22d, section 102, line 45

                                     οὐ γὰρ τοὺς καλοὺς 
καὶ τὰς καλὰς ἐγὼ προξενῶ, ἀλλ' ἑτέρως οὐκ ἔνι εἰπεῖν, 
ἑπόμενον τῷ Ἰνδῶν λόγῳ. 



Joannes Stobaeus Anthologus, Anthologium 
Book 4, chapter 28, section 19, line 36

                               ὥστ' οὔτε χρυσὸν ἀμφιθήσεται 
ἢ λίθον Ἰνδικὸν ἢ χώρης ἐόντα ἄλλης, οὐδὲ πλέξεται πο-
λυτεχνίῃσι τρίχας, οὐδ' ἀλείψεται Ἀραβίης ὀδμῆς ἐμ-
πνέοντα, οὐδὲ χρίσεται πρόσωπον λευκαίνουσα ἢ ἐρυθραί-
νουσα τοῦτο ἢ μελαίνουσα ὀφρύας τε καὶ ὀφθαλμοὺς καὶ 
τὴν πολιὴν τρίχα βαφαῖσι τεχνεωμένη, οὐδὲ λούσεται θα-
μινά. 



Joannes Stobaeus Anthologus, Anthologium 
Book 4, chapter 36, section 22, line 1

Κλειτοφῶντος Ῥοδίου ἐν αʹ Ἰνδικῶν (Ibid. 
25, 1. 3 l. c. p. 327). 



Joannes Stobaeus Anthologus, Anthologium 
Book 4, chapter 36, section 22, line 3

Ἰνδὸς ποταμός ἐστι τῆς Ἰνδίας. 



Joannes Stobaeus Anthologus, Anthologium 
Book 4, chapter 55, section 18, line 2

Τοῦ αὐτοῦ (fr. 143 M.). 
 Ἰνδοὶ συγκατακαίουσιν, ὅταν τελευτήσωσι, τῶν γυναι-
κῶν τὴν προσφιλεστάτην. 

\end{greek}



\section{Syriani, Sopatri Et Marcellini Scholia (?)}

?
\blockquote[From Wikipedia\footnote{\url{}}]{}
\begin{greek}

Syriani, Sopatri Et Marcellini Scholia Ad Hermogenis Status, Scholia ad Hermogenis librum περὶ στάσεων (2047: 001)
“Rhetores Graeci, vol. 4”, Ed. Walz, C.
Stuttgart: Cotta, 1833, Repr. 1968.
Volume 4, page 71, line 4

Εἶτα ἐπεὶ καὶ ἰατρῶν ἐστιν ἐπὶ μέρους ζητήματα· οἷον 
τί μὲν τοῖς ἐν Ἰνδίᾳ συντελεῖ πρὸς ὑγείαν, τί δὲ τοῖς 
ἐν Σκυθία, προσέθηκε τὸ ἐκ τῶν παρ' ἑκάστοις κειμέ-
νων· οὐ γὰρ ἐκ τούτων ἀμφισβητεῖ ὁ ἰατρὸς, ἀλλ' ὁ 
ῥήτωρ· εἰ δὲ πολλάκις καὶ ῥήτορες καθολικὰ μεταχει-
ρίζονται, ὡς ὁ Δημοσθένης· οἷον οὐκ ἔστιν ἐπιορκοῦντα 
καὶ ψευδόμενον δύναμιν βεβαίαν, κτήσασθαι· καὶ πά-
λιν πέρας γὰρ ἅπασιν ἀνθρώποις τοῦ βίου ὁ θάνατος· 
καὶ παρ' Αἰσχίνῃ· οὐδεὶς γὰρ πώποτε πλοῦτος πονηροῦ 
τρόπου περιεγένετο· εἰδέναι δὲ χρὴ, ὅτι αἱ καθόλου 
ἐργασίαι τοῦ ἐπὶ μέρους ἐγένοντο. 



Syriani, Sopatri Et Marcellini Scholia Ad Hermogenis Status, Scholia ad Hermogenis librum περὶ στάσεων 
Volume 4, page 747, line 25

                          καὶ ἐπὶ ζητημάτων δὲ φανερὸν 
τὸ τοιοῦτον· οἷον Ἀλεξάνδρου ἐν Ἰνδοῖς ὄντος συμβου-
λεύει Δημοσθένης ἀντιλαμβάνεσθαι τῶν πραγμάτων. 


\end{greek}



\section{Flavius Claudius Julianus Imperator}
\blockquote[From Wikipedia\footnote{\url{http://en.wikipedia.org/wiki/Julian_\%28emperor\%29}}]{Julian (Latin: Flavius Claudius Julianus Augustus, Greek: Φλάβιος Κλαύδιος Ἰουλιανός Αὔγουστος;[1] 331/332[2]  – 26 June 363), also known as Julian the Apostate, as well as Julian the Philosopher, was Roman Emperor from 361 to 363 and a noted philosopher and Greek writer.[3]}
\begin{greek}

Flavius Claudius Julianus Imperator Phil., Ἐγκώμιον εἰς τὸν αὐτοκράτορα Κωνστάντιον (2003: 001)
“L'empereur Julien. Oeuvres complètes, vol. 1.1”, Ed. Bidez, J.
Paris: Les Belles Lettres, 1932.
Section 28, line 33

       Ὁ δὲ μικρὰ μὲν ἐνόμιζεν εἶναι τὰ παρόντα καὶ 
πόνον οὐ πολὺν τῆς σῆς συνέσεως καὶ ῥώμης κρατῆσαι, 
τοὺς Ἰνδῶν δὲ ἐσκόπει πλούτους καὶ Περσῶν τὴν πολυ-
τέλειαν· [καὶ] τοσοῦτον αὐτῷ περιῆν ἀνοίας καὶ θράσους ἐκ 
μικροῦ παντελῶς περὶ τοὺς κατασκόπους πλεονεκτήματος, 
οὓς ἀφυλάκτους ὅλῃ τῇ στρατιᾷ λοχήσας ἔκτεινεν. 



Flavius Claudius Julianus Imperator Phil., Περὶ τῶν τοῦ αὐτοκράτορος πράξεων ἢ περὶ βασιλείας (2003: 003)
“L'empereur Julien. Oeuvres complètes, vol. 1.1”, Ed. Bidez, J.
Paris: Les Belles Lettres, 1932.
Section 1, line 43

τοῦ τῶν Ἑλλήνων βασιλέως εἶναι ἐθέλοντα, ὥστε ὁ μὲν 
ἠτίμαζε τοὺς ἀρίστους, σὺ δὲ οἶμαι καὶ τῶν φαύλων 
πολλοῖς τὴν συγγνώμην νέμεις, τὸν Πιττακὸν ἐπαινῶν τοῦ 
λόγου, ὃς τὴν συγγνώμην τῆς τιμωρίας προὐτίθει, αἰσχυ-
νοίμην <ἄν>, εἰ μὴ τοῦ Πηλέως φαινοίμην εὐγνωμονέστερος 
καὶ ἐπαινοίην εἰς δύναμιν τὰ προσόντα σοι, οὔτι φημὶ 
χρυσὸν καὶ ἁλουργῆ χλαῖναν, οὐδὲ μὰ Δία πέπλους παμποι-
κίλους, <γυναικῶν ἔργα Σιδωνίων>, οὐδὲ ἵππων Νισαίων 
κάλλη καὶ χρυσοκολλήτων ἁρμάτων ἀστράπτουσαν αἴγλην,   
οὐδὲ τὴν Ἰνδῶν λίθον εὐανθῆ καὶ χαρίεσσαν. 



Flavius Claudius Julianus Imperator Phil., Περὶ τῶν τοῦ αὐτοκράτορος πράξεων ἢ περὶ βασιλείας 
Section 11, line 11

             Ταύτην δὴ τὴν πόλιν στρατὸς ἀμήχανος πλήθει 
Παρθυαίων ξὺν Ἰνδοῖς περιέσχεν, ὁπηνίκα ἐπὶ τὸν τύραν-
νον βαδίζειν προὔκειτο· καὶ ὅπερ Ἡρακλεῖ φασιν ἐπὶ τὸ 
Λερναῖον ἰόντι θηρίον συνενεχθῆναι, τὸν θαλάττιον καρ-
κίνον, τοῦτο ἦν ὁ Παρθυαίων βασιλεὺς ἐκ τῆς ἠπείρου 
Τίγρητα διαβὰς καὶ ἐπιτειχίζων τὴν πόλιν χώμασιν· εἶτα 
εἰς ταῦτα δεχόμενος τὸν Μυγδόνιον, λίμνην ἀπέφηνε τὸ 
περὶ τῷ ἄστει χωρίον καὶ ὥσπερ νῆσον ἐν αὐτῇ συνεῖχε   
τὴν πόλιν, μικρὸν ὑπερεχουσῶν καὶ ὑπερφαινομένων τῶν 
ἐπάλξεων. 



Flavius Claudius Julianus Imperator Phil., Περὶ τῶν τοῦ αὐτοκράτορος πράξεων ἢ περὶ βασιλείας 
Section 11, line 40

                                                   Ταῦτα δὲ ἐξ 
Ἰνδῶν εἵπετο, καὶ ἔφερεν ἐκ σιδήρου πύργους τοξοτῶν 
πλήρεις. 



Flavius Claudius Julianus Imperator Phil., Περὶ τῶν τοῦ αὐτοκράτορος πράξεων ἢ περὶ βασιλείας 
Section 12, line 2

Ἐπειδὴ γὰρ οἱ Παρθυαῖοι κοσμηθέντες ὅπλοις αὐτοί 
τε καὶ ἵπποι ξὺν τοῖς Ἰνδικοῖς θηρίοις προσῆγον τῷ 
τείχει, λαμπροὶ ταῖς ἐλπίσιν ὡς αὐτίκα μάλα διαρπασό-
μενοι, καὶ ἐδέδοτό σφιν τοῦ πρόσω χωρεῖν τὸ σημεῖον,   
ὠθοῦντο ξύμπαντες, αὐτός τις ἐθέλων πρῶτος ἐσαλέσθαι 
τὸ τεῖχος καὶ οἴχεσθαι φέρων τὸ ἐπ' αὐτῷ κλέος, εἶναί τε 
οὐδὲν ἐτόπαζον δέος· οὐδὲ γὰρ ὑπομενεῖν σφῶν τὴν ὁρμὴν 
τοὺς ἔνδον ἠξίουν. 



Flavius Claudius Julianus Imperator Phil., Περὶ τῶν τοῦ αὐτοκράτορος πράξεων ἢ περὶ βασιλείας 
Section 13, line 38

Ἑλλήνων μὲν Αἴαντε καὶ οἱ Λαπίθαι καὶ Μενεσθεὺς τοῦ 
τείχους εἶξαν καὶ περιεῖδον τὰς πύλας συντριβομένας 
ὑφ' Ἕκτορος καὶ τῶν ἐπάλξεων ἐπιβεβηκότα τὸν Σαρπη-
δόνα· οἱ δὲ οὐδὲ διαρραγέντος αὐτομάτως τοῦ τείχους 
ἐνέδοσαν, ἀλλὰ ἐνίκων μαχόμενοι καὶ ἀπεκρούοντο Παρ-
θυαίους ξὺν Ἰνδοῖς ἐπιστρατεύσαντας· εἶτα ὁ μὲν ἐπιβὰς 
τῶν νεῶν ἀπὸ τῶν ἰκρίων ὥσπερ ἐρύματος πεζὸς διαγωνί-
ζεται, οἱ δὲ πρότερον ἀπὸ τῶν τειχῶν ἐναυμάχουν· τέλος 
δὲ οἱ μὲν τῶν ἐπάλξεων εἶξαν καὶ τῶν νεῶν, οἱ δὲ ἐνίκων 
ναυσί τε ἐπιόντας καὶ πεζῇ τοὺς πολεμίους. 



Flavius Claudius Julianus Imperator Phil., Περὶ τῶν τοῦ αὐτοκράτορος πράξεων ἢ περὶ βασιλείας 
Section 18, line 7

Καὶ εἰ βούλεσθε τὸ κεφάλαιον ἀθρόως ἑλεῖν τοῦ λόγου, 
ὑπομνήσθητε τῆς τοῦ Μακεδόνος ἐπὶ τοὺς Ἰνδοὺς πορείας, 
οἳ τὴν πέτραν ἐκείνην κατῴκουν, ἐφ' ἣν οὐδὲ τῶν ὀρνίθων 
ἦν τοῖς κουφοτάτοις ἀναπτῆναι, ὅπως ἑάλω, καὶ οὐδὲν 
πλέον ἀκούειν ἐπιθυμήσετε, πλὴν τοσοῦτον μόνον, ὅτι 
Ἀλέξανδρος μὲν ἀπέβαλε πολλοὺς Μακεδόνας ἐξελὼν τὴν 
πέτραν, ὁ δὲ ἡμέτερος ἄρχων καὶ στρατηγὸς οὐδὲ χιλίαρχον 
ἀποβαλὼν ἢ λοχαγόν τινα, ἀλλ' οὐδὲ ὁπλίτην τῶν ἐκ κατα-
λόγου, καθαρὰν καὶ ἄδακρυν περιεποιήσατο τὴν νίκην. 



Flavius Claudius Julianus Imperator Phil., Περὶ τῶν τοῦ αὐτοκράτορος πράξεων ἢ περὶ βασιλείας 
Section 24, line 11

                                                Ταύτην δὲ τῇ 
ψυχῇ φασιν ἐμφύεσθαι καὶ αὐτὴν ἀποφαίνειν εὐδαίμονα 
καὶ βασιλικὴν καὶ ναὶ μὰ Δία πολιτικὴν καὶ στρατηγικὴν 
καὶ μεγαλόφρονα καὶ πλουσίαν γε ἀληθῶς, οὐ τὸ Κολο-
φώνιον ἔχουσαν χρυσίον, 
  Οὐδ' ὅσα λάινος οὐδὸς ἀφήτορος ἐντὸς ἔεργε 
  Τὸ πρὶν ἐπ' εἰρήνης, 
ὅτε ἦν ὀρθὰ τὰ τῶν Ἑλλήνων πράγματα, οὐδὲ ἐσθῆτα 
πολυτελῆ καὶ ψήφους Ἰνδικὰς καὶ γῆς πλέθρων μυριάδας 
πάνυ πολλάς, ἀλλὰ ὃ πάντων ἅμα τούτων καὶ κρεῖττον καὶ 
θεοφιλέστερον, ὃ καὶ ἐν ναυαγίαις ἔνεστι διασώσασθαι καὶ 
ἐν ἀγορᾷ καὶ ἐν δήμῳ καὶ ἐν οἰκίᾳ καὶ ἐπ' ἐρημίαις ἐν 
λῃσταῖς μέσοις καὶ ἀπὸ τυράννων βιαίων. 



Flavius Claudius Julianus Imperator Phil., Ἐπὶ τῇ ἐξόδῳ τοῦ ἀγαθωτάτου Σαλουστίου παραμυθητικὸς εἰς ἑαυτόν (2003: 004)
“L'empereur Julien. Oeuvres complètes, vol. 1.1”, Ed. Bidez, J.
Paris: Les Belles Lettres, 1932.
Section 5, line 57

             Ταῦτα ἐννοοῦντες, τούτοις στρεφόμενοι τοῖς 
εἰδώλοις, τυχὸν οὐκ ὀνείρων νυκτερινῶν ἰνδάλμασι προσέ-
ξομεν, οὐδὲ κενὰ καὶ μάταια προσβαλεῖ τῷ νῷ φαντάσματα 
πονηρῶς ὑπὸ τῆς τοῦ σώματος κράσεως αἴσθησις διακει-
μένη. 



Flavius Claudius Julianus Imperator Phil., Πρὸς Ἡράκλειον κυνικὸν περὶ τοῦ πῶς κυνιστέον καὶ εἰ πρέπει τῷ κυνὶ μύθους πλάττειν (2003: 007)
“L'empereur Julien. Oeuvres complètes, vol. 2.1”, Ed. Rochefort, G.
Paris: Les Belles Lettres, 1963.
Section 2, line 6

             Εἰ δέ, ὥσπερ ἱππεῖς ἐν Θράκῃ καὶ Θετταλίᾳ, 
τοξόται δὲ καὶ τὰ κουφότερα τῶν ὅπλων ἐν Ἰνδίᾳ καὶ 
Κρήτῃ καὶ Καρίᾳ, τῇ φύσει τῆς χώρας ἀκολουθούντων 
οἶμαι τῶν ἐπιτηδευμάτων, οὕτω τις ὑπολαμβάνει καὶ ἐπὶ 
τῶν ἄλλων πραγμάτων, ἐν οἷς ἕκαστα τιμᾶται, μάλιστα 
παρὰ τούτων αὐτὰ καὶ πρῶτον εὑρῆσθαι. 



Flavius Claudius Julianus Imperator Phil., Πρὸς Ἡράκλειον κυνικὸν περὶ τοῦ πῶς κυνιστέον καὶ εἰ πρέπει τῷ κυνὶ μύθους πλάττειν 
Section 16, line 9

                             Ἐπεὶ δὲ ἐδέδοκτο τῷ Διὶ κοινῇ 
πᾶσιν ἀνθρώποις ἐνδοῦναι ἀρχὴν καταστάσεως ἑτέρας καὶ 
μεταβαλεῖν αὐτοὺς ἐκ τοῦ νομαδικοῦ βίου πρὸς τὸν ἡμε-
ρώτατον, ἐξ Ἰνδῶν ὁ Διόνυσος αὔτοπτος ἐφαίνετο δαίμων, 
ἐπιφοιτῶν τὰς πόλεις, ἄγων μεθ' ἑαυτοῦ στρατιὰν πολλὴν 
δαιμόνων τινὰ καὶ διδοὺς ἀνθρώποις κοινῇ μὲν ἅπασι 
σύμβολον τῆς ἐπιφανείας αὐτοῦ τὸ τῆς ἡμερίδος φυτόν, 
ὑφ' οὗ μοι δοκοῦσιν, ἐξημερωθέντων αὐτοῖς τῶν περὶ τὸν 
βίον, Ἕλληνες τῆς ἐπωνυμίας αὐτὸ ταύτης ἀξιῶσαι, 
μητέρα δὲ αὐτοῦ προσειπεῖν τὴν Σεμέλην διὰ τὴν πρόρ-
ρησιν, ἄλλως τε καὶ τοῦ θεοῦ τιμῶντος αὐτὴν ἅτε πρώτην 
ἱερόφαντιν τῆς ἔτι μελλούσης ἐπιφοιτήσεως. 



Flavius Claudius Julianus Imperator Phil., Συμπόσιον ἢ Κρόνια sive Caesares (2003: 010)
“L'empereur Julien. Oeuvres complètes, vol. 2.2”, Ed. Lacombrade, C.
Paris: Les Belles Lettres, 1964.
Section 25, line 29

                                                             Ἐγὼ 
δὲ ἐν οὐδὲ ὅλοις ἐνιαυτοῖς δέκα πρὸς τούτοις καὶ Ἰνδῶν 
γέγονα κύριος. 



Flavius Claudius Julianus Imperator Phil., Συμπόσιον ἢ Κρόνια sive Caesares 
Section 31, line 32

             Ἀλλ' ἡνίκα,» εἶπεν, «ἐν Ἰνδοῖς ἐτρώθης καὶ 
ὁ Πευκέστης ἔκειτο παρὰ σέ, σὺ δὲ ἐξήγου ψυχορραγῶν 
τῆς πόλεως, ἆρα ἥττων ἦσθα τοῦ τρώσαντος, ἢ καὶ ἐκεῖνον 
ἐνίκας; 

\end{greek}

\section{Asterius of Amasea}
\blockquote[From Wikipedia\footnote{\url{http://en.wikipedia.org/wiki/Asterius_of_Amasia}}]{Saint Asterius of Amasea (c. 350 – c. 410 AD)[1] was made Bishop of Amasea between 380 and 390 AD, after having been a lawyer.[1][2] He was born in Cappadocia and probably died in Amasea in modern Turkey, then in Pontus. Significant portions of his lively sermons survive, which are especially interesting from the point of view of art history, and social life in his day. Asterius, Bishop of Amasea is not to be confused with the Arian polemicist, Asterius the Sophist.[2]

Asterius of Amasea was the younger contemporary of Amphilochius of Iconium and the three great Cappadocian Fathers.[2] Little is known about his life, except that he was educated by a Scythian slave. Like Amphilochius, he had been a lawyer before becoming bishop between 380 and 390 AD, and he brought the skills of the professional rhetorician to his sermons.[3] Sixteen homilies and panegyrics on the martyrs still exist, showing familiarity with the classics, and containing an unusual concentration of details of everyday life in his time. One of them, Oration 4: Adversus Kalendarum Festum attacks the pagan customs and abuses of the New Years feast, denying everything that Libanius had said supporting it - see Lord of Misrule for extensive quotations. That sermon was preached on January 1, 400 AD, which provides the main evidence, with a reference in another to his great age, to the dating of his career.[4]

Texts

An English translation exists of five sermons by Asterius, which were published in 1904 in the US under the title "Ancient Sermons for Modern Times", and issued as a reprint in 2007. This is the main portion of his works to exist in English, and has been transcribed online.[2][12] Oration 11 has also been translated.[13]

Other sermons by Asterius of Amasea existed in the time of Photius, who referred to a further ten sermons not now known in Bibliotheca codex 271.[2] One of these lost sermons indicates that Asterius lived to a great age.[2]

Fourteen genuine sermons have been printed by Migne in the Patrologia Graeca 40, 155-480, with a Latin translation.[2] along with other sermons "by Asterius" that were written by Asterius the Sophist. Another two genuine sermons were discovered in manuscript at Mount Athos by M. Bauer. Those two were first printed by A. Bretz (TU 40.1, 1914). Eleven sermons have also been translated into German.[2]}

\begin{greek}

Asterius Scr. Eccl., Homiliae 1–14 (2060: 001)
“Asterius of Amasea. Homilies i–xiv”, Ed. Datema, C.
Leiden: Brill, 1970.
Homily 1, chapter 5, section 3, line 6

                                                                    Αὔξουσα γὰρ 
καθ' ἡμέραν ἐπὶ τὸ περιεργότερον ἡ τρυφὴ ἤδη καὶ τῶν ἐξ Ἰνδικῆς 
ἀρωμάτων παρεγχέει τοῖς ὄψοις, καὶ μᾶλλον τῶν ἰατρῶν οἱ μυροπῶλαι 
τοῖς μαγείροις ὑπηρετοῦσιν. 

\end{greek}

\section{Stephanus Med.}%???
7th century? Don't confuse with Stephanus Phil. (also 7th c).

\blockquote[From Wikipedia\footnote{\url{}}]{}
\begin{greek}

Stephanus Med., Phil., Collyrium ophthalmicum (olim sub auctore Stephano Archiatro) (0724: 003)
“Index lectionum in universitate litterarum Vratislaviensi per hiemem anni 1888–1889”, Ed. Studemund, W.
Breslau: Breslau University Press, 1889.
Page 13, line 7

                                                                               ἀκαίρως γὰρ οὐδὲν ὀνίνησιν· ἀλλὰ καὶ ὀδύνην   
μεγάλην ἐργάζεται τῶ διατείνειν τοὺς χιτῶνας· χρεία οὖν ἐν ἀπόροις εὑρεῖν· καὶ διὰ τοῦτο πρῶτον μὲν τοῦ ὅλου 
σώματος πρόνοιαν ποιήσασθαι χρὴ ἤτοι δια (sic) φλεβοτομίας· ἢ καθάρσεως, ἢ δι' ἀμφοτέρων εἰ δέοι· εἶτα τῆς κεφαλῆς· 
μετὰ δὲ ταύτης πρόνοιαν καὶ τὴν πρόσφορον δίαιταν χρήσασθαι τῶ τοιούτω φαρμάκω· (manus recens in margine 
adscripsit: Comp, id est Compositio) ἔστι δὲ ἡ σύνθεσις αὐτοῦ τοιάδε· πομφόλυγος πεπλυμένης 𐅻 <β>· ἀμύλου καλῶς 
πεπλυμένου προσφάτου καὶ ἀποίου 𐅻 <ε>· καδμίας κεκαυμένης (inc. fol. 323) καὶ πεπλυμένης 𐅻 <δ>· μολίβδου κεκαυ-
μένου καὶ πεπλυμένου 𐅻 <γ>· ψιμμιθίου πεπλυμένου 𐅻 <α>· λίθου αἱματίτου 𐅻 <ζ>· ὀπίου 𐅻 <α> 𐅶· ῥόδων χυλοῦ 𐅻 <γ>· λιβάνου 
𐅻 <α> 𐅶· κρόκου 𐅻 τὸ 𐅶· ἀλόης ἰνδικῆς· ὡσαύτως σμύρνης 𐅻 <α> 𐅶· σαρκοκόλλης πεπλυμένης 𐅻 <η>· σάχαρος πεφωγμένου 
𐅻 <γ>· τραγακάνθης 𐅻 ι<β>· ἀποβραχείσης ἐν χυλῶ τήλεως· ἔστι δὲ ὁ μὲν πεπλυμένος ποφόλυξ (sic)· ξηραίνων ἀδήκτως· 
εἴπέρ τι καὶ ἄλλως ἐστὶ· καὶ διὰ τοῦτο χρώμεθα αὐτῶ πρὸς τὰ λεπτὰ καὶ δριμέα ῥεύματα· ἔτι δὲ καὶ πρὸς ἕλκη 
πρὸς τούτοις ὁ πομφόλυξ ἔχει βραχὺ τὶ καὶ στυπτικὸν· ἡ δὲ ἀρίστη καδμία καυθεῖσα καὶ πλυθεῖσα ἀδηκτοτάτη 
γίνεται φάρμακον· ἔχει δέ τι βραχὺ καὶ ῥυπτικὸν ἐάν τε μετὰ τὴν καύσιν (sic)· ἐάν τε καὶ χωρὶς ταύτης πλυθῆ. 

\end{greek}



\section{Joannes Philoponus}
\blockquote[From Wikipedia\footnote{\url{http://en.wikipedia.org/wiki/Joannes_Philoponus}}]{
Jump to: navigation, search

John Philoponus (play /fɨˈlɒpənəs/; Ancient Greek: Ἰωάννης ὁ Φιλόπονος; 490 – 570) also known as John the Grammarian or John of Alexandria, was a Christian and Aristotelian commentator and the author of a considerable number of philosophical treatises and theological works. A rigorous, sometimes polemical writer and an original thinker who was controversial in his own time, John Philoponus broke from the Aristotelian-Neoplatonic tradition, questioning methodology and eventually leading to empiricism in the natural sciences.

He was posthumously condemned as a heretic by the Orthodox Church in 680-81 because of what was perceived of as a tritheistic interpretation of the Trinity.}
\begin{greek}

Joannes Philoponus Phil., In Aristotelis meteorologicorum librum primum commentarium (4015: 005)
“Ioannis Philoponi in Aristotelis meteorologicorum librum primum commentarium”, Ed. Hayduck, M.
Berlin: Reimer, 1901; Commentaria in Aristotelem Graeca 14.1.
Volume 14,1, page 17, line 33

λέγουσι δὲ καὶ Ἀλέξανδρον ἐξ Ἰνδῶν Ἀριστοτέλει γράψαι, ὡς ἕτεροί φασιν, 
ἐκ Βαβυλῶνος ὡς καὶ οἱ ἐνταῦθα σοφοὶ σώματος ἑτέρου φασὶν εἶναι τὸν 
οὐρανόν. 





Joannes Philoponus Phil., De opificio mundi (4015: 011)
“Joannis Philoponi de opificio mundi libri vii”, Ed. Reichardt, W.
Leipzig: Teubner, 1897.
Page 89, line 14

       – ἀλλ' ἴστω ὡς οὐχ ἡ αὐτὴ παρὰ πᾶσιν οὔτε 
ἡμέρα ἐστὶν ἀπαραλλάκτως, οὔτε νύξ· ἡ γὰρ παρ' ἡμῖν 
τρίτη φέρε τῆς ἡμέρας ὥρα παρ' Ἰνδοῖς μὲν ἕκτη τυ-
χὸν οὖσα τυγχάνει, τοῖς δὲ περὶ τὸν δυτικὸν ὠκεανὸν 
πρώτη φέρε, εἰ οὕτω τύχοι, καὶ ἄλλως παρ' ἄλλοις· 
οὔτε γὰρ παρὰ πᾶσιν αἱ αὐταί εἰσιν ἀνατολαὶ καὶ δύ-
σεις, ὡς ἐκ τῶν ἐκλείψεων ἡλίου τε καὶ σελήνης 
ὑπάρχει δῆλον, οὐ κατὰ τὴν αὐτὴν φαινομένων παρὰ 
πᾶσιν ὥραν. 



Joannes Philoponus Phil., De opificio mundi 
Page 126, line 7

μεθόδῳ ληπτῇ τοῖς ἐθέλουσι τὰ περὶ αὐτοὺς τοὺς 
φωστῆρας συμβαίνοντα προλέγοντες καὶ εἰς ἔργον ἐπ' 
ὄψεσι πάντων δεικνύντες ἐκβαίνοντα, τόν τε χρόνον   
καθ' ὃν ἐκλείπειν ἄρχονται, τόν τε μέσον καὶ τὸν 
ἔσχατον, ἐκ μέρους τε ποίου τῶν φωστήρων ἡ τοῦ 
φωτὸς αὐτῶν ἄρχεται κρύψις καὶ μέχρι τίνος πρόεισι, 
πόθεν τε ἀνακαθαίρεσθαι ἄρχονται καὶ ποῦ λήγουσι, 
καὶ διὰ τί μιᾶς οὔσης τῆς ἐκλείψεως μὴ τὸν αὐτὸν 
χρόνον ἐν ἑκάστῳ τόπῳ ἡ αὐτὴ γινομένη φαίνεται, 
ἀλλὰ παρ' Ἰνδοῖς μὲν ἐνάτην ὥραν τυχόν, ἐν Αἰγύπτῳ 
δὲ πέμπτην φέρε ἢ ἁπλῶς ἐλάττονα, τοῖς δὲ περὶ τὸν 
δυτικὸν ὠκεανὸν τρίτην ἢ δευτέραν, εἰ οὕτως ἡ μέ-
θοδος εὕροι, τοῖς δὲ ἐπὶ τὸ δυτικώτερον ἔτι πρώτην 
τυχὸν ἢ ὑπὸ γῆν ἔτι τοῦ φωστῆρος ὄντος, τῆς αἰτίας 
ἑκάστου τούτων τοῖς ἐθέλουσι μαθεῖν ληπτῆς οὔσης· 
οἱ οὖν ταῦτα καὶ τὰ τοιαῦτα δι' ἐπιστήμης ἐγνωκότες 
καὶ αὐταῖς ὄψεσι τοῖς φαινομένοις ἐπιστήσαντες, ὅταν 
ἐντύχωσιν τοῖς τοῦ καλοῦ Θεοδώρου ἤ τινος τῶν κατ' 




Joannes Philoponus Phil., De vocabulis quae diversum significatum exhibent secundum differentiam accentus (4015: 012)
“Iohannis Philoponi de vocabulis quae diversum significatum exhibent secundum differentiam accentus”, Ed. Daly, L.W.
Philadelphia: American Philosophical Society, 1983.
Recensio a, alphabetic letter iota, entry 4, line 1

<Ἱέρων>· τὸ κύριον παροξύνεται, 
<ἱερῶν>· ὁ ἀνατεθημένος, ἡ μετοχὴ περισπᾶται. 
<Ἴνδος>· ὁ ποταμὸς Ἰνδίας παροξύνεται, 
<Ἰνδός>· τὸ ἐθνικὸν ὀξύνεται. 



Joannes Philoponus Phil., De vocabulis quae diversum significatum exhibent secundum differentiam accentus 
Recensio a, alphabetic letter iota, entry 4, line 1

<Ἴνδος>· ὁ ποταμὸς Ἰνδίας παροξύνεται, 
<Ἰνδός>· τὸ ἐθνικὸν ὀξύνεται. 



Joannes Philoponus Phil., De vocabulis quae diversum significatum exhibent secundum differentiam accentus 
Recensio b, alphabetic letter iota, entry 9, line 1

<ἰωνιά>· ὁ τόπος τῶν ἴων ὀξύνεται, 
<Ἰωνία>· †ἡ κακία† ἢ ἡ χώρα παροξύνεται. 
<Ἴνδος>· ὁ ποταμὸς Ἰνδίας παροξύνεται, 
<Ἰνδός>· τὸ ἐθνικὸν ὀξύνεται. 



Joannes Philoponus Phil., De vocabulis quae diversum significatum exhibent secundum differentiam accentus 
Recensio b, alphabetic letter iota, entry 9, line 1

<Ἴνδος>· ὁ ποταμὸς Ἰνδίας παροξύνεται, 
<Ἰνδός>· τὸ ἐθνικὸν ὀξύνεται. 



Joannes Philoponus Phil., De vocabulis quae diversum significatum exhibent secundum differentiam accentus 
Recensio c, alphabetic letter iota, entry 6, line 1

<Ἴων>· τὸ κύριον, 
<ἰών>· ὁ πορευόμενος, 
<ἰῶν>· ὁ ἰοῦ ἀνάμεστος. 
<Ἴνδος>· ὁ ποταμὸς Ἰνδίας, 
<Ἰνδός>· τὸ ἐθνικόν. 



Joannes Philoponus Phil., De vocabulis quae diversum significatum exhibent secundum differentiam accentus 
Recensio c, alphabetic letter iota, entry 6, line 1

<Ἴνδος>· ὁ ποταμὸς Ἰνδίας, 
<Ἰνδός>· τὸ ἐθνικόν. 



Joannes Philoponus Phil., De vocabulis quae diversum significatum exhibent secundum differentiam accentus 
Recensio d, alphabetic letter iota, entry 7, line 1

<ἴδε>· τὸ ῥῆμα τὸ προστακτικὸν τὸ σημαῖνον τὸ θεάσασθαι 
<ἰδού>· τὸ ἐπίρρημα. 
<Ἴνδος>· ὁ ποταμὸς Ἰνδίας, 
<Ἰνδός>· τὸ ἐθνικόν. 



Joannes Philoponus Phil., De vocabulis quae diversum significatum exhibent secundum differentiam accentus 
Recensio d, alphabetic letter iota, entry 7, line 1

<Ἴνδος>· ὁ ποταμὸς Ἰνδίας, 
<Ἰνδός>· τὸ ἐθνικόν. 



Joannes Philoponus Phil., De vocabulis quae diversum significatum exhibent secundum differentiam accentus 
Recensio e, alphabetic letter iota, entry 3, line 1

<Ἱέρων>· κύριον, 
<ἱερῶν>· τῶν πραγμάτων. 
<Ἴνδος>· ποταμὸς Ἰνδίας, 
<Ἰνδός>· ἐθνικόν. 



Joannes Philoponus Phil., De vocabulis quae diversum significatum exhibent secundum differentiam accentus 
Recensio e, alphabetic letter iota, entry 3, line 1

<Ἴνδος>· ποταμὸς Ἰνδίας, 
<Ἰνδός>· ἐθνικόν. 

\end{greek}



\section{Anonymi Geographiae Expositio Compendiaria}%???
Date?
\blockquote[From Wikipedia\footnote{\url{}}]{}
\begin{greek}

Anonymi Geographiae Expositio Compendiaria, Geographiae expositio compendiaria (0092: 001)
“Geographi Graeci minores, vol. 2”, Ed. Müller, K.
Paris: Didot, 1861, Repr. 1965.
Section 1, line 5

                 Τοῦτο δέ ἐστι τὸ ἀπὸ Γάγγου ποτα-
μοῦ ἐκβολῆς, τοῦ ἐν Ἰνδοῖς ἑωθινωτάτου, ἐπὶ τὸ δυτι-
κώτατον τῆς ὅλης οἰκουμένης ἀκρωτήριον, ὃ καλεῖται 
μὲν Ἱερὸν, τῆς Λυσιτανίας δ' ἐστὶν ἄκρον· τόδ' ἐστὶ 
τῶν Ἡρακλέους στηλῶν δυτικώτερον σταδίοις που 
τρισχιλίοις. 



Anonymi Geographiae Expositio Compendiaria, Geographiae expositio compendiaria 
Section 19, line 10

                                   Ταύτης δὲ ἔχεται πρὸς 
ἀνατολὰς ἡ Σκυθία· αὕτη δὲ περὶ μὲν τὰς ἀρχὰς οὐ 
σφόδρα πλατύνεται, περὶ δὲ τὰς ἀνατολὰς καὶ πάνυ· 
ὀλίγου γὰρ δεῖν συνάπτει τῇ Ἰνδικῇ. 



Anonymi Geographiae Expositio Compendiaria, Geographiae expositio compendiaria 
Section 24, line 3

Τὴν δὲ λοιπὴν τὴν μέχρι τῶν Θινῶν ἤπειρον 
ἅπασαν, πλείστην οὖσαν καὶ ὑπὸ πολλῶν ἐθνῶν κατοι-
κουμένην, Ἰνδοὶ κατανέμονται, ἀφοριζομένην πρὸς 
μὲν ἀνατολαῖς Σίναις, πρὸς δὲ ταῖς δύσεσι Γεδρωσίᾳ, 
πρὸς δὲ ταῖς ἄρκτοις Παροπανισάδαις καὶ Ἀραχωσίᾳ, 
Σογδιανοῖς τε καὶ Σάκαις, Σκυθίᾳ τε καὶ τῇ Ση-
ρικῇ. 



Anonymi Geographiae Expositio Compendiaria, Geographiae expositio compendiaria 
Section 25, line 2

Ἔστι δὲ καὶ τῆς ἠπείρου ταύτης κατὰ μὲν τὸ 
Ἰνδικὸν πέλαγος μεγίστη νῆσος, ἡ πάλαι μὲν Σιμοῦνδα 
καλουμένη, νῦν δὲ Σαλικὴ, ἐν ᾗ φασι πάντα γίνεσθαι 
τὰ πρὸς τὴν χρῆσιν τὴν βιωτικὴν, ἔχειν τε παντοῖα 
μέταλλα, καὶ τοὺς κατοικοῦντας αὐτὴν ἄνδρας μαλ-
λοῖς γυναικείοις ἀναδεῖσθαι τὰς κεφαλάς· κατὰ δὲ τὴν 
καθ' ἡμᾶς θάλασσαν ἡ Κύπρος. 



Anonymi Geographiae Expositio Compendiaria, Geographiae expositio compendiaria 
Section 26, line 13

                                            Τῶν δὲ ἐν αὐ-  
ταῖς ἐθνῶν, τῆς μὲν Εὐρώπης μεγίστη ἐστὶν Ἱσπανία 
τε καὶ Ἰταλία, Γερμανία τε καὶ Σαρματία, τῶν δὲ 
ἐν τῇ Λιβύῃ ἥ τε Ἀφρικὴ καὶ ἡ Αἴγυπτος, καὶ τῶν 
Ἀσιανῶν παρὰ πάντα μὲν ἰδίως ἡ Ἰνδικὴ, μεγίστη δὲ 
καὶ Σκυθία Σηρική τε καὶ ἡ Εὐδαίμων. 



Anonymi Geographiae Expositio Compendiaria, Geographiae expositio compendiaria 
Section 28, line 5

Ἔστι δὲ καὶ τῶν ὀρῶν μέγιστα, ἐν μὲν τῇ 
Ἀσίᾳ ὅ τ' Ἰμάϊος καὶ τὰ Ἠμωδὰ καὶ τὰ Καυκάσια· 
ταῦτα δὲ καὶ τὰ Ῥίπαιά φασι παρὰ πάντα ὑψηλότατα 
εἶναι· μέγιστον δ' ὄρος καὶ ὁ Παροπάνισος καὶ ὁ 
Ταῦρος, τῶν τε κατὰ τὴν Ἰνδικὴν τὰ πλεῖστα· τῶν 
δὲ Λιβυκῶν ὑψηλότατα μὲν ὅ τε μέγας Ἄτλας καὶ τὸ 
τῶν Θεῶν ὄχημα, μεγάλα δὲ καὶ ἐπιμηκέστατα τὰ 
Αἰθιοπικά· ἀρξάμενα γὰρ ἀπὸ τῆς μεθορίας τῆς κατ' 
Αἴγυπτον κάτεισιν ἐπὶ μεσημβρίαν συνεκτεινόμενα τῇ 
τοῦ Νείλου πορείᾳ. 



Anonymi Geographiae Expositio Compendiaria, Geographiae expositio compendiaria 
Section 29, line 3

Τῶν δὲ ἐν τῇ οἰκουμένῃ ποταμῶν μέγιστοι μέν 
εἰσιν, ἐν μὲν τῇ Ἀσίᾳ πολλῶν ὄντων παρὰ πάντας ὅ 
τε Γάγγης καὶ ὁ Ἰνδός· οὗτοι γὰρ ἀρξάμενοι ἀπὸ τῶν 
βορειοτάτων τῆς οἰκουμένης, καὶ προσλαμβάνοντες 
σχεδὸν πάντας τοὺς ἀξιολόγους, ὅσοι διαρρέουσι τὴν 
ὅλην Ἰνδικὴν (εἰσὶ δὲ καὶ ὅσοι πλεῖστοι), ἐκδιδόασιν ἐπὶ 
τὴν πρὸς νότον θάλασσαν. 



Anonymi Geographiae Expositio Compendiaria, Geographiae expositio compendiaria 
Section 34, line 3

Ταύτης δὲ τῆς θαλάσσης ὑπέρκειται ἡ λοιπὴ 
ἡ παρὰ τὴν ἤπειρον· καὶ ἔστι μὲν αὐτῆς μέγιστον μὲν 
Ἰνδικὸν πέλαγος, ἐν ᾧ χερρόνησοι καὶ κόλποι πάνυ   
μεγάλοι, ὅ τε Θηριώδης καὶ ὁ Μέγας καὶ ὁ Γαγγη-
τικός. 



Anonymi Geographiae Expositio Compendiaria, Geographiae expositio compendiaria 
Section 35, line 1

Τοῦ δὲ Ἰνδικοῦ πελάγους ἔχεται τὸ Καρμάνιον, 
προϊὸν ὡς ἐπὶ δύσεις, τούτου δὲ ἡ Ἐρυθρὰ θάλασσα, 
ὧν περὶ τὰς συμβολὰς καὶ τὸ τοῦ Περσικοῦ κόλπου 
στόμα κεῖται. 



Anonymi Geographiae Expositio Compendiaria, Geographiae expositio compendiaria 
Section 43, line 1

Τῆς δὲ κατὰ τὸ Ἰνδικὸν πέλαγος Βραχείας θα-
λάσσης, ἐπείπερ αὕτη παρὰ τὰς ἄλλας ἐπὶ πλεῖστον 
πρὸς ἀνατολὰς καὶ δύσεις ἐκτείνεται, τὸ ἀπὸ Ἐσιναῦ 
ἐμπορίου τῆς Βαρβαρίας ἢ τῶν Ῥαπτῶν τῆς μητρο-
πόλεως ἐπὶ Κοττίαριν ποταμὸν τῶν Σινῶν σταδίων 
μυριάδες εʹ καὶ ͵βφʹ, μίλια δὲ ͵ζ· πλάτος δὲ τὸ ἀπὸ 
τοῦ μυχοῦ τοῦ Μεγάλου κόλπου ἐπὶ τὴν ἄγνωστον στά-
δια [μυρία] τρισχίλια, ἤτοι μίλια ͵αψ[λ]γʹ. 



Anonymi Geographiae Expositio Compendiaria, Geographiae expositio compendiaria 
Section 45, line 14

Τῶν δὲ λοιπῶν δύο κλιμάτων τὸ μὲν κατ' ἀνατολὰς 
Ἑῷον πέλαγος καὶ Ἰνδικὸς ὠκεανὸς, τὸ δὲ κατὰ τὴν 
δύσιν, ἀφ' οὗπερ ἡ καθ' ἡμᾶς θάλασσα πληροῦται, 
Ἑσπέριος ὠκεανὸς, καὶ κατ' ἐξοχὴν Ἀτλαντικὸν προς-
αγορεύεται πέλαγος. 

\end{greek}


\section{Philostorgius}
\blockquote[From Wikipedia\footnote{\url{http://en.wikipedia.org/wiki/Philostorgius}}]{

Philostorgius (Greek: Φιλοστόργιος; 368 – ca. 439) was an Anomoean Church historian of the 4th and 5th centuries. Anomoeanism questioned the Trinitarian account of the relationship between God the Father and Christ and was considered a heresy by the Orthodox Church, which adopted the term "homoousia" in the Nicene Creed. Very little information about his life is available. He was born in Borissus, Cappadocia to Eulampia and Carterius,[1] and later lived in Constantinople.

He wrote a history of the Arian controversy titled History of the Church, of which only an epitome by Photius survives, as well as a treatise against Porphyry, which is lost.[2]}
\begin{greek}

Philostorgius Scr. Eccl., Historia ecclesiastica (fragmenta ap. Photium) (2058: 001)
“Philostorgius. Kirchengeschichte, 3rd edn.”, Ed. Winkelmann, F. (post J. Bidez)
Berlin: Akademie–Verlag, 1981; Die griechischen christlichen Schriftsteller.
Book 2, fragment 6, line 1

Ὅτι τοὺς ἐνδοτάτω Ἰνδούς, ὅσοι Χριστὸν ἔμαθον τιμᾶν ἐκ τῆς 
Βαρθολομαίου τοῦ ἀποστόλου διδασκαλίας, τὸ ἑτεροούσιον πρεσβεύειν 
ὁ δυσσεβής φησι. 



Philostorgius Scr. Eccl., Historia ecclesiastica (fragmenta ap. Photium) 
Book 2, fragment 6, line 3

                   καὶ τὸν Θεόφιλον εἰσάγει τὸν Ἰνδὸν τὸ τοιοῦτον 
ἀσπαζόμενον φρόνημα, παραγενέσθαι τε εἰς αὐτοὺς καὶ τὴν αὐτῶν 
ἐκδιηγεῖσθαι δόξαν. 



Philostorgius Scr. Eccl., Historia ecclesiastica (fragmenta ap. Photium) 
Book 2, fragment 6, line 5

                       τὸ δὲ τῶν Ἰνδῶν ἔθνος τοῦτο Σάβας μὲν 
πάλαι ἀπὸ τῆς Σαβᾶ μητροπόλεως, τὰ νῦν δὲ Ὁμηρίτας καλεῖσθαι. 



Philostorgius Scr. Eccl., Historia ecclesiastica (fragmenta ap. Photium) 
Book 3, fragment 4, line 18

Ταύτης τῆς πρεσβείας ἐν τοῖς πρώτοις ἦν καὶ Θεόφιλος ὁ Ἰνδός. 



Philostorgius Scr. Eccl., Historia ecclesiastica (fragmenta ap. Photium) 
Book 3, fragment 4, line 21


ὃς πάλαι μέν, Κωνσταντίνου τοῦ πάλαι βασιλεύοντος, ἔτι τὴν ἡλικίαν 
νεώτατος, καθ' ὁμηρίαν πρὸς τῶν Διβηνῶν καλουμένων εἰς Ῥωμαίους 
ἐστάλη· Διβοῦς δ' ἐστὶν αὐτοῖς ἡ νῆσος χώρα, τῶν Ἰνδῶν δὲ καὶ 
οὗτοι φέρουσι τὸ ἐπώνυμον. 



Philostorgius Scr. Eccl., Historia ecclesiastica (fragmenta ap. Photium) 
Book 3, fragment 5, line 5

                                               κἀκεῖθεν εἰς τὴν ἄλλην 
ἀφίκετο Ἰνδικήν, καὶ πολλὰ τῶν παρ' αὐτοῖς οὐκ εὐαγῶς δρωμένων 
ἐπηνωρθώσατο. 



Philostorgius Scr. Eccl., Historia ecclesiastica (fragmenta ap. Photium) 
Book 3, fragment 10, line 25

                                                                       οὗτος, 
ὡς ἔστι συμβαλεῖν, ἐξορμῶν τοῦ Παραδείσου, πρὶν ἐπὶ τὸ οἰκούμενον 
φθάσαι καταδυόμενος, ἔπειτα τὴν Ἰνδικὴν θάλατταν ὑπελθὼν ἔτι 
καὶ κύκλῳ γε αὐτὴν περιελιχθείς, ὡς εἰκάσαι (τίς γὰρ ἀνθρώπων 
ἀκριβώσειε τοῦτο; 



Philostorgius Scr. Eccl., Historia ecclesiastica (fragmenta ap. Photium) 
Book 3, fragment 11, line 32

                                     ὃν καὶ ὁ τῶν Ἰνδῶν βασιλεὺς 
Κωνσταντίῳ ἀπεστάλκει. 



Philostorgius Scr. Eccl., Historia ecclesiastica (fragmenta ap. Photium) 
Book 3, fragment 15, line 72

                                                            ταῦτα δὲ κατ' 
ἐκείνους τοὺς καιροὺς Κωνσταντίου ἦν καθ' οὓς καὶ ὁ Θεόφιλος ἐκ 
τῶν Ἰνδῶν ἐπανελθὼν διῆγεν ἐν Ἀντιοχείᾳ. 



Philostorgius Scr. Eccl., Historia ecclesiastica (fragmenta ap. Photium) 
Book 4, fragment 1, line 9

          συναπῄει δ' αὐτῷ καὶ Θεόφιλος ὁ Ἰνδός. 



Philostorgius Scr. Eccl., Historia ecclesiastica (fragmenta ap. Photium) 
Book 8, fragment 2, line 18

                                        πρὸς δὲ τὴν ἐν τῇ κοίλῃ Συρίᾳ Ἀντι-
όχειαν μετ' οὐ πολὺν χρόνον ἐθελοντὴς ἀφικνεῖται Θεόφιλος ὁ Ἰνδός, 
ἐφ' ᾧ τὸν Εὐζώϊον μὲν κατὰ τὸ προηγούμενον ἀναστῆσαι εἰς τέλος 
ἀγαγεῖν τὰ ὑπὲρ Ἀετίου ἐγνωσμένα· εἰ δὲ μή, αὐτός γε καθηγήσεσθαι 
τοῦ ἐκεῖσε πλήθους ὅσον τὴν ἐκείνου γνώμην ἠσπάζετο. 



Philostorgius Scr. Eccl., Historia ecclesiastica (fragmenta ap. Photium) 
Book 9, fragment 1, line 4

ΕΚ ΤΗΣ ΕΝΝΑΤΗΣ ΙΣΤΟΡΙΑΣ


 Ὅτι τῷ Φιλοστοργίῳ ὁ ἔννατος λόγος Ἀετίου χειρῶν ὑπερφυῆ 
ἔργα Εὐνομίου τε καὶ Λεοντίου διαπλάττει· καὶ δὴ καὶ Κανδίδου καὶ 
Εὐαγρίου καὶ Ἀρριανοῦ καὶ Φλωρεντίου καὶ μάλιστά γε Θεοφίλου 
τοῦ Ἰνδοῦ, καί τινων ἄλλων οὓς ἡ αὐτὴ τῆς ἀσεβείας λύσσα θερμο-
τέρους ἐπεδείκνυ. 



Philostorgius Scr. Eccl., Historia ecclesiastica (fragmenta ap. Photium) 
Book 9, fragment 18, line 7

                             τοῦ δὲ θᾶττον τελειωθέντος, Ἰωάννην 
ἀντικαθιστῶσιν· καὶ σὺν αὐτῷ ἀπὸ Κωνσταντινουπόλεως αὐτός τε 
Εὐνόμιος καὶ Ἀρριανὸς καὶ Εὐφρόνιος ἐπὶ τὴν Ἑῴαν ἀφικνοῦνται, 
ὡς ἐκεῖσε τόν τε Ἰουλιανὸν ἐκ τῆς Κιλικίας ἄξοντες καὶ Θεόφιλον 
τὸν Ἰνδὸν ἐν τῇ Ἀντιοχείᾳ καταληψόμενοι καὶ τὰ τῆς ἄλλης Ἑῴας 
καταστησόμενοι. 

\end{greek}



\section{John of Damascus}
\blockquote[From Wikipedia\footnote{\url{http://en.wikipedia.org/wiki/John_of_Damascus}}]{Saint John of Damascus (Greek: Ἰωάννης ὁ Δαμασκηνός Iōannēs ho Damaskēnos; Latin: Iohannes Damascenus; also known as John Damascene, Χρυσορρόας/Chrysorrhoas, "streaming with gold"—i.e., "the golden speaker") (c. 645 or 676 – 4 December 749; Arabic: يوحنا الدمشقي Yuḥannā Al Demashqi) was a Syrian monk and priest. Born and raised in Damascus, he died at his monastery, Mar Saba, near Jerusalem.[1]}
\begin{greek}

Joannes Damascenus Scr. Eccl., Theol., Expositio fidei (2934: 004)
“Die Schriften des Johannes von Damaskos, vol. 2”, Ed. Kotter, B.
Berlin: De Gruyter, 1973; Patristische Texte und Studien 12.
Section 23, line 28

                                        Ἐντεῦθεν αἱ δύο θάλασσαι αἱ τὴν 
Αἴγυπτον περιέχουσαι (μέση γὰρ αὕτη τῶν δύο κεῖται θαλασσῶν) 
συνέστησαν, διάφορα πελάγη καὶ ὄρη καὶ νήσους καὶ ἀγκῶνας καὶ λι-
μένας ἔχουσαι καὶ κόλπους διαφόρους περιέχουσαι αἰγιαλούς τε καὶ 
ἀκτάς – αἰγιαλὸς μὲν γὰρ ὁ ψαμμώδης λέγεται, ἀκτὴ δὲ ἡ πετρώδης καὶ 
ἀγχιβαθὴς ἤτοι ἡ εὐθέως ἐν τῇ ἀρχῇ βάθος ἔχουσα – , ὁμοίως καὶ ἡ 
κατὰ τὴν ἀνατολήν, ἥτις λέγεται Ἰνδική, καὶ ἡ βορεινή, ἥτις λέγεται 
Κασπία· καὶ αἱ λίμναι δὲ ἐντεῦθεν συνήχθησαν. 



Joannes Damascenus Scr. Eccl., Theol., Expositio fidei 
Section 23, line 38

  Ὄνομα τῷ ἑνὶ Φεισών», τουτέστι Γάγγης ὁ Ἰνδικός. 



Joannes Damascenus Scr. Eccl., Theol., Expositio fidei 
Section 24b, line 24

Ἀσίας ἠπείρου μεγάλης ἐπαρχίαι μηʹ, κανόνες ιβʹ· αʹ Βιθυνία Πόντου βʹ Ἀσία ἡ 
ἰδίως, πρὸς τῇ Ἐφέσῳ γʹ Φρυγία μεγάλη δʹ Λυκία εʹ Γαλατία ϛʹ Παφλαγονία ζʹ 
Παμφυλία ηʹ Καππαδοκία θʹ Ἀρμενία μικρά ιʹ Κιλικία ιαʹ Σαρματία ἡ ἐντὸς Ἀσίας ιβʹ 
Κολχίς ιγʹ Ἰβηρία ιδʹ Ἀλβανία ιεʹ Ἀρμενία μεγάλη ιϛʹ Κύπρος νῆσος ιζʹ Συρία 
κοίλη ιηʹ Συρία Φοινίκη ιθʹ Συρία Παλαιστίνη κʹ Ἀραβία Πετραία καʹ Μεσοποταμία 
κβʹ Ἀραβία ἔρημος κγʹ Βαβυλωνία κδʹ Ἀσσυρία κεʹ Σουσιανή κϛʹ Μηδία κζʹ 
Περσίς κηʹ Παρθία κθʹ Καρμανία ἔρημος λʹ Καρμανία ἑτέρα λαʹ Ἀραβία εὐδαίμων λβʹ 
Ὑρκανία λγʹ Μαργιανή λδʹ Βακτριανή λεʹ Σογδιανή λϛʹ Σακῶν λζʹ Σκυθία ἡ ἐντὸς 
Ἰμάου ὄρους ληʹ Σκυθία ἡ ἐκτὸς Ἰμάου ὄρους λθʹ Σηρική μʹ Ἀρεία μαʹ Παροπανισάδαι 
μβʹ Δραγγιανή μγʹ Ἀραχωσία μδʹ Γεδρωσία μεʹ Ἰνδικὴ ἡ ἐντὸς Γάγγου τοῦ 
ποταμοῦ μʹ2ʹ Ἰνδικὴ ἡ ἐκτὸς Γάγγου τοῦ ποταμοῦ μζʹ Σῖναι μηʹ Ταπροβάνη νῆσος. 



Joannes Damascenus Scr. Eccl., Theol., Expositio fidei 
Section 24b, line 25

ἰδίως, πρὸς τῇ Ἐφέσῳ γʹ Φρυγία μεγάλη δʹ Λυκία εʹ Γαλατία ϛʹ Παφλαγονία ζʹ 
Παμφυλία ηʹ Καππαδοκία θʹ Ἀρμενία μικρά ιʹ Κιλικία ιαʹ Σαρματία ἡ ἐντὸς Ἀσίας ιβʹ 
Κολχίς ιγʹ Ἰβηρία ιδʹ Ἀλβανία ιεʹ Ἀρμενία μεγάλη ιϛʹ Κύπρος νῆσος ιζʹ Συρία 
κοίλη ιηʹ Συρία Φοινίκη ιθʹ Συρία Παλαιστίνη κʹ Ἀραβία Πετραία καʹ Μεσοποταμία 
κβʹ Ἀραβία ἔρημος κγʹ Βαβυλωνία κδʹ Ἀσσυρία κεʹ Σουσιανή κϛʹ Μηδία κζʹ 
Περσίς κηʹ Παρθία κθʹ Καρμανία ἔρημος λʹ Καρμανία ἑτέρα λαʹ Ἀραβία εὐδαίμων λβʹ 
Ὑρκανία λγʹ Μαργιανή λδʹ Βακτριανή λεʹ Σογδιανή λϛʹ Σακῶν λζʹ Σκυθία ἡ ἐντὸς 
Ἰμάου ὄρους ληʹ Σκυθία ἡ ἐκτὸς Ἰμάου ὄρους λθʹ Σηρική μʹ Ἀρεία μαʹ Παροπανισάδαι 
μβʹ Δραγγιανή μγʹ Ἀραχωσία μδʹ Γεδρωσία μεʹ Ἰνδικὴ ἡ ἐντὸς Γάγγου τοῦ 
ποταμοῦ μʹ2ʹ Ἰνδικὴ ἡ ἐκτὸς Γάγγου τοῦ ποταμοῦ μζʹ Σῖναι μηʹ Ταπροβάνη νῆσος. 



Joannes Damascenus Scr. Eccl., Theol., Expositio fidei 
Section 24b, line 31

Ἔθνη δὲ οἰκεῖ τὰ πέρατα· κατ' ἀπηλιώτην Βακτριανοί, κατ' εὗρον Ἰνδοί, κατὰ 
Φοίνικα Ἐρυθρὰ θάλασσα καὶ Αἰθιοπία, κατὰ λευκόνοτον οἱ ὑπὲρ Σύρτιν Γεράμαντες, 
κατὰ λίβα Αἰθίοπες καὶ δυσμικοὶ Ὑπέρμαυροι, κατὰ ζέφυρον Στῆλαι καὶ ἀρχαὶ Λιβύης 
καὶ Εὐρώπης, κατὰ ἀργέστην Ἰβηρία ἡ νῦν Ἱσπανία, κατὰ δὲ θρασκίαν Κελτοὶ καὶ τὰ 
ὅμορα, κατὰ ἀπαρκτίαν οἱ ὑπὲρ Θρᾴκην Σκύθαι, κατὰ βορρᾶν Πόντος Μαιῶτις καὶ 
Σαρμάται, κατὰ καικίαν Κασπία θάλασσα καὶ Σάκες. 



Joannes Damascenus Scr. Eccl., Theol., De sacris jejuniis (2934: 021); MPG 95.
Volume 95, page 73, line 33

Τὸ μέντοι ἅγιον Πάσχα τῆς ἑνδεκάτης Ἰνδικτίωνος 
σὺν Θεῷ ἐπιτελοῦμεν· κατὰ μὲν Αἰγυπτίους, μηνὸς 
Φαρμουθὶ εἰκάδι πέμπτῃ· κατὰ δὲ Ῥωμαίους, μηνὸς 
Ἀπριλλίου εἰκάδι, πρὸ δεκαδύο καλανδῶν Μαΐων· 
ἀρχόμενοι τῆς νηστείας τῶν ἑπτὰ ἑβδομάδων ἐξ αὐ-
τῆς δευτέρας ἡμέρας, κατὰ μὲν Αἰγυπτίους, ὀγδόῃ 
τοῦ Φανεμὼθ μηνός· κατὰ δὲ Ῥωμαΐους Μαρτίου 
τρίτῃ. 



Joannes Damascenus Scr. Eccl., Theol., Epistula ad Theophilum imperatorem de sanctis et venerandis imaginibus [Sp.] (2934: 050); MPG 95.
Volume 95, page 376, line 50

                                       Καὶ μὲν δὴ τούτων 
πλείονα καὶ θρήνων ἄξια, «δι' ἃ ἦλθεν ἡ ὀργὴ τοῦ 
Θεοῦ ἐπὶ τὸν λαὸν τῆς ἀπειθείας,» ὡς ἀνωτέρω 
δεδήλωται, καὶ πᾶσιν ἡμῖν πρόδηλα γεγόνασι· λιμοὶ, 
λοιμοὶ, σεισμοὶ, καταποντισμοὶ, θάνατοι ἐξαίσιοι, 
πόλεμοι ἐμφύλιοι, ἐθνῶν ἐπαναστάσεις, ἐμπρησμοὶ 
ἐκκλησιῶν, ἐρημώσεις χωρῶν καὶ πόλεων, αἰχμ-
αλωσίαι λαῶν ὡσεὶ πρόβατα εἰς σφαγὴν πορευόμενα, 
μέχρις Αἰθιόπων, καὶ Ἰνδῶν, καὶ εἰς Ἀνατολὰς 
γῆς δοῦλοι καὶ αἰχμάλωτοι, νεάνιδες καὶ παρθένοι, 
πρεσβύτεροι μετὰ τῶν νεωτέρων· καὶ πᾶσα ἡλικία 
ἄρδην συντετέλεσται. 



Joannes Damascenus Scr. Eccl., Theol., Commentarii in epistulas Pauli [Dub.] (2934: 053); MPG 95.
Volume 95, page 681, line 31

Τουτέστι, τοσαῦται γλῶσσαι, τοσαῦται φωναὶ, 
Σκυθῶν, Θρᾳκῶν, Ῥωμαίων, Περσῶν, Μαύρων, 
Ἰνδῶν, Αἰγυπτίων, ἑτέρων μυρίων ἐθνῶν. 



Joannes Damascenus Scr. Eccl., Theol., Sermo in annuntiationem Mariae [Sp.] (2934: 057); MPG 96.
Volume 96, page 657, line 46

                                                    Χαῖρε, 
ὅτι πολλοὶ τῶν φιλοχρίστων βασιλίδων λιθολαμπεῖς 
διὰ σὲ στεφάνους, καὶ χρυσονήμους ἁλουργίδας, 
ἀράχνης εὐτελέστερα ἐλογίσαντο, Χαῖρε, ὅτι πολλοὶ 
τῶν εὐγενῶν Ἰνδικοὺς διὰ σὲ καὶ χρυσταλλίζοντας 
λίθους περιεφρόνησαν. 



Joannes Damascenus Scr. Eccl., Theol., Vita Barlaam et Joasaph [Sp.] (2934: 066)
“[St. John Damascene]. Barlaam and Joasaph”, Ed. Woodward, G.R., Mattingly, H.
Cambridge, Mass.: Harvard University Press, 1914, Repr. 1983.
Page 2, line t3

ΒΑΡΛΑΑΜ ΚΑΙ ΙΩΑΣΑΦ 
ΙΣΤΟΡΙΑ ΨΥΧΩΦΕΛΗΣ ΕΚ ΤΗΣ ΕΝΔΟΤΕΡΑΣ ΤΩΝ ΑΙΘΙΟΠΩΝ 
ΧΩΡΑΣ, ΤΗΣ ΙΝΔ*ωΝ ΛΕΓΟΜΕΝΗΣ, ΠΡΟΣ ΤΗΝ ΑΓΙΑΝ ΠΟΛΙΝ 
ΜΕΤΕΝΕΧΘΕΙΣΑ ΔΙΑ ΙΩΑΝΝΟΥ ΜΟΝΑΧΟΥ, ΑΝΔΡΟΣ ΤΙΜΙΟΥ 
ΚΑΙ ΕΝΑΡΕΤΟΥ ΜΟΝΗΣ ΤΟΥ ΑΓΙΟΥ ΣΑΒΑ· ΕΝ ΗΙ Ο ΒΙΟΣ 
ΒΑΡΛΑΑΜ ΚΑΙ ΙΩΑΣΑΦ ΤΩΝ ΑΟΙΔΙΜΩΝ ΚΑΙ ΜΑΚΑΡΙΩΝ. 




Joannes Damascenus Scr. Eccl., Theol., Vita Barlaam et Joasaph [Sp.] 
Page 4, line 27

                    τούτῳ οὖν ἐγὼ στοιχῶν τῷ 
κανόνι, ἄλλως δὲ καὶ τὸν ἐπηρτημένον τῷ δούλῳ 
κίνδυνον ὑφορώμενος, ὅς, λαβὼν παρὰ τοῦ δεσπό-
του τὸ τάλαντον, εἰς γῆν ἐκεῖνο κατώρυξε καὶ τὸ 
δοθὲν πρὸς ἐργασίαν ἔκρυψεν ἀπραγμάτευτον, 
ἐξήγησιν ψυχωφελῆ ἕως ἐμοῦ καταντήσασαν οὐ-
δαμῶς σιωπήσομαι· ἥνπερ μοι ἀφηγήσαντο ἄνδρες 
εὐλαβεῖς τῆς ἐνδοτέρας τῶν Αἰθιόπων χώρας, 
οὕστινας Ἰνδοὺς οἶδεν ὁ λόγος καλεῖν, ἐξ ὑπομνη-
μάτων ταύτην ἀψευδῶν μεταφράσαντες, ἔχει δὲ 
οὕτως. 



Joannes Damascenus Scr. Eccl., Theol., Vita Barlaam et Joasaph [Sp.] 
Page 6, line 1

I 
 Ἡ τῶν Ἰνδῶν λεγομένη χώρα πόρρω μὲν διά-
κειται τῆς Αἰγύπτου, μεγάλη οὖσα καὶ πολυ-
άνθρωπος· περικλύζεται δὲ θαλάσσαις καὶ ναυσι-
πόροις πελάγεσι τῷ κατ' Αἴγυπτον μέρει· ἐκ δὲ 
τῆς ἠπείρου προσεγγίζει τοῖς ὁρίοις Περσίδος, 
ἥτις πάλαι μὲν τῷ τῆς εἰδωλομανίας ἐμελαίνετο 
ζόφῳ, εἰς ἄκρον ἐκβεβαρβαρωμένη καὶ ταῖς ἀθέ-
σμοις ἐκδεδιῃτημένη τῶν πράξεων. 



Joannes Damascenus Scr. Eccl., Theol., Vita Barlaam et Joasaph [Sp.] 
Page 8, line 7

ἔθνη φωτίσαι τοὺς ἐν σκότει τῆς ἀγνοίας καθη-
μένους, καὶ βαπτίζειν αὐτοὺς εἰς τὸ ὄνομα τοῦ 
Πατρὸς καὶ τοῦ Υἱοῦ καὶ τοῦ Ἁγίου Πνεύματος,   
ὡς ἐντεῦθεν τοὺς μὲν αὐτῶν τὰς ἑῴας λήξεις, τοὺς 
δὲ τὰς ἑσπερίους λαχόντας περιέρχεσθαι, βόρειά 
τε καὶ νότια διαθέειν κλίματα, τὸ προστεταγμένον 
αὐτοῖς πληροῦντας, διάγγελμα τότε καὶ ὁ ἱερώ-
τατος Θωμᾶς, εἷς ὑπάρχων τῆς δωδεκαρίθμου 
φάλαγγος τῶν μαθητῶν τοῦ Χριστοῦ, πρὸς τὴν 
τῶν Ἰνδῶν ἐξεπέμπετο, κηρύττων αὐτοῖς τὸ σω-
τήριον κήρυγμα. 



Joannes Damascenus Scr. Eccl., Theol., Vita Barlaam et Joasaph [Sp.] 
Page 8, line 22

Ἐπεὶ δὲ καὶ ἐν Αἰγύπτῳ ἤρξατο μοναστήρια 
συνίστασθαι καὶ τὰ τῶν μοναχῶν ἀθροίζεσθαι 
πλήθη, καὶ τῆς ἐκείνων ἀρετῆς καὶ ἀγγελομιμήτου 
διαγωγῆς ἡ φήμη τὰ πέρατα διελάμβανε τῆς 
οἰκουμένης, καὶ εἰς Ἰνδοὺς ἧκε, πρὸς τὸν ὅμοιον 
ζῆλον καὶ τούτους διήγειρεν, ὡς πολλοὺς αὐτῶν, 
πάντα καταλιπόντας, καταλαβεῖν τὰς ἐρήμους 
καὶ ἐν σώματι θνητῷ τὴν πολιτείαν ἀνειληφέναι 
τῶν ἀσωμάτων. 



Joannes Damascenus Scr. Eccl., Theol., Vita Barlaam et Joasaph [Sp.] 
Page 14, line 4

II 
 Τῆς τοιαύτης οὖν σκοτομήνης τὴν τῶν Ἰνδῶν 
καταλαβούσης, καὶ τῶν μὲν πιστῶν πάντοθεν 
ἐλαυνομένων, τῶν δὲ τῆς ἀσεβείας ὑπασπιστῶν 
κρατυνομένων, αἵμασί τε καὶ κνίσαις τῶν θυσιῶν 
καὶ αὐτοῦ δὴ τοῦ ἀέρος μολυνομένου, εἷς τῶν τοῦ 
βασιλέως, ἀρχισατράπης τὴν ἀξίαν, ψυχῆς παρα-
στήματι, μεγέθει τε καὶ κάλλει, καὶ πᾶσιν ἄλλοις, 
οἷς ὥρα σώματος καὶ γενναιότης ψυχῆς ἀνδρείας 
χαρακτηρίζεσθαι πέφυκε, τῶν ἄλλων ἐτύγχανε 
διαφέρων. 



Joannes Damascenus Scr. Eccl., Theol., Vita Barlaam et Joasaph [Sp.] 
Page 62, line 13

                 καί, ἀμείψας τὸ ἑαυτοῦ σχῆμα, 
ἱμάτιά τε κοσμικὰ ἀμφιασάμενος, καὶ νηὸς ἐπιβάς, 
ἀφίκετο εἰς τὰ τῶν Ἰνδῶν βασίλεια, καὶ ἐμπόρου 
ὑποδὺς προσωπεῖον, τὴν πόλιν καταλαμβάνει, 
ἔνθα δὴ ὁ τοῦ βασιλέως υἱὸς τὸ παλάτιον εἶχε. 



Joannes Damascenus Scr. Eccl., Theol., Vita Barlaam et Joasaph [Sp.] 
Page 388, line 4

                             ὡσαύτως δὲ καὶ τοὺς 
μύστας καὶ νεωκόρους τῶν εἰδώλων καὶ σοφοὺς 
τῶν Χαλδαίων καὶ Ἰνδῶν, τοὺς κατὰ πᾶσαν τὴν 
ὑπ' αὐτὸν ἀρχὴν ὄντας, συνεκαλέσατο, καί τινας 
οἰωνοσκόπους καὶ γόητας καὶ μάντεις, ὅπως ἂν 
Χριστιανῶν περιγένοιντο. 



Joannes Damascenus Scr. Eccl., Theol., Vita Barlaam et Joasaph [Sp.] 
Page 606, line 22

Προστάγματι δέ τινος φοβερωτάτου κατ' ὄναρ 
κραταιῶς ἐπισκήπτοντος πεισθείς, ὁ τοῦτον 
κηδεύσας ἀναχωρητὴς τὰ βασίλεια καταλαμ-
βάνει Ἰνδῶν, καὶ τῷ βασιλεῖ Βαραχίᾳ προσελθὼν 
πάντα αὐτῷ δῆλα τὰ περὶ τοῦ Βαρλαὰμ καὶ τοῦ 
μακαρίου τούτου τίθησιν Ἰωάσαφ. 

\end{greek}



\section{\emph{Chronicon Paschale}}
\blockquote[From Wikipedia\footnote{\url{http://en.wikipedia.org/wiki/Chronicon_Paschale}}]{
Jump to: navigation, search

Chronicon Paschale ("the Paschal Chronicle, also Chronicum Alexandrinum or Constantinopolitanum, or Fasti Siculi ) is the conventional name of a 7th-century Greek Christian chronicle of the world. Its name comes from its system of chronology based on the Christian paschal cycle; its Greek author named it "Epitome of the ages from Adam the first man to the 20th year of the reign of the most August Heraclius."

The Chronicon Paschale follows earlier chronicles. For the years 600 to 627 the author writes as a contemporary historian - that is, through the last years of emperor Maurice, the reign of Phocas, and the first seventeen years of the reign of Heraclius.}

\begin{greek}

Chronicon Paschale, Chronicon paschale (2371: 001)
“Chronicon paschale, vol. 1”, Ed. Dindorf, L.

Chronicon Paschale, Chronicon paschale 
Page 48, line 14

                                      αἱ δὲ χῶραι αὐτῶν εἰσι κατὰ 
τὰς φυλὰς αὐτῶν αὗται· ἡ Λυχνῖτις, Μηδία, Ἀδριακή, ἀφ' 
ἧς τὸ Ἀδριακὸν πέλαγος, Ἀλβανία, Γαλλία, Ἀμαζονίς, Ἰταλία, 
Ἀρμενία μικρά τε καὶ μεγάλη, Θουσκηνή, Καππαδοκία, Λυσι-
τανία, Παφλαγονία, Μεσσαλία, Γαλατία, Κελτίς, Κολχίς, 
Σπανογαλλία, Ἰνδική, Ἰβηρία, Ἀχαΐα, Σπανία ἡ μεγάλη, Βο-
σπορηνή, Μαιῶτις, Δέῤῥις, Σαρματίς, Ταυριαννίς, Βασταρ-
νίς, Σκυθία, Θρᾴκη, Μακεδονία, Δελματία, Κολχίς, Θεττα-
λίς, Λοκρίς, Βοιωτία, Αἰτωλία, Ἀττική, Ἀχαΐα, Πελοπόννη-
σος, Ἀκαρνία, Ἠπειρῶτις, Ἰλλυρίς. 



Chronicon Paschale, Chronicon paschale 
Page 49, line 14

Οὗτος Μεσραεὶμ ὁ Αἰγύπτιος μετέπειτα ἐπὶ τὰ ἀνατολικὰ 
μέρη οἰκήσας οἰκήτωρ γίνεται Βάκτρων, τὴν ἐσωτέραν Περσίδος 
λέγει Ἄσοα τῶν μεγάλων Ἰνδῶν. 



Chronicon Paschale, Chronicon paschale 
Page 52, line 13

                  τὰ δὲ ὀνόματα τῶν χωρῶν τοῦ Χάμ ἐστι ταῦ-
τα· Αἴγυπτος σὺν τοῖς περὶ αὐτὴν πᾶσιν, Αἰθιοπία ἡ βλέπουσα 
κατὰ Ἰνδούς, καὶ ἑτέρα Αἰθιοπία, ὅθεν ἐκπορεύεται ὁ τῶν Αἰθιό-
πων ποταμὸς ὁ καλούμενος Νεῖλος ὁ καὶ Γήων, Θηβαΐς, Λι-
βύη ἡ παρεκτείνουσα μέχρι Κυρήνης, Μαρμαρὶς καὶ τὰ περὶ αὐ-
τὴν πάντα, Σύρτις ἔχουσα ἔθνη τρία, Νασαμῶνας, Μάκας, 
Ταυταμαίους, Λιβύη ἑτέρα ἡ ἀπὸ Λέπτεως παρεκτείνουσα μέ-
χρις Ἡρακλεωτικῶν στηλῶν κατέναντι Γαδείρων. 



Chronicon Paschale, Chronicon paschale 
Page 53, line 18

Εἶτα πάλιν Ἐπιφάνιος, Τῷ δὲ Σὴμ πρώτῳ υἱῷ τοῦ Νῶε 
ὑπέπεσεν ὁ κλῆρος ὁ ἀπὸ Περσίδος καὶ Βάκτρων ἕως Ἰνδικῆς. 



Chronicon Paschale, Chronicon paschale 
Page 54, line 21

  Ἐλμωδάδ, ἐξ οὗ οἱ Ἰνδοί. 



Chronicon Paschale, Chronicon paschale 
Page 55, line 13

                                                             πάν-
των δὲ τῶν υἱῶν τοῦ Σὴμ ἡ κατοικία ἐστὶν ἀπὸ Βάκτρων ἕως 
Ῥινοκορούρων τῆς ὁριζούσης Συρίαν καὶ Αἴγυπτον καὶ τὴν ἐρυ-
θρὰν θάλασσαν ἀπὸ στόματος τοῦ κατὰ Ἀρσενοΐτην τῆς Ἰνδικῆς. 



Chronicon Paschale, Chronicon paschale 
Page 55, line 18


Ταῦτα δέ εἰσιν τὰ ἐξ αὐτοῦ γενόμενα ἔθνη· αʹ Ἑβραῖοι οἱ 
καὶ Ἰουδαῖοι, βʹ Πέρσαι, γʹ Ἀσσύριοι δεύτεροι, δʹ Αἱλυμαῖοι, εʹ 
Χαλδαῖοι, ϛʹ Ἀραμοσσυνοί, ζʹ Ἄραβες οἱ δεύτεροι, ηʹ Μῆδοι, 
θʹ Ὑρκανοί, ιʹ Μακαρδοί, ιαʹ Κοσσαῖοι, ιβʹ Σκύθαι, ιγʹ Σα-
λαθιαῖοι, ιδʹ Γυμνοσοφισταί, ιεʹ Παίονες, ιʹ2ʹ Ἰνδοὶ πρῶτοι, ιζʹ 
Πάρθοι, ιηʹ Ἄραβες ἀρχαῖοι, ιθʹ Καρμήλιοι, κʹ Βακτριανοί, 
καʹ Ἀῤῥιανοί, κβʹ Ἰνδοὶ δεύτεροι, κγʹ Γερμανοί, κδʹ Κεδρούσιοι, 
κεʹ Γασφηνοί, κϛʹ Ἑρμαῖοι. 



Chronicon Paschale, Chronicon paschale 
Page 56, line 2

Οἱ δὲ ἐπιστάμενοι αὐτῶν γράμματα Ἑβραῖοι οἱ καὶ Ἰουδαῖοι, 
Πέρσαι, Μῆδοι, Χαλδαῖοι, Ἰνδοί, Ἀσσύριοι. 



Chronicon Paschale, Chronicon paschale 
Page 56, line 4

Ἔστι δὲ ἡ κατοικία τῶν υἱῶν Σὴμ παρεκτείνουσα κατὰ μῆ-
κος μὲν ἀπὸ τῆς Ἰνδικῆς ἕως Ῥινοκορούρων, πλάτος δὲ ἀπὸ Περ-
σίδος καὶ Βάκτρων ἕως τῆς Αἰθιοπίας καὶ τῆς Κιλικίας. 



Chronicon Paschale, Chronicon paschale 
Page 56, line 9

Τὰ δὲ ὀνόματα τῶν χωρῶν τῶν υἱῶν τοῦ Σήμ, πρωτοτόκου 
υἱοῦ Νῶε, ἐστὶν ταῦτα· αʹ Περσίς, βʹ Βακτριανή, γʹ Ὑρκανία, 
δʹ Βαβυλωνία, εʹ Κορδυαία, ϛʹ Ἀσσυρία, ζʹ Μεσοποταμία, ηʹ 
Ἰνδική, [ϛʹ Ἐλυμαΐς, ζʹ Ἀραβία ἡ ἀρχαία] θʹ Ἀραβία ἡ εὐδαί-
μων; 



Chronicon Paschale, Chronicon paschale 
Page 56, line 19

Τὰ δὲ ἔθνη ἃ διέσπειρε κύριος ὁ θεὸς ἐπὶ τῆς γῆς μετὰ τὸν 
κατακλυσμὸν ἐν ταῖς ἡμέραις Φαλὲγ καὶ Ἰεκτὰν τοῦ ἀδελφοῦ 
αὐτοῦ ἐν τῇ πυργοποιίᾳ, ὅτε συνεχύθησαν αἱ γλῶσσαι αὐτῶν, 
ἐστὶν ταῦτα· αʹ Ἑβραῖοι οἱ καὶ Ἰουδαῖοι, βʹ Ἀσσύριοι, γʹ Χαλ-
δαῖοι, δʹ Μῆδοι, εʹ Πέρσαι, ϛʹ Ἄραβες πρῶτοι καὶ Ἄραβες 
δεύτεροι, ζʹ Μαδιναῖοι, ηʹ Μαδιναῖοι δεύτεροι, θʹ Ταϊανοί, ιʹ 
Ἀλαμοσυνοί, ιαʹ Σαρακηνοί, ιβʹ Μάγοι, ιγʹ Κάσπιοι, ιδʹ Ἀλβα-
νοί, ιεʹ Ἰνδοὶ πρῶτοι, Ἰνδοὶ δεύτεροι, ιϛʹ Αἰθίοπες πρῶτοι, Αἰ-
θίοπες δεύτεροι, ιζʹ Αἰγύπτιοι καὶ Θηβαῖοι, ιηʹ Λίβυες πρῶ-
τοι, Λίβυες δεύτεροι, ιθʹ Χετταῖοι, κʹ Χαναναῖοι, καʹ Φε-  
ρεζαῖοι, κβʹ Εὐαῖοι, κγʹ Ἀμοῤῥαῖοι, κδʹ Γεργεσαῖοι, κεʹ Ἰεβουσαῖοι, 
κϛʹ Ἰδουμαῖοι, κζʹ Σαρμάται, κηʹ Φοίνικες, κθʹ Σύροι, λʹ Κίλικες, 
λαʹ Καππάδοκες, λβʹ Ἀρμένιοι, λγʹ Ἴβηρες, λδʹ Βεβρανοί, λεʹ 
Σκύθες, λϛʹ Κόλχοι, λζʹ Σάννιοι, ληʹ Βοσποριανοί, λθʹ Ἀσια-
νοί, μʹ Ἴσαυροι, μαʹ Λυκάονες, μβʹ Πισίδαι, μγʹ Γαλάται, 
μδʹ Παφλαγόνες, μεʹ Φρύγες, μϛʹ Ἕλληνες οἱ καὶ Ἀχαιοί, μζʹ 
Θετταλοί, μηʹ Μακεδόνες, μθʹ Θρᾷκες, νʹ Μυσοί, ναʹ Βέσσοι, 




Chronicon Paschale, Chronicon paschale 
Page 61, line 17

                                                             ποτα-
μοὶ γάρ εἰσιν ὀνομαστοὶ μʹ· αʹ Ἰνδὸς ὁ καὶ Φεισών, βʹ Νεῖλος ὁ 
καὶ Γήων, γʹ Τίγρις, δʹ Εὐφράτης, εʹ Ἰορδάνης, ϛʹ Κηφινσός, 
ζʹ Τάναϊς, ηʹ Ἰσμηνός, θʹ Ἐρύμανθος, ιʹ Ἅλυς, ιαʹ Ἀσωπός, 
ιβʹ Θερμωδών, ιγʹ Ἐρασινός, ιδʹ Ῥεῖος, ιεʹ Ἀλφειός, ιϛʹ Βορυ-
σθένης, ιζʹ Ταῦρος, ιηʹ Εὐρώτας, ιθʹ Μαίανδρος, κʹ Εἶρμος,   
καʹ Ἄξιος, κβʹ Πύραμος, κγʹ Βοῖος, κδʹ Ἕβρος, κεʹ Σαγάριος, 
κϛʹ Ἀχελῶος, κζʹ Πηνειός, κηʹ Εὔηνος, κθʹ Σπερχειός, λʹ Κάϋ-
στρος, λαʹ Σιμόεις, λβʹ Σκάμανδρος, λγʹ Στρυμών, λδʹ Παρ-
θένιος, λεʹ Ἴστρος, λϛʹ Βαῖτις, λζʹ Ῥῆνος, ληʹ Ῥοδανός, λθʹ 




Chronicon Paschale, Chronicon paschale 
Page 64, line 11

Περὶ ἀστρονομίας


Ἐν τοῖς χρόνοις τῆς πυργοποιίας ἐκ τοῦ γένους – τοῦ Ἀρφα-
ξὰδ ἀνήρ τις Ἰνδὸς ἀνεφάνη σοφὸς ἀστρονόμος, ὀνόματι Ἀν-
δουβάριος, ὃς καὶ συνεγράψατο πρῶτος Ἰνδοῖς ἀστρονομίαν. 



Chronicon Paschale, Chronicon paschale 
Page 268, line 8

          ὁ δὲ αὐτὸς καὶ ἄλλας βασιλείας περιεῖλεν Ἀσίας, Κα-
ρίας, Λυκίας, Ἰνδῶν πρὸς βοῤῥᾶν καὶ Σακῶν καὶ Σκυθῶν. 

\end{greek}




\section{Choricius of Gaza}
\blockquote[From Wikipedia\footnote{\url{http://en.wikipedia.org/wiki/Choricius}}]{Choricius, of Gaza (Greek: Χορίκιος), Greek sophist and rhetorician, flourished in the time of Anastasius I (AD 491-518).

He was the pupil of Procopius of Gaza, who must be distinguished from Procopius of Caesarea, the historian. A number of his declamations and descriptive treatises have been preserved. The declamations, which are in many cases accompanied by explanatory commentaries, chiefly consist of panegyrics, funeral orations and the stock themes of the rhetorical schools. The wedding speeches, wishing prosperity to the bride and bridegroom, strike out a new line.

Choricius was also the author of descriptions of works of art after the manner of Philostratus. The moral maxims, which were a constant feature of his writings, were largely drawn upon by Macanus Chrysocephalas, metropolitan of Philadelphia (middle of the 14th century), in his Rodonia (rose-garden), a voluminous collection of ethical sayings.

The style of Choricius is praised by Photius as pure and elegant, but he is censured for lack of naturalness. A special feature of his style is the persistent avoidance of hiatus, peculiar to what is called the school of Gaza.
}
\begin{greek}

Choricius Rhet., Soph., Opera (4094: 001)
“Choricii Gazaei opera”, Ed. Foerster, R., Richtsteig, E.
Leipzig: Teubner, 1929.
Oration-declamation-dialexis 3, section 2, paragraph 67, line 3

   σκοπεῖτε γάρ· νῆσός 
ἐστιν ὄνομα μὲν Ἰοτάβη, τὸ δὲ ἔργον αὐτῆς ὑποδοχὴ φορ-
τίων τῶν Ἰνδικῶν, ὧν μέγας φόρος τὰ τέλη. 


Choricius Rhet., Soph., Opera 
Oration-declamation-dialexis 10, section 1, paragraph 6, line 5

   οὕτω 
γὰρ Ἀχιλλέα τε μᾶλλον κοσμήσει καὶ σφαλερωτέραν ἀπο-
δείξει τὴν Τροίαν τοῦ σώζειν εἰωθότος ἀνῃρημένου μηδὲ 
τῶν ἀρτίως ἐλθόντων τῇ Τροίᾳ συμμάχων Αἰθιόπων, 
Ἰνδῶν, Ἀμαζόνων εἰς ἐπικουρίαν ἀποχρῆν δεικνυμένων. 


Choricius Rhet., Soph., Opera 
Oration-declamation-dialexis 10, section 2, paragraph 24, line 1

Πρὸς τούτῳ μοι καὶ Ἰνδοὺς καὶ Αἰθίοπας λέγε· τὸ 
γὰρ αὐτό μοι κατὰ πάντων εἰπεῖν ὑπάρξει δικαίως, εἴπερ 
καὶ Αἰθίοπες καὶ Ἰνδοὶ πάρεισιν ἐγνωκότες, ὡς ἦν Ἕκτωρ 
ἡμῖν εὖ μάλα συγκεκροτημένος τὰ τοῦ πολέμου καὶ 
δύναμιν ἠσκημένος ἀξίαν θαυμάσαι δεινὸς μὲν ἀθυμοῦντα 
στρατὸν ἀγαθῶν ἐλπίδων πληρῶσαι, εὖ δὲ παρασχὸν καὶ 
ταῖς ὁλκάσιν αὐταῖς φλόγα προσάγειν. 


Choricius Rhet., Soph., Opera 
Oration-declamation-dialexis 12, section 1, paragraph 6, line 4

ἀλλὰ περιττὸς ὁ Φθιώτης τῇ Τροίᾳ δειχθήσεται συλλαμ-
βανόντων αὐτῇ τῶν Ὀλυμπίων τῷ περὶ τὴν Ἕκτορος 
ἀτιμίαν ἐλέῳ καὶ πολλῆς ἄρτι παραγενομένης ἐπικουρίας 
Αἰθιόπων, Ἰνδῶν, Ἀμαζόνων. 


Choricius Rhet., Soph., Opera 
Oration-declamation-dialexis 12, section 2, paragraph 28, line 1

Εἶεν· τὴν δὲ τῶν Ἰνδῶν ἐπικουρίαν, τὴν δὲ τῶν 
Αἰθιόπων προσθήκην ποῦ χοροῦ τάξομεν; 

\end{greek}



\section{Eutecnius}
\blockquote[From Brill's New Pauly\footnote{\url{http://referenceworks.brillonline.com/entries/brill-s-new-pauly/eutecnius-e407140?s.num=0}}]{(Εὐτέκνιος; Eutéknios). The famous Cod. Vindobonensis med. gr. 1 (late 5th cent. AD) with the herbal of Pedanius Dioscorides also contains prose paraphrases on  Nicander's	Thēriaká and Alexiphármaka [4; 2; 5]. A remark in a manuscript attributes them to a ‘rhetor’ (σοφιστής; sophistḗs) by the name of E., who is to be dated sometime between the 3rd and 5th cents. AD [3. 34-37]; without any solid proof, the following anonymous paraphrases are also attributed to the same E.: …}

\begin{greek}

Eutecnius Soph., Paraphrasis in Nicandri theriaca (0752: 001)
“Eutecnii paraphrasis in Nicandri theriaca”, Ed. Gualandri, I.
Milan: Istituto Editoriale Cisalpino, 1968.
Page 67, line 15

                     <Καὶ> μὴν καὶ ἡ Ἰνδῶν ὁπόσα γῆ καὶ 
ὁ Χοάσπης ποταμὸς ἀρώματα φέρει σὺν ταύταις μίγνυε, 
καὶ πιστακίων ἀκρέμονας (ταῖς σμικραῖς ἀμυγδάλαις ὁ 
τῶν πιστακίων πως παρέοικε καρπός)· καυκαλίδες τε καὶ 
μύρτα καὶ ἄρμινθος ἡ βοτάνη καὶ μάραθον χλωρόν· ἔτι 
μὲν τοῖσδε παρέστω σοι καὶ ἐρύσιμος βοτάνη καὶ τῶν 
ἀγρίων ἐρεβίνθων ὁ καρπὸς βαλλέσθω σὺν αὐτοῖς τοῖς 
κλάδοις (ὀδμὴν δὲ βαρεῖαν ὁ ἄγριος οὗτος ἐρέβινθος ἔχει, 
καί ἐστιν ἐπαχθής) σισύμβριόν τε δὴ καὶ τοῦτο ἐπειδὰν 
ἐν τῇ ὥρᾳ τοῦ ἄνθους γένηται· ὠφελιμότατον γὰρ τοῖς 

Eutecnius Soph., Paraphrasis in Oppiani cynegetica (fort. auctore Eutecnio) (0752: 003)
“Die Paraphrase des Euteknios zu Oppians Kynegetika”, Ed. Tüselmann, O.
Berlin: Weidmann, 1900; Abhandlungen der königlichen Gesellschaft der Wissenschaften zu Göttingen, Philol.–hist. Kl., N.F. 4.1.
Page 25, line 4

                                                            Ἀλλ' ἔμαθον οὐκ εἰς μακρὰν 
τῆς ἑταιρείας τὸ ἀσύμφορον καὶ πικρῶς τῆς συνηθείας ἀπώναντο, ἀντὶ φίλων καὶ 
συνήθων ἀλλήλοις καταστάντες ἐχθροὶ καὶ ἐπίβουλοι· κυνηγῶν γὰρ ἐπιφροσύναις καὶ 
μηχανήμασι δόρκοι τε περδίκων ψευδέσιν ἑάλωσαν ἀπατηθέντες ἰνδάλμασι καὶ δόρκων 
ἔμπαλιν πέρδικες. 



Eutecnius Soph., Paraphrasis in Oppiani cynegetica (fort. auctore Eutecnio) 
Page 39, line 26

                                                                         Οὐδὲ Γάγγῃ τῷ 
παρ' Ἰνδοῖς ποταμῷ τοσοῦτον βρύχημα, ὃς ἐξ ἀποτόμων καταῤῥέων πετρῶν ἐστι μὲν 
<καὶ> καθ' ἑαυτὸν πολὺς, μείζων δὲ γίνεται ποταμῶν ἐπιμιγνυμένων ἄλλων καὶ συνεισβαλ-
λόντων ἐκείνῳ τὰ ῥεύματα, ὑφ' ὧν εἰς τὸ μετέωρον κυρτούμενος πλάτει μὲν καὶ 
μεγέθει χώρας καλύπτει τὰς παραιγιαλίτιδας· οὕτως ὁ θὴρ ἐπιβρέμεται τῷ φοβερῷ 
τοῦ βρυχήματος, καὶ ἀπὸ τῶν ὀρέων ἦχον προκαλούμενος ὥρμηται κατὰ τῶν ἀνδρῶν 
σαρκῶν ἀνθρωπείων ἐμφορηθῆναι διψῶν· οἱ δὲ ὑπομένουσι τὴν ὁρμὴν ἑστῶτες ἀμε-
ταστρεφεῖς καὶ ἀτίνακτοι. 

\end{greek}


\section{Mantissa Proverbiorum}%???
%date?
%\blockquote[From Wikipedia\footnote{\url{}}]{}
%http://www26.us.archive.org/stream/moralsignifican00houggoog/moralsignifican00houggoog_djvu.txt
\begin{greek}
Mantissa Proverbiorum, Mantissa proverbiorum (0200: 001)
“Corpus paroemiographorum Graecorum, vol. 2”, Ed. von Leutsch, E.L.
Göttingen: Vandenhoeck \& Ruprecht, 1851, Repr. 1958.
Centuria 2, section 11, line 3
                                                                   ὁ 
δὲ ποηφάγος ζῷόν ἐστιν ἐν Ἰνδοῖς. 
\end{greek}

\section{Marcian of Heraclea}
\blockquote[From Wikipedia\footnote{\url{http://en.wikipedia.org/wiki/Marcianus_Heracleensis}}]{Marcian of Heraclea (Marcianus Heracleensis) was a minor Greek geographer of Late Antiquity (fl. ca. 4th century). His surviving works are:

    Periplus maris externi, ed. Müller (1855),515-562.
    Menippi periplus maris interni (epitome Marciani), ed. Müller (1855), 563-572.
    Artemidori geographia (epitome Marciani), ed. Müller (1855), 574-576.
}
\begin{greek}

Marcianus Geogr., Periplus maris exteri (4003: 001)
“Geographi Graeci minores, vol. 1”, Ed. Müller, K.
Paris: Didot, 1855, Repr. 1965.
Book 1, section A, line 6

Τῶν δεξιῶν μερῶν τοῦ τε Ἀραβίου κόλπου καὶ τῆς Ἐρυθρᾶς 
θαλάσσης καὶ τοῦ Ἰνδικοῦ πελάγους περίπλους. 



Marcianus Geogr., Periplus maris exteri 
Book 1, section A, line 8

Τῶν ἀριστερῶν μερῶν τοῦ τε Ἀραβίου κόλπου καὶ τῆς Ἐρυ-
θρᾶς θαλάσσης καὶ τοῦ Ἰνδικοῦ πελάγους περίπλους. 



Marcianus Geogr., Periplus maris exteri 
Book 1, section A, line 16

Γεδρωσίας περίπλους. 
 Ἰνδικῆς τῆς ἐντὸς Γάγγου ποταμοῦ καὶ τῶν ἐν αὐτῇ κόλπων 
καὶ νήσων περίπλους. 



Marcianus Geogr., Periplus maris exteri 
Book 1, section A, line 20

Τοῦ Γαγγητικοῦ κόλπου περίπλους.   
 Ἰνδικῆς τῆς ἐκτὸς Γάγγου ποταμοῦ καὶ τῶν ἐν αὐτῇ κόλπων 
περίπλους. 



Marcianus Geogr., Periplus maris exteri 
Book 1, section 1, line 27

τάτου Πτολομαίου ἔκ τε τῆς Πρωταγόρου τῶν σταδίων 
ἀναμετρήσεως, ἣν τοῖς οἰκείοις τῆς γεωγραφίας βιβλίοις 
προστέθεικεν, ἔτι μὴν καὶ ἑτέρων πλείστων ἀρχαίων 
ἀνδρῶν, τὸν περίπλουν ἀναγράψαι προειλόμεθα ἐν βιβλίοις 
δυσὶ, τὸν μὲν ἑῷον καὶ μεσημβρινὸν ὠκεανὸν ἐν τῷ προ-
τέρῳ βιβλίῳ, τὸν δ' ἑσπέριον καὶ τὸν ἀρκτῷον ἐν τῷ 
δευτέρῳ, ἅμα ταῖς ἐν αὐτοῖς κειμέναις μεγίσταις νήσοις, 
τῇ τε Ταπροβάνῃ καλουμένῃ, τῇ Παλαισιμούνδου 
λεγομένῃ πρότερον, καὶ ταῖς Πρεττανικαῖς ἀμφοτέραις   
νήσοις· ὧν τὴν μὲν πρώτην κατὰ [τὸ] μεσαίτατον τοῦ 
Ἰνδικοῦ πελάγους κεῖσθαι συνέστηκε, τὰς δ' ἑτέρας δύο 
ἐν τῷ ἀρκτῴῳ ὠκεανῷ. 



Marcianus Geogr., Periplus maris exteri 
Book 1, section 6, line 4

Τῶν δὲ τριῶν θαλασσῶν τῷ μεγέθει τυγχάνει πρώτη 
μὲν ἡ κατὰ τὸ Ἰνδικὸν πέλαγος· δευτέρα δὲ ἡ καθ' ἡμᾶς 
ἡ μεταξὺ Λιβύης καὶ Εὐρώπης, ἀρχομένη μὲν ἀπὸ 
Γαδείρων ἤτοι τοῦ Ἡρακλείου πορθμοῦ, διήκουσα δὲ 
μέχρι τῆς Ἀσίας· τρίτη δὲ ἡ Ὑρκανία. 



Marcianus Geogr., Periplus maris exteri 
Book 1, section 6, line 10

Μέγεθος δὲ τῆς οἰκουμένης, τὸ μὲν ἀπὸ ἀνατολῆς ἐπὶ 
δύσιν ἀναμεμέτρηται σταδίων Μ ζʹ, ͵ηφμεʹ· τοῦτο δέ   
ἐστι τὸ ἀπὸ Γάγγου ποταμοῦ ἐκβολῶν, τοῦ ἐν Ἰνδοῖς 
ἀνατολικωτάτου ποταμοῦ, ἐπὶ τὸ δυτικώτατον τῆς ὅλης 
οἰκουμένης ἀκρωτήριον, ὃ καλεῖται μὲν Ἱερὸν ἄκρον, 
τῆς δὲ Ἰβηρίας ἐστὶ τῶν Λυσιτανῶν ἔθνους. 



Marcianus Geogr., Periplus maris exteri 
Book 1, section 10, line 4

Τῶν μὲν οὖν ἀριστερῶν τῆς Ἀσίας μερῶν, τουτ-
έστι τῆς τε Ἀραβίας τῆς Εὐδαίμονος καὶ τῆς Ἐρυθρᾶς 
θαλάσσης καὶ μετ' ἐκείνην τοῦ Περσικοῦ κόλπου καὶ 
τοῦ Ἰνδικοῦ πελάγους παντὸς ἄχρι τοῦ Σινῶν (τοῦ) 
ἔθνους καὶ τοῦ πέρατος τῆς ἐγνωσμένης γῆς τὸν ἀκρι-
βέστατον ποιησόμεθα περίπλουν καὶ τὴν τῶν σταδίων 
ἀναμέτρησιν. 



Marcianus Geogr., Periplus maris exteri 
Book 1, section 10, line 18

Τούτων μὲν γὰρ τῶν δεξιῶν μερῶν ἐπιδρομή ἐστιν ἃ 
τῆς ἀναμετρήσεως πεποιήμεθα σαφῆ, μιᾶς ἕνεκα τῆς 
θέσεως τῆς τε γῆς καὶ τῆς θαλάσσης, ἥνπερ ἔχει πρὸς 
τὰς ἀντιπέρα τῆς Ἀσίας χώρας, τουτέστι τῶν τε Ἀρά-
βων καὶ τῶν Ἰνδῶν καὶ τῶν ἄλλων ἐθνῶν· τῶν δὲ ἀρι-
στερῶν μερῶν μετὰ τῆς προειρημένης ἐπαγγελίας τὸν 
περίπλουν σπουδῇ ἐποιησάμεθα. 



Marcianus Geogr., Periplus maris exteri 
Book 1, section 11T, line 2

ΤΩΝ ΔΕΞΙΩΝ ΜΕΡΩΝ ΤΟΥ ΤΕ ΑΡΑΒΙΟΥ ΚΟΛΠΟΥ 
 ΚΑΙ ΤΗΣ ΕΡΥΘΡΑΣ ΘΑΛΑΣΣΗΣ ΚΑΙ ΤΟΥ ΙΝΔ*ιΚΟΥ 
 ΠΕΛΑΓΟΥΣ ΠΕΡΙΠΛΟΥΣ. 




Marcianus Geogr., Periplus maris exteri 
Book 1, section 12, line 3

Ἐκπλεύσαντι δὲ τὸν κόλπον καὶ τὴν Ἐρυθρὰν 
θάλασσαν, ἠρέμα πως μετὰ τὸν κόλπον κατὰ τὸ ἀκρω-
τήριον στενουμένην, ἐκδέχεται τὸ Ἰνδικὸν πέλαγος 
ἀναπεπταμένον ἐπὶ πολὺ καὶ τῷ μὲν μήκει διῆκον πρὸς 
τὴν ἕω καὶ τὰς ἀνατολὰς τοῦ ἡλίου μέχρι Σινῶν τοῦ 
ἔθνους, ὅπερ ἐπὶ τέλει τῆς οἰκουμένης τυγχάνει κείμε-
νον κατὰ τὴν πρὸς ταῖς ἀνατολαῖς ἄγνωστον γῆν, τῷ 
δὲ πλάτει πρὸς μεσημβρίαν ἀναχεόμενον ἐπὶ πλεῖστον, 
μέχρι τῆς ἑτέρας ἀγνώστου γῆς τῆς κατὰ τὴν μεσημ-
βρίαν ὑπαρχούσης, καθ' ἣν καὶ ἡ Πρασώδης καλου-
μένη διατείνει θάλασσα παρ' ὅλην τὴν μεσημβρινὴν 




Marcianus Geogr., Periplus maris exteri 
Book 1, section 12, line 12


ἀναπεπταμένον ἐπὶ πολὺ καὶ τῷ μὲν μήκει διῆκον πρὸς 
τὴν ἕω καὶ τὰς ἀνατολὰς τοῦ ἡλίου μέχρι Σινῶν τοῦ 
ἔθνους, ὅπερ ἐπὶ τέλει τῆς οἰκουμένης τυγχάνει κείμε-
νον κατὰ τὴν πρὸς ταῖς ἀνατολαῖς ἄγνωστον γῆν, τῷ 
δὲ πλάτει πρὸς μεσημβρίαν ἀναχεόμενον ἐπὶ πλεῖστον, 
μέχρι τῆς ἑτέρας ἀγνώστου γῆς τῆς κατὰ τὴν μεσημ-
βρίαν ὑπαρχούσης, καθ' ἣν καὶ ἡ Πρασώδης καλου-
μένη διατείνει θάλασσα παρ' ὅλην τὴν μεσημβρινὴν 
ἄγνωστον γῆν μέχρι τῆς ἕω, τοῦ μὲν Ἰνδικοῦ πελάγους 
ὑπάρχουσα, ταύτην δὲ διὰ τὴν χροιὰν λαχοῦσα τὴν 
προσηγορίαν. 



Marcianus Geogr., Periplus maris exteri 
Book 1, section 14, line 3

Καὶ ἡ μὲν ὅλη θέσις καὶ περιγραφὴ τῶν δεξιῶν 
μερῶν τοῦ τε Ἀραβίου κόλπου καὶ τῆς Ἐρυθρᾶς θα-
λάσσης καὶ προσέτιγε τοῦ Ἰνδικοῦ πελάγους τοῦ πρὸς 
τὴν μεσημβρίαν ἀποκλίνοντος, τοῦτον ἔχει τὸν τρόπον· 
τὰ δὲ κατὰ μέρος οὕτω πως ἔχει· 
    [Λείπει τὰ κατὰ μέρος. 



Marcianus Geogr., Periplus maris exteri 
Book 1, section 15T, line 2

ΤΩΝ ΑΡΙΣΤΕΡΩΝ ΜΕΡΩΝ ΤΟΥ ΤΕ ΑΡΑΒΙΟΥ ΚΟΛΠΟΥ 
 ΚΑΙ ΤΗΣ ΕΡΥΘΡΑΣ ΘΑΛΑΣΣΗΣ ΚΑΙ ΤΟΥ ΙΝΔ*ιΚΟΥ 
 ΠΕΛΑΓΟΥΣ ΠΑΝΤΟΣ ΠΕΡΙΠΛΟΥΣ. 




Marcianus Geogr., Periplus maris exteri 
Book 1, section 15, line 15

                        Ἐν τούτῳ δὲ τῷ μέρει τῆς θα-
λάσσης καὶ τὸ τῶν Ὁμηριτῶν ἔθνος τυγχάνει τῆς τῶν 
Ἀράβων ὑπάρχον γῆς, μέχρι τῆς ἀρχῆς τοῦ Ἰνδικοῦ 
διῆκον πελάγους. 



Marcianus Geogr., Periplus maris exteri 
Book 1, section 15, line 17

                   Μετὰ δὲ τὴν Ἐρυθρὰν θάλασσαν 
ἑξῆς ἐστι τὸ Ἰνδικὸν πέλαγος. 



Marcianus Geogr., Periplus maris exteri 
Book 1, section 16, line 3

Ἐκπλεύσαντι δὲ τὸν κόλπον καὶ πρὸς τὴν ἕω 
τὸν πλοῦν ποιουμένῳ ἀριστεράν τε ὁμοίως τὴν ἤπειρον 
ἔχοντι, ἐκδέχεται πάλιν τὸ Ἰνδικὸν πέλαγος, ᾧ τὸ λει-
πόμενον τῆς Καρμανίας ἔθνος παροικεῖ. 



Marcianus Geogr., Periplus maris exteri 
Book 1, section 16, line 6

                                             Καὶ μετὰ 
τοῦτο τὸ τῆς Γεδρωσίας ἔθνος κείμενον τυγχάνει· ἑξῆς 
δὲ τούτων ἐστὶν ἡ Ἰνδικὴ ἡ ἐντὸς Γάγγου ποταμοῦ κει-
μένη, ἧς κατὰ τὸ μεσαίτατον τῆς ἠπείρου νῆσος κατ-
αντικρὺ κεῖται μεγίστη Ταπροβάνη καλουμένη. 



Marcianus Geogr., Periplus maris exteri 
Book 1, section 16, line 9

                                                  Μετὰ 
δὲ ταύτην ἡ ἑτέρα ἐστὶν Ἰνδικὴ ἡ ἐκτὸς Γάγγου ποτα-
μοῦ, ὅρου τυγχάνοντος ἑκατέρων τῶν Ἰνδικῶν γαιῶν. 



Marcianus Geogr., Periplus maris exteri 
Book 1, section 16, line 11

Ἐν δὲ τῇ ἐκτὸς Γάγγου Ἰνδικῇ ἡ Χρυσῆ καλουμένη 
χερσόνησός ἐστι· μεθ' ἣν ὁ καλούμενος Μέγας κόλπος· 
οὗ κατὰ τὸ μεσαίτατον οἱ ὅροι τῆς ἐκτὸς Γάγγου Ἰνδι-
κῆς καὶ τῶν Σινῶν εἰσιν. 



Marcianus Geogr., Periplus maris exteri 
Book 1, section 17, line 4

Καὶ ἡ μὲν ὅλη τῶν τόπων θέσις καὶ περιγραφὴ 
τῶν ἀριστερῶν τῆς Ἀσίας μερῶν, τοῦ τε Ἀραβίου κόλ-
που καὶ τῆς Ἐρυθρᾶς θαλάσσης καὶ προσέτιγε τοῦ 
Περσικοῦ κόλπου καὶ τοῦ Ἰνδικοῦ πελάγους παντὸς, 
τοῦτον ἔχει τὸν τρόπον· τὰ δὲ κατὰ μέρος οὕτω πως 
ἔχει. 



Marcianus Geogr., Periplus maris exteri 
Book 1, section 17, line 6

Ἡ Εὐδαίμων Ἀραβία περιορίζεται ἀπὸ μὲν 
ἄρκτων ταῖς πλευραῖς τῆς τε Πετραίας Ἀραβίας καὶ 
ἔτι τῆς Ἐρήμου Ἀραβίας καὶ τῷ νοτίῳ μέρει τοῦ Περ-
σικοῦ κόλπου μέχρι τῶν ἐκβολῶν τοῦ Τίγριδος ποταμοῦ, 
[ἀπὸ δὲ ἀνατολῶν μέρει τε τοῦ Περσικοῦ κόλπου] 
καὶ μέρει τῆς Ἰνδικῆς θαλάσσης, ἀπὸ δὲ μεσημβρίας 
τῇ Ἐρυθρᾷ θαλάσσῃ, [ἀπὸ δὲ δύσεως τῷ Ἀραβίῳ 
κόλπῳ]. 



Marcianus Geogr., Periplus maris exteri 
Book 1, section 17, line 10

                                                   Προπέ-
πτωκε πρὸς τὴν μεσημβρίαν εἰς τὴν Ἐρυθρὰν θάλασσαν 
καὶ τὸ Ἰνδικὸν πέλαγος ἐπὶ πλεῖστον, καὶ ὥσπερ χερ-
σόνησος μεγίστη πλατυτάτῳ ἰσθμῷ προσεχομένη πε-
ριρρεῖται τῇ θαλάσσῃ. 



Marcianus Geogr., Periplus maris exteri 
Book 1, section 18, line 16

<Χαδραμωτῖται>, ἔθνος περὶ τὸν Ἰνδικὸν κόλπον, 
τῷ Πρίονι παροικοῦντες ποταμῷ, ὥς φησι Μαρκιανὸς 
ἐν τῷ Περίπλῳ αὐτοῦ. 



Marcianus Geogr., Periplus maris exteri 
Book 1, section 18, line 19

<Ἀσκῖται>, ἔθνος παροικοῦν τὸν Ἰνδικὸν κόλπον καὶ   
ἐπὶ ἀσκῶν πλέον, ὡς Μαρκιανὸς ἐν τῷ Περίπλῳ αὐτοῦ· 
»Παροικεῖ αὐτὸν ἔθνος καὶ αὐτὸ καλούμενον Σαχαλιτῶν· 
ἔτι μὴν καὶ Ἀσκιτῶν ἕτερον ἔθνος [ἐπὶ ἀσκῶν πλέον]». 



Marcianus Geogr., Periplus maris exteri 
Book 1, section 26, line 2

Ἡ Καρμανία μέρει μέν τινι κατὰ τὸν Περσι-
κὸν κεῖται κόλπον, μέρει δὲ παρὰ τὸ Ἰνδικὸν πέλαγος 
[τὸ] μετὰ τὸν κόλπον τὸν Περσικόν. 



Marcianus Geogr., Periplus maris exteri 
Book 1, section 26, line 10

                                          Περιορίζεται   
δὲ ἀπὸ μὲν ἄρκτων τῇ ἐρήμῳ Καρμανίᾳ, ἀπὸ δὲ δύ-
σεως τῇ προρρηθείσῃ Περσίδι καὶ τῷ προειρημένῳ Βα-
γράδᾳ ποταμῷ καὶ ἔτι τῷ λειπομένῳ μέρει τοῦ Περ-
σικοῦ κόλπου, (διὰ τὸ πρὸς δύσιν ὁρᾶν αὐτὸν) καλουμένῳ 
Καρμανικῷ· ἀπὸ δὲ ἀνατολῶν Γεδρωσίᾳ τῷ ἔθνει παρὰ 
τὰ Παρσικὰ ὄρη· ἀπὸ δὲ μεσημβρίας μετὰ τὰ στενὰ 
τοῦ Περσικοῦ κόλπου τῷ Ἰνδικῷ πελάγει. 



Marcianus Geogr., Periplus maris exteri 
Book 1, section 28, line 2

Μετὰ δὲ τὴν Κάρπελλαν ἄκραν] ἐκδέχεται 
τὸ Ἰνδικὸν πέλαγος πρὸς ἀνατολὰς ἐκτεινόμενον· ᾧ τὸ 
λειπόμενον μέρος τῆς Καρμανίας παρήκει μέχρι Μου-
σαρναίων γῆς. 



Marcianus Geogr., Periplus maris exteri 
Book 1, section 30, line 7

                    Οἱ πάντες ἀπὸ τοῦ Καρπέλλης 
ἀκρωτηρίου μέχρι Μουσάρνων πόλεως τοῦ περίπλου 
τῆς Καρμανίας τῆς παρὰ τὸ Ἰνδικὸν πέλαγος στάδιοι 
͵εϡνʹ. 



Marcianus Geogr., Periplus maris exteri 
Book 1, section 31, line 4

Ἡ Γεδρωσία περιορίζεται ἀπὸ μὲν ἄρκτων τῇ 
Δραγγιανῇ καὶ τῇ Ἀραχωσίᾳ, ἀπὸ δὲ δύσεως τῇ προει-
ρημένῃ Καρμανίᾳ μέχρι θαλάσσης, ἀπὸ δὲ ἀνατολῶν τῷ 
τῆς Ἰνδικῆς μέρει τῷ παρὰ τὸν Ἰνδὸν ποταμὸν μέχρι 
τοῦ πρὸς τῇ μνημονευθείσῃ Ἀραχωσίᾳ ὁρίου, ἀπὸ δὲ 
μεσημβρίας τῷ Ἰνδικῷ πελάγει. 



Marcianus Geogr., Periplus maris exteri 
Book 1, section 32, line 12

Ἐντεῦθεν ἄρχεται ἡ Παταληνὴ χώρα, ἧς τὸ πλεῖστον 
ὁ Ἰνδὸς ποταμὸς τοῖς στόμασιν ἐμπεριείληφε· καὶ αὐ-
τὴν δὲ τὴν μητρόπολιν καλουμένην Πάταλα μετὰ τὸ γʹ 
στόμα τοῦ Ἰνδοῦ ποταμοῦ ὥσπερ νῆσον κεῖσθαι συμ-
βέβηκε, καὶ ἑτέρας πόλεις πλείστας. 



Marcianus Geogr., Periplus maris exteri 
Book 1, section 34T, line 1

                        Οἱ πάντες ἀπὸ Μουσάρνων πό-
λεως εἰς Ῥίζανα τῆς τῶν Γεδρωσίων παραλίας στά-
διοι γωνʹ. 
ΙΝΔ*ιΚΗΣ ΤΗΣ ΕΝΤΟΣ ΓΑΓΓΟΥ ΠΟΤΑΜΟΥ, ΚΑΙ ΤΩΝ 
 ΕΝ ΑΥΤΗι ΚΟΛΠΩΝ ΚΑΙ ΝΗΣΩΝ ΠΕΡΙΠΛΟΥΣ. 




Marcianus Geogr., Periplus maris exteri 
Book 1, section 34, line 1

Ἡ ἐντὸς Γάγγου ποταμοῦ Ἰνδικὴ περιορίζεται 
ἀπὸ μὲν ἄρκτων τῷ Ἰμάῳ ὄρει παρὰ τοὺς ὑπερκειμέ-
νους αὐτοῦ Σογδιανοὺς καὶ Σάκας, ἀπὸ δὲ δύσεως πρὸς 
μὲν τῇ θαλάσσῃ τῇ προειρημένῃ Γεδρωσίᾳ, κατὰ δὲ τὴν 
μεσόγειον τῇ Ἀραχωσίᾳ καὶ ἀνωτέρω τοῖς Παροπανι-
σάδαις, ἀπὸ δὲ ἀνατολῶν τῷ Γάγγῃ ποταμῷ, ἀπὸ δὲ 
μεσημβρίας τῷ Ἰνδικῷ πελάγει. 



Marcianus Geogr., Periplus maris exteri 
Book 1, section 34, line 12

                                      Καὶ ἡ μὲν ὅλη πε-
ριγραφὴ τοιαύτη· [τὰ δὲ κατὰ μέρος οὕτως ἔχει·] 
[Λείπει τὰ κατὰ μέρος] 
Ὁ δὲ πᾶς περίπλους ἀπὸ τοῦ Ναυστάθμου λιμένος μέ-  
χρι τοῦ Κῶρυ ἀκρωτηρίου τοῦ μέρους τοῦ προειρημέ-
νου τῆς ἐντὸς Γάγγου Ἰνδικῆς σταδίων ͵͵β ͵αψκεʹ. 



Marcianus Geogr., Periplus maris exteri 
Book 1, section 35, line 1

Τῷ ἀκρωτηρίῳ τῆς Ἰνδικῆς τῷ καλουμένῳ Κῶρυ 
ἀντίκειται τὸ τῆς Ταπροβάνης νήσου ἀκρωτήριον τὸ 
καλούμενον Βόρειον. 



Marcianus Geogr., Periplus maris exteri 
Book 1, section 35F, line 2N

[Μάργανα, πόλις τῆς Ἰνδικῆς. 



Marcianus Geogr., Periplus maris exteri 
Book 1, section 36, line 12

                                                  Πάλιν 
δὲ ἐπανήξομεν ἐπὶ τὸν παράπλουν τῆς ἐντὸς Γάγγου 
Ἰνδικῆς. 



Marcianus Geogr., Periplus maris exteri 
Book 1, section 37, line 5

Ἀπὸ τοῦ Ἀφετηρίου τούτου ἐκδέχεται ὁ Γαγγη-
τικὸς καλούμενος κόλπος μέγιστος ὢν σφόδρα, οὗ κατὰ   
τὸν μυχὸν ὁ Γάγγης ἐξίησι ποταμὸς, πέντε στόμασι τὴν 
ἐκβολὴν ποιούμενος, ὃν ἔφαμεν ὅριον εἶναι τῆς ἐντὸς 
Γάγγου Ἰνδικῆς καὶ τῆς ἐκτός. 



Marcianus Geogr., Periplus maris exteri 
Book 1, section 38, line 1

[Λείπει τὰ κατὰ μέρος] 
 Ἔστι δὲ τῆς ἐντὸς Γάγγου ποταμοῦ Ἰνδικῆς τὸ 
μὲν μῆκος, ᾗ μακροτάτη τυγχάνει, ἀπὸ τοῦ πέμπτου 
στόματος τοῦ Γάγγου ποταμοῦ λεγομένου Ἀντιβολὴ 
ἕως τοῦ Ναυστάθμου λιμένος, τοῦ ἐν τῷ Κάνθι κόλπῳ, 
σταδίων ͵͵α ͵ησϙʹ· τὸ δὲ πλάτος, ἀπὸ τοῦ ἀκρωτηρίου 
τοῦ καλουμένου Ἀφετηρίου ἕως τῶν πηγῶν τοῦ Γάγγου 
ποταμοῦ, σταδίων ͵͵α͵γ. 



Marcianus Geogr., Periplus maris exteri 
Book 1, section 39, line 7

                  Οἱ δὲ σύμπαντες ἀπὸ τοῦ Ναυστάθμου 
λιμένος ἕως τοῦ πέμπτου στόματος τοῦ Γάγγου ποτα-
μοῦ, ὃ καλεῖται Ἀντιβολὴ, τοῦ περίπλου παντὸς τῆς 
ἐντὸς Γάγγου ποταμοῦ Ἰνδικῆς στάδιοι <͵͵γ ͵εχϙεʹ>. 



Marcianus Geogr., Periplus maris exteri 
Book 1, section 40, line 1

Ἡ Ἰνδικὴ ἡ ἐκτὸς Γάγγου ποταμοῦ περιορίζε-
ται ἀπὸ μὲν ἄρκτων τοῖς μέρεσι τῆς Σκυθίας καὶ τῆς 
Σηρικῆς, ἀπὸ δὲ δύσεως αὐτῷ τῷ Γάγγῃ ποταμῷ, ἀπὸ 
δὲ ἀνατολῶν τοῖς Σίναις μέχρι τοῦ καλουμένου Μεγάλου 
κόλπου καὶ αὐτῷ τῷ κόλπῳ, ἀπὸ δὲ μεσημβρίας τῷ τε 
Ἰνδικῷ πελάγει καὶ μέρει τῆς Πρασώδους θαλάσσης, 
ἥτις ἀπὸ τῆς Μενουθιάδος νήσου ἀρξαμένη διατείνει 
κατὰ παράλληλον γραμμὴν μέχρι τῶν ἀντικειμένων 
μερῶν τῷ Μεγάλῳ κόλπῳ, καθὰ προειρήκαμεν. 



Marcianus Geogr., Periplus maris exteri 
Book 1, section 41, line 1

                                
 Ἔστι δὲ τῆς ἐκτὸς Γάγγου ποταμοῦ Ἰνδικῆς 
τὸ μὲν μῆκος, ᾗ μακροτάτη τυγχάνει, σταδίων ͵͵α ͵αχνʹ· 
τὸ δὲ πλάτος, ᾗ πλατυτάτη ἐστὶ, σταδίων ͵͵α͵θ. 



Marcianus Geogr., Periplus maris exteri 
Book 1, section 42, line 4

Οἱ πάντες ἀπὸ τοῦ [Μεγάλου] ἀκρωτηρίου 
μέχρι τοῦ πρὸς Σίνας ὁρίου τοῦ περίπλου τοῦ μέρους 
τοῦ Μεγάλου κόλπου τοῦ παρὰ τὴν ἐκτὸς Γάγγου Ἰν-  
δικὴν τυγχάνοντος στάδιοι ͵͵α͵βφνʹ. 



Marcianus Geogr., Periplus maris exteri 
Book 1, section 42, line 8

                                             Οἱ δὲ σύμπαντες 
ἀπὸ τοῦ πέμπτου στόματος τοῦ Γάγγου ποταμοῦ, ὃ 
καλεῖται Ἀντιβολὴ, μέχρι τῶν πρὸς τοὺς Σίνας τὸ ἔθ-
νος ὅρων τοῦ περίπλου παντὸς τῆς παραλίας τῆς ἐκτὸς 
Γάγγου Ἰνδικῆς στάδιοι ͵͵δ͵ετνʹ. 



Marcianus Geogr., Periplus maris exteri 
Book 1, section 43, line 3

Τὸ τῶν Σινῶν ἔθνος περιορίζεται ἀπὸ μὲν ἄρκ-
των μέρει τῆς Σηρικῆς, ἀπὸ δὲ δύσεως τῇ ἐκτὸς Γάγγου 
ποταμοῦ Ἰνδικῇ κατὰ τὸ προειρημένον ἐν τῷ Μεγάλῳ 
κόλπῳ ὅριον, ἀπὸ δὲ ἀνατολῶν ἀγνώστῳ γῇ, ἀπὸ δὲ 
μεσημβρίας τῇ τε μεσημβρινῇ θαλάττῃ καὶ τῇ μεσημ-
βρινῇ ἀγνώστῳ γῇ. 



Marcianus Geogr., Periplus maris exteri 
Book 1, section 44, line 6

                                               Δύο γὰρ 
ἀγνώστους ὑπονοεῖν χρὴ γᾶς, τήν τε παρὰ τὴν ἀνατο-
λὴν διήκουσαν, ᾗ παροικεῖν εἰρήκαμεν τοὺς Σίνας, 
καὶ τὴν παρὰ τὴν μεσημβρίαν, ἥτις διήκει παρὰ πᾶσαν 
τὴν Ἰνδικὴν θάλασσαν ἤτοι τὴν Πρασώδη καλουμένην, 
μέρος οὖσαν τῆς Ἰνδικῆς θαλάσσης, ὥστε συνάπτουσαν 
ἑκατέρας τὰς ἀγνώστους γᾶς καθάπερ τινὰ γωνίαν ἀπο-
τελεῖν περὶ τὸν τῶν Σινῶν κόλπον. 



Marcianus Geogr., Periplus maris exteri 
Book 1, section 48, line 2

Οἱ πάντες ἀπὸ τοῦ ἐν τῷ Μεγάλῳ κόλπῳ τῶν 
Σινῶν ὁρίου τοῦ ὄντος πρὸς τῇ Ἰνδικῇ τῇ ἐκτὸς Γάγγου 
ποταμοῦ ἐπὶ Κοττιάριος ποταμοῦ ἐκβολὰς τοῦ περίπλου 
παντὸς τῆς τῶν Σινῶν παραλίας στάδιοι ͵͵α͵βχνʹ. 



Marcianus Geogr., Periplus maris exteri 
Book 1, section 50, line 4

Καὶ τὸν μὲν ὅλον περίπλουν καὶ περιγραφὴν τῆς 
παραθαλασσίου χώρας τοῦ τῆς Ἀσίας μέρους, τοῦ τε 
Ἀραβίου κόλπου καὶ τῆς Ἐρυθρᾶς θαλάσσης καὶ ἔτι 
τοῦ Περσικοῦ κόλπου καὶ τοῦ Ἰνδικοῦ πελάγους, τοῦ-
τον ἔχειν τὸν τρόπον συμβέβηκε· τὸ δὲ σύμπαν ἐστὶ 
διάστημα, τῶν κόλπων ἁπάντων περιπλεομένων ἀπὸ 
τοῦ Αἰλανίτου μυχοῦ ἕως Κοττιάριος ποταμοῦ ἐκβο-
λῶν τοῦ ἐν τῷ κόλπῳ Σινῶν τυγχάνοντος, σταδίων 
͵͵͵ι ͵͵ε ͵γσϙεʹ. 



Marcianus Geogr., Periplus maris exteri 
Book 1, section 51, line 6

Ἀπὸ δὲ τῶν στενῶν τοῦ Ἀραβίου κόλπου τοῦ περί-
πλου τῆς τε Ἐρυθρᾶς θαλάσσης καὶ μέρους τοῦ Ἰνδι-
κοῦ πελάγους στάδιοι ͵͵β ͵αφλʹ. 



Marcianus Geogr., Periplus maris exteri 
Book 1, section 51, line 18

Ἀπὸ δὲ τῶν προρρηθέντων ὅρων τῆς Γεδρωσίας 
καὶ ἔτι τοῦ πρώτου καὶ δυσμικωτάτου στόματος τοῦ 
Ἰνδοῦ ποταμοῦ τοῦ λεγομένου Σάγαπα, μέχρι τοῦ 
πέμπτου στόματος τοῦ Γάγγου ποταμοῦ, ὃ καλεῖται 
Ἀντιβολὴ, τῆς παραλίας τῆς ἐντὸς Γάγγου ποταμοῦ 
Ἰνδικῆς στάδιοι ͵͵γ͵εχϙεʹ. 



Marcianus Geogr., Periplus maris exteri 
Book 1, section 51, line 26

Ἀπὸ δὲ τοῦ πέμπτου στόματος τοῦ Γάγγου ποταμοῦ, 
ὃ καλεῖται Ἀντιβολὴ, μέχρι τῶν ὅρων τῶν πρὸς τοὺς 
Σίνας, οἵτινες ἐν τῷ μεσαιτάτῳ τοῦ καλουμένου Μεγά-
λου κόλπου τυγχάνουσι, τῆς ἐκτὸς Γάγγου ποταμοῦ 
Ἰνδικῆς στάδιοι ͵͵δ ͵ετνʹ. 



Marcianus Geogr., Periplus maris exteri 
Book 1, section 52, line 3

Τέλος τοίνυν ἐνθάδε τοῦ πρώτου βιβλίου ποιησό-
μεθα, παντὸς μὲν τοῦ Ἀραβίου κόλπου, πάσης δὲ τῆς 
Ἐρυθρᾶς θαλάσσης, οὐ μὴν ἀλλὰ καὶ τοῦ Ἰνδικοῦ πε-
λάγους τῶν τε δεξιῶν μερῶν, ἔτι μὴν καὶ τῶν ἀριστε-
ρῶν, ὅσα τῇ τῶν ἀνθρώπων ἐπιμελείᾳ καὶ φιλομαθείᾳ 
γέγονεν ἐφικτὰ, μέχρι τῆς ἀγνώστου γῆς καθ' ἑκατέρας 
τὰς ἠπείρους, τῆς τε ἑῴας καὶ τῆς μεσημβρινῆς, τὸν 
περίπλουν ἀναγράψαντες. 



Marcianus Geogr., Periplus maris exteri 
Book 2, section 2, line 14

γράφου, ὃν νομίζομεν τῆς καθ' ἡμᾶς θαλάσσης ἐπιμε-
λέστατον ἐν τοῖς τῆς γεωγραφίας τὸν περίπλουν πε-
ποιῆσθαι· τῆς δὲ ἔξω θαλάσσης, ἥτις ὠκεανὸς παρὰ 
τῶν πλείστων καλεῖται, εἰ καὶ μετρίως τινῶν μερῶν 
ὁ προειρημένος ἐμνημόνευσεν Ἀρτεμίδωρος, ἀλλ' ὅμως 
τὸν ἀκριβέστατον ταύτης περίπλουν ἐκ τῆς τοῦ θειοτά-
του Πτολομαίου γεωγραφίας καὶ προσέτιγε τοῦ Πρω-
ταγόρου καὶ ἑτέρων παλαιῶν ἀνδρῶν ἐξελόντες, τοῦ 
μὲν Ἀραβίου κόλπου καὶ τῆς Ἐρυθρᾶς θαλάσσης ἑκα-
τέρων τῶν ἠπείρων καὶ ἔτι γε τοῦ Ἰνδικοῦ πελάγους 
παντὸς μέχρι τῆς ἑῴας καὶ τῆς ἀγνώστου γῆς μετὰ 
τῆς ἐνδεχομένης ἀκολουθίας ἐν τῷ προτέρῳ βιβλίῳ διεξ-
ήλθομεν· νυνὶ δὲ τὰ περὶ τὸν ἑσπέριον ὠκεανὸν ἐπε-
λευσόμεθα. 



Marcianus Geogr., Periplus maris exteri 
Book 2, section 46, line 6

                                                  Ὥσπερ δὲ 
ἐν τῷ προτέρῳ βιβλίῳ τῶν μὲν παρὰ τὴν Λιβύην δεξιῶν 
μερῶν τοῦ Ἀραβίου κόλπου καὶ τῆς Ἐρυθρᾶς θαλάς-
σης καὶ τοῦ Ἰνδικοῦ ὠκεανοῦ τοῦ πρὸς τὴν μεσημβρίαν 
ὁρῶντος τὸν περίπλουν ἐπὶ κεφαλαίων ἐποιησάμεθα, 
σαφηνείας ἕνεκα διὰ μακροῦ τὸν τῶν σταδίων ἀριθμὸν 
ἀποδόντες, τῶν δὲ παρὰ τὴν Ἀσίαν ἀριστερῶν ἁπάντων 
μερῶν μέχρι Σινῶν τοῦ ἔθνους καὶ τῆς ἀγνώστου γῆς 
ἀκριβῆ τὸν περίπλουν ἀνεγράψαμεν, τῶν διαστημάτων 
ἁπάντων τοὺς σταδίους σημάναντες· οὕτω κἀνταῦθα 
τῶν δεξιῶν μερῶν τοῦ ὠκεανοῦ τοῦ παρὰ τὴν Εὐρώπην 
ὄντος ἀπὸ τῶν Ἡρακλείων στηλῶν μέχρι τῆς ἀγνώστου 
γῆς καὶ τοῦ παρ' αὐτὴν περατουμένου Σαρματικοῦ 




Marcianus Geogr., Menippi periplus maris interni (epitome Marciani) (4003: 002)
“Geographi Graeci minores, vol. 1”, Ed. Müller, K.
Paris: Didot, 1855, Repr. 1965.
Section 2, line 14

                                              Οἱ γὰρ δὴ 
δοκοῦντες ταῦτα μετὰ λόγων ἐξητακέναι, Τιμοσθένης 
ὁ Ῥόδιός ἐστιν, ἀρχικυβερνήτης τοῦ δευτέρου Πτολε-
μαίου γεγονὼς, καὶ μετ' ἐκεῖνον Ἐρατοσθένης, ὃν Βῆτα 
ἐκάλεσαν οἱ τοῦ Μουσείου προστάντες, πρὸς δὲ τούτοις 
Πυθέας τε ὁ Μασσαλιώτης καὶ Ἰσίδωρος ὁ Χαρακη-
νὸς καὶ Σώσανδρος ὁ κυβερνήτης, τὰ κατὰ τὴν Ἰν-
δικὴν γράψας, Σιμμέας τε ὁ τῆς οἰκουμένης ἐνθεὶς τὸν 
περίπλουν· ἔτι μὴν Ἀπελλᾶς ὁ Κυρηναῖος καὶ Εὐθυ-
μένης ὁ Μασσαλιώτης καὶ Φιλέας ὁ Ἀθηναῖος καὶ 
Ἀνδροσθένης ὁ Θάσιος καὶ Κλέων ὁ Σικελιώτης, Εὔ-
δοξός τε ὁ Ῥόδιος καὶ Ἄννων ὁ Καρχηδόνιος, οἱ μὲν 
μερῶν τινων, οἱ δὲ τῆς ἐντὸς πάσης θαλάττης, οἱ δὲ 
τῆς ἐκτὸς περίπλουν ἀναγράψαντες· οὐ μὴν ἀλλὰ καὶ 
Σκύλαξ ὁ Καρυανδεὺς καὶ Βωτθαῖος· οὗτοι δὲ ἑκάτεροι   
διὰ τῶν ἡμερησίων πλῶν, οὐ διὰ τῶν σταδίων τὰ δια-
στήματα τῆς θαλάσσης ἐδήλωσαν. 

\end{greek}


\section{Aëtius}%???
%date uncertain
\blockquote[From Wikipedia\footnote{\url{http://en.wikipedia.org/wiki/Aëtius_of_Amida}}]{Aëtius of Amida (Greek: Ἀέτιος Ἀμιδηνός, Latin: Aëtius Amidenus) (fl. mid-5th century to mid-6th century) was a Byzantine physician and medical writer,[1] particularly distinguished by the extent of his erudition.[2] Historians are not agreed about his exact date. He is placed by some writers as early as the 4th century; but it is plain from his own work that he did not write till the very end of the 5th or the beginning of the 6th, as he refers not only to Patriarch Cyril of Alexandria, who died 444,[3] but also to Petrus archiater, who could be identified with the physician of Theodoric the Great,[4] whom he defines a contemporary. He is himself quoted by Alexander of Tralles,[5] who lived probably in the middle of the 6th century. He was probably a Christian,[citation needed] which may account perhaps for his being confounded with Aëtius of Antioch, a famous Arian who lived in the time of the Emperor Julian.



Aetius seems to be the first Greek medical writer among the Christians who gives any specimen of the spells and charms so much in vogue with the Egyptians, such as that of Saint Blaise in removing a bone which sticks in the throat,[8] and another in relation to a fistula.[9]

The division of his work Sixteen Books on Medicine (Βιβλία Ιατρικά Εκκαίδεκα) into four tetrabibli was not made by himself, but (as Fabricius observes) was the invention of some modern translator, as his way of quoting his own work is according to the numerical series of the books. Although his work does not contain much original matter, and is heavily indebted to Galen and Oribasius,[10] it is nevertheless one of the most valuable medical remains of antiquity, as being a very judicious compilation from the writings of many authors, many from the Alexandrian Library, whose works have been long since lost.[11]

In the manuscript for book 8.13, the word άκμή (acme) is written as άκνή, the origin of the modern word acne.[12]}

\begin{greek}

Aëtius Med., Iatricorum liber i (0718: 001)
“Aëtii Amideni libri medicinales i–iv”, Ed. Olivieri, A.
Leipzig: Teubner, 1935; Corpus medicorum Graecorum, vol. 8.1.
Chapter 131, line 39

                                                                      στάχους 
λι κιναμώμου λι καρυοφύλλων λι ἀμώμου λι σχινάνθων λι καλάμου 
ἀρωματικοῦ λι ξυλαλόης λι καρύων μυριστικῶν λι καχρύου λι ξανθο-
καρύων λι μάκερ λι γαλαγγὰ λι βαλσάμου λι καρποβαλσάμου λι ξυλο-
βαλσάμου λι μυροβαλάνου λι φύλλου ἰνδικοῦ λι κασίας λι ξηροκαρυο-
φύλλου λι πεπέρεως μακροῦ λι πεπέρεως λευκοῦ λι πεπέρεως κοινοῦ 
λι ἄσαρ χαλδαικοῦ λι κελτικοῦ λι θυμιάματος λι σμύρνης τρωγλίτιδος 
λι κόστου λι μόσχου λι ἄμπαρ λι γομφίτου λαδάνου λι τερεβίνθης λι 
οἴνου εὐώδους τὸ ἀρκοῦν. 



Aëtius Med., Iatricorum liber i 
Chapter 132, line 4

                                Ἐσκεύασα ταύτην ἐν Ἀλεξανδρείᾳ καί 
ἐστι πάνυ καλλίστη· ἀσπαλάθου 𐆄 ϛʹ ξυλοβαλσάμου 𐆄 <θ> κυπέρων 
𐆄 <δ> ἑλενίου 𐆄 ϛʹ ἴρεως 𐆄 ϛʹ καλάμου γρ <ιη> σχοίνου ἄνθους 𐆄 <β>ς 
στύρακος λιπαροῦ 𐆄 <β> κάρυα ἰνδικὰ <β> φύλλου γρ <ιη> ναρδοστάχυος 
𐆄 <α> καρυοφύλλου 𐆄 <α>ς ἀρνάβω 𐆄 <α>ς ἀμώμου 𐆄 γʹ κασίας 𐆄 <β> 
κόστου 𐆄 <α> σμύρνης 𐆄 <α> ὕπνου 𐆄 <γ> ξυλοκασίας 𐆄 <γ> ἐλαίου ξ̸ε9 <ι>. 



Aëtius Med., Iatricorum liber i 
Chapter 261, line 11

         τὸ δὲ ἰνδικὸν εἰς ἅπαντα χρησιμώτερον. 



Aëtius Med., Iatricorum liber i 
Chapter 265, line 1

Μάκερ φλοιός ἐστιν ἐκ τῆς Ἰνδικῆς κομιζόμενος· στύφει δὲ μετὰ 
βραχείας δριμύτητος. 



Aëtius Med., Iatricorum liber i 
Chapter 289, line 8

                                                   σύγκειται δὲ ἔκ τε 
στυφούσης αὐτάρκως οὐσίας καὶ δριμείας θερμῆς οὐ πολλῆς καί τινος 
ὑποπίκρου βραχείας· ὅθεν καὶ πρὸς ἧπαρ καὶ στόμαχον εὐλόγως ἁρ-
μόττει πινομένη τε καὶ ἔξωθεν ἐπιτιθεμένη καὶ οὖρα κινεῖ καὶ δήξεις 
ἰᾶται στομάχου καὶ τὰ κατὰ τὴν γαστέρα καὶ τὰ ἔντερα ῥεύματα ξη-
ραίνει καὶ πρὸς τούτοις ἔτι τὰ κατὰ τὴν κεφαλὴν καὶ τὸν θώρακα· 
ἰσχυροτέρα δὲ ἡ Ἰνδική, μελαντέρα τῆς Συριακῆς ὑπάρχουσα. 



Aëtius Med., Iatricorum liber ii (0718: 002)
“Aëtii Amideni libri medicinales i–iv”, Ed. Olivieri, A.
Leipzig: Teubner, 1935; Corpus medicorum Graecorum, vol. 8.1.
Chapter 30, line 1

                Ὁ δὲ ἱερακίτης καὶ ὁ Ἰνδικὸς τὰς αἱμορροίδας ἀναξη-
ραίνουσι περιαπτόμενοι δεξιῷ μηρῷ, ὧν καὶ ἡμεῖς ἐπειράθημεν. 



Aëtius Med., Iatricorum liber ii 
Chapter 30, line 9

                                                                                ὁ 
δὲ Ἰνδικὸς τὴν μὲν χρόαν ἐστὶν ὑπόπυρρος· ἀνέει δὲ τριβόμενος πορ-
φυροειδῆ χυλόν, οὔτε πυκνός ἐστιν οὔτε καρτερὸς καὶ δύναται μετ' 
οἴνου πινόμενος ἀκράτου αἱμοπτυικοὺς ὠφελεῖν. 



Aëtius Med., Iatricorum liber iii (0718: 003)
“Aëtii Amideni libri medicinales i–iv”, Ed. Olivieri, A.
Leipzig: Teubner, 1935; Corpus medicorum Graecorum, vol. 8.1.


Aëtius Med., Iatricorum liber iv (0718: 004)
“Aëtii Amideni libri medicinales i–iv”, Ed. Olivieri, A.
Leipzig: Teubner, 1935; Corpus medicorum Graecorum, vol. 8.1.
Chapter 10, line 1

                                 Λύκιον ἰνδικὸν μάλιστα, εἰ δὲ μή γε, 
τὸ παταρικὸν ἀνέσας μετὰ γάλακτος, ἐπίχριε τὰ ἄνω βλέφαρα σὺν ταῖς 
ὀφρύσιν· ἐνίοτε δὲ καὶ αὐτοῖς τοῖς ὀφθαλμοῖς ἔνσταζε ἐξυδαρώσας αὐτὸ 
τῷ γάλακτι. 






Aëtius Med., Iatricorum liber vi (0718: 006)
“Aëtii Amideni libri medicinales v–viii”, Ed. Olivieri, A.
Berlin: Akademie–Verlag, 1950; Corpus medicorum Graecorum, vol. 8.2.
Chapter 24, line 90

        σαγαπηνοῦ 𐅻 <β> ὀπίου Θηβαικοῦ 𐅻 <β> κρόκου γρʹ <α> ς λυκίου Ἰνδικοῦ 
γρʹ <α> ς σαρκὸς καρύων μὴ πάνυ παλαιῶν 𐅻 <μ>, ἀναλάμβανε καὶ δίδου   
𐅻 <α> σὺν ὕδατι ὀμβρίῳ θερμῷ εἰς νύκτα, μετὰ τὴν ἀκμὴν τοῦ πυρετοῦ, 
ὥστε ἕωθεν μὲν τῆς διὰ καρκίνων, εἰς ἑσπέραν δὲ τῆς διὰ καρύων. 



Aëtius Med., Iatricorum liber vi 
Chapter 24, line 117

                                                ὠφελίμως δὲ δίδοται καὶ τῆς ἀγρίας 
συκῆς ὁ φλοιὸς τῶν ῥάβδων ξηραινόμενος καὶ κοπτόμενος καὶ ποτιζό-
μενος σὺν ὕδατι καὶ τὸ λύκιον τὸ Ἰνδικὸν καὶ τὸ ἀψίνθιον καὶ τὸ 
σκόρδιον καὶ τὸ μικρὸν κενταύριον ἀριστολοχία ἀρτεμισία χαμαίδρυς   
βρυωνίας ῥίζα πόλιον λάσαρ καρκίνων ποταμίων ἀφέψημα ἀνήθου 
πολλοῦ συνεψομένου. 




Aëtius Med., Iatricorum liber vi 
Chapter 65, line 17

μεγάλας ἐχόντων τὰς τρίχας ἁρμόττειν, οἷόν ἐστι τὸ κεκαυμένον νί-
τρον καὶ ὁ ἀφρὸς τοῦ νίτρου καὶ τὸ ἀφρόνιτρον καὶ τὰ καλλάϊνα 
ὄστρακα καὶ τὰ τῶν κεραμίδων καὶ τὰ τοῦ κλιβάνου ὄστρακα κίσσηρις 
ἄκαυστός τε καὶ κεκαυμένη κηρύκων τε καὶ πορφυρῶν καὶ τῶν λοιπῶν 
ὀστρέων ὄστρακα κεκαυμένα (τὸ δὲ τῆς σηπίας καὶ ἄκαυστον) ἀλκυ-
όνιά τε καὶ στρουθίου ῥίζαι, οἵ τε ἐλλέβοροι καὶ ἡ τῆς βρυωνίας ῥίζα 
καὶ ἡ τοῦ δρακοντίου καὶ ἀριστολοχίας καὶ πάνακος ῥίζα καὶ κάχρυ 
καὶ τὰ τοιαῦτα· εὐώδη δὲ αὐτὰ ποιῆσαι βουλόμενος μίξεις κυπέρου 
καὶ μελιλώτου καὶ ῥόδων ξηρῶν καὶ σχοίνου ἄνθους καὶ ἴρεως καὶ 
μελισσοφύλλου, τοῖς δὲ πλουσίοις καὶ νάρδου Κελτικῆς καὶ Ἰνδικῆς καὶ 
ἀμώμου καὶ φύλλου <μαλαβάθρου> καὶ σμύρνης καὶ κόστου· λεπτύνει 
δὲ τρίχας καὶ τέφρα κληματίνη καὶ συκίνη καὶ τιθυμαλλίνη παρατρι-
βομένη λεία ἐν βαλανείῳ καὶ ἡ ἀπὸ τῶν κεκαυμένων γιγάρτων τέφρα 
στυπτηρία σχιστὴ κόπρος κυνεία ξηρά. 



Aëtius Med., Iatricorum liber vi 
Chapter 91, line 64

                                                                    καδμίας τρὶς 
κεκαυμένης καὶ οἴνῳ σβεσθείσης, μολυβδαίνης μολύβδου κεκαυμένου 
λιθαργύρου ναρδοστάχυος λιβάνου κυπαρίσσου φύλλων βράθυος κη-
κίδων λυκίου Ἰνδικοῦ κόμμεως ἀνὰ 𐅻 <δ> ψιμμυθίου γλαυκίου μίλτου 
φύλλου βαλαυστίων σχοίνου ἄνθους σιδίων ἀλόης ἀκακίας ἀνὰ Γρ <η> 
ῥοῶν φύλλων ῥόδων ἄνθους ἀνὰ Γρ <ϛ>, κόψας σήσας λειώσας ὄξει ἀνα-
λάμβανε τροχίσκους καὶ χρῶ μετ' οἴνου ἢ ἑψήματος. 



Aëtius Med., Iatricorum liber vii (0718: 007)
“Aëtii Amideni libri medicinales v–viii”, Ed. Olivieri, A.
Berlin: Akademie–Verlag, 1950; Corpus medicorum Graecorum, vol. 8.2.
Chapter 40, line 27

ῥυπτικὸν δέ τι καὶ τὸ καλούμενον Ἀρμένιον ἔχει, ᾧ χρῶνται οἱ ζω-
γράφοι, καὶ τὸ μέλαν τὸ Ἰνδικὸν καὶ διὰ τοῦτο τοῖς ἀφλεγμάντοις 
ἕλκεσιν ἀλύπως ὁμιλεῖ· μικτῆς δέ πώς ἐστι δυνάμεως ἡ ἀλόη, καθάπερ 
τὸ ῥόδον· ἔχει μὲν γάρ τι πικρόν, ᾧ ῥύπτειν πέφυκεν· ἔχει δέ τι καὶ 
στυπτικόν, ᾧ συνάγει τε καὶ συνουλοῖ τὰ ἕλκη. 



Aëtius Med., Iatricorum liber vii 
Chapter 41, line 56

                            κοχλιῶν κεκαυμένων 𐅻 <γ> χαλκοῦ κεκαυμένου 
𐅻 <δ> λεπίδος χαλκοῦ 𐅻 <ϛ> λεπίδος στομώματος σιδήρου 𐅻 <ιβ> ἰοῦ 𐅻 <ϛ> 
στυπτηρίας σχιστῆς 𐅻 <ϛ> λίθου σχιστοῦ 𐅻 <α> ἀλόης 𐅻 <α> ὀμφακίου ξηροῦ 
𐅻 <β> λυκίου Ἰνδικοῦ 𐅻 <δ> χαλκίτεως 𐅻 <γ> σμύρνης 𐅻 <γ> λιβάνου 𐅻 <γ>   
φλοιοῦ λιβάνου 𐅻 <α> κρόκου 𐅻 <β> κροκομάγματος 𐅻 <β> ναρδοστάχυος 
𐅻 <γ> κυτίνων 𐅻 <β> κόμμεως 𐅻 <η>, λείου ὕδατι καὶ ἀνάπλασσε κολλύρια 
καὶ χρῶ σὺν ὕδατι· καὶ ξηρίον δὲ εἰ βούλει ποιῆσαι, λεάνας τὸ κολλύ-
ριον χρῶ ξηρῷ. 



Aëtius Med., Iatricorum liber vii 
Chapter 80, line 26

                ἁπλᾶ δέ ἐστι ποιοῦντα πρὸς τοὺς πτίλους καὶ τὰ περι-
βεβρωμένα βλέφαρα ἀμόργη ἡψημένη, λύκιον ἰνδικόν, ἀρμένιον, ᾧ 
χρῶνται οἱ ζωγράφοι· σὺν ὕδατι γὰρ ἐγχριόμενον ἐκδαπανᾷ τὴν κακο-
χυμίαν καὶ αὔξει τὰς κατὰ φύσιν τρίχας· ἰὸς σιδήρου ἐπὶ πολλὰς 
ἡμέρας ἐν ἡλίῳ λειωθεὶς μετ' οἴνου καὶ σμύρνης καὶ ἀναπλασθεὶς εἰς 
κολλύριον· σπόδιον ἀναληφθὲν κρομμύου χυλῷ. 



Aëtius Med., Iatricorum liber vii 
Chapter 99, line 5

           ἀλόης λυκίου Ἰνδικοῦ ῥόδων χλωρῶν κρόκου ὀπίου σμύρνης, 
ἑκάστου τὸ ἴσον οἴνῳ λεάνας, ἀνάπλαττε τροχίσκους καὶ ξήραινε ἐν 
σκιᾷ. 



Aëtius Med., Iatricorum liber vii 
Chapter 101, line 47

                            αἰγὸς θηλείας χολὴν 𐅻 <η> λυκίου Ἰνδικοῦ 𐅻 <π> 
πεπέρεως 𐅻 <δ> περιστερεῶνος ὀρθοῦ χυλὸν 𐆄 <ιϛ> καὶ ξηροῦ 𐅻 <η> μέλιτος 
Ἀττικοῦ 𐆄 <ϛ>, κόψας σήσας λεπτοτάτῳ κοσκίνῳ τὸ πέπερι κἄπειτα λειο-
τριβήσας ἐν θυίᾳ ἐπὶ πολὺ ἐπίβαλλε τὸν ξηρὸν χυλὸν καὶ συλλείου· 
εἶτα τὸν ὑγρὸν χυλόν, ἔπειτα λύκιον καὶ ὅταν λεῖα γένηται, ἐπίβαλλε 
τὸ μέλι καὶ οὕτως τὴν χολὴν καὶ ἑνώσας ἀναλάμβανε καὶ χρῶ. 



Aëtius Med., Iatricorum liber vii 
Chapter 101, line 126

                        περιστερεῶνος ὀρθοῦ χυλὸν λυκίου Ἰνδικοῦ ἴσα 
τουτέστι ἀνὰ 𐆄 <α> ἑκάστου, μέλιτος τὸ ἴσον. 



Aëtius Med., Iatricorum liber vii 
Chapter 102, line 22

                                                 γλαυκίων 𐅻 <μη> σαρκο-
κόλλης 𐅻 <ιϛ> κρόκου 𐅻 <η> ὀπίου λυκίου Ἰνδικοῦ ἀνὰ 𐅻 <δ> ῥόδων χυλοῦ 
𐅻 <δ> μανδραγόρου χυλοῦ 𐅻 <δ> ὑοσκυάμου χυλοῦ κωνείου χυλοῦ ἀνὰ 
𐅻 <δ> τραγακάνθης 𐅻 <ιϛ> κόμμεως 𐅻 <η>· ἀναλαμβάνεται μελιλώτου 
ἀφεψήματι· συντίθεται δὲ τὸν τρόπον τοῦτον· λαβὼν μελιλώτων 
λίτραν μίαν καὶ ὕδατος ὀμβρίου ξ̸ <ϛ>, ἕψεται εἰς τρίτον καὶ διηθήσας 
τὸ ὑγρὸν πρὸς τὴν τοῦ φαρμάκου σκευασίαν. 



Aëtius Med., Iatricorum liber vii 
Chapter 102, line 28

            χυλοῦ πολυγόνου 𐆄 <ϛ> λυκίου Ἰνδικοῦ 𐅻 <ϛ> ἀλόης σμύρνης 
λιβάνου ἀνὰ 𐅻 <δ> ὀπίου 𐅻 <γ> ἀκακίας μελαίνης πρωτείας 𐅻 <ιβ> οἴνου 
παλαιοῦ καὶ εὐώδους τὸ ἀρκοῦν. 



Aëtius Med., Iatricorum liber vii 
Chapter 103, line 16

                τοιαῦτα δέ ἐστι κρόκος καὶ σμύρνα καὶ λύκιον Ἰνδικὸν καστό-
ριόν τε καὶ λιβανωτός, ἃ ἄνευ τοῦ στύφειν πέττει ἅμα καὶ διαφορεῖ. 



Aëtius Med., Iatricorum liber vii 
Chapter 104, line 93

                                            καδμίας χαλκοῦ ὀπίου ἀνὰ 
𐆄 <α> ς κρόκου 𐅻 <δ> ἀλόης 𐆄 <κ> λυκίου Ἰνδικοῦ 𐆄 <β> ἀκακίας 𐆄 <β> σμύρνης 
𐅻 <δ> τραγακάνθης 𐆄 <α> κόμμεως 𐆄 <ϛ>, ὕδωρ. 



Aëtius Med., Iatricorum liber vii 
Chapter 104, line 97

                                                        καδμίας 𐆄 <η> ναρ-
δοστάχυος 𐆄 <α> ς χαλκοῦ 𐆄 <α> ς κρόκου σμύρνης ὀπίου ἀνὰ 𐅻 <δ> μᾶλλον 
δὲ 𐆄 δύο ἀλόης 𐆄 <η> λυκίου Ἰνδικοῦ 𐆄 <β> ἀκακίας 𐆄 <κ> κόμμεως 𐆄 <κ>, 
ὕδωρ. 



Aëtius Med., Iatricorum liber vii 
Chapter 112, line 33

                                                           καδμίας 𐅻 <ιγ> χαλκοῦ   
κεκαυμένου 𐅻 <ε> λίθου σχιστοῦ 𐅻 <γ> λίθου αἱματίτου κασσίας ἀνὰ 𐅻 <α> 
πεπέρεως κόκκους <κα> κρόκου 𐅻 <β> ναρδοστάχυος 𐅻 <γ> φύλλου 𐅻 <α> ς 
λυκίου Ἰνδικοῦ ὀπίου ἀνὰ 𐅻 <α> ἀλόης ἀκακίας κιρρᾶς ἀνὰ 𐅻 <β> σμύρνης 
𐅻 <δ> κόμμεως 𐅻 <ϛ>, λείου οἴνῳ Χίῳ ἢ ἑτέρῳ αὐστηρῷ παλαιῷ, χρῶ ὡς 
ἐνεργεστάτῳ, καὶ πρὸς ὑποπύους καὶ τραχώματα ἐν ἀρχαῖς μετ' ὠοῦ, 
εἶτα ὕδατι· ἔστι δὲ ὑπὲρ τὴν ὑπόσχεσιν ἡ ἐνέργεια. 



Aëtius Med., Iatricorum liber vii 
Chapter 112, line 86

                          ἔχει δὲ οὕτως· κυτίνων, ῥοᾶς ἄνθους τοῦ 
ἐοικότος τῷ τῆς ἀνεμώνης ἄνθει 𐅻 <κε> ἀκακίας ξανθῆς 𐅻 <λε> γλαυκίων 
𐅻 <ιϛ> ἀλόης 𐅻 <ε> σμύρνης τρωγλίτιδος 𐅻 <γ> μέλανος Ἰνδικοῦ 𐅻 <ζ> φύλλου 
𐅻 <α> ὀμφακίου ξηροῦ 𐅻 <γ> κόμμεως 𐅻 <β> ὀποβαλσάμου, πρὸς τὰς ἀνα-
λήψεις 𐅻 <ε>, οἴνῳ αὐστηρῷ παλαιῷ λείου. 



Aëtius Med., Iatricorum liber vii 
Chapter 114, line 138

                καδμίας 𐅻 <ιϛ> ἰοῦ 𐅻 <δ> μέλανος Ἰνδικοῦ 𐅻 <ιϛ> πεπέρεως 
λευκοῦ 𐅻 <η> ὀποῦ Μηδικοῦ ὀποβαλσάμου ἀνὰ 𐅻 <δ> κόμμεως 𐅻 <ιβ>· ἡ 
χρῆσις μεθ' ὕδατος, ἡ κρᾶσις πρὸς τὰς διαθέσεις. 



Aëtius Med., Iatricorum liber vii 
Chapter 114, line 140

                                                                  Ἄλλο Ἰνδικὸν βασι-
λικὸν ἐπιγραφόμενον· ποιεῖ πρὸς ἀρχὰς ὑποχύσεως καὶ πᾶσαν ἀμ-
βλυωπίαν καὶ οὐλὰς ἀποσμήχει. 



Aëtius Med., Iatricorum liber vii 
Chapter 114, line 143

                                      καδμίας κεκαυμένης πεπλυμένης 𐅻 <ιϛ> 
μέλανος Ἰνδικοῦ 𐆄 <ϛ> ψιμμυθίου 𐆄 <δ> πεπέρεως λευκοῦ 𐆄 <ϛ> χολῆς ὑαίνης 
τὸ ὅλον, ὅσον ἔχει, σκάρων ἰχθύων χολὰς <ι> περδίκων χολὰς <δ> ὀπίου 
𐆄 <α> ὀποβαλσάμου ὀποπάνακος σαγαπηνοῦ ἀνὰ 𐆄 <β> κόμμεως λ̸ <α>. 



Aëtius Med., Iatricorum liber vii 
Chapter 117, line 5

           καδμίας 𐅻 <ιβ> χαλκοῦ κεκαυμένου 𐅻 <η> στίμμεως 𐅻 <η> ψιμ-  
μυθίου καστορίου νάρδου Συριακῆς ἀνὰ 𐅻 <δ> ἀλόης 𐅻 <α> ς φύλλου 
κρόκου χαλκίτεως ὀπτῆς χαλκάνθου λυκίου Ἰνδικοῦ ὀπίου ἀνὰ 𐅻 <α> 
σμύρνης 𐅻 <β> ἀκακίας 𐅻 <κ> κόμμεως <κ>, ὕδωρ. 



Aëtius Med., Iatricorum liber vii 
Chapter 117, line 19

                             καδμίας πεπλυμένης στίμμεως πεπλυμένου 
ἀκακίας κόμμεως ἀνὰ 𐆄 <β> ῥόδων ξηρῶν κεκαθαρμένων 𐅻 <ιβ> χαλκοῦ 
κεκαυμένου πεπλυμένου σμύρνης τρωγλίτιδος ἀνὰ 𐅻 <η> καστορίου 
λυκίου Ἰνδικοῦ κρόκου φύλλου ναρδοστάχυος χαλκίτεως ὀπτῆς ψιμμυ-
θίου γλαυκίων ἐρείκης καρποῦ ὀπίου κηκίδων ὀμφακίνων ἀνὰ 𐅻 <β>, 
ὕδωρ· χρῶ ἐν ἀρχαῖς μὲν μετ' ὠοῦ ἐγχυματίζων, ἐν παρακμαῖς δὲ 
ὕδατι ἐγχρίων καθ' ὑποβολήν. 



Aëtius Med., Iatricorum liber vii 
Chapter 117, line 32

                                                                       ἔχει δὲ 
οὕτως· καδμίας πεπλυμένης 𐅻 <ιϛ> φιμμυθίου πεπλυμένου καστορίου 
ἀλόης ἀνὰ 𐅻 <ϛ> νάρδου Ἰνδικῆς κασσίας ἀνὰ 𐅻 <δ> στίμμεως πεπλυμένου 
𐅻 <μ> σμύρνης 𐅻 <θ> λεπίδος χαλκοῦ 𐅻 <ε> χαλκοῦ κεκαυμένου ὀπίου πεπλυ-
μένου ἀνὰ 𐅻 <ιϛ> ῥόδων 𐅻 <κ> λυκίου Ἰνδικοῦ ὀπίου ἀνὰ 𐅻 <γ> λίθου 
σχιστοῦ πεπλυμένου 𐅻 <δ> ς κρόκου 𐅻 <ϛ> μολύβδου κεκαυμένου πεπλυ-
μένου 𐅻 <κ> ἀκακίας κόμμεως ἀνὰ 𐅻 <μ>, ὕδωρ. 



Aëtius Med., Iatricorum liber vii 
Chapter 117, line 43

Κολλύριον τὸ ἰνδάριον. 



Aëtius Med., Iatricorum liber vii 
Chapter 117, line 50

                                                   καδμίας πεπλυμένης 𐅻 <ιϛ> 
χαλκοῦ κεκαυμένου πεπλυμένου 𐅻 <η> στίμμεως πεπλυμένου 𐅻 <ιϛ> κρόκου 
𐅻 <γ> καστορίου 𐅻 <δ> ἰοῦ 𐅻 <α> ἀλόης 𐅻 <δ> σμύρνης τρωγλίτιδος 𐅻 <γ> 
λυκίου Ἰνδικοῦ ὀπίου ἀνὰ 𐅻 <β> ἀκακίας κόμμεως ἀνὰ 𐅻 <κδ> ὕδωρ· ἡ 
χρῆσις δι' ὠοῦ, ἡ κρᾶσις παχυτέρα. 



Aëtius Med., Iatricorum liber vii 
Chapter 117, line 62

                     ἔχει δὲ οὕτως· καδμίας 𐅻 <κ> στίμμεως πεπλυμένου 
𐅻 <ιε> ναρδοστάχυος 𐅻 <γ> σμύρνης τρωγλίτιδος καστορίου χαλκοῦ κεκαυ-  
μένου ἰοῦ ἀνὰ 𐅻 <β> κρόκου 𐅻 <α> λεπίδος σιδήρου στομώματος 𐅻 <δ> 
ψιμμυθίου τὸ ἴσον πεπέρεως λυκίου Ἰνδικοῦ ὀπίου λιβάνου ὀποβαλ-
σάμου κασσίας ἀνὰ 𐅻 <β> λίθου σχιστοῦ 𐅻 <δ> ἰοῦ σκώληκος 𐅻 <γ> χαλ-
κάνθου 𐅻 <β> μαράθρου χυλοῦ χολῆς αἰγείας ἀνὰ 𐅻 <δ> κόμμεως 𐅻 <κ>, 
οἴνῳ λείου Ἀμηναίῳ ἢ Φαλερίνῳ ἢ ἑτέρῳ αὐστηρῷ παλαιῷ εὐώδει. 



Aëtius Med., Iatricorum liber vii 
Chapter 117, line 68

                                   καδμίας κεκαυμένης πεπλυμένης χαλκοῦ 
πεπλυμένου ἀνὰ 𐆄 <η> στίμμεως πεπλυμένου 𐆄 <κε> ῥόδων ξηρῶν ψιμμυ-
θίου χαλκίτεως κηκίδων ἰοῦ κρόκου λυκίου Ἰνδικοῦ ἀλόης ναρδοστάχυος 
σμύρνης τρωγλίτιδος ἐβένου ῥινήματος καστορίου μέλανος Ἰνδικοῦ 
λιβάνου ὀπίου σαρκοκόλλης πομφόλυγος ἀνὰ 𐆄 <α> [κέρατος ἐλαφείου 
πεπλυμένου καὶ κεκαυμένου 𐆄 <α>] ἀκακίας κόμμεως ἀνὰ 𐆄 <κε>, ὕδωρ. 



Aëtius Med., Iatricorum liber vii 
Chapter 117, line 88

                                                 λαμβάνει δὲ καδμίας 𐅻 <κδ> χαλκοῦ 
κεκαυμένου 𐅻 <ιβ> στίμμεως 𐅻 <μ> ψιμμυθίου 𐅻 <η> χαλκίτεως ὀπτῆς 𐅻 <δ> 
μίσυος ὀπτοῦ 𐅻 <δ> ἐβένου ῥινήματος 𐅻 <β> νάρδου Κελτικῆς φύλλου 
ναρδοστάχυος κρόκου καστορίου λυκίου Ἰνδικοῦ ῥόδων ξηρῶν ἀνὰ 𐅻 <δ> 
ἀλόης σμύρνης ἀνὰ 𐅻 <η> ὀμφακίου ὀπίου ἀνὰ 𐅻 <β> ἀκακίας κόμμεως 
σκώληκος ἀνὰ 𐅻 <μ>. 



Aëtius Med., Iatricorum liber viii (0718: 008)
“Aëtii Amideni libri medicinales v–viii”, Ed. Olivieri, A.
Berlin: Akademie–Verlag, 1950; Corpus medicorum Graecorum, vol. 8.2.
Chapter 25, line 30

                                                  τὰ δὲ γεγυμνωμένα οὖλα σαρκοῖ 
τοῦτο· ῥόδων σπέρματος μὴ παλαιοῦ μέρη <β> καλάμου ἀρωματικοῦ ἢ 
Ἰνδικοῦ μέρος <α>, ξηρῷ λείῳ προσάπτου. 



Aëtius Med., Iatricorum liber viii 
Chapter 31, line 40

                                                                             ἐνίοτε 
δὲ τῶν λευκῶν ἐλαιῶν τεθλασμένων καὶ ἐκπιεσθέντων τὸν χυλὸν μετὰ 
πηγάνου χυλοῦ καὶ μέλιτος μίξαντες παρατρίβομεν τοὺς ὀδόντας, 
ἔπειτα παραπάσσομεν νάρδῳ Ἰνδικῇ μετὰ στυπτηρίας στρογγύλης ἴσοις 
καὶ ἅλατος τὸ διπλοῦν. 



Aëtius Med., Iatricorum liber viii 
Chapter 37, line 48

                                                                ἀστραγάλους 
προβατίους καύσας, λείοις χρῶ· καλῶς ποιοῦσι καὶ κήρυκες πληρω-
θέντες ἁλσὶ καὶ καυθέντες καὶ κίσσηρις ὀπτὴ οἴνῳ σβεσθεῖσα, σηπίας 
ὄστρακα καέντα κεκαυμένοι μύακες κοχλίας χερσαῖος καεὶς σὺν μέλιτι·   
ἑκάστῳ δὲ τούτων εὐωδίας χάριν μίγνυται ἴρεως Ἰλλυρικῆς βραχὺ ἢ 
σχίνου ἄνθους ἢ καλάμου Ἰνδικοῦ ἢ ναρδοστάχυος. 



Aëtius Med., Iatricorum liber viii 
Chapter 47, line 15

                              εἰ δὲ πολυτελέστερον ἐθέλεις ποιῆσαι τὸ φάρ-
μακον ἢ πλουσίοις σκευάζων ἐμβάλλεις κασσίας φλοιὸν νάρδον Ἰνδικὴν 
ἢ Κελτικὴν ἢ μαλαβάθρου φύλλα· μετὰ δὲ τὴν ἀρχήν, ὅταν στῇ τὸ 
ἐπιρρέον, μόνον ἀρκεῖ τὸ διὰ μόρων τοῦ κρόκου τι καὶ τῆς σμύρνης 
προσειληφὸς εἰς τὸ πέψαι τὴν φλεγμονήν. 



Aëtius Med., Iatricorum liber ix (0718: 009)
“”Ἀετίου Ἀμιδηνοῦ λόγος ἔνατος””, Ed. Zervos, S., 1911; Athena 23.
Chapter 31, line 159

                               Ἔστι δὲ οὐκ ὀλίγον κἀν τῇ ἰνδικῇ νάρδῳ 
τὸ πεπτικὸν τῶν ψυχρῶν διαθέσεων· ὥσπερ δὲ τῶν ἀρωματικῶν τὴν 
νάρδον ἐνέβαλλεν ὁ Φίλων, οὕτως οἱ μετ' αὐτὸν ἄλλος ἄλλο προσέ-  
θεσαν, σχίνου ἄνθος, κασίαν, ἄμωμόν τε καὶ κόστον· ὥσπερ δ' αὖ 
πάλιν ὁ Φίλων τὸν κρόκον ἐνέβαλλεν πεπτικὸν φάρμακον καὶ χυμῶν 
καὶ διαθέσεων εὐπέπτων, οὕτως ἄλλοι σμύρνης τε καὶ καστορίου ἔμι-
ξαν· οἱ πλεῖστοι δὲ αὐτῶν καλῶς ποιοῦντες, καὶ τὰ συνήθη ἡμῖν 
σπέρματα [παρέμιξαν], λέγω δὴ σελίνου τὸ σπέρμα καὶ κυμίνου, ἀνή-
σου τε καὶ δαύκου καὶ πετροσελίνου καὶ ὅσα τοιαῦτα, ὥστε παρα-
μυθήσασθαι τὴν ἀηδίαν τῶν πικρῶν φαρμάκων, εἰς ἀνάδοσίν τε καὶ 




Aëtius Med., Iatricorum liber xi (0718: 011)
“Oeuvres de Rufus d'Éphèse”, Ed. Daremberg, C., Ruelle, C.É.
Paris: Imprimerie Nationale, 1879, Repr. 1963.
Chapter 11, line 63

Τοῦ Πρεσβύτου τοῦ Ἰνδοῦ πρὸς λιθιῶντας, φασὶ δέ τινες, καὶ τῶν ἔξωθεν λίθων 
δύνασθαι θρυβεῖν, ὡς τὸ πρὸ αὐτοῦ· Ἀκόρου, φοῦ, ὑπερικοῦ ἀνὰ 𐅻 ϛʹ, πράσου 
σπέρματος 𐅻 ιβʹ, ναρδουστάχυος 𐅻 ιʹ, κασίας, λινοσπέρμου, κυπέρου ἀνὰ 𐅻 κεʹ· 
μέλιτι ἀναλάμβανε· ἡ δόσις κυάμου μέγεθος. 



Aëtius Med., Iatricorum liber xi 
Chapter 29, line 94

                                                         Παραλαμβανέσθω δὲ καὶ τὰ διὰ 
στόματος διδόμενα, τῆς μὲν δυσουρίας ἐπειγούσης, μήκωνος λευκῆς πεφωγμένης 
σπέρμα λεῖον· ἐμπάσσεται δὲ ὅσον 𐅻 αʹ εἰς κυάθους δʹ ἀφεψήματος σχοίνου ἄνθους ἢ 
καλάμου ἰνδικοῦ, ἢ γλυκυῤῥίζης· βιαιότερα δέ ἐστι τούτων μῆον, φοῦ, ἄκορον, δαῦ-
κος. 



Aëtius Med., Iatricorum liber xii (0718: 012)
“Ἀετίου λόγος δωδέκατος”, Ed. Kostomiris, G.A.
Paris: Klincksieck, 1892.
Chapter 33, line 18

                                                               οἱ δὲ βάρβαροι ἰνδοὶ 
κόπτοντες καὶ σήθοντες τὰ ξηρὰ φύλλα τῆς κύπρου ἀποτιθέασιν, ἐπὶ δὲ 
τῆς χρείας ὕδατι ζέοντι φυρῶντες καὶ προσραίνοντες ὄξους ὀλίγον ἢ ῥοῦ 
μαγειρικοῦ ἀφέψημα ἐπιτιθέασι τῷ φλεγμαίνοντι τόπῳ, ἄνωθεν ἐπιτι-
θέντες φύλλα κίκεως ἢ καρπάσου, εἰ δὲ μή, κράμβης. 



Aëtius Med., Iatricorum liber xii 
Chapter 53, line 35

        Ἐλαίου κικίνου, ἐλαίου τηλίνου, μανδραγόρου χυλοῦ ἀνὰ 𐅻 λβʹ, 
ἀρτεμισίας χυλοῦ, λαπάθου χυλοῦ ἀνὰ 𐅻 κδʹ, πιτυΐνης, τερεβινθίνης, 
στέατος χηνείου ἀνὰ 𐅻 μηʹ, βάτου χυλοῦ 𐅻 ιϛʹ, βουτύρου νεαροῦ 𐅻 μηʹ, 
λυκίου ἰνδικοῦ 𐅻 κδʹ, αἵματος τραγείου ξηροῦ 𐅻 ιϛʹ, κιττοῦ δακρύου 
𐅻 λβʹ, χαλβάνης 𐅻 μηʹ, ὀποπάνακος 𐅻 λβʹ. 



Aëtius Med., Iatricorum liber xii 
Chapter 68, line 44

      Παραδόξως δὲ ποιεῖ ἐπ' αὐτῶν καὶ τὸ μέγα ξηρίον, ὃ ἀσκληπιὸν 
ὀνομάζουσι καὶ τὸ ἰνδὸν ξηρίον, ἐμπασσόμενον αὐτοῖς τοῖς ὀδυνωμένοις 
τόποις ἐν τῷ λουτρῷ· πάνυ καλόν. 



Aëtius Med., Iatricorum liber xv (0718: 015)
“”Ἀετίου Ἀμιδηνοῦ λόγος δέκατος πέμπτος””, Ed. Zervos, S., 1909; Athena 21.
Chapter 13, line 141

Ἑλκύσματος, ἐλαίου γλυκέος ἀνὰ δραχμὰς β, πίσσης, κηροῦ, κο-
λοφωνίας ἀνὰ οὐγγίας, ιστ, συμφύτου ῥίζης κεκομμένης λεπτοτάτης   
καὶ σεσησμένης οὐγγίας β, αἵματος δρακοντίου οὐγγίας β, ὕδατος 
θαλασσίου οὐγγίας κ. Ἕψε τὴν λιθάργυρον, τὸ ἔλαιον καὶ τὸ θα-
λάσσιον ὕδωρ μέχρις ἀμολύντου, εἶτα τήξας τὰ τηκτὰ καὶ διηθήσας 
ἐπίβαλλε τοῖς ἑψηθεῖσι· καὶ ἄρας ἀπὸ τοῦ πυρὸς ἐπίπασσε τὸ δρα-
κόντιον καὶ σύμφυτον, καὶ ἑνώσας καὶ ψύξας καὶ μαλάξας ἀνάπλας-
σε μαζία καὶ χρῶ· συνάγεται δὲ τὸ δρακόντιον αἷμα ἐν τῇ Ἰνδικῇ 
χώρᾳ ἐκ τῆς δρακοντίου βοτάνης. 



Aëtius Med., Iatricorum liber xvi (0718: 016)
“Gynaekologie des Aëtios”, Ed. Zervos, S.
Leipzig: Fock, 1901.
Chapter 118, line 8

                             Δεῖ οὖν ἐπὶ τούτων τοῖς στύφουσι 
προσαντλεῖν καὶ προσθέτοις χρῆσθαι τοῖς ἐπὶ δακτυλίου προει-
ρημένοις, μάλιστα δὲ σιδίοις μετὰ βουτύρου ἢ λυκίῳ ἰνδικῷ μετὰ 
ὑσσώπου, καὶ τοῖς ὁμοίοις ἐπὶ τῶν <ἐν ἕδρᾳ παθῶν> προγεγραμ-
μένοις. 



Aëtius Med., Iatricorum liber xvi 
Chapter 121, line 8

ἐπὶ τούτων ἁρμόζει κάθαρσις δι' <ὀνείου γάλακτος>, καὶ ἔμετοι 
ἀπὸ δείπνου, ἀφιδρώσεις ἐν βαλανείῳ καὶ ἐμβάσεις εἰς θερμὸν ὕδωρ, 
καὶ κλυσμὸς δι' οἴνου θερμοῦ μετὰ νίτρου, ἢ τρυγὸς οἴνου κε-
καυμένης· μετὰ δὲ ταῦτα θερμῷ ὕδατι καταχρίειν τοὺς τόπους, ἢ 
λυκίῳ ἰνδικῷ, ἢ ἀμόργῃ, ἢ βουτύρῳ μετὰ θείου, ἢ κηρωτῇ μυρσι-
νίνῃ μετὰ λιθαργύρου, ἢ στυπτηρίᾳ μετὰ μέλιτος, ἢ σιδίοις 
μετὰ μυρσίνης καὶ μέλιτος. 



Aëtius Med., Iatricorum liber xvi 
Chapter 126, line 7

   ὀποβαλσάμου γοα· ἐλαίου ἰνδικοῦ ἢ ἑτέρου γογ. 



Aëtius Med., Iatricorum liber xvi 
Chapter 142, line 7

                                                        α. καλάμου ἰν-
δικοῦ γοστ, ἤτοι οὐγ. 



Aëtius Med., Iatricorum liber xvi 
Chapter 142, line 7

                              κάρυα ἰνδικὰ γ. σανδαράχης γράμματα 
ιστ. 



Aëtius Med., Iatricorum liber xvi 
Chapter 150, line 1


  
ΘΥΜΙΑΜΑ ΡΟΔΑΤΟΝ ΤΟΥ ΕΜΒΟΛΑΡΧΟΥ.


       
 Κασίας, σμύρνης, βδελλίου, ἀρναβῶ, καλάμου ἰνδικοῦ, σαρούα, 
καρποβαλσάμου, λαδάνου λιπαροῦ, ὕπνου, φύλλων, ἀνὰ γογ. 



Aëtius Med., Iatricorum liber xvi 
Chapter 151, line 2

καλάμου ἰνδικοῦ, ναρδοστάχυος, ὀνύχων μεγάλων, βδελλίου, καρ-
ποβαλσάμου, κρόκου, κασίας, ἀνὰ γογ. 

\end{greek}




\section{Anonymi De Astrologia Dialogus}%???
\blockquote[From Wikipedia\footnote{\url{}}]{}
\begin{greek}
%http://archive.org/details/anonymichristia00viergoog
%date?

Anonymi De Astrologia Dialogus Astrol., De astrologia dialogus (= Hermippus) (fort. auctore Joanne Catrario) (4374: 001)
“Anonymi christiani Hermippus De astrologia dialogus”, Ed. Kroll, W., Viereck, P.
Leipzig: Teubner, 1895; Bibliotheca scriptorum Graecorum et Romanorum Teubneriana.
Page 51, line 20

                                   οὔτε γὰρ καρκίνον μὲν 
λέγουσιν Ἀρμενίας καὶ Ἀφρικῆς κυριεύειν οὔτ' αὖ 
αἰγοκέρωτα Συρίας καὶ Ἰνδικῆς, ἔτι δὲ καὶ Θρᾴκης 
εὖ φρονῶν ἄν τις πιστεύσειεν. 



Anonymi De Astrologia Dialogus Astrol., De astrologia dialogus (= Hermippus) (fort. auctore Joanne Catrario) 
Page 51, line 26

                                                       ἢ πῶς 
τὸ αὐτὸ καὶ ἓν ζῴδιον Ἰνδικῆς ἅμα καὶ Θρᾴκης ἐφέξει 
τὴν ἐφορείαν; 



Anonymi De Astrologia Dialogus Astrol., De astrologia dialogus (= Hermippus) (fort. auctore Joanne Catrario) 
Page 52, line 3

                 οὔτε γὰρ ταὐτοῦ κλίματός εἰσιν οὔτε   
ὑπὸ τὸν αὐτὸν παράλληλον πίπτουσιν, εἴ γε Θρᾴκη 
μὲν τοῦ ἰσημερινοῦ τὸ μέσον ἀπόστημα πέντε καὶ 
τεσσαράκοντα μοίρας ἀφίσταται, Ἰνδικὴ δὲ τὸ μέγιστον 
πεντεκαίδεκα. 

\end{greek}



\section{Timotheus of Gaza}
\blockquote[From Wikipedia\footnote{\url{http://en.wikipedia.org/wiki/Timotheus_of_Gaza}}]{Timotheus of Gaza (sometimes referred to as Timothy of Gaza) was a Greek grammarian active during the reign of Anastasius, i.e. 491-518. He is the author of a book on animals[1] which may have been the source of the Arabic Nu'ut al-Hayawan.[2]}
\begin{greek}

Timotheus Gramm., Excerpta ex libris de animalibus (e cod. Paris. gr. 2422) (2449: 003)
“”Excerpta ex Timothei Gazaei libris de animalibus””, Ed. Haupt, M., 1869; Hermes 3.
Section 5, line 28

                                         ἐὰν δὲ ἄρσην ᾖ, ἡ δὲ 
κύων θήλεια, τίκτεται Λακωνικὸς κύων, ὥσπερ συγγινομέ-
νων κυνὸς καὶ τίγριδος τίκτεται ὁ Ἰνδικὸς κύων. 



Timotheus Gramm., Excerpta ex libris de animalibus (e cod. Paris. gr. 2422) 
Section 14, line 3

ὅτι ὁ Ἰνδικὸς πάνθηρ μύρου ὄζων διὰ τῆς εὐωδίας τὰ 
θηρία ἐφελκόμενος ἐπὶ τὸν ἴδιον ἄγει φωλεὸν καὶ κατεσθίει. 



Timotheus Gramm., Excerpta ex libris de animalibus (e cod. Paris. gr. 2422) 
Section 24, line 2

                                   ὅτι ἡ καμηλοπάρ-
δαλις ζῷόν ἐστιν Ἰνδικόν· γίνεται δὲ ἀπὸ ἐπιμιξίας ζῴων 
ἑτερογενῶν. 



Timotheus Gramm., Excerpta ex libris de animalibus (e cod. Paris. gr. 2422) 
Section 24, line 4

ὅτι διὰ Γάζης παρῆλθέ τις ἀνὴρ ἀπὸ τῶν Ἰνδικῶν, 
Ἀελίσιος δὲ τὸ γένος, ἄγων δύο καμηλοπαρδάλεις καὶ ἐλέ-
φαντα τῷ βασιλεῖ Ἀναστασίῳ. 



Timotheus Gramm., Excerpta ex libris de animalibus (e cod. Paris. gr. 2422) 
Section 24, line 8

                                   τοῦτο ἐθεάθη καὶ ἐφ' 
ἡμῶν· τῷ γὰρ βασιλεῖ τῷ Μονομάχῳ καὶ ἄμφω ταυτὶ τὰ 
ζῷα προσαχθέντα ἐξ Ἰνδίας ὡς θαῦμα ἐπὶ τοῦ τῆς Κων-
σταντινουπόλεως θεάτρου ἑκάστοτε τῷ λαῷ ἐπεδείκνυντο. 



Timotheus Gramm., Excerpta ex libris de animalibus (e cod. Paris. gr. 2422) 
Section 25, line 15

             τούτους δὲ γοητεύοντες οἱ Ἰνδοὶ κοιμίζουσι 
καὶ ἀναιροῦσι καὶ ἀφαιροῦνται τοὺς λίθους· πολλάκις δὲ 
καὶ ὑπ' αὐτῶν εἰς τοὺς φωλεοὺς ἕλκονται οἱ θηρευταὶ καὶ 
ἀπόλλυνται. 



Timotheus Gramm., Excerpta ex libris de animalibus (e cod. Paris. gr. 2422) 
Section 25, line 19

ὅτι οἱ Ἰνδοὶ ἐσθίοντες τὴν τῶν δρακόντων καρδίαν ἢ 
τὸ ἧπαρ νοοῦσι τί τὰ ἄλογα ζῷα φθέγγονται. 



Timotheus Gramm., Excerpta ex libris de animalibus (e cod. Paris. gr. 2422) 
Section 31, line 14

ὅτι εἰσὶ σύες μονώνυχες καὶ ὄνος Ἰνδικὸς μονώνυξ καὶ 
κερατώδης· τὸ δὲ κέρας αὐτοῦ ποιεῖ θεραπείαν καὶ μόνῳ 
ἀποφέρεται τῷ βασιλεῖ. 



Timotheus Gramm., Excerpta ex libris de animalibus (e cod. Paris. gr. 2422) 
Section 32, line 6

ὅτι ἐν μεσημβρίᾳ πλέον τῶν ἄλλων ζῴων ὁδεύουσι καὶ 
πρὸς ἀστέρας ἐν νυκτὶ τὰς ὁδοὺς γινώσκουσιν· ὅθεν ἐπ' 
αὐταῖς οἱ Ἰνδοὶ τὴν χρυσῖτιν κόνιν τῶν Ἰνδικῶν μυρμήκων 
κλέπτουσι πρὸς ἀνατολὰς ὁδεύοντες. 



Timotheus Gramm., Excerpta ex libris de animalibus (e cod. Paris. gr. 2422) 
Section 32, line 9

ὅτι τῶν μυρμήκων ἐν τῷ καύματι ἐν τοῖς φωλεοῖς δια-
τριβόντων κλέπτουσιν οἱ Ἰνδοὶ τὴν αὐτῶν χρυσόκονιν. 



Timotheus Gramm., Excerpta ex libris de animalibus (e cod. Paris. gr. 2422) 
Section 45, line 8

ὅτι παρὰ τοῖς Ἰνδοῖς βόες λέγονται, ἐρχόμενοι δὲ παρὰ 
τὸν Νεῖλον ῥινοκέρωτες. 



Timotheus Gramm., Excerpta ex libris de animalibus (e cod. Paris. gr. 2422) 
Section 51, line 13

ὅτι οὗτοι παρὰ τοῖς Ἰνδοῖς τρυγῶσι τὰ πεπέρια, τέχνῃ 
ἀπατώμενοι καὶ μιμήσει. 

\end{greek}

\section{Salaminius Hermias Sozomenus}
\blockquote[From Wikipedia\footnote{\url{http://en.wikipedia.org/wiki/Salaminius_Hermias_Sozomenus}}]{Salminius Hermias Sozomenus[1] (Σωζομενός) (c. 400 – c. 450) was a historian of the Christian Church.

Sozomen's second work continues approximately where his first work left off. He wrote it in Constantinople, around the years 440 to 443 and dedicated it to Emperor Theodosius II.

The work is structured into nine books, roughly arranged along the reigns of Roman Emperors:

    Book I: from the conversion of Constantine I until the Council of Nicea (312-325)
    Book II: from the Council of Nicea to Constantine's death (325-337)
    Book III: from the death of Constantine I to the death of Constans I (337-350)
    Book IV: from the death of Constans I to the death of Constantius II (350-361)
    Book V: from the death of Constantius I to the death of Julian the Apostate (361-363)
    Book VI: from the death of Julian to the death of Valens (363-375)
    Book VII: from the death of Valens to the death of Theodosius I (375-395)
    Book VIII: from the death of Theodosius I to the death of Arcadius (375-408).
    Book IX: from the death of Arcadius to the accession of Valentinian III (408-25).

Book IX is incomplete. In his dedication of the work, he states that he intended cover up to the 17th consulate of Theodosius II, that is, to 439. The extant history ends about 425. Scholars disagree on why the end is missing. Albert Guldenpenning supposed that Sozomen himself suppressed the end of his work because in it he mentioned the Empress Aelia Eudocia, who later fell into disgrace through her supposed adultery. However, it appears that Nicephorus, Theophanes, and Theodorus Lector did read the end of Sozomen's work, according to their own histories later. Therefore most scholars believe that the work did actually come down to that year, and that consequently it has reached us only in a damaged condition.}
\begin{greek}


Salaminius Hermias Sozomenus Scr. Eccl., Historia ecclesiastica (2048: 001)
“Sozomenus. Kirchengeschichte”, Ed. Bidez, J., Hansen, G.C.
Berlin: Akademie–Verlag, 1960; Die griechischen christlichen Schriftsteller 50.
Book 2, chapter 24, section 1, line 2

Ὑπὸ δὲ τοῦτον τὸν χρόνον παρειλήφαμεν καὶ τοὺς ἔνδον τῶν καθ' ἡμᾶς 
Ἰνδῶν, ἀπειράτους μείναντας τῶν Βαρθολομαίου κηρυγμάτων, μετασχεῖν 
τοῦ δόγματος ὑπὸ Φρουμεντίῳ, ἱερεῖ καὶ καθηγητῇ γενομένῳ παρ' αὐτοῖς 
τῶν ἱερῶν μαθημάτων. 



Salaminius Hermias Sozomenus Scr. Eccl., Historia ecclesiastica 
Book 2, chapter 24, section 1, line 5

                         ἵνα δὲ γνοίημεν καὶ ἐν τῷ παραδόξῳ τοῦ συμβάντος 
περὶ τοὺς Ἰνδοὺς οὐκ ἐξ ἀνθρώπων, ὥς τισι τερατολογεῖσθαι δοκεῖ, τὴν 
σύστασιν λαβεῖν τὸ τῶν Χριστιανῶν δόγμα, ἀναγκαῖον καὶ τὴν αἰτίαν τῆς 
Φρουμεντίου χειροτονίας διεξελθεῖν· ἔχει δὲ ὧδε. 



Salaminius Hermias Sozomenus Scr. Eccl., Historia ecclesiastica 
Book 2, chapter 24, section 5, line 2

                                               οὓς ζηλώσας Μερόπιός τις φι-
λόσοφος Τύριος τῆς Φοινίκης παρεγένετο εἰς Ἰνδούς. 



Salaminius Hermias Sozomenus Scr. Eccl., Historia ecclesiastica 
Book 2, chapter 24, section 5, line 4

                                                       ἱστορήσας δὲ τῆς Ἰνδικῆς   
ὅσα γε αὐτῷ ἐξεγένετο, τῆς ἐπανόδου εἴχετο νηὸς ἐπιτυχὼν στελλομένης εἰς 
Αἴγυπτον. 



Salaminius Hermias Sozomenus Scr. Eccl., Historia ecclesiastica 
Book 2, chapter 24, section 5, line 7

             συμβὰν δὲ κατὰ χρείαν ὕδατος ἢ τῶν ἄλλων ἐπιτηδείων εἰς ὅρμον 
τινὰ προσσχεῖν τὴν ναῦν, καταδραμόντες οἱ τῇδε Ἰνδοὶ κτείνουσι πάντας καὶ 
τὸν Μερόπιον· ἔτυχον γὰρ τότε λύσαντες τὰς πρὸς Ῥωμαίους σπονδάς. 



Salaminius Hermias Sozomenus Scr. Eccl., Historia ecclesiastica 
Book 2, chapter 24, section 8, line 1

                                                                            ἀντι-
βολοῦσαν δὲ τὴν βασιλίδα ᾐδέσθησαν, καὶ τὰ βασίλεια καὶ τὴν ἡγεμονίαν 
Ἰνδῶν διῴκουν. 



Salaminius Hermias Sozomenus Scr. Eccl., Historia ecclesiastica 
Book 2, chapter 24, section 8, line 3

                    ὁ δὲ Φρουμέντιος θείαις ἴσως προτραπεὶς ἐπιφανείαις 
ἢ καὶ αὐτομάτως τοῦ θεοῦ κινοῦντος ἐπυνθάνετο, εἴ τινες εἶεν Χριστιανοὶ 
παρ' Ἰνδοῖς ἢ Ῥωμαῖοι τῶν εἰσπλεόντων ἐμπόρων. 



Salaminius Hermias Sozomenus Scr. Eccl., Historia ecclesiastica 
Book 2, chapter 24, section 10, line 2

                                         συντυχὼν δὲ Ἀθανασίῳ τῷ προϊστα-
μένῳ τῆς Ἀλεξανδρέων ἐκκλησίας τὰ κατ' Ἰνδοὺς διηγήσατο καὶ ὡς ἐπι-
σκόπου δέοι αὐτοῖς τῶν αὐτόθι Χριστιανῶν ἐπιμελησομένου. 



Salaminius Hermias Sozomenus Scr. Eccl., Historia ecclesiastica 
Book 2, chapter 24, section 10, line 5

                                                                    ὁ δὲ Ἀθα-
νάσιος τοὺς ἐνδημοῦντας ἱερέας ἀγείρας ἐβουλεύσατο περὶ τούτου· καὶ χειρο-
τονεῖ αὐτὸν τῆς Ἰνδικῆς ἐπίσκοπον, λογισάμενος ἐπιτηδειότατον εἶναι τοῦ-
τον καὶ ἱκανὸν πολλὴν ποιῆσαι τὴν θρησκείαν, παρ' οἷς πρῶτος αὐτὸς ἔδειξε 
τὸ Χριστιανῶν ὄνομα καὶ σπέρμα παρέσχετο τῆς τοῦ δόγματος μετουσίας. 



Salaminius Hermias Sozomenus Scr. Eccl., Historia ecclesiastica 
Book 2, chapter 24, section 11, line 1

ὁ δὲ Φρουμέντιος πάλιν εἰς Ἰνδοὺς ὑποστρέψας λέγεται τοσοῦτον εὐκλεῶς 
τὴν ἱερωσύνην μετελθεῖν, ὡς ἐπαινεθῆναι παρὰ πάντων τῶν αὐτοῦ πειρα-  
θέντων, οὐχ ἧττον ἢ τοὺς ἀποστόλους θαυμάζουσι, καθότι καὶ ἐπισημό-
τατον αὐτὸν ὁ θεὸς ἀπέφηνε, πολλὰς καὶ παραδόξους ἰάσεις καὶ σημεῖα καὶ 
τέρατα δι' αὐτοῦ δημιουργήσας. 



Salaminius Hermias Sozomenus Scr. Eccl., Historia ecclesiastica 
Book 2, chapter 24, section 11, line 5

                                   ἡ μὲν δὴ παρ' Ἰνδοῖς ἱερωσύνη ταύτην 
ἔσχεν ἀρχήν. 



Salaminius Hermias Sozomenus Scr. Eccl., Historia ecclesiastica 
Book 7, chapter 15, section 6, line 6

                                   προθυμοτέρους δὲ τοὺς ἐν τῷ Σεραπείῳ 
παρεσκεύαζεν εἶναι τὸ συνειδέναι σφίσιν ἃ τετολμήκασιν, ἔπειτα δὲ καὶ 
Ὀλύμπιός τις ἐν φιλοσόφου σχήματι συνὼν αὐτοῖς καὶ πείθων χρῆναι μὴ 
ἀμελεῖν τῶν πατρίων, ἀλλ' εἰ δέοι ὑπὲρ αὐτῶν θνῄσκειν· καθαιρουμένων 
δὲ τῶν ξοάνων ἀθυμοῦντας ὁρῶν συνεβούλευε μὴ ἐξίστασθαι τῆς θρησκείας, 
ὕλην φθαρτὴν καὶ ἰνδάλματα λέγων εἶναι τὰ ἀγάλματα καὶ διὰ τοῦτο ἀφα-
νισμὸν ὑπομένειν, δυνάμεις δέ τινας ἐνοικῆσαι αὐτοῖς καὶ εἰς οὐρανοὺς ἀπο-
πτῆναι. 



Salaminius Hermias Sozomenus Scr. Eccl., Historia ecclesiastica 
Book 7, chapter 26, section 3, line 2

       τὸ δὲ τὸν σίελον εἰς τὸ στόμα δεξάμενον αὐτίκα κατέπεσε· καὶ 
νεκρὸν κείμενον οὐ μεῖον τῶν παρ' Ἰνδοῖς ἱστορουμένων ἑρπετῶν διεφάνη 
τὸ μέγεθος· ἀμέλει τοι, ὡς ἐπυθόμην, ὑπὸ ζεύγεσιν ὀκτὼ εἰς τὸ πλησίον 
πεδίον ἐξελκύσαντες αὐτὸ οἱ ἐπιχώριοι κατέκαυσαν, ὅπως μὴ διασαπεὶς τὸν 
ἀέρα λυμήνηται καὶ λοιμώδη ποιήσῃ. 

\end{greek}

\section{Papyri magicae}%???
\blockquote[From Wikipedia\footnote{\url{}}]{}
\begin{greek}

What dates for this?

Magica, Papyri magicae 
Preisendanz number 13, line 19

            ἀπηρτίσθω δὲ ἡ τράπεζα τοῖς ἐπιθύμα-
σι τούτοις, συνγενικοῖς οὖσι τοῦ θεοῦ – ἐκ δὲ ταύτης τῆς 
βίβλου Ἑρμῆς κλέψας τὰ ἐπιθύματα ζʹ προσεφώνησεν <ἐν> 
ἑαυτοῦ ἱερᾷ βύβλῳ ἐπικαλουμένῃ ‘Πτέρυγι’ – τοῦ μὲν 
Κρόνου στύραξ (ἔστιν γὰρ βαρὺς καὶ εὐώδης), τοῦ δὲ Διὸς 
μαλάβαθρον, τοῦ δὲ Ἄρεως κόστος, τοῦ δὲ Ἡλίου λίβανον, 
τῆς δὲ Ἀφροδίτης νάρδος Ἰνδικός, τοῦ δὲ Ἑρμοῦ κασία, 
τῆς δὲ Σελήνης ζμύρνα. 

\end{greek}


\section{Procopius}
\blockquote[From Wikipedia\footnote{\url{http://en.wikipedia.org/wiki/Procopius}}]{Procopius of Caesarea (Latin: Procopius Caesarensis, Greek: Προκόπιος ὁ Καισαρεύς; c. AD 500 – c. AD 565) was a prominent Byzantine scholar from Palaestina Prima. Accompanying the general Belisarius in the wars of the Emperor Justinian I, he became the principal historian of the 6th century, writing the Wars of Justinian, the Buildings of Justinian and the celebrated Secret History. He is commonly held to be the last major historian of the ancient world.}
\begin{greek}
Procopius Hist., De bellis (4029: 001)
“Procopii Caesariensis opera omnia, vols. 1–2”, Ed. Wirth, G. (post J. Haury)
Leipzig: Teubner, 1:1962; 2:1963.
Book 1, chapter 19, section 3, line 1

                   αὕτη δὲ ἡ θάλασσα ἐξ Ἰνδῶν ἀρχο-
μένη ἐνταῦθα τελευτᾷ τῆς Ῥωμαίων ἀρχῆς. 



Procopius Hist., De bellis 
Book 1, chapter 19, section 16, line 1

μεθ' οὓς δὴ τὰ γένη τῶν Ἰνδῶν ἐστιν. 



Procopius Hist., De bellis 
Book 1, chapter 19, section 23, line 1

Πλοῖα μέντοι ὅσα ἔν τε Ἰνδοῖς καὶ ἐν ταύτῃ τῇ 
θαλάσσῃ ἐστὶν, οὐ τρόπῳ τῷ αὐτῷ ᾧπερ αἱ ἄλλαι νῆες 
πεποίηνται. 



Procopius Hist., De bellis 
Book 1, chapter 19, section 25, line 1

αἴτιον δὲ οὐχ ὅπερ οἱ πολλοὶ οἴονται, πέτραι τινὲς 
ἐνταῦθα οὖσαι καὶ τὸν σίδηρον ἐφ' ἑαυτὰς ἕλκουσαι 
(τεκμήριον δέ· ταῖς γὰρ Ῥωμαίων ναυσὶν ἐξ Αἰλᾶ 
πλεούσαις ἐς θάλασσαν τήνδε, καίπερ σιδήρῳ πολλῷ 
ἡρμοσμέναις, οὔποτε τοιοῦτον ξυνηνέχθη παθεῖν), 
ἀλλ' ὅτι οὔτε σίδηρον οὔτε ἄλλο τι τῶν ἐς ταῦτα 
ἐπιτηδείων Ἰνδοὶ ἢ Αἰθίοπες ἔχουσιν. 



Procopius Hist., De bellis 
Book 1, chapter 20, section 9, line 6

Τότε δὲ Ἰουστινιανὸς [ὁ] βασιλεὺς ἐν μὲν Αἰθίοψι 
βασιλεύοντος Ἑλλησθεαίου, Ἐσιμιφαίου δὲ ἐν Ὁμηρί-
ταις, πρεσβευτὴν Ἰουλιανὸν ἔπεμψεν, ἀξιῶν ἄμφω 
Ῥωμαίοις διὰ τὸ τῆς δόξης ὁμόγνωμον Πέρσαις πολε-
μοῦσι ξυνάρασθαι, ὅπως Αἰθίοπες μὲν ὠνούμενοί τε   
τὴν μέταξαν ἐξ Ἰνδῶν ἀποδιδόμενοί τε αὐτὴν ἐς Ῥω-
μαίους, αὐτοὶ μὲν κύριοι γένωνται χρημάτων μεγάλων, 
Ῥωμαίους δὲ τοῦτο ποιήσωσι κερδαίνειν μόνον, ὅτι 
δὴ οὐκέτι ἀναγκασθήσονται τὰ σφέτερα αὐτῶν χρή-
ματα ἐς τοὺς πολεμίους μετενεγκεῖν (αὕτη δέ ἐστιν ἡ 
μέταξα, ἐξ ἧς εἰώθασι τὴν ἐσθῆτα ἐργάζεσθαι, ἣν 
πάλαι μὲν Ἕλληνες Μηδικὴν ἐκάλουν, τανῦν δὲ ση-
ρικὴν ὀνομάζουσιν), Ὁμηρῖται δὲ ὅπως Καϊσὸν τὸν 
φυγάδα φύλαρχον Μαδδηνοῖς καταστήσωνται καὶ στρατῷ 
μεγάλῳ αὐτῶν τε Ὁμηριτῶν καὶ Σαρακηνῶν τῶν

Μαδ-



Procopius Hist., De bellis 
Book 1, chapter 20, section 12, line 2

                                                     τοῖς τε 
γὰρ Αἰθίοψι τὴν μέταξαν ὠνεῖσθαι πρὸς τῶν Ἰνδῶν 
ἀδύνατα ἦν, ἐπεὶ ἀεὶ οἱ Περσῶν ἔμποροι πρὸς αὐτοῖς   
τοῖς ὅρμοις γινόμενοι, οὗ δὴ τὰ πρῶτα αἱ τῶν Ἰνδῶν 
νῆες καταίρουσιν, ἅτε χώραν προσοικοῦντες τὴν ὅμο-
ρον, ἅπαντα ὠνεῖσθαι τὰ φορτία εἰώθασι, καὶ τοῖς 
Ὁμηρίταις χαλεπὸν ἔδοξεν εἶναι χώραν ἀμειψαμένοις 
ἔρημόν τε καὶ χρόνου πολλοῦ ὁδὸν κατατείνουσαν ἐπ' 
ἀνθρώπους πολλῷ μαχιμωτέρους ἰέναι. 



Procopius Hist., De bellis 
Book 2, chapter 25, section 3, line 1

ἔκ τε γὰρ Ἰνδῶν καὶ τῶν πλησιοχώρων Ἰβήρων πάν-  
των τε ὡς εἰπεῖν τῶν ἐν Πέρσαις ἐθνῶν καὶ Ῥωμαίων 
τινῶν τὰ φορτία ἐσκομιζόμενοι ἐνταῦθα ἀλλήλοις ξυμ-
βάλλουσι. 



Procopius Hist., De bellis 
Book 7, chapter 35, section 23, line 2

Ἐν ᾧ δὲ ταῦτα ἐπράσσετο τῇδε ᾗπέρ μοι εἴρηται,   
ἐν τούτῳ τῶν τις Βελισαρίου δορυφόρων, Ἰνδοὺλφ 
ὄνομα, βάρβαρος γένος, θυμοειδής τε καὶ δραστήριος, 
ὃς δὴ ἐν Ἰταλίᾳ λειφθεὶς ἔτυχε, Τουτίλᾳ τε καὶ Γότ-
θοις προσεχώρησεν οὐδενὶ λόγῳ. 



Procopius Hist., De bellis 
Book 7, chapter 35, section 29, line 2

                                         ἅπερ ἅπαντα Ἰν-
δούλφ τε καὶ Γότθοι ἑλόντες κτείναντές τε τοὺς ἐν 
ποσὶν ἅπαντας καὶ τὰ χρήματα ληϊσάμενοι παρὰ Του-
τίλαν ἦλθον. 



Procopius Hist., De bellis 
Book 8, chapter 3, section 10, line 1

                                 τὰ γὰρ ἐπιτηδεύματα 
μέχρι ἐς τοὺς ἀπογόνους παραπεμπόμενα τῶν προγε-
γενημένων τῆς φύσεως ἴνδαλμα γίνεται. 



Procopius Hist., De bellis 
Book 8, chapter 17, section 1, line 2

Ὑπὸ τοῦτον τὸν χρόνον τῶν τινες μοναχῶν 
ἐξ Ἰνδῶν ἥκοντες, γνόντες τε ὡς Ἰουστινιανῷ βασιλεῖ 
διὰ σπουδῆς εἴη μηκέτι πρὸς Περσῶν τὴν μέταξαν 
ὠνεῖσθαι Ῥωμαίους, ἐς βασιλέα γενόμενοι οὕτω δὴ τὰ 
ἀμφὶ τῇ μετάξῃ διοικήσεσθαι ὡμολόγουν, ὡς μηκέτι 
Ῥωμαῖοι ἐκ Περσῶν τῶν σφίσι πολεμίων ἢ ἄλλου του 
ἔθνους τὸ ἐμπόλημα τοῦτο ποιήσωνται· χρόνου γὰρ 
κατατρῖψαι μῆκος ἐν χώρᾳ ὑπὲρ Ἰνδῶν ἔθνη τὰ 
πολλὰ οὔσῃ, ἥπερ Σηρίνδα ὀνομάζεται, ταύτῃ τε ἐς 
τὸ ἀκριβὲς ἐκμεμαθηκέναι ὁποίᾳ ποτὲ μηχανῇ γίνε-
σθαι τὴν μέταξαν ἐν γῇ τῇ Ῥωμαίων δυνατὰ εἴη. 



Procopius Hist., De bellis 
Book 8, chapter 23, section 2, line 1

       τινὲς δὲ αὐτὸν Ἰνδοὺλφ ἐκάλουν. 



Procopius Hist., De bellis 
Book 8, chapter 35, section 38, line 1

       Γότθοι μὲν οὖν μεταξὺ χίλιοι τοῦ στρατοπέδου 
ἐξαναστάντες ἐς Τικινόν τε πόλιν καὶ χωρία τὰ ὑπὲρ 
ποταμὸν Πάδον ἐχώρησαν, ὧν ἄλλοι τε ἡγοῦντο καὶ   
Ἰνδοὺλφ, οὗπερ πρότερον ἐπεμνήσθην. 



Procopius Hist., Historia arcana (= Anecdota) (4029: 002)
“Procopii Caesariensis opera omnia, vol. 3”, Ed. Wirth, G. (post J. Haury)
Leipzig: Teubner, 1963.
Chapter 17, section 34, line 3

                                             Χρυσομαλλὼ 
δὲ αὕτη πάλαι μὲν ὀρχηστρὶς ἐγεγόνει καὶ αὖθις ἑταίρα, 
τότε δὲ ξὺν ἑτέρᾳ Χρυσομαλλοῖ καὶ Ἰνδαροῖ ἐν Παλα-
τίῳ τὴν δίαιταν εἶχεν. 



Procopius Hist., De aedificiis (lib. 1–6) (4029: 003)
“Procopii Caesariensis opera omnia, vol. 4”, Ed. Wirth, G. (post J. Haury)
Leipzig: Teubner, 1964.


Procopius Hist., De aedificiis (lib. 1-6) 
Book 6, chapter 1, section 6, line 1

Νεῖλος μὲν ὁ ποταμὸς ἐξ Ἰνδῶν ἐπ' Αἰγύπτου φερό-
μενος δίχα τέμνει τὴν ἐκείνῃ γῆν ἄχρι ἐς θάλασσαν. 
\end{greek}



\section{Proclus Phil.}

\blockquote[From Wikipedia\footnote{\url{http://en.wikipedia.org/wiki/Proclus}}]{Proclus Lycaeus (play /ˈprɒkləs ˌlaɪˈsiːəs/; 8 February 412 – 17 April 485 AD), called the Successor (Greek Πρόκλος ὁ Διάδοχος, Próklos ho Diádokhos), was a Greek Neoplatonist philosopher, one of the last major Classical philosophers (see Damascius). He set forth one of the most elaborate and fully developed systems of Neoplatonism. He stands near the end of the classical development of philosophy, and was very influential on Western medieval philosophy (Greek and Latin) as well as Islamic thought.

        "Wherever there is number, there is beauty."

    Proclus, quoted by M. Kline, Mathematical Thought from Ancient to Modern Times }

\begin{greek}

Proclus Phil., In Platonis Cratylum commentaria (4036: 009)
“Procli Diadochi in Platonis Cratylum commentaria”, Ed. Pasquali, G.
Leipzig: Teubner, 1908.
Section 71, line 77

                    μετέχουσιν δ' ἄλλως ἄλλοι καὶ τούτων, οἷον 
Αἰγύπτιοι κατὰ τὴν ἐπιχώριον φωνὴν τοιαῦτα παρὰ τῶν θεῶν 
ἔλαβον ὀνόματα, Χαλδαῖοι δὲ καὶ Ἰνδοὶ ἄλλως κατὰ τὴν οἰ-
κείαν γλῶσσαν καὶ Ἕλληνες ὡσαύτως κατὰ τὴν σφετέραν διά-
λεκτον. 

Proclus Phil., In Platonis Timaeum commentaria 
Volume 1, page 208, line 18

φιλεῖ, τοῖς δὲ θεοῖς ὁ σπουδαῖος ὁμοιότατος, καὶ διότι <ἐν 
φρουρᾷ> ὄντες οἱ τῆς ἀρετῆς ἀντεχόμενοι καὶ ὑπὸ τοῦ 
σώματος ὡς δεσμωτηρίου συνειλημμένοι δεῖσθαι τῶν θεῶν 
ὀφείλουσι περὶ τῆς ἐντεῦθεν μεταστάσεως, καὶ ὅτι ὡς παῖδας 
πατέρων ἀποσπασθέντας εὔχεσθαι προσήκει περὶ τῆς πρὸς 
τοὺς ἀληθινοὺς ἡμῶν πατέρας, τοὺς θεούς, ἐπανόδου, καὶ 
ὅτι ἀπάτορές τινες ἄρα καὶ ἀμήτορες ἐοίκασιν εἶναι οἱ μὴ 
ἀξιοῦντες εὔχεσθαι μηδὲ ἐπιστρέφειν εἰς τοὺς κρείττονας, καὶ 
ὅτι καὶ ἐν πᾶσι τοῖς ἔθνεσιν οἱ σοφίᾳ διενεγκόντες περὶ 
εὐχὰς ἐσπούδασαν, Ἰνδῶν μὲν Βραχμᾶνες, Μάγοι δὲ Περσῶν, 
Ἑλλήνων δὲ οἱ θεολογικώτατοι, οἳ καὶ τελετὰς κατεστήσαντο 
καὶ μυστήρια· Χαλδαῖοι δὲ καὶ τὸ ἄλλο θεῖον ἐθεράπευσαν καὶ 
αὐτὴν τὴν ἀρετὴν τῶν θεῶν θεὸν εἰπόντες ἐσέφθησαν, πολ-
λοῦ δέοντες διὰ τὴν ἀρετὴν ὑπερφρονεῖν τῆς ἱερᾶς θρησκείας· 
καὶ ἐπὶ πᾶσι τούτοις, ὅτι μέρος ὄντας τοῦ παντὸς δεῖσθαι 
προσήκει τοῦ παντός· παντὶ γὰρ ἡ πρὸς τὸ ὅλον ἐπιστροφὴ 
παρέχεται τὴν σωτηρίαν· εἴτε οὖν ἀρετὴν ἔχεις, παρακλητέον 
σοι τὸ τὴν ὅλην ἀρετὴν προειληφός· τὸ γὰρ πᾶν ἀγαθὸν 
αἴτιόν ἐστι καὶ σοὶ τοῦ ἀγαθοῦ τοῦ σοὶ προσήκοντος· εἴτε 

\end{greek}




\section{Agathias Scholasticus}

\blockquote[From Wikipedia\footnote{\url{http://en.wikipedia.org/wiki/Agathias}}]{Agathias or Agathias Scholasticus (Ancient Greek: Ἀγαθίας σχολαστικός) c. AD 530[1]-582[1]/594), of Myrina (Mysia), an Aeolian city in western Asia Minor (now in Turkey), was a Greek poet and the principal historian of part of the reign of the Roman emperor Justinian I between 552 and 558.}

\begin{greek}

Agathias Scholasticus Epigr., Hist., Historiae (4024: 001)
“Agathiae Myrinaei historiarum libri quinque”, Ed. Keydell, R.
Berlin: De Gruyter, 1967; Corpus fontium historiae Byzantinae 2. Series Berolinensis.
Page 73, line 24

                            4 πρῶτοι μὲν γὰρ ὧν ἀκοῇ ἴσμεν Ἀσσύριοι 
λέγονται ἅπασαν τὴν Ἀσίαν χειρώσασθαι πλὴν Ἰνδῶν τῶν ὑπὲρ 
Γάγγην ποταμὸν ἱδρυμένων. 

\end{greek}



\section{Olympiodorus the Younger}

\blockquote[From Wikipedia\footnote{\url{http://en.wikipedia.org/wiki/Olympiodorus_the_Younger}}]{Olympiodorus the Younger (Greek: Ὀλύμπιόδωρος ὁ Νεώτερος)(c. 495-570) was a Neoplatonist philosopher, astrologer and teacher who lived in the early years of the Byzantine Empire, after Justinian's Decree of 529 A.D. which closed Plato's Academy in Athens and other pagan schools. Olympiodorus was the last pagan to maintain the Platonist tradition in Alexandria (see Alexandrian School); after his death the School passed into the hands of Christian Aristotelians, and was eventually moved to Constantinople.

Among the extant writings of Olympiodorus the Younger are a biography of Plato, commentaries on several dialogues of Plato and on Aristotle, and an introduction to Aristotelian philosophy. Olympiodorus also provides information on the work of the earlier Neoplatonist Iamblichus which is not found elsewhere. The surviving works are:

    Commentary on Plato's Alcibiades
    Commentary on Plato's Gorgias
    Commentary on Plato's Phaedo
    Life of Plato
    Introduction (prolegomena) to Aristotle's logic
    Commentary on the Aristotle's Meteorology
    Commentary on the Aristotle's Categories

In addition, a Commentary by Olympiodorus is extant on Paulus Alexandrinus' Introduction to astrology (which was written in 378 AD). Although the manuscript of the Commentary is credited in two later versions to a Heliodorus, L.G. Westerink argues that it is actually the outline of a series of lectures given by Olympiodorus in Alexandria between May and July 564 AD. The Commentary is an informative expatiation of Paulus' tersely written text, elaborating on practices and sources. The Commentary also illuminates the developments in astrological theory in the 200 years after Paulus.
}

\begin{greek}

Olympiodorus Phil., In Aristotelis meteora commentaria (4019: 003)
“Olympiodori in Aristotelis meteora commentaria”, Ed. Stüve, G.
Berlin: Reimer, 1900; Commentaria in Aristotelem Graeca 12.2.


Olympiodorus Phil., In Aristotelis meteora commentaria 
Page 192, line 1n

<Τὸ γὰρ ἀφ' Ἡρακλείων στηλῶν μέχρι τῆς Ἰνδικῆς τοῦ 
ἐξ Αἰθιοπίας πρὸς τὴν Μαιῶτιν καὶ τοὺς ἐσχατεύοντας τῆς 
 Σκυθίας τόπους πλέον ἢ πέντε πρὸς τρία τὸ μέγεθος. 




Olympiodorus Phil., In Aristotelis meteora commentaria 
Page 192, line 6

        καὶ τοῦ μὲν μήκους ἐπὶ δυσμὰς τὰς Ἡρακλείους στήλας, πρὸς δὲ 
ἀνατολὰς τὴν ἐσχάτην Ἰνδίαν, παρ' οἷς ἐστιν ἡ Ἐρυθρὰ θάλασσα. 



Olympiodorus Phil., In Platonis Alcibiadem commentarii (4019: 004)
“Olympiodorus. Commentary on the first Alcibiades of Plato”, Ed. Westerink, L.G.
Amsterdam: Hakkert, 1956, Repr. 1982.
Section 10, line 9

          δεύτερον ἡμῖν συμβάλλεται πρὸς τὴν γνῶσιν πάντων τῶν ὄντων· 
εἰ γὰρ γινώσκομεν τὴν ψυχήν, γνωσόμεθα καὶ οὓς ἔχει λόγους ἐν αὑτῇ, 
πάντων δὲ τῶν ὄντων ἔχει τοὺς λόγους καὶ τοὺς τύπους ὡς ἴνδαλμα 
τούτων οὖσα· συμβάλλεται ἡμῖν ἄρα ἡ αὐτῆς γνῶσις καὶ πρὸς τὴν τῶν 
ὄντων πάντων. 



Olympiodorus Phil., In Platonis Alcibiadem commentarii 
Section 164, line 7

                                                                 οὐ μόνον δὲ 
οὗτοι τοιοῦτοι, ἀλλὰ καὶ Ἰνδοὶ ἔχονται τῷ πάθει τούτῳ. 



Olympiodorus Phil., In Platonis Alcibiadem commentarii 
Section 165, line 22

                                                              ἱματίων’> δὲ 
’ἕλξεις’> φησὶ δύο μέρη λέγων, διότι ποδήρεις χιτῶνας φοροῦσιν οἱ 
Πέρσαι (διὸ καὶ ‘ἑλκεσίπεπλοι’) καὶ Ἰνδοί, καθάπερ καὶ οἱ Ἴωνες. 


\end{greek}


\section{Basilius}

Is this Basil of Caesarea?

\blockquote[From Wikipedia\footnote{\url{http://en.wikipedia.org/wiki/Basil_of_Caesarea}}]{Basil of Caesarea, also called Saint Basil the Great, (329 or 330[5] – January 1, 379) (Greek: Ἅγιος Βασίλειος ὁ Μέγας) was the Greek bishop of Caesarea Mazaca in Cappadocia, Asia Minor (modern-day Turkey). He was an influential theologian who supported the Nicene Creed and opposed the heresies of the early Christian church, fighting against both Arianism and the followers of Apollinaris of Laodicea. His ability to balance his theological convictions with his political connections made Basil a powerful advocate for the Nicene position.}

\begin{greek}

Basilius Theol., Homiliae in hexaemeron (2040: 001)
“Basile de Césarée. Homélies sur l'hexaéméron, 2nd edn.”, Ed. Giet, S.
Paris: Cerf, 1968; Sources chrétiennes 26 bis.
Homily 3, section 6, line 9

                 Ἐκ μέν γε τῆς ἕω, ἀπὸ μὲν χειμερινῶν 
τροπῶν ὁ Ἰνδὸς ῥεῖ ποταμὸς ῥεῦμα πάντων ποταμίων 
ὑδάτων πλεῖστον, ὡς οἱ τὰς περιόδους τῆς γῆς ἀναγράφοντες 
ἱστορήκασιν· ἀπὸ δὲ τῶν μέσων τῆς ἀνατολῆς ὅ τε Βάκτρος, 
καὶ ὁ Χοάσπης, καὶ ὁ Ἀράξης, ἀφ' οὗ καὶ ὁ Τάναϊς 
ἀποσχιζόμενος εἰς τὴν Μαιῶτιν ἔξεισι λίμνην. 



Basilius Theol., Homiliae in hexaemeron 
Homily 4, section 3, line 39

                 Ὅτι γὰρ ταπεινοτέρα τῆς ἐρυθρᾶς θαλάσσης 
ἡ Αἴγυπτος, ἔργῳ ἔπεισαν ἡμᾶς οἱ θελήσαντες ἀλλήλοις τὰ 
πελάγη συνάψαι, τό τε Αἰγύπτιον καὶ τὸ Ἰνδικὸν, ἐν ᾧ ἡ 
ἐρυθρά ἐστι θάλασσα. 



Basilius Theol., Homiliae in hexaemeron 
Homily 6, section 9, line 28

                             Σημεῖον δὲ, ὅτι καὶ Ἰνδοὶ καὶ 
Βρεττανοὶ τὸν ἴσον βλέπουσιν. 



Basilius Theol., Homiliae in hexaemeron 
Homily 7, section 2, line 34

                                   Ἄλλα γνωρίζουσιν οἱ τὴν 
Ἰνδικὴν ἁλιεύοντες θάλασσαν· ἄλλα, οἱ τὸν Αἰγύπτιον 
ἀγρεύοντες κόλπον· ἄλλα, νησιῶται· καὶ ἄλλα, Μαυρούσιοι. 



Basilius Theol., Homiliae in hexaemeron 
Homily 8, section 8, line 16

       Ὁποῖα καὶ περὶ τοῦ Ἰνδικοῦ σκώληκος ἱστορεῖται 
τοῦ κερασφόρου· ὃς εἰς κάμπην τὰ πρῶτα μεταβαλὼν, εἶτα 
προϊὼν βομβυλιὸς γίνεται, καὶ οὐδὲ ἐπὶ ταύτης ἵσταται τῆς 
μορφῆς, ἀλλὰ χαύνοις καὶ πλατέσι πετάλοις ὑποπτεροῦται. 



Basilius Theol., Epistulae (2040: 004)
“Saint Basile. Lettres, 3 vols.”, Ed. Courtonne, Y.
Paris: Les Belles Lettres, 1:1957; 2:1961; 3:1966.
Epistle 1, section 1, line 32

         Δοκῶ γάρ μοι, εἰ μὴ ὥσπερ τι θρέμμα θαλλῷ 
προδεικνυμένῳ ἑπόμενος ἀπηγόρευσα, ἐπέκεινα ἄν σε καὶ 
Νύσης τῆς Ἰνδικῆς ἐλθεῖν ἀγόμενον, καί, εἴ τι ἔσχατον 
τῆς καθ' ἡμᾶς οἰκουμένης χωρίον, καὶ τοῦτο ἐπιπλανη-
θῆναι. 



Basilius Theol., Enarratio in prophetam Isaiam [Dub.] (2040: 009)
“San Basilio. Commento al profeta Isaia, 2 vols.”, Ed. Trevisan, P.
Turin: Società Editrice Internazionale, 1939.
Chapter 13, section 269, line 13

                                                          – Ἔοικε 
δὲ χώραν τινὰ λέγειν ἐν τῷ ἔθνει τῷ Ἰνδικῷ τὴν Σουφεὶρ, 
περὶ ἣν οἱ πολυτίμητοι τῶν λίθων πεφύκασι γίνεσθαι. 

\end{greek}

\section{Theodoretus}

\blockquote[From Wikipedia\footnote{\url{http://en.wikipedia.org/wiki/Theodoretus}}]{Theodoret of Cyrus or Cyrrhus (Greek: Θεοδώρητος Κύρρου; c. 393 – c. 457) was an influential author, theologian, and Christian bishop of Cyrrhus, Syria (423-457). He played a pivotal role in many early Byzantine church controversies that led to various ecumenical acts and schisms. He is considered blessed or a saint by the Eastern Orthodox Church.[1]}

\begin{greek}

Theodoretus Scr. Eccl., Theol., Graecarum affectionum curatio (4089: 001)
“Théodoret de Cyr. Thérapeutique des maladies helléniques, 2 vols.”, Ed. Canivet, P.
Paris: Cerf, 1958; Sources chrétiennes 57.
Book 1, section 25, line 6

Εἰ δὲ ἄρα τοῦτό φατε, ὡς ἔξω μὲν τῆς Ἑλλάδος καὶ ἔφυσαν 
οἵδε οἱ ἄνδρες καὶ ἐτράφησαν, τὴν δέ γε Ἑλληνικὴν ἠσκήθησαν 
γλῶτταν, πρῶτον μὲν ὁμολογεῖτε καὶ ἐν ἄλλοις ἔθνεσιν ἄνδρας 
γεγενῆσθαι σοφούς· καὶ γὰρ δὴ καὶ Ζάμολξιν τὸν Θρᾷκα καὶ 
Ἀνάχαρσιν τὸν Σκύθην ἐπὶ σοφίᾳ θαυμάζετε, καὶ τῶν Βραχμά-
νων πολὺ παρ' ὑμῖν τὸ κλέος· Ἰνδοὶ δὲ οὗτοι, οὐχ Ἕλληνες. 



Theodoretus Scr. Eccl., Theol., Graecarum affectionum curatio 
Book 2, section 53, line 1

          Εἶτα διδάσκει σαφῶς, ὡς οὐδὲν αὐτῷ τῶν ὁρωμένων 
προσέοικε, καὶ παντάπασιν ἀπαγορεύει μηδεμίαν εἰκόνα πρὸς 
μίμησίν τινος τῶν ὁρωμένων κατασκευάσαι καὶ νομίσαι τοῦτο 
δείκηλον εἶναι καὶ ἴνδαλμα τοῦ ἀοράτου Θεοῦ. 



Theodoretus Scr. Eccl., Theol., Graecarum affectionum curatio 
Book 5, section 58, line 7

                                                 Καὶ γὰρ Ἀνάχαρσιν 
θαυμάζουσιν, ἄνδρα Σκύθην, οὐκ Ἀθηναῖον οὐδὲ Ἀργεῖον οὐδέ 
γε Κορίνθιον οὐδὲ Τεγεάτην ἢ Σπαρτιάτην, καὶ τοὺς Βραχμᾶνας 
ὑπεράγανται, Ἰνδοὺς ὄντας, οὐ Δωριέας οὐδὲ Αἰολέας οὐδέ γε 
Ἴωνας· ἐπαινοῦσι δὲ καὶ Αἰγυπτίους ὡς σοφωτάτους· πολλὰς 
γάρ τοι καὶ παρὰ τούτων ἔμαθον ἐπιστήμας. 



Theodoretus Scr. Eccl., Theol., Graecarum affectionum curatio 
Book 5, section 66, line 7

Καὶ ἡ Ἑβραίων φωνὴ οὐ μόνον εἰς τὴν Ἑλλήνων μετεβλήθη, 
ἀλλὰ καὶ εἰς τὴν Ῥωμαίων καὶ Αἰγυπτίων καὶ Περσῶν καὶ 
Ἰνδῶν καὶ Ἀρμενίων καὶ Σκυθῶν καὶ Σαυροματῶν καὶ ξυλ-
λήβδην εἰπεῖν εἰς ἁπάσας τὰς γλώττας, αἷς ἅπαντα τὰ ἔθνη 
κεχρημένα διατελεῖ. 



Theodoretus Scr. Eccl., Theol., Graecarum affectionum curatio 
Book 5, section 73, line 1

             Τοὺς δέ γε Ἰνδοὺς καὶ τούτων πολλῷ σοφωτέρους 
εἶναί φασιν. 




Theodoretus Scr. Eccl., Theol., Graecarum affectionum curatio 
Book 8, section 6, line 11

      Ἡνίκα μὲν γὰρ μετὰ τῶν σωμάτων ἐπολιτεύοντο, νῦν μὲν 
παρὰ τούτους, νῦν δὲ παρ' ἐκείνους ἐφοίτων, καὶ ἄλλοτε μὲν 
Ῥωμαίοις, ἄλλοτε δὲ Ἱσπανοῖς ἢ Κελτοῖς διελέγοντο· ἐπειδὴ 
δὲ πρὸς ἐκεῖνον ἐξεδήμησαν, ὑφ' οὗ κατεπέμφθησαν, ἅπαντες 
αὐτῶν ἐνδελεχῶς ἀπολαύουσιν, οὐ μόνον Ῥωμαῖοι, καὶ ὅσοι γε 
τὸν τούτων ἀγαπῶσι ζυγὸν καὶ ὑπὸ τούτων ἰθύνονται, ἀλλὰ καὶ 
Πέρσαι καὶ Σκύθαι καὶ Μασσαγέται καὶ Σαυρομάται καὶ Ἰνδοὶ 
καὶ Αἰθίοπες, καὶ ξυλλήβδην εἰπεῖν ἅπαντα τῆς οἰκουμένης τὰ 
τέρματα. 



Theodoretus Scr. Eccl., Theol., Graecarum affectionum curatio 
Book 9, section 15, line 4

                                                             Καὶ οὐ 
μόνον Ῥωμαίους καὶ τοὺς ὑπὸ τούτοις τελοῦντας, ἀλλὰ καὶ τὰ 
Σκυθικὰ καὶ τὰ Σαυροματικὰ ἔθνη καὶ Ἰνδοὺς καὶ Αἰθίοπας καὶ 
Πέρσας καὶ Σῆρας καὶ Ὑρκανοὺς καὶ Βακτριανοὺς καὶ Βρεττα-
νοὺς καὶ Κίμβρους καὶ Γερμανοὺς καὶ ἁπαξαπλῶς πᾶν ἔθνος καὶ 
γένος ἀνθρώπων δέξασθαι τοῦ σταυρωθέντος τοὺς νόμους ἀνέπει-
σαν, οὐχ ὅπλοις χρησάμενοι καὶ πολλαῖς μυριάσι λογάδων οὐδὲ 
τῇ τῆς Περσικῆς ὠμότητος χρώμενοι βίᾳ, ἀλλὰ πείθοντες καὶ 
δεικνύντες ὀνησιφόρους τοὺς νόμους, καὶ οὐδὲ δίχα κινδύνων 
τοῦτο ποιοῦντες, ἀλλὰ πολλὰς μὲν κατὰ πόλιν ὑπομένοντες πα-
ροινίας, πολλὰς δὲ καὶ παρὰ τῶν τυχόντων δεχόμενοι μάστιγας 
καὶ στρεβλούμενοι καὶ καθειργνύμενοι καὶ πᾶσαν ἰδέαν

κολαστη-



Theodoretus Scr. Eccl., Theol., Eranistes (4089: 002)
“Theodoret of Cyrus. Eranistes”, Ed. Ettlinger, G.H.
Oxford: Clarendon Press, 1975.
Page 64, line 31

                              Τὸν μέντοι ἄνθρωπον ἁπλῶς ἀκούσας, 
οὐκ εἰς τὸ ἄτομον ἀπερείδει τὸν νοῦν, ἀλλὰ καὶ τὸν Ἰνδὸν καὶ τὸν 
Σκύθην καὶ τὸν Μασσαγέτην καὶ ἁπαξαπλῶς πᾶν γένος ἀνθρώπων 
λογίζεται. 



Theodoretus Scr. Eccl., Theol., Historia ecclesiastica (4089: 003)
“Theodoret. Kirchengeschichte, 2nd edn.”, Ed. Parmentier, L., Scheidweiler, F.
Berlin: Akademie–Verlag, 1954; Die griechischen christlichen Schriftsteller 44.
Page 2, line 21

     Περὶ τῆς Ἰνδῶν πίστεως. 



Theodoretus Scr. Eccl., Theol., Historia ecclesiastica 
Page 73, line 1

Παρὰ δὲ Ἰνδοῖς κατὰ τοῦτον ἀνέτειλε τὸν χρόνον τῆς θεογνω-
σίας τὸ φῶς. 



Theodoretus Scr. Eccl., Theol., Historia ecclesiastica 
Page 73, line 7

                                       τότε τις Τύριος τῆς θύραθεν 
φιλοσοφίας μετέχων, τὴν ἐσχάτην Ἰνδίαν ἱστορῆσαι ποθήσας, σὺν 
δύο μειρακίοις ἀδελφιδοῖς ἐξεδήμησεν· ὧν ἐπόθησε δὲ τυχών, ναυ-
τιλίᾳ χρώμενος ἐπανῄει. 



Theodoretus Scr. Eccl., Theol., Historia ecclesiastica 
Page 74, line 5

                         ὁ δὲ Φρουμέντιος τὴν περὶ τὰ θεῖα σπουδὴν 
τῆς τῶν γεγεννηκότων προτετίμηκε θέας καὶ τὴν Ἀλεξάνδρου κατα-
λαβὼν πόλιν τὸν τῆς ἐκκλησίας ἐδίδαξε πρόεδρον, ὡς Ἰνδοὶ λίαν 
ποθοῦσι τὸ νοερὸν εἰσδέξασθαι φῶς. 



Theodoretus Scr. Eccl., Theol., Historia ecclesiastica 
Page 74, line 17

                   ἀποστολικαῖς γὰρ κεχρημένος θαυματουργίαις τοὺς 
ἀντιλέγειν τοῖς λόγοις πειρωμένους ἐθήρευε, καὶ ἡ τερατουργία μαρ-
τυροῦσα τοῖς λεγομένοις παμπόλλους καθ' ἑκάστην ἡμέραν ἐζώγρει. 
 Ἰνδῶν μὲν οὖν ὁ Φρουμέντιος πρὸς θεογνωσίαν ἐγένετο ποδηγός. 



Theodoretus Scr. Eccl., Theol., Historia religiosa (= Philotheus) (4089: 004)
“Théodoret de Cyr. L'histoire des moines de Syrie, 2 vols.”, Ed. Canivet, P., Leroy–Molinghen, A.
Paris: Cerf, 1:1977; 2:1979; Sources chrétiennes 234, 257.




Theodoretus Scr. Eccl., Theol., Epistulae: Collectio Patmensis (epistulae 1–52) (4089: 005)
“Théodoret de Cyr. Correspondance I”, Ed. Azéma, Y.
Paris: Cerf, 1955; Sources chrétiennes 40.
Epistle 18, line t

                                                               Διά 
τοι τοῦτο μικρὰν ἀναβολὴν ἐπαγγέλλω· ἐλπίζομεν γάρ, ὡς τὸ 
ζοφῶδες τοῦτο καὶ τετριγὸς νέφος ὁ φιλάνθρωπος ἡμῶν ὅτι 
τάχιστα διασκεδάσει δεσπότης. 
ΑΡΕΟΒΙΝΔ*ῳ ΣΤΡΑΤΗΛΑΤῌ. 




Theodoretus Scr. Eccl., Theol., Epistulae: Collectio Patmensis (epistulae 1-52) 
Epistle 21, line t

             Διὰ ταύτην τοίνυν τὴν ἱερὰν καὶ φιλτάτην τῷ Θεῷ 
κεφαλὴν ἀπολαυσάτω τῆς ὑμετέρας κηδεμονίας καὶ σωθήτω 
τῇ πόλει τῇ ἡμετέρᾳ τὸ σχῆμα. 
ΑΡΕΟΒΙΝΔ*ᾳ ΠΑΤΡΙΚΙῼ. 




Theodoretus Scr. Eccl., Theol., Epistulae: Collectio Sirmondiana (epistulae 1–95) (4089: 006)
“Théodoret de Cyr. Correspondance II”, Ed. Azéma, Y.
Paris: Cerf, 1964; Sources chrétiennes 98.
Epistle 23, line t

                  Ἀνιῶμαι δὲ μὴ πάντα ἐπαινῶν τὰ ὑμέτερα, 
ἀλλὰ τὸ κεφάλαιον τῶν ἀγαθῶν ἐλλεῖπον τοῖς ἐπαίνοις ὁρῶν· 
ὅπερ εἰ δοίη προσγενέσθαι Θεός, ἐν ἅπασι τοῖς τῆς ἀρετῆς 
εἴδεσι κατὰ πάντων σχήσετε τὸ κράτος, τῶν τὴν αὐτὴν ὑμῖν 
βιοτὴν μετιόντων.   
ΑΡΕΟΒΙΝΔ*ᾳ ΠΑΤΡΙΚΙῼ. 




Theodoretus Scr. Eccl., Theol., Commentaria in Isaiam (4089: 008)
“Théodoret de Cyr. Commentaire sur Isaïe, vols. 1–3”, Ed. Guinot, J.–N.
Paris: Cerf, 1:1980; 2:1982; 3:1984; Sources chrétiennes 276, 295, 315.
Section 14, line 394

                         Εἶτα καθολικῶς· Ἐγένετο τὰ γλυπτὰ 
αὐτῶν εἰς θηρία καὶ κτήνη. Οὐ γὰρ μόνον ἀνθρωπόμορφα 
κατεσκεύαζον εἴδωλα ἀλλὰ καὶ θηρίοις καὶ κτήνεσιν ἐοικότα· 
καὶ διαφερόντως Αἰγύπτιοι πιθήκων καὶ κυνῶν καὶ λεόντων 
καὶ προβάτων καὶ κροκοδίλων προσεκύνουν ἰνδάλματα, 
Ἀκαρωνῖται δὲ καὶ μυίας εἶχον εἰκόνα, ἄλλοι δὲ νυκτερίδων 
προσεκύνουν εἰκάσματα· καὶ τούτων ἐν τοῖς προοιμίοις ὁ 
προφητικὸς κατηγόρησε λόγος. 



Theodoretus Scr. Eccl., Theol., Commentaria in Isaiam 
Section 17, line 181

                    Ἀλλὰ τούτους καταλύσας ὁ δεσπότης 
Χριστὸς τὰ τούτων σκῦλα τοῖς ἀποστόλοις διένειμε, τοὺς 
μὲν Ῥωμαίων, τοὺς δὲ Αἰγυπτίων, τοὺς δὲ Ἰνδῶν διδασκά-
λους χειροτονήσας. 


Theodoretus Scr. Eccl., Theol., Quaestiones in Octateuchum (4089: 022)
“Theodoreti Cyrensis quaestiones in Octateuchum”, Ed. Fernández Marcos, N., Sáenz–Badillos, A.
Madrid: Poliglota Matritense, 1979; Textos y Estudios «Cardenal Cisneros» 17.


Theodoretus Scr. Eccl., Theol., Quaestiones in Octateuchum 
Page 15, line 17

              πληθυντικῶς δὲ πάλιν τὰς «<συναγωγὰς>« ὠνόμασεν, 
ἐπειδὴ ἄλλο μὲν τὸ Ἰνδικὸν πέλαγος, ἄλλο δὲ τὸ Ποντικὸν καὶ τὸ Τυρρηνι-
κὸν ἕτερον· καὶ ἄλλη μὲν ἡ Προποντίς, ἄλλος δὲ ὁ Ἑλλήσποντος, καὶ ὁ Αἰ-
γαῖος ἕτερος καὶ ἄλλος πάλιν ὁ Ἰώνιος κόλπος. 



Theodoretus Scr. Eccl., Theol., Quaestiones in libros Regnorum et Paralipomenon (4089: 023); MPG 80.
Volume 80, page 697, line 31

Σοφερὰ ποία ἐστίν;


 Χώρα τις ἔστι τῆς Ἰνδίας, ἣν οἱ γεωγράφοι χρυ-
σῆν ὀνομάζουσι γῆν. 



Theodoretus Scr. Eccl., Theol., Quaestiones in libros Regnorum et Paralipomenon 
Volume 80, page 697, line 36

                Ἐντεῦθεν δὲ αὐτοὺς κεῖσθαί φασι τῆς 
θαλάσσης τῆς Ἰνδικῆς. 



Theodoretus Scr. Eccl., Theol., Quaestiones in libros Regnorum et Paralipomenon 
Volume 80, page 700, line 29

Ποία πόλις ἐστὶν ἡ Θαρσεῖς;


 Ἐνταῦθα Ἰνδικήν τινα χώραν ὠνόμασεν. 



Theodoretus Scr. Eccl., Theol., Quaestiones in libros Regnorum et Paralipomenon 
Volume 80, page 837, line 34

           Καὶ ἤδη δὲ ἔφην, ὅτι πόλις ἦν αὕτη τῷ 
Ἰνδικῷ πελάγει παρακειμένη, Αἰθίοπας οἰκήτορας 
ἔχουσα. 



Theodoretus Scr. Eccl., Theol., Interpretatio in Psalmos (4089: 024); MPG 80.
Volume 80, page 1204, line 3

                                    Ὁ μὲν γὰρ 
ἔδραμε πρὸς Ἰνδοὺς, ὁ δὲ πρὸς Ἱσπανούς· καὶ ὁ 
μὲν τὴν Αἴγυπτον, ὁ δὲ τὴν Ἑλλάδα κατέλαβε· καὶ 
ἕτεροι μὲν τὴν Ἰουδαίαν ἐπιστεύθησαν ἄρδειν, ἕτε-
ροι δὲ Συρίαν καὶ Κιλικίαν, ἄλλοι δὲ ἄλλων 
ἐθνῶν τὴν γεωργίαν ἐνεχειρίσθησαν. 



Theodoretus Scr. Eccl., Theol., Interpretatio in Psalmos 
Volume 80, page 1384, line 29

                              Τούτους δὲ τοῖς ἱεροῖς 
διένειμεν ἀποστόλοις· τὸν μὲν Ῥωμαίων, τὸν δὲ Ἑλ-
λήνων διδάσκαλον προστησάμενος· καὶ τοὺς μὲν Ἰν-
δῶν, τοὺς δὲ Αἰγυπτίων κήρυκας ἀποφήνας. 



Theodoretus Scr. Eccl., Theol., Interpretatio in Psalmos 
Volume 80, page 1805, line 30

                        Τούτῳ πειθόμενοι τῷ νόμῳ, 
πᾶσαν γῆν καὶ θάλασσαν ἔδραμον, καὶ ὁ μὲν Ἰν-
δοὺς, ὁ δὲ Αἰγυπτίους, ὁ δὲ Αἰθίοπας προσήγαγε 
τῷ Χριστῷ. 



Theodoretus Scr. Eccl., Theol., Interpretatio in Psalmos 
Volume 80, page 1916, line 27

                                          Ἀτλαντικὸς 
γὰρ κόλπος, καὶ Ὠκεανὸς, καὶ Τυῤῥηνικὸς, καὶ Ἰό-
νιός τε, καὶ Αἰγαῖος, καὶ Ἀραβικὸς, καὶ Ἰνδικὸς, 
καὶ Εὔξεινος Πόντος, καὶ Προποντὶς, καὶ Ἑλλήσπον-
τος, καὶ ἕτερα πελάγη πολλαπλάσια τῶν εἰρημέ-
νων. 


Theodoretus Scr. Eccl., Theol., Interpretatio in Jeremiam 
Volume 81, page 736, line 37

                          Ἐμπόριον δὲ ἦν τοῦτο πά-
λαι λαμπρὸν, καὶ νῦν οἱ πρὸς Ἰνδοὺς ἀποπλέοντες 
ἐκεῖθεν ἀνάγονται. 




Theodoretus Scr. Eccl., Theol., Interpretatio in Ezechielem 
Volume 81, page 1036, line 36

Ἕλληνες μὲν γὰρ, καὶ Ῥωμαῖοι, τὸν Ἄρηα κατὰ 
τὸ ἴδιον αὑτῶν τῆς σκευῆς ἐξοπλίζουσι σχῆμα· 
Πέρσαι δὲ κατὰ τὸ σφέτερον· καὶ ἄλλως Ἰνδοὶ καὶ 
ἑτέρως Αἰθίοπες, ὡσαύτως δὲ καὶ ἕκαστον τῶν ὀνο-
μαζομένων ἐθνῶν. 




Theodoretus Scr. Eccl., Theol., Interpretatio in Ezechielem 
Volume 81, page 1084, line 13

                                               Αἰθιοπικὰ 
τοίνυν ταῦτα καὶ Ἰνδικὰ ἔθνη· δηλοῖ δὲ καὶ τὰ ἐκεῖ-
θεν κομιζόμενα, λίθοι τίμιοι, καὶ χρυσίον, καὶ ἡδύ-
σματα. 



Theodoretus Scr. Eccl., Theol., Interpretatio in Ezechielem 
Volume 81, page 1084, line 26

Σαβᾶ, καὶ Ἀσσοὺρ, καὶ Χαρμὰν ἔμποροί σου· 
(κδʹ.) Φέροντες ἐμπορίαν ἐν Μαχαλὶμ, καὶ ἐν 
Γαλιμά· ὑάκινθον, καὶ ποικιλίαν, καὶ θησαυ-
ροὺς ἐκλεκτοὺς ἐν μαγώζοις συγκειμένους, 
καταδεδεμένους ἐν σχοινίοις καὶ ἐν κυπα-
ρισσίνοις πλοίοις, ἐν αὐτοῖς ἡ ἐμπορία σου.  –  
Σαβᾶ ἔθνος Ἰνδικὸν ὀνομάζει· εὑρίσκομεν γὰρ καὶ 
ἐν τῇ τοῦ Σὴμ γενεαλογίᾳ τοῦτο τὸ ὄνομα· Ἀσσοὺρ 
δὲ τὸν Ἀσσύριον· Χαρμὰν δὲ τὴν λεγομένην Καρ-
μαήνην. 



Theodoretus Scr. Eccl., Theol., Interpretatio in Ezechielem 
Volume 81, page 1204, line 48

                        Σαβὰ δὲ, ὡς ἤδη ἔφην, 
ἔθνος ἐστὶν Αἰθιοπικὸν καὶ Ἰνδικόν· ἡ δὲ Δαιδὰν 
πλησιόχωρος τῆς Ἰδουμαίας· ἔστι δὲ καὶ ἄλλη Αἰ-
θιοπική· Θαρσεῖς δὲ ἡ Καρχηδών. 




Theodoretus Scr. Eccl., Theol., Interpretatio in xii prophetas minores (4089: 029); MPG 81.


Theodoretus Scr. Eccl., Theol., Interpretatio in xii prophetas minores 
Volume 81, page 1724, line 27

                                            Τὴν δὲ Θαρ-
σὶς, τινὲς μὲν Ταρσὸν ἀπὸ τῆς τοῦ ὀνόματος συγγε-
νείας ὑπέλαβον εἶναι, τινὲς δὲ τὴν Ἰνδίαν οὕτως 
ἔφασαν ὠνομᾶσθαι· συνιδεῖν οὐκ ἐθελήσαντες, ὡς 
τῶν Ἀσσυρίων ἡ βασιλεία τῆς Ἰνδῶν ἐστιν ὅμορος· 
ἔθος δὲ τοῖς φεύγουσι τὰ ἑῷα ἐπὶ τὴν ἑσπέραν χω-
ρεῖν, καὶ τοῖς τὰ νότια δραπετεύουσιν ἐπὶ τὰ βόρεια 
τρέχειν· ἄλλως τε καὶ εἰς τὴν Ἰόππην κατῆλθε, 
πόλιν παραλίαν τῆς Παλαιστίνης, ἵνα ἐκεῖθεν ἀπάρῃ· 
ἐπίκειται δὲ τῇ πρὸς ἑσπέραν κειμένῃ θαλάττῃ. 



Theodoretus Scr. Eccl., Theol., Interpretatio in xii prophetas minores 
Volume 81, page 1724, line 36

Διὰ τούτου δὲ τοῦ πελάγους οὐκ ἂν ναυτιλίᾳ τις χρώ-
μενος εἰς Ἰνδίαν ἀπέλθοι· μεταξὺ γὰρ τῆς τε ἡμε-
τέρας θαλάττης, καὶ τῆς Ἰνδικῆς, ἤπειρός ἐστι με-
γίστη, ἡ μὲν οἰκουμένη, ἡ δὲ παντελῶς ἔρημος· καὶ 
ὄρη δὲ πλεῖστα καὶ μέγιστα, μεθ' ἃ τῆς Ἐρυθρᾶς 
θαλάσσης ὁ κόλπος, ᾧ τὸ Ἰνδικὸν ἀναμέμικται πέ-
λαγος. 



Theodoretus Scr. Eccl., Theol., Interpretatio in xii prophetas minores 
Volume 81, page 1725, line 5

                          Ἐξ ὧν ποδηγηθέντες, τὸν 
μακάριόν φαμεν Ἰωνᾶν οὐκ εἰς Ἰνδίαν, ἀλλ' εἰς 
Καρχηδόνα ποιήσασθαι τὴν φυγήν. 



Theodoretus Scr. Eccl., Theol., Interpretatio in xiv epistulas sancti Pauli (4089: 030); MPG 82.

Theodoretus Scr. Eccl., Theol., Interpretatio in xiv epistulas sancti Pauli 
Volume 82, page 337, line 44

                        Ἐδόθη γὰρ τοῦτο τοῖς κήρυξι 
διὰ τὰς διαφόρους τῶν ἀνθρώπων φωνάς· ἵνα πρὸς 
Ἰνδοὺς ἀφικνούμενοι, τῇ ἐκείνων χρώμενοι γλώττῃ, 
τὸ θεῖον προσφέρωσι κήρυγμα· καὶ Πέρσαις πάλιν 
διαλεγόμενοι, καὶ Σκύθαις, καὶ Ῥωμαίοις, καὶ Αἰ-
γυπτίοις, ταῖς ἑκάστων κεχρημένοι φωναῖς τὴν εὐ-
αγγελικὴν διδασκαλίαν κηρύττωσι. 



Theodoretus Scr. Eccl., Theol., Haereticarum fabularum compendium (4089: 031); MPG 83.
Volume 83, page 381, line 1

                                   Καὶ τὸν μὲν   
Ἀλδὰν Σύροις ἀπέστειλε κήρυκα, Ἰνδοῖς δὲ τὸν 
Θωμᾶν. 



Theodoretus Scr. Eccl., Theol., De providentia orationes decem (4089: 032); MPG 83.

\end{greek}

\section{Paulus (med.)}

\blockquote[From Wikipedia\footnote{\url{http://en.wikipedia.org/wiki/Paul_of_Aegina}.}]{Paul of Aegina or Paulus Aegineta (Aegina, 625?–690?) was a 7th-century Byzantine Greek physician best known for writing the medical encyclopedia Medical Compendium in Seven Books. For many years in the Byzantine Empire, this work contained the sum of all Western medical knowledge and was unrivaled in its accuracy and completeness.



The Medical Compendium in Seven Books is a medical treatise written in Greek the 7th century CE by Paul of Aegina a.k.a. Paulus Aegineta. The title in Greek is Epitomes iatrikes biblio hepta.

     

Although Byzantine medicine drew largely on ancient Greek and Roman knowledge, however, his works also contained many new ideas as he was a teacher from Alexandria. For example, in several volumes Paul of Aegina talks about bone structure and fractures, as shown below: \ldots}


Paulus Med., Epitomae medicae libri septem (0715: 001)
“Paulus Aegineta, 2 vols.”, Ed. Heiberg, J.L.
Leipzig: Teubner, 9.1:1921; 9.2:1924; Corpus medicorum Graecorum, vols. 9.1 \& 9.2.

\begin{greek}

Paulus Med., Epitomae medicae libri septem 
Book 2, chapter 53, section 1, line 10

                            Ἀρχιγένης δέ φησιν· καὶ ὁ ἃλς ὁ Ἰνδικός, 
χρόᾳ μὲν καὶ συστάσει ὅμοιος τῷ κοινῷ ἁλί, γεύσει δὲ μελιτώδης, φα-
κοῦ δὲ μέγεθος ἢ τό γε πλεῖστον κυάμου, διατρωχθεὶς σφόδρα καθυ-
γραίνειν δύναται. 



Paulus Med., Epitomae medicae libri septem 
Book 3, chapter 22, section 16, line 7

          στίμμεως ὀπτοῦ πεπλυμένου 𐅻 <α>, μολίβδου κεκαυμένου καὶ 
πεπλυμένου 𐅻 <δ>, κρόκου 𐅻 <δ>, νάρδου Ἰνδικῆς 𐅻 <γ>· λείοις χρῶ. 



Paulus Med., Epitomae medicae libri septem 
Book 3, chapter 24, section 8, line 7

                                            χαλκῖτιν λεάνας ἀνάλαβε δεδευμένῳ 
ὕδατι ἐλλυχνίῳ ἢ πριαπίσκῳ καὶ ἐντίθει τοῖς μυξωτῆρσιν ἢ ᾠοῦ τὸ 
ὄστρακον καύσας μίσγε αὐτῷ καὶ κηκῖδος τὸ ἥμισυ καὶ ὡσαύτως χρῶ, 
ἢ λυκίῳ Ἰνδικῷ διάψα ἢ ὀνίδα καύσας ἐμφύσα τὴν σποδὸν ἢ χυλίσας 
τὴν ὀνίδα ἔνσταζε τὸν χυλόν, ἢ μυλίτου λίθου σβεσθέντος ὄξει τὴν ἀτ-
μίδα ὀσφραινέσθω, ἢ καὶ παρεμπλαστικῷ τούτῳ κέχρησο· μάννης λιβά-
νου μέρος <α>, ἀλόης μέρος 𐅵ʹ, ᾠοῦ τῷ λευκῷ ἀναλάμβανε καὶ χρῶ δι' 
ἐλλυχνιωτοῦ προστιθεὶς ἔξωθεν τῷ χρίσματι λαγωοῦ τρίχας, ἢ τὴν κα-
λουμένην λυχνίδα ἔνθες τῷ μυκτῆρι, ἢ σικύαν κούφην κατὰ τοῦ κατ' 
εὐθὺ τοῦ αἱμορραγοῦντος μυκτῆρος ὑποχονδρίου πρόσθες μεγάλην τε 
καὶ ἐπιμόνως, ἢ τὰ ὦτα στερρῶς ἐμφραττέτω, καὶ τὸ μέτωπον σπόγ-  
γοις ἐξ ὕδατος ψυχροῦ καταβρεχέσθω, ἢ σικύαν ἰνίῳ κολλᾶν μεθ' αἵ-
ματος ἀφαιρέσεως, ἔσθ' ὅτε δὲ καὶ φλεβοτομεῖν, εἰ μηδὲν κωλύει, καὶ 




Paulus Med., Epitomae medicae libri septem 
Book 3, chapter 37, section 2, line 3

                                                                       ψυκτικὸν 
δὲ καὶ τονωτικὸν πλαδῶντος στομάχου τοῦτο· ῥόδων χλωρῶν τῶν 
φύλλων 𐅻 <ϛ>, γλυκυρίζης χυλοῦ 𐅻 <δ>, νάρδου Ἰνδικῆς 𐅻 <δ>· οἴνῳ γλυκεῖ 
ἀναλάμβανε καὶ ποίει ὑπογλώττια. 



Paulus Med., Epitomae medicae libri septem 
Book 3, chapter 42, section 3, line 4

            κάλλιστος δὲ καὶ οὗτος· ὀπίου, κρόκου, λυκίου Ἰνδικοῦ, 
ἀκακίας, ῥοός, λιβάνου, κηκῖδος, ὑποκιστίδος, σιδίων, σμύρνης, ἀλόης 
ἴσα· ὕδατι δίδου τριώβολον. 



Paulus Med., Epitomae medicae libri septem 
Book 3, chapter 46, section 6, line 13

                                                                       <δ>· ὀξύμελί τε δο-
τέον αὐτοῖς ἁπλᾶ τε βοηθήματα, οἷον ἄσαρον, νάρδον Κελτικὴν ἢ Ἰν-
δικήν, σχοῖνον, πετροσέλινον· καὶ δεῖ ἐρεθίζειν τὴν γαστέρα διὰ τῆς 
ἀκαλήφης ἢ λινοζώστεως ἐσθιομένων ἑφθῶν. 



Paulus Med., Epitomae medicae libri septem 
Book 3, chapter 62, section 2, line 8

                                                                                 ἐνεργῶς 
δὲ πρὸς τοῦτο ποιοῦσι καὶ αἱ κολλητικαὶ τῶν ἐμπλάστρων, οἷον ἁρμο-
νία, Ἱκέσιος, Ἀθηνᾶ, μηλίνη, Ἰνδικὴ καὶ αἱ παραπλήσιοι. 



Paulus Med., Epitomae medicae libri septem 
Book 4, chapter 1, section 7, line 5

                                                                    αἱ δὲ ὀχθώδεις 
ἐπαναστάσεις φλεγμαίνουσαι ἢ εἱλκωμέναι καταχριέσθωσαν λυκίῳ Ἰν-
δικῷ ἢ γλαυκίῳ ἢ ἀλόῃ ἢ τῷ Ἀνδρωνίῳ τροχίσκῳ ἢ τῇ Πολυείδου 
σφραγίδι ἢ καταπλασσέσθωσαν χόνδρῳ μετὰ χυλοῦ πολυγόνου ἢ ἀρνο-
γλώσσου ἢ ἑλξίνῃ λείᾳ. 



Paulus Med., Epitomae medicae libri septem 
Book 4, chapter 41, section 2, line 7

ἔτι δὲ πρὸς τὰ ῥυπαρὰ τῶν ἑλκῶν ἥ τε Αἰγυπτία ποιεῖ καὶ αἱ 
δι' ἁλῶν κηρωταὶ συντακεῖσαι ἥ τε Ἰνδικὴ καὶ ἡ Ἀθηνᾶ καὶ αἱ χλωραὶ 
ἀνιέμεναι τό τε διὰ κισήρεως καὶ τὰ δι' ὀρόβου ξηρὰ καὶ ὁ μελάγχλωρος 
τροχίσκος· ὁμοίως δὲ καὶ ὁ κριογενής. 



Paulus Med., Epitomae medicae libri septem 
Book 4, chapter 54, section 9, line 5

                                                            τινὲς δὲ καὶ πολυ-
συνθέτοις ἐπὶ αὐτῶν εἰώθασι χρῆσθαι φαρμάκοις, ὁποῖά ἐστι τό τε διὰ 
τῶν μεταλλικῶν καὶ αἱ βάρβαροι προσαγορευόμεναι τό τε κίσσινον καὶ 
τὸ διὰ τοῦ ἠριγέροντος καὶ τὸ μελάγχλωρον ἥ τε Ἰνδικὴ καὶ ἡ ἁρμο-
νία καὶ ἡ Ἀθηνᾶ, ὧν τὰς συνθέσεις καὶ τὸν τῆς χρήσεως τρόπον κατὰ 
τὸ ἕβδομον εὑρήσεις βιβλίον. 



Paulus Med., Epitomae medicae libri septem 
Book 4, chapter 58, section 1, line 1

                             >


 Ἐν Ἰνδικῇ καὶ τοῖς ἄνω τῆς Αἰγύπτου τόποις τὰ λεγόμενα δρα-
κόντια συνίστανται, καθάπερ ἑλμινθώδη τινὰ ζῷα, ἐν τοῖς μυώδεσι τῶν   
μορίων, οἷον βραχίοσι, μηροῖς, κνήμαις, ἐπὶ δὲ τῶν παιδίων καὶ πλευ-
ροῖς, ὑπὸ τῷ δέρματι συνιστάμενα καὶ κινούμενα σαφῶς· εἶθ' ὅταν 
χρονίσῃ, κατά τι πέρας τοῦ δρακοντίου πυοῦται ὁ τόπος, καὶ τοῦ 
δέρματος ἀναστομουμένου ἔξω προέρχεται τοῦ δρακοντίου ἡ ἀρχή, 
ἑλκόμενον δὲ τὸ δρακόντιον ἀλγηδόνας ἐμποιεῖ, καὶ μάλιστα ὅταν 
ἀπορραγείη. 



Paulus Med., Epitomae medicae libri septem 
Book 7, chapter 3, section 1, line 7

Ἀγάλοχον ξύλον ἐστὶν Ἰνδικὸν παραπλήσιον θύᾳ εὐῶδες, ὃ δια-
μασώμενον πρὸς εὐωδίαν στόματος ποιεῖ· ἔστι δὲ καὶ θυμίαμα. 



Paulus Med., Epitomae medicae libri septem 
Book 7, chapter 3, section 10, line 87

Καρυόφυλλον οὐ πρὸς τοὔνομα καὶ τὴν οὐσίαν ἔχει, ἀλλ' ἐκ τῆς 
Ἰνδίας οἷον ἄνθη τινὰ δένδρου καρφοειδῆ μέλανα, ὅσον δακτύλου σύν-
εγγυς τὸ μῆκος, φέρεται ἀρωματίζοντα καὶ δριμέα, ὑπόπικρα, θερμά 
τε καὶ ξηρὰ περί που τρίτης τάξεως· ἃ πολύχρηστά ἐστιν ἐν ὄψοις 
τε καὶ ἑτέροις φαρμάκοις. 



Paulus Med., Epitomae medicae libri septem 
Book 7, chapter 3, section 11, line 120

                                 τὸν ἱερακίτην δὲ καὶ Ἰνδικὸν λίθον φασὶ 
περιαπτόμενον τὸ ἐκ τῶν αἱμορροΐδων ἱστᾶν αἷμα, τὸν δὲ σάπφειρον 
πινόμενον τοὺς ὑπὸ σκορπίου πληγέντας ὠφελεῖν καὶ τὸν ἀφροσέ-
λινον τοὺς ἐπιλήπτους. 



Paulus Med., Epitomae medicae libri septem 
Book 7, chapter 3, section 11, line 153

Λύκιον ἐξ ἑτερογενῶν σύγκειται δυνάμεων, τῆς μὲν θερμῆς τε 
καὶ λεπτομεροῦς καὶ διαφορητικῆς, τῆς δὲ γεώδους ψυχρᾶς καὶ ἠρέμα 
στυφούσης, ὥστε ξηραίνειν κατὰ τὴν δευτέραν ἀπόστασιν, κατὰ δὲ τὸ 
θερμαίνειν καὶ ψύχειν μέσον· διόπερ ὡς ῥυπτικῷ μὲν αὐτῷ χρῶνται 
ἐπὶ τῶν ἐπισκοτούντων ταῖς κόραις, ὡς δὲ στυπτικῷ ἐπὶ κοιλιακῶν τε 
καὶ δυσεντερικῶν καὶ τῶν κακοήθων ἑλκῶν, ἐπὶ δὲ φλεγμονῶν ὡς 
διαφοροῦντι· προτερεύει δὲ τὸ Ἰνδικόν. 



Paulus Med., Epitomae medicae libri septem 
Book 7, chapter 3, section 12, line 2

                            >


 Μάκερ φλοιός ἐστιν ἐκ τῆς Ἰνδικῆς κομιζόμενος, ξηραίνων μὲν 
κατὰ τὴν τρίτην τάξιν, μέσος δὲ κατὰ θερμότητα καὶ ψῦξιν· ἔστι 
δὲ καὶ στυπτικὸς λεπτομερής· ὅθεν κοιλιακοῖς τε καὶ δυσεντερικοῖς 
ἁρμόττει. 



Paulus Med., Epitomae medicae libri septem 
Book 7, chapter 3, section 12, line 27

Μέλαν Ἰνδικόν, ὥς φησι Διοσκουρίδης, τῶν ψυχόντων ἐλαφρῶς 
ἐστι καὶ ῥυσούντων φλεγμονὰς καὶ οἰδήματα ἕλκη τε ἀνακαθαιρόντων. 



Paulus Med., Epitomae medicae libri septem 
Book 7, chapter 3, section 13, line 2

                            >


 Ναόκαφθον, οἱ δὲ νάκαφθον, Ἰνδικόν ἐστιν ἄρωμα πρὸς μήτραν 
ἐστεγνωμένην ὑπατμιζόμενον. 



Paulus Med., Epitomae medicae libri septem 
Book 7, chapter 3, section 13, line 9

     ἰσχυροτέρα δέ ἐστιν ἡ Ἰνδικὴ τῆς Συριακῆς καὶ μελαντέρα. 



Paulus Med., Epitomae medicae libri septem 
Book 7, chapter 3, section 15, line 36

Ὄνυχες πώματά εἰσι κογχυλίων Ἰνδικῶν, οἳ θυμιαθέντες ἐγεί-
ρουσι τάς τε ὑστερικῶς πνιγομένας καὶ ἐπιληπτικούς, ποθέντες δὲ 
κοιλίαν ταράσσουσιν. 



Paulus Med., Epitomae medicae libri septem 
Book 7, chapter 11, section 2, line 4

                                                       >


 Ἀσπαλάθου ῥίζης φλοιοῦ, καλάμου ἀρωματικοῦ, κόστου, ἀσάρου, 
ξυλοβαλσάμου, φοῦ, ἀμαράκου, μαστίχης ἀνὰ 𐅻 <ϛ>, σχοίνου ἄνθους 𐅻 <ιβ>, 
κιναμώμου 𐅻 <κδ>, ἀμώμου, κασσίας, ῥέου ἀνὰ 𐅻 <κ>, νάρδου Ἰνδικῆς, 
φύλλου ἀνὰ 𐅻 <ιβ>, σμύρνης 𐅻 <κδ>, κρόκου 𐅻 <ιβ>· οἴνῳ καλῷ ἀναλάμβανε 
καὶ ἀνάπλασσε τροχίσκους ὀποβαλσάμου παραπτόμενος. 



Paulus Med., Epitomae medicae libri septem 
Book 7, chapter 12, section 6, line 4

                         >


 Ἀκακίας, κόμμεως, ῥόδων ἄνθους, βαλαυστίων, ὑποκιστίδος χυλοῦ, 
κηκίδων ἀνὰ 𐅻 <γ>, ῥόδων χλωρῶν χυλοῦ, ἀρνογλώσσου σπέρματος ἀνὰ 
𐅻 <β>, λυκίου Ἰνδικοῦ 𐅻 <α>. 



Paulus Med., Epitomae medicae libri septem 
Book 7, chapter 14, section 10, line 7

                                  >


 Ἀνίσου σπέρματος, σελίνου σπέρματος, σχοίνου ἄνθους, ἄμεως σπέρ-
ματος, στυπτηρίας σχιστῆς, ἴρεως Ἰλλυρικῆς, βησασά, ὅ τινες ἁρμαλὰ 
καλοῦσιν (ἔστι δὲ τὸ ἄγριον πήγανον), κιναμώμου, σμύρνης τρωγλίτιδος, 
κρόκου, κηκῖδος ἀνὰ 𐆄 <α>, ἀριστολοχίας μακρᾶς, κασσίας, κροκομάγματος, 
ῥόδων ξηρῶν ἀνὰ 𐆄 <β>, κόστου, χελιδόνων σποδοῦ προσφάτου ἀνὰ 𐆄 <γ>, 
νάρδου Ἰνδικῆς, ἀμώμου ἀνὰ 𐆄 𐅵ʹ, μέλιτος τὸ ἀρκοῦν. 



Paulus Med., Epitomae medicae libri septem 
Book 7, chapter 16, section 12, line 3

                                                             >


 Καδμίας ἁπαλῆς 𐅻 <κδ>, ψιμυθίου 𐅻 <ιϛ>, ἰοῦ ξυστοῦ 𐅻 <ιβ>, στίμμεως 
𐅻 <η>, στυπτηρίας σχιστῆς 𐅻 <γ>, χαλκίτεως κεκαυμένης 𐅻 <γ>, νάρδου Ἰνδι-
κῆς 𐅻 <δ>, ὀμφακίου 𐅻 <β>, χαλκοῦ 𐅻 <α>, λεπίδος χαλκοῦ 𐅻 <η>, ἐρείκης καρ-
ποῦ 𐅻 <ιγ>, ὀποῦ μήκωνος 𐅻 <κδ>, κρόκου 𐅻 <δ>, καστορίου 𐅻 <γ>, σμύρνης 𐅻 <ϛ>, 
λυκίου Ἰνδικοῦ, ἀκακίας, κόμμεως ἀνὰ 𐅻 <δ>, ῥόδων νεαρῶν 𐅻 <οβ>· οἴνῳ 
Φαλερίνῳ ἢ Σουρεντίνῳ ἢ Ἀμινναίῳ ἢ Χίῳ αὐστηρῷ λείου. 



Paulus Med., Epitomae medicae libri septem 
Book 7, chapter 16, section 24, line 6

                                       >


 Ἀκακίας, ναρδοστάχυος, λιβάνου ἀνὰ 𐅻 <η>, χαλκοῦ κεκαυμένου καὶ 
πεπλυμένου, στίμμεως κεκαυμένου καὶ πεπλυμένου, ψιμυθίου κεκαυ-
μένου καὶ πεπλυμένου, καδμίας ἀνὰ 𐅻 <ιβ>, σμύρνης, ὀπίου πεφωγμένου 
ἀνὰ 𐅻 <δ>, κρόκου 𐅻 <ε>, ἰοῦ ξυστοῦ 𐅻 <γ>, λίθου σχιστοῦ, λεπίδος ἐρυθρᾶς, 
λυκίου Ἰνδικοῦ, ὀμφακίου ἀνὰ 𐅻 <α>, καστορίου, ῥόδων ἄνθους ἀνὰ 𐅻 <β>, 
φοινικοβαλάνων 𐅻 <δ>, ὁμοίως τὰ ὀστᾶ τῶν φοινίκων κεκαυμένα ἀριθμῷ <ε>, 
κόμμεως 𐆄 <ε>· ὕδωρ ὄμβριον, ἔστωσαν δὲ εἰς τὸ ὕδωρ ἐμβρεχόμενα 
τρία νυχθήμερα καλάμου ἀρωματικοῦ, ὑοσκυάμου σπέρματος, ῥόδων 
ξηρῶν ἀνὰ 𐅻 <δ>, φύλλου 𐅻 <α>. 



Paulus Med., Epitomae medicae libri septem 
Book 7, chapter 16, section 44, line 2

    
<Τὸ διὰ χυλοῦ μαράθρου.>


 Καδμίας 𐅻 <ιζ>, μέλανος Ἰνδικοῦ 𐅻 <ιϛ>, πεπέρεως μακροῦ 𐅻 <ιγ> καὶ 
λευκοῦ 𐅻 <ιβ>, ὀποῦ Κυρηναϊκοῦ 𐅻 <η>, ὀποβαλσάμου 𐅻 <ϛ>, ναρδοστάχυος 
𐅻 <ϛ>, σαγαπηνοῦ, ὀποπάνακος ἀνὰ 𐅻 <ε>, ὀπίου 𐅻 <δ>, εὐφορβίου 𐅻 <α>, κόμ-
μεως 𐅻 <α>· λείου χυλῷ μαράθρου. 



Paulus Med., Epitomae medicae libri septem 
Book 7, chapter 16, section 46, line 2

                             >


 Καδμίας 𐆄 <η>, ἰοῦ 𐆄 <β>, μέλανος Ἰνδικοῦ 𐆄 <η>, πεπέρεως λευκοῦ 𐆄 <δ>, 
ὀποῦ Μηδικοῦ 𐆄 <β>, ὀποβαλσάμου 𐆄 <β>, κόμμεως 𐆄 <ϛ>· ὕδωρ. 



Paulus Med., Epitomae medicae libri septem 
Book 7, chapter 16, section 48, line 3

                                            >


 Καδμίας 𐅻 <ιϛ>, χαλκοῦ κεκαυμένου καὶ πεπλυμένου 𐅻 <ιδ>, ὀπίου, 
λυκίου Ἰνδικοῦ, μαλαβάθρου, νάρδου Ἰνδικῆς, κρόκου, ἀλόης ἀνὰ 𐅻 <β>, 
καστορίου 𐅻 <η>, σμύρνης 𐅻 <δ>, ἀκακίας, στίμμεως ἀνὰ 𐅻 <μ>· ὕδατι. 



Paulus Med., Epitomae medicae libri septem 
Book 7, chapter 16, section 50, line 4

                              >


 Χαλκοῦ κεκαυμένου, καδμίας πλακίτιδος ἀνὰ 𐆄 <θ>, λίθου αἱματίτου 
πεπλυμένου 𐆄 <ϛ>, κρόκου, σμύρνης, ἀλόης, ἀμμωνιακοῦ θυμιάματος ἀνὰ 
𐆄 <γ>, λυκίου Ἰνδικοῦ, ναρδοστάχυος ἀνὰ 𐆄 <α> 𐅵ʹ, πεπέρεως λευκοῦ κόκκους 
<ρν>, ἀκακίας κιρρᾶς 𐆄 <θ>, κόμμεως 𐆄 <γ>· οἴνῳ Φαλερινῷ ἢ Ἀμινναίῳ λείου. 



Paulus Med., Epitomae medicae libri septem 
Book 7, chapter 17, section 56, line 1

<Ἡ Ἰνδὴ κολλητικὴ πρὸς νομάς τε καὶ αἱμοπτυϊκούς. 




Paulus Med., Epitomae medicae libri septem 
Book 7, chapter 18, section 4, line 3

                                                           >


 Κηροῦ, τερεβινθίνης, βδελλίου, ἀμμωνιακοῦ θυμιάματος, καρδαμώ-
μου, κυπέρου ἀνὰ μνᾶν <α>, ἀμώμου, νάρδου Ἰνδικῆς, κρόκου, σμύρνης, 
λιβάνου, ξυλοκιναμώμου ἀνὰ 𐅻 <κε>, ἐλαίου κυπρίνου κοτύλην <α>, οἴνου 
Ἰταλικοῦ ὅσον ἐξαρκεῖ· σκεύαζε καὶ χρῶ, ποτὲ μὲν ἀκράτῳ, ἔστι δ' 
ὅτε ἀνειμένῳ κηρωτῇ κυπρίνῃ. 



Paulus Med., Epitomae medicae libri septem 
Book 7, chapter 25, section 2, line 3

ἀντὶ ἀλόης Ἰνδικῆς ἀλόης χλωρᾶς φύλλα ἢ γλαυκίας ἢ κόπρος 
  ἴβεως ἢ λύκιον ἢ κενταύριον. 



Paulus Med., Epitomae medicae libri septem 
Book 7, chapter 25, section 2, line 19

ἀντὶ ἀμυγδάλων πικρῶν ἀψίνθιον 
 ἀντὶ ἀρμενίου μέλαν Ἰνδικόν. 



Paulus Med., Epitomae medicae libri septem 
Book 7, chapter 25, section 10, line 9

ἀντὶ κροκομάγματος ἀλόη Ἰνδική. 



Paulus Med., Epitomae medicae libri septem 
Book 7, chapter 25, section 12, line 1

ἀντὶ μαλαβάθρου κασσία ἢ νάρδος Ἰνδική. 

\end{greek}

\section{Nonnosus}

Late antique.

\blockquote[From New Pauly's]{Author of a lost Greek report on the travels of a legation to the ruler of Kinda in central Arabia and then to Ethiopia and southern Arabia in the year AD 530/1, the existence of which is known only from the  ‘Library of Photius (cod. 3). Similar journeys had been undertaken by 502 by N.'s grandfather Euphrasius, and several in 524 and later by his father Abram. According to Photius, the report emphasised the courage of N. in hazardous situations and contained information on the r…}

%See also \url{http://www.amazon.com/The-Mediterranean-World-Late-Antiquity/dp/0415579619/ref=sr_1_1?ie=UTF8&qid=1347514741&sr=8-1&keywords=Nonnosus+%28historian%29} and \url{http://www.amazon.com/Political-Communication-Antique-Cambridge-Medieval/dp/0521096383/ref=sr_1_2?ie=UTF8&qid=1347514741&sr=8-2&keywords=Nonnosus+%28historian%29}.

Nonnosus Hist., Fragmenta (4393: 001)
“FHG 4”, Ed. Müller, K.
Paris: Didot, 1841–1870.
Fragment 1, line 6

\begin{greek}

Malalas Chron.: 
Ὁ δὲ βασιλεὺς Ῥωμαίων ἀκούσας 
παρὰ τοῦ πατρικίου Ῥουφίνου τὴν παρὰ Κωάδου, βα-
σιλέως Περσῶν, παράβασιν, ποιήσας θείας κελεύσεις 
κατέπεμψε πρὸς τὸν βασιλέα τῶν Αὐξουμιτῶν· ὅστις 
βασιλεὺς Ἰνδῶν συμβολὴν ποιήσας μετὰ τοῦ βασιλέως 
τῶν Ἀμεριτῶν (***) Ἰνδῶν, κατὰ κράτος νικήσας παρέ-
λαβε τὰ βασίλεια αὐτοῦ καὶ τὴν χώραν αὐτοῦ πᾶσαν, 
καὶ ἐποίησεν ἀντ' αὐτοῦ βασιλέα τῶν Ἀμεριτῶν Ἰνδῶν 
ἐκ τοῦ ἰδίου γένους Ἀγγάνην διὰ τὸ εἶναι καὶ τὸ τῶν 
Ἀμεριτῶν Ἰνδῶν βασίλειον ὑπ' αὐτόν. 



Nonnosus Hist., Fragmenta 
Fragment 1, line 11

Ὁ δὲ βασιλεὺς Ῥωμαίων ἀκούσας 
παρὰ τοῦ πατρικίου Ῥουφίνου τὴν παρὰ Κωάδου, βα-
σιλέως Περσῶν, παράβασιν, ποιήσας θείας κελεύσεις 
κατέπεμψε πρὸς τὸν βασιλέα τῶν Αὐξουμιτῶν· ὅστις 
βασιλεὺς Ἰνδῶν συμβολὴν ποιήσας μετὰ τοῦ βασιλέως 
τῶν Ἀμεριτῶν (***) Ἰνδῶν, κατὰ κράτος νικήσας παρέ-
λαβε τὰ βασίλεια αὐτοῦ καὶ τὴν χώραν αὐτοῦ πᾶσαν, 
καὶ ἐποίησεν ἀντ' αὐτοῦ βασιλέα τῶν Ἀμεριτῶν Ἰνδῶν 
ἐκ τοῦ ἰδίου γένους Ἀγγάνην διὰ τὸ εἶναι καὶ τὸ τῶν 
Ἀμεριτῶν Ἰνδῶν βασίλειον ὑπ' αὐτόν. 



Nonnosus Hist., Fragmenta 
Fragment 1, line 13

                                              Καὶ ἀποπλεύ-
σας ὁ πρεσβευτὴς Ῥωμαίων ἐπὶ Ἀλεξανδρείαν διὰ τοῦ 
Νείλου ποταμοῦ καὶ τῆς Ἰνδικῆς θαλάσσης κατέφθασε 
τὰ Ἰνδικὰ μέρη. 



Nonnosus Hist., Fragmenta 
Fragment 1, line 15

                     Καὶ εἰσελθὼν παρὰ τῷ βασιλεῖ 
τῶν Ἰνδῶν, μετὰ χαρᾶς πολλῆς ἐξενίσθη ὁ βασιλεὺς 
Ἰνδῶν, ὅτι διὰ πολλῶν χρόνων ἠξιώθη μετὰ τοῦ βασι-
λέως Ῥωμαίων κτήσασθαι φιλίαν. 



Nonnosus Hist., Fragmenta 
Fragment 1, line 19

                                     Ὡς δὲ ἐξηγήσα-
το (*) ὁ αὐτὸς πρεσβευτὴς, ὅτε ἐδέξατο αὐτὸν ὁ τῶν 
Ἰνδῶν βασιλεὺς, ὑφηγήσατο τὸ σχῆμα τῆς βασιλι-
κῆς τῶν Ἰνδῶν καταστάσεως, ὅτι γυμνὸς ὑπῆρχε, 
καὶ κατὰ τοῦ ζώσματος εἰς τὰς ψύας αὐτοῦ λινόχρυσα 
ἱμάτια, κατὰ δὲ τῆς γαστρὸς καὶ τῶν ὤμων φορῶν 
σχιαστὰς διὰ μαργαριτῶν καὶ 
κλαβία ἀνὰ πέντε, καὶ 
χρυσᾶ ψέλια εἰς τὰς χεῖρας αὐτοῦ, ἐν δὲ τῇ κεφαλῇ 
αὐτοῦ λινόχρυσον φακιόλιον ἐσφενδονισμένον, ἔχον ἐξ 
ἀμφοτέρων τῶν μερῶν σειρὰς τέσσαρας, καὶ μανιάκιν 
χρυσοῦν ἐν τῷ τραχήλῳ αὐτοῦ· καὶ ἵστατο ὑπεράνω 




Nonnosus Hist., Fragmenta 
Fragment 1, line 33

                             Καὶ ἵστατο ἐπάνω ὁ βασιλεὺς 
τῶν Ἰνδῶν βαστάζων σκουτάριον μικρὸν κεχρυσωμένον 
καὶ δύο λαγκίδια καὶ αὐτὰ κεχρυσωμένα κατέχων ἐν 
ταῖς χερσὶν αὐτοῦ. 



Nonnosus Hist., Fragmenta 
Fragment 1, line 39

Καὶ εἰσενεχθεὶς ὁ πρεσβευτὴς Ῥωμαίων κλίνας τὸ 
γόνυ προσεκύνησε· καὶ ἐκέλευσεν ὁ βασιλεὺς Ἰνδῶν 
ἀναστῆναί με καὶ ἀναχθῆναι πρὸς αὐτόν. 



Nonnosus Hist., Fragmenta 
Fragment 1, line 50

                                                      Λύ-
σας δὲ καὶ ἀναγνοὺς δι' ἑρμηνέως τὰ γράμματα, εὗρε   
περιέχοντα ὥστε ὁπλίσασθαι αὐτὸν κατὰ Κωάδου, βα-
σιλέως Περσῶν, καὶ τὴν πλησιάζουσαν αὐτῷ χώραν 
ἀπολέσαι καὶ τοῦ λοιποῦ μηκέτι συνάλλαγμα ποιῆσαι 
μετ' αὐτοῦ, ἀλλὰ δι' ἧς ὑπέταξε χώρας τῶν Ἀμεριτῶν 
Ἰνδῶν διὰ τοῦ Νείλου ἐπὶ τὴν Αἴγυπτον ἐν Ἀλεξαν-
δρείᾳ τὴν πραγματείαν ποιεῖσθαι. 



Nonnosus Hist., Fragmenta 
Fragment 1, line 52

                                      Καὶ εὐθέως ὁ βασι-
λεὺς Ἰνδῶν Ἐλεσβόας 
ἐπ' ὄψεσι τοῦ πρεσβευτοῦ Ῥωμαίων 
ἐκίνησε πόλεμον κατὰ Περσῶν, προπέμψας δὲ τοὺς ὑπ' 
αὐτὸν Ἰνδοὺς Σαρακηνοὺς, ἐπῆλθε τῇ Περσικῇ χώρᾳ 
ὑπὲρ Ῥωμαίων, δηλώσας τῷ βασιλεῖ Περσῶν τοῦ δέ-
ξασθαι τὸν βασιλέα Ἰνδῶν πολεμοῦντα αὐτῷ καὶ ἐκ-
πορθῆσαι πᾶσαν τὴν ὑπ' αὐτοῦ βασιλευομένην γῆν. 



Nonnosus Hist., Fragmenta 
Fragment 1, line 59

                                                        Καὶ 
πάντων οὕτως προβάντων, ὁ βασιλεὺς Ἰνδῶν κρατήσας 
τὴν κεφαλὴν τοῦ πρεσβευτοῦ Ῥωμαίων, δεδωκὼς εἰρή-
νης φίλημα, ἀπέλυσεν ἐν πολλῇ θεραπείᾳ. 



Nonnosus Hist., Fragmenta 
Fragment 1, line 62

                                                Κατέπεμψε 
γὰρ καὶ σάκρας διὰ Ἰνδοῦ πρεσβευτοῦ καὶ δῶρα τῷ 
βασιλεῖ Ῥωμαίων. 
\end{greek}

\section{Epiphanius}%???

Which of the many Ipiphanii is this?

\blockquote[From New Pauly's]{(Ἀλλογενής; Allogenḗs, the ‘different’). Name of  Seth as son of Adam and Eve in Sethian  Gnosticism (Epiphanius, Panarii libri 40,7,2). His seven sons are the Allogeneis (40,7,5). Books are also ascribed to him, which are likewise called Allogeneis (39,5,1; 40,2,2).

Graf, Fritz (Columbus, OH)
Citation
Graf, Fritz (Columbus, OH). " Allogenes." Brill’s New Pauly. Antiquity volumes edited by: Hubert Cancik and , Helmuth Schneider. Brill Online , 2012. Reference. 13 September 2012 <http://referenceworks.brillonline.com/entries/brill-s-new-pauly/allogenes-e116140> \footnote{\url{http://referenceworks.brillonline.com/entries/brill-s-new-pauly/allogenes-e116140?s.num=1&s.f.s2_parent=s.f.book.brill-s-new-pauly&s.q=epiphanius}.}}

What is \textquote{Ἰνδικτιῶνος}?

\begin{greek}

Epiphanius Scr. Eccl., Ancoratus (2021: 001)
“Epiphanius, Band 1: Ancoratus und Panarion”, Ed. Holl, K.
Leipzig: Hinrichs, 1915; Die griechischen christlichen Schriftsteller 25.
Chapter 58, section 2, line 3

        καὶ Φεισὼν μέν ἐστιν ὁ Γάγγης παρὰ τοῖς Ἰνδοῖς καλούμενος 
καὶ Αἰθίοψιν, Ἕλληνες δὲ τοῦτον καλοῦσιν Ἰνδὸν ποταμόν. 



Epiphanius Scr. Eccl., Ancoratus 
Chapter 60, section 5, line 7

                                          τὸ ἔτος γὰρ τοῦτό ἐστιν   
ἐνενηκοστὸν Διοκλητιανοῦ, Οὐαλεντινιανοῦ καὶ Οὐάλεντος <ι>, Γρα-
τιανοῦ δὲ ἔτος <ϛ>, ὑπατείᾳ Γρατιανοῦ Αὐγούστου τὸ τρίτον καὶ 
Ἐκουιτίου λαμπροτάτου, Ἰνδικτιῶνος <β>. 



Epiphanius Scr. Eccl., Ancoratus 
Chapter 112, section 3, line 2

                                καὶ διαμερίζει μὲν ὡς κληρονόμος τοῦ 
κόσμου καταστὰς ὑπὸ τοῦ θεοῦ τοῖς τρισὶν υἱοῖς αὐτοῦ τὸν πάντα 
κόσμον, ὑπὸ κλήρους διελὼν καὶ ἑκάστην μερίδα κατὰ κλῆρον ἑκάστῳ 
ἀπονέμων· καὶ τῷ μὲν Σὴμ τῷ πρωτοτόκῳ ὑπέπεσεν ὁ κλῆρος ἀπὸ   
Περσίδος καὶ Βάκτρων ἕως Ἰνδικῆς <τὸ μῆκος, πλάτος δὲ ἀπὸ Ἰνδικῆς> 
ἕως τῆς χώρας Ῥινοκουρούρων· κεῖται δὲ αὕτη ἡ Ῥινοκουρούρων 
ἀνὰ μέσον Αἰγύπτου καὶ Παλαιστίνης, ἀντικρὺ τῆς ἐρυθρᾶς θαλάσσης. 



Epiphanius Scr. Eccl., Ancoratus 
Chapter 113, section 2, line 2

                          τὰ δὲ ὀνόματα αὐτῶν ἐστι τάδε· Ἐλυμαῖοι 
Παίονες Λαζόνες Κοσσαῖοι Γασφηνοὶ [Παλαιστινοὶ] Ἰνδοὶ Σύροι 
Ἄραβες οἱ καὶ <Ταϊ>ανοὶ Ἀριανοὶ Μάρδοι Ὑρκανοὶ Μαγουσαῖοι Τρω-
γλοδύται Ἀσσύριοι Γερμανοὶ Λυδοὶ Μεσοποταμῖται Ἑβραῖοι Κοιληνοὶ 
Βακτριανοὶ Ἀδιαβηνοὶ Καμήλιοι Σαρακηνοὶ Σκύθαι † Χίονες Γυμνο-  
σοφισταὶ Χαλδαῖοι Πάρθοι Ἐῆται Κορδυληνοὶ Μασσυνοὶ Φοίνικες 
Μαδιηναῖοι Κομμαγηνοὶ Δαρδάνιοι Ἐλαμασηνοὶ Κεδρούσιοι Ἐλαμῖται 
Ἀρμένιοι Κίλικες [Αἰγύπτιοι] Καππάδοκες [Φοίνικες] Ποντικοὶ [Μαρμα-
ρίδαι] † Βίονες [Κᾶρες] Χάλυβες [Ψυλλῖται] Λαζοὶ [Μοσσύνοικοι] 
Ἴβηρες [Φρύγες]. 



Epiphanius Scr. Eccl., Panarion (= Adversus haereses) (2021: 002)
“Epiphanius, Bände 1–3: Ancoratus und Panarion”, Ed. Holl, K.
Leipzig: Hinrichs, 1:1915; 2:1922; 3:1933; Die griechischen christlichen Schriftsteller 25, 31, 37.
Volume 1, page 291, line 13

       διὰ τοῦτο γὰρ ὁ ἱερεὺς προσετάγη ὑπ' αὐτοῦ τοῦ νομοθετή-
σαντος, φησίν, ἔχειν κώδωνας, ἵν' ὅταν εἰσέρχηται ἱερατεῦσαι, τὸν 
κτύπον ἀκούων κρύπτηται ὁ προσκυνούμενος, ἵνα μὴ φωραθῇ τὸ 
ἰνδαλτικὸν αὐτοῦ τῆς μορφῆς πρόσωπον. 



Epiphanius Scr. Eccl., Panarion (= Adversus haereses) 
Volume 2, page 216, line 9

                                 τὸν δὲ γάμον σαφῶς τοῦ διαβόλου 
ὁρίζονται· ἔμψυχα δὲ βδελύσσονται, ἀπαγορεύοντες οὐχ ἕνεκεν ἐγκρα-
τείας οὔτε πολιτείας, ἀλλὰ κατὰ φόβον καὶ ἰνδαλμὸν τοῦ μὴ κατα-
δικασθῆναι ἀπὸ τῆς τῶν ἐμψύχων μεταλήψεως. 



Epiphanius Scr. Eccl., Panarion (= Adversus haereses) 
Volume 3, page 16, line 8

                                     ἀεὶ δὲ στελλόμενος τὴν πορείαν ἐπὶ 
τὴν τῶν Ἰνδῶν χώραν πραγματείας χάριν πολλὴν ἐμπορίαν ἐποιεῖτο. 



Epiphanius Scr. Eccl., Panarion (= Adversus haereses) 
Volume 3, page 17, line 4

                                                                             ὅθεν 
πολλὰ κτησάμενος ἐν τῷ κόσμῳ καὶ διὰ τῆς Θηβαΐδος διιών, ὅρμοι 
γὰρ τῆς ἐρυθρᾶς θαλάσσης διάφοροι, ἐπὶ τὰ στόμια τῆς Ῥωμανίας 
διακεκριμένοι, ὁ μὲν εἷς ἐπὶ τὴν Αἰλᾶν, ἥτις ἐστὶν ἐν τῇ θείᾳ γραφῇ 
Αἰλῶν· ἔνθα που ἡ ναῦς Σολομῶντος διὰ τριῶν ἐτῶν ἐρχομένη ἔφερε   
χρυσὸν καὶ ὀδόντας ἐλεφαντίνους, ἀρώματά τε καὶ ταῶνας καὶ τὰ ἄλλα, 
ὁ δὲ ἕτερος ὅρμος ἐπὶ τὸ Κάστρον τοῦ Κλύσματος, ἄλλος δὲ ἀνωτάτω 
ἐπὶ τὴν Βερνίκην καλουμένην, δι' ἧς Βερνίκης καλουμένης ἐπὶ τὴν Θηβαΐδα 
φέρονται, καὶ τὰ ἀπὸ τῆς Ἰνδικῆς ἐρχόμενα εἴδη ἐκεῖσε τῇ Θηβαΐδι 
διαχύνεται ἢ ἐπὶ τὴν Ἀλεξανδρέων διὰ τοῦ Χρυσορρόᾳ ποταμοῦ, Νείλου 
δέ φημι, τοῦ καὶ Γεὼν ἐν ταῖς γραφαῖς λεγομένου, καὶ ἐπὶ πᾶσαν τῶν 
Αἰγυπτίων γῆν καὶ ἐπὶ τὸ Πηλούσιον φέρεται· καὶ οὕτως εἰς τὰς ἄλλας 
πατρίδας διὰ θαλάσσης διερχόμενοι οἱ ἀπὸ τῆς Ἰνδικῆς ἐπὶ τὴν Ῥω-
μανίαν ἐμπορεύονται. 



Epiphanius Scr. Eccl., Panarion (= Adversus haereses) 
Volume 3, page 17, line 8

διακεκριμένοι, ὁ μὲν εἷς ἐπὶ τὴν Αἰλᾶν, ἥτις ἐστὶν ἐν τῇ θείᾳ γραφῇ 
Αἰλῶν· ἔνθα που ἡ ναῦς Σολομῶντος διὰ τριῶν ἐτῶν ἐρχομένη ἔφερε   
χρυσὸν καὶ ὀδόντας ἐλεφαντίνους, ἀρώματά τε καὶ ταῶνας καὶ τὰ ἄλλα, 
ὁ δὲ ἕτερος ὅρμος ἐπὶ τὸ Κάστρον τοῦ Κλύσματος, ἄλλος δὲ ἀνωτάτω 
ἐπὶ τὴν Βερνίκην καλουμένην, δι' ἧς Βερνίκης καλουμένης ἐπὶ τὴν Θηβαΐδα 
φέρονται, καὶ τὰ ἀπὸ τῆς Ἰνδικῆς ἐρχόμενα εἴδη ἐκεῖσε τῇ Θηβαΐδι 
διαχύνεται ἢ ἐπὶ τὴν Ἀλεξανδρέων διὰ τοῦ Χρυσορρόᾳ ποταμοῦ, Νείλου 
δέ φημι, τοῦ καὶ Γεὼν ἐν ταῖς γραφαῖς λεγομένου, καὶ ἐπὶ πᾶσαν τῶν 
Αἰγυπτίων γῆν καὶ ἐπὶ τὸ Πηλούσιον φέρεται· καὶ οὕτως εἰς τὰς ἄλλας 
πατρίδας διὰ θαλάσσης διερχόμενοι οἱ ἀπὸ τῆς Ἰνδικῆς ἐπὶ τὴν Ῥω-
μανίαν ἐμπορεύονται. 



Epiphanius Scr. Eccl., Panarion (= Adversus haereses) 
Volume 3, page 17, line 17

              ἐν ἀρχῇ τοίνυν οὗτος ὁ Σκυθιανὸς πλούτῳ πολλῷ ἐπαρ-
θεὶς καὶ κτήμασιν ἡδυσμάτων καὶ τοῖς ἄλλοις τοῖς ἀπὸ τῆς Ἰνδίας καὶ 
ἐλθὼν περὶ τὴν Θηβαΐδα εἰς Ὑψηλὴν πόλιν οὕτω καλουμένην, εὑρὼν 
ἐκεῖ γύναιον ἐξωλέστατον καὶ κάλλει σώματος πρόοπτον ἐκπλῆξάν τε 
αὐτοῦ τὴν ἀσυνεσίαν, ἀνελόμενός τε τοῦτο ἀπὸ τοῦ στέγους (ἕστηκε 
γὰρ ἡ τοιαύτη ἐν τῇ πολυκοίνῳ ἀσεμνότητι) ἐπεκαθέσθη τῷ γυναίῳ 
καὶ ἐλευθερώσας αὐτὸ συνήφθη αὐτῷ πρὸς γάμον. 



Epiphanius Scr. Eccl., Panarion (= Adversus haereses) 
Volume 3, page 19, line 18

                                            ὡς δὲ οὐκ ἴσχυσέ τι ἀνύσαι, 
ἀλλὰ τὸ ἧττον μᾶλλον ἀπηνέγκατο, ἐπετήδευσε δι' ὧν εἶχε μαγικῶν 
βιβλίων – καὶ γὰρ καὶ γόης ἦν, ἀπὸ τῆς τῶν Ἰνδῶν καὶ Αἰγυπτίων 
[καὶ] ἐθνομύθου σοφίας ἐμπορευσάμενος τὰ δεινὰ καὶ ὀλετήρια τῆς γοη-
τείας – φαντασίαν τινά· ἐπὶ δώματος <γὰρ> ἀνελθὼν καὶ ἐπιτηδεύσας, 
ὅμως οὐδὲν ἰσχύσας, ἀλλὰ καταπεσὼν ἐκ τοῦ δώματος, τέλει τοῦ βίου 
ἐχρήσατο. 



Epiphanius Scr. Eccl., Panarion (= Adversus haereses) 
Volume 3, page 32, line 22

         καὶ τοῦ Τρύφωνος ἰνδαλλομένου, ἀποκριθέντος τε αὐτῷ πρὸς 
ἔπος ὧν ᾔτησε κατὰ τὴν ἐκ θεοῦ δοθεῖσαν αὐτῷ σύνεσιν καὶ στροβήσαντος 
τὸν ἀπατεῶνα, ἠρέμα πως ἐν ᾧ ἑαυτῷ ἐνεδοίαζεν, ἀνακύπτει ὁ Ἀρχέλαος 
ὥσπερ ἰσχυρὸς οἰκοδεσπότης τῶν ἰδίων ἐπιμελούμενος καὶ μετὰ παρρησίας 
ἐπελθὼν τῷ συλᾶν ἐπιχειροῦντι ἐνεβριμεῖτο. 



Epiphanius Scr. Eccl., Panarion (= Adversus haereses) 
Volume 3, page 89, line 9

Τὰ δὲ ἄλλα λοιπὸν τῆς φλυαρίας, ὡς ἡ παρθένος φαίνεται τοῖς ἄρ-
χουσι, ποτὲ μὲν εἰς ἀνδρὸς σχῆμα, ποτὲ δὲ εἰς θηλείας, τάχα τὸν ἑρμα-
φρόδιτον τοῦ αὐτοῦ δαίμονος ἰνδαλλόμενος τὰ ἑαυτοῦ πάθη εἰσηγεῖται 
τῆς ἐπιθυμίας. 



Epiphanius Scr. Eccl., Panarion (= Adversus haereses) 
Volume 3, page 103, line 22

                            εἰ δὲ πολεμούμενον τὸ φῶς διώκεται 
ὑπὸ τοῦ σκότους, ἄρα δυνατώτερόν ἐστι τοῦ φωτὸς τὸ σκότος, ἐπειδὴ 
ἀποδιδράσκει ἀπὸ προσώπου τοῦ σκότους καὶ οὐχ ὑφίσταται 
στῆναι ἰνδαλλόμενον διὰ τὸ δυνατώτερον σκότος. 



Epiphanius Scr. Eccl., Panarion (= Adversus haereses) 
Volume 3, page 297, line 29

                                                                            νῦν 
δὲ μετὰ τὴν ἐκείνων τελευτήν, ὅτε εἰς πλάτος ἐλήλακεν ἡ αὐτῶν κακοδοξία 
καὶ μετὰ παρρησίας τυγχάνουσι διὰ τὴν τῆς σαρκὸς δεξιάν, [καὶ] μηκέτι 
ἐμποδιζόμενοι κηρύττουσι σαφῶς τὸ αὐτῶν ἐπιχείρημα, οὐκέτι οὔτε 
αἰδοῖ τινι κατεχόμενοι οὔτε ἰνδαλλόμενοι ἀπό τινος προστάγματος. 



Epiphanius Scr. Eccl., Panarion (= Adversus haereses) 
Volume 3, page 509, line 26

10. Καὶ οὗτοι μὲν οἳ ἐξ Ἑλλήνων εἰς γνῶσιν ἡμῶν ἐληλύθασιν· 
ἄλλοι δὲ ὅσοι κατὰ τὴν βάρβαρον καὶ Ἑλλάδα Ῥωμα<ν>ίαν τε καὶ τὰ 
ἄλλα κλίματα τῆς οἰκουμένης· ἑβδομήκοντα δύο μὲν ἀηδεῖς φιλοσοφίαι ἐν 
τῇ τῶν Ἰνδῶν ἐμφέρονται φατρίᾳ, τῶν τε γυμνοσοφιστῶν, τῶν τε Βραχ-
μάνων, ἐπαινετῶν τούτων μόνων, τῶν τε Ψευδοβραχμάνων, τῶν τε νεκυο-
φάγων, τῶν τε αἰσχροποιῶν, τῶν τε ἀπηλγημένων· ὧν τὸ κατ' εἶδος 
λέγειν καὶ τὰ παρ' αὐτῶν μυσαρὰ γινόμενα περιττὸν ἡγησάμεθα καὶ 
οὐδὲ<ν> ἄξιον, διὰ τὴν πολλὴν ἐν τοῖς ἀνθρώποις φθοράν, κακίας τε 
καὶ * ἐργασίαν. 



Epiphanius Scr. Eccl., Panarion (= Adversus haereses) 
Volume 3, page 512, line 20

                                           ἑτέρων δὲ πάλιν μυστηρίων πολλῶν 
καὶ αἱρεσιαρχῶν καὶ σχισματοποιῶν † ὧν μὲν ἀρχηγοὶ παρὰ Πέρσαις 
Μαγουσαῖοι, παρὰ δὲ Αἰγυπτίοις προφῆται καλούμενοι, τῶν ἀδύτων 
τε καὶ ἱερῶν ἀρχηγοί, καὶ μάγων Βαβυλωνίων δὲ οἵ τε καλούμενοι Γαζαρη-
νοί, σοφοί τε καὶ ἐπαοιδοί, Ἰνδῶν δὲ οἱ Εὐίλεοι καλούμενοι καὶ Βραχ-
μᾶνες, Ἑλλήνων <δὲ> ἱεροφάνται τε καὶ νεωκόροι, Κυνικῶν πλῆθος 
καὶ ἄλλων ἀμυθήτων φιλοσόφων ἀρχηγοί. 



Epiphanius Scr. Eccl., De xii gemmis (2021: 004)
“Les lapidaires de l'antiquité et du Moyen Age, vol. 2.1”, Ed. Ruelle, C.É.
Paris: Leroux, 1898.
Chapter 1, section 2, line 2

Λίθος <τοπάζιον,> ἐρυθρὸς τῷ εἴδει ὑπὲρ τὸν ἄνθρακα· γίνεται δὲ ἐν Τοπάζῃ, 
πόλει τῆς Ἰνδίας, ὑπὸ τῶν ἐκεῖσέ ποτε λίθους λατομούντων, ἐν καρδίᾳ ἑτέρου 
λίθου ὃν οἱ λατομοῦντες θεασάμενοι φαιδρόν, καὶ ὑποδείξαντες ἀλάβαστρόν τισιν 
ἀπέδοντο ὀλίγου τιμήματος. 



Epiphanius Scr. Eccl., De xii gemmis 
Chapter 1, section 3, line 12

                                                                                              Φεισσὼν 
δέ ἐστιν ὁ παρὰ τοῖς Ἕλλησιν Ἰνδὸς καλούμενος, παρὰ τοῖς βαρβάροις δὲ Γάγγης. 



Epiphanius Scr. Eccl., De xii gemmis 
Chapter 1, section 5, line 4

                                                      Καὶ οὗτος δὲ λέγεται εἶναι ἐν τῇ 
Ἰνδίᾳ καὶ Αἰθιοπείᾳ. 



Epiphanius Scr. Eccl., De xii gemmis 
Chapter 1, section 5, line 4

                              Διὸ τὸ τέμενος παρὰ Ἰνδοῖς φασιν εἶναι τοῦ Διονύσου τξεʹ 
ἀναβαθμοὺς ἔχον ἐκ σαπφείρου λίθου, εἰ καὶ τοῖς πολλοῖς ὑπάρχει ἄπιστον. 



Epiphanius Scr. Eccl., De xii gemmis (fragmenta ap. Anastasium Sinaïtam, Quaestiones et responsiones) (2021: 005); MPG 89.
Volume 89, page 588, line 14

Τοπάζιον ἐρυθρὸς μέν ἐστιν ὑπὲρ τὸν ἄνθρακα 
λίθον, γίνεται δὲ ἐν Τοπάζῃ πόλει τῆς Ἰνδικῆς. 



Epiphanius Scr. Eccl., De xii gemmis (fragmenta ap. Anastasium Sinaïtam, Quaestiones et responsiones) 
Volume 89, page 588, line 24

Σμάραγδος χλωρὸς μέν ἐστιν, ἐν δὲ τοῖς ὄρεσι 
τοῖς Ἰνδικοῖς ὀρύγοντες οἱ βάρβαροι, κόπτουσιν 
αὐτόν. 



Epiphanius Scr. Eccl., De xii gemmis (fragmenta ap. Anastasium Sinaïtam, Quaestiones et responsiones) 
Volume 89, page 588, line 37

Σάπφειρος πορφυρίζων μέν ἐστι, γίνεται δὲ 
ἐν Αἰθιοπίᾳ καὶ Ἰνδίᾳ. 



Epiphanius Scr. Eccl., Index apostolorum [Sp.] (2021: 023)
“Prophetarum vitae fabulosae”, Ed. Schermann, T.
Leipzig: Teubner, 1907.
Page 110, line 8

      Βαρθολομαῖος δὲ ὁ ἀπόστολος Ἰνδοῖς τοῖς καλουμέ-
νοις Εὐδαίμοσιν ἐκήρυξε τὸ εὐαγγέλιον τοῦ Χριστοῦ καὶ τὸ 
κατὰ Ματθαῖον ἅγιον εὐαγγέλιον αὐτοῖς τῇ ἰδίᾳ διαλέκτῳ 
αὐτῶν συγγράψας ἐκοιμήθη δὲ ἐν Ἀλβανίᾳ πόλει τῆς μεγάλης 
Ἀρμενίας καὶ ἐκεῖ ἐτάφη. 



Epiphanius Scr. Eccl., Index apostolorum [Sp.] 
Page 111, line 4

     Θωμᾶς δὲ ὁ ἀπόστολος, καθὼς ἡ παράδοσις περιέχει, 
ἦν μὲν ἀπὸ τῆς Πανιάδος πόλεως τῆς Γαλιλαίας, Πάρθοις 
δὲ καὶ Μήδοις ἐκήρυξε τὸ εὐαγγέλιον τοῦ κυρίου, καὶ 
Πέρσαις δὲ καὶ Γερμανοῖς καὶ Ὑρκανοῖς [καὶ Ἰνδοῖς] καὶ 
Βάκτροις καὶ Μάγοις, ἐκοιμήθη ἐν πόλει Καλαμηνῇ τῆς 
Ἰνδικῆς. 



Epiphanius Scr. Eccl., Index apostolorum [Sp.] 
Page 115, line 17

            Ἐστὶν οὖν ὁ πᾶς χρόνος, ἐξ οὗ ἐμαρτύρησε 
τριακόσια τριάκοντα ἔτη μέχρι τῆς παρούσης ταύτης ὑπατίας, 
τετάρτης μὲν Ἀρκαδίου, τρίτης δὲ Ὁνωρίου τῶν δύο ἀδελ-
φῶν αὐτοκρατόρων Αὐγούστων, ἐννάτης ἰνδικτίωνος τῆς 
πεντεκαιδεκαετηρικῆς περιόδου μηνὸς Ἰουνίου εἰκοστῆς ἐν-
νάτης ἡμέρας. 



Epiphanius Scr. Eccl., De mensuris et ponderibus (2021: 033)
“”Τὸ ‘Περὶ μέτρων καὶ σταθμῶν’ ἔργον Ἐπιφανίου τοῦ Σαλαμῖνος””, Ed. Moutsoulas, E., 1973; Θεολογία 44.
Line 271

                                                                  Ἀκούομεν δὲ   
ἔτι πολὺ πλῆθος ἐν τῷ κόσμῳ ὑπάρχειν, παρά τε Αἰθίοψι καὶ Ἰνδοῖς, Πέρ-
σαις τε καὶ Ἐλαμίταις καὶ Βαβυλωνίοις, Ἀσσυρίοις τε καὶ Χαλδαίοις, παρὰ 
Ῥωμαίοις τε καὶ Φοίνιξι, Σύροις τε καὶ τοῖς ἐν τῇ Ἑλλάδι Ῥωμαίοις οὔπω 
Ῥωμαίοις καλουμένοις ἀκμὴν ἀλλὰ Λατίνοις. 

\end{greek}


\section{Stephanus Gramm.}
Double--check that Stephanus Gramm. = Stephen of Byzantium (author of \emph{Ethnica}).

\blockquote[From Wikipedia\footnote{\url{}}]{Stephen of Byzantium, also known as Stephanus Byzantinus (Greek: Στέφανος Βυζάντιος; fl. 6th century AD), was the author of an important geographical dictionary entitled Ethnica (Ἐθνικά). Of the dictionary itself only meagre fragments survive, but we possess an epitome compiled by one Hermolaus}

\begin{greek}

Stephanus Gramm., Ethnica (epitome) (4028: 001)
“Stephan von Byzanz. Ethnika”, Ed. Meineke, A.
Berlin: Reimer, 1849, Repr. 1958.
Page 11, line 20

<Ἀγαθοῦ δαίμονος,> νῆσος ἐν τῇ Ἰνδικῇ θαλάσσῃ. 



Stephanus Gramm., Ethnica (epitome) 
Page 44, line 19

       ἐκλήθη καὶ Μύσρα ἡ χώρα ὑπὸ Φοινίκων, καὶ Ἀερία 
καὶ Ποταμῖτις, καὶ Ἀετία ἀπό τινος Ἰνδοῦ Ἀετοῦ. 



Stephanus Gramm., Ethnica (epitome) 
Page 71, line 11

                           τετάρτη πόλις Ὠριτῶν, ἔθνους Ἰχθυο-
φάγων, κατὰ τὸν περίπλουν τῆς Ἰνδικῆς. 



Stephanus Gramm., Ethnica (epitome) 
Page 71, line 12

                                               πέμπτη ἐν τῇ 
Ὠπιανῇ, κατὰ τὴν Ἰνδικήν. 



Stephanus Gramm., Ethnica (epitome) 
Page 71, line 12

                                   ἕκτη πάλιν Ἰνδικῆς. 



Stephanus Gramm., Ethnica (epitome) 
Page 71, line 13

                                                             ἑβδόμη ἐν 
Ἀρίοις, ἔθνει Παρθυαίων κατὰ τὴν Ἰνδικήν. 



Stephanus Gramm., Ethnica (epitome) 
Page 71, line 18

                                                τεσσαρεσκαιδε-
κάτη παρὰ Σωριανοῖς, Ἰνδικῷ ἔθνει. 



Stephanus Gramm., Ethnica (epitome) 
Page 71, line 19

                                             πεντεκαιδεκάτη παρὰ 
τοῖς Ἀραχώτοις, ὁμοροῦσα τῇ Ἰνδικῇ. 



Stephanus Gramm., Ethnica (epitome) 
Page 101, line 4

                             καὶ τρίτη Ἰνδικῆς, ἣν ἀναγράφει 
Φίλων καὶ Δημοδάμας ὁ Μιλήσιος. 



Stephanus Gramm., Ethnica (epitome) 
Page 108, line 3

<Ἄραβις,> ποταμὸς Ἰνδικῆς, ἐν αὐτονόμῳ χώρᾳ, περὶ ὃν 
οἰκοῦσιν Ἀραβῖται, ὡς ὠκεανῖται. 



Stephanus Gramm., Ethnica (epitome) 
Page 110, line 13

<Ἀραχωτοί,> πόλις Ἰνδικῆς, ἀπὸ Ἀραχωτοῦ ποταμοῦ, 
ῥέοντος ἀπὸ τοῦ Καυκάσου, ὡς Φαβωρῖνος καὶ Στράβων ἑνδε-
κάτῃ. 



Stephanus Gramm., Ethnica (epitome) 
Page 111, line 8

<Ἄρβις,> ποταμὸς τῆς Ἰνδικῆς. 



Stephanus Gramm., Ethnica (epitome) 
Page 111, line 21

<Ἀργάντη,> πόλις Ἰνδίας, ὡς Ἑκαταῖος. 



Stephanus Gramm., Ethnica (epitome) 
Page 112, line 1

                                                      τὸ ἐθνικὸν ἔδει   
Ἀργανταῖος, ἀλλὰ ὁ τύπος τῶν Ἰνδῶν ἢ Ἀργαντηνός ἢ Ἀρ-
γαντίτης. 



Stephanus Gramm., Ethnica (epitome) 
Page 115, line 1

<Ἀργυρᾶ,> μητρόπολις [τῆς] ἐν Ἰνδικῇ Ταπροβάνης νήσου, 
ὅ ἐστι κριθῆς νήσου· καὶ γὰρ εὐφορωτάτη ἐστὶ καὶ πλεῖστον 
ποιεῖ χρυσόν. 



Stephanus Gramm., Ethnica (epitome) 
Page 122, line 14

                                                      ἔστι καὶ <Ἅρ-
ματα> πόλις πληθυντικῶς Ἰνδικῆς. 



Stephanus Gramm., Ethnica (epitome) 
Page 133, line 4

<Ἀσκῖται,> ἔθνος παροικοῦν τὸν Ἰνδικὸν κόλπον καὶ ἐπὶ 
ἀσκῶν πλέον, ὡς Μαρκιανὸς ἐν τῷ περίπλῳ αὐτοῦ “παροικεῖ 
αὐτὸν ἔθνος καὶ αὐτὸ καλούμενον Σαχαλιτῶν. 



Stephanus Gramm., Ethnica (epitome) 
Page 135, line 20

<Ἀσσακηνοί,> ἔθνος Ἰνδικόν. 



Stephanus Gramm., Ethnica (epitome) 
Page 157, line 5

                                                            τὰ εἰς <δος> 
δισύλλαβα ἔχοντα πρὸ τοῦ <δ> ἄφωνον βαρύνεται, εἰ μὴ ἐπιθετικὰ 
εἴη· Ἰνδός ὅμοιον τῷ ποταμῷ, τὸ λορδός μυνδός ὁ ἄφωνος 
ὀξύνεται, ἀφ' οὗ τὸ “μυνδότεροι νεπόδων” παρὰ Καλλιμάχῳ. 



Stephanus Gramm., Ethnica (epitome) 
Page 164, line 21

<Βέρεξ,> ἔθνος μεταξὺ Ἰνδίας καὶ Αἰθιοπίας, ὡς Τιμο-
κράτης ὁ Ἀδραμυττηνός. 



Stephanus Gramm., Ethnica (epitome) 
Page 168, line 5

<Βήσσυγα,> οὐδετέρως, ἐμπόριον τῆς Ἰνδικῆς, καὶ Βησσύ-
γας ποταμός, καὶ Βησσυγῖται οἱ ἄνθρωποι, οὕς φασιν ἀν-
θρωποφάγους. 



Stephanus Gramm., Ethnica (epitome) 
Page 175, line 12

                                                                ἔστι 
καὶ Ἰνδικῆς <Βουκεφάλα,> ἣν ἔκτισεν Ἀλέξανδρος “ἐπ' ἀμ-
φοτέραις ταῖς ὄχθαις τοῦ Ὑδάσπου ποταμοῦ πόλεις ᾤκισε, 
Νίκαιαν Βουκεφάλαν δὲ ἔνθα διαβάντος καὶ μαχομένου ἀπέ-
θανεν αὐτοῦ ὁ ἵππος Βουκεφάλας προσαγορευόμενος”. 



Stephanus Gramm., Ethnica (epitome) 
Page 179, line 2

                            ἔστι καὶ ἄλλη τῆς Ἰνδικῆς. 



Stephanus Gramm., Ethnica (epitome) 
Page 181, line 14

<Βουκεφάλεια,> πόλις ἐπὶ τῷ Βουκεφάλῳ ἵππῳ, ἣν 
ἔκτισεν Ἀλέξανδρος ἐν Ἰνδίᾳ παρὰ τὸν Ὑδάσπην ποταμόν. 



Stephanus Gramm., Ethnica (epitome) 
Page 184, line 18

<Βραχμᾶνες,> Ἰνδικὸν ἔθνος σοφώτατον, οὓς καὶ <Βράχ-
μας> καλοῦσιν. 



Stephanus Gramm., Ethnica (epitome) 
Page 190, line 10

                ἔστι καὶ Βυζάντιον ἕτερον ἐν τῇ Ἰνδικῇ. 



Stephanus Gramm., Ethnica (epitome) 
Page 191, line 1

<Βωλίγγαι,> ἔθνος Ἰνδικόν. 



Stephanus Gramm., Ethnica (epitome) 
Page 194, line 19

<Γάζος,> πόλις Ἰνδική, κατὰ Διονύσου πολεμήσασα μετὰ 
Δηριάδου, λινοῦν ἔχουσα τεῖχος, καθὰ [Διονύσιος] ἐν τρίτῃ 
Βασσαρικῶν 
  Γήρειαν Ῥοδόην τε καὶ οἳ λινοτειχέα Γάζον 
  τοῖόν μιν κλωστοῖο λινοῦ πέρι τεῖχος ἐέργει,   
  ἀστύφελον δηίοισι, καὶ εἰ παγχάλκεοι εἶεν, 
  εὖρος μὲν μάλα δή τι διαμπερὲς ὀργυιῇσιν 
  μετρητὸν πισύρεσσιν, ἀτὰρ μῆκός τε καὶ ἰθύν 
  ὅσσον ἀνὴρ δοιοῖσιν ἐν ἠελίοισιν ἀνύσσει, 
  ἠῶθεν κνέφας ἄκρον ἐπειγόμενος ποσὶν οἷσιν. 



Stephanus Gramm., Ethnica (epitome) 
Page 198, line 13

<Γανδάραι,> Ἰνδῶν ἔθνος. 



Stephanus Gramm., Ethnica (epitome) 
Page 200, line 15

<Γεδρωσία> χώρα καὶ <Γεδρώσιοι> ἔθνος Ἰνδικόν. 



Stephanus Gramm., Ethnica (epitome) 
Page 203, line 1

<Γέντα,> πόλις Ἰνδικὴ τῆς ἐκτὸς Γάγγου. 



Stephanus Gramm., Ethnica (epitome) 
Page 207, line 13

<Γήρεια,> πόλις Ἰνδική, τελοῦσα ὑπὸ Δηριάδῃ τῷ βα-
σιλεῖ τῶν Ἰνδῶν πρὸς Διόνυσον πολεμοῦντι. 



Stephanus Gramm., Ethnica (epitome) 
Page 216, line 13

                               ἔστι καὶ Ἰνδικῆς. 



Stephanus Gramm., Ethnica (epitome) 
Page 218, line 5

<Δάονες,> ἔθνος τῆς Ἰνδικῆς, ἀπὸ Δάονος. 



Stephanus Gramm., Ethnica (epitome) 
Page 218, line 8

<Δάρδαι,> Ἰνδικὸν ἔθνος ὑπὸ Δηριάδῃ πολεμῆσαν Διο-
νύσῳ, ὡς Διονύσιος ἐν γʹ Βασσαρικῶν. 



Stephanus Gramm., Ethnica (epitome) 
Page 219, line 14

<Δαρσανία,> πόλις Ἰνδική, ἐν ᾗ αὐθημερὸν ἱμάτιον 
ἱστουργοῦσι γυναῖκες, ὡς Διονύσιος Βασσαρικῶν τρίτῃ   
  ἢ οἳ Δαρσανίην ναῖον πόλιν εὐρυάγυιαν, 
  ἔνθα τε πέπλα γυναῖκες Ἀθηναίης ἰότητι 
  αὐτῆμαρ κροκόωσιν ἐφ' ἱστοπόδων τανύουσαι, 
  αὐτῆμαρ δ' ἐτάμοντο [καὶ ἐξ ἱστῶν] ἐρύσαντο. 



Stephanus Gramm., Ethnica (epitome) 
Page 233, line 8

γʹ τῆς Ἰνδικῆς. 



Stephanus Gramm., Ethnica (epitome) 
Page 242, line 8

<Δυρβαῖοι,> ἔθνος καθῆκον εἰς Βάκτρους καὶ τὴν Ἰνδι-
κήν. 



Stephanus Gramm., Ethnica (epitome) 
Page 242, line 10

      Κτησίας ἐν Περσικῶν ιʹ “χώρη δὲ πρὸς αὐτὸν πρόσκει-
ται Δυρβαῖοι, πρὸς τὴν Βακτρίην καὶ Ἰνδικὴν κατατείνοντες. 



Stephanus Gramm., Ethnica (epitome) 
Page 259, line 1

<Ἔαρες,> ἔθνος Ἰνδικὸν τῶν μετὰ Δηριάδου Διονύσῳ 
πολεμησάντων. 



Stephanus Gramm., Ethnica (epitome) 
Page 293, line 5

<Ζάβιοι,> ἔθνος Ἰνδικόν, πολεμῆσαν μετὰ Δηριάδου 
Διονύσῳ. 



Stephanus Gramm., Ethnica (epitome) 
Page 303, line 19

ιαʹ μεταξὺ Σκυθίας καὶ Ἰνδικῆς. 



Stephanus Gramm., Ethnica (epitome) 
Page 332, line 7

                                                                       καὶ 
διὰ τοῦ <ι> Ἰνάχιον, καὶ Ἰναχίτης καὶ Ἰναχιεύς. 
 <Ἰνδάρα,> Σικανῶν πόλις. 



Stephanus Gramm., Ethnica (epitome) 
Page 332, line 8

                                                 τὸ ἐθνικὸν Ἰν-
δαραῖος ὡς Ἱμεραῖος. 



Stephanus Gramm., Ethnica (epitome) 
Page 332, line 10

                            τὸ ἐθνικὸν Ἰνδικῆται. 



Stephanus Gramm., Ethnica (epitome) 
Page 332, line 11

<Ἰνδός,> ποταμός, ἀφ' οὗ Ἰνδοί, ἀφ' οὗ Ἰνδικός καὶ 
Ἰνδική “καὶ Ἰνδικὸν οἶδμα θαλάσσης”. 



Stephanus Gramm., Ethnica (epitome) 
Page 332, line 12

                                                 λέγεται καὶ Ἰνδῷος. 



Stephanus Gramm., Ethnica (epitome) 
Page 346, line 13

<Κάθαια,> πόλις Ἰνδική. 



Stephanus Gramm., Ethnica (epitome) 
Page 347, line 24

<Καλατίαι,> γένος Ἰνδικόν. 



Stephanus Gramm., Ethnica (epitome) 
Page 360, line 3

<Καρμανία,> χώρα τῆς Ἰνδικῆς. 



Stephanus Gramm., Ethnica (epitome) 
Page 360, line 10

<Κάρμινα,> νῆσος Ἰνδική. 



Stephanus Gramm., Ethnica (epitome) 
Page 364, line 10

<Κάσπειρος,> πόλις Πάρθων προσεχὴς τῇ Ἰνδικῇ. 



Stephanus Gramm., Ethnica (epitome) 
Page 364, line 15

καὶ πάλιν 
  Κοσσαῖος γενεὴν Κασπειρόθεν, οἵ ῥά τε πάντων 
  Ἰνδῶν ὅσσοι ἔασιν ἀφάρτερα γούνατ' ἔχουσιν· 
  ὅσσον γάρ τ' ἐν ὄρεσσιν ἀριστεύουσι λέοντες, 
  ἢ ὁπόσον δελφῖνες ἔσω ἁλὸς ἠχηέσσης, 
  αἰετὸς εἰν ὄρνισι μεταπρέπει ἀγρομένοισιν, 
  ἵπποι τε πλακόεντος ἔσω πεδίοιο θέοντες, 
  τόσσον ἐλαφρότατοισι περιπροφέρουσι πόδεσσιν 
  Κάσπειροι μετὰ φῦλα τά τ' ἄφθιτος ἔλλαχεν ἠώς. 



Stephanus Gramm., Ethnica (epitome) 
Page 365, line 19

<Κασσίτερα,> νῆσος ἐν τῷ ὠκεανῷ, τῇ Ἰνδικῇ προσεχής, 
ὡς Διονύσιος ἐν Βασσαρικοῖς. 



Stephanus Gramm., Ethnica (epitome) 
Page 430, line 9

<Μαλοί,> ἔθνος Ἰνδικόν, τῶν ἀνθεστηκότων τῷ Διονύσῳ 
μετὰ Δηριάδου, ὡς Διονύσιος Βασσαρικῶν γʹ. 



Stephanus Gramm., Ethnica (epitome) 
Page 432, line 9

<Μαράχη,> πόλις Ἰνδική. 



Stephanus Gramm., Ethnica (epitome) 
Page 432, line 13

<Μάργανα,> πόλις τῆς Ἰνδικῆς. 



Stephanus Gramm., Ethnica (epitome) 
Page 436, line 4

<Μάσσακα,> πόλις Ἰνδῶν. 



Stephanus Gramm., Ethnica (epitome) 
Page 436, line 4

                                   Ἀρριανὸς ἐν Ἰνδικοῖς. 



Stephanus Gramm., Ethnica (epitome) 
Page 466, line 16

<Μωριεῖς,> ἔθνος Ἰνδικόν, ἐν ξυλίνοις οἰκοῦντες οἴκοις, 
ὡς Εὐφορίων. 



Stephanus Gramm., Ethnica (epitome) 
Page 474, line 20

                                             τετάρτη ἐν Ἰνδοῖς. 



Stephanus Gramm., Ethnica (epitome) 
Page 479, line 9

                             ἑβδόμη ἐν Ἰνδοῖς. 



Stephanus Gramm., Ethnica (epitome) 
Page 494, line 1

<Ὀξυδράκαι,> ἔθνος Ἰνδικόν, ἀφ' ὧν σώσας Ἀλέξανδρον 
Πτολεμαῖος σωτὴρ ἐκλήθη. 



Stephanus Gramm., Ethnica (epitome) 
Page 494, line 14

<Ὀρβῖται,> ἔθνος Ἰνδικόν, ὡς Ἀπολλόδωρος δευτέρῳ, 
περὶ Ἀλεξάνδρειαν. 



Stephanus Gramm., Ethnica (epitome) 
Page 497, line 6

<Παλίμβοθρα,> πόλις Ἰνδική. 



Stephanus Gramm., Ethnica (epitome) 
Page 499, line 4

<Παναίουρα,> πόλις Ἰνδικὴ περὶ τὸν Ἰνδὸν ποταμόν. 



Stephanus Gramm., Ethnica (epitome) 
Page 499, line 15

<Πάνδαι,> ἔθνος [Ἰνδικὸν κατὰ] Διονύσου μετὰ Δηριάδου 
στρατευσάμενον, καθὰ Διονύσιος. 



Stephanus Gramm., Ethnica (epitome) 
Page 507, line 1

<Παροπάνισσος,> πόλις ὄρος Ἰνδικῆς, ἀφ' οὗ Παροπα-
νισσάδαι οἱ παροικοῦντες. 



Stephanus Gramm., Ethnica (epitome) 
Page 510, line 11

<Πάταλα,> πόλις Ἰνδική. 



Stephanus Gramm., Ethnica (epitome) 
Page 534, line 18

<Πράσιοι,> ἔθνος Ἰνδικὸν Διονύσῳ πολεμῆσαν. 



Stephanus Gramm., Ethnica (epitome) 
Page 546, line 8

<Ῥοδόη,> πόλις Ἰνδική. 



Stephanus Gramm., Ethnica (epitome) 
Page 548, line 8

<Ῥωγάνη,> πόλις ἐν τῇ Ἰνδικῇ. 



Stephanus Gramm., Ethnica (epitome) 
Page 550, line 11

                                                     ἔστι δὲ καὶ 
ἕτερον ἔθνος Ἰνδικόν. 



Stephanus Gramm., Ethnica (epitome) 
Page 554, line 15

<Σάνεια,> πόλις Ἰνδική, ὡς Ἀδριανὸς Ἀλεξανδριάδος ἑβ-
δόμῳ. 



Stephanus Gramm., Ethnica (epitome) 
Page 556, line 5

<Σάραπις,> νῆσος ἐν τῷ Ἰνδικῷ κόλπῳ. 



Stephanus Gramm., Ethnica (epitome) 
Page 558, line 12

<Σαυρομάται,> ἔθνος Ἰνδικόν. 



Stephanus Gramm., Ethnica (epitome) 
Page 562, line 3

<Σεσίνδιον,> πόλις Ἰνδική. 



Stephanus Gramm., Ethnica (epitome) 
Page 562, line 20

<Σῆρες,> ἔθνος Ἰνδικόν, ἀπροσμιγὲς ἀνθρώποις, ὡς Οὐ-
ράνιος ἐν τρίτῳ Ἀραβικῶν. 



Stephanus Gramm., Ethnica (epitome) 
Page 563, line 12

<Σίβαι,> Ἰνδικὸν ἔθνος, ἅμα Δηριάδῃ μαχεσάμενον Διο-
νύσῳ, καθά φησι Διονύσιος. 



Stephanus Gramm., Ethnica (epitome) 
Page 569, line 25

<Σίνδα,> πόλις πρὸς τῷ μεγάλῳ κόλπῳ τῆς Ἰνδικῆς, ἔν-
θεν οἱ καλούμενοι Σίνδαι. 



Stephanus Gramm., Ethnica (epitome) 
Page 596, line 19

<Σώλιμνα,> πόλις Ἰνδίας, ὡς Ἡρωδιανὸς ἐν ἑνδεκάτῳ. 



Stephanus Gramm., Ethnica (epitome) 
Page 602, line 8

<Τάξιλα,> πόλις Ἰνδική. 



Stephanus Gramm., Ethnica (epitome) 
Page 602, line 16

<Ταπροβάνη,> νῆσος μεγίστη ἐν τῇ Ἰνδικῇ θαλάσσῃ. 



Stephanus Gramm., Ethnica (epitome) 
Page 624, line 12

                                           καὶ ποταμὸς Ἰνδός. 



Stephanus Gramm., Ethnica (epitome) 
Page 628, line 16

<Τόπαζος,> νῆσος Ἰνδική. 



Stephanus Gramm., Ethnica (epitome) 
Page 643, line 7

                                                                   ἔστι καὶ 
πόλις Ἰνδίας καὶ Λυδίας καὶ Πισιδίας. 



Stephanus Gramm., Ethnica (epitome) 
Page 645, line 9

<Ὑδάρκαι,> ἔθνος Ἰνδικὸν ἀντιταξάμενον Διονύσῳ, ὡς 
Διονύσιος Βασσαρικῶν τρίτῳ. 



Stephanus Gramm., Ethnica (epitome) 
Page 677, line 11

<Χαδραμωτῖται,> ἔθνος περὶ τὸν Ἰνδικὸν κόλπον, τῷ 
Πρίονι παροικοῦντες ποταμῷ, ὥς φησι Μαρκιανὸς ἐν τῷ 
περίπλῳ αὐτοῦ. 



Stephanus Gramm., Ethnica (epitome) 
Page 697, line 10

                                               ἔστι καὶ ἄλλη χερ-
ρόνησος τῆς Ἰνδικῆς, Μαρκιανὸς ἐν περίπλῳ “ἐν δὲ τῇ ἐκτὸς 
Γάγγου Ἰνδικῇ χρυσῆ καλουμένη χερρόνησος”. 



Stephanus Gramm., Ethnica (epitome) 
Page 708, line 15

<Ὠπίαι,> ἔθνος Ἰνδικόν. 



Stephanus Gramm., Ethnica (epitome) 
Page 708, line 16

                                     Ἑκαταῖος Ἀσίᾳ “ἐν δὲ αὐτοῖσι 
οἰκέουσι ἄνθρωποι παρὰ τὸν Ἰνδὸν ποταμὸν Ὠπίαι, ἐν δὲ 
τεῖχος βασιλήιον. 



Stephanus Gramm., Ethnica (epitome) 
Page 708, line 18

                    μέχρι τούτου Ὠπίαι, ἀπὸ δὲ τούτων ἐρημίη 
μέχρις Ἰνδῶν”. 



Stephanus Gramm., Ethnica (epitome) 
Page 710, line 9

<Ὠρῖται,> ἔθνος Ἰνδῶν αὐτόνομον. 



Stephanus Gramm., Ethnica (epitome) 
Page 710, line 10

                                                Στράβων πεντεκαιδε-
κάτῃ “τῷ ὁρίζοντι αὐτοὺς ἀπὸ τῶν ἑξῆς Ὠριτῶν· Ἰνδῶν δέ 
ἐστι καὶ αὕτη μερίς, ἔθνος αὐτόνομον”. 



Stephanus Gramm., Ethnica (epitome) 
Page 710, line 13

                                                 καὶ Ἀπολλόδωρος 
δευτέρῳ “ἔπειτα δ' Ὠρίτας τε καὶ Γεδρωσίους, ὧν τοὺς μὲν 
Ἰνδοὺς ὡς ἐνοικοῦντας πέτραν . 

\end{greek}


\section{Scholia In Demosthenem}
\blockquote[From Wikipedia\footnote{\url{http://www.oxfordscholarship.com/view/10.1093/acprof:oso/9780199259205.001.0001/acprof-9780199259205-chapter-5}}]{The Demosthenes Scholia

    Malcolm Heath (Contributor Webpage)

DOI:10.1093/acprof:oso/9780199259205.003.0005

This chapter presents a source-critical analysis of the scholia (explanatory notes) found in the medieval manuscripts of Demosthenes. These scholia are based on remnants of commentaries composed in late antiquity. It is shown that the scholia have been transmitted in three main strands of tradition. A lightly redacted version of Menander’s commentary is the sole source of one strand. In the other two strands, material derived from Menander has been combined with material from other commentators: one of these commentators was probably Zosimus (fifth century AD), the other is unidentified. Some material, which is identified as Menander’s, is attributed to Ulpian in manuscript superscriptions. The identity of this Ulpian and the nature of his contribution to the formation of the scholia is unknown.}

\begin{greek}
Scholia In Demosthenem, Scholia in Demosthenem (scholia vetera) (fort. auctore Ulpiano) (5017: 001)
“Scholia Demosthenica, 2 vols.”, Ed. Dilts, M.R.
Leipzig: Teubner, 1:1983; 2:1986.
Oration 17, section 2, line 43

ὅτι δὲ οὐ διὰ γῆρας οὐδὲ δι' ἀτονίαν λόγου τοῦτο συμβέβηκεν, εὔδηλον ἐκ 
τοῦ περὶ στεφάνου λόγου, ὃς πολὺ μεταγενέστερός ἐστι ταύτης τῆς δημη-
γορίας· ὁ μὲν γὰρ εἴρηται ἐν ἀρχῇ τῆς Ἀλεξάνδρου καταστάσεως, ὁ δὲ περὶ 
τοῦ στεφάνου λόγος Ἀλεξάνδρου ὄντος ἐν Ἰνδοῖς ἢ Πέρσαις. 

\end{greek}

\section{Joannes Chrysostomus}
\blockquote[From Wikipedia\footnote{\url{http://en.wikipedia.org/wiki/Joannes_Chrysostomus}}]{John Chrysostom (c. 347–407, Greek: Ἰωάννης ὁ Χρυσόστομος), Archbishop of Constantinople, was an important Early Church Father. He is known for his eloquence in preaching and public speaking, his denunciation of abuse of authority by both ecclesiastical and political leaders, the Divine Liturgy of St. John Chrysostom, and his ascetic sensibilities. After his death in 407 (or, according to some sources, during his life) he was given the Greek epithet chrysostomos, meaning "golden mouthed" in English, and Anglicized to Chrysostom.[2][5]}
\begin{greek}

Joannes Chrysostomus Scr. Eccl., De incomprehensibili dei natura (= Contra Anomoeos, homiliae 1–5) (2062: 012)
“Jean Chrysostome. Sur l'incompréhensibilité de Dieu”, Ed. Malingrey, A.–M.
Paris: Cerf, 1970; Sources chrétiennes 28 bis.
Homily 2, line 263

                                               Μὴ ἁπλῶς 
παρέλθῃς τὸν λόγον, ἀλλ' ἀνάπτυξον τὸ εἰρημένον καλῶς   
καὶ ἐξέτασον· ἀναλόγισαι πάντα τὰ ἔθνη, Σύρους, Κίλικας, 
Καππαδόκας, Βιθυνούς, τοὺς τὸν Εὔξεινον πόντον οἰκοῦντας, 
Θρᾴκην, Μακεδονίαν, τὴν Ἑλλάδα πᾶσαν, τοὺς ἐν ταῖς 
νήσοις, τοὺς ἐν τῇ Ἰταλίᾳ, τοὺς ὑπὲρ τὴν καθ' ἡμᾶς 
οἰκουμένην, τοὺς ἐν ταῖς νήσοις ταῖς Βρεττανικαῖς, Σαυ-
ρομάτας, Ἰνδούς, τοὺς τὴν τῶν Περσῶν οἰκοῦντας γῆν, 
τὰ ἄλλα τὰ ἄπειρα γένη καὶ φῦλα ὧν οὐδὲ τὰ ὀνόματα 
ἴσμεν· ἀλλὰ πάντα ταῦτα τὰ ἔθνη «. 

Joannes Chrysostomus Scr. Eccl., Ad populum Antiochenum (homiliae 1–21) (2062: 024); MPG 49.
Vol 49, pg 106, ln 8

            Εἰ μὲν γὰρ διὰ βιβλίων ἐπαίδευσε καὶ διὰ 
γραμμάτων, ὁ μὲν εἰδὼς γράμματα ἔμαθεν ἂν τὰ ἐγγε-
γραμμένα, ὁ δὲ οὐκ εἰδὼς ἀπῆλθεν ἂν μηδὲν ἐκεῖθεν 
ὠφεληθεὶς, εἰ μή τις ἐνήγαγεν ἕτερος· καὶ ὁ μὲν εὔπο-
ρος ἐπρίατο ἂν τὸ βιβλίον, ὁ δὲ πένης οὐκ ἂν ἴσχυσε 
κτήσασθαι· πάλιν ὁ μὲν τὴν φωνὴν ἐκείνην εἰδὼς τὴν 
διὰ τῶν γραμμάτων σημαινομένην ἔγνω ἂν τὰ ἐγ-
κείμενα, ὁ δὲ Σκύθης, καὶ ὁ βάρβαρος, καὶ ὁ Ἰνδὸς, καὶ 
ὁ Αἰγύπτιος, καὶ πάντες οἱ τῆς γλώττης ἐκείνης ἀπεστε-
ρημένοι ἀπῆλθον ἂν μηδὲν μαθόντες· ἐπὶ δὲ τοῦ οὐ-
ρανοῦ οὐκ ἔστι τοῦτο εἰπεῖν, ἀλλὰ καὶ Σκύθης, καὶ βάρ-
βαρος, καὶ Ἰνδὸς, καὶ Αἰγύπτιος, καὶ πᾶς ἄνθρωπος 
ἐπὶ τῆς γῆς βαδίζων ταύτης ἀκούσεται τῆς φωνῆς· οὐ 
γὰρ δι' ὤτων, ἀλλὰ καὶ δι' ὄψεως εἰς τὴν διάνοιαν ἐμπί-
πτει τὴν ἡμετέραν. 



Joannes Chrysostomus Scr. Eccl., Ad populum Antiochenum (homiliae 1-21) 
Vol 49, pg 106, ln 12

γραμμένα, ὁ δὲ οὐκ εἰδὼς ἀπῆλθεν ἂν μηδὲν ἐκεῖθεν 
ὠφεληθεὶς, εἰ μή τις ἐνήγαγεν ἕτερος· καὶ ὁ μὲν εὔπο-
ρος ἐπρίατο ἂν τὸ βιβλίον, ὁ δὲ πένης οὐκ ἂν ἴσχυσε 
κτήσασθαι· πάλιν ὁ μὲν τὴν φωνὴν ἐκείνην εἰδὼς τὴν 
διὰ τῶν γραμμάτων σημαινομένην ἔγνω ἂν τὰ ἐγ-
κείμενα, ὁ δὲ Σκύθης, καὶ ὁ βάρβαρος, καὶ ὁ Ἰνδὸς, καὶ 
ὁ Αἰγύπτιος, καὶ πάντες οἱ τῆς γλώττης ἐκείνης ἀπεστε-
ρημένοι ἀπῆλθον ἂν μηδὲν μαθόντες· ἐπὶ δὲ τοῦ οὐ-
ρανοῦ οὐκ ἔστι τοῦτο εἰπεῖν, ἀλλὰ καὶ Σκύθης, καὶ βάρ-
βαρος, καὶ Ἰνδὸς, καὶ Αἰγύπτιος, καὶ πᾶς ἄνθρωπος 
ἐπὶ τῆς γῆς βαδίζων ταύτης ἀκούσεται τῆς φωνῆς· οὐ 
γὰρ δι' ὤτων, ἀλλὰ καὶ δι' ὄψεως εἰς τὴν διάνοιαν ἐμπί-
πτει τὴν ἡμετέραν. 



Joannes Chrysostomus Scr. Eccl., Ad populum Antiochenum (homiliae 1-21) 
Vol 49, pg 165, ln 41

                                              Ἰδοὺ γοῦν ἐξ ἐκεί-
νου μέχρι νῦν πόσος διαγέγονε χρόνος, καὶ λαμπρότε-
ρον τὸ ὄνομα τοῦ δεσμίου γέγονε τούτου· καὶ ὕπατοι μὲν 
ἅπαντες, ὅσοι γεγόνασιν ἐν τοῖς ἔμπροσθεν χρόνοις, σε-
σίγηνται καὶ οὐδὲ ἐκ προσηγορίας εἰσὶ γνώριμοι τοῖς 
πολλοῖς· τὸ δὲ τοῦ δεσμίου τούτου ὄνομα τοῦ μακαρίου 
Παύλου πολὺ μὲν ἐνταῦθα, πολὺ δὲ ἐν τῇ βαρβάρων χώ-
ρᾳ, πολὺ δὲ παρὰ Σκύθαις καὶ Ἰνδοῖς, κἂν πρὸς αὐτὰ 
τῆς οἰκουμένης ἔλθῃς τὰ πέρατα, ταύτης ἀκούσῃ τῆς 
προσηγορίας, καὶ ὅπουπερ ἄν τις ἀφίκηται, Παῦλον παν-
ταχοῦ ἐν τοῖς ἁπάντων στόμασι περιφερόμενον εἴσεται. 



Joannes Chrysostomus Scr. Eccl., De sancta pentecoste (homiliae 1–2) (2062: 037); MPG 50.
Vol 50, pg 459, ln 19

                   Ὁ βαπτιζόμενος εὐθέως ἐφθέγγετο 
τῇ τῶν Ἰνδῶν φωνῇ, τῇ τῶν Αἰγυπτίων, τῇ τῶν 
Περσῶν, τῇ τῶν Σκυθῶν, τῇ τῶν Θρᾳκῶν, καὶ εἷς 
ἄνθρωπος πολλὰς ἐλάμβανε γλώσσας, καὶ οὗτοι οἱ 
νῦν εἰ ἦσαν τότε βαπτισθέντες, εὐθέως ἂν ἤκουσας 
αὐτῶν διαφόροις φθεγγομένων φωναῖς. 



Joannes Chrysostomus Scr. Eccl., In principium Actorum (homiliae 1–4) (2062: 064); MPG 51.
Vol 51, pg 87, ln 45

                              Καὶ ὅτι πανταχοῦ τῆς οἰ-
κουμένης τὰς Γραφὰς ἥπλωσεν, ἄκουσον τοῦ προφή-
του λέγοντος· Εἰς πᾶσαν τὴν γῆν ἐξῆλθεν ὁ φθόγ-
γος αὐτῶν, καὶ εἰς τὰ πέρατα τῆς οἰκουμένης τὰ 
ῥήματα αὐτῶν. Κἂν πρὸς Ἰνδοὺς ἀπέλθῃς, οὓς 
πρώτους ἀνίσχων ὁ ἥλιος ὁρᾷ, κἂν εἰς τὸν ὠκεανὸν 
ἀπέλθῃς, κἂν πρὸς τὰς Βρεταννικὰς νήσους ἐκείνας, 
κἂν εἰς τὸν Εὔξεινον πλεύσῃς πόντον, κἂν πρὸς τὰ 
νότια ἀπέλθῃς μέρη, πάντων ἀκούσῃ πανταχοῦ τὰ 
ἀπὸ τῆς Γραφῆς φιλοσοφούντων, φωνῇ μὲν ἑτέρᾳ,    
πίστει δὲ οὐχ ἑτέρᾳ, καὶ γλώσσῃ μὲν διαφόρῳ, δια-
νοίᾳ δὲ συμφώνῳ. 



Joannes Chrysostomus Scr. Eccl., In principium Actorum (homiliae 1-4) 
Vol 51, pg 88, ln 28

                           Οὐ γὰρ τὸν Τίγρητα, οὐδὲ 
τὸν Εὐφράτην, οὐδὲ τὸν Αἰγύπτιον Νεῖλον, οὐδὲ τὸν 
Ἰνδὸν Γάγγην, ἀλλὰ μυρίους ἀφίησι ποταμοὺς αὕτη 
ἡ πηγή. 



Joannes Chrysostomus Scr. Eccl., In principium Actorum (homiliae 1-4) 
Vol 51, pg 92, ln 47

Ἐπειδὴ γὰρ ἔτι ἀσθενέστερον διέκειντο οἱ τότε, καὶ 
τὰ νοητὰ χαρίσματα ὁρᾷν οὐκ ἠδύναντο τοῖς ὀφθαλ-
μοῖς τῆς σαρκὸς, ἐδίδοτο αἰσθητὸν χάρισμα, ὥστε τὸ 
νοητὸν γενέσθαι καταφανές· καὶ ὁ βαπτισθεὶς εὐθέως 
καὶ τῇ ἡμετέρᾳ γλώσσῃ, καὶ τῇ τῶν Περσῶν, καὶ τῇ 
τῶν Ἰνδῶν, καὶ τῇ τῶν Σκυθῶν ἐφθέγγετο, ὥστε 
μαθεῖν καὶ τοὺς ἀπίστους, ὅτι Πνεύματος ἁγίου ἠξίωτο. 



Joannes Chrysostomus Scr. Eccl., Ad eos qui scandalizati sunt (2062: 087)
“Jean Chrysostome. Sur la providence de Dieu”, Ed. Malingrey, A.–M.
Paris: Cerf, 1961; Sources chrétiennes 79.
Chapter 22, section 9, line 6

Ἕκαστος γὰρ τὸ Εὐαγγέλιον ἀναγινώσκων τοῦτο, 
λέγει· «Οὐκ ἔξεστί σοι ἔχειν τὴν γυναῖκα Φιλίππου 
τοῦ ἀδελφοῦ σου»· καὶ τοῦ Εὐαγγελίου χωρὶς ἐν συλ-
λόγοις καὶ συνουσίαις, ταῖς οἴκοι, ταῖς ἐν ἀγορᾷ, ταῖς 
ἁπανταχοῦ, κἂν εἰς τὴν Περσῶν χώραν ἀπέλθῃς, κἂν εἰς 
τὴν Ἰνδῶν, κἂν εἰς τὴν Μαύρων, κἂν ὅσην ἥλιος ἐφορᾷ 
γῆν καὶ πρὸς αὐτὰς τὰς ἐσχατιάς, ταύτης ἀκούσῃς τῆς 
φωνῆς καὶ ὄψει τὸν δίκαιον ἐκεῖνον ἔτι καὶ νῦν βοῶντα, 
ἐνηχοῦντα καὶ τὴν κακίαν ἐλέγχοντα τοῦ τυράννου καὶ 
οὐδέποτε σιγῶντα, οὐδὲ τῷ πλήθει τοῦ χρόνου τὸν ἔλεγχον 
μαραινόμενον. 



Joannes Chrysostomus Scr. Eccl., De Chananaea [Dub.] (2062: 101); MPG 52.
Vol 52, pg 453, ln 48

                   ἡ δὲ οἰκουμένη πᾶσα ἔρημος, 
Σκύθαι, Θρᾷκες, Ἰνδοὶ, Μαῦροι, Κίλικες, Καππά-
δοκες, Σύροι, Φοίνικες, ὅσην ὁ ἥλιος ἐφορᾷ γῆν; 



Joannes Chrysostomus Scr. Eccl., De Chananaea [Dub.] 
Vol 52, pg 460, ln 2

              Ὅπου ἐὰν ἀπέλθῃς, ἀκούεις τοῦ 
Χριστοῦ λέγοντος, Ὦ γύναι, μεγάλη σου ἡ πίστις. 
Εἴσελθε εἰς Περσῶν τὴν ἐκκλησίαν, καὶ ἀκούσεις τοῦ    
Χριστοῦ λέγοντος, Ὦ γύναι, μεγάλη σου ἡ πίστις· 
εἰς τὴν Γότθων, εἰς τὴν βαρβάρων, εἰς τὴν Ἰνδῶν, 
εἰς τὴν Μαύρων, ὅσην ἥλιος ἐφορᾷ γῆν· ἕνα λόγον ὁ 
Χριστὸς ἐφθέγξατο, καὶ οὐ σιωπᾷ ὁ λόγος, ἀλλὰ 
μεγάλῃ τῇ φωνῇ ἀνακηρύττει τὴν πίστιν αὐτῆς, λέ-
γων, Ὦ γύναι, μεγάλη σου ἡ πίστις· γενηθήτω 
σοι ὡς θέλεις. Οὐκ εἶπε, Θεραπευθήτω τὸ θυγάτριόν 
σου· ἀλλ', Ὡς θέλεις. Σὺ αὐτὴν θεράπευσον· σὺ 
γενοῦ ἰατρός· σοὶ ἐγχειρίζω τὸ φάρμακον· ὕπαγε, 
ἐπίθες, Γενηθήτω σοι ὡς θέλεις. Τὸ θέλημά σου 
θεραπευσάτω αὐτήν. 



Joannes Chrysostomus Scr. Eccl., In pentecosten (sermo 1) [Sp.] (2062: 107); MPG 52.
Vol 52, pg 808, ln 12

         Ὅπου δ' ἂν ἀπέλθῃς, εἰς Ἰνδοὺς, εἰς Μαυ-
ροὺς, εἰς Βρετανοὺς, εἰς τὴν οἰκουμένην, εὑρήσεις, 
Ἐν ἀρχῇ ἦν ὁ Λόγος, καὶ βίον ἐνάρετον. 



Joannes Chrysostomus Scr. Eccl., In Genesim (homiliae 1–67) (2062: 112); MPG 53:21–385; 54:385–580.
Vol 53, pg 258, ln 21

       Κἂν πρὸς Ἰνδοὺς γὰρ ἀπέλθῃς, κἂν πρὸς Σκύθας, 
κἂν πρὸς αὐτὰ τὰ πέρατα τῆς οἰκουμένης, κἂν εἰς αὐτὸν 
τὸν ὠκεανὸν, πανταχοῦ εὑρήσεις τοῦ Χριστοῦ τὴν διδασκα-
λίαν καταυγάζουσαν τὰς ἁπάντων ψυχάς. 



Joannes Chrysostomus Scr. Eccl., De Anna (sermones 1–5) (2062: 114); MPG 54.
Vol 54, pg 664, ln 8

         Βασιλεῖς μὲν γὰρ καὶ στρατηγοὶ καὶ δυνάσται,    
πολλὰ πραγματευσάμενοι πολλάκις, ὥστε αὐτῶν ἄλη-
στον γενέσθαι τὴν μνήμην, καὶ τάφους λαμπροὺς οἰκο-
δομησάμενοι, καὶ ἀνδριάντας ἀναστήσαντες, καὶ εἰκό-
νας πολλὰς πολλαχοῦ, καὶ κατορθωμάτων ὑπομνήματα 
μυρία καταλιπόντες, σεσίγηνται, καὶ οὐδὲ ἀπὸ ψιλῆς 
προσηγορίας εἰσί τινι γνώριμοι· αὕτη δὲ ἡ γυνὴ παν-
ταχοῦ τῆς οἰκουμένης ᾄδεται νῦν· κἂν εἰς Σκυθίαν 
ἀπέλθῃς, κἂν εἰς Αἴγυπτον, κἂν εἰς Ἰνδοὺς, κἂν εἰς 
αὐτὰ τὰ πέρατα τῆς οἰκουμένης, πάντων ἀκούσῃ τὰ 
κατορθώματα τῆς γυναικὸς ταύτης ᾀδόντων, καὶ πᾶ-
σαν ἁπλῶς, ὅσην ἥλιος ἐφορᾷ γῆν, ἡ δόξα τῆς Ἄννης 
καταλαμβάνει. 



Joannes Chrysostomus Scr. Eccl., Homilia de capto Eutropio [Dub.] (2062: 142); MPG 52.
Vol 52, pg 409, ln 19

Πρὸς Θρᾷκας, πρὸς Σκύθας, πρὸς Ἰνδοὺς, πρὸς Μαύ-
ρους, πρὸς Σαρδονίους, πρὸς Γοτθοὺς, πρὸς θηρία 
ἄγρια, καὶ μετέβαλε πάντα. 



Joannes Chrysostomus Scr. Eccl., Expositiones in Psalmos (2062: 143); MPG 55.
Vol 55, pg 58, ln 4

      Οὐ γὰρ οὕτω Σκύθαι, οὐδὲ Θρᾷκες, οὐ Σαυρομά-
ται, καὶ Ἰνδοὶ, καὶ Μαῦροι, καὶ ὅσα ἄγρια ἔθνη πολε-
μεῖν εἰώθασιν, ὡς λογισμὸς ἀτοπώτατος ἐνδομυχῶν τῇ 
ψυχῇ, καὶ ἐπιθυμία ἀκόλαστος, καὶ χρημάτων ἔρως, 
καὶ δυναστείας πόθος, καὶ τῶν βιωτικῶν πραγμάτων 
ἡ προσπάθεια. 



Joannes Chrysostomus Scr. Eccl., Expositiones in Psalmos 
Vol 55, pg 203, ln 7

                                                Ὁ Ῥω-
μαίων βασιλεὺς οὐκ ἂν δύναιτο νομοθετεῖν Πέρσαις, 
οὐδὲ ὁ Περσῶν Ῥωμαίοις· οἱ δὲ Παλαιστινοὶ οὗτοι 
καὶ Πέρσαις, καὶ Ῥωμαίοις, καὶ Θρᾳξὶ, καὶ Σκύ-
θαις, καὶ Ἰνδοῖς, καὶ Μαύροις, καὶ πάσῃ τῇ οἰκου-
μένῃ νόμους ἔθηκαν· καὶ οὐχὶ ζώντων αὐτῶν ἐκρά-
τησαν μόνον, ἀλλὰ καὶ τελευτησάντων· καὶ μυριά-
κις ἂν ἕλοιντο οἱ νομοθετηθέντες τὴν ψυχὴν ἀφεῖναι, 
ἢ τῶν νόμων ἀποστῆναι ἐκείνων. 



Joannes Chrysostomus Scr. Eccl., Expositiones in Psalmos 
Vol 55, pg 390, ln 16

                  Ἔφη γὰρ, Ἐν ταῖς θαλάσσαις 
καὶ ἐν πάσαις ταῖς ἀβύσσοις. Ἥ τε γὰρ Κασπία, 
ἥ τε Ἰνδικὴ, ἥ τε Ἐρυθρὰ διῃρημέναι σχεδὸν ταύ-
της εἰσὶ, καὶ ἔξωθεν περικείμενος ὁ Ὠκεανός. 



Joannes Chrysostomus Scr. Eccl., Expositiones in Psalmos 
Vol 55, pg 467, ln 57

μένων καὶ περὶ ἡμᾶς, τροπὰς ἐτησίους, ἡμέρας, 
παραδείσους, λειμῶνας, ἄνθη ποικίλα, ὕδωρ πότιμον 
καὶ γλυκὺ, καὶ τὸ ἐκ τῶν ὑετῶν χρήσιμον, τῆς γῆς 
τὰς ὠδῖνας, τοὺς ποικίλους καρποὺς, τὰ δένδρα τὰ 
διάφορα, τοὺς ἀνέμους τοὺς προσηνεῖς, τὴν ἡλιακὴν 
ἀκτῖνα, τὴν σεληναίαν λαμπάδα, τὸν ποικίλον τῶν 
ἄστρων χορὸν, τὸ προσηνὲς τῆς νυκτός· καὶ ἐπὶ τῶν 
ἀλόγων, πρόβατα, καὶ βοῦς, καὶ αἶγας· καὶ ἐπὶ τῶν 
ἀγρίων, δορκάδας, καὶ ἐλάφους, λαγωοὺς, καὶ ἕτερα 
πλείονα· καὶ ἐπὶ τῶν πετεινῶν, τοὺς ὄρνιθας τοὺς 
Ἰνδικούς· καὶ ἐν τοῖς ἔργοις αὐτοῖς ἴδοι τις ἂν οὐ 
κολάζοντα μόνον, ἀλλὰ καὶ εὐεργετοῦντα πολλῷ    
πλεῖον ἢ κολάζοντα. 



Joannes Chrysostomus Scr. Eccl., In illud Isaiae: Ego dominus deus feci lumen (2062: 148); MPG 56.
Vol 56, pg 143, ln 53

              Οἷόν τι λέγω, Ἑλλάδι διαλεγομένῳ μοι 
γλώττῃ, ἂν τοίνυν τὴν φωνὴν εἰδῇ τις, ἐκεῖνος 
ἀκούσεταί μου· ὁ δὲ Σκύθης, καὶ ὁ Θρᾷξ, καὶ ὁ Μαῦ-
ρος, καὶ ὁ Ἰνδὸς οὐκέτι· ἡ γὰρ διαφορὰ τῆς γλώττης 
οὐκ ἀφίησιν εὔσημον αὐτῷ γενέσθαι τὴν ἐμὴν διά-
λεξιν. 



Joannes Chrysostomus Scr. Eccl., In illud Isaiae: Ego dominus deus feci lumen 
Vol 56, pg 144, ln 18

                                                       Οὐ 
γάρ ἐστι λαλιὰ, φησὶ, τουτέστιν, οὐκ ἔστιν ἔθνος, οὐκ 
ἔστι φωνὴ, ἔνθα μὴ ἀκούεται ἡ φωνὴ τοῦ οὐρανοῦ· 
ἀλλὰ καὶ ὁ Σκύθης, καὶ ὁ Θρᾷξ, καὶ ὁ Μαῦρος, καὶ ὁ 
Ἰνδὸς, καὶ ὁ Σαυρομάτης, καὶ πᾶσα λαλιὰ, καὶ πᾶσα 
γλῶττα, καὶ πᾶν ἔθνος δυνήσεται ταύτης ὑπακούειν 
τῆς φωνῆς. 



Joannes Chrysostomus Scr. Eccl., In illud Isaiae: Ego dominus deus feci lumen 
Vol 56, pg 144, ln 39

                            Ἐπειδὴ γὰρ οὐκ ἔστιν ἀκοῇ 
ταῦτα μαθεῖν, ἀλλ' ὄψει καὶ θεωρίᾳ, ὄψις δὲ πᾶσι 
μία, εἰ καὶ ἡ γλῶττα διάφορος, καὶ Βάρβαρος, καὶ 
Σκύθαι, καὶ Θρᾷκες, καὶ Μαῦροι, καὶ Ἰνδοὶ ταύτης 
ἀκούουσι τῆς φωνῆς, τουτέστι, τὸ θαῦμα βλέποντες, 
τὸ κάλλος ἐκπληττόμενοι, τὴν φαιδρότητα, τὸ μέγε-
θος, τὰ ἄλλα ἅπαντα τὰ πρὸς τὸν οὐρανὸν, δόξαν ἀνα-
φέρουσι τῷ Δημιουργῷ οἱ καλῶς φρονοῦντες. 



Joannes Chrysostomus Scr. Eccl., De prophetiarum obscuritate (homiliae 1–2) (2062: 150); MPG 56.
Vol 56, pg 179, ln 24

                                   Οὐκ ἦν ἑτερόγλωσσος 
ἀπ' ἀρχῆς, οὐκ ἦν ἑτερόφωνος, οὐκ ἦν Ἰνδὸς, οὔτε 
Θρᾲξ, οὔτε Σκύθης, ἀλλὰ πάντες μιᾷ διελέγοντο 
γλώσσῃ. 



Joannes Chrysostomus Scr. Eccl., In Matthaeum (homiliae 1–90) (2062: 152); MPG 57:13–472; 58:471–794.
Vol 58, pg 725, ln 53

καὶ στρατηγῶν ἀνδραγαθήματα, ὧν καὶ τὰ ὑπομνήματα 
μένει, σεσίγηνται· καὶ πόλεις ἀναστήσαντες, καὶ τείχη 
περιβαλόντες, καὶ πολέμους νικήσαντες, καὶ τρόπαια 
στήσαντες, καὶ ἔθνη πολλὰ δουλωσάμενοι, οὐδὲ ἐξ ἀκοῆς, 
οὐδὲ ἐξ ὀνόματός εἰσι γνώριμοι, καίτοι καὶ ἀνδριάντας 
ἀναστήσαντες καὶ νόμους θέντες· ὅτι δὲ πόρνη γυνὴ 
ἔλαιον ἐξέχεεν ἐν οἰκίᾳ λεπροῦ τινος, δέκα ἀνδρῶν 
παρόντων, τοῦτο πάντες ᾄδουσι κατὰ τὴν οἰκουμένην, 
καὶ χρόνος τοσοῦτος διῆλθε, καὶ ἡ μνήμη τοῦ γενομέ-
νου οὐκ ἐμαράνθη· ἀλλὰ καὶ Πέρσαι, καὶ Ἰνδοὶ, 
καὶ Σκύθαι, καὶ Θρᾷκες, καὶ Σαυρομάται, καὶ 
τὸ Μαύρων γένος, καὶ οἱ τὰς Βρεττανικὰς νήσους 
οἰκοῦντες τὸ ἐν Ἰουδαίᾳ γενόμενον λάθρα ἐν οἰκίᾳ παρὰ 
γυναικὸς πεπορνευμένης περιφέρουσι. 



Joannes Chrysostomus Scr. Eccl., In Joannem (homiliae 1–88) (2062: 153); MPG 59.
Vol 59, pg 32, ln 20

                                                        Ἀλλ' οὐ 
τὰ τοῦ ἰδιώτου καὶ ἀγραμμάτου οὕτως· ἀλλὰ καὶ Σύροι, 
καὶ Αἰγύπτιοι, καὶ Ἰνδοὶ, καὶ Πέρσαι, καὶ Αἰθίοπες, καὶ 
μυρία ἕτερα ἔθνη, εἰς τὴν αὐτῶν μεταβαλόντες γλῶτταν 
τὰ παρὰ τούτου δόγματα εἰσαχθέντα, ἔμαθον ἄνθρωποι 
βάρβαροι φιλοσοφεῖν. 



Joannes Chrysostomus Scr. Eccl., In Joannem (homiliae 1-88) 
Vol 59, pg 361, ln 58

                                                      Ὁ ἐν 
Ῥώμῃ καθήμενος τοὺς Ἰνδοὺς μέλος εἶναι νομίζει    
ἑαυτοῦ. 



Joannes Chrysostomus Scr. Eccl., In Acta apostolorum (homiliae 1–55) (2062: 154); MPG 60.
Vol 60, pg 47, ln 32

                      Καὶ ὁ μὲν πολλὰ ληρήσας Πλά-
των, σεσίγηκεν· οὗτος δὲ φθέγγεται, οὐχὶ παρ' 
οἰκείοις μόνοις, ἀλλὰ καὶ παρὰ Πάρθοις, παρὰ Μή-
δοις, παρὰ Ἐλαμίταις, καὶ ἐν Ἰνδίᾳ, καὶ πανταχοῦ 
γῆς, καὶ εἰς τὰ πέρατα τῆς οἰκουμένης. 



Joannes Chrysostomus Scr. Eccl., In Acta apostolorum (homiliae 1-55) 
Vol 60, pg 56, ln 26

                                      Κἂν εἰς Ἰν-
δοὺς ἀπέλθῃς, ἀκούσῃ τούτων· κἂν εἰς Ἱσπανίαν, 
κἂν πρὸς αὐτὰ τῆς γῆς τὰ τέρματα, οὐδεὶς ἀν-
ήκοος τυγχάνει, πλὴν εἰ μὴ παρὰ τὴν οἰκείαν ῥᾳθυ-
μίαν. 



Joannes Chrysostomus Scr. Eccl., In Acta apostolorum (homiliae 1-55) 
Vol 60, pg 220, ln 18

                                     Καὶ ἐν Ἰνδοῖς δὲ τὸ μέγα 
θηρίον καὶ φοβερὸν τῶν ἐλεφάντων λέγεται καὶ δεκα-
πενταέτει παιδὶ γεγονότι μετὰ πολλῆς εἴκειν τῆς προθυ-
μίας. 



Joannes Chrysostomus Scr. Eccl., In epistulam ad Romanos (homiliae 1–32) (2062: 155); MPG 60.
Vol 60, pg 517, ln 18

                         Καὶ οὐ παρ' ἡμῖν μόνον, ἀλλὰ 
καὶ παρὰ Σκύθαις καὶ Θρᾳξὶ καὶ Ἰνδοῖς καὶ Πέρσαις, 
καὶ ἑτέροις δὲ βαρβάροις πλείοσι, καὶ παρθένων χοροὶ 
καὶ μαρτύρων δῆμοι καὶ μοναχῶν συμμορίαι, καὶ πλεί-
ους οὗτοι λοιπὸν τῶν γεγαμηκότων εἰσὶ, καὶ νηστείας 
ἐπίτασις καὶ ἀκτημοσύνης ὑπερβολή· ἅπερ, πλὴν ἑνὸς 
ἢ δυεῖν, οὐ φαντασθῆναι ὄναρ οἱ κατὰ τὸν νόμον ἠδυνή-
θησαν πολιτευόμενοι. 



Joannes Chrysostomus Scr. Eccl., In epistulam i ad Corinthios (homiliae 1–44) (2062: 156); MPG 61.
Vol 61, pg 52, ln 3

πῶς δ' ἂν τὰ γραφέντα, καὶ εἰς τὴν βαρβάρων καὶ 
εἰς τὴν Ἰνδῶν, καὶ πρὸς αὐτὰ τοῦ ὠκεανοῦ τὰ πέ-
ρατα ἀφίκετο, οὐκ ὄντων τῶν λεγόντων ἀξιοπίστων; 



Joannes Chrysostomus Scr. Eccl., In epistulam i ad Corinthios (homiliae 1-44) 
Vol 61, pg 53, ln 12

                                        Ἡμεῖς δὲ βουλό-
μεθα πολλῆς ἀπολαύειν τρυφῆς καὶ ἀναπαύσεως καὶ 
ἀδείας· ἀλλ' οὐκ ἐκεῖνοι, ἀλλ' ἐβόων, Ἄχρι τῆς 
ἄρτι ὥρας καὶ πεινῶμεν καὶ διψῶμεν καὶ γυμνη-
τεύομεν καὶ κολαφιζόμεθα καὶ ἀστατοῦμεν. Καὶ 
ὁ μὲν ἀπὸ Ἱερουσαλὴμ μέχρι τοῦ Ἰλλυρικοῦ ἔτρε-
χεν, ὁ δὲ εἰς τὴν Ἰνδῶν, ὁ δὲ εἰς τὴν Μαύρων, ἄλ-
λος δὲ πρὸς ἄλλα μέρη τῆς οἰκουμένης· ἡμεῖς δὲ 
οὐδὲ τῆς πατρίδος ἐξελθεῖν τολμῶμεν, ἀλλὰ τρυφὴν 
ζητοῦμεν καὶ οἰκίας λαμπρὰς καὶ πᾶσαν τὴν ἄλλην 
ἀφθονίαν. 



Joannes Chrysostomus Scr. Eccl., In epistulam i ad Corinthios (homiliae 1-44) 
Vol 61, pg 68, ln 22

διὰ τοῦτο αὐτὸ μάλιστα θαυμάζειν ἐχρῆν, ὅτι ἀν-
θρώπους βαρβάρους τοιαύτην ἔπεισαν καταδέξασθαι 
πίστιν, καὶ χρηστὰς περὶ τῶν μελλόντων ἔχειν 
ἐλπίδας, καὶ τὸ πρότερον τῶν ἁμαρτημάτων φορτίον 
ἀπεσκευασμένους μετὰ πολλῆς τῆς προθυμίας εἰς τὸ 
ἐπιὸν τῶν ὑπὲρ τῆς ἀρετῆς ἅπτεσθαι πόνων, καὶ 
πρὸς αἰσθητὸν μὲν μηδὲν κεχηνέναι, πάντων δὲ 
ἀνωτέρους τῶν σωματικῶν γεγενημένους νοερὰς δέ-
ξασθαι δωρεὰς, καὶ τὸν Πέρσην καὶ τὸν Σαυρομάτην, 
καὶ τὸν Μαῦρον καὶ τὸν Ἰνδὸν εἰδέναι ψυχῆς καθ-
αρμὸν, καὶ Θεοῦ δύναμιν καὶ φιλανθρωπίαν ἄφατον, 
καὶ πίστεως φιλοσοφίαν, καὶ Πνεύματος ἁγίου ἐπι-
φοίτησιν, καὶ σωμάτων ἀνάστασιν, καὶ ζωῆς ἀθανάτου 
δόγματα. 



Joannes Chrysostomus Scr. Eccl., In epistulam i ad Corinthios (homiliae 1-44) 
Vol 61, pg 239, ln 39

                        Ἐπειδὴ γὰρ ἀπὸ τῶν εἰδώλων 
προσιόντες, οὐδὲν εἰδότες σαφῶς, οὐδὲ ταῖς παλαιαῖς 
ἐντραφέντες βίβλοις, βαπτισθέντες εὐθέως Πνεῦμα 
ἐλάμβανον, τὸ δὲ Πνεῦμα οὐχ ἑώρων· ἀόρατον γάρ 
ἐστιν· αἴσθητόν τινα ἔλεγχον ἐδίδου τῆς ἐνεργείας 
ἐκείνης ἡ χάρις· καὶ ὁ μὲν τῇ Περσῶν, ὁ δὲ τῇ Ῥω-
μαίων, ὁ δὲ τῇ Ἰνδῶν, ὁ δὲ ἑτέρᾳ τινὶ τοιαύτῃ εὐ-
θέως ἐφθέγγετο γλώσσῃ· καὶ τοῦτο ἐφανέρου τοῖς 
ἔξωθεν, ὅτι Πνεῦμά ἐστιν ἐν αὐτῷ τῷ φθεγγομένῳ. 



Joannes Chrysostomus Scr. Eccl., In epistulam i ad Corinthios (homiliae 1-44) 
Vol 61, pg 296, ln 45

                  Καὶ ὥσπερ ἐν τῷ καιρῷ τῆς πυρ-
γοποιίας ἡ μία γλῶττα εἰς πολλὰς διετέμνετο· οὕτω 
τότε αἱ πολλαὶ πολλάκις εἰς ἕνα ἄνθρωπον ᾔεσαν, καὶ 
ὁ αὐτὸς καὶ τῇ Περσῶν καὶ τῇ Ῥωμαίων καὶ τῇ 
Ἰνδῶν καὶ ἑτέραις πολλαῖς διελέγετο γλώτταις, τοῦ 
Πνεύματος ἐνηχοῦντος αὐτῷ· καὶ τὸ χάρισμα ἐκαλεῖτο 
χάρισμα γλωττῶν, ἐπειδὴ πολλαῖς ἀθρόον ἐδύνατο 
λαλεῖν φωναῖς. 



Joannes Chrysostomus Scr. Eccl., In epistulam i ad Corinthios (homiliae 1-44) 
Vol 61, pg 299, ln 2

                                     Τοσαῦτα, εἰ τύχοι, 
γένη φωνῶν ἐστιν ἐν κόσμῳ, καὶ οὐδὲν αὐτῶν 
ἄφωνον. Τουτέστι, Τοσαῦται γλῶσσαι, τοσαῦται φω-  
ναὶ, Σκυθῶν, Θρᾳκῶν, Ῥωμαίων, Περσῶν, Μαύρων, 
Ἰνδῶν, Αἰγυπτίων, ἑτέρων μυρίων ἐθνῶν. 



Joannes Chrysostomus Scr. Eccl., In epistulam ii ad Corinthios (homiliae 1–30) (2062: 157); MPG 61.
Vol 61, pg 506, ln 45

Τοιούτους τοὺς ἁμαξοβίους εἶναί φασι, τοὺς 
παρὰ Σκύθαις νομάδας, τοὺς γυμνοσοφιστὰς τοὺς 
τῶν Ἰνδῶν. 



Joannes Chrysostomus Scr. Eccl., In epistulam ad Philippenses (homiliae 1–15) (2062: 160); MPG 62.
Vol 62, pg 237, ln 17

                                         ἀλλ' οὐχ οὕτως, 
ὡς οἱ Ἰνδικοὶ μύρμηκες. 



Joannes Chrysostomus Scr. Eccl., In epistulam i ad Thessalonicenses (homiliae 1–11) (2062: 162); MPG 62.
Vol 62, pg 405, ln 16

Ὥσπερ οὖν, εἰ περί τινος ἔλεγον φυτοῦ ἐν Ἰνδίᾳ 
τικτομένου, οὗ μηδεὶς μηδὲ πεῖραν ἔλαβεν, οὐκ ἂν 
ἴσχυσεν ὁ λόγος παραστῆσαι, κἂν εἰ μυρία εἶπον· 
οὕτω καὶ νῦν ὅσα ἂν εἴπω, εἰκῆ ἐρῶ· οὐδεὶς γὰρ 
ἐπιστῆναι δυνήσεται. 



Joannes Chrysostomus Scr. Eccl., In epistulam i ad Timotheum (homiliae 1–18) (2062: 164); MPG 62.
Vol 62, pg 513, ln 15

             Καυχᾷ ἐπὶ πράγματι, ὃ σκώληκες τί-
κτουσι, καὶ ἀπολλύουσι· λέγονται γὰρ Ἰνδικά τινα 
ζωΰφια εἶναι, ὅθεν τὰ νήματα ταῦτα κατασκευάζεται. 



Joannes Chrysostomus Scr. Eccl., In epistulam i ad Timotheum (homiliae 1-18) 
Vol 62, pg 513, ln 39

Τί ἄν τις εἴποι τὴν τῶν ἀρωμάτων πολυτέλειαν, τῶν 
Ἰνδικῶν, τῶν Ἀραβικῶν, τῶν Περσικῶν. 



Joannes Chrysostomus Scr. Eccl., In epistulam i ad Timotheum (homiliae 1-18) 
Vol 62, pg 596, ln 38

                                                     Ἐν δὲ 
τῇ ἀρωματοφόρῳ Ἀραβίᾳ καὶ Ἰνδίᾳ, ἔνθα εἰσὶν οἱ 
λίθοι, πολλὰ τοιαῦτα ἔστιν εὑρεῖν. 



Joannes Chrysostomus Scr. Eccl., Homilia habita postquam presbyter Gothus concionatus fuerat (2062: 177); MPG 63.
Vol 63, pg 501, ln 5

                                                     οὐκ ἐν 
Ἰουδαίᾳ μόνον, ἀλλὰ καὶ ἐν τῇ τῶν βαρβάρων γλώττῃ, 
καθὼς ἠκούσατε σήμερον, ἡλίου φανότερον διαλάμπει· καὶ 
Σκύθαι καὶ Θρᾷκες καὶ Σαυρομάται καὶ Μαῦροι καὶ Ἰνδοὶ 
καὶ οἱ πρὸς αὐτὰς ἀπῳκισμένοι τὰς ἐσχατιὰς τῆς οἰκουμέ-
νης, πρὸς τὴν οἰκείαν ἕκαστος μεταβαλόντες γλῶτταν, τὰ 
εἰρημένα φιλοσοφοῦσι ταῦτα· ἃ μηδὲ ὄναρ ἐφαντάσθησαν 
οἱ παρὰ τοῖς Ἕλλησι τὸν πώγωνα ἕλκοντες, καὶ 
τῇ βακτηρίᾳ τοὺς ἀπαντῶντας ἐπὶ τῆς ἀγορᾶς σοβοῦντες, 
καὶ τοὺς βοστρύχους ἀπὸ τῆς κεφαλῆς σείοντες, λεόντων 
μᾶλλον ἢ ἀνθρώπων ἐπιδεικνύμενοι πρόσωπα. 



Joannes Chrysostomus Scr. Eccl., In illud: Messis quidem multa (2062: 179); MPG 63.
Vol 63, pg 519, ln 55

                           Σκύθαι μὲν γὰρ καὶ Θρᾷκες 
καὶ Μαῦροι καὶ Ἰνδοὶ καὶ Πέρσαι καὶ Σαυρομάται, καὶ 
οἱ τὴν Ἑλλάδα καὶ οἱ τὴν Ἤπειρον οἰκοῦντες, καὶ πᾶσα, 
ὡς εἰπεῖν, ἡ ὑφ' ἡλίῳ τοῖς δαίμοσιν ἐτελεῖτο, καὶ ὑπὸ 
τῶν ἀλαστόρων ἐκείνων ἐξεβακχεύετο καὶ χώρα καὶ 
πόλις καὶ ἔρημος, καὶ γῆ καὶ θάλαττα, καὶ βάρβαρος 
καὶ Ἑλλὰς, καὶ ὄρη καὶ νάπαι καὶ βουνοί· μόνον δὲ τὸ 
Ἰουδαίων ἔθνος τὸ δοκοῦν εὐσεβεῖν προφήτας εἶχε    
καὶ θεογνωσίας σπέρματα μικρά· ἀλλὰ καὶ ταῦτα χρόνῳ 
κατεχώννυτο, καὶ οἱ διδάσκαλοι τοῦ γένους ἐκείνου κατ-
ήγοροι πικροὶ τῶν ἁμαρτανομένων αὐτοῖς ἐγίνοντο· 




Joannes Chrysostomus Scr. Eccl., Fragmenta in Jeremiam (in catenis) (2062: 186); MPG 64.
Vol 64, pg 829, ln 5

         Σαβὰ δὲ χώρα ἐστὶν Ἰνδῶν. 



Joannes Chrysostomus Scr. Eccl., De perfecta caritate [Sp.] (2062: 211); MPG 56.
Vol 56, pg 282, ln 7

                                                        Ὥς-
περ γὰρ εἰ ἔλεγον περί τινος φυτοῦ ἐν Ἰνδίᾳ τικτομέ-
νου, καὶ οὗ μηδεὶς πεῖραν ἔλαβεν, οὐκ ἂν ἴσχυσεν ὁ 
λόγος παραστῆσαι, κἂν μυρία περὶ αὐτοῦ ἔλεγον· οὕτω 
καὶ νῦν ὅσα ἂν εἴπω, μάτην ἐρῶ· οὐ γὰρ συνιοῦσί τινες 
τὰ λεγόμενα. 



Joannes Chrysostomus Scr. Eccl., In Genesim (sermo 3) [Sp.] (2062: 216); MPG 56.
Vol 56, pg 527, ln 17

            Διὸ ταύτην ὑποδεξώμεθα, τὸ τίμιον δῶρον, 
τὸ ἅγιον κειμήλιον, τὸ τῆς ἀληθείας ἴνδαλμα, τῆς εὐσε-
βείας τὸ κεφάλαιον, τὸν τῆς πνευματικῆς διδασκαλίας 
συνήγορον, τῶν παθῶν τὴν νέκρωσιν, τῆς ἁμαρτίας τὴν 
ἀναίρεσιν, τῆς κακίας τὸν ἀντίπαλον, τῆς παρθενίας τὴν 
ὁμόδοξον, τῶν δαιμόνων τὴν ἀφανίστριαν, τοῦ διαβόλου 
τὴν ἀποτίμησιν, τῶν εἰδώλων τὴν καθαίρεσιν, τῶν Ἐκ-
κλησιῶν τὴν εὐπρέπειαν, τῶν βασιλέων τὸ κράτος, τῶν 
ἱερέων τὸ ἐγκαλλώπισμα, τῶν ἀνδρῶν τὴν φρόνησιν, τῶν 
γυναικῶν τὴν σωφροσύνην, τῶν νηπίων τὴν παιδαγωγίαν, 
τῶν δούλων τὴν ἀνάῤῥυσιν, τῶν πτωχῶν τὴν

παραμυ-





Joannes Chrysostomus Scr. Eccl., In adorationem venerandae crucis [Sp.] (2062: 326); MPG 62.
Vol 62, pg 753, ln 38

                                           Τοῦτο τὸ τί-
μιον καὶ σεβάσμιον ξύλον ὑπὸ πάντων τιμώμενον προς-
κυνεῖται· Ἕλληνές τε καὶ βάρβαροι, Μακεδόνες καὶ 
Θετταλοὶ, Παίονες, καὶ Ἰλλυριοὶ, Ἀθηναῖοι καὶ Ἀργεῖοι 
καὶ Λάκωνες, Πάρθοι καὶ Μῆδοι καὶ Ἐλαμῖται, καὶ οἱ 
κατοικοῦντες τὴν Μεσοποταμίαν, Ἰουδαίαν τε καὶ Καπ-
παδοκίαν, Πόντον καὶ Ἀσίαν, Αἴγυπτον καὶ τὰ μέρη τῆς 
Λιβύης τῆς κατὰ Κυρήνην, Κρῆτες καὶ Ἄραβες, 
Ἰνδοὶ καὶ Αἰθίοπες, καὶ Ὁμηρῖται, καὶ πάντες οἱ λοι-
ποὶ τῶν ἐθνῶν, ὅσους ὁ ἥλιος ἐφορᾷ, τὴν ἑαυτῶν κατα-
λιπόντες ἀπάτην, τῷ σταυρῷ σημειούμενοι προσκυ-
νοῦσι, φεύγοντες τὰς πολυπλόκους σειρὰς τοῦ διαβόλου. 



Joannes Chrysostomus Scr. Eccl., In illud: Si qua in Christo nova creatura [Sp.] (2062: 359); MPG 64.
Vol 64, pg 27, ln 1

                                                        ἀλλ', 
ὅπου ἂν ἀπέλθῃς, ἀκούεις, ὅτι Ἐν ἀρχῇ ἦν ὁ Λόγος· 
κἂν ἐν χώρᾳ, κἂν ἐν πόλει, προέλαβεν, Ἐν ἀρχῇ ἦν   
ὁ Λόγος· καὶ ἐν Περσίδι καὶ ἐν Ἰνδίᾳ καὶ ἐν τῇ Μαυ-
ριτανῶν χώρᾳ λάμπει τὸ ῥῆμα τοῦτο ἡλίου φανερώ-
τερον. 



Joannes Chrysostomus Scr. Eccl., De laudibus sancti Pauli apostoli (homiliae 1–7) (2062: 486)
“Jean Chrysostome. Panégyriques de S. Paul”, Ed. Piédagnel, A.
Paris: Cerf, 1982; Sources chrétiennes 300.
Homily 4, section 8, line 3

           Πάντως ἠκούσατε, ὅτι καὶ παρὰ Πέρσαις καὶ 
Ἰνδοῖς πολλοὶ γεγόνασι μάγοι, καί εἰσιν ἔτι καὶ νῦν· ἀλλ' 
οὐδὲ ὄνομα αὐτῶν ἐστιν οὐδαμοῦ. 



Joannes Chrysostomus Scr. Eccl., De laudibus sancti Pauli apostoli (homiliae 1-7) 
Homily 4, section 10, line 8

                                                 Ἄνθρωπος 
γὰρ ἐπ' ἀγορᾶς ἑστηκώς, περὶ δέρματα τὴν τέχνην ἔχων, 
τοσοῦτον ἴσχυσεν, ὡς καὶ Ῥωμαίους, καὶ Πέρσας, καὶ 
Ἰνδούς, καὶ Σκύθας, καὶ Αἰθίοπας, καὶ Σαυρομάτας, καὶ 
Πάρθους, καὶ Μήδους, καὶ Σαρακηνούς, καὶ ἅπαν ἁπλῶς 
τὸ τῶν ἀνθρώπων γένος πρὸς τὴν ἀλήθειαν ἐπαναγαγεῖν 
ἐν ἔτεσιν οὐδὲ ὅλοις τριάκοντα. 



Joannes Chrysostomus Scr. Eccl., Commentarius in Job (2062: 505)
“Johannes Chrysostomos. Kommentar zu Hiob”, Ed. Hagedorn, U., Hagedorn, D.
Berlin: De Gruyter, 1990; Patristische Texte und Studien 35.



\end{greek}


\section{Fragmenta Alchemica, Tractatus alchemicus (fragmenta) (date?)}
%when is this from?
%instances of τὸν ἰνδικὸν in here
\blockquote[From Wikipedia\footnote{\url{}}]{?}
\begin{greek}

Fragmenta Alchemica, Tractatus alchemicus (fragmenta) (P. Holm.) (1379: 002)
“Les alchimistes grecs, vol. 1 [Papyrus de Leyde, Papyrus de Stockholm, fragments de recettes]”, Ed. Halleux, R.
Paris: Les Belles Lettres, 1981.

Fragmenta Alchemica, Tractatus alchemicus (fragmenta) (P. Holm.) 
Fragment 60, line 3

                       | 
 Ἄν δέ ποτε τῇ χρήσει τὸ ἀληθινὸν μάργαρον | ἐκπνεῦσαν 
ῥυπωθῇ, ὧδε ἀποσμῶσιν Ἰνδοί· | ἐνιᾶσιν ὡς ἐδωδὴν 
ἀλεκ̣τρυόνι τὴν ψῆφον | ἑσπέρας. 



Fragmenta Alchemica, Tractatus alchemicus (fragmenta) (P. Holm.) 
Fragment 63, line 6

        Καὶ τῇ τὸν {κρυστα} | κρύσταλλον βήρυλλον 
ἕξεις αὐτὸν {ἀρα} | ἀραίωσας τὸν λίθον καὶ μετὰ τὴν 
πύρωσι(ν) |‖ {τὴν πύρωσιν} ἀποψυγέντα εἴς τε τὴν 
προειρη|μένην ῥητίνην ἰνδικῷ τε τῷ φαρμάκῳ συμ|μεμιγ-
μένῳ ἐμβαλών. 



Fragmenta Alchemica, Tractatus alchemicus (fragmenta) (P. Holm.) 
Fragment 64, line 2

                          ‖ 
 Ἐλύδριον μειχθὲν ἰνδικῷ χρώματι γίνε|ται χλωρόν. 



Fragmenta Alchemica, Tractatus alchemicus (fragmenta) (P. Holm.) 
Fragment 84, line 3

                       Χρυσοκόλλης Σʹθʹ, ἰ|οῦ μείξας Σʹαʹ, 
ἐλυδρ<ί>ου Σʹαʹ, ἰνδικοῦ τρι|ώβολον ῥητίνην χρῖε. 



Fragmenta Alchemica, Lexicon alchemicum (= Λεξικὸν κατὰ στοιχεῖον τῆς χρυσοποιίας) (e cod. Venet. Marc. 299, fol. 131r) (1379: 008)
“Collection des anciens alchimistes grecs, vol. 2”, Ed. Berthelot, M., Ruelle, C.É.
Paris: Steinheil, 1888, Repr. 1963.
Volume 2, page 9, line 4

<Ἰός> ἐστι ξάνθωσις, καὶ ὕδωρ θεῖον ἄθικτον, καὶ κώμαρις σκυθικὴ, 
 καὶ ἴσατις ἰνδικὴ, καὶ βατράχιον, καὶ χρυσόπρασον, καὶ χρυσό-
 κολλα. 



Fragmenta Alchemica, Lexicon alchemicum (= Λεξικὸν κατὰ στοιχεῖον τῆς χρυσοποιίας) (e cod. Venet. Marc. 299, fol. 131r) 
Volume 2, page 11, line 6

<Μία φύσις> ἐστὶ θεῖον καὶ ὑδράργυρος διαφόρως οἰκονομηθέντα. 
<Μέλαν>9 ἰνδικὸν ἀπὸ ἰσάτιος γίνεται καὶ χρυσολίθου. 



Fragmenta Alchemica, Περὶ ποιήσεως κινναβάρεως (e cod. Venet. Marc. 299, fol. 106r) (1379: 024)
“Collection des anciens alchimistes grecs, vol. 2”, Ed. Berthelot, M., Ruelle, C.É.
Paris: Steinheil, 1888, Repr. 1963.
Volume 2, page 38, line 12

                               – Δεῖ γινώσκειν ὅτι ἡ μαγνησία 
ἡ ὑελουργικὴ ταύτη ἐστὶν ἡ τῆς Ἀσίας, δι' ἧς ὁ ὕελος τὰς βαφὰς 
δέχεται, καὶ ὁ ἰνδικὸς σίδηρος γίνεται, καὶ τὰ θαυμάσια ξίφη. 



Fragmenta Alchemica, Περὶ βαφῆς σιδήρου (e cod. Venet. Marc. 299, fol. 104r) (1379: 028)
“Collection des anciens alchimistes grecs, vol. 2”, Ed. Berthelot, M., Ruelle, C.É.
Paris: Steinheil, 1888, Repr. 1963.
Volume 2, page 343, line 21

                           – Ἔστι δέ τις καὶ ἄλλη βαφῆς ἰδέα, ἡ οὐ 
μόνον τὸ κοινὸν τῶν σιδήρων ἀποβάπτουσα, στίλβον τε καὶ λαμπρότερον 
ἤπερ ἡ προειρημένη βαφὴ, ἀπεργάζεται, ἀλλά γε καὶ τὸν ὀνομαζόμενον 
ἰνδικὸν παραπλησίως ἢ μικρὸν πλέον στομοῦσα. 



Fragmenta Alchemica, Περὶ βαφῆς σιδήρου (e cod. Venet. Marc. 299, fol. 104r) 
Volume 2, page 345, line 1

                                                                       Αὕτη   
ἐστὶν ἡ μυστικωτάτη βαφὴ, τὸν ἰνδικὸν ἐκβάπτουσα σίδηρον. 



Fragmenta Alchemica, Βαφὴ τοῦ ἰνδικοῦ σιδήρου γραφεῖσα ἀπὸ ἀρχῆς Φιλίππου (e cod. Venet. Marc. 299, fol. 118v) (1379: 030)
“Collection des anciens alchimistes grecs, vol. 2”, Ed. Berthelot, M., Ruelle, C.É.
Paris: Steinheil, 1888, Repr. 1963.
Volume 2, page 347, line 8t

ΒΑΦΗ ΤΟΥ ΙΝΔ*ιΚΟΥ ΣΙΔΗΡΟΥ, ΓΡΑΦΕΙΣΑ 
Τῼ ΑΥΤῼ ΧΡΟΝῼ. 




Fragmenta Alchemica, Βαφὴ τοῦ ἰνδικοῦ σιδήρου γραφεῖσα ἀπὸ ἀρχῆς Φιλίππου (e cod. Venet. Marc. 299, fol. 118v) 
Volume 2, page 348, line 7

                                                           Ηὑρέθη δὲ ὑπὸ τῶν 
Ἰνδῶν, καὶ ἐξεδόθη Πέρσαις, καὶ παρ' ἐκείνων ἦλθεν εἰς ἡμᾶς. 



Fragmenta Alchemica, Καταβαφὴ λίθων καὶ σμαράγδων καὶ λιχνιτῶν καὶ ὑακίνθων (e cod. Paris. B.N. gr. 2327, fol. 147r) (1379: 032)
“Collection des anciens alchimistes grecs, vol. 2”, Ed. Berthelot, M., Ruelle, C.É.
Paris: Steinheil, 1888, Repr. 1963.
Volume 2, page 351, line 24

                                                          Φησὶ γὰρ καὶ ἡ 
<Μαρία>· «Ἐὰν μὲν χλωρὸν θέλῃς, συμμάλασσε τὸν ἰὸν τοῦ χαλκοῦ 
μετὰ χολῆς χελώνης, ἐὰν δὲ κάλλιον βούλῃς, τῆς ἰνδικῆς χελώνης, 
ἐπίβαλε, καὶ ἔσται πάνυ πρωτεῖον· ἐὰν δὲ μὴ εὕρῃς χολὴν χελώνης 
πνεύμονι θαλασσίῳ τῷ κυανέῳ χρῶ, καὶ κάλλιον ποιήσεις· συντελες-
θέντες δὲ, φέγγος βάλλουσιν· ὥστε τὰς μὲν χολὰς τῶν 
ζώων καὶ τὸν ἰὸν τοῦ χαλκοῦ <Ὀστάνης,> ἐπὶ τῶν σμαράγδων ἐξέλαβε, 
μὴ προσθεὶς τὸ θαλάσσιον· ἐπὶ ὑακίνθου δὲ, πόαν ὑάκινθον, καὶ μέλαν   
ἰνδικὸν, καὶ ἰσάτιδος ῥίζαν· ἐπὶ δὲ τοῦ λυχνίτου, τὴν ἄγχουσαν καὶ τὸ 
δρακόντειον αἷμα· ἡ δὲ <Μαρία,> τὸν ἰὸν τοῦ χαλκοῦ καὶ τὰς χολὰς 
τῶν θαλασσίων ζώων· ἐπὶ δὲ τοῦ νυκτοφανοῦς δῆλον <ὅτι> καλοῦσιν 
ὑάκινθον οἱ περὶ λίθων σοφοί. 



Fragmenta Alchemica, Καταβαφὴ λίθων καὶ σμαράγδων καὶ λιχνιτῶν καὶ ὑακίνθων (e cod. Paris. B.N. gr. 2327, fol. 147r) 
Volume 2, page 360, line 22

                                     – Λαβὼν χαλκοῦ κεκαυμένου ἰὸν, καὶ 
ἔλαιον δᾴδινον, καὶ ὀλίγον ἰνδικὸν, καὶ χρυσοκόλλης καὶ ἐλυδρίου μέρη 
γʹ, ἔμβαλε ἐντὸς τοῦ ἄγγους ἔνθα τὸ ἔλαιον, καὶ ἕψει μαλθακῷ πυρὶ 
ἐπὶ ἀνθράκων. 



Fragmenta Alchemica, Καταβαφὴ λίθων καὶ σμαράγδων καὶ λιχνιτῶν καὶ ὑακίνθων (e cod. Paris. B.N. gr. 2327, fol. 147r) 
Volume 2, page 362, line 14

               Εἶτα ὀλίγον τοῦ ἰνδικοῦ ἐπίβαλε, καὶ τὴν χρυσόκολλαν, 
τριπλασίαν τοῦ ἰνδικοῦ. 




Fragmenta Alchemica, Διαφοραὶ μολίβδου καὶ χρυσοπετάλου (e cod. Venet. Marc. 299, fol. 130r) (1379: 042)
“Collection des anciens alchimistes grecs, vol. 2”, Ed. Berthelot, M., Ruelle, C.É.
Paris: Steinheil, 1888, Repr. 1963.
Volume 2, page 38, line 11

    Ἐπὶ χρυσολίθου Νο αʹ πηχῶν ζʹ, μίξεως μύσεως, κασσιτέρου 
παλαιοῦ, ἀρτεμισίας ἰνδικῆς. 


\end{greek}


\section{Anonymi In Aristotelis (date ?)}
\blockquote[From Wikipedia\footnote{\url{}}]{?}
\begin{greek}

Anonymi In Aristotelis Sophisticos Elenchos Phil., Scholia in sophisticos elenchos (= commentarium 2) (4193: 002)
“Commentators and commentaries on Aristotle's sophistici elenchi. A study of post–Aristotelian ancient and medieval writings on fallacies, vol. 2”, Ed. Ebbesen, S.
Leiden: Brill, 1981; Corpus Latinum commentariorum in Aristotelem Graecorum 7.
Bekker pagëline 165a8-9, line of scholion 9

                              ὁ Σωκράτης ἄρα καὶ ὁ Ἰν-
δὸς ποταμὸς οἱ αὐτοί. 



Anonymi In Aristotelis Sophisticos Elenchos Phil., Scholia in sophisticos elenchos (= commentarium 2) 
Bekker pagëline 165a8-9, line of scholion 18

            ὁ Πλάτων ἄρα καὶ ὁ Ἰνδὸς ποταμὸς οἱ αὐτοί. 



Anonymi In Aristotelis Sophisticos Elenchos Phil., In Aristotelis sophisticos elenchos paraphrasis (4193: 012)
“Anonymi in Aristotelis sophisticos elenchos paraphrasis”, Ed. Hayduck, M.
Berlin: Reimer, 1884; Commentaria in Aristotelem Graeca 23.4.
Section 24, line 12

                                       καὶ πάλιν ὁ Ἰνδὸς μέλας ὢν ἁπλῶς λευκός 
ἐστιν τοὺς ὀδόντας, ὥστε λευκὸς καὶ οὐ λευκός. 



Anonymi In Aristotelis Sophisticos Elenchos Phil., In Aristotelis sophisticos elenchos paraphrasis 
Section 79, line 15

                                                                               ἐν οἷς 
τε μὴ ὀνόμασι σημαίνεται τὸ καθόλου, ἀλλὰ τῇ κατὰ μέρος ὁμοιότητι λόγῳ 
τὸ κοινὸν θηρᾶται, χρηστέον πρὸς τὸ συμφέρον· οἷον ἐπεὶ τὸ κερασφόρον 
μὴ ὄνομα ὂν ἀλλὰ λόγος τῷ εἶναι σύνθετον, καὶ ἐξ ὁμοιότητος ἐπὶ τὰ μὴ 
τῶν ζώων ἀμφόδοντα εἴληπται, οἷον αἶγα, πρόβατον, βοῦν καὶ τὰ λοιπά, 
χρηστέον ταῖς ἐπαγωγαῖς πρὸς τὸ συμφέρον, ὡς ἐπεὶ τὸ καὶ τὸ κερασφόρα 
ὄντα οὐκ ἀμφόδοντα, καὶ πᾶν ἄρα· καὶ ἄλλου μὲν ἐπάγοντος αὐτὸν φέρειν 
ἔνστασιν τὸν ὄνον τὸν Ἰνδικὸν ἀμφόδοντα ὄντα καὶ κερασφόρον, αὐτὸν δὲ 
χρώμενον προσομολογεῖν οὕτως ἐπὶ πάντων. 

\end{greek}


\chapter{Byzantine}%byz
\minitoc


\section{Georgius Acropolites}
\blockquote[From Wikipedia\footnote{\url{http://en.wikipedia.org/wiki/George_Akropolites}}]{George Akropolites, Latinized as Acropolites or Acropolita (Greek: Γεῶργιος Ἀκροπολίτης, Georgios Akropolitês, 1217 or 1220 – 1282), was a Byzantine Greek historian and statesman born at Constantinople.}
\begin{greek}

Georgius Acropolites Hist., Epitaphius in Joannem Ducam (3141: 009)
“Georgii Acropolitae opera, vol. 2”, Ed. Heisenberg, A.
Leipzig: Teubner, 1903, Repr. 1978 (1st edn. corr. P. Wirth).
Section 13, line 16

                   ἀλλὰ τὰ μὲν τῆς ἀνδρείας ἐκείνου γνω-
ρίσματα τοῦ στερροῦ καὶ καρτερικοῦ καὶ βεβηκότος ὡς τὸ 
τετράγωνον, τοῦ ὀξυτέρου πνεύματος ἢ πυρὸς ὁπότε καιρὸς 
καλοῖ καὶ ἀκμὴ κατεπείγει καιροῦ, κατὰ ταὐτὰ δὲ καὶ τὰ 
τῆς φρονήσεως σύμβολα, ἣ πασῶν τῶν ἀρετῶν ὑπερτέθειται 
ἢ μᾶλλον κρεῖττον εἰπεῖν ἣ τὰς ἀρετὰς εἰδοποίησε καὶ ἀρε-
τὰς εἶναι καὶ ὀνομάζεσθαι πέπεικε, τούτων δὴ πάντων ὡς 
ἐνὸν ἡμῖν καὶ ὡς ὁ καιρὸς δίδωσιν οἱ χαρακτῆρες καὶ τὰ 
ἰνδάλματα προκεκέντηνται· ἐπεὶ δὲ καὶ καλοκαγαθίας καὶ 
ἡμερότητος ἐμνήσθημεν, ἐνταῦθά μοι τοῦ λόγου τὰ 
δάκρυα προπηδᾷ ἀναλογιζομένῳ τὴν τούτου περὶ τοὺς 
ἐπταικότας συμπάθειαν, οἶμαι δὲ καὶ πάντες ἐν τούτοις 
πλέον θρηνήσετε. 



Georgius Acropolites Hist., In Gregorii Nazianzeni sententias (3141: 012)
“Georgii Acropolitae opera, vol. 2”, Ed. Heisenberg, A.
Leipzig: Teubner, 1903, Repr. 1978 (1st edn. corr. P. Wirth).
Section 8, line 3

                                                             πᾶσι 
δὲ λέγω Χαλδαίοις, Αἰγυπτίοις, Ἰνδοῖς, αὐτοῖς Ἕλλησιν, ἁπα-
ξαπλῶς τοῖς ὁπωσοῦν ἁψαμένοις θεολογίας καὶ βεβουλημένοις 
διαλεχθῆναι περὶ θεοῦ τοῦτο γοῦν τὸ ὂν καὶ δημιουργικὸν καθω-
μολόγηται αἴτιον. 



Georgius Acropolites Hist., In Gregorii Nazianzeni sententias 
Section 9, line 8

                                             καὶ τοῦτο εἰκότως· 
τὰ γὰρ ἐκ διαμέτρου πρὸς ἄλληλα κείμενα ἰνδάλματά τινα 
προβάλλουσιν ὁμοιότητος. 



Georgius Acropolites Hist., Laudatio Petri et Pauli (3141: 013)
“Georgii Acropolitae opera, vol. 2”, Ed. Heisenberg, A.
Leipzig: Teubner, 1903, Repr. 1978 (1st edn. corr. P. Wirth).
Section 17, line 54

                                                     καὶ ἵνα συντόμως 
ἐρῶ· Πέτρον μὲν εὑρίσκω τῆς πρακτικῆς εἰκόνας ἀκριβεστάτας 
ἐμφαίνοντα, Παῦλον δὲ θεωρίας ὡς ἐναργῆ τὰ ἰνδάλματα. 

\end{greek}


\section{Testamenta XII Patriarcharum}
\blockquote[From Wikipedia\footnote{\url{}}]{}
\begin{greek}

Testamenta XII Patriarcharum, Testamenta xii patriarcharum (1700: 001)
“Testamenta xii patriarcharum, 2nd edn.”, Ed. de Jonge, M.
Leiden: Brill, 1970; Pseudepigrapha veteris testamenti Graece 1.
Testamentum 11, chapter 11, section 2, line 2

   Ἐλθὼν δὲ εἰς 
Ἰνδοκολπίτας μετὰ τῶν Ἰσμαηλιτῶν, ἠρώτουν με· κἀγὼ 
εἶπον ὅτι δοῦλος αὐτῶν εἰμι ἐξ οἴκου, ἵνα μὴ αἰσχύνω τοὺς 
ἀδελφούς μου. 

\end{greek}


\section{Geoponica}
\blockquote[From Wikipedia\footnote{\url{http://en.wikipedia.org/wiki/Geoponica}}]{The Geoponica (also spelled Geoponika) is a twenty-book collection of agricultural lore, compiled during the 10th century in Constantinople for the Byzantine emperor Constantine VII Porphyrogenitus. The Greek word Geoponica signifies "agricultural pursuits" in its widest sense.

The 10th century collection is sometimes (wrongly) ascribed to the 7th century author Cassianus Bassus, whose collection, also titled Geoponica, was integrated into the extant work. Bassus drew heavily on the work on another agricultural compiler, Vindonius Anatolius (4th century). The ultimate sources of the Geoponica include Pliny, various lost Hellenistic and Roman-period Greek agriculture and veterinary authors, the Carthaginian agronomist Mago, and even works passing under the name of the Persian prophet Zoroaster. (The names of the principal sources for each section are attached to the text, although the age and correctness of these attributions remains in doubt.) The Greek manuscript tradition is extremely complex and not fully understood. Syriac, Pahlavi, Arabic and Armenian translations attest to its worldwide popularity and complicate the manuscript tradition still further.
Contents

The Geoponica embraces all manner of "agricultural" information, including celestial and terrestrial omina, viticulture, oleoculture, apiculture, veterinary medicine, the construction of fishponds and much more.}
\begin{greek}

Geoponica, Geoponica (4080: 001)
“Geoponica”, Ed. Beckh, H.
Leipzig: Teubner, 1895.
Book 2, chapter 6, section 23, line 5

   σχοίνους 
τε γὰρ φυομένας <ἅς> τινες ὁλοσχοίνους καλοῦσι, καὶ 
βούτομον, καὶ βάτους, καὶ κύπειρον, ἥν τινες ζέρναν 
καλοῦσιν· ἔτι δὲ καὶ ἄγρωστιν πολλὴν καὶ εὔτροφον, 
καλάμους τε τοὺς καλουμένους Ἰνδικούς, ὑπό τινων   
δὲ μεστοκαλάμους, ὑπ' ἐνίων βαλίτας, καὶ σύριγγας 
δασεῖς καὶ ἁπαλοὺς σημαίνειν φασίν, ὡς ὕδωρ ἔχοντος 
τοῦ τόπου. 



Geoponica, Geoponica 
Book 6, chapter 8, section 1, line 2

                        βʹ νάρδου Ἰνδικῆς ἢ Κελτικῆς 
λι. 



Geoponica, Geoponica 
Book 8, chapter 22, section 2, line 3

                                           θʹ νάρδου 
Ἰνδικῆς δραχ. 



Geoponica, Geoponica 
Book 16, chapter 22, section 3, line 3

   τὴν 
δὲ Βακτριανὴν κάμηλον, ἐν τοῖς ὄρεσι τοῖς πρὸς τῇ 
Ἰνδικῇ, ὑπὸ συάγρων τῶν συννεμομένων συλλαμβά-
νειν, ὁ αὐτὸς Δίδυμός φησιν. 



Geoponica, Geoponica 
Book 16, chapter 22, section 9, line 1

   Ἐγὼ δὲ ἀπὸ τῆς Ἰνδίας ἐνεχθεῖ-
σαν ἐθεασάμην ἐν Ἀντιοχείᾳ καμηλοπάρδαλιν. 

\end{greek}

\section{Constantine VII}
\blockquote[From Wikipedia\footnote{\url{http://en.wikipedia.org/wiki/Constantine_VII}}]{Constantine VII Porphyrogennetos or Porphyrogenitus, "the Purple-born" (that is, born in the imperial bed chambers) (Greek: Κωνσταντῖνος Ζ΄ Πορφυρογέννητος, Kōnstantinos VII Porphyrogennētos; September 2, 905 – November 9, 959), was the fourth Emperor of the Macedonian dynasty of the Byzantine Empire, reigning from 913 to 959. He was the son of the emperor Leo VI and his fourth wife, Zoe Karbonopsina, and the nephew of his predecessor, the emperor Alexander.

Most of his reign was dominated by co-regents: from 913 until 919 he was under the regency of his mother, while from 920 until 945 he shared the throne with Romanos Lekapenos, whose daughter Helena he married, and his sons. Constantine VII is best known for his four books, De Administrando Imperio (bearing in Greek the heading Προς τον ίδιον υιόν Ρωμανόν), De Ceremoniis (Περί τῆς Βασιλείου Τάξεως), De Thematibus (Περί θεμάτων Άνατολῆς καί Δύσεως), and Vita Basilii (Βίος Βασιλείου).}
\begin{greek}

Constantinus VII Porphyrogenitus Imperator Hist., De legationibus (3023: 001)
“Excerpta historica iussu imp. Constantini Porphyrogeniti confecta, vol. 1: excerpta de legationibus, pts. 1–2”, Ed. de Boor, C.
Berlin: Weidmann, 1903.
Page 91, line 28

3. Ὅτι Ἰουστινιανὸς ὁ βασιλεύς, ἐν μὲν Αἰθίοψι βασιλεύ-
οντος Ἑλλησθεαίου, Ἐσιμιφαίου δὲ ἐν Ὁμηρίταις, πρεσβευτὴν 
Ἰουλιανὸν <ἔπεμψεν> ἀξιῶν ἄμφω Ῥωμαίοις διὰ τὸ τῆς γνώμης 
ὁμόγνωμον Πέρσαις πολεμοῦσι ξυνάρασθαι, ὅπως Αἰθίοπες μὲν 
ὠνούμενοί τε τὴν μέταξαν ἐξ Ἰνδῶν, ἀποδιδόμενοι δὲ αὐτὴν ἐς 
Ῥωμαίους, αὐτοὶ μὲν κύριοι γένωνται χρημάτων μεγάλων, Ῥωμαίους 
δὲ τοῦτο ποιήσωσι κερδαίνειν μόνον, ὅτι δὴ οὐκ ἔτι ἀναγκασθή-
σονται τὰ σφέτερα αὐτῶν χρήματα ἐς τοὺς πολεμίους μετενεγκεῖν   
(αὕτη δέ ἐστιν ἡ μέταξα, ἐξ ἧς εἰώθασι τὴν ἐσθῆτα ἐργάζεσθαι, 
ἣν πάλαι μὲν Ἕλληνες Μηδικὴν ἐκάλουν, τὰ νῦν δὲ σηρικὴν ὀνομά-
ζουσιν), Ὁμηρῖται δὲ ὅπως Κάϊσόν τε τὸν φυγάδα φύλαρχον Μααλ-
δηνοῖς καταστήσωνται καὶ στρατῷ μεγάλῳ αὐτῶν τε Ὁμηριτῶν 
καὶ Σαρακηνῶν τῶν Μααλδηνῶν ἐσβάλωσιν ἐς τὴν Περσῶν γῆν. 



Constantinus VII Porphyrogenitus Imperator Hist., De legationibus 
Page 92, line 11

τοῖς τε γὰρ Αἰθίοψι τὴν μέταξαν ὠνεῖσθαι πρὸς τῶν Ἰνδῶν 
ἀδύνατα ἦν, ἐπειδὴ ἀεὶ οἱ Περσῶν ἔμποροι πρὸς αὐτοῖς <τοῖς 
ὅρμοις> γενόμενοι, οὗ δὴ τὰ πρῶτα αἱ τῶν Ἰνδῶν νῆες καταίρουσιν, 
ἅτε χώραν προσοικοῦντες τὴν ὅμορον, ἅπαντα ὠνεῖσθαι τὰ φορτία 
εἰώθασιν· καὶ τοῖς Ὁμηρίταις χαλεπὸν ἔδοξεν εἶναι χώραν ἀμει-
ψαμένοις ἔρημόν τε καὶ χρόνου πολλοῦ ὁδὸν κατατείνουσαν ἐπ' 
ἀνθρώπους πολλῷ μαχιμωτέρους ἰέναι. 



Constantinus VII Porphyrogenitus Imperator Hist., De legationibus 
Page 123, line 30

                                  ἡμῶν δὲ ἐς ἕτερα τρεψάντων τὸν 
λόγον καὶ φιλοφροσύνῃ τὸν σφῶν αὐτῶν καταπραϋνάντων θυμόν, 
μετὰ τὸ δεῖπνον ὡς διανέστημεν, δώροις ὁ Μαξιμῖνος Ἐδέκωνα 
καὶ Ὀρέστην ἐθεράπευσε σηρικοῖς ἐσθήμασι καὶ λίθοις Ἰνδικοῖς. 



Constantinus VII Porphyrogenitus Imperator Hist., De legationibus 
Page 132, line 12

      ἐπιμεληθέντες δὲ καὶ τῶν ἵππων καὶ τῶν λοιπῶν ὑποζυγίων 
παρὰ τὴν βασιλίδα ἀφικόμεθα, καὶ αὐτὴν ἀσπασάμενοι καὶ δώ-
ροις ἀμειψάμενοι, τρισί τε ἀργυραῖς φιάλαις καὶ ἐρυθροῖς δέρ-
μασι καὶ τῷ ἐξ Ἰνδίας πεπέρει καὶ τῷ καρπῷ τῶν φοινίκων καὶ 
ἑτέροις τραγήμασι διὰ τὸ μὴ ἐπιχωριάζειν τοῖς βαρβάροις οὖσι 
τιμίοις, ὑπέξιμεν εὐξάμενοι αὐτῇ ἀγαθὰ τῆς ξενίας πέρι. 



Constantinus VII Porphyrogenitus Imperator Hist., De legationibus 
Page 428, line 10

49. Ὅτι πλεῖσται ὅσαι πρεσβεῖαι πρὸς Τραιανὸν παρὰ βαρ-
βάρων ἄλλων τε καὶ Ἰνδῶν ἀφίκοντο. 



Constantinus VII Porphyrogenitus Imperator Hist., De legationibus 
Page 478, line 9

                                              ὁπηνίκα δ' εἶδε τὸ 
Ἰνδικὸν ζῶον ὁ Χαγάνος ἐλέφαντα, παραυτίκα καταλύει τὸ θέα-
τρον καὶ παλιννοστεῖν προστάττει τὸ θηρίον παρὰ τὸν Καίσαρα, 
ἢ καταπλαγεὶς ἢ ἀποφαυλίσας τὸ θαυμαζόμενον, οὐκ ἔχω εἰπεῖν· 
οὐ γὰρ ἂν ἐκρυψάμην. 



Constantinus VII Porphyrogenitus Imperator Hist., De legationibus 
Page 488, line 21

                                 τετάρτη δὲ ἡμέρα, καὶ τῆς Ῥω-
μαϊκῆς δυνάμεως εὐσθενούσης τοῖς πρὸς ζωὴν ἀναγκαιοτάτοις τῷ 
βίῳ, ὁ Χαγάνος πρέσβεις ἐξέπεμψεν Ἰνδικὰς ὑπὸ τοῦ Πρίσκου 
λαβεῖν ἀξιῶν καρυκείας. 



Constantinus VII Porphyrogenitus Imperator Hist., De legationibus 
Page 488, line 23

                            ὁ μὲν οὖν Πρίσκος τοῦ βαρβάρου τὴν 
ἀξίωσιν ἐθεράπευε πέπερί τε ἐξέπεμψε καὶ φύλλον Ἰνδικὸν κας-
σίαν τε καὶ τὸ λεγόμενον κόστον. 



Constantinus VII Porphyrogenitus Imperator Hist., De legationibus 
Page 514, line 24

                                                    αὐτῷ δὲ τὰ Ἰνδῶν 
ἔφη ἐν τῷ τότε μέλειν. 



Constantinus VII Porphyrogenitus Imperator Hist., De legationibus 
Page 514, line 32

ἧκον δὲ καὶ παρὰ τῶν αὐτονόμων τῶν Ἰνδῶν πρέσβεις παρὰ Ἀλέ-
ξανδρον καὶ παρὰ Πώρου ἄλλου του ὑπάρχου Ἰνδῶν. 



Constantinus VII Porphyrogenitus Imperator Hist., De legationibus 
Page 515, line 7

6. Ὅτι παρεγένοντο παρὰ Ἀλέξανδρον τῶν Μαλλῶν τῶν ὑπο-
λειπομένων πρέσβεις ἐνδιδόντες τὸ ἔθνος, καὶ παρὰ Ὀξυδρακῶν 
οἵ τε ἡγεμόνες τῶν πόλεων καὶ οἱ νομάρχαι αὐτοὶ καὶ ἄλλοι ἅμα 
τούτοις ἑκατὸν καὶ πεντήκοντα οἱ γνωριμώτατοι αὐτοκράτορες περὶ 
σπονδῶν δῶρά τε ὅσα μέγιστα παρ' Ἰνδοῖς κομίζοντες καὶ τὸ 
ἔθνος καὶ οὗτοι ἐνδιδόντες. 



Constantinus VII Porphyrogenitus Imperator Hist., De legationibus 
Page 515, line 11

                                   συγγνωστὰ δὲ ἁμαρτεῖν ἔφασαν οὐ 
πάλαι παρ' αὐτὸν πρεσβευσάμενοι· ἐπιθυμεῖν γάρ, ὥσπερ τινὲς 
ἄλλοι, ἔτι μᾶλλον αὐτοὶ ἐλευθερίας τε καὶ αὐτόνομοι εἶναι, ἥντινα 
ἐλευθερίαν ἐξ ὅτου Δι<όνυσος ἐς Ἰνδοὺς ἧκε σώαν σφίσιν εἶναι 
ἐς Ἀλέξανδρον· εἰ δὲ Ἀλεξάνδρῳ δοκοῦν ἐστιν>, ὅτι καὶ Ἀλέξαν-
δρον ἀπὸ θεοῦ γενέσθαι λόγος κατέχει, ξατράπην τε ἀναδέξασθαι 
ὅντινα τάττοι Ἀλέξανδρος καὶ φόρους ἀποίσειν τοὺς Ἀλεξάνδρῳ 
δόξαντας· διδόναι δὲ καὶ ὁμήρους ἐθέλειν ὅσους ἂν αἰτῇ Ἀλέ-
ξανδρος. 



Constantinus VII Porphyrogenitus Imperator Hist., De legationibus 
Page 515, line 18

         ὁ δὲ χιλίους ᾔτησε τοὺς κρατιστεύοντας τοῦ ἔθνους, οὕς, 
εἰ μὲν βούλοιντο, ἀντὶ ὁμήρων καθέξειν, εἰ δὲ μή, ξυστρατεύοντας 
ἕξειν, ἔστ' ἂν διαπολεμηθῇ αὐτῷ πρὸς τοὺς ἄλλους Ἰνδούς. 



Constantinus VII Porphyrogenitus Imperator Hist., De legationibus 
Page 515, line 24

7. Ὅτι καὶ παρὰ Ὀσσαδίων γένους αὐτονόμου Ἰνδικοῦ πρέ-
σβεις ἧκον ἐνδιδόντες καὶ οὗτοι τοὺς Ὀσσαδίους. 



Constantinus VII Porphyrogenitus Imperator Hist., De legationibus 
Page 516, line 8

9. Ὅτι παρελθόντι Ἀλεξάνδρῳ ἐς Βαβυλῶνα πρεσβεῖαι παρὰ 
τῶν Ἑλλήνων ἐνέτυχον, ὑπὲρ ὅτων μὲν ἕκαστοι οὐκ ἀναγέγραπται· 
δοκεῖν δ' ἔμοιγε αἱ πολλαὶ στεφανούντων τε αὐτὸν ἦσαν καὶ ἐπαι-
νούντων ἐπὶ ταῖς νίκαις ταῖς τε ἄλλαις καὶ μάλιστα ταῖς Ἰνδικαῖς, 
καὶ ὅτι σῶος ἐξ Ἰνδῶν ἐπανήκει χαίρειν φασκόντων. 



Constantinus VII Porphyrogenitus Imperator Hist., De virtutibus et vitiis (3023: 002)
“Excerpta historica iussu imp. Constantini Porphyrogeniti confecta, vol. 2: excerpta de virtutibus et vitiis, pts. 1 \& 2”, Ed. Büttner–Wobst, T., Roos, A.G.
Berlin: Weidmann, 2.1:1906; 2.2:1910.
Volume 1, page 147, line 23

Ἐπὶ τούτου γὰρ τοῦ μακαρίου καὶ οἱ ἐνδότεροι Ἰνδοὶ καὶ 
Ἴβηρες προσῆλθον τῷ ἁγίῳ βαπτίσματι, καὶ οἱ Ἀρμένιοι τελείως 
ἐπίστευσαν μετὰ καὶ τοῦ βασιλέως αὐτῶν Τιριδάτου διὰ τοῦ πο-
λυάθλου μάρτυρος καὶ μεγάλου Γρηγορίου ἀρχιεπισκόπου αὐτῶν. 



Constantinus VII Porphyrogenitus Imperator Hist., De virtutibus et vitiis 
Volume 1, page 246, line 30

       (17, 108, 4). Ὅτι Ἅρπαλος ὁ τῶν ἐν Βαβυλῶνι θησαυρῶν 
καὶ τῶν προσόδων τὴν φυλακὴν πεπιστευμένος, ἐπειδὴ ὁ βασιλεὺς 
εἰς τὴν Ἰνδικὴν ἐστράτευσεν, ἀπέγνω τὴν ἐπάνοδον αὐτοῦ· δοὺς 
δ' ἑαυτὸν εἰς τρυφὴν καὶ πολλῆς χώρας ἀποδεδειγμένος σατρά-
πης τὸ μὲν πρῶτον εἰς ὕβρεις γυναικῶν καὶ παρανόμους ἔρωτας 
ἐξετράπη καὶ πολλὰ τῆς γάζης ἀκρατεστάταις ἡδοναῖς κατηνάλω-
σεν· ἀπὸ δὲ τῆς Ἐρυθρᾶς θαλάσσης πολὺ διάστημα κομίζων 
ἰχθύων πλῆθος καὶ δίαιταν πολυδάπανον ἐνιστάμενος ἐβλασφη-  
μεῖτο. 



Constantinus VII Porphyrogenitus Imperator Hist., De virtutibus et vitiis 
Volume 1, page 300, line 26

       (33, 18). Ὅτι ὁ Ἀρσάκης ὁ βασιλεὺς ἐπιείκειαν καὶ φι-
λανθρωπίαν ζηλώσας αὐτομάτην ἔσχε τὴν ἐπίρροιαν τῶν ἀγαθῶν 
καὶ τὴν βασιλείαν ἐπὶ πλεῖον ηὔξησε· μέχρι <γὰρ> τῆς Ἰνδικῆς 
διατείνας τῆς ὑπὸ τὸν Πῶρον γενομένης χώρας ἐκυρίευσεν ἀκιν-
δύνως. 



Constantinus VII Porphyrogenitus Imperator Hist., De virtutibus et vitiis 
Volume 2, page 253, line 14

                                                      καὶ αὐτῷ διὰ 
τοῦτο πολλοὶ μὲν δῆμοι πολλοὶ δὲ καὶ δυνάσται, ἄλλοι τε καὶ 
Ἰνδίβολις καὶ Μανδόνιος Ἰαγερτανοί, προσεχώρησαν. 



Constantinus VII Porphyrogenitus Imperator Hist., De insidiis (3023: 003)
“Excerpta historica iussu imp. Constantini Porphyrogeniti confecta, vol. 3: excerpta de insidiis”, Ed. de Boor, C.
Berlin: Weidmann, 1905.
Page 3, line 24

2. Ὅτι μετὰ τὸν Ἰνδικὸν πόλεμον Σεμίραμις, ἐπεὶ ὁδοιποροῦσα 
ἐγένετο ἐν Μήδοις, ἀναβᾶσα ἐπί τι ὑψηλὸν ὄρος πάντοθεν πλὴν 
καθ' ἓν μέρος περιερρωγὸς καὶ ἄβατον λισσάδι καὶ ἀποτόμῳ πέτρᾳ, 
ἐθεᾶτο τὴν στρατιὰν ἀπό τινος ἐξέδρας, ἣν παραχρῆμα ᾠκοδομή-
σατο. 



Constantinus VII Porphyrogenitus Imperator Hist., De insidiis 
Page 47, line 12

Ὁ δὲ νεκρὸς ἔτι ἔκειτο ἔνθα ἔπεσεν ἀτίμως πεφυρμένος αἵ-
ματι, ἀνδρὸς ἐλάσαντος μὲν πρὸς ἑσπέραν ἄχρι Βρεττανῶν τε καὶ 
Ὠκεανοῦ, διανοουμένου δ' ἐλαύνειν πρὸς ἕω ἐπὶ τὰ Πάρθων ἀρ-
χεῖα καὶ Ἰνδῶν, ὡς ἄν, κἀκείνων ὑπηκόων γενομένων, εἰς μίαν 
ἀρχὴν κεφαλαιωθείη γῆς πάσης καὶ θαλάττης τὰ κράτη· τότε δ' 
οὖν ἔκειτο, μηδενὸς τολμῶντος ὑπομένειν καὶ τὸν νεκρὸν ἀναι-
ρεῖσθαι. 



Constantinus VII Porphyrogenitus Imperator Hist., De insidiis 
Page 130, line 12

                                   ὁ δὲ τοῦ βασιλέως γαμβρὸς 
Ζήνων τὴν ὕπατον ἔχων ἀρχὴν ἔστελλε τοὺς τὸν Ἰνδακὸν ἀπο-
στήσοντας ἀπὸ τοῦ λεγομένου Παπιρίου λόφου. 



Constantinus VII Porphyrogenitus Imperator Hist., De insidiis 
Page 130, line 15

                                                    τοῦτον γὰρ 
πρῶτος Νέων ἐφώλευε· μεθ' ὃν Παπίριος καὶ ὁ τοῦδε παῖς 
Ἰνδακός, τοὺς προσοίκους ἅπαντας βιαζόμενοι καὶ τοὺς διοδεύον-
τας ἀναιροῦντες. 



Constantinus VII Porphyrogenitus Imperator Hist., De insidiis 
Page 138, line 4

        ὁ δὲ Ἰλλοῦς τὴν τοῦ φρουρίου φυλακὴν ἐπιτρέψας Ἰνδακῷ 
Κοττούνῃ τὸ λοιπὸν ἐσχόλαζεν ἀναγνώσει βιβλίων, καὶ ὁ Λεόντιος 
ἐν νηστείᾳ τε καὶ θρήνοις διετέλει. 



Constantinus VII Porphyrogenitus Imperator Hist., De insidiis 
Page 139, line 10

Ἐπράχθη δὲ καὶ ἡ τοῦ φρουρίου Χέρρις κατάληψις τρόπῳ 
τοιῷδε. Ἰνδακὸς ὁ Κοττούνης πάλαι τὴν προδοσίαν μελετῶν, ἅμα 
δὲ καὶ τὴν φυλακὴν τοῦ ἐρύματος ἐπιτετραμμένος, πείθει τὸν 
Ἰλλοῦν ἔξω τοῦ φρουρίου τοὺς ἀμφ' αὐτὸν παρασκευάσαι, ὡς δὴ 
τῶν ἐναντίων διὰ τῆς νυκτὸς ἐπιόντων, αὐτόν τε ἅμα Λεοντίῳ ἐν 
τῷ συνήθει κατευνασθῆναι κοιτῶνι. 



Constantinus VII Porphyrogenitus Imperator Hist., De insidiis 
Page 139, line 18

καὶ πρῶτα μὲν οἱ τῶν πυλῶν φύλακες ἀποσφάττονται, ἔπειτα βοῆς 
ἀκουσθείσης, ὡς ἔθος ἐστὶ Ῥωμαίοις λέγειν· Ζήνων Αὔγουστε 
τούμβικας, παραχρῆμα μὲν Ἰνδακὸς καὶ οἱ σὺν αὐτῷ προδόντες 
ἀναιροῦνται, Ἰλλοῦς δὲ καὶ Λεόντιος εἰς τὸ τέμενος τοῦ μάρτυρος 
Κόνωνος καταφεύγουσιν. 



Constantinus VII Porphyrogenitus Imperator Hist., De insidiis 
Page 158, line 17

                 καὶ δεξάμενος τὸ ἕτοιμον τῆς ἀπολογίας αὐτῶν 
συνεχώρησε τὸ πταῖσμα <καὶ> ἰνδουλγεντίας αὐτοῖς παρασχὼν 
ἐδέξατο αὐτούς, καὶ ἔκτισεν αὐτοῖς δημόσιον λουτρὸν τὸ Σεβήριον 
καὶ ἱερά. 



Constantinus VII Porphyrogenitus Imperator Hist., De insidiis 
Page 168, line 8

                    καὶ πάλιν συνεκροτήθη πόλεμος μεταξὺ Ἰσαύρων, 
καὶ εἰσηνέχθησαν αἰχμάλωτοι Σιλούντιος καὶ Ἴνδης καὶ ὁ ἀδελφὸς 
αὐτοῦ ἀπὸ τῆς Ἰσαυρίας, καὶ Λογγῖνος ὁ ἀδελφὸς Ζήνωνος τοῦ 
βασιλέως ἐξωρίσθη. 



Constantinus VII Porphyrogenitus Imperator Hist., De insidiis 
Page 174, line 4

                                               οἱ δὲ τὴν αὐτὴν σκέψιν τῆς 
ἐπιβουλῆς μελετήσαντες ἦσαν οὗτοι· Ἀβλάβιος ὁ κατὰ Μελτιάδην 
ὁ μελιστής, καὶ Μάρκελλος ὁ ἀργυροπράτης ὁ τῶν Κιλίκων ὁ ἔχων 
τὸ ἐργαστήριον πλησίον τῆς ἁγίας Εἰρήνης τῆς ἀρχαίας καὶ νέας   
ὁ κατὰ Αἰθέριον τὸν κουράτορα, καὶ Σέργιος ὁ ἀνεψιὸς τοῦ αὐ-
τοῦ Αἰθερίου, ἵνα καθημένου τοῦ βασιλέως ἐν τῷ παλατίῳ ὀψίας 
πρὸ μινσῶν σφάξωσιν αὐτὸν στήσαντες καὶ ἀνθρώπους ἰδίους εἴς 
τε τὸ Ἅρμα καὶ εἰς τὸ Σελεντιαρίκιν καὶ κατὰ τοὺς Ἰνδοὺς καὶ κατὰ 
τὸν Ἀρχάγγελον, ἵνα, γινομένης τῆς αὐτῆς ἐπιβουλῆς, ταραχὴν 
ποιήσωσιν. 



Constantinus VII Porphyrogenitus Imperator Hist., De insidiis 
Page 194, line 1

10. Ὅτι οἱ Σκύθαι τὸν ὅμορον χῶρον οἰκοῦντες τῆς Ἰνδικῆς ἐξ 
ἀρχῆς μὲν ὀλίγην ἐνέμοντο χώραν, ὕστερον δὲ κατ' ὀλίγον αὐξη-
θέντες διὰ τὰς ἀλκὰς καὶ τὴν ἀνδρείαν πολλὴν μὲν κατεκτήσαντο 
χώραν, τὸ δὲ ἔθνος εἰς μεγάλην ἡγεμονίαν καὶ δόξαν προήγαγον. 



Constantinus VII Porphyrogenitus Imperator Hist., De sententiis (3023: 004)
“Excerpta historica iussu imp. Constantini Porphyrogeniti confecta, vol. 4: excerpta de sententiis”, Ed. Boissevain, U.P.
Berlin: Weidmann, 1906.
Page 62, line 23

                                                    τὸν δὲ Καύκασον 
τὸ ὄρος ἐκ τοῦ Πόντου ἐς τὰ πρὸς ἕω μέρη τῆς γῆς καὶ τὴν 
Παραπαμισαδῶν χώραν ὡς ἐπὶ Ἰνδοὺς μετάγειν τῷ λόγῳ τοὺς 
Μακεδόνας, Παραπάμισον ὄντα τὸ ὄρος αὐτοὺς καλοῦντας Καύ-
κασον τῆς Ἀλεξάνδρου ἕνεκα δόξης, ὡς ὑπὲρ τὸν Καύκασον ἄρα 
ἐλθόντα Ἀλέξανδρον. 



Constantinus VII Porphyrogenitus Imperator Hist., De sententiis 
Page 62, line 26

                         ἔν τε αὐτῇ τῇ Ἰνδῶν γῇ βοὺς ἰδόντας ἐγκε-
καυμένας ῥόπαλον τεκμηριοῦσθαι ἐπὶ τῷδε ὅτι Ἡρακλῆς ἐς Ἰνδοὺς 
ἀφίκετο. 



Constantinus VII Porphyrogenitus Imperator Hist., De sententiis 
Page 63, line 11

                                                             καὶ Ἀλέξανδρος 
τούτῳ ἔτι μᾶλλον τῷ λόγῳ ἡσθεὶς τήν τε ἀρχὴν τῷ Πώρῳ τῶν 
τε αὐτῶν Ἰνδῶν ἔδωκε καὶ ἄλλην ἔτι χώραν πρὸς τῇ πάλαι οὔσῃ 
πλείονα τῆς πρόσθεν προσέθηκεν. 



Constantinus VII Porphyrogenitus Imperator Hist., De sententiis 
Page 64, line 8

18. Ὅτι φησὶν ὁ Ἀρριανός· ἐπὶ τῷδε ἐπαινῶ τοὺς σοφιστὰς 
τῶν Ἰνδῶν, ὧν λέγουσί τινας συλληφθέντας ὑπ' Ἀλεξάνδρου ὑπαι-
θρίους ἐν λειμῶνι, ἵναπερ αὐτοῖς διατριβαὶ ἦσαν, ἄλλο μὲν οὐδὲν 
ποιῆσαι πρὸς τὴν ὄψιν αὐτοῦ τε καὶ τῆς στρατιᾶς, κρούειν δὲ 
τοῖς ποσὶ τὴν γῆν ἐφ' ἧς βεβηκότες ἦσαν. 



Constantinus VII Porphyrogenitus Imperator Hist., De sententiis 
Page 356, line 6

295. Ὅτι ὁ Ἰνδιβέλης ὁ Κελτίβηρ συγγνώμης τυχὼν παρὰ 
Σκιπίωνος καιρὸν εὑρὼν ἐπιτήδειον πάλιν ἐξέκαυσε πόλεμον. 



Constantinus VII Porphyrogenitus Imperator Hist., De strategematibus (olim sub auctore Herone Byzantio) (3023: 005)
“Griechische Poliorketiker”, Ed. Schneider, R.
Berlin: Weidmann, 1909, Repr. 1970; Abhandlungen der königlichen Gesellschaft der Wissenschaften zu Göttingen, Philol.–hist. Kl., N.F. 11, no. 1.
Wescher page 203, line 15

                                                    Οὐκ ἀπεικὸς οὖν 
πρὸς τοὺς πολυγραφοῦντας καὶ εἰς οὐκ ἀναγκαίους λόγους τὸν 
χρόνον καταναλίσκοντας, ἀνθηρολεκτοῦντας πρὸς κενοὺς λόγους, 
ἄψυχα ἐκφράζοντας κοσμεῖν καὶ ζῶα αἰνοῦντας ἢ ψέγοντας οὐ 
κατ' ἀξίαν δι' ἔμφασιν τῆς ἑαυτῶν πολυμαθείας, καὶ Κάλανον 
τὸν Ταξιληνὸν Ἰνδὸν εἰρηκέναι· Ἑλλήνων φιλοσόφοις οὐκ ἐξο-
μοιούμεθα παρ' οἷς ὑπὲρ μικρῶν καὶ ἀφελῶν πραγμάτων πολ-
λοὶ καὶ δεινοὶ ἀναλίσκονται λόγοι· ἡμεῖς γὰρ ὑπὲρ τῶν μεγίστων 
καὶ βιωφελεστάτων ἐλάχιστα καὶ ἁπλᾶ, ὡς πᾶσιν εὐμνημόνευτα, 
παραγγέλλειν εἰώθαμεν. 



Constantinus VII Porphyrogenitus Imperator Hist., De administrando imperio (3023: 008)
“Constantine Porphyrogenitus. De administrando imperio, 2nd edn.”, Ed. Moravcsik, G.
Washington, D.C.: Dumbarton Oaks, 1967; Corpus fontium historiae Byzantinae 1 (= Dumbarton Oaks Texts 1).
Chapter 16, line 6

Ἐξῆλθον οἱ Σαρακηνοὶ μηνὶ Σεπτεμβρίῳ τρίτῃ, ἰνδικτιῶνος 
δεκάτης, εἰς τὸ δωδέκατον ἔτος Ἡρακλείου, ἔτος ἀπὸ κτίσεως κόσμου 
͵ϛρλʹ. 



Constantinus VII Porphyrogenitus Imperator Hist., De administrando imperio 
Chapter 27, line 54

                                   Εἰσὶ δὲ μέχρι τῆς σήμερον, ἥτις ἐστὶν 
ἰνδικτιὼν ζʹ, ἔτη ἀπὸ κτίσεως κόσμου ͵ϛυνζʹ, ἀφ' οὗ ἐμερίσθη ἡ Λαγου-
βαρδία, ἔτη σʹ. 



Constantinus VII Porphyrogenitus Imperator Hist., De administrando imperio 
Chapter 29, line 234

                                        Ἀφ' οὗ δὲ ἀπὸ Σαλῶνα μετῴκη-
σαν εἰς τὸ Ῥαούσιον, εἰσὶν ἔτη φʹ μέχρι τῆς σήμερον, ἥτις ἰνδικτιὼν ζʹ 
ἔτους ͵ϛυνζʹ. 



Constantinus VII Porphyrogenitus Imperator Hist., De administrando imperio 
Chapter 45, line 40

Ἀπὸ δὲ τῆς ἐξ Ἱερουσαλὴμ μετοικήσεως αὐτῶν εἰς τὴν νῦν οἰκουμένην 
παρ' αὐτῶν χώραν εἰσὶν ἔτη υʹ ἢ καὶ φʹ μέχρι τῆς σήμερον, ἥτις ἐστὶν 
ἰνδικτιὼν ιʹ, ἔτος ἀπὸ κτίσεως κόσμου ͵ϛυξʹ ἐπὶ τῆς βασιλείας Κωνσταν-
τίνου καὶ Ῥωμανοῦ, τῶν φιλοχρίστων καὶ πορφυρογεννήτων βασιλέων 
Ῥωμαίων. 



Constantinus VII Porphyrogenitus Imperator Hist., De thematibus (3023: 009)
“Costantino Porfirogenito. De thematibus”, Ed. Pertusi, A.
Vatican City: Biblioteca Apostolica Vaticana, 1952; Studi e Testi 160.
Asia-Europe Asia, chapter 1, line 5

                                                                          Πρὸς δὲ 
τοὺς κατοικοῦντας τὴν Μεσοποταμίαν Συρίας καὶ τὴν μεγάλην Ἀσίαν, ἐν 
ᾗ κατοικοῦσιν Ἰνδοὶ καὶ Αἰθίοπες καὶ Αἰγύπτιοι, λέγεται δυτικὸν μέσον 
καὶ Ἀσία μικρά· ἡ γὰρ Ἀνατολή, καθώσπερ ἔφημεν, Ἰνδῶν ἐστι καὶ 
Αἰθιόπων καὶ Αἰγυπτίων καὶ τῶν λοιπῶν τῶν Ἀνατολὴν κατοικούντων. 



Constantinus VII Porphyrogenitus Imperator Hist., De thematibus 
Asia-Europe Europ, chapter 8, line 11

                                                                              Ἀπὸ 
δὲ τοῦ ἀκρωτηρίου τοῦ καλουμένου Ἀκτίου καὶ τὰς καλουμένας ἰνδικτιῶνας 
ἐκάλεσεν. 



Constantinus VII Porphyrogenitus Imperator Hist., De thematibus 
Asia-Europe Europ, chapter 8, line 12

            Οὕτω γὰρ γράφει Ἡσύχιος ὁ Ἰλλούστριος· «ἰνδικτιὼν τουτέστιν 
ἰνακτιὼν ἡ περὶ τὸ Ἄκτιον νίκη· διὰ τοῦτο ἄρχεται μὲν ἰνδικτιὼν ἀπὸ 
πρώτης καὶ καταλήγει μέχρι τῆς ιεʹ· καὶ πάλιν ὑποστρέφει καὶ ἄρχεται 
ἀπὸ πρώτης, διὰ τὸ τὸν Ἀντώνιον συνάρχοντα γενέσθαι Αὐγούστῳ τῷ 
Καίσαρι μέχρι τοῦ ιεʹ χρόνου, μετὰ δὲ ταῦτα μόνος ἐκράτησεν Αὔγουστος». 



Constantinus VII Porphyrogenitus Imperator Hist., De cerimoniis aulae Byzantinae (lib. 1.84–2.56) (3023: 010)
“Constantini Porphyrogeniti imperatoris de cerimoniis aulae Byzantinae libri duo, vol. 1”, Ed. Reiske, J.J.
Bonn: Weber, 1829; Corpus scriptorum historiae Byzantinae.
Page 433, line 4

                                                    τῇ οὖν τετάρτῃ 
τοῦ Ἀπριλίου μηνὸς ἰνδ. εʹ, μαγίστρου ὄντος Τατιανοῦ, 
ἐκέλευσεν σιλέντιον καὶ κομέντον καὶ τὰς σχολὰς καὶ τὰ 
στρατεύματα πάντα παραγενέσθαι ἐν τῷ δέλφακι. 



Constantinus VII Porphyrogenitus Imperator Hist., De cerimoniis aulae Byzantinae (lib. 1.84-2.56) 
Page 433, line 15

Τελευτήσαντος Ῥωμανοῦ βασιλέως τοῦ νέου, υἱοῦ Κων-
σταντίνου τοῦ μεγάλου καὶ πορφυρογεννήτου βασιλέως Ῥω-
μαίων τοῦ Μακεδόνος, εἰς μῆνα Μάρτιον ιεʹ, ἰνδ. ϛʹ, ἔτους 
͵ϛυοαʹ, τῇ τεσσαρακοστῇ τῶν νηστειῶν, κατέλειπεν τὴν ἑαυ-
τοῦ βασιλείαν Βασιλείῳ καὶ Κωνσταντίνῳ, τοὺς νηπίους υἱ-
οὺς αὐτοῦ καὶ τὴν ἰδίαν γαμετὴν καὶ αὐγούσταν Θεοφανῶ 
βασιλεύειν τῆς Ῥωμαίων ἀρχῆς. 



Constantinus VII Porphyrogenitus Imperator Hist., De cerimoniis aulae Byzantinae (lib. 1.84-2.56) 
Page 434, line 2

ἐκράτησεν δὲ ἡ τῶν ῥηθέντων προσώπων ἐξουσία ἀπὸ πεντε-
καιδεκάτην μηνὸς Μαρτίου ἰνδ. ϛʹ μέχρι Αὐγούστου πεντε-
καιδεκάτης, ἰνδ. 



Constantinus VII Porphyrogenitus Imperator Hist., De cerimoniis aulae Byzantinae (lib. 1.84-2.56) 
Page 434, line 3

                                   ϛʹ μέχρι Αὐγούστου πεντε-
καιδεκάτης, ἰνδ. τῆς αὐτῆς. 



Constantinus VII Porphyrogenitus Imperator Hist., De cerimoniis aulae Byzantinae (lib. 1.84-2.56) 
Page 434, line 3

                                 Ἰουλίου δὲ μηνὸς δευτέρᾳ, ἰνδ. 
ὁμοίως, ἀνηγορεύθη ἐν τοῖς τῆς ἀνατολῆς μέρεσιν ὁ εὐσεβὴς 
καὶ φιλόχριστος βασιλεὺς ἡμῶν Νικηφόρος παρὰ τοῦ ἰδίου 
στρατοπέδου βασιλεὺς Ῥωμαίων. 



Constantinus VII Porphyrogenitus Imperator Hist., De cerimoniis aulae Byzantinae (lib. 1.84-2.56) 
Page 438, line 3

                                           καὶ τῇ ἐπαύριον, ἑξ-
καιδεκάτῃ τοῦ αὐτοῦ Αὐγούστου μηνὸς, ἰνδ. ϛʹ, ἡμέρᾳ κυ-
ριακῇ πρωῒ ἐμβὰς εἰς τὸ βασιλικὸν δρομόνιον προσέβαλεν ἐν 
τῇ χρυσῇ πόρτῃ. 



Constantinus VII Porphyrogenitus Imperator Hist., De cerimoniis aulae Byzantinae (lib. 1.84-2.56) 
Page 511, line 1

                                    περὶ τῆς δοχῆς τῆς γε-
 νομένης ἐν τῷ αὐτῷ τρικλίνῳ ἐπὶ Κωνσταντίνου καὶ 
 Ῥωμανοῦ ἐπὶ τῇ παρουσίᾳ τῶν παρὰ τοῦ Ἀμεριμνῆ 
 ἀπὸ Ταρσοῦ ἐλθόντων πρέσβεων περὶ τοῦ ἀλλαγίου καὶ   
 τῆς εἰρήνης, μηνὶ Μαΐῳ λαʹ, ἡμέρᾳ αʹ, ἰνδικτίωνι δʹ. 



Constantinus VII Porphyrogenitus Imperator Hist., De cerimoniis aulae Byzantinae (lib. 1.84-2.56) 
Page 514, line 11

     Ἡ κατὰ τῆς αὐτῆς θεολέστου Κρήτης γενομένη ἐκστρα-
 τεία καὶ ἔξοδος ἐπὶ Κωνσταντίνου καὶ Ῥωμανοῦ τῶν 
 Πορφυρογεννήτων εἰς ἰνδικτίονα ζʹ. 



Constantinus VII Porphyrogenitus Imperator Hist., De cerimoniis aulae Byzantinae (lib. 1.84-2.56) 
Page 570, line 15t

Περὶ τῆς γενομένης δοχῆς ἐν τῷ περιβλέπτῳ καὶ μεγάλῳ τρικλίνῳ 
τῆς μανναύρας ἐπὶ Κωνσταντίνου καὶ Ῥωμανοῦ τῶν Πορφυρογεν-
νήτων ἐν Χριστῷ βασιλέων Ῥωμαίων, ἐπὶ τῇ παρουσίᾳ τῶν παρὰ 
τοῦ Ἀμεριμνῆ ἀπὸ τῆς Ταρσοῦ ἐλθόντων πρεσβέων περὶ τοῦ ἀλλα-
γίου καὶ τῆς εἰρήνης, μηνὶ Μαΐῳ λαʹ, ἡμέρᾳ αʹ, ἰνδικτ. 




Constantinus VII Porphyrogenitus Imperator Hist., De cerimoniis aulae Byzantinae (lib. 1.84-2.56) 
Page 588, line 16t

Περὶ τοῦ γεγονότος ἱπποδρομίου ἐπὶ τῇ ἐλεύσει τῶν φίλων Σαρα-
κηνῶν, διὰ τὴν εἰρήνην καὶ τὸ ἀλλάγιον, εἰς ἰνδ. δʹ ἐπὶ Κων-
σταντίνου καὶ Ῥωμανοῦ τῶν Πορφυρογεννήτων βασιλέων. 




Constantinus VII Porphyrogenitus Imperator Hist., De cerimoniis aulae Byzantinae (lib. 1.84-2.56) 
Page 627, line 17

Χρὴ εἰδέναι, ὅτι κατὰ τὴν τετάρτην τοῦ Ἰουλίου μη-
νὸς, ἰνδ. ιαʹ, ὁ αὐτοκράτωρ καὶ μέγας βασιλεὺς θελήσας 
ἀναγορεῦσαι Ἡράκλειον τὸν τούτου υἱὸν ἀπὸ τῆς ἀξίας τοῦ   
καίσαρος εἰς τὸ σχῆμα τῆς βασιλείας, ἐποίησεν οὕτως. 



Constantinus VII Porphyrogenitus Imperator Hist., De cerimoniis aulae Byzantinae (lib. 1.84-2.56) 
Page 628, line 23

Χρὴ εἰδέναι, ὡς τῇ πρώτῃ τοῦ Ἰαννουαρίου μηνὸς, ἰνδ.   
ιβʹ, ἐποίησεν πρόκενσον ὁ βασιλεὺς ἐν τῇ ἁγιωτάτῃ μεγάλῃ 
ἐκκλησίᾳ, καὶ ἐξῆλθεν μετ' αὐτοῦ Κωνσταντῖνος ὁ δεσπότης, 
φορῶν χλανίδιον, καὶ Ἡράκλειος ὁ δεσπότης καὶ υἱὸς αὐ-
τοῦ, φορῶν πραίσεκστον, καὶ παρὰ τοῦ ἰδίου ἀδελφοῦ πα-
ρακρατούμενος. 



Constantinus VII Porphyrogenitus Imperator Hist., De cerimoniis aulae Byzantinae (lib. 1.84-2.56) 
Page 630, line 14

Χρὴ εἰδέναι, ὅτι τῇ ιγʹ τοῦ Δεκεμβρίου μηνὸς, ἰνδ. ιβʹ, 
ἐτελειώθη Σέργιος ὁ πατριάρχης Κωνσταντινουπόλεως ἡμέρᾳ 
κυριακῇ· καὶ μετὰ τὸ δέξασθαι τὸν βασιλέα τοὺς ἄρχοντας 
κατὰ τὸ εἰωθὸς, ἀπέστειλεν αὐτοὺς εἰς τὴν κηδείαν τοῦ αὐ-
τοῦ πατριάρχου. 



Constantinus VII Porphyrogenitus Imperator Hist., De cerimoniis aulae Byzantinae (lib. 1.84-2.56) 
Page 641, line 3

                                             καὶ εἰς ἰνδ. δʹ 
ἀνεκαινίσθη ἐξ αὐτῶν ιβʹ, καὶ τῶν ϛʹ τὰ ἔργα εἰσὶν κατακλα-
σμένα μὴ ἔχοντα περιποίησιν. 



Constantinus VII Porphyrogenitus Imperator Hist., De cerimoniis aulae Byzantinae (lib. 1.84-2.56) 
Page 660, line 14t

Διὰ τῶν ἐν Λαγοβαρδίᾳ ταξειδευσάντων ἐπὶ τοῦ κυροῦ Ῥωμανοῦ 
τοῦ βασιλέως εἰς ἰνδ. ηʹ. 




Constantinus VII Porphyrogenitus Imperator Hist., De cerimoniis aulae Byzantinae (lib. 1.84-2.56) 
Page 660, line 16

Τὰ κατελθόντα μετὰ τοῦ πρωτοσπαθαρίου Ἐπιφανίου 
βασιλοπλόϊμα χελάνδια εἰς ἰνδικτίονα ηʹ ιαʹ· τὰ προκατελ-
θόντα μετὰ τοῦ πατρικίου Κοσμᾶ χελάνδια εἰς ἰνδικτίονα 
ζʹ ιαʹ· Ῥῶς καράβια ζʹ ἔχοντα ἄνδρας υιεʹ. 



Constantinus VII Porphyrogenitus Imperator Hist., De cerimoniis aulae Byzantinae (lib. 1.84-2.56) 
Page 664, line 6t

Ἡ κατὰ τῆς νήσου Κρήτης γενομένη ἐκστρατεία καὶ ἐξόπλισις τῶν 
τε πλοΐμων καὶ καβαλλαρικῶν ἐπὶ Κωνσταντίνου καὶ Ῥωμανοῦ τῶν 
Πορφυρογεννήτων ἐν Χριστῷ πιστῶν βασιλέων εἰς ἰνδικτίονα ζʹ. 



Constantinus VII Porphyrogenitus Imperator Hist., De cerimoniis aulae Byzantinae (lib. 1.84-2.56) 
Page 665, line 2

                                  κατελείφθη δὲ καὶ μία οὐσία 
εἰς τὸ κόψαι τὴν τῆς ὀγδόης ἰνδικτίονος ξυλήν. 



Constantinus VII Porphyrogenitus Imperator Hist., De cerimoniis aulae Byzantinae (lib. 1.84-2.56) 
Page 665, line 11

        κατελείφθη δὲ καὶ εἰς τὸ κόψαι τὴν τῆς ὀγδόης ἰν-
δικτίονος ξυλὴν οὐσίαι βʹ. 



Constantinus VII Porphyrogenitus Imperator Hist., De cerimoniis aulae Byzantinae (lib. 1.84-2.56) 
Page 691, line 20

                  εἰς τὸν ὑπερέχοντα κυριεύοντα Ἰνδίας· “Κων-
σταντῖνος καὶ Ῥωμανὸς, πιστοὶ ἐν Χριστῷ τῷ Θεῷ μεγάλοι 
αὐτοκράτορες βασιλεῖς Ῥωμαίων, πρὸς ὁ δεῖνα τὸν ὑπερ-
έχοντα κύριον τῆς Ἰνδίας, τὸν ἠγαπημένον ἡμῶν φίλον. 



Constantinus VII Porphyrogenitus Imperator Hist., De cerimoniis aulae Byzantinae (lib. 1.84-2.56) 
Page 702, line 6t

ΑΚΡΙΒΟΛΟΓΙΑ ΤΗΣ ΤΩΝ ΒΑΣΙΛΙΚΩΝ ΚΛΗΤΩΡΙΩΝ ΚΑΤΑ-
ΣΤΑΣΕΩΣ, ΚΑΙ ΕΚΑΣΤΟΥ ΤΩΝ ΑΞΙΩΜΑΤΩΝ ΠΡΟΣΚΛΗΣΙΣ 
ΚΑΙ ΤΙΜΗ, ΣΥΝΤΑΘΧΕΙΣΑ ΕΞ ΑΡΧΑΙΩΝ ΚΛΗΤΩΡΟΛΟΓΙΩΝ 
ΕΠΙ ΛΕΟΝΤΟΣ ΤΟΥ ΦΙΛΟΧΡΙΣΤΟΥ ΚΑΙ ΣΟΦΩΤΑΤΟΥ ΗΜΩΝ 
ΒΑΣΙΛΕΩΣ, ΜΗΝΙ ΣΕΠΤΕΜΒΡΙΩι, ΙΝΔ*ιΚΤ. 



Constantinus VII Porphyrogenitus Imperator Hist., De cerimoniis aulae Byzantinae (lib. 1.1–92) (3023: 011)
“Le livre des cérémonies, vols. 1–2”, Ed. Vogt, A.
Paris: Les Belles Lettres, 1:1935; 2:1939, Repr. 1967.
Volume 1, page 127, line 22

                                                           Καὶ εἶθ' 
οὕτως ἐπαίρουσιν οἱ δεσπόται τὴν λιτὴν ἀπὸ τοῦ ναοῦ τῆς 
ὑπεραγίας Θεοτόκου, καὶ ἀπέρχονται λιτανεύοντες εἰς τὸν 
ναὸν τοῦ Ἁγίου Βασιλείου, κἀκεῖσε ἀποδιδόντες τὴν λιτήν, 
ἵστανται μέχρι τῆς ἀπολύσεως τοῦ ἁγίου Εὐαγγελίου, καὶ 
μετὰ τὴν ἀπόλυσιν τῆς ἐκτενοῦς εἰσέρχονται πάλιν 
οἰκειακῶς ἐν τῷ Χρυσοτρικλίνῳ, 
[Συνέβη δὲ καὶ τοῦτο γενέσθαι τῇ αὐτῇ ἡμέρᾳ, ἰνδικ-
τιῶνι γʹ, τῆς λιτῆς τελεσθείσης, καθὼς προείρηται, μετὰ   
τὴν τῆς λειτουργίας ἀπόλυσιν ἐγένετο μεταστάσιμον καὶ 
ἀπῆλθον πάντες οἱ ἄρχοντες ἐν τῇ Μανναύρᾳ. 



Constantinus VII Porphyrogenitus Imperator Hist., De cerimoniis aulae Byzantinae (lib. 1.1-92) 
Volume 2, page 42, line 15

                          Καὶ μηνύεται ἀφ' ἑσπέρας πᾶσα ἡ 
σύγκλητος, ἵνα προέλθωσιν ἐπὶ προελεύσει, καὶ τὸ πρωῒ 
ἀλλάσσει ἡ σύγκλητος ἐν τῷ μάκρωνι τῶν κανδιδάτων, καὶ 
οἱ πατρίκιοι ἀλλάσσουσιν εἰς τοὺς Ἰνδούς, μὴ ἔχοντες 
ἄδειαν εἰσιέναι ἐν τῷ Κονσιστωρίῳ, ἱσταμένου τοῦ σένζου. 



Constantinus VII Porphyrogenitus Imperator Hist., De cerimoniis aulae Byzantinae (lib. 1.1-92) 
Volume 2, page 43, line 24

               Ὁ δὲ προβληθεὶς μάγιστρος ἐξέρχεται εἰς 
τοὺς Ἰνδούς, καὶ ἀλλάσσει σαγίον ἀληθινὸν ἐπάνω τοῦ στι-
χαρίου αὐτοῦ, καὶ ἀναχωρεῖ εἰς τὸν οἶκον αὐτοῦ, καὶ εἰ 
μέν ἐστιν πλησιάζων τῷ παλατίῳ ὁ οἶκος αὐτοῦ, ὀψικεύε-
ται ὑπὸ ἀξιωματικῶν καὶ δομεστίκων πεδίτου καὶ σχολαρίων 
πεδίτου καὶ σκουταρίων τοῦ ἀριθμοῦ καὶ διαιταρίων καὶ δεκα-
νῶν· εἰ δέ ἐστιν μηκόθεν, οἱ αὐτοὶ ἄνευ τῶν ἀξιωματικῶν. 


\end{greek}



\section{Patria of Constantinople}
\blockquote[From Wikipedia\footnote{\url{http://en.wikipedia.org/wiki/Patria_of_Constantinople}}]{The Patria of Constantinople (Greek: Πάτρια Κωνσταντινουπόλεως),[1] also known by the Latin name Scriptores originum Constantinopolitarum ("writers on the origins of Constantinople"), is a Byzantine collection of historical works on the history and monuments of the Byzantine imperial capital of Constantinople (modern Istanbul, Turkey).[2]

Although in the past attributed to the 14th-century writer George Kodinos,[3] the collection in fact dates from earlier centuries, being probably first compiled in ca. 995 in the reign of Basil II (r. 976–1025) and then revised and added to in the reign of Alexios I Komnenos (r. 1081–1118).[4]}
\begin{greek}

Patria Constantinopoleos, Διήγησις περὶ τῆς Ἁγίας Σοφίας (3120: 002)
“Scriptores originum Constantinopolitanarum, pt. 1”, Ed. Preger, T.
Leipzig: Teubner, 1901, Repr. 1975.
Section 23, line 14

                                      Ἐπεκύρωσε δὲ καὶ κτήματα 
τξεʹ ἀπὸ Αἰγύπτου καὶ Ἰνδίας καὶ πάσης ἑῴας καὶ δύσεως 
ὑπάρχοντα εἰς διοίκησιν τοῦ ναοῦ· ἐκτυπώσας μίαν ἑκάστην 
ἑορτὴν δίδοσθαι ἔλαιον μέτρα χιλιάδα μίαν καὶ οἴνου μέτρα 
τʹ, ἄρτους τῆς προθέσεως χιλίους. 

\end{greek}

\section{Joannes Rhet (?)}
?
\blockquote[From Wikipedia\footnote{\url{}}]{}
\begin{greek}

Joannes Rhet., Prolegomena in Hermogenis librum περὶ ἰδεῶν (4235: 001)
“Prolegomenon sylloge”, Ed. Rabe, H.
Leipzig: Teubner, 1931; Rhetores Graeci 14.
Volume 14, page 413, line 14

ἐπὶ τὴν θεωρίαν, ἔνθα δείξομεν κατὰ ποσὸν καταληπτοὺς 
εἶναι τοὺς χαρακτῆρας, εἴ γε καὶ οἱ περὶ Διονύσιον εἰς 
ὡρισμένον αὐτοὺς ἀποδεδώκασιν ἀριθμόν· ἐκεῖνο γὰρ 
λῆρος τὸ τὰς ἐπιτάσεις τούτων καὶ ἀνέσεις ἀντ' ἄλλων 
παρὰ τοὺς ἐξ ἀρχῆς ὑπολαμβάνειν· ὡρισμένοι γὰρ ὄντες 
τὴν ἀρχήν, κἂν αὔξησιν ἢ μείωσιν δέξωνται, οὐδὲν 
ἧττον οἱ αὐτοί εἰσιν· ἐπεὶ καὶ τῶν σωμάτων οἱ χαρακτῆ-
ρες ἄπειροι δοκοῦντες ἐκ λευκοῦ καὶ μέλανος συν-
εστήκασιν, ὧν ὃ μὲν πρὸς ἑκατέραν ἐπίτασιν λευκὸν 
λέγει τὸν χαρακτῆρα εἶναι ἢ μέλανα, Σκύθην τυχὸν ἢ 
Ἰνδόν, ἡ δὲ ἐκ τούτων μῖξις, κἂν ἐπὶ τὸ μᾶλλον καὶ 
ἧττον γένηται, ὑπόλευκον ποιεῖ καὶ εἶναι καὶ λέγεσθαι 
ἢ ὑπομέλανα ἢ σιτόχρουν καὶ ἡ συνήθεια καλεῖ· κἂν 
μέντοι ὑπέρυθρόν τι εἴη ἢ φοινικοῦν ἢ ἄλλο τι τῶν 
τοιούτων, οὐ διὰ τοῦτο ἐρυθρίας λέγεται ὁ ἔχων ἢ φοι-
νικοῦς, ἀλλά τι τῶν ὡρισμένων, καὶ ἀπερρίφθω τὰ ἐκ 
πάθους, παρατροπὴ γάρ ἐστι φύσεως. 



Joannes Rhet., Commentarium in Hermogenis librum περὶ ἰδεῶν (4235: 002)
“Rhetores Graeci, vol. 6”, Ed. Walz, C.
Stuttgart: Cotta, 1834, Repr. 1968.
6, Page 87, line 13

τὸ τελέως, ὡς οὗτοι λέγουσιν, αἰσθήσει καταλαμβάνεται, 
καὶ οὐδὲν διαφέρει τοῦ ἀκριβοῦς κατὰ ταύτην, ὥς τινες 
οἴονται· ἀλλ' ὁ μὲν τὸ ὅλον ταύτῃ ἀνεὶς, περὶ τὴν ταύ-
της τέχνην ἀλόγως ἔχει, καὶ τὴν ἐκείνης διοίκησιν ἀγνοεῖ, 
μᾶλλον δὲ καὶ πολὺ ῥαπίζεται καὶ ἀποῤῥίπτεται· ὥσπερ 
ἴσμεν τινὰ τοῖς θεολογικοῖς ἐπιστήσαντα, τοῦτο δὴ τὸ τοῦ 
λόγου ἡλίῳ νυκτάλωπα· ὁ δὲ κατὰ μέρος ἔχων ἐπιστημό-
νως καὶ ἐμβατεύων μετὰ λόγου ταῖς θεωρίαις, ὃ ἂν βού-
ληται λογισμῷ καὶ τέχνῃ ποιεῖ, καὶ καθάπερ ὁ ἐπ' Ἀλε-
ξάνδρου τοξότης Ἰνδός. 

\end{greek}


\section{Michael Apostolius}
\blockquote[From Wikipedia\footnote{\url{http://en.wikipedia.org/wiki/Michael_Apostolius}}]{Michael Apostolius (Μιχαὴλ Ἀποστόλιος or Μιχαὴλ Ἀποστόλης; c. 1420 in Constantinople – after 1474 or 1486, possibly in Venetian Crete )[1] or Apostolius Paroemiographus, i.e. Apostolius the proverb-writer, was a Greek teacher, writer and copyist who lived in the 15th century CE.

Of his numerous works a few have been printed:

    Παροιμίαι (Paroemiae, Greek for "proverbs"), a collection of proverbs in Greek
        an edition published in Basel in 1538, now exceedingly rare
        a fuller edition edited by Daniel Heinsius ("Curante Heinsio") and published in Leiden in 1619[2]
    "Oratio Panegyrica ad Fredericum III." in Freher's Scriptores Rerum Germanicarum, vol. ii. (Frankfort, 1624)
    Georgii Gemisthi Plethonis et Mich. Apostolii Orationes funebres duae in quibus de Immortalitate Animae exponitur (Leipzig, 1793)
    a work against the Latin Church and the council of Florence in Le Moine's Varia Sacra.}

\begin{greek}

Michael Apostolius Paroemiogr., Collectio paroemiarum (9009: 001)
“Corpus paroemiographorum Graecorum, vol. 2”, Ed. von Leutsch, E.L.
Göttingen: Vandenhoeck \& Ruprecht, 1851, Repr. 1958.
Centuria 5, section 9, line 2

<Βοῦς εἰς ἀμητόν:> ἐπὶ τῶν ἐπ' ὠφελείᾳ καμνόντων⁝ 
<Πτολεμαίῳ τῷ> δευτέρῳ, φασίν, ἐξ Ἰνδῶν κέρας ἐκομίσθη 
καὶ τρεῖς ἀμφορέας ἐχώρησεν. 



Michael Apostolius Paroemiogr., Collectio paroemiarum 
Centuria 7, section 4, line 3

<Ἐλέφαντος οὐδὲν διαφέρεις:> ἐπὶ τῶν ἀναισθή-
των· παρόσον καὶ τὸ ζῷον τοιοῦτον⁝ <Ἐλέφαντος> πω-
λίῳ περιτυγχάνει λευκῷ πωλευτὴς Ἰνδός, καὶ παραλαβὼν 
ἔτρεφεν ἔτι νεαρὸν καὶ κατὰ μικρὰ ἀπέφηνε χειροήθη, καὶ 
ἐπωχεῖτο αὐτῷ· ὁ τοίνυν βασιλεὺς τῶν Ἰνδῶν πυθόμενος, 
ᾔτει λαβεῖν τὸν ἐλέφαντα. 



Michael Apostolius Paroemiogr., Collectio paroemiarum 
Centuria 7, section 4, line 10

                                 ὁ δὲ ὡς ἐρώμενον ζηλοτυπῶν 
καὶ μέντοι περιαλγῶν, εἰ ἔμελλε δεσπόσειν αὐτοῦ ἄλλος, 
οὐκ ἔφατο δώσειν, καὶ ᾤχετο ἀπιὼν ἐς τὴν ἔρημον, ἀνα-
βὰς τὸν ἐλέφαντα· ἀγανακτεῖ ὁ βασιλεύς, καὶ πέμπει κατ' 
αὐτοῦ τοὺς ἀφαιρησομένους, καὶ ἅμα καὶ τὸν Ἰνδὸν ἐπὶ 
τὴν δίκην ἄξοντας· ἐπεὶ δὲ ἧκον, ἐπειρῶντο πεῖραν προς-  
φέρειν· οὐκοῦν καὶ ὁ ἄνθρωπος ἔβαλλεν αὐτοὺς ἄνωθεν, 
καὶ τὸ θηρίον ὡς ἀδικούμενον συνημύνετο· καὶ τὰ μὲν 
πρῶτα ἦν τοιαῦτα. 



Michael Apostolius Paroemiogr., Collectio paroemiarum 
Centuria 7, section 4, line 14

                      ἐπεὶ δὲ βληθεὶς ὁ Ἰνδὸς κατώλισθε, 
περιβαίνει μὲν τὸν τροφέα ὁ ἐλέφας κατὰ τοὺς ὑπερασπί-
ζοντας ἐν τοῖς ὅπλοις, καὶ τῶν ἐπιόντων πολλοὺς ἀπέ-
κτεινε, τοὺς δὲ ἄλλους ἐτρέψατο· περιβαλὼν δὲ τῷ τροφεῖ 
τὴν προβοσκίδα, αἴρει τε αὐτὸν καὶ ἐπὶ τὰ αὔλια κομίζει, 
καὶ παρέμεινεν ὡς φίλῳ φίλος πιστός. 



Michael Apostolius Paroemiogr., Collectio paroemiarum 
Centuria 7, section 8, line 5

<Ἐλέφας μῦν οὐκ ἀλεγίζει:> ἐπὶ τῶν τὰ μικρὰ καὶ 
φαῦλα ὑπερορώντων⁝ <Λόγος> τίς ἐστιν, ἐλέφαντα μὴ πρό-
τερον πίνειν, πρὶν ἂν ὑποθολώσῃ τῇ προνομαίᾳ τὰ νά-
ματα· τὸ δ' αἴτιον, ὅτι μορμολύττεται τὴν ἑαυτοῦ σκιὰν 
ἐν ὕδατι ὁ ἐλέφας θεώμενος· διὸ καὶ τοὺς Ἰνδοὺς ἐπιτη-
ρεῖν ἀσέληνον νύκτα, ὁπηνίκα διαπορθμεύωσι, τουτὶ δεδιό-
τες τὸ ζῷον. 



Michael Apostolius Paroemiogr., Collectio paroemiarum 
Centuria 7, section 74, line 1

<Ἔποπος Ἰνδοῦ στοργή:> ἐπὶ τῶν ἄγαν φιλούντων 
προσγενεῖς καὶ φίλους, καὶ ὑπὲρ τούτων θανεῖν αἱρουμένων⁝ 
<Περὶ τοῦ Ἰνδοῦ> ἔποπος μῦθόν φασιν οἱ Βραχμᾶνες. 



Michael Apostolius Paroemiogr., Collectio paroemiarum 
Centuria 7, section 74, line 4

                                                                   Ἐγέ-
νετο παῖς Ἰνδῶν βασιλεῖ, καὶ ἀδελφοὺς εἶχεν, οἵπερ οὖν 
ἀνδρωθέντες ἐκδικώτατοί τε γίνονται καὶ λεωργότατοι, καὶ 
τούτου μὲν ὡς νεωτάτου καταφρονοῦσι, τὸν δὲ πατέρα   
ἐκερτόμουν καὶ τὴν μητέρα, τὸ γῆρας αὐτῶν ἐκφαυλίσαν-
τες· ἀναίνονται οὖν ἐκεῖνοι τὴν σὺν τούτοις διατριβήν, 
καὶ ᾤχοντο φεύγοντες ὅ τε παῖς καὶ οἱ γέροντες. 



Michael Apostolius Paroemiogr., Collectio paroemiarum 
Centuria 7, section 74, line 29

ὃς ἔφασκε λέγων κορυδὸν πάντων πρώτην 
       ὄρνιθα γενέσθαι, 
 προτέραν τῆς γῆς· κἄπειτα νόσῳ τὸν πατέρα 
       αὐτῆς ἀποθνήσκειν· 
 γῆν δὲ οὐκ εἶναι· τὸν δὲ προκεῖσθαι πεμ-
       πταῖον, τὴν δὲ ἀποροῦσαν 
 ὑπ' ἀμηχανίας τὸν πατέρα αὐτῆς ἐν τῇ κε-
       φαλῇ κατορύξαι·> 
ἔοικεν οὖν ἐξ Ἰνδῶν τὸ μυθολόγημα ἐπ' ἄλλου μὲν ὄρνι-
θος, ἐπιῤῥεῦσαι δ' οὖν καὶ τοῖς Ἕλλησιν· ὠγύγιον γάρ τι 
χρόνου μῆκος λέγουσι Βραχμᾶνες, ἐξ οὗ αὐτῷ τῷ ἔποπι 
τῷ Ἰνδῷ, ἔτι ἀνθρώπῳ ὄντι καὶ παιδὶ τήν γε ἡλικίαν, ἐς 
τοὺς γειναμένους πέπρακται. 



Michael Apostolius Paroemiogr., Collectio paroemiarum 
Centuria 7, section 74, line 32

προτέραν τῆς γῆς· κἄπειτα νόσῳ τὸν πατέρα 
       αὐτῆς ἀποθνήσκειν· 
 γῆν δὲ οὐκ εἶναι· τὸν δὲ προκεῖσθαι πεμ-
       πταῖον, τὴν δὲ ἀποροῦσαν 
 ὑπ' ἀμηχανίας τὸν πατέρα αὐτῆς ἐν τῇ κε-
       φαλῇ κατορύξαι·> 
ἔοικεν οὖν ἐξ Ἰνδῶν τὸ μυθολόγημα ἐπ' ἄλλου μὲν ὄρνι-
θος, ἐπιῤῥεῦσαι δ' οὖν καὶ τοῖς Ἕλλησιν· ὠγύγιον γάρ τι 
χρόνου μῆκος λέγουσι Βραχμᾶνες, ἐξ οὗ αὐτῷ τῷ ἔποπι 
τῷ Ἰνδῷ, ἔτι ἀνθρώπῳ ὄντι καὶ παιδὶ τήν γε ἡλικίαν, ἐς 
τοὺς γειναμένους πέπρακται. 



Michael Apostolius Paroemiogr., Collectio paroemiarum 
Centuria 9, section 75, line 2

<Κερκιώνων ἐλευθερία:> ἐπὶ τῶν ἀδεσπότως ζῆν 
ἐθελόντων⁝ <Γίνεται ἐν τοῖς>20 Ἰνδοῖς ὄρνις ὄνομα κερ-
κίων, μέγεθος κατὰ ψᾶρας, καὶ ἔστι ποικίλον καὶ μουσω-
θὲν ἀνθρώπων φωνήν. 



Michael Apostolius Paroemiogr., Collectio paroemiarum 
Centuria 9, section 87, line 2

<Κορώνη γράμμα κομίζει:> ἐπὶ τῶν ἀγγελίας ταχείας 
φερόντων⁝ <Τῷ γὰρ βασιλεῖ> τῶν Ἰνδῶν, Μάρης οὗτος 
ἐκαλεῖτο, ἦν κορώνη θρέμμα πάνυ ἥμερον· καὶ τῶν ἐπι-
στολῶν ἃς ἐβούλετό οἱ κομισθῆναί που θᾶττον ἐκόμιζεν 
αὕτη, καὶ ἦν ἀγγέλων ὠκίστη, καὶ ἀκούσασα ᾔδει, ἔνθα 
ἰθῦναι χρὴ τὸ πτερὸν καὶ τίνα χρὴ παραδραμεῖν χῶρον, 
καὶ ὅπου ἥκουσαν ἀναπαύσεσθαι, ἀνθ' ὧν ἀποθανοῦσαν 
αὐτὴν ὁ Μάρης ἐτίμησε καὶ στήλῃ καὶ τάφῳ⁝ <Λέγεται 
καὶ τοῦτο> περὶ κορωνῶν, ὅτι ἀλλήλαις εἰοὶ πιστόταται, 
καὶ ὅταν εἰς κοινωνίαν συνέλθωσι, πάνυ σφόδρα ἀγαπῶσιν 
ἀλλήλας· καὶ οὐκ ἂν ἴδοι τις μιγνύμενα ταῦτα τὰ ζῷα 




Michael Apostolius Paroemiogr., Collectio paroemiarum 
Centuria 10, section 32, line 1

<Κύνες Ἰνδικοί:> θηρία δὲ οἵδε τὴν ψυχὴν θυμοει-
δέστατα· καὶ τῶν μὲν ἄλλων ζῴων ὑπερφρονοῦσι, λέοντι 
δὲ ὁμόσε χωρεῖ κύων Ἰνδικὸς καὶ ἐγκείμενον ὑπομένει καὶ 
βρυχωμένῳ ἀνθυλακτεῖ καὶ ἀντιδάκνει δάκνοντα· καὶ πολλὰ 
λυπήσας τελευτῶν ἡττᾶται ὁ κύων. 



Michael Apostolius Paroemiogr., Collectio paroemiarum 
Centuria 10, section 32, line 6

                                       εἴη δ' ἂν καὶ λέων ἡττη-
θεὶς ὑπὸ κυνὸς Ἰνδοῦ, καὶ μέντοι καὶ δακὼν ὁ κύων ἔχε-
ται καὶ μάλα ἐγκρατῶς· κἂν προσελθὼν μαχαίρᾳ τὸ σκέλος 
ἀποκόπτῃς τοῦ κυνός, ὁ δὲ οὐκ ἄγει σχολὴν ἀλγήσας ἀνεῖ-
ναι τὸ δῆγμα, ἀλλ' ἀπεκόπη μὲν πρότερον τὸ σκέλος, νε-
κρὸς δὲ ἀνῆκε τὸ στόμα⁝ <Ἀνθρώπου μόνου> καὶ κυ-
νὸς κορεσθέντων ἀναπλεῖ ἡ τροφή. 



Michael Apostolius Paroemiogr., Collectio paroemiarum 
Centuria 10, section 37, line 1

<Κώνωπος ἐλέφας Ἰνδὸς οὐκ ἀλεγίζει. 



Michael Apostolius Paroemiogr., Collectio paroemiarum 
Centuria 11, section 81, line 1

<Μύδας ὀφθαλμοὺς Ἰνδικὸς δοκιμάζει:> ἔστι δὲ 
ὁ μύδας λίθος ἐν Ἰνδίᾳ, διττὰ ἀφιεὶς τὰ σέλα, τοῖς μὲν 
πονήρως ἔχουσι τῶν ὀμμάτων δριμὺ καὶ πυρῶδες καὶ ἐπιει-
κῶς φλογωπόν, τοῖς δὲ κατὰ φύσιν καὶ ῥωστικὸν καὶ σω-
τήριον. 



Michael Apostolius Paroemiogr., Collectio paroemiarum 
Centuria 12, section 82, line 4

<Ὄνος λύρας· ἀκούων κινεῖ τὰ ὦτα:> ἐπὶ τῶν ἀπαι-
δεύτων· ἢ ἐπὶ τῶν συγκαταθεμένων μηδὲ ἐπαινούντων⁝ 
<Πέπυσμαι> ὄνους ἀγρίους οὐκ ἐλάττονας ἵππων τὰ με-  
γέθη ἐν Ἰνδοῖς γίνεσθαι· κέρας δὲ ἔχειν ἐπὶ τῷ μετώπῳ, 
ὅσον πήχεως τὸ μέγεθος καὶ ἡμίσεος προσέτι, καὶ τὸ μὲν 
κάτω μέρος τοῦ κέρατος εἶναι λευκόν, τὸ δὲ ἄνω φοινι-
κοῦν, τό γε μὴν μέσον μέλαν δεινῶς. 



Michael Apostolius Paroemiogr., Collectio paroemiarum 
Centuria 14, section 71, line 1

<Ποηφάγου δειλότερος: τουτὶ τὸ> ζῷον ἐν Ἰνδοῖς 
ἐστί, καὶ πέφυκέ γε διπλάσιον ἵππου τὸ μέγεθος· οὐρὰν 
δὲ ἔχει δασυτάτην καὶ μελαίνης ἀκράτως χρόας· καί εἰσιν 
αὗται αἱ τρίχες τῶν ἀνθρωπίνων λεπτότεραι ἂν καὶ ἐν 
μεγάλῳ τίθενται ταύτας ἔχειν Ἰνδῶν αἱ γυναῖκες· καὶ γάρ 
τοι παραπλέκονται μάλα ἐξ αὐτῶν καὶ κοσμοῦνται ὡραίως, 
ταῖς πλοκαμῖσι ταῖς συμφύτοις καὶ ταύτας ὑποδέουσαι. 



Michael Apostolius Paroemiogr., Collectio paroemiarum 
Centuria 17, section 71, line 2

<Ὗς Διαροίδων:> ἐπὶ τῶν σκαιῶν καὶ ἀπαιδεύτων· 
Κράτης⁝ <Ὗν οὔτε> ἄγριον οὔτε ἥμερον ἐν Ἰνδοῖς γενέ-
σθαι λέγει Κτησίας, πρόβατα δὲ τὰ ἐκείνων οὐρὰς πήχεως 
ἔχειν τὸ πλάτος που φησίν. 


\end{greek}

\section{Pseudo-Macarius}
\blockquote[From Wikipedia\footnote{\url{http://en.wikipedia.org/wiki/Pseudo-Macarius}}]{Macarius of Egypt (ca. 300 – 391) was an Egyptian Christian monk and hermit. He is also known as Macarius the Elder, Macarius the Great and The Lamp of the Desert.

Fifty Spiritual Homilies were ascribed to Macarius a few generations after his death, and these texts had a widespread and considerable influence on Eastern monasticism and Protestant pietism. [5] This was particularly in the context of the debate concerning the 'extraordinary giftings' of the Holy Spirit in the post-apostolic age, since the Macarian Homilies could serve as evidence in favour of a post-apostolic attestation of 'miraculous' Pneumatic giftings to include healings, visions, exorcisms, etc. The Macarian Homilies have thus influenced Pietist groups ranging from the Spiritual Franciscans (West) to Eastern Orthodox monastic practice to John Wesley to modern charismatic Christianity.

However, modern patristic scholars have established that it is not likely that Macarius the Egyptian was their author.[6] Exactly who the author of these fifty Spiritual Homilies was has not been definitively established, although it is evident from statements in them that the author was from Upper Mesopotamia, where the Roman Empire bordered the Persian Empire, and that they were not written later than 534.[7]}

\begin{greek}

Pseudo-Macarius Scr. Eccl., Sermones 64 (collectio B) (2109: 001)
“Makarios/Symeon Reden und Briefe, 2 vols.”, Ed. Berthold, H.
Berlin: Akademie–Verlag, 1973; Die griechischen christlichen Schriftsteller.
Homily 34, chapter 7, section 1, line 3

                                      ὅθεν ἐν τῇ Ἰνδικῇ ἀπὸ τοσούτου διαστήματος 
μόνον ἀκούσαντες βάρβαροι περὶ Ἰησοῦ, ὅτι ἐν τῇ Παλαιστίνῃ ἐπεφάνη ἐν 
σώματι ἀνθρώπου, καὶ ἐποίησε σημεῖα σταυρωθεὶς καὶ ἀναστάς, ἐκ τῆς φήμης 
ἐπίστευσαν τοσοῦτον, ὅτι θεός ἐστιν, ὥστε καὶ εἰς μαρτυρίαν καὶ καῦσιν τὰ 
σώματα αὐτῶν παραδοῦναι. 



Pseudo-Macarius Scr. Eccl., Sermones 64 (collectio B) 
Homily 34, chapter 8, section 1, line 1

Ὥσπερ ἵνα ᾖ πόλις καὶ εἰσελθόντες Ἰνδοὶ καὶ Σαρακηνοὶ κατάσχωσιν αὐτήν, 
ἔλθῃ δὲ ὁ βασιλεύς, οὗ ἐστιν ἡ πόλις, καὶ ἐκβάλῃ αὐτοὺς καὶ δώῃ αὐτῇ ἰδίαν 
βασιλείαν καὶ ἰδίαν στρατιάν, ὥστε εἶναι ἐκείνην τὴν πόλιν ἐν χαρᾷ καὶ εἰρήνῃ 
βαθυτάτῃ, ἔχουσαν ἥμερον βασιλέα. 

\end{greek}


\section{Michael Attaleiates}
\blockquote[From Wikipedia\footnote{\url{http://en.wikipedia.org/wiki/Michael_Attaliates}}]{Michael Attaleiates or Attaliates (Greek: Μιχαήλ Ἀτταλειάτης) (c.1022-1080) was a Byzantine public servant and historian active in Constantinople and around the empire's provinces in the second half of the eleventh century.[1] He was a younger contemporary (possibly even a student) of Michael Psellos and likely an older colleague of John Skylitzes, the two other Byzantine historians of the eleventh century whose work survives.

Michael Attaleiates was probably a native of Attaleia (now Antalya, in Turkey) and moved to Constantinople between 1030 and 1040 to pursue studies in law.[2] During years of service in the empire's judicial system he built a small private fortune. Prominence on the judge's bench also brought him to the attention of a number of emperors who rewarded him with some of the highest honours available to civil servants (patrikios and anthypatos).

In 1072 Attaleiates compiled for Emperor Michael VII a synopsis of law, known as the Ponema Nomikon, based on the late ninth-century Basilika.

In addition he drew up an Ordinance for the Poor House and Monastery which he founded at Constantinople in the mid-1070s. This work, known as the Diataxis, is of value for students of the social, economic, cultural and religious history of Byzantium in Constantinople and the provinces during the eleventh century. It also provides invaluable information regarding the life of Attaleiates himself. It includes a catalogue of the books available in the monastery's library, while also offering details about the founder's fortune in the capital and in Thrace. From the Diataxis we learn that Attaleiates owned numerous properties (both farms and urban real estate) in Constantinople, Raidestos (mod. Tekirdag), Selymbria (mod. Silivri).

Around 1079/80 Michael Attaleiates circulated The History, a political and military history of the Byzantine Empire from 1034 to 1079. This vivid and largely reliable presentation of the empire's declining fortunes after the end of the Macedonian dynasty, offered Attaleiates the opportunity to engage with political questions of his time also addressed, albeit often from a different point of view, by his contemporary Michael Psellos.[3] The History concludes with a long encomium to Emperor Nikephoros III Botaneiates, to whom the whole work is dedicated. On account of this encomium and dedication, Attaleiates was for years considered an honest supporter of this elderly and largely ineffective emperor. Careful reading of his text, however, suggests that the words of praise may be less than honest. Instead Attaleiates appears to be partial towards the young military commander and future emperor Alexios Komnenos.[4]

Attaleiates probably died around 1080, shortly before the beginning of the Komnenian era. He therefore had no chance to rededicate his work to the founder of the Komnenian dynasty, Alexios I Komnenos, whom The History treats as a potential saviour of the Byzantine state. He was outlived by his son Theodore, who died sometime before 1085. Their bodies, along with those of the judge's two wives, Eirene and Sophia, were put to rest on the grounds of the church of St. George of the Cypresses in the southwestern side of Constantinople. This was the area where the family's Constantinopolitan estates were likely clustered, close to the monastery of Christ Panoikteirmon, of which the Attaleiatai were patrons. One may still visit the church of St George (Samatya Aya Yorgi Rum Ortodoks Kilisesi), which today, after two fires and extensive reconstruction, bears no resemblance to the church of Attaleiates' day.[5]}

\begin{greek}

Michael Attaliates Hist., Historia (3079: 001)
“Michaelis Attaliotae historia”, Ed. Bekker, I.
Bonn: Weber, 1853; Corpus scriptorum historiae Byzantinae.

Michael Attaliates Hist., Historia 
Page 85, line 12

                                                          καὶ ἐῴ-
κει τῇ μυθευομένῃ τοῦ Διονύσου στρατιᾷ, ὅτε μετὰ τῶν μαι-
νάδων ἐκεῖνος καὶ τῶν Σειληνῶν ταύτην ἐπ' Ἰνδοὺς ἤλαυνεν. 



Michael Attaliates Hist., Historia 
Page 148, line 9

Διελθὼν οὖν ἡμέραν ἐξ ἡμέρας τὴν προκειμένην ὁδόν, 
κατέλαβε τὴν Θεοδοσίου πόλιν, ἐπὶ μὲν τῷ πρὸ τοῦ χρόνῳ 
παραμεληθεῖσαν καὶ ἀοίκητον γενομένην διὰ τὸ ἐν τῇ πολι-
τείᾳ τοῦ Ἄρτζη πλησίον οὔσῃ καὶ ἐν καλῷ τῆς θέσεως ὁρω-
μένῃ μεταθέσθαι τοὺς ἀνθρώπους τὴν οἴκησιν, καὶ μεγάλην 
ἐγκαταστῆσαι χωρόπολιν καὶ παντοίων ὠνίων, ὅσα Περσική 
τε καὶ Ἰνδικὴ καὶ ἡ λοιπὴ Ἀσία φέρει, πλῆθος οὐκ εὐαρίθ-
μητον φέρουσαν, πρὸ ὀλίγων δὲ χρόνων ἀνοικοδομηθεῖσαν καὶ 
κατοχυρωθεῖσαν, τὴν Θεοδοσίου πόλιν λέγω, τάφρῳ καὶ τεί-
χεσι διὰ τὴν τῶν Τούρκων ἐκ τοῦ ἀνελπίστου γειτνίασιν, δι' 
ὧν ἐξ ἐπιδρομῆς ἡ πολιτεία τοῦ Ἄρτζη παμπληθεὶ τὴν σφα-
γὴν προϋπέμεινε καὶ τὴν ἅλωσιν. 

\end{greek}


\section{George Pachymeres}
\blockquote[From Wikipedia\footnote{\url{http://en.wikipedia.org/wiki/Georgius_Pachymeres}}]{ation, search
Georgios Pachymeres

Georgius Pachymeres (Greek: Γεώργιος Παχυμέρης) (1242 – c. 1310), a Byzantine Greek historian, philosopher and miscellaneous writer, was born at Nicaea, in Bithynia, where his father had taken refuge after the capture of Constantinople by the Latins in 1204. Upon the recovery of The City from the Latin Empire by Michael VIII Palaeologus, Pachymeres settled in Constantinople, studied law, entered the church, and subsequently became chief advocate of the church and chief justice of the imperial court. His literary activity was considerable, his most important work being a Byzantine history in thirteen books, in continuation of that of George Acropolites from 1261 (or rather 1255) to 1308, containing the history of the reigns of Michael and Andronicus II Palaeologus. He was also the author of rhetorical exercises on philosophical themes; of a Quadrivium (arithmetic, music, geometry, astronomy), valuable for the history of music and astronomy in the Middle Ages; a general sketch of Aristotelian philosophy; a paraphrase of the speeches and letters of Pseudo-Dionysius the Areopagite; poems, including an autobiography; and a description of the square of the Augustaeum, and the column erected by Justinian in the church of Hagia Sophia to commemorate his victories over the Persians. The History was first published in print by I Bekker (1835) in the Corpus scriptorum hist. byzantinae; also in JP Migne, Patrologia Graeca, vol. cxliii, cxliv; for editions of the minor works see Karl Krumbacher, Geschichte der byzantinischen Litteratur (1897). A more recent edition with French translation of the 'History' by Faiiler and Laurent was published in 1984. An English translation of Books I and II (up to the recovery of Constantinople in 1261), with commentary, exists in the form of a PhD thesis (author Nathan Cassidy) held in the Reid Library of the University of Western Australia.}
\begin{greek}

Georgius Pachymeres Hist., Συγγραφικαὶ ἱστορίαι (libri vi de Michaele Palaeologo) (3142: 001)
“Georges Pachymérès. Relations historiques, 2 vols.”, Ed. Failler, A., Laurent, V.
Paris: Les Belles Lettres, 1984; Corpus fontium historiae Byzantinae 24.1–2. Series Parisiensis.

Georgius Pachymeres Hist., Συγγραφικαὶ ἱστορίαι (libri vii de Andronico Palaeologo) 
Page 459, line 5

                                                             οὗτος   
αὐτανέψιον ἔχων Τουκταΐν, ᾧ δὴ προσῆκεν ἐκ γένους καὶ ἡ ἀρ-
χή, ἐπεὶ πρὸς θανάτῳ ἦν καὶ οὐ τοῖς ἰδίοις τρόποις τοὺς ἐκείνου 
συμβαίνοντας ὑπετόπαζε, παριδὼν αὐτὸν ἔφεδρον εἰς ἀρχὴν ἐκ 
τοῦ ἀναγκαίου ὄντα, πέμψας μετακαλεῖται τὸν αὐτοῦ ἀδελφὸν 
περί που τὰ τῆς Ἰνδίας μέρη σὺν ἰδίῳ στρατεύματι διατρίβοντα, 
ᾧ δὴ Χαρμπαντᾶς τοὔνομα· ὀρεοκόμον εἴπῃ τις ἂν ἐκεῖνον, 
οὕτω συμβὰν ἐπὶ τῇ γεννήσει, φανέντος εὐθὺς τοιούτου, ὡς 
εἴθιστο σφίσι γεννωμένοις ποιεῖν κατά τι νόμιμον. 

\end{greek}


%not really belongs here but dispersed by author

\section{Anthologia Graeca}
\blockquote[From Wikipedia\footnote{\url{http://en.wikipedia.org/wiki/Anthologia_Graeca}}]{The Greek Anthology (also called Anthologia Graeca) is a collection of poems, mostly epigrams, that span the classical and Byzantine periods of Greek literature. Most of the material of the Greek Anthology comes from two manuscripts, the Palatine Anthology of the 10th century and the Anthology of Planudes (or Planudean Anthology) of the 14th century.[1][2]}

\begin{greek}

Anthologia Graeca, Anthologia Graeca (7000: 001)
“Anthologia Graeca, 4 vols., 2nd edn.”, Ed. Beckby, H.
Munich: Heimeran, 1–2:1965; 3–4:1968.
Book 4, epigram 3, line 80

Ἀλλ' ἴθι νῦν ἀφύλακτος ὅλην ἤπειρον ὁδεύων, 
Αὐσόνιε, σκίρτησον, ὁδοιπόρε· Μασσαγέτην δὲ 
ἀμφιθέων ἀγκῶνα καὶ ἄξενα τέμπεα Σούσων 
Ἰνδῴης ἐπίβηθι κατ' ὀργάδος· ἐν δὲ κελεύθοις 
εἴ ποτε διψήσειας, ἀρύεο δοῦλον Ὑδάσπην. 



Anthologia Graeca, Anthologia Graeca 
Book 5, epigram 132, line 8

εἰ δ' Ὀπικὴ καὶ Φλῶρα καὶ οὐκ ᾄδουσα τὰ Σαπφοῦς, 
 καὶ Περσεὺς Ἰνδῆς ἠράσατ' Ἀνδρομέδης. 



Anthologia Graeca, Anthologia Graeca 
Book 5, epigram 251, line 1

ΕΙΡΗΝΑΙΟΥ ΡΕΦΕΡΕΝΔΑΡΙΟΥ


Ὄμματα δινεύεις κρυφίων ἰνδάλματα πυρσῶν, 
 χείλεα δ' ἀκροβαφῆ λοξὰ παρεκτανύεις, 
καὶ πολὺ κιχλίζουσα σοβεῖς εὐβόστρυχον αἴγλην, 
 ἐκχυμένας δ' ὁρόω τὰς σοβαρὰς παλάμας. 



Anthologia Graeca, Anthologia Graeca 
Book 5, epigram 270, line 5

μάργαρα σῆς χροιῆς ἀπολείπεται, οὐδὲ κομίζει 
 χρυσὸς ἀπεκτήτου σῆς τριχὸς ἀγλαΐην· 
Ἰνδῴη δ' ὑάκινθος ἔχει χάριν αἴθοπος αἴγλης, 
 ἀλλὰ τεῶν λογάδων πολλὸν ἀφαυροτέρην. 



Anthologia Graeca, Anthologia Graeca 
Book 6, epigram 261, line 1

ΚΡΙΝΑΓΟΡΟΥ


Χάλκεον ἀργυρέῳ με πανείκελον, Ἰνδικὸν ἔργον, 
 ὄλπην, ἡδίστου ξείνιον εἰς ἑτάρου, 
ἦμαρ ἐπεὶ τόδε σεῖο γενέθλιον, υἱὲ Σίμωνος, 
 πέμπει γηθομένῃ σὺν φρενὶ Κριναγόρης. 



Anthologia Graeca, Anthologia Graeca 
Book 7, epigram 153, line p1

ΟΜΗΡΟΥ, οἱ δὲ ΚΛΕΟΒΟΥΛΟΥ ΤΟΥ ΛΙΝΔ*ιΟΥ


Χαλκῆ παρθένος εἰμί, Μίδα δ' ἐπὶ σήματι κεῖμαι. 



Anthologia Graeca, Anthologia Graeca 
Book 9, epigram 524, line 10

ΑΔΕΣΠΟΤΟΝ


Μέλπωμεν βασιλῆα φιλεύιον, εἰραφιώτην, 
ἁβροκόμην, ἀγροῖκον, ἀοίδιμον, ἀγλαόμορφον, 
Βοιωτόν, βρόμιον, βακχεύτορα, βοτρυοχαίτην, 
γηθόσυνον, γονόεντα, γιγαντολέτην, γελόωντα, 
Διογενῆ, δίγονον, διθυραμβογενῆ, Διόνυσον, 
Εὔιον, εὐχαίτην, εὐάμπελον, ἐγρεσίκωμον, 
ζηλαῖον, ζάχολον, ζηλήμονα, ζηλοδοτῆρα, 
ἤπιον, ἡδυπότην, ἡδύθροον, ἠπεροπῆα, 
θυρσοφόρον, Θρήικα, θιασώτην, θυμολέοντα, 
Ἰνδολέτην, ἱμερτόν, ἰοπλόκον, ἰραφιώτην, 
κωμαστήν, κεραόν, κισσοστέφανον, κελαδεινόν, 
Λυδόν, ληναῖον, λαθικηδέα, λυσιμέριμνον, 
μύστην, μαινόλιον, μεθυδώτην, μυριόμορφον, 
νυκτέλιον, νόμιον, νεβρώδεα, νεβριδόπεπλον, 
ξυστοβόλον, ξυνόν, ξενοδώτην, ξανθοκάρηνον, 
ὀργίλον, ὀβριμόθυμον, ὀρέσκιον, οὐρεσιφοίτην, 
πουλυπότην, πλαγκτῆρα, πολυστέφανον, πολύκωμον, 
ῥηξίνοον, ῥαδινόν, ῥικνώδεα, ῥηνοφορῆα, 
σκιρτητήν, Σάτυρον, Σεμεληγενέτην, Σεμελῆα, 




Anthologia Graeca, Anthologia Graeca 
Book 9, epigram 544, line 1

ΦΙΛΙΠΠΟΥ


Θεσσαλίης εὔιππος ὁ ταυρελάτης χορὸς ἀνδρῶν, 
 χερσὶν ἀτευχήτοις θηρσὶν ὁπλιζόμενος, 
κεντροτυπεῖς πώλους ζεῦξεν σκιρτήματι ταύρων, 
 ἀμφιβαλεῖν σπεύδων πλέγμα μετωπίδιον· 
ἀκρότατον δ' ἐς γῆν κλίνας ἅμα κεὔροπον ἅμμα 
 θηρὸς τὴν τόσσην ἐξεκύλισε βίην.   
ΑΔΑΙΟΥ\}1 
Ἰνδὴν βήρυλλόν με Τρύφων ἀνέπεισε Γαλήνην 
 εἶναι, καὶ μαλακαῖς χερσὶν ἀνῆκε κόμας· 
ἠνίδε καὶ χείλη νοτερὴν λειοῦντα θάλασσαν, 
 καὶ μαστούς, τοῖσιν θέλγω ἀνηνεμίην. 




Anthologia Graeca, Anthologia Graeca 
Book 11, epigram 428, line 1

<ΤΟΥ ΑΥΤΟΥ>


Εἰς τί μάτην νίπτεις δέμας Ἰνδικόν; 



Anthologia Graeca, Anthologia Graeca 
Book 16, epigram 39, line 1

ΑΡΑΒΙΟΥ ΣΧΟΛΑΣΤΙΚΟΥ


Νεῖλος, Περσίς, Ἴβηρ, Σόλυμοι, Δύσις, Ἀρμενίς, Ἰνδοὶ 
 καὶ Κόλχοι σκοπέλων ἐγγύθι Καυκασίων 
καὶ πεδία ζείοντα πολυσπερέων Ἀγαρηνῶν 
 Λογγίνου ταχινῶν μάρτυρές εἰσι πόνων· 
ὡς δὲ ταχὺς βασιλῆι διάκτορος ἦεν ὁδεύων, 
 καὶ ταχὺς εἰρήνην ὤπασε κευθομένην. 



Anthologia Graeca, Anthologia Graeca 
Book 16, epigram 183, line 6

καὶ γὰρ ἐμοὶ πολέμων φίλιον κλέος· οἶδεν ἅπας μοι 
 ἠῴου δμηθεὶς Ἰνδὸς ἀπ' ὠκεανοῦ. 

\end{greek}


\section{Theophanes Continuatus}
\blockquote[From Wikipedia\footnote{\url{http://en.wikipedia.org/wiki/Theophanes_Continuatus}}]{Theophanes Continuatus (Greek: συνεχισταί Θεοφάνους) or Scriptores post Theophanem (Οἱ μετὰ Θεοφάνην, "those after Theophanes") is the Latin name commonly applied to a collection of historical writings preserved in the 11th-century Vat. gr. 167 manuscript.[1] Its name derives from its role as the continuation, covering the years 813–961, of the chronicle of Theophanes the Confessor, which reaches from 285 to 813. The manuscript consists of four distinct works, in style and form very unlike the annalistic approach of Theophanes.[2]

The first work, of four books consists of a series of biographies on the emperors reigning from 813 to 867 (from Leo the Armenian to Michael III). As they were commissioned by Emperor Constantine VII (r. 913–959), they reflect the point of view of the reigning Macedonian dynasty. The unknown author probably used the same sources as Genesios.[2] The second work is known as the Vita Basilii (Latin for "Life of Basil"), a biography of Basil I the Macedonian (r. 867–886) written by his grandson Constantine VII probably around 950. The work is essentially a panegyric, praising Basil and his reign while vilifying his predecessor, Michael III.[3] The third work is a history of the years 886–948, in form and content very close to the history of Symeon Logothetes, and the final section continues it until 961. It was probably written by Theodore Daphnopates, shortly before 963.[4]}
\begin{greek}
Theophanes Continuatus, Chronographia (lib. 1–6) (4153: 001)
“Theophanes Continuatus, Ioannes Cameniata, Symeon Magister, Georgius Monachus”, Ed. Bekker, I.
Bonn: Weber, 1838; Corpus scriptorum historiae Byzantinae.

Theophanes Continuatus, Chronographia (lib. 1-6) 
Page 55, line 6

                  ὅθεν τοῦ μὲν βουλεύματος οὐ διήμαρτεν τοῦ οἰ-  
κείου, ἀλλὰ καὶ στέφους μεταλαγχάνει καὶ αὐτοκράτωρ ἀναγο-
ρεύεται παρὰ τοῦ τὸν ἐν Ἀντιοχείᾳ θρόνον τηνικαῦτα μεταποιου-
μένου Ἰακώβ, καὶ χεῖρα συλλέγει πολλήν, μᾶλλον δὲ λαμβάνει 
πρὸς τὴν αὐτοῦ κραταίωσιν· οὐ γὰρ Ἀγαρηνῶν μόνον τούτων δὴ 
τῶν ἡμῖν γειτονούντων καὶ ὁμορούντων, ἀλλὰ καὶ αὐτῶν τῶν ἐν-
δότερον οἰκούντων, Αἰγυπτίων Ἰνδῶν Περσῶν Ἀσσυρίων Ἀρμε-
νίων Χάλδων Ἰβήρων Ζηχῶν Καβείρων καὶ πάντων τῶν δὴ Μά-
νεντος συστοιχούντων δόγμασι καὶ θεσπίσμασι. 



Theophanes Continuatus, Chronographia (lib. 1-6) 
Page 330, line 23

                        ἡ δὲ διείργουσα τὰ ἄδυτα τοῦ θείου 
οἴκου τούτου κιγκλίς, Ἡράκλεις, ὅσον ὄλβον ἐν ἑαυτῇ περιείλη-
φεν! ἧς οἱ στῦλοι μὲν καὶ τὰ κάτωθεν ἐξ ἀργύρου διόλου τὴν σύ-
στασιν ἔχουσιν, ἡ δὲ ταῖς κεφαλίσι τούτων ἐπικειμένη δοκὸς ἐκ 
καθαροῦ χρυσίου πᾶσα συνέστηκε, τὸν πλοῦτον πάντα τὸν ἐξ Ἰν-
δῶν περικεχυμένον πάντοθεν ἔχουσα· ἐν ᾗ κατὰ πολλὰ μέρη καὶ   
ἡ θεανδρικὴ τοῦ κυρίου μορφὴ μετὰ χυμεύσεως ἐκτετύπωται. 



%!what is this from?
                                                               καὶ 
ταῖς γὰρ πολυτελέσι καὶ πολυόψοις ἐκείναις τραπέζαις τὴν σύγ-
κλητον ἅπασαν δεξιούμενος χορηγίαις εὐεργετικωτέραις τὸ φαιδρὸν 
τῆς ἑορτῆς ἐπολλαπλασίαζεν, σηρικῶν περιβολαίων ἐπιδίδων, ἀρ-
γυρίων πολλῶν καὶ ἀπείρων, ἐσθημάτων ἁλουργῶν, ξύλων Ἰνδι-
κῶν εὐωδίας, ἃ οὔ τις ἀκήκοεν ἢ γεγονότα τεθέαται. 

\end{greek}

\section{Manuel Philes}
\blockquote[From Wikipedia\footnote{\url{http://en.wikipedia.org/wiki/Manuel_Philes}}]{Manuel Philes (c. 1275–1345), of Ephesus, Byzantine poet.

At an early age he removed to Constantinople, where he was the pupil of Georgius Pachymeres, in whose honour he composed a memorial poem. Philes appears to have travelled extensively, and his writings contain much information concerning the imperial court and distinguished Byzantines. Having offended one of the emperors by indiscreet remarks published in a chronography, he was thrown into prison and only released after an abject apology.

Philes is the counterpart of Theodorus Prodromus in the time of the Comneni; his character, as shown in his poems, is that of a begging poet, always pleading poverty, and ready to descend to the grossest flattery to obtain the favorable notice of the great. With one unimportant exception, his productions are in verse, the greater part in dodecasyllabic iambic trimeters, the remainder in the fifteen-syllable "political" measure.

Philes was the author of poems on a great variety of subjects: on the characteristics of animals, chiefly based upon Aelian and Oppian, a didactic poem of some 2000 lines, dedicated to Michael IX Palaiologos; on the elephant; on plants; a necrological poem, probably written on the death of one of the sons of the imperial house; a panegyric on John VI Kantakouzenos, in the form of a dialogue; a conversation between a man and his soul; on ecclesiastical subjects, such as church festivals, Christian beliefs, the saints and fathers of the church; on works of art, perhaps the most valuable of all his pieces for their bearing on Byzantine iconography, since the writer had before him the works he describes, and also the most successful from a literary point of view; occasional poems, many of which are simply begging letters in verse.}
\begin{greek}
Manuel Philes Poeta, Scr. Rerum Nat., Carmina (2718: 001)
“Manuelis Philae Carmina, vols. 1–2”, Ed. Miller, E.
Paris, 1855–1857, Repr. 1967.
Chapter 1, poem 213, line 63

Φρενῶν δὲ μεστὴν εὐτυχεῖς τὴν καρδίαν, 
Ἥρακλες Ἑρμῆ, καὶ σφριγᾷς πρὸς τὰς μάχας 
Καὶ τοῖς πονηροῖς ἐξ ἀπόπτου συμπλέκῃ 
Πρηστῆρσιν ὀργῆς τὴν καλὴν θήγων σπάθην, 
Καὶ πρὶν μὲν ἐλθεῖν εἰς βελῶν περιστάσεις, 
Ὁρᾷς ἱλαρὸν ὀπτικὸν χέων μέλι· 
Τομῶς δὲ χωρῶν εἰς τὸ πῦρ τοῦ κινδύνου,   
Τὸ βλέμμα γοργὸν καὶ φλογῶδες δεικνύεις 
Καθάπερ ὀξὺς καὶ πολύστροφος δράκων 
Ὅταν πρὸς ἐλέφαντας Ἰνδοὺς ἑρπύσῃ 
Καὶ τὸν μαχιμώτατον αὐτῶν ἁρπάσῃ. 




Manuel Philes Poeta, Scr. Rerum Nat., Carmina 
Chapter 2, poem 95, line 107

Καὶ τοῦτ' ἀπαρχὴν τῷ θεῷ δοὺς τοῦ κράτους, 
Εἶτα προχωρεῖς εἰς τὰ λοιπὰ τῆς τύχης, 
Καὶ τῶν μὲν ἐθνῶν τὰς παρατάξεις λύεις, 
Στρατευμάτων ἄπειρον ἰσχὺν συλλέγων· 
Καὶ τὸ ξίφος πρόκωπον εἰς πάντας φέρων, 
Οὓς εἶδεν ἐχθροὺς ὁ δρομεὺς τότε χρόνος, 
Τῷ δὲ κράτει σχοίνισμα τὰς πράξεις δίδως 
Αἷς βαρβάρων ᾕρηκας ἀρχισατράπας, 
Καὶ πᾶσαν ἁπλῶς δυσμενῶν ὁμαιχμίαν, 
Ὡς ἄχρις Ἰνδῶν καὶ Σκυθῶν καὶ Περσίδος 
Καὶ γῆς Ἰταλῶν καὶ Μυσῶν πολυσπόρων, 
Τὴν σὴν διελθεῖν εὐχερῶς κραταρχίαν· 
Ποῖος γὰρ οὐκ ἔγνω σε πορθμὸς δεσπότην; 



Manuel Philes Poeta, Scr. Rerum Nat., Carmina 
Chapter 2, poem 214, line 22

Γαλῆ δὲ μυὸς ἐξ ὀπῆς δεδραγμένη 
Τοῖς ὀστέοις βέβρυχε τοῦ θηράματος· 
Ὀξὺς δὲ κυνὸς καὶ πολύστροφος δρόμος 
Αἱρεῖ λαγωὸν εἰς φυγὴν ἠπειγμένον· 
Φεύγει δὲ τοὺς δράκοντας Ἰνδὸς ἐλέφας,   
Στυγῶν τὸν ὁλκὸν ὡς ταχὺν πρὸς ἀγχόνην, 
Τάχα δὲ καὶ πῦρ καὶ κριὸν καὶ δέλφακα, 
Καὶ μῦν κρεμαστὸν εἰς λινόπλοκον βρόχον. 



Manuel Philes Poeta, Scr. Rerum Nat., Carmina 
Chapter 3, poem 58, line 71

Εἰς τόξα πυκνὰ καὶ βελῶν περικλάσεις, 
Εἰς ἀγρίας φάραγγας, εἰς τραχεῖς τόπους, 
Εἰς ἄξυλον γῆν, εἰς περίξυλον λόφον, 
Εἰς ὑδάτων ἔρημον οὐχ ἅπαξ τρίβον, 
Εἰς θῆρας ὀξεῖς, εἰς ἀνίκμους σκορπίους, 
Εἰς νιφετούς τε καὶ κρυμοὺς ὀλεθρίους, 
Εἰς ἡλιακὰς ἐν μεσημβρίᾳ ζέσεις, 
Εἰς ἀκρατεῖς λαίλαπας, εἰς ἐπομβρίας, 
Εἰς Πέρσας, εἰς Ἄραβας, εἰς πτηνοὺς Σκύθας, 
Εἰς βαρβάρων δύναμιν, εἰς Ἰνδῶν θράσος, 
Τὰ παντοδαπὰ δυσχερῆ τῆς Περσίδος; 

\end{greek}



\section{Joannes Zonaras}
\blockquote[From Wikipedia\footnote{\url{http://en.wikipedia.org/wiki/Joannes_Zonaras}}]{

Ioannes (John) Zonaras (Greek: Ἰωάννης Ζωναρᾶς; fl. 12th century) was a Byzantine chronicler and theologian, who lived at Constantinople.

Under Emperor Alexios I Komnenos he held the offices of head justice and private secretary (protasēkrētis) to the emperor, but after Alexios' death, he retired to the monastery of St Glykeria, where he spent the rest of his life in writing books.

His most important work, Extracts of History (Greek: Ἐπιτομὴ Ἱστοριῶν, Latin: Epitome Historiarum), in eighteen books, extends from the creation of the world to the death of Alexius (1118). The earlier part is largely drawn from Josephus; for Roman history he chiefly followed Cassius Dio up to the early third century. Contemporary scholars are particularly interested in his account of the third and fourth centuries, which depend upon sources, now lost, whose nature is fiercely debated. Central to this debate is the work of Bruno Bleckmann, whose arguments tend to be supported by continental scholars but rejected in part by English-speaking scholars.[1] An English translation of these important sections has recently been published: Thomas Banchich and Eugene Lane, The History of Zonaras from Alexander Severus to the Death of Theodosius the Great (Routledge 2009). The chief original part of Zonaras' history is the section on the reign of Alexios Komnenos, whom he criticizes for the favour shown to members of his family, to whom Alexios entrusted vast estates and significant state offices. His history was continued by Nicetas Acominatus.}
\begin{greek}
Joannes Zonaras Gramm., Hist., Epitome historiarum (lib. 1–12) (3135: 001)
“Ioannis Zonarae epitome historiarum, 3 vols.”, Ed. Dindorf, L.
Leipzig: Teubner, 1:1868; 2:1869; 3:1870.
Volume 1, page 16, line 8

καὶ Φεισὼν μὲν ὄνομα τῷ ἑνί· πληθὺν δὲ τοῦτο δη-
λοῖ· τοῖς δ' Ἕλλησι Γάγγης οὗτος ὠνόμασται, τὴν 
Ἰνδικὴν διιὼν καὶ ἐκδιδοὺς εἰς τὸ πέλαγος. 



Joannes Zonaras Gramm., Hist., Epitome historiarum (lib. 1-12) 
Volume 1, page 24, line 14

Σὴμ δὲ τῷ υἱῷ Νῶε πέντε τίκτονται παῖδες, οἳ 
τὴν μέχρις ὠκεανοῦ τοῦ κατ' Ἰνδίαν οἰκοῦσιν Ἀσίαν, 
ἀπ' Εὐφράτου ἀρξάμενοι. 


Joannes Zonaras Gramm., Hist., Epitome historiarum (lib. 1-12) 
Volume 1, page 200, line 16

                                        τῇ τε γὰρ Ἰνδίᾳ 
προσέβαλε καὶ τὸν Πῶρον ἐνίκησε καὶ τὸν Ταξίλην 
ᾠκειώσατο καὶ ἄλλα μέρη τῆς Ἰνδικῆς κατέσχεν. 



Joannes Zonaras Gramm., Hist., Epitome historiarum (lib. 1-12) 
Volume 1, page 228, line 6

ταῦτα δὲ διανοηθεὶς τούς τε ὑφ' ἑαυτὸν ἡτοίμαζεν, 
ἔπεμψε δὲ καὶ πρὸς Κροῖσον τὸν βασιλέα Λυδῶν καὶ 
πρὸς ἄμφω τοὺς Φρύγας, πρὸς Παφλαγόνας τε καὶ 
Ἰνδοὺς καὶ πρὸς Κᾶρας καὶ Κίλικας, αἰτῶν συμμα-
χήσειν αὐτῷ κατὰ Μήδων ὡρμημένῳ, ὡς καὶ αὐτοῖς 
τοῦ πολέμου συμφέροντος, δυνατὸν εἶναι λέγων τὸ 
ἔθνος, ἐπιγαμίαν τε πρὸς Πέρσας πεποιημένον καὶ 
τὴν παρ' ἐκείνων προσκτήσασθαι ἀρωγήν, καὶ θάτε-
ρον συγκροτεῖσθαι παρὰ θατέρου, ὥστε εἰ μή τις αὐ-
τοὺς φθάσας ἀσθενώσει, ἑκάστῳ τῶν ἐθνῶν ἐπιόν-
τας κρατήσειν αὐτῶν. 



Joannes Zonaras Gramm., Hist., Epitome historiarum (lib. 1-12) 
Volume 1, page 228, line 31

ἐν τούτοις δὲ παρὰ Κυαξάρου ἧκεν ἄγγελος λέγων 
ὅτι Ἰνδῶν παρείη πρεσβεία, καί “δεῖ παρεῖναι καὶ 
σέ. 



Joannes Zonaras Gramm., Hist., Epitome historiarum (lib. 1-12) 
Volume 1, page 229, line 2

     φέρω δέ σοι καὶ στολὴν τὴν καλλίστην· βούλεται   
γάρ σε προσάγειν ἐστολισμένον λαμπρότατα, ἵν' οὕτω 
τοῖς Ἰνδοῖς ὀφθείης. 



Joannes Zonaras Gramm., Hist., Epitome historiarum (lib. 1-12) 
Volume 1, page 229, line 5

                   κληθέντες δὲ οἱ Ἰνδοὶ εἶπον ἐστάλ-
θαι παρὰ τοῦ σφετέρου βασιλέως ἐρωτῶντος ἐξ οὗ ὁ 
πόλεμος εἴη Μήδοις τε καὶ τῷ Ἀσσυρίῳ, τὰ αὐτὰ δὲ 
πυθέσθαι κἀκείνου, καὶ ἀμφοτέροις εἰπεῖν ὅτι ὁ Ἰν-
δῶν βασιλεὺς τὸ δίκαιον σκεψάμενος μετὰ τοῦ ἠδι-
κημένου ἔσται. 



Joannes Zonaras Gramm., Hist., Epitome historiarum (lib. 1-12) 
Volume 1, page 229, line 13

                     ὁ δὲ Κῦρος εἶπεν “εἰ παρ' ἡμῶν 
ἀδικεῖσθαί φησιν ὁ Ἀσσύριος, ὦ Ἰνδοί, αὐτὸν αἱ-
ρούμεθα δικαστὴν τὸν βασιλέα ὑμῶν. 



Joannes Zonaras Gramm., Hist., Epitome historiarum (lib. 1-12) 
Volume 1, page 245, line 3

Ἦλθον δὲ τῷ Κύρῳ παρὰ τοῦ Ἰνδοῦ χρήματα, 
καὶ οἱ ἄγοντες αὐτὰ ἀπήγγελλον αὐτῷ ὅτι ὁ Ἰνδὸς 
λέγει ὡς “ἥδομαι, ὦ Κῦρε, ὅτι μοι περὶ ὧν ἐδέου 
ἐδήλωσας, καὶ βούλομαί σοι ξένος εἶναι, καὶ πέμπω 
σοι χρήματα, κἂν ἄλλων δέῃ, μεταπέμπου· ἐντέταλται 
δὲ τοῖς παρ' ἐμοῦ ποιεῖν ἃ ἂν σὺ κελεύῃς. 



Joannes Zonaras Gramm., Hist., Epitome historiarum (lib. 1-12) 
Volume 1, page 245, line 12

                                                       ὁ δὲ 
Κῦρος “κελεύω τοίνυν” εἶπε “τοὺς μὲν ἄλλους μένον-
τας ἔνθα κατεσκηνώσατε φυλάττειν τὰ χρήματα, τρεῖς 
δέ μοι ἐλθόντες ὑμῶν εἰς τοὺς πολεμίους ὡς παρὰ 
τοῦ Ἰνδοῦ περὶ συμμαχίας, καὶ τὰ ἐκεῖ μαθόντες ὅ, 
τι ἂν λέγωσί τε καὶ ποιῶσιν ὡς τάχιστα ἀπαγγείλατε 
ἐμοί τε καὶ τῷ Ἰνδῷ. 



Joannes Zonaras Gramm., Hist., Epitome historiarum (lib. 1-12) 
Volume 1, page 245, line 14

                               οἱ μὲν δὴ Ἰνδοὶ συσκευασά-
μενοι τῇ ὑστεραίᾳ ἐπορεύοντο, ὁ δὲ Κῦρος τὰ πρὸς 
τὸν πόλεμον παρεσκευάζετο μεγαλοπρεπῶς. 



Joannes Zonaras Gramm., Hist., Epitome historiarum (lib. 1-12) 
Volume 1, page 245, line 24

                              οὕτω δὲ διατιθεμένων τῷ 
Κύρῳ τῶν τοῦ πολέμου ἧκον οἱ Ἰνδοὶ ἐκ τῶν πολε-
μίων καὶ ἔλεγον ὅτι Κροῖσος ἡγεμὼν καὶ στρατηγὸς 
ᾕρηται, καὶ πολλοὶ μὲν βασιλεῖς, πολλὰ δ' ἔθνη καὶ 
Ἕλληνες συμμαχήσειν ἡτοίμασται. 



Joannes Zonaras Gramm., Hist., Epitome historiarum (lib. 1-12) 
Volume 1, page 297, line 11

Μέλλων δὲ εἰς τὴν Ἰνδικὴν ἐμβάλλειν, συνε-
σκευασμένων τῶν ἁμαξῶν πρώταις μὲν ταῖς οἰκείαις 
ἐνῆκε πῦρ, εἶτα καὶ ταῖς τῶν φίλων, καὶ μετὰ ταῦτα 
καὶ τὰς τῶν Μακεδόνων καταπρῆσαι ἐκέλευσε. 



Joannes Zonaras Gramm., Hist., Epitome historiarum (lib. 1-12) 
Volume 1, page 298, line 7

Ὁ μέντοι Ταξίλης μοίρας ἄρχων τῆς Ἰνδικῆς παμ-
φόρου τε καὶ εὐδαίμονος, οὐκ ἀποδεούσης Αἰγύπτου, 
σοφὸς δὲ ὢν ἀνήρ, πέμψας ἠσπάσατο τὸν Ἀλέξαν-
δρον καί “τί δεῖ πολέμων ἡμῖν” ἔφη, “εἰ μήτε ὕδωρ 
ἀφαιρησόμενος ἡμῶν ἀφῖξαι μήτε τροφὴν ἀναγκαίαν; 



Joannes Zonaras Gramm., Hist., Epitome historiarum (lib. 1-12) 
Volume 1, page 298, line 19

                                          σπεισάμενος 
δέ τινι πόλει τῶν Ἰνδικῶν, ἀπιόντας ἐκεῖθεν τοὺς 
ἐν αὐτῇ μισθοφοροῦντας τῶν μαχιμωτάτων Ἰνδῶν 
ἀπέκτεινεν ἅπαντας· ὃ τοῖς αὐτοῦ πολεμικοῖς ἔργοις 
οἷά τις κηλὶς πρόσεστιν. 



Joannes Zonaras Gramm., Hist., Epitome historiarum (lib. 1-12) 
Volume 1, page 298, line 23

                              εἶτα πρὸς Πῶρον ἐμαχέ-
σατο, καὶ τοῦτον χώρας Ἰνδικῆς βασιλεύοντα, τὸ μέ-
γεθος τοῦ σώματος ἔχοντα εἰς τέσσαρας πήχεις ἀνα-
τρέχον καὶ σπιθαμήν. 



Joannes Zonaras Gramm., Hist., Epitome historiarum (lib. 1-12) 
Volume 1, page 299, line 17

                                ἐν δὲ Μαλλοῖς γεγονώς, 
μαχιμωτάτοις οὖσιν Ἰνδῶν, μικροῦ ἐκινδύνευσε. 



Joannes Zonaras Gramm., Hist., Epitome historiarum (lib. 1-12) 
Volume 1, page 300, line 28

            εἶτα ἀναστρέφων τὰς μὲν ναῦς παραπλεῖν 
ἐν δεξιᾷ τὴν Ἰνδικὴν ἐχούσας ἐκέλευσεν, αὐτὸς δὲ 
πεζῇ πορευόμενος εἰς ἐσχάτην ἀπορίαν κατήντησε καὶ   
πλῆθος τοσοῦτον ἀπώλεσεν ὥστε τῆς στρατιᾶς μηδὲ 
τὸ τέταρτον ἐκ τῆς Ἰνδικῆς ἀνακομισθῆναι διὰ νό-
σους καὶ πονηρὰς διαίτας καὶ καύματα καὶ λιμόν. 



Joannes Zonaras Gramm., Hist., Epitome historiarum (lib. 1-12) 
Volume 2, page 280, line 8

                                        καὶ μετὰ τοῦτο 
τὴν μισθοφορὰν τοῖς ἄλλοις δοὺς ἐπὶ τὸν Ἰνδίβιλιν 
καὶ ἐπὶ τὸν Μανδόνιον ἐστράτευσε. 



Joannes Zonaras Gramm., Hist., Epitome historiarum (lib. 1-12) 
Volume 2, page 442, line 14

         καὶ ὁ Αὔγουστος ἐθνῶν ἡγεμονίας τισὶ δεδω-
κὼς ἐπανῆλθεν εἰς Σάμον, κἀκεῖ καὶ αὖθις ἐχείμασε 
καὶ πολλὰ διῴκησεν· ἀφίκοντο γὰρ ἐνταῦθα πρες-
βεῖαι πλεῖσται· καὶ οἱ Ἰνδοὶ τότε φιλίαν ἐποιήσαντο, 
δῶρα πέμψαντες ἄλλα τε καὶ τίγρεις, πρῶτον τότε 
Ῥωμαίοις ὀφθείσας. 



Joannes Zonaras Gramm., Hist., Epitome historiarum (lib. 1-12) 
Volume 3, page 17, line 16

           ὁ δὲ τὸν τοῦ Ἀλεξάνδρου, ὡς ἔλεγε, θώ-
ρακα ἐνδυσάμενος καὶ ἐπ' αὐτῷ χλαμύδα σηρικὴν 
ἁλουργῆ πολὺ μὲν χρυσίον, πολλοὺς δὲ λίθους Ἰνδι-
κοὺς ἔχουσαν, καὶ ξίφος περιζωσάμενος καὶ ἀσπίδα 
λαβὼν δρυΐ τε στεφανωσάμενος, σπουδῇ καθάπερ 
ἐπὶ πολεμίαν εἰς τὴν πόλιν εἰσήλασε, παμπληθεῖς 
ἱππεῖς τε καὶ πεζοὺς ὡπλισμένους ἐπαγόμενος· καὶ 
ἄλλα δέ τινα τοιαῦτα ποιήσας καὶ ἑαυτὸν ἀποσεμνύ-
νας ἐν δημηγορίᾳ διὰ ταῦτα ἐς τὸν Δαρεῖον καὶ τὸν 
Ξέρξην ἀπέσκωπτεν, ὡς πολλαπλάσιον ἢ ἐκεῖνοι τῆς 
θαλάσσης μέτρον ζεύξας αὐτός. 



Joannes Zonaras Gramm., Hist., Epitome historiarum (lib. 1-12) 
Volume 3, page 69, line 24

                                         ἐνενόει δὲ καὶ 
Ἰνδούς, καὶ ἔλεγεν ὡς “εἰ νέος ἔτι ἦν, καὶ ἐπ' αὐ-
τοὺς ἂν ἐπεραιώθην. 



Joannes Zonaras Gramm., Hist., Epitome historiarum (lib. 13–18) (3135: 002)
“Ioannis Zonarae epitomae historiarum libri xviii, vol. 3”, Ed. Büttner–Wobst, T.
Bonn: Weber, 1897; Corpus scriptorum historiae Byzantinae.
Page 156, line 18

Ἐν δὲ τῇ στάσει ταύτῃ, ὡς εἴρηται, τῆς μεγάλης ἐκκλη-
σίας καυθείσης, ἧς δομήτωρ ἦν ὁ Κωνστάντιος, ἑτέραν πολλῷ 
μείζω καὶ περιφανεστέραν ὁ βασιλεὺς Ἰουστινιανὸς ἀπήρξατο 
καινουργεῖν, τῆς οἰκοδομῆς αὐτῆς ἀρχθείσης κατὰ τὸ ͵ϛμʹ ἔτος, 
ἰνδικτιῶνος πεντεκαιδεκάτης ἐνισταμένης ἐν Φευρουαρίῳ μηνί. 



Joannes Zonaras Gramm., Hist., Epitome historiarum (lib. 13-18) 
Page 172, line 6

               μοναχοὶ δὲ δύο τινὲς πρὸς τὸ Βυζάντιον ἐξ Ἰνδίας 
ἀφικόμενοι τὴν ταύτης γένεσιν ἀφηγήσαντο καὶ ὑπισχνοῦντο 
κομίσαι τῶν σκωλήκων ἐκείνων γόνον, ᾠὰ ὄντα τὸν ὄγκον 
βραχύτατα, καὶ δεῖξαι Ῥωμαίοις ὅπως ἐκεῖνα ζωογονοῦνται 
θαλπόμενα καὶ εἰς σκώληκας μεταμείβονται, καὶ ὅπως δημιουρ-
γοῦσι τὴν μέταξαν, τὴν φύσιν σχόντα διδάσκαλον. 



Joannes Zonaras Gramm., Hist., Epitome historiarum (lib. 13-18) 
Page 241, line 6

ἃ μαθὼν Ἰουστινιανὸς τοὺς μὲν τοῦ Ἡλία παῖδας ἐν τῷ κόλπῳ 
κατέσφαξε τῆς μητρός, ἐκείνην δὲ δούλῳ αὐτῆς Ἰνδῷ μαγείρῳ 
συνέζευξεν. 



Joannes Zonaras Gramm., Hist., Epitome historiarum (lib. 13-18) 
Page 634, line 11

ἐπεὶ δ' ἡ Περσῶν ἀρχὴ ἢ μᾶλλον ἡ Μακεδόνων, ἣ τὴν Περ-
σῶν βασιλείαν καθεῖλεν, ὑπὸ Σαρακηνῶν καθῄρητο, εἶτα καὶ 
οὗτοι πρὸς ἀλλήλους στασιάσαντες εἰς ἀντιπάλους μοίρας διῄρηντο 
καὶ ἀλλήλοις ἐμάχοντο, Μουχούμετ ὁ τοῦ Ἰμβραήλ, Περσίδος 
ἄρχων καὶ Χορασμίων καὶ Μηδίας καί τινων ἄλλων, πόλεμον 
ἤρατο κατὰ τοὺς χρόνους Βασιλείου τοῦ βασιλέως κατὰ Βα-
βυλωνίων τε καὶ Ἰνδῶν, ἡττώμενος δὲ συμμαχικὸν ἐκ Τούρκων 
μετεπέμψατο· ἦν δὲ τοῖς εἰς συμμαχίαν ἐλθοῦσι τῷ Μουχούμετ 
στρατηγὸς Ταγγρολίπιξ Μουκάλετ. 

\end{greek}



\section{\emph{Scholia In Clementem Alexandrinum}}
\blockquote[From Wikipedia\footnote{\url{http://en.wikipedia.org/wiki/Clement_of_Alexandria}}]{Titus Flavius Clemens (c.150 – c. 215), known as Clement of Alexandria, was a Christian theologian who taught at the Catechetical School of Alexandria. A convert to Christianity, he was an educated man who was familiar with classical Greek philosophy and literature. As his three major works demonstrate, Clement was influenced by Hellenistic philosophy to a greater extent than any other Christian thinker of his time, and in particular by Plato and the Stoics.[1] His secret works, which exist only in fragments, attest that he was also familiar with pre-Christian Jewish esotericism and Gnosticism. Among his pupils were Origen and Alexander of Jerusalem.

Clement is regarded as a Church Father, and he is venerated as a saint in Orthodox Christianity, Eastern Catholicism and Anglicanism. He was previously revered in the Roman Catholic Church, but his cult was suppressed in 1586 by Pope Sixtus V due to concerns about his orthodoxy.}
\begin{greek}

Scholia In Clementem Alexandrinum, Scholia in protrepticum et paedagogum (scholia recentiora partim sub auctore Aretha) (5048: 001)
“Clemens Alexandrinus, vol. 1, 3rd edn.”, Ed. Stählin, O., Treu, U.
Berlin: Akademie–Verlag, 1972; Die griechischen christlichen Schriftsteller 12.
Page 337, line 22

253, 21 ὄρνεις Ἰνδικοὺς] ψιττακούς φησι. 

\end{greek}



\section{Phalaridis Epistulae}%???
Who is this?
%\blockquote[From Wikipedia\footnote{\url{}}]{}
\begin{greek}

Phalaridis Epistulae, Epistulae (0053: 001)
“Epistolographi Graeci”, Ed. Hercher, R.
Paris: Didot, 1873, Repr. 1965.
Epistle 86, section 1, line 3

Πολλὰ λέγειν ἔχων καὶ κατὰ σοῦ καὶ περὶ ἧς κατ' 
ἐμοῦ πεφλυάρηκας ἐν Λεοντίνοις δημοκοπίας οὐδὲν   
ἐρῶ περισσότερον πλὴν ὅτι κώνωπος ἐλέφας Ἰνδὸς οὐκ 
ἀλεγίζει. 

\end{greek}


\section{Theodorus Scutariota}%???
Who is this?
%\blockquote[From Wikipedia\footnote{\url{}}]{}
\begin{greek}
Theodorus Scutariota Hist., Additamenta ad Georgii Acropolitae historiam (3157: 001)
“Georgii Acropolitae opera, vol. 1”, Ed. Heisenberg, A.
Leipzig: Teubner, 1903, Repr. 1978 (1st edn. corr. P. Wirth).
Fragment 33, line 46

                                    ἐν δὲ τῇ κατὰ Λυδίαν 
Μαγνησίᾳ, ὅπου καὶ τὰ πλείω τῶν χρημάτων ἀπέθετο, τί 
τίς ἂν ἐζήτησεν ἀφ' ὧν ἄνθρωποι χρῄζομεν, καὶ οὐχ εὑρὼν 
ἐκληρώσατο τὴν ἀπόλαυσιν, οὐ τῶν ἐν τοῖς ἡμετέροις τό-
ποις εὑρισκομένων ἀλλὰ καὶ ὅσα ἐνιαχοῦ τῆς οἰκουμένης, 
κατ' Αἴγυπτόν φημι καὶ Ἰνδίαν καὶ ἀλλαχοῦ; 
\end{greek}



\section{Nikephoros I of Constantinople}
\blockquote[From Wikipedia\footnote{\url{http://en.wikipedia.org/wiki/Nikephoros_I_of_Constantinople}}]{St. Nikephoros I or Nicephorus I (Greek: Νικηφόρος Α΄, Nikēphoros I ), (c. 758 – April 5, 828) was a Christian Byzantine writer and Ecumenical Patriarch of Constantinople from April 12, 806, to March 13, 815.}

\begin{greek}

Nicephorus I Scr. Eccl., Hist., Theol., Breviarium historicum de rebus gestis post imperium Mauricii (e cod. Vat. gr. 977) (3086: 001)
“Nicephori archiepiscopi Constantinopolitani opuscula historica”, Ed. de Boor, C.
Leipzig: Teubner, 1880, Repr. 1975.


Nicephorus I Scr. Eccl., Hist., Theol., Breviarium historicum de rebus gestis post imperium Mauricii (e cod. Vat. gr. 977) 
Page 46, line 15

τούτων αἰσθόμενος Ἰουστινιανὸς καὶ μείζονι θυμῷ ἐξαπτό-
μενος τὰ μὲν Ἡλία τέκνα τῷ μητρῴῳ κόλπῳ φερόμενα 
ἀναλίσκει, τὴν δὲ αὐτοῦ γυναῖκα τῷ ἰδίῳ μαγείρῳ ζευχθῆ-
ναι ἠνάγκασεν, Ἰνδῷ τῷ γένει καὶ ὅλῳ δυσειδεῖ τυγχάνοντι. 

Nicephorus I Scr. Eccl., Hist., Theol., Chronographia brevis [Dub.] (recensiones duae) (3086: 002)
“Nicephori archiepiscopi Constantinopolitani opuscula historica”, Ed. de Boor, C.
Leipzig: Teubner, 1880, Repr. 1975.
Page 98, line 20

Τῷ ζʹ ἔτει αὐτοῦ ἐπληρώθη κύκλος αʹ τοῦ ἁγίου πάσχα 
 ἐτῶν φλβʹ, ἐξότε ὁ κύριος ἡμῶν Ἰησοῦς Χριστὸς ἐσταυ-
 ρώθη 8[ἰνδ. ϛʹ] ἔτους ἀπὸ κτίσεως κόσμου ͵ϛξεʹ. 

Nicephorus I Scr. Eccl., Hist., Theol., Chronographia brevis [Dub.] (recensiones duae) 
Page 99, line 11

Ἔτους γʹ τῆς βασιλείας αὐτοῦ ὁ Πέρσης 8[Χοσρόης] 
 πλεῖστον μέρος τῆς Ῥωμαίων παρέλαβε πολιτείας καὶ 
 τὰ Ἱεροσόλυμα καὶ τοὺς σεβασμίους τόπους ἐνέπρησε 
 πλήθη τε λαῶν ᾐχμαλώτευσε σὺν τῷ πατριάρχῃ Ζαχαρίᾳ 
 καὶ τοῖς τιμίοις ξύλοις εἰς Περσίδα ἀπήγαγεν· ἔτει δὲ 
 αὐτοῦ ιβʹ Χοσρόης 8[ὁ Πέρσης] ἀνῃρέθη καὶ ἡ αἰχμα-
 λωσία ἀνεκλήθη, καὶ ὁ ζωοποιὸς σταυρὸς τοῖς ἰδίοις 
 τόποις ἀπεκατέστη 8[ἀνεγερθεῖσιν]. 
 Οἱ δὲ Σαρακηνοὶ ἤρξαντο τῆς τοῦ παντὸς ἐρημώσεως 
 τῷ ͵ϛρκϛʹ ἔτει ἰνδ. ζʹ. 


Nicephorus I Scr. Eccl., Hist., Theol., Chronographia brevis [Dub.] (recensiones duae) 
Page 102, line 22

Ἀπὸ δὲ Κωνσταντίνου ἕως Θεοφίλου ἰνδ. εʹ ἔτη φλʹ. 



Nicephorus I Scr. Eccl., Hist., Theol., Refutatio et eversio definitionis synodalis anni 815 (3086: 012)
“Nicephori Patriarchae Constantinopolitani Refutatio et Eversio Definitionis Synodalis Anni 815”, Ed. Featherstone, J.M.
Turnhout: Brepols, 1997; Corpus Christianorum, Series Graeca 33.
Chapter 2, line 23

λα περιεστολίζοντο· ὅτε θεῖος φόβος ταῖς ψυχαῖς τῶν εὐσε-
βούντων ἐνίδρυτο καὶ ἡ περὶ τὰ θεῖα αἰδὼς καὶ εὐλάβεια· ὅτε 
τὸ τῆς ἀγάπης δῶρον πανταχοῦ διαφοιτῶν περιηγγέλλετο· ὅτε 
τὰ τῆς ἱερωσύνης δίκαια κομῶντα συνδιεφυλάσσετο καὶ οἱ 
ἱερεῖς κυρίου δικαιοσύνην ἐνδεδύκεσαν, ὥσπερ στολὴν ἁγίαν 
περιχλαινιζόμενοι τὴν εὐσέβειαν, ἀμφιεννύμενοι δὲ κρῖμα ἶσα 
διπλοΐδι, καὶ ἡ εὐλογία κυρίου πᾶσιν ἐπέπρεπεν, εἰ δεῖ τι καὶ 
τῶν Ἰὼβ φθέγξασθαι ῥημάτων· ὅτε καὶ βασιλεῖς μέγα ἐφρό-
νουν ἐπ' εὐσεβείᾳ μᾶλλον ἢ τῷ διαδήματι καὶ χρυσῷ καὶ λίθοις 
τοῖς Ἰνδικοῖς λαμπρὸν ἀπαστράπτουσιν, εὐθύτητι δὲ δογμάτων 
ἢ τῷ ἁλουργῷ ἐκαλλωπίζοντο χρώματι καὶ ὅσον <***> 
τρόπαια στήσωσι. 

\end{greek}


\section{Josephus Genesius}
\blockquote[From Wikipedia\footnote{\url{http://en.wikipedia.org/wiki/Joseph_Genesius}}]{

Genesius (Greek: Γενἐσιος, Genesios) is the conventional name given to the anonymous Greek author of the tenth century chronicle, On the reign of the emperors. His first name is sometimes given as Joseph, combining him with a "Joseph Genesius" quoted in the preamble to John Skylitzes. Traditionally, he has been regarded as the son or grandson of Constantine Maniakes.

Composed at the court of Constantine VII, the chronicle opens in 814, covers the Second Iconoclast period and ends in 886. It presents the events largely from the view of the Macedonian dynasty, though with a skew less marked than the authors of Theophanes Continuatus, a collection of mostly anonymous chronicles meant to continue the work of Theophanes the Confessor.

The chronicle describes the reigns of the four emperors from Leo V down to Michael III in detail; and more briefly that of Basil I. It uses Constantine VII's Life of Basil as a source, though it appears to have been finished before Theophanes Continuatus, and gives information present in neither Continuatus nor Skylitzes.
Modern editions

English

    Genesios, Joseph, A. Kaldellis. (trans.) On the reigns of the emperors. Byzantina Australiensia, 11. Canberra: Australian Association for Byzantine Studies, 1998. ISBN 0-9593626-9-X.

Greek

    A. Lesmüller-Werner, and H. Thurn, Corpus Fontium Historiae Byzantinae, Vol. XIV, Series Berolinensis. Berlin: De Gruyter, 1973. ISSN 0589-8048.

}
\begin{greek}

Josephus Genesius Hist., Βασιλεῖαι (3040: 001)
“Iosephi Genesii regum libri quattuor”, Ed. Lesmüller–Werner, A., Thurn, J.
Berlin: De Gruyter, 1978; Corpus fontium historiae Byzantinae 14. Series Berolinensis.



Josephus Genesius Hist., Βασιλεῖαι 
Book 2, section 2, line 33

              ποιεῖται τοίνυν σπονδὰς μετ' Ἀγαρηνῶν, εἰδήσει τοῦ 
αὐτῶν ἀρχηγοῦ ἀναδεῖται στέφος βασίλειον παρὰ τοῦ ἀρχιερέως 
Ἀντιοχείας Ἰώβ, εἶτα μετ' Ἀγαρηνῶν Ἰνδῶν Αἰγυπτίων Ἀσσυρίων 
Μήδων Ἀβασίων Ζηχῶν Ἰβήρων Καβείρων Σκλάβων Οὔννων Βανδή-
λων Γετῶν καὶ ὅσοι τῆς Μάνεντος βδελυρίας μετεῖχον, Λαζῶν τε καὶ 
Ἀλανῶν Χάλδων τε καὶ Ἀρμενίων καὶ ἑτέρων παντοίων ἐθνῶν πολυθρύλ-
λητον πανστρατιὰν στρατοπεδευσάμενος ἁπάσης τῆς ἀνατολῆς ἐκυρί-
ευσεν, τελευταῖον μέρεσι τοῖς κατὰ Θρᾴκην προσεμπελάσας ἑλεπολεῖν 
τὸ Βυζάντιον ἐκβιάζεται ἱππεῦσιν εὐόπλοις καὶ πετροβολισταῖς τοῖς 
ὑπὸ χεῖρα πεζοῖς, ἔτι καὶ σφενδονισταῖς γε καὶ πελτασταῖς ἀμέτροις 
ἐπιρρωννύμενος, προσέτι μὴν καὶ πολιορκητικοῖς οὐκ ὀλίγοις τεχνάσμασι 
κρατυνόμενος. 


Josephus Genesius Hist., Βασιλεῖαι 
Book 2, section 5, line 47

                  μετ' οὐ πολὺ δὲ καὶ αὐτὸς σὺν πʹ χιλιάσιν ἐφίσταται 
τῇ πόλει, ἐπὶ δὲ τῷ προτέρως ποιηθέντι υἱῷ καὶ ἕτερον ἀναδείκνυσιν, 
Ἀναστάσιον ὄνομα, πάλαι μὲν εἰς τοὺς καλουμένους τελέσαντα μοναχούς, 
εἰς κοσμικῶν δὲ τρόπων φαυλότητι τάξιν ἐληλυθότα, αἰσχρὸν τὸ εἶδος, 
ὥστε δοκεῖν ἐξ οἰνοποσίας ἰνδογενὴς εἶναι, μοχθηρότερον τῇ ψυχῇ 
ὑπὸ ἐμπληξίας ἐσχάτης. 

\end{greek}



\section{Nicephorus Gregoras}
\blockquote[From Wikipedia\footnote{\url{http://en.wikipedia.org/wiki/Nicephorus_Gregoras}}]{Nikephoros Gregoras, Latinized as Nicephorus Gregoras (Greek: Νικηφόρος Γρηγορᾶς; c. 1295-1360), Byzantine astronomer, historian, man of learning and religious controversialist, was born at Heraclea Pontica.}

\begin{greek}

Nicephorus Gregoras Hist., Historia Romana (4145: 001)
“Nicephori Gregorae historiae Byzantinae, 3 vols.”, Ed. Schopen, L., Bekker, I.
Bonn: Weber, 1:1829; 2:1830; 3:1855; Corpus scriptorum historiae Byzantinae.
Volume 1, page 9, line 22

                             οὐ μὴν ἀλλ' ἔσθ' ὅτε καὶ δι' ἀμαθίαν 
τοῦ βελτίονος καὶ ἀπειρίαν πραγμάτων ἅπερ ὁτουοῦν ἠκηκόει-
σαν, πρὶν βασανίσαι, εἰ τὰ μὲν τῶν εἰκότων τάδ' ἥκιστα, καὶ 
τὰ μὲν ἔοικεν ἀληθείας οἴκοις ἐνδιαιτᾶσθαι, τάδ' ὑπερόριον ἀλη-
θείας τείνουσι γλῶσσαν, οὕτω ταῦτ' ἐφαπλοῦσι ταῖς ἑαυτῶν 
συγγραφαῖς καὶ τῷ χρόνῳ, αἰτιώμενοί τε τὰ ἀναίτια, καὶ φά-
σκοντες ἃ μήτ' ἐγένοντο, μήτε γενέσθαι τῶν δυνατῶν ἦν· οἵας 
τοῦ Πλάτωνος τὰς ἰδέας ἀκούομεν, καὶ ὅσοι τοὺς τραγελάφους 
ἐκ τῶν τῆς Ἰνδίας τεράτων ἐς τὰς τῆς Ἀσίας διαβιβάζουσιν ἀκοὰς, 
ἐκ μὴ ὄντων αὖθις μὴ ὄντα καθιστῶντες, ἵνα μᾶλλον ἐκπλήττω-
σι τοὺς ἀκούοντας. 



Nicephorus Gregoras Hist., Historia Romana 
Volume 1, page 38, line 6

                 (Δ.) Ἀλλὰ γὰρ ἦρος ἐπιγενομένου, ὅτε πᾶν τὸ 
πρόσωπον τῆς γῆς τὴν χλόην τῆς πόας ἐνδύεται, τὰ παρὰ τοὺς 
πρόποδας τῶν ὀρῶν χειμάδια καταλιπόντες οἱ Σκύθαι, καθάπερ 
αἰπόλια καὶ βουκόλια, κατὰ πλῆθος τὰς κορυφὰς τῶν ὀρῶν 
ὑπερβάλλουσι, ῥέουσί τε κατὰ τῶν ὑποκειμένων ἐθνῶν καὶ πάν-
τας ἐν λόγῳ λείας ποιούμενοι, καταντῶσιν ἐς Ἰνδικὴν, ὁπόση 
ἐφ' ἑκάτερα κεῖται τοῦ μεγίστου τῶν ποταμῶν Ἰνδοῦ. 



Nicephorus Gregoras Hist., Historia Romana 
Volume 1, page 107, line 5

                               οἱ γὰρ κατ' Αἴγυπτον Ἄραβες 
πλείστην προσειληφότες δύναμιν διὰ τοῦ Σκυθικοῦ στρατεύμα-  
τος ἐκείνου, καθάπερ ἔφθημεν εἰρηκότες, πλεῖστον ὅσον μάλα ἐξῆν 
τοὺς οἰκείους παρέδραμον ὅρους· πρὸς μὲν ἑσπέραν Λιβύας καὶ 
ὅσα Μαυρουσίων ἔθνη δουλωσάμενοι· πρὸς δ' ἀνατέλλοντα 
ἥλιον ἔνθεν μὲν Ἀραβίαν εὐδαίμονα πᾶσαν ὅσην τά τε ἄκρα 
τῶν Ἰνδικῶν ὁρίζει θαλασσῶν καὶ ἑκατέρωθεν ὅ, τε Περσικὸς καὶ 
ὁ Ἀραβικὸς τειχίζουσι κόλποι· ἔνθεν δὲ τήν τε Κοίλην Συρίαν 
καὶ τὴν Φοινίκην πᾶσαν, ὅσην ὁ ποταμὸς Ὀρόντης ἔνδον ποιεῖ-
ται, τοὺς τῶν Κελτογαλατῶν ἐκείνων ἐκγόνους τοὺς μὲν ἐκεῖθεν 
ἀποβήσαντες, τοὺς δ' ἐς ὄλεθρον, οἷον πολέμιος ὑποτίθεται νό-
μος, ὀλίγου παραπέμψαντες χρόνου. 


Nicephorus Gregoras Hist., Historia Romana 
Volume 1, page 188, line 21

             τοιοῦτον μέντοι καὶ τῶν ἐν Ἰνδοῖς σοφιστῶν ἡ παραί-
νεσις ὑφηγεῖται τὸν ἄρχειν βουλόμενον· οὕτω γὰρ ἂν τὰ μάλιστά 
φησι φιληθείη τοῖς ὑπ' αὐτὸν, ἂν φύσει τούτων ὑπέρτερος ὢν,   
ὁ δ' ἔπειτ' ἐπιεικὴς ἑκών γε εἶναί σφισιν ὁρᾶται καὶ μέτριος. 



Nicephorus Gregoras Hist., Historia Romana 
Volume 1, page 332, line 23

                                       πείθομαι γὰρ μηδ' ἂν τοῖς 
ἐπ' ἔσχατα γῆς Κελτοῖς, οὐδ' ἂν οὐδέσιν ὅσοι πρὸς τῷ ὠκεανῷ 
τυγχάνουσιν ὄντες, οὐ μέντοι οὐδ' ἂν ἐνδεῖν σου τῆς φήμης, 
οὐδ' αὐτοῖς Ἰνδοῖς· ἀλλὰ κἀκεῖθεν τὸ κῦρός σε δέχεσθαι τοῦ 
νικᾷν περιουσίᾳ φρονήσεως πάντα ἀνθρώπων γένη. 



Nicephorus Gregoras Hist., Historia Romana 
Volume 1, page 369, line 13

Ἀλεξανδρεῖς μὲν γὰρ πρὸ τριῶν ἡμερῶν τῆς πρώτης τοῦ καθ' 
ἡμᾶς σεπτεμβρίου τὴν ἀρχὴν τοῦ σφετέρου τίθενται ἔτους· Αἰγύ-
πτιοι δὲ νῦν μὲν αὐτὴν, νῦν δ' ἑτέραν, καὶ ἄλλοτ' ἄλλην ἀεί· 
καὶ Πέρσαι δὲ καὶ Μῆδοι καὶ Ἰνδοὶ τούτοις τε πᾶσι καί σφισιν 
αὐτοῖς ἀλλήλοις ἀσύμφωνα. 




Nicephorus Gregoras Hist., Historia Romana 
Volume 2, page 807, line 24

            ᾧ δ' ἑκατέροις τοῖς βίοις ἐν πείρᾳ γενέσθαι τετύχη-
κεν, ἄρχεσθαι μᾶλλον οὗτος ἢ βασιλεύειν ἕλοιτ' ἂν οἶμαι μάλα 
προθύμως, καὶ πένητα μᾶλλον τρίβειν βίον, ἢ ὃς μυρίαις περι-
στοιχίζεται δόξαις καὶ χρήμασιν· εἰ δ' οὖν, λεγέτω τις παρελθὼν, 
πῶς Ἀλέξανδρος ἐκεῖνος ὁ μέγας, ὁ μέχρις Ἰνδῶν τὰ τῆς Εὐρώ-  
πης διαβιβάσας ὅπλα, τὸν εὐτελῆ τοῦ Διογένους πίθον καὶ τὴν 
διεῤῥωγυῖαν ἀμπέχεσθαι μᾶλλον ἐσθῆτα ποθεῖν ὡμολόγει, ἢ τὴν 
τῆς ὅλης Ἀσίας τε καὶ Εὐρώπης ἔχειν ἀρχὴν, καὶ τὸν Βαβυλώ-
νιον ἐκεῖνον περιβεβλῆσθαι πλοῦτον· ὅθεν καὶ τὸ πολὺ τῆς ῥᾳ-
στώνης χεθὲν Δαρείῳ καὶ Πέρσαις ὄνειρον ἔδειξεν εἶναι σαφῶς 
τὰς δοκούσας εὐδαίμονας τύχας τοῦ βίου. 





Nicephorus Gregoras Hist., Historia Romana 
Volume 3, page 19, line 3

             ἐξ ὅτου γὰρ τὸ Σκυθικὸν ἐπιρρεῦσαν γένος καὶ   
ἐκχυθὲν ἄνωθέν ποθεν ἐξ ἀρκτικῶν πηγῶν καθάπερ ἀχανοῦς 
τινὸς ὕδωρ πελάγους, Πέρσας τε ἐδουλώσατο καὶ Μήδους καὶ 
πᾶσαν εἰπεῖν ταυτηνὶ τὴν Ἀσίαν ἄχρις Ἰνδῶν τε ἐκείνων πρὸς 
ἕω καὶ ἄχρις Ἀράβων τουτωνὶ πρὸς νότον, οὐ μόνον τὰ πλείω 
τῶν ἐγχωρίων ἠθῶν ἐκείνων ἔσβη καὶ ἐτεθνήκει, ἀλλὰ καὶ 
αὐτὰ τῶν ἐθνῶν ἐκείνων τὰ τοπικὰ διαστήματά τε καὶ ὅρια 
συγκέχυται καὶ παντάπασίν ἐστι δυσείκαστα νῦν. 




Nicephorus Gregoras Hist., Historia Romana 
Volume 3, page 354, line 8

                     οὔτε γὰρ οὐδὲν πρὸς ἔπος ἔοικε λέγειν, 
ἀλλ' ἄλλην τρέχων ἄλλην ἐβάδισεν, οὔτε εἰ ἀντιλέγειν ἐβού-
λετο, τὰς ὁμόσε προσηκούσας ἀντιθέσεις ἐπήνεγκεν, ἀλλ' ὅμοιον 
ποιεῖ ὥσπερ ἂν εἰ τὴν πρὸς ἕω τῶν τε Αἰθιόπων καὶ Ἰνδῶν 
εἰπεῖν ἀπαιτούμενος οἴκησιν, ὃ δ' ἐκ διαμέτρου τοὺς ἑσπε-
ρίους ἐπειρᾶτο δεικνύειν Κελτούς, καὶ ὅποι τὰ Βρετανῶν 
προσοικοῦσιν ἔθνη. 
\end{greek}

\section{Anthologiae Graecae Appendix}%???
\blockquote[From Wikipedia\footnote{\url{http://en.wikipedia.org/wiki/Greek_Anthology}}]{The Greek Anthology (also called Anthologia Graeca) is a collection of poems, mostly epigrams, that span the classical and Byzantine periods of Greek literature. Most of the material of the Greek Anthology comes from two manuscripts, the Palatine Anthology of the 10th century and the Anthology of Planudes (or Planudean Anthology) of the 14th century.[1][2]}

%is the appendix different than the Anthology?
\begin{greek}

Anthologiae Graecae Appendix, Epigrammata sepulcralia (7052: 002)
“Epigrammatum anthologia Palatina cum Planudeis et appendice nova, vol. 3”, Ed. Cougny, E.
Paris: Didot, 1890.

Anthologiae Graecae Appendix, Epigrammata sepulcralia 
Epigram 402, line 7

Ἀλλὰ σὺ, Γαῖα, πέλοις ἀγαθὴ κούφη τ' Ἀκυλίνῳ, 
 καὶ δὲ παρὰ πλευρὰς ἄνθεα λαρὰ φύοις, 
ὅσσα κατ' Ἀραβίους τε φέρεις, ὅσα τ' ἐστι κατ' Ἰνδοὺς, 
 ὡς ἂν ἀπ' εὐόδμου χρωτὸς ἰοῦσα δρόσος 
ἀγγέλλῃ τὸν παῖδα θεοῖς φίλον ἔνδοθι κεῖσθαι, 
 λοιβῆς καὶ θυέων ἄξιον, οὐχὶ γόων. 

Anthologiae Graecae Appendix, Epigrammata demonstrativa (7052: 003)
“Epigrammatum anthologia Palatina cum Planudeis et appendice nova, vol. 3”, Ed. Cougny, E.
Paris: Didot, 1890.
Epigram 55, line 3

  ................ 
Ὑσμίνην δεδάηκας ἀμετροβίων ἐλεφάντων· 
 Ἰνδοφόρων κρατεροὺς οὐ τρομέεις πολέμους. 



Anthologiae Graecae Appendix, Epigrammata demonstrativa 
Epigram 76, line 1

Ἀμφιλόχου τοῦ Λάγου Ποντωρέως. 
Ἥκει καὶ Νείλου προχοὰς καὶ ἐπ' ἔσχατον Ἰνδὸν 
τέχνας Ἀμφιλόχοιο μέγα κλέος ἄφθιτον ἀεί. 


Anthologiae Graecae Appendix, Epigrammata exhortatoria et supplicatoria (7052: 004)
“Epigrammatum anthologia Palatina cum Planudeis et appendice nova, vol. 3”, Ed. Cougny, E.
Paris: Didot, 1890.
Epigram 77, line 19

Ἄπελθε τοίνυν εἰς τόπους τῆς Ἰνδίας, 
εἴς τ' Ἀγησύμβων, εἴς τε Βλεμμύων πόλεις, 
ὅπου λέγουσιν ἀμπέλους μὴ βλαστάνειν· 
ἐκεῖσε δεῖξον σὴν ἰατρικὴν, σοφέ. 



Anthologiae Graecae Appendix, Oracula (7052: 006)
“Epigrammatum anthologia Palatina cum Planudeis et appendice nova, vol. 3”, Ed. Cougny, E.
Paris: Didot, 1890.
Epigram 313, line 3

Ἐς δίνας Ἴστροιο διιπετέος ποταμοῖο 
ἐσβαλέειν κέλομαι δοίους Κυβέλης θεράποντας, 
θῆρας ὀρειτρεφέας, καὶ ὅσα τρέφει Ἰνδικὸς ἀὴρ 
ἄνθεα καὶ βοτάνας εὐώδεας· αὐτίκα δ' ἔσται 
νίκη καὶ μέγα κῦδος ἅμ' εἰρήνῃ ἐρατεινῇ. 


\end{greek}



\section{Nicetas Choniates}
\blockquote[From Wikipedia\footnote{\url{http://en.wikipedia.org/wiki/Nicetas_Choniates}}]{Niketas or Nicetas Choniates (Νικήτας Χωνιάτης, ca. 1155 to 1215 or 1216), sometimes called Acominatos, was a Greek historian – like his brother Michael Acominatus, whom he accompanied from their birthplace Chonae to Constantinople. Nicetas wrote a history of the Eastern Roman Empire from 1118 to 1207.}

\begin{greek}
Nicetas Choniates Hist., Scr. Eccl., Rhet., Historia (= Χρονικὴ διήγησις) (3094: 001)
“Nicetae Choniatae historia, pars prior”, Ed. van Dieten, J.
Berlin: De Gruyter, 1975; Corpus fontium historiae Byzantinae 11.1. Series Berolinensis.
Reign Alex2, page 243, line of page 4

Ἐπεὶ δὲ ἡ κυρία τῆς ἀνόδου ἐνειστήκει, ὁπόσοι τῶν ἐν τέλει καὶ ὅσοι 
τοῦ βήματος τοῦ καλοῦ ἔτρεφον ἔρωτα καὶ ὁ τῆς πόλεως ἅπας δῆμος εἰς 
τὸ ἱερὸν φροντιστήριον συνδραμόντες λαμπροτάτην ἐκείνῳ συντελοῦσι 
τὴν πρόοδον μύροις τὰς ἀγυιὰς τέγγοντες καὶ τοῖς Ἰνδικοῖς ξύλοις καὶ 
καρυκευτοῖς ἀρώμασι τὸν ἀέρα εὐωδιάζοντες. 

\end{greek}



\section{Pseudo-Codinus}
\blockquote[From Wikipedia\footnote{\url{http://en.wikipedia.org/wiki/Pseudo-Codinus}}]{
Jump to: navigation, search

George Kodinos or Codinus (Greek: Γεώργιος Κωδινός), also Pseudo-Kodinos, kouropalates in the Byzantine court, is the reputed 14th-century author of three extant works in late Byzantine literature.

Their attribution to him is merely a matter of convenience, two of them being anonymous in the manuscripts. Οf Kodinos himself nothing is known; it is supposed that he lived towards the end of the 15th century. The works referred to are the following:

    Patria (Πάτρια Κωνσταντινουπόλεως), treating of the history, topography, and monuments of Constantinople. It is divided into five sections: (a) the foundation of the city; (b) its situation, limits and topography; (c) its statues, works of art, and other notable sights; (d) its buildings; (e) and the construction of the Hagia Sophia. It was written in the reign of Basil II (976-1025), revised and rearranged under Alexios I Komnenos (1081–1118), and perhaps copied by Codinus, whose name it bears in some (later) manuscripts. The chief sources are: the Patria of Hesychius Illustrius of Miletus, the anonymous Parastaseis syntomoi chronikai, and an anonymous account (ἔκφρασις) of St Sophia (ed. Theodor Preger in Scriptores originum Constantinopolitanarum, fasc. i, 1901, followed by the Patria of Codinus). Procopius, De Aedificiis and the poem of Paulus Silentiarius on the dedication of St. Sophia should be read in connexion with this subject.
    De Officiis (Τακτικόν περί των οφφικίων του Παλατίου Kωνσταντινουπόλεως και των οφφικίων της Μεγάλης Εκκλησίας), a treatise, written in an unattractive style between 1347 and 1368, of the court and higher ecclesiastical dignities and of the ceremonies proper to different occasions, as they had evolved by the middle Palaiologan period. It should be compared with the earlier De Ceremoniis of Constantine Porphyrogenitus and other Taktika of the 9th and 10th centuries.
    A chronological outline of events from the beginning of the world to the taking of Constantinople by the Turks (called Agarenes in the manuscript title). It is of little value.

Complete editions are (by Immanuel Bekker) in the Bonn Corpus scriptorum Hist. Byz. (1839–1843, where, however, some sections of the Patria are omitted), and in JP Migne, Patrologia graeca civil.; see also Karl Krumbacher, Geschichte der byzantinischen Litteratur (1897).}
\begin{greek}

Pseudo-Codinus Hist., De officiis (= officia palatii Constantinopoleos) (3168: 001)
“Pseudo–Kodinos. Traité des offices”, Ed. Verpeaux, J.
Paris: Centre National de la Recherche Scientifique, 1966; Le monde byzantin 1.



Pseudo-Codinus Hist., De annis ab orbe condito (3168: 004)
“Die byzantinischen Kleinchroniken, vol. 1”, Ed. Schreiner, P.
Vienna: Österreichische Akademie der Wissenschaften, 1975; Corpus fontium historiae Byzantinae 12.1. Series Vindobonensis.



Pseudo-Codinus Hist., Patria Constantinopoleos 
Book 2a, section 1, line 7

        Ἡ πρώτη σύνοδος γέγονεν ἐν τῇ Νικαίᾳ τῆς Βιθυ-
νίας ὑπὸ τοῦ μεγάλου Κωνσταντίνου συνελθόντων τῶν τρια-
κοσίων δέκα καὶ ὀκτὼ ἁγίων πατέρων καὶ Σιλβέστρου πάπα 
Ῥώμης· οἳ καὶ καθεῖλαν Ἄρειον, ὅστις ἦν πρῶτος πρεσβύ-
τερος Ἀλεξανδρείας· τὸν γὰρ κύριον ἡμῶν Ἰησοῦν Χριστὸν 
ψιλὸν ἄνθρωπον ἔλεγεν εἶναι· Ἴβηρες δὲ καὶ Ἰνδοὶ τότε 
ἐχριστιάνισαν. 


Pseudo-Codinus Hist., Patria Constantinopoleos 
Book 3, section 89, line 2

(c255, m262) Ἐν τοῖς χρόνοις τοῦ μεγάλου Θεοδο-
δοσίου ἤχθη ἐλέφας μικρὸς ἀπὸ Ἰνδίας καὶ ἐκεῖσε ἔτρεφον 
αὐτὸν εἰς τὰ οἰκήματα· καὶ ἱππικοῦ γενομένου ἔφερον 
αὐτόν. 


\end{greek}


\section{Chronicon Paschale}
\blockquote[From Wikipedia\footnote{\url{http://en.wikipedia.org/wiki/Chronicon_Paschale}}]{Chronicon Paschale ("the Paschal Chronicle, also Chronicum Alexandrinum or Constantinopolitanum, or Fasti Siculi ) is the conventional name of a 7th-century Greek Christian chronicle of the world. Its name comes from its system of chronology based on the Christian paschal cycle; its Greek author named it "Epitome of the ages from Adam the first man to the 20th year of the reign of the most August Heraclius."

The Chronicon Paschale follows earlier chronicles. For the years 600 to 627 the author writes as a contemporary historian - that is, through the last years of emperor Maurice, the reign of Phocas, and the first seventeen years of the reign of Heraclius.

Like many chroniclers, the author of this popular account relates anecdotes, physical descriptions of the chief personages (which at times are careful portraits), extraordinary events such as earthquakes and the appearance of comets, and links Church history with a supposed Biblical chronology. Sempronius Asellio points out the difference in the public appeal and style of composition which distinguished the chroniclers (Annales) from the historians (Historia) of the Eastern Roman Empire.

The "Chronicon Paschale" is a huge compilation, attempting a chronological list of events from the creation of Adam. The principal manuscript, the 10th-century Codex Vaticanus græcus 1941, is damaged at the beginning and end and stops short at AD 627. The Chronicle proper is preceded by an introduction containing reflections on Christian chronology and on the calculation of the Paschal (Easter) cycle. The so-called 'Byzantine' or 'Roman' era (which continued in use in Greek Orthodox Christianity until the end of Turkish rule as the 'Julian calendar') was adopted in the Chronicum as the foundation of chronology; in accordance with which the date of the creation is given as the 21st of March, 5507.

The author identifies himself as a contemporary of the Emperor Heraclius (610-641), and was possibly a cleric attached to the suite of the œcumenical Patriarch Sergius. The work was probably written during the last ten years of the reign of Heraclius.

The chief authorities used were: Sextus Julius Africanus; the consular Fasti; the Chronicle and Church History of Eusebius; John Malalas; the Acta Martyrum; the treatise of Epiphanius, bishop of Constantia (the old Salamis) in Cyprus (fl. 4th century), on Weights and Measures.}
\begin{greek}

Andromachus Poet. Med., Fragmentum (0280: 001)
“Die griechischen Dichterfragmente der römischen Kaiserzeit, vol. 2”, Ed. Heitsch, E.
Göttingen: Vandenhoeck \& Ruprecht, 1964.
Line 133

ἢ ἔτι καὶ σμύρνης καὶ εὐόδμου κόστοιο 
 καὶ κρόκου, ὅν τ' ἄντρον θρέψατο Κωρύκιον, 
καὶ κασίην Ἰνδήν τε βάλοις εὐώδεα νάρδον 
 καὶ σχοῖνον νομάδων θαῦμα φέροις Ἀράβων 
καὶ λιβάνου μίσγοιο καὶ ἀγλαΐην στήσαιο 
 ἄμμιγα κυανέῳ κατθέμενος πεπέρει 
δικτάμνου τε κλῶνας ἰδὲ χλοεροῦ πρασίοιο 
 καὶ ῥῆον, στοιχὰς δ' οὐκ ἀπάνευθε μένοι, 
οὐδέ νυ πετροσέλινον ἰδ' εὐώδης καλαμίνθη 
 δριμύ τε τερμίνθου δάκρυ Λιβυστιάδος, 
θερμὸν ζιγγίβερι κεὔκλωνον πενταπέτηλον· 
 τὰς δοιὰς δραχμῶν πάντα φέροι τριάδας. 

\end{greek}


\section{Laonicus Chalcocondyles Hist.}

\blockquote[From Wikipedia\footnote{\url{http://en.wikipedia.org/wiki/Laonicus_Chalcocondyles}}]{Laonikos Chalkokondyles, Latinized as Laonicus Chalcondyles (Greek: Λαόνικος Χαλκοκονδύλης, from λαός "people", νικᾶν "to be victorious", an anagram of Nikolaos which bears the same meaning; c. 1423 – 1490) was a Byzantine Greek scholar from Athens.}

\begin{greek}

Laonicus Chalcocondyles Hist., Historiae (3139: 001)
“Laonici Chalcocandylae historiarum demonstrationes, 2 vols. in 3”, Ed. Darkó, E.
Budapest: Academia Litterarum Hungarica, 1:1922; 2.1:1923; 2.2:1927.
Volume 1, page 4, line 1

                                                       μετὰ δὲ ταῦτα 
ὕστερον οὐ πολλαῖς γενεαῖς Ἀλέξανδρον τὸν Φιλίππου, Μακε-  
δόνων βασιλέα Πέρσας ἀφελόμενον τὴν ἡγεμονίαν καὶ Ἰνδοὺς 
καταστρεψάμενον καὶ Λιβύης μοῖραν οὐκ ὀλίγην, πρὸς δὲ καὶ 
Εὐρώπης, τοῖς μεθ' ἑαυτὸν τὴν βασιλείαν καταλιπεῖν, ἐς ὃ δὴ 
Ῥωμαίους ἐπὶ τὴν τῆς οἰκουμένης μεγίστην ἀρχὴν ἀφικομένους, 
ἰσοτάλαντον ἔχοντας τύχην τῇ ἀρετῇ, ἐπιτρέψαντας Ῥώμην τῷ 
μεγίστῳ αὐτῶν ἀρχιερεῖ καὶ διαβάντας ἐς Θρᾴκην, ὑφηγουμένου 
ἐπὶ τάδε τοῦ βασιλέως, καὶ Θρᾴκης ἐπὶ χώραν, ἥτις ἐς τὴν 
Ἀσίαν ἐγγυτάτω ᾤκηται, Βυζάντιον Ἑλληνίδα πόλιν μητρόπολιν 
σφῶν ἀποδεικνύντας, πρὸς Πέρσας, ὑφ' ὧν ἀνήκεστα ἐπεπόν-
θεισαν, τὸν ἀγῶνα ποιεῖσθαι, Ἕλληνάς τε τὸ ἀπὸ τοῦδε

Ῥω-



Laonicus Chalcocondyles Hist., Historiae 
Volume 1, page 110, line 11

διώρυχα μέντοι ἐπυθόμην ἔγωγε ἀπὸ ταύτης διήκειν καὶ ἐς τὴν 
Ἰνδικὴν θάλασσαν ἐκδιδοῖ. 



Laonicus Chalcocondyles Hist., Historiae 
Volume 1, page 110, line 14

                                καὶ ἰχθύας μὲν φέρει αὕτη ἡ θά-
λασσα πολλούς τε καὶ ἀγαθούς, φέρει δὲ καὶ ὄστρεα μαργαρί-
τας ἔχοντα, ᾗπερ δὴ καὶ ἡ Ἰνδικὴ θάλασσα. 



Laonicus Chalcocondyles Hist., Historiae 
Volume 1, page 120, line 20

               ἔστι δὲ τοῦτο τὸ γένος ἄλκιμόν τε τῶν κατὰ τὴν 
Ἀσίαν καὶ πολεμικώτατον, καὶ σὺν τούτοις λέγεται Τεμήρης 
τὴν ἡγεμονίαν τῶν ἐν τῇ Ἀσίᾳ παραλαβεῖν, πλὴν Ἰνδῶν. 



Laonicus Chalcocondyles Hist., Historiae 
Volume 1, page 124, line 19

                                                                   ἔστι 
μέντοι, ᾗ πυνθάνομαι, καὶ τὰ ὑπὲρ τὴν Κασπίαν θάλασσαν καὶ 
τοὺς Μασσαγέτας ἔθνος Ἰνδικὸν ἐς ταύτην τετραμμένον τὴν 
θρησκείαν τοῦ Ἀπόλλωνος. 



Laonicus Chalcocondyles Hist., Historiae 
Volume 1, page 135, line 4

                                                        ὁ γάρ τοι 
τῆς Χαταΐης βασιλεὺς τῶν ἐννέα καλούμενος, οὗτος δ' ἂν καὶ 
ὁ τῆς Ἰνδίας βασιλεύς, διαβὰς τὸν Ἀράξην τήν τε χώραν ἐπέ-
δραμε τοῦ Τεμήρεω, καὶ ἀνδράποδα ὡς πλεῖστα ἀπάγων ᾤχετο 
αὖθις ἐπ' οἴκου ἀποχωρῶν. 



Laonicus Chalcocondyles Hist., Historiae 
Volume 1, page 139, line 20

καὶ τοὺς Τριβαλλοὺς αὐτοῦ δορυφόρους, ἐς μυρίους μάλιστά που 
γενομένους τούτους, ἐφ' οἷς δὲ μέγα ἐφρόνει ὡς, ὅποι παρα-
τυγχάνοιεν, ἀνδρῶν ἀγαθῶν γενομένων, καὶ προθέμενος ὡς Ἀλέ-
ξανδρος ὁ Φιλίππου τοὺς Μακεδόνας ἔχων μεθ' ἑαυτοῦ καὶ ἐς 
τὴν Ἀσίαν διαβάς, Δαρεῖον αἰτιασάμενος τῆς ἐς τοὺς Ἕλληνας 
Ξέρξεω ἐλάσεως, τῷ ἑαυτοῦ ἐλάσσονι δὴ στρατῷ ἐπιὼν κατε-
στρέψατο, καὶ τὴν Ἀσίαν ὑφ' αὑτῷ ἐποιήσατο, ἔστε ἐπὶ Ὕφασιν 
τῆς Ἀσίας ἐληλάκει· ἐπίστευε δὲ καὶ αὐτὸς τῷ ἑαυτοῦ στρατεύ-
ματι ἐπιὼν καθαιρήσειν ταχὺ πάνυ τὴν Τεμήρεω βασιλείαν καὶ 
ἐπὶ Ἰνδοὺς ἀφικέσθαι. 



Laonicus Chalcocondyles Hist., Historiae 
Volume 1, page 151, line 15

Οὕτω μὲν οὖν ᾕρει Τεμήρης τὰς πόλεις· ὡς δὲ ἤδη ἔαρ 
ὑπέφαινεν, ἀφίκετο παρ' αὐτὸν ἀγγελία, ὡς τοῦ Ἰνδῶν βασιλέως 
πρεσβεία ἀφικομένη ἐπὶ Χεσίην μεγάλῃ χειρὶ δεινά τε τὴν πόλιν 
ἐργάσαιτο, καὶ ἐπὶ τοὺς θησαυροὺς παριὼν τοῦ βασιλέως τόν τε 
φόρον λαβὼν οἴχοιτο, καὶ ἀπειλοίη, ὡς οὐκέτι ἐμμένοι ταῖς 
σπονδαῖς ὁ Ἰνδῶν βασιλεύς. 



Laonicus Chalcocondyles Hist., Historiae 
Volume 1, page 151, line 20

                                 ταῦτα ὡς ἐπύθετο, περιδεὴς γενό-
μενος, μὴ ἐπειδὴ ἀφίκοιτο ἡ πρεσβεία παρὰ βασιλέα τῶν Ἰνδῶν, 
ἐπιὼν καταστρέφοιτο τὴν ἑαυτοῦ χώραν, σχόντος αὐτοῦ ἀμφὶ   
τοὺς ἐπήλυδας πολέμους, καὶ ἅμα ἐσῄει αὐτὸν καὶ τὰ ἀνθρώ-
πεια ἐν οὐδενὶ ἑστηκότα ἀσφαλεῖ, καὶ δεινὰ ποιησάμενος τοὺς 
Ἰνδοὺς πρέσβεις ἐξυβρίσαι ἐς αὐτὸν οὕτως ἀναίδην, ἤλαυνεν, ὡς 
εἶχε τάχιστα, ἐπὶ Χεσίης, τόν τε Παιαζήτην ἔχων μεθ' ἑαυτοῦ 
καὶ τὸν παῖδα αὐτοῦ. 



Laonicus Chalcocondyles Hist., Historiae 
Volume 1, page 152, line 15

Ὁ δὲ Ἰνδῶν βασιλεὺς οὗτος ἐστὶν ὁ τῶν ἐννέα βασιλέων 
τοὔνομα ἔχων, Τζαχατάης βασιλεύς. 



Laonicus Chalcocondyles Hist., Historiae 
Volume 1, page 152, line 20

                              Σίνης τε βασιλεύει καὶ Ἰνδίας [καὶ] 
ξυμπάσης, καὶ διήκει αὐτῷ ἡ χώρα ἐπὶ Ταπροβάνην νῆσον, ἐς 
Ἰνδικὴν θάλασσαν, ἐς ἣν οἱ μέγιστοι τῆς Ἰνδίας χώρας ποταμοὶ   
ἐκδιδοῦσιν, ὅ τε Γάγγης, Ἰνδός, Ἀκεσίνης, Ὑδάσπης, Ὑδραώτης, 
Ὕφασις, μέγιστοι δὴ οὗτοι ὄντες τῆς χώρας. 



Laonicus Chalcocondyles Hist., Historiae 
Volume 1, page 153, line 2

                                                      φέρει δὲ ἡ Ἰνδικὴ 
χώρα ἀγαθὰ μὲν πολλὰ καὶ ὄλβον πολύν, καὶ ὅ τε βασιλεὺς 
ξυμπάσης τῆς χώρας ὑπ' αὐτὸν γενομένης. 



Laonicus Chalcocondyles Hist., Historiae 
Volume 1, page 153, line 5

                                               ὁρμώμενος δὲ οὗτος 
ἀπὸ τῆς ὑπὲρ Γάγγην χώρας καὶ τῆς παραλίου Ἰνδικῆς καὶ 
Ταπροβάνης, ἐλθεῖν ἐπὶ τὸν βασιλέα Χαταΐης, τῆς χώρας τῆς 
ἐντὸς Γάγγου καὶ Ἰνδοῦ, καὶ καταστρεψάμενον τὴν ταύτῃ χώραν 
τὰ βασίλεια ἐν ταύτῃ δὴ τῇ πόλει ποιήσασθαι· ξυμβῆναι δὲ 
τότε γενέσθαι ὑφ' ἑνὶ βασιλεῖ ξύμπασαν τὴν Ἰνδικὴν χώραν. 



Laonicus Chalcocondyles Hist., Historiae 
Volume 1, page 153, line 19

                                         φέρει δὲ ἡ Ἰνδική, ὡς 
λέγουσι, τοσοῦτον τὸ μέγεθος, ὥστε ἀπ' αὐτοῦ ναυπηγεῖσθαι πλοῖα 
μεδίμνων τεσσαράκοντα Ἑλληνικῶν. 



Laonicus Chalcocondyles Hist., Historiae 
Volume 1, page 154, line 4

                         γένος μέντοι ἰσχυρότατον γενόμενον τὸ 
παλαιὸν τούς τε Περσῶν βασιλεῖς καὶ Ἀσσυρίων, ἡγουμένους 
τῆς Ἀσίας, θεραπεύειν μὲν τοὺς Ἰνδῶν βασιλεῖς, ἐπεί τε Σεμί-
ραμις καὶ Κῦρος ὁ τοῦ Καμβύσου τὸν Ἀράξην διαβάντες με-
γάλῳ τῷ πολέμῳ ἐχρήσαντο. 



Laonicus Chalcocondyles Hist., Historiae 
Volume 1, page 154, line 7

                                  ἥ τε γὰρ Σεμίραμις τῶν Ἀσσυρίων 
βασίλισσα ἐπὶ τῶν Ἰνδῶν βασιλέα ἐλαύνουσα μεγάλῃ παρασκευῇ, 
ἐπεί τε τὸν ποταμὸν διέβη, ἐπεπράγει τε χαλεπώτατα καὶ αὐτοῦ 
ταύτῃ ἐτελεύτησε. 



Laonicus Chalcocondyles Hist., Historiae 
Volume 1, page 154, line 20

                  βασιλεὺς δὲ Τεμήρης ὡς ἐγένετο ἐς τὰ βασίλεια 
τὰ ἑαυτοῦ, τά τε ἐν τῇ ἀρχῇ αὐτοῦ καθίστη, ᾗ ἐδόκει κάλλιστα 
ἔχειν αὐτῷ, καὶ πρὸς τὸν Ἰνδῶν βασιλέα διενεχθεὶς ἐπολέμει. 



Laonicus Chalcocondyles Hist., Historiae 
Volume 1, page 156, line 3

                                          πρὸς τοῦτον Μπαϊμπούρης   
τῶν ἐννέα βασιλέων ἐπιγαμίαν ποιησάμενος καὶ ἐπιτραφθεὶς ἔσχε 
τὴν βασιλείαν· καὶ τὰ Σαμαρχάνδης πράγματα κατασχών, καὶ 
Ἰνδῶν συμμαχίαν ἐπαγόμενος, πρός τε τὸν Τζοκίην Παϊαγγούρεω 
ἐπολέμει παῖδα. 

\end{greek}


\section{Etymologicum Gudianum}

\blockquote[From Wikipedia\footnote{\url{http://en.wikipedia.org/wiki/Etymologicum_Genuinum}}]{The Etymologicum Genuinum (standard abbreviation E Gen) is the conventional modern title given to a lexical encyclopedia compiled at Constantinople in the mid ninth century. The anonymous compilator drew on the works of numerous earlier lexicographers and scholiasts, both ancient and recent, including Aelius Herodianus, Georgius Choeroboscus, Saint Methodius, Orion of Thebes, Oros of Alexandria and Theognostus the Grammarian.[1] The Etymologicum Genuinum was possibly a product of the intellectual circle around Photius. It was an important source for the subsequent Byzantine lexicographical tradition, including the Etymologicum Magnum, Etymologicum Gudianum and Etymologicum Symeonis.[2]

Modern scholarship discovered the Etymologicum Genuinum only in the nineteenth century. It is preserved in two tenth-century manuscripts, codex Vaticanus graecus 1818 (= A) and codex Laurentianus Sancti Marci 304 (= B; AD 994). Neither contains the earliest recension nor the complete text, but rather two different abridgements. The manuscript evidence and citations in later works suggest that the original title was simply τὸ Ἐτυμολογικόν and later τὸ μέγα Ἐτυμολογικόν. Its modern name was coined in 1897 by Richard Reitzenstein, who was the first to edit a sample section.[3] The Etymologicum Genuinum remains for the most part unpublished except for specimen glosses.[4] Two editions are in long-term preparation, one begun by Ada Adler and continued by Klaus Alpers,[5] the other by François Lasserre and Nikolaos Livadaras.[6] The latter edition is published under the title Etymologicum Magnum Genuinum, but this designation is not widely used and is a potential source of confusion with the twelfth-century lexical compendium conventionally titled the Etymologicum Magnum.[7]}

\begin{greek}

Etymologicum Gudianum, Etymologicum Gudianum (ἀάλιον – ζειαί) (4098: 001)
“Etymologicum Gudianum, fasc. 1 \& 2”, Ed. de Stefani, A.
Leipzig: Teubner, 1:1909; 2:1920, Repr. 1965.
Alphabetic entry alpha, page 196, line 10

                 τὸ δὲ <ινδην> . 



Etymologicum Gudianum, Etymologicum Gudianum (ζείδωρος – ὦμαι) (4098: 002)
“Etymologicum Graecae linguae Gudianum et alia grammaticorum scripta e codicibus manuscriptis nunc primum edita”, Ed. Sturz, F.W.
Leipzig: Weigel, 1818, Repr. 1973.



Etymologicum Gudianum, Etymologicum Gudianum (ζείδωρος – ὦμαι) 




Etymologicum Gudianum, Additamenta in Etymologicum Gudianum (ἀάλιον – ζειαί) (e codd. Vat. Barber. gr. 70 [olim Barber. I 70] + Paris. suppl. gr. 172) (4098: 003)
“Etymologicum Gudianum, fasc. 1 \& 2”, Ed. de Stefani, A.
Leipzig: Teubner, 1:1909; 2:1920, Repr. 1965.
Alphabetic entry delta, page 348, line 20

                       ἔνιοι δὲ αὐτὸν Δεύνυσον ὀνομάζεσθαί φασιν, ἐπειδὴ ἐβασίλευσε 
Νύσης· κατὰ γὰρ τὴν τῶν Ἰνδῶν φωνὴν δεῦνος ὁ βασιλεύς. 

Etymologicum Gudianum, Additamenta in Etymologicum Gudianum (ἀάλιον – ζειαί) (e codd. Vat. Barber. gr. 70 [olim Barber. I 70] + Pari 
Alphabetic entry epsilon, page 519, line 16

                                                                                      οἱ 
δὲ τοὺς Ἰνδούς· παρὰ τὸ Ἔρεβος· μέλανες γάρ. 

\end{greek}



\section{Nikephoros Bryennios}

\blockquote[From Wikipedia\footnote{\url{http://en.wikipedia.org/wiki/Nikephoros_Bryennios_the_Younger}.}]{Byzantine general, statesman and historian, was born at Orestias (Orestiada, Adrianople) in the theme of Macedonia



At the suggestion of his mother-in-law he wrote a history ("Materials for a History", Greek: Ὕλη Ἱστορίας or Ὕλη Ἱστοριῶν) of the period from 1057 to 1081, from the victory of Isaac I Komnenos over Michael VI to the dethronement of Nikephoros III Botaneiates by Alexios I. The work has been described as a family chronicle rather than a history, the object of which was the glorification of the house of Komnenos. Part of the introduction is probably a later addition.

In addition to information derived from older contemporaries (such as his father and father-in-law) Bryennios made use of the works of Michael Psellos, John Skylitzes and Michael Attaleiates. As might be expected, his views are biased by personal considerations and his intimacy with the royal family, which at the same time, however, afforded him unusual facilities for obtaining material. His model was Xenophon, whom he has imitated with a tolerable measure of success; he abstains from an excessive use of simile and metaphor, and his style is concise and simple.}


Nicephorus Bryennius Hist., Historiae (3088: 002)
“Nicéphore Bryennios. Histoire”, Ed. Gautier, P.
Brussels: Byzantion, 1975; Corpus fontium historiae Byzantinae 9. Series Bruxellensis.
Book 1, section 7, line 24

\begin{greek}
ἐπικρατείας μὴ μόνον Περσίδος καὶ Μηδίας 
καὶ Βαβυλῶνος καὶ Ἀσσυρίων κυριευούσης, 
ἀλλ' ἤδη καὶ Αἰγύπτου καὶ Λιβύης καὶ μέρους 
οὐκ ἐλαχίστου τῆς Εὐρώπης, ἐπείπερ ἀλλήλων 
καταστασιάσαντες οἱ ἐξ Ἄγαρ τὴν μεγίστην ἀρχὴν 
εἰς πολλὰς ἐμερίσαντο, ἄλλος ἄλλης κατάρχων, καὶ εἰς ἐμφυλίους 
πολέμους τὸ ἔθνος ἐχώρησεν, ἀρχηγὸς Περσίδος καὶ Χω-
ρασμίων καὶ Ἀβριτανῶν καὶ Μηδίας ὑπάρχων τότε 
Μουχούμετ ὁ τοῦ Ἰμβραὴλ κατὰ τοὺς χρόνους 
Βασιλείου τοῦ αὐτοκράτορος καὶ πολεμῶν Ἰνδοῖς 
καὶ Βαβυλωνίοις, ἐπειδὴ πρὸς τὸ κατόπιν ἑώρα χωροῦντα   
ἑαυτῷ τὰ πράγματα, ἔγνω δεῖν πρὸς τὸν Οὔννων δια-
πρεσβεύσασθαι ἄρχοντα καὶ ξυμμαχίαν ἐκεῖθεν 
αἰτήσασθαι. 



Nicephorus Bryennius Hist., Historiae 
Book 1, section 7, line 38

            Ἐπανελθὼν δὲ εἰς τὴν ἑαυτοῦ ἔσπευσε καὶ 
πρὸς τοὺς πολεμοῦντας Ἰνδοὺς μετὰ τῶν ξυμμά-
χων διαγωνίσασθαι. 

\end{greek}

\section{Pseudo--Sphrantzes}

\blockquote[From Wikipedia\footnote{\url{}}]{}

\begin{greek}


Pseudo-Sphrantzes Hist., Chronicon sive Maius (partim sub auctore Macario Melisseno) (3176: 001)
“Georgios Sphrantzes. Memorii 1401–1477”, Ed. Grecu, V.
Bucharest: Academie Republicii Socialiste România, 1966; Scriptores Byzantini 5.


Pseudo-Sphrantzes Hist., Chronicon sive Maius (partim sub auctore Macario Melisseno) 
Page 352, line 2

                                                                     Ὁ δὲ κύριος αὐτοῦ ἔμπο-
ρος ὢν καὶ μετὰ ἑτέρων πολλῶν ἐμπόρων θέλοντες ἐλθεῖν κατὰ τὰ τῶν Ἰνδῶν μέρη 
ποιῆσαι τὴν αὐτῶν νενομισμένην ἐμπορίαν καὶ περιπατοῦντες ἡμέρας οὐκ ὀλίγας, ἤγ-
γισαν ἔνδον τῆς τῶν Ἰνδῶν χώρας. 



Pseudo-Sphrantzes Hist., Chronicon sive Maius (partim sub auctore Macario Melisseno) 
Page 352, line 12

                                                     Ἐκεῖ δὲ καὶ τὰ μεγάλα Ἰνδικὰ γίνονται 
κάρυα καὶ δυσπόριστα ἡμῖν καὶ πάνυ ἐπιθυμητὰ ἀρώματα καὶ ὁ μαγνήτης λίθος. 



Pseudo-Sphrantzes Hist., Chronicon sive Maius (partim sub auctore Macario Melisseno) 
Page 352, line 34

                                                            Καὶ ἐνεθυμήθη, ἵνα ἐν τῇ πατρίδι 
ἐπανέλθῃ· καί τινι τῶν ἐντοπίων τὴν γνώμην αὑτοῦ εἰπών, παρέλαβεν αὐτὸν καὶ ἤγα-
γεν ἐν τόπῳ τινί, ἔνθα ἐκ τῶν ἔξωθεν Ἰνδῶν ἀκάτια ἐρχόμενα καὶ φόρτον ποιοῦντα 
ἀρωμάτων ἦν. 



Pseudo-Sphrantzes Hist., Chronicon sive Maius (partim sub auctore Macario Melisseno) 
Page 498, line 11

                                                                                      Ὁ δὲ ἕτερος 
Λουκᾶς, Ἑλληνικῶς δέ· καὶ ἐδόθη εἰς τὴν Ἀσίαν καὶ Αἰθιοπίαν καὶ Περσίαν καὶ 
Ἰνδίαν καὶ Ἀραβίαν. 

\end{greek}


\section{\emph{Encomium Heraclii ducis}}

\blockquote[From Wikipedia\footnote{\url{}}]{}

\begin{greek}
Epica Adespota (GDRK), Encomium Heraclii ducis (PSI 3.253) (1816: 015)
“Die griechischen Dichterfragmente der römischen Kaiserzeit, vol. 1, 2nd edn.”, Ed. Heitsch, E.
Göttingen: Vandenhoeck \& Ruprecht, 1963.
Line 50

τῷ δὲ μ̣έλας β̣οέοιο φ[ερώ]νυμ[ος] 
πορφυρέ<ε>ι κ[...........] ο...ω[] 
Ἰνδῶν ἠλ[ιβάτων] 
κυαν.[] 
[].λ̣...α̣..ε̣υ̣ μαστιγ..[] 
[].ν̣ε.....βασιληιο̣[] 
[] 
[] 
[].ε....των κλ̣.[] 
[] 
[ο]ὗ̣τος ἀνὴρ μ̣ε[] 
[].μ̣ιν..αδιην[] 

\end{greek}


\section{Joannes VI Cantacuzenus}
\blockquote[From Wikipedia\footnote{\url{http://en.wikipedia.org/wiki/Joannes_Cantacuzene}}]{John VI Kantakouzenos or Cantacuzenus (Greek: Ἰωάννης ΣΤʹ Καντακουζηνός, Iōannēs VI Kantakouzēnos) (c. 1292 – 15 June 1383) was the Byzantine emperor from 1347 to 1354.

His History in four books deals with the years 1320–1356. An apologia for his own actions, it needs to be read with caution; fortunately it can be supplemented and corrected by the work of a contemporary, Nikephoros Gregoras. It possesses the merit of being well arranged and homogenous, the incidents being grouped round the chief actor in the person of the author, but the information is defective on matters with which he is not directly concerned. Kantakouzenos also wrote a defence of Hesychasm, a Greek mystical doctrine.}

\begin{greek}
Joannes VI Cantacuzenus, Historiae (3169: 001)
“Ioannis Cantacuzeni eximperatoris historiarum libri iv, 3 vols.”, Ed. Schopen, L.
Bonn: Weber, 1:1828; 2:1831; 3:1832; Corpus scriptorum historiae Byzantinae.

Joannes VI Cantacuzenus, Historiae 
Volume 2, page 331, line 17

          βασιλεὺς δὲ ἐπεὶ γένοιτο μακρὰν Φερῶν, συνιδὼν 
ὡς ἡ ἑπομένη στρατιὰ τῶν Τριβαλῶν ὄχλος μόνον ἀνόνητός 
εἰσι, (τοὺς ἀρίστους γὰρ αὐτῶν ἀπολεξάμενος πρότερον 
ὁ Κράλης, φρουρὰν κατέλιπε ταῖς πόλεσιν, ἃς Χρέλη ἔ-
χοντος μετὰ τὴν ἐκείνου τελευτὴν ἔλαβεν αὐτὸς,) ἄλλως 
τε καὶ τῶν ὑπολειφθέντων χρονίῳ τε στρατείᾳ τεταλαιπω-
ρηκότων, (ἦσαν γὰρ πλέον ἢ δυσὶ πρότερον μησὶ Κράλῃ 
ἑπόμενοι στρατευομένῳ,) ἄλλως τε καὶ δέει ἀσχέτῳ κατε-
χομένων καὶ νομιζόντων, οὐκ εἰς Θρᾴκην, \xecolor{red}{ἀλλ' εἰς Πάρ-
θους ἢ Ἰνδοὺς στρατεύεσθαι}, ὅθεν οὐκ ἐξέσται μηχανῇ οὐ-
δεμιᾷ οἴκαδε ἐπανελθεῖν, καὶ διὰ τοῦτο καὶ ἵππους πολεμι-
στηρίους καὶ ὅπλα καὶ εἴ τι ἐπεφέροντο χρήσιμον πρὸς τὸν 
πόλεμον, οἴκαδε ἀποπεμπόντων, ἵνα ταῦτα γοῦν τοῖς παισὶν 
ὑπολειφθείη, ὡς ἐκείνων ἀπολουμένων πάντως· ταῦτα δὴ 
συνορῶν ὁ βασιλεὺς καὶ βουλόμενος εἰς Διδυμότειχον μὴ οὕ-
τως ἀφικέσθαι ἀσθενὴς, ὥσθ' ὑπὸ Βυζαντίων καὶ αὐτὸς πο-
λιορκεῖσθαι, (οὔτ' αὐτῷ γὰρ οὔτε τοῖς συνοῦσι τοῦτο μάλι-  
στα συνοίσειν,) ἐσκέψατο εἰς Κράλην ἀναστρέφειν καὶ στρα-
τιὰν ἀξιόχρεων αἰτεῖν, ὥστε εἰς Διδυμότειχον φοβερὸν τοῖς 

\end{greek}


\section{Constantinus Manasses}
\blockquote[From Wikipedia\footnote{\url{http://en.wikipedia.org/wiki/Constantinus_Manasses}}]{Constantine Manasses (Greek: Κωνσταντῖνος Μανασσῆς; c. 1130 - c. 1187) was a Byzantine chronicler who flourished in the 12th century during the reign of Manuel I Komnenos (1143-1180). He was the author of a chronicle or historical synopsis of events from the creation of the world to the end of the reign of Nikephoros Botaneiates (1081), sponsored by Irene Komnene, the emperor's sister-in-law. It consists of about 7000 lines in the so-called political verse. It obtained great popularity and appeared in a free prose translation; it was also translated into Bulgarian and Roman Slavic in the 14th century and enjoyed a great popularity.}
\begin{greek}


Constantinus Manasses Hist., Poeta, Compendium chronicum (3074: 001)
“Constantini Manassis breviarium historiae metricum”, Ed. Bekker, I.
Bonn: Weber, 1837; Corpus scriptorum historiae Byzantinae.
Line 924

Οὗτος Περσῶν ἐκράτησεν, οὗτος Ἰνδῶν κατῆρξε, 
τούτῳ καθυπετάγησαν Συρία καὶ Φοινίκη 
καὶ πᾶν ἔθνος καὶ πάσης γῆς χωράρχαι καὶ σατράπαι 
ἀπ' ἄκρων τῶν ἀνατολῶν μέχρι δυσμῶν ἐσχάτων. 



Constantinus Manasses Hist., Poeta, Compendium chronicum 
Line 1367

τὸν οὖν Ταντάνην τῶν Ἰνδῶν Πρίαμος ἱκετεύει, 
καὶ μετὰ πλήθους στέλλεται Μέμνων ἀπειραρίθμου. 



Constantinus Manasses Hist., Poeta, Compendium chronicum 
Line 1369

ὁ δὲ στρατὸς ἦσαν Ἰνδοὶ πάντες μελανοχρῶτες· 
οὕσπερ ἰδόντες Ἕλληνες ἐν ἀλλοκότῳ θέᾳ, 
καὶ δειλιάσαντες αὐτῶν μορφὴν καὶ πανοπλίαν, 
καὶ ζῷα περιτρέσαντες ἅπερ Ἰνδία τρέφει, 
νύκτωρ φυγεῖν ἐσκέπτοντο καὶ προλιπεῖν τὴν Τροίαν. 



Constantinus Manasses Hist., Poeta, Compendium chronicum 
Line 1375

ἀλλ' ὅμως ἀντετάξαντο πρὸς τοὺς κελαινοχρῶτας,   
καὶ τῶν Ἰνδῶν τοῖς αἵμασιν ἠρύθρωσαν ἀρούρας, 
καὶ τοῦ Σκαμάνδρου τὰς ῥοὰς ἐφοίνιξαν τοῖς λύθροις. 



Constantinus Manasses Hist., Poeta, Compendium chronicum 
Line 2931

ὁ Βασιλίσκος γὰρ πολλῷ φαρμακευθεὶς χρυσίῳ 
ἔβλεψε πρῶτος εἰς φυγὴν κατὰ τὰς ὑποσχέσεις, 
κἀντεῦθεν ἀνετράπησαν τὰ πράγματα Ῥωμαίοις, 
καὶ τὸν Ἰνδοὺς φοβήσαντα καὶ τοὺς ἐν Βρεττανίᾳ 
καὶ πᾶν ἔθνος καὶ πᾶσαν γῆν στόλον τὸν φρικαλέον   
μόνη χρυσίου στιλβηδὼν ἴσχυσεν ἀφανίσαι 
ἄνευ χειρῶν, ἄνευ βελῶν, ἄτερ ὁπλοφορίας. 



Constantinus Manasses Hist., Poeta, Compendium chronicum 
Line 3129

φιλίαν γάρ τοι καθαρὰν καὶ πρὸ τῆς βασιλείας   
πρὸς τὸν Ἰνδέριχον πλουτῶν τῶν Οὐανδήλων ῥῆγα, 
καὶ γράμματα δεξάμενος ὡς ὁ σκαιὸς Γελίμερ 
ἐπανασταίη κατ' αὐτοῦ καὶ κατακλείσας ἔχοι 
αὐτόν τε τὸν Ἰνδέριχον καὶ γαμετὴν καὶ τέκνα, 
καὶ τὴν ἀρχὴν ἀφέλοιτο σύμπασαν Ἰνδερίχου, 
ταῦτα μαθὼν ὁ βασιλεύς, πληγείς τε τὴν καρδίαν 
καὶ μέγα παθηνάμενος ὑπὲρ τοῦ δυσπραγοῦντος, 
στόλον κατὰ Γελίμερος μυρίανδρον ἐκπέμπει, 
στολάρχην δὲ καθίστησι καὶ στρατηγὸν τῆς μάχης 
τὸν μέγαν Βελισάριον, τὴν τῶν Ῥωμαίων χεῖρα, 


\end{greek}


\section{Michael Glycas}
\blockquote[From Wikipedia\footnote{\url{http://en.wikipedia.org/wiki/Michael_Glycas}}]{Michael Glycas or Glykas (Greek: Μιχαὴλ Γλυκᾶς; 12th century) was a Byzantine historian, theologian, mathematician, astronomer and poet. He was probably from Corfu and lived in Constantinople (now Istanbul).

His chief work is his Chronicle of events from the creation of the world to the death of Alexius I Comnenus (1118). It is extremely brief and written in a popular style, much space is devoted to theological and scientific matters. Glycas was also the author of a theological treatise and a number of letters on theological questions.

A poem of some 15-syllable verses, written in 1158/1159 during his imprisonment on a charge of slandering a neighbor and containing an appeal to the emperor Manuel I, is extant, and is commonly regarded as the first dated work of Modern Greek literature, since it contains several vernacular proverbs. The exact nature of his offence is not known, but his punishment was to be blinded.}

\begin{greek}

Michael Glycas Astrol., Hist., Annales (3047: 001)
“Michaelis Glycae annales”, Ed. Bekker, I.
Bonn: Weber, 1836; Corpus scriptorum historiae Byzantinae.
Page 27, line 13

                ὁ μὲν οὖν Θεοδώρητός φησιν ὅτι καλῶς εἶπε 
καὶ συναγωγὴν καὶ συναγωγὰς ἐπὶ τῶν ὑδάτων· τὰ γὰρ πε-
λάγη κατὰ τὴν ἔξω ἐπιφάνειαν διῃρημένα δοκοῦσιν (ἄλλο γὰρ 
τὸ Ἰνδικὸν πέλαγος καὶ ἄλλο τὸ Ὑρκανικόν), κάτωθεν δὲ συν-
ήρμοσται διά τινων ὑπογείων πόρων. 



Michael Glycas Astrol., Hist., Annales 
Page 82, line 7

      λέγεται γὰρ ὅτι γενόμενος ἔγκυος ὁ γὺψ πορεύεται εἰς 
τὴν Ἰνδικήν, καὶ λαβὼν λίθον τὸ καλούμενον εὐτόκιον ἐπά-
νω αὐτοῦ κάθηται συνεχόμενος ὠδῖσι, καὶ οὕτως ἀποτίκτει. 



Michael Glycas Astrol., Hist., Annales 
Page 100, line 7

                                              διὸ καὶ τοὺς Ἰνδοὺς 
ἐπιτηρεῖν ἀσέληνον νύκτα φασίν, ὁπηνίκα διαπορθμεύουσι 
τουτὶ δι' ὑδάτων τὸ ζῶον. 



Michael Glycas Astrol., Hist., Annales 
Page 100, line 11

                         τὸν γὰρ Ἰνδὸν ποταμὸν ἔστιν ὅτε δια-
περαιούμενοι οὐχ ἁπλῶς καὶ ὡς ἔτυχε τὴν ἐκεῖσε ποιοῦνται 
διάβασιν, ἀλλ' ὁ μικρότατος πάντων τῆς περαιώσεως πρῶτος 
ἀπάρχεται, μετὰ τοῦτον ὁ μείζων αὐτοῦ, ἔπειτα καὶ ὁ ἀμφο-
τέρων ἐπέκεινα. 



Michael Glycas Astrol., Hist., Annales 
Page 105, line 2

Γίνωσκε δὲ τοῦτο, ὅτι γεννήτορες τοῦ ἀρώματος μόσχου 
οἱ ἐν Ἰνδίᾳ νεμόμενοι ἔλαφοι. 


Michael Glycas Astrol., Hist., Annales 
Page 236, line 15

                                                             ἀλ-
λὰ καὶ συναγόμενα ὁρῶντες τὰ θηρία πάντα, καθά φησιν ὁ 
θεῖος Ἐφραΐμ, ἐλέφαντας μὲν ἀπὸ Ἰνδικῆς καὶ Περσίδος 
ἐρχόμενα, λέοντας καὶ παρδάλεις μετὰ προβάτων καὶ αἰ-
γῶν μιγάδας, ἑρπετὰ καὶ πετεινὰ ἄνευ τινὸς διώκοντος ἐρχό-
μενα καὶ κύκλῳ τῆς κιβωτοῦ αὐλιζόμενα, οὐδαμῶς κατε-
νύγησαν. 



Michael Glycas Astrol., Hist., Annales 
Page 269, line 2

Οὗτος μὲν οὖν Ἀλέξανδρος καὶ μέχρι τῶν ἐνδοτάτων 
Ἰνδῶν καὶ αὐτοῦ τοῦ ὠκεανοῦ καὶ τῆς μεγίστης νήσου τῶν 
Βραχμάνων φθάσας, ὧν καὶ τὸν θαυμάσιον βίον θαυμάσας καὶ 
τὴν εἰς θεὸν εὐσέβειάν τε καὶ λατρείαν ἐξεπλάγη, ἐν ᾧ τό-
πῳ καὶ στήλην στήσας ὑπέγραψεν “ἐγὼ μέγας Ἀλέξανδρος 
βασιλεὺς ἔφθασα μέχρι τούτου. 



Michael Glycas Astrol., Hist., Annales 
Page 269, line 17

                                           καὶ οἱ μὲν ἄνδρες εἰς 
ποταμὸν παροικοῦσιν, αἱ δὲ γυναῖκες αὐτῶν ἐντεῦθέν εἰσι 
τοῦ ποταμοῦ Γάγγου τοῦ παραρρέοντος εἰς τὸν ὠκεανὸν ἐπὶ 
τὸ μέρος τῆς Ἰνδίας. 



Michael Glycas Astrol., Hist., Annales 
Page 318, line 3

                 τούτων οὖν ἕνεκεν ὁ παρὰ τῷ βασιλεῖ μέγι-  
στος Ἀμμὰν λυπηθεὶς ἔπεισε τὸν βασιλέα κατὰ Ἰουδαίων 
ἐπεξελθεῖν, καὶ γράμματα στεῖλαι τοῖς τῶν εἴκοσι καὶ ἑπτὰ 
χωρῶν ἄρχουσιν, ἀπὸ Ἰνδικῆς ἕως Αἰθιοπίας, τοὺς ὁπουδή 
ποτε παρευρισκομένους Ἰουδαίους τελείως κατασφάττειν κατὰ 
τὴν τεσσαρεσκαιδεκάτην τοῦ δωδεκάτου μηνός. 



Michael Glycas Astrol., Hist., Annales 
Page 501, line 17

                μοναχοὶ δὲ δύο τινὲς ἐξ Ἰνδίας ἐλθόντες τὴν 
γένεσιν αὐτῆς διηγήσαντο, καὶ ὑπισχνοῦντο κομίσαι τὸν τῶν 
σκωλήκων ἐκείνων γόνον, ᾠὰ ὄντα καὶ πάνυ σμικρά, δεῖξαί 
τε Ῥωμαίοις ὅπως ἐκεῖνα ζωογονοῦνται θαλπόμενα καὶ εἰς 
σκώληκας μεταβάλλονται, καὶ ὅπως δημιουργοῦσι τὴν με-  
τάξαν. 

\end{greek}

\section{Lexica In Opera Gregorii Nazianzeni}
\blockquote[From Wikipedia\footnote{\url{}}]{}
\begin{greek}


Lexica In Opera Gregorii Nazianzeni, Lexicon in carmina Gregorii Nazianzeni (ordine alphabetico) (e cod. Paris. Coislin. 394) (4303: 004)
“Λεξικὰ τῶν ἐπῶν Γρηγορίου τοῦ Θεολόγου μετὰ γενικῆς θεωρήσεως τῆς πατερικῆς λεξικογραφίας”, Ed. Kalamakis, D.
Athens: Papadakis, 1992.
Alphabetic letter iota, lemma 38, line 1

<ἱκέσσιον>· ἱκέτην 
<ἰκμαλέω τε>· καὶ διΰγρῳ, ἱδρωτοποιῷ 
<ἵκωμαι>· παραγένωμαι 
<ἵλαος <εἴης>>· <εὐμενὴς> γένοιο 
<ἵλαος>· εὐμενής 
<ἵλαθι>· ἵλεως γενοῦ   
<ἰλυόεντι>· καὶ πηλώδει 
<ἰλύος>· γῆς, πηλοῦ 
<ἱμείρων>· ἐπιθυμῶν 
<ἱμερόεντα>· ἐπιθυμητικόν 
<ἴνδαλμα>· ὁμοίωμα, εἴδωλον 
<Ἰ<ν>δοῖσι>· τοῖς Ἰνδοῖς 
<ἰνδάλματα>· εἴδωλα, ἀπεικάσματα 
<ἵξεται>· παραγενήσεται 
<ἴομεν>· βαδίσωμεν 
<ἰόντα>· περαιούμενον 
<ἰόντι>· πορευομένῳ 
<ἰότητι>· τῇ βουλήσει 
<ἰούσης>· συντρεχούσης 
<†ιρείαις> (nusquam)· ταῖς διαφόροις τῇ ἴριδι ἐοικυίαις 




Lexica In Opera Gregorii Nazianzeni, Lexicon in carmina Gregorii Nazianzeni (ordine alphabetico) (e cod. Paris. Coislin. 394) 
Alphabetic letter iota, lemma 39, line 1

<ἰκμαλέω τε>· καὶ διΰγρῳ, ἱδρωτοποιῷ 
<ἵκωμαι>· παραγένωμαι 
<ἵλαος <εἴης>>· <εὐμενὴς> γένοιο 
<ἵλαος>· εὐμενής 
<ἵλαθι>· ἵλεως γενοῦ   
<ἰλυόεντι>· καὶ πηλώδει 
<ἰλύος>· γῆς, πηλοῦ 
<ἱμείρων>· ἐπιθυμῶν 
<ἱμερόεντα>· ἐπιθυμητικόν 
<ἴνδαλμα>· ὁμοίωμα, εἴδωλον 
<Ἰ<ν>δοῖσι>· τοῖς Ἰνδοῖς 
<ἰνδάλματα>· εἴδωλα, ἀπεικάσματα 
<ἵξεται>· παραγενήσεται 
<ἴομεν>· βαδίσωμεν 
<ἰόντα>· περαιούμενον 
<ἰόντι>· πορευομένῳ 
<ἰότητι>· τῇ βουλήσει 
<ἰούσης>· συντρεχούσης 
<†ιρείαις> (nusquam)· ταῖς διαφόροις τῇ ἴριδι ἐοικυίαις 
<†ιρι (ἶριν)>· †ο ἴριδι ἐοικυῖα· ἐστὶ δὲ ὡς τόξον ἐν οὐρανῷ 




Lexica In Opera Gregorii Nazianzeni, Lexicon in carmina Gregorii Nazianzeni (ordine alphabetico) (e cod. Paris. Coislin. 394) 
Alphabetic letter iota, lemma 39, line 1

<ἵκωμαι>· παραγένωμαι 
<ἵλαος <εἴης>>· <εὐμενὴς> γένοιο 
<ἵλαος>· εὐμενής 
<ἵλαθι>· ἵλεως γενοῦ   
<ἰλυόεντι>· καὶ πηλώδει 
<ἰλύος>· γῆς, πηλοῦ 
<ἱμείρων>· ἐπιθυμῶν 
<ἱμερόεντα>· ἐπιθυμητικόν 
<ἴνδαλμα>· ὁμοίωμα, εἴδωλον 
<Ἰ<ν>δοῖσι>· τοῖς Ἰνδοῖς 
<ἰνδάλματα>· εἴδωλα, ἀπεικάσματα 
<ἵξεται>· παραγενήσεται 
<ἴομεν>· βαδίσωμεν 
<ἰόντα>· περαιούμενον 
<ἰόντι>· πορευομένῳ 
<ἰότητι>· τῇ βουλήσει 
<ἰούσης>· συντρεχούσης 
<†ιρείαις> (nusquam)· ταῖς διαφόροις τῇ ἴριδι ἐοικυίαις 
<†ιρι (ἶριν)>· †ο ἴριδι ἐοικυῖα· ἐστὶ δὲ ὡς τόξον ἐν οὐρανῷ 

\end{greek}

\printbibliography

\printindex

\end{document}

